\chapter{Frames and descriptions}
% \chapter[Common nouns, intransitive verbs, frames, the Partee puzzle
%   and passengers][Common Nouns and Frames]{Common nouns, intransitive verbs, frames, the Partee puzzle
%   and passengers}
%\markboth{CHAPTER \thechapter. COMMON NOUNS AND FRAMES}{}
\label{ch:commonnouns}
\setcounter{equation}{0}

\section[Montague's treatment of common nouns and individual
concepts][Montague's treatment of common nouns]{Montague's treatment of
  common nouns and  individual concepts}
The treatment of common nouns in Chapter~\ref{ch:gram} is encapsulated
in Lex$_{\mathrm{CommonNoun}}$ and carried over into
Chapter~\ref{ch:propnames} where it was modified to accommodate
parametric contents.
The idea is that for a common noun such as \textit{dog} there should
be a corresponding predicate `dog' with arity
$\langle$\textit{Ind}$\rangle$ as well as a phonological type
``dog''.  Then an utterance event of the type ``dog'' will be
associated with the content in \nexteg{a} whose foreground is of type
\nexteg{b}. 
\begin{ex} 
\begin{subex}
\item $\ulcorner\lambda c$:\textit{Rec} . $\lambda r$:\smallrecord{\smalltfield{x}{\textit{Ind}}}
  . \smallrecord{\smalltfield{e}{dog($r$.x)}}$\urcorner$ 
\item
  (\textit{Rec}$\rightarrow$(\smallrecord{\smalltfield{x}{\textit{Ind}}} $\rightarrow$ \textit{RecType}))
\end{subex}
\label{ex:dog}
\end{ex}
Once we have applied the foreground of \preveg{a} to a context
(modelled as a record) we will obtain the function \nexteg{a} of type
\nexteg{b}
\begin{ex}
  \label{ex:non-paradog}
\begin{subex} 
 
\item  $\lambda r$:\smallrecord{\smalltfield{x}{\textit{Ind}}}
  . \smallrecord{\smalltfield{e}{dog($r$.x)}}
\item
  (\smallrecord{\smalltfield{x}{\textit{Ind}}} $\rightarrow$ \textit{RecType})
 

 
\end{subex} 
   
\end{ex} 
There is a correspondence between this and Montague's treatment of
common nouns.  
\cite{Montague1973} introduces predicates corresponding to common
nouns which in his type system are of the type $\langle\langle
s,e\rangle,t\rangle$.  The type $\langle s,e\rangle$ for Montague is
the type of individual concepts.  These are modelled as functions from
world-time pairs (of type $s$) to individuals (of type $e$).  The
reason that Montague used this type rather than the simpler type
$\langle e,t\rangle$, that is, the type of functions from individuals
to truth-values, has to do with his treatment of the Partee puzzle
concerning temperatures and prices which we will take up below.  Much
subsequent research has abandoned Montague's analysis using
individual concepts and used the simpler type $\langle e,t\rangle$.
This alternative would correspond to \nexteg{} in our terms.
\begin{ex} 
\begin{subex} 
 
\item $\lambda x$:\textit{Ind}
  . \smallrecord{\smalltfield{e}{dog($x$)}}  
 
\item (\textit{Ind} $\rightarrow$ \textit{RecType}) 
 
\end{subex} 
   
\end{ex}
\textit{Ind} corresponds to Montague's type $e$, the type of entities
or individuals. Instead of Montague's $t$, the type of truth values we
have \textit{RecType}.  Record types serve as our propositions.  Thus
instead of mapping to a truth value (for Montague, following Frege,
the denotation corresponding to a proposition), we map to the
proposition itself (see the discussion in Chapter~\ref{ch:gram}, Section~\ref{sec:gram-semantics}).
We will argue that (\ref{ex:dog}) is preferable to \preveg{} in that records which
are arguments to such a function are frames and that, among other
things, frames as arguments enable us to account for the Partee
puzzle. One way of seeing this is that we are using frames to replace
Montague's use of individual concepts to approach this problem.  In
this way our proposal is closely related to work by \cite{Loebner2015}.  We made this proposal in previous work
\citep{Cooper2010,Cooper2012}.  Here we will present a modification of
that proposal which uses frames to introduce scales and measure
functions and yields a more general treatment of the semantics of
verbs like \textit{rise} than we were able to give in the earlier
treatment.  In addition it gives us a way of treating nouns like
\textit{passenger} where, at least on some readings, we seem to be
predicating of passenger events, rather than individual passengers.
We will also relate our treatment to other recent work on the
introduction of frame semantics into formal semantics.  

\section{The Partee puzzle}

% Perhaps the most recent discussion of the Partee puzzle is that of
% \cite{Lobnerinprep}.  As we will see, his proposal is closely related
% to our own.
The puzzle is one that Barbara Partee raised while
sitting in on an early presentation of the material that led to
\cite{Montague1973}.  In its simplest form it is that \nexteg{c}
should follow from \nexteg{a,b} given some otherwise apparently
harmless assumptions.
\begin{ex} 
\begin{subex} 
 
\item The temperature is rising 
 
\item The temperature is ninety

\item Ninety is rising 
 
\end{subex} 
\label{ex:parteepuzzle}   
\end{ex}
Clearly, our intuitions are that \preveg{c} does not follow from
\preveg{a,b}. The assumptions that the error relies on are those given
in \nexteg{}.
\begin{ex} 
\begin{subex} 
 
\item  \textit{temperature} is a
predicate of individuals
 
\item \textit{is} in (\ref{ex:parteepuzzle}b) represents identity
between individuals
 
\end{subex} 
   
\end{ex}
Montague's solution was to abandon \preveg{a} and say that
`temperature' is a predicate not of individuals but of
individual \textit{concepts}, in his terms functions from world-time
pairs to individuals, thus introducing intensionality into predication
by common nouns.  When we say (\ref{ex:parteepuzzle}a) we are
predicating `rise' not of an individual but of a function.
When we say (\ref{ex:parteepuzzle}b) we are saying that the
\textit{value} of the function at the current world and time is
identical with ninety.  The technical machinery that Montague uses to
achieve this involves his predilection for general treatments.  
He treats all common nouns as being predicates of individual
concepts.  But in the case of all nouns other than \textit{price} or
\textit{temperature} in his fragment he requires that the individual
concepts are rigid designators, that is, they are constant functions
which return the same individual for every world-time pair.  Similarly
intransitive verbs will correspond to predicates of individual
concepts but in the case of verbs other than \textit{rise} and
\textit{change} (in his fragment) there will be a predicate of the
value of the individual concept which holds just in case the verb
predicate holds of the individual concept.  Finally \textit{be} is
treated as representing identity of the values of individual concepts
and a given time and world
and not identity of the individual concepts.  Thus two distinct
individual concepts can have identical values at a given world and
time.

Given this machinery we can analyze the Partee puzzle represented in
(\ref{ex:parteepuzzle}) as follows.  When we say that the temperature
is rising we are predicating `rise' of an individual concept, a
function from world-time pairs.  Montague does not say what it might
mean for such a function to rise.  There is, however,  something
obvious that we could say, namely that if $f$ is such that
temperature($f$) at world $w$ and time $t$, then rise($f$) is true at world $w$ and time $t$ just
in case there is some time $t'$, $t'< t$ (``$t'$ is earlier than $t$''), and some time $t''$,
$t<t''$, such that $f(w,t')$ is less than $f(w,t'')$.  (We may assume
that $f$ returns  a (real) number for any world and time.) When we say
that the temperature $f$ is ninety at world $w$ and time $t$, what we
mean is that $f(w,t)=90$ (assuming that the interpretation of
\textit{ninety} is an individual concept $g$ such that for any world,
$w$, and time, $t$, $g(w,t)=90$).  From this is does not follow that
ninety is rising, that is, rise($g$).  After all, we have just said
that \textit{ninety} corresponds to a constant function which always
returns the same value and rising functions have to return different
values at different times.

We have now shown that Montague's analysis prevents the offending
inference from going through but we must also show that the inference
does go through in ``normal'' cases according to his analysis.
Consider \nexteg{}.
\begin{ex} 
\begin{subex} 
 
\item The dog is barking 
 
\item The dog is Fido

\item Fido is barking 
 
\end{subex} 
   
\end{ex} 
Here we do want \preveg{c} to follow as a conclusion from the premises
\preveg{a,b}.  When we say that the dog is barking we are predicating
`bark' of a constant function $f$ since for an individual
concept to fall under the predicate `dog' it must be rigid,
i.e. return the same object for each world and time.  Furthermore,
there is a predicate, call it `bark$_*$', such that for any $w$ and $t$, `bark$_*$' holds of
$f(w,t)$ just in case
`bark' holds of $f$.  So in effect by predicating `bark'
of $f$ at $w$ and $t$, we are predicating `bark$_*$' of $f(w,t)$.
(Given Montague's notion of proposition, bark($f$) and
bark$_*$($f(w,t)$) are the \textit{same} proposition since they are
true at exactly the same possible worlds and times.)  When we
say that the dog is Fido at $w$ and $t$ what we mean is that
$f(w,t)=g(w,t)$ where $g$ is the individual concept corresponding to
\textit{Fido}.  According to Montague's theory of proper names $g$ too
will be a constant function always returning the same individual, say,
`fido'.  Is Fido barking given these assumptions, that is, is
bark($g$) true at $w$ and $t$?  There are a couple of ways to make the
argument.  Since both $f$ and $g$ are constant functions if they have
the same value at any world and time they will have the same value at
all worlds and times, that is, given the classical set theoretic view
of functions that Montague is using, $f$ and $g$ will in fact be the
same function.  Thus if we predicate anything of $f$ it will also hold
of $g$, since they are identical.  The other argument involves the
nature of the predicate `bark'.  Since bark($f$) is equivalent to
bark$_*$($f(w,t)$) and, given that $f(w,t)=g(w,t)$, bark$_*$($f(w,t)$)
is equivalent to bark$_*$($g(w,t)$), which in turn is equivalent to
bark($g$), then bark($f$) and bark($g$) are equivalent.  Thus if
bark($f$) is true, then so is bark($g$). 

Despite the obvious ingenuity and formal correctness of this solution
it fell into disuse.  As \cite{Loebner2015} points out one objection
is to the interpretation of (\ref{ex:parteepuzzle}) as an identity
statement rather than the location of the temperature value on a
scale.  This point was made by \cite{Jackendoff1979}, a paper which
has given rise to a trickle of remarks and replies in
\textit{Linguistic Inquiry} over a period of thirty years:
\cite{Lobner1981,Lasersohn2005a,Romero2008}.  Part of Jackendoff's
argument is that in addition to \nexteg{a} we can also say \nexteg{b},
just as we can say \nexteg{c}.  
\begin{ex} 
\begin{subex} 
 
\item The temperature is ninety 
 
\item The temperature is at ninety

\item The airplane is at 6000 feet 
 
\end{subex} 
   
\end{ex} 
We do not, he argues, feel the temptation to conclude \nexteg{c} from
\nexteg{a,b}.
\begin{ex} 
\begin{subex} 
 
\item The airplane is at 6000 feet 
 
\item The airplane is rising

\item 6000 feet is/are rising 
 
\end{subex} 
   
\end{ex} 
So neither should we feel the temptation to draw the offending
conclusion in the temperature puzzle since even though we say
\textit{the temperature is ninety} we mean \textit{the temperature is
  at ninety}.  Jackendoff does not point out, however, that there is
an important difference between the temperature and the airplane case,
namely that \nexteg{} does not mean the same as \preveg{a}, and to the
extent that it means anything it means something absurd which involves an
equality between an airplane and 6000 feet.
\begin{ex} 
The airplane is 6000 feet 
\end{ex} 
If Jackendoff were right that \textit{is} can mean \textit{is at} why
would this be the case?  \cite{Lobner1981} has a stronger argument
against Jackendoff.  He points out that we cannot conclude \nexteg{c}
from \nexteg{a,b}
\begin{ex} 
\begin{subex} 
 
\item The temperature of the air in my refrigerator is the same as the
  temperature of the air in your refrigerator

 
\item The temperature of the air in my refrigerator is rising

\item The temperature of the air in your refrigerator is rising  
 
\end{subex} 
   
\end{ex} 
\cite{Lasersohn2005a} gives the example in \nexteg{} based on L�bner's
example.
\begin{ex} 
\begin{subex} 
 
\item The temperature in Chicago is rising 
 
\item The temperature in Chicago is the very same as the temperature
  in St. Louis

\item The temperature in St. Louis is rising
 
\end{subex} 
   
\end{ex} 
These examples are meant to show that there are similar cases to the
original Partee puzzle where the construction seems clearly equative
rather than locative.  Note that we can mention identity explicitly as
in \nexteg{}.
\begin{ex} 
The temperature in Chicago is identical with the temperature in St. Louis 
\end{ex} 
\cite{Romero2008} discusses examples with prices where it seems
intuitive that there are two readings, one where the inference does
not go through and one where it does.
\begin{ex} 
\begin{subex} 
 
\item The prices in supermarket $A$ are (the very same as) the prices
  in supermarket $B$ 
 
\item Most prices in supermarket $A$ are rising

\item Most prices in supermarket $B$ are rising 
 
\end{subex} 
   
\end{ex} 
On one reading (not the preferred one, I think) \preveg{a} means that
at the current time the prices just happen to be the same.  In this
case the inference does not go through.  The other reading is that the
prices in the two supermarkets are pegged to each other, perhaps
because they are owned by the same chain even though they have
different names.  (Note that this is not quite the same as saying that
the prices are \textit{necessarily} the same which is the case that
Romero discusses.  This is a matter of business strategy, not logic.
The supermarket owners \textit{could} have chosen not to peg the
prices to each other.)  In this case the inference does go
through.\footnote{Actually, there is a further complication with these
  examples involving plural quantifiers, which Romero does not
  discuss.  We also need an assumption that the two supermarkets have
  sufficiently similar stock.  If most of the prices are rising in
  supermarket $A$ and supermarket $B$ only stocks those items whose
  prices are not rising in supermarket $A$, then even though the
  prices in the two supermarkets are the same (and pegged to each
  other), the prices in supermarket $B$ are not rising.}  

Despite all this discussion there is an important intuition in
Jackendoff's observation that the interpretation of \textit{the
  temperature is ninety} involves the placement of the temperature on
a scale.  In a sense  Montague was recognizing this by modelling
temperature in terms of his individual concepts.  He was giving us a
function which returns for each world and time an individual
(presumably a number) representing the temperature.  Thus he could
account for the fact that the temperature is different at different
times.  The problem is, though, that possible worlds (that is, total
ways the universe could be) do not have a single temperature, even at
a single point of time.  The notion of individual concept he has is
simply not fine-grained enough to deal with temperature.  One can
understand why Montague might not have wanted to pursue this matter
further in PTQ.  He wanted to include the treatment of temperature in
his general treatment of intensions (functions from possible worlds
and times to objects of various types) but in order to get temperature
right he would have had to change this.  One strategy would be to use
possible situations (parts of possible worlds).  Another strategy
would have been to use an additional index, not just worlds and times
but also locations.  But if he had done this for temperature and
maintained a general theory of intensions he would
have had to make all intensions be functions defined on triples of worlds, times
and locations and this would have raised issues about the relationship
between intensionality and indexicality which he was probably wise to
avoid at that point in the development.  Nevertheless, it is an
important issue which nags at some of the central assumptions of 
formal semantics as Montague was proposing it: namely, the use of possible worlds and evaluation
with respect to a finite set of indices some of which are in the
domain of intensions and some of which are contextual parameters.

L�bner's early work on this topic \citep{Lobner1979,Lobner1981}
treated this problem by removing what he called \textit{functional
  concepts} (\textit{Funktionalbegriffe}) from the general notion of
intension and allowing them to have different numbers and 
types of argument roles.  These insights led him in later work
\citep{Lobner2014,Loebner2015} to adopt a frame semantic approach
where the parameters that are relevant for interpretation can vary
between different words and phrases and there is no fixed set of
indices as there was in the original work on formal semantics.  This
is very much the same kind of proposal as in
\cite{Cooper2010,Cooper2012} although the historical precursors we had in
mind were different.  In my case, the precursors were early work on
situation semantics such as \cite{BarwisePerry1983} and frame
semantics of the kind suggested in \cite{Fillmore1982,Fillmore1985}
and taken as a foundation for FrameNet
\citep[][\url{https://framenet.icsi.berkeley.edu}]{RuppenhoferEllsworthPetruckJohnsonScheffczyk2006}.
In L�bner's case, the inspiration for frames comes from the
psychological work of \cite{Barsalou1992,Barsalou1992a,Barsalou1999}.

\section{Frames as records}

Our leading idea in modelling frames is that they correspond to
records and that the \textit{roles} (or \textit{frame elements} in the
terminology of FrameNet) are represented by the record fields.
Records are in turn what we use to model situations so frames and
situations in our view turn out to be the same.  Given that we are
working in a type theory which makes a clear distinction between types
and the objects which belong to those types it is a little unclear
whether what we call frame should be a record or a record type.  We
need both and we will talk of frames (records) and frame types (record
types).  For example, when we look up the frame Ambient\_temperature
(\url{https://framenet2.icsi.berkeley.edu/fnReports/data/frameIndex.xml?frame=Ambient_temperature})
in FrameNet we will take that to be an informal description of a frame
type which can be instantiated by the kinds of situations which are
described in the examples there.  In our terms we can characterize a
type corresponding to a very stripped down version of FrameNet's
Ambient\_temperature which is sufficient for us to make the argument
we wish to make.  This is the type \textit{AmbTempFrame} defined in \nexteg{}.
\begin{ex} 
\record{\tfield{x}{\textit{Real}} \\
        \tfield{loc}{\textit{Loc}} \\
        \tfield{e}{temp(loc, x)}}
\label{ex:AmbTempFrame} 
\end{ex}



This is different from the earlier proposal we made in
\cite{Cooper2012} which is given in \nexteg{}.
\begin{ex} 
\record{\tfield{x}{\textit{Ind}} \\
        \tfield{e-time}{\textit{Time}} \\
        \tfield{e-location}{\textit{Loc}} \\
        \tfield{c$_{\mathrm{temp\_at\_in}}$}{temp\_at\_in(e-time, e-location, x)}} 
\end{ex} 
The new proposal in (\ref{ex:AmbTempFrame}) differs from the old one in two
ways.  Firstly we have removed the field for time.  This is because we
now want to treat time in terms of strings of events rather than
introducing time-points as such. This follows 
Fernando's strategy (for example in \citealp{Fernando2011}) and
relates to the discussion of the Russell-Wiener construction of time
in \cite{Kamp1979}.  Secondly we have made the type in the `x'-field
(the field which will contain `ninety' in our example) be
\textit{Real} (``real number'')  rather than \textit{Ind}
(``individual'').  As \cite{Lasersohn2005a} points out the issue was
raised early in the literature as to whether numbers (or temperature
measurements at any rate) should be treated
as individuals in these examples or should be counted as belonging to
a separate type \citep{Bennett1974,Thomason1979}.  In our earlier work
we assumed that temperatures were to be considered as individuals
because we had no reason to do otherwise.  In the current analysis,
however, we want to build in a notion of scale which involves a
mapping to real numbers and therefore we will model temperatures as
real numbers.  As we will see this will lead to a slight complication
in the compositional semantics so there is still an open issue as to
whether this is the right decision.

A scale is a function which maps frames (situations) to a real
number.  Thus a scale for ambient temperature will be of the type
\nexteg{a} and the obvious function to choose of that type is the
function in \nexteg{b} which maps any ambient temperature frame to the
real number in its `x'-field.
\begin{ex} 
\begin{subex} 
 
\item (\textit{AmbTempFrame} $\rightarrow$ \textit{Real}) 
 
\item $\lambda r$:\textit{AmbTempFrame} . $r$.x 
 
\end{subex} 
\label{ex:scaleambtempframe}   
\end{ex} 
Let us call \preveg{b} $\zeta_{\mathrm{temp}}$.  As a first approximation we can take an event of
a temperature rise to be a string of two temperature frames,
$r_1^\frown r_2$, where $\zeta_{\mathrm{temp}}(r_1) <
\zeta_{\mathrm{temp}}(r_2)$.  Using a notation where $T^n$ is the type of
strings of length $n$ each of whose members are of type $T$ and where
for a given string, $s$, $s[0]$ is the first member of $s$, $s[1]$ the
second and so on, a first approximation to the type of temperature
rises could be \nexteg{}. 
\begin{ex} 
\record{\tfield{e}{\textit{AmbTempFrame}$^2$}\\
        \tfield{c$_{\mathrm{rise}}$}{$\zeta_{\mathrm{temp}}$(e[0]) $<$ $\zeta_{\mathrm{temp}}$(e[1])}}
\end{ex} 
In the c$_{\mathrm{rise}}$-field of \preveg{} we are using $<$ as an
infix notation for a predicate `less-than' with arity
$\langle$\textit{Real}, \textit{Real}$\rangle$ which obeys the
constraint in \nexteg{}.
\begin{ex} 
less-than($n$, $m$) is non-empty (``true'') iff $n<m$ 
\end{ex}
One way to meet this constraint is to specify the witness conditions
for `less-than($n$, $m$)' as in \nexteg{}.
\begin{ex} 
$e$ : less-than($n$, $m$) iff $n\varepsilon e$, $m\varepsilon e$ and $n<m$ 
\end{ex} 
  

A more general type for temperature rises is given by \nexteg{} where
we abstract away from the particular temperature scale used by
introducing a field for the scale into the record type.  This, for
example, allows for an event to be a temperature rise independent of
whether it is measured on the Fahrenheit or Celsius scales.  
\begin{ex} 
\record{\tfield{scale}{(\textit{AmbTempFrame} $\rightarrow$ \textit{Real})}\\
        \tfield{e}{\textit{AmbTempFrame}$^2$}\\
        \tfield{c$_{\mathrm{rise}}$}{scale(e[0]) $<$ scale(e[1])}} 
\label{ex:ambtempwithscale}
\end{ex} 
This type, though, is now too general to count as the type of
temperature rising events.  To be of this type, it is
enough for there to be some scale on which the rise condition holds
and the scale is allowed to be any arbitrary function from temperature
frames to real numbers.  Of course, it is possible to find some
arbitrary function which will meet the rise condition even if the
temperature is actually going down.  For example, consider a function
which returns the number on the Celsius scale but with the sign (plus
or minus)
reversed making temperatures above 0 to be below 0 and \textit{vice
  versa}. There are two ways we can approach this problem.  One is to
make the type in the scale-field a subtype of (\textit{AmbTempFrame}
$\rightarrow$ \textit{Real}) which limits the scale to be one of a
number of standardly
accepted scales.  This may be an obvious solution in the case of
temperature where it is straightforward to identify the commonly used
scales.  However, scales are much more generally used in linguistic
meaning and people create new scales depending on the situation at
hand. % (Horn ????)
This makes it difficult to specify the
nature of the relevant scales in advance and we therefore  prefer our
second way of approaching this problem.

The second way is to parametrize the type of temperature rising
events.  By this we mean using a dependent type which maps a record
providing a scale to
a record type modelling the type of temperature rising events
according to that scale.  The function in \nexteg{} is a dependent
type which is related in an obvious way to the record type in
\preveg{}.
\begin{ex} 
$\lambda r$:\smallrecord{\smalltfield{scale}{(\textit{AmbTempFrame}
    $\rightarrow$ \textit{Real})}} .\\  
\hspace*{2em}\record{\tfield{e}{\textit{AmbTempFrame}$^2$}\\
        \tfield{c$_{\mathrm{rise}}$}{$r$.scale(e[0]) $<$ $r$.scale(e[1])}}
\end{ex} 
According to (\ref{ex:ambtempwithscale}) an event will be a
temperature rise if there is some scale according to which the appropriate
relation holds between the temperatures of the two stages of the event
which we are comparing.  According to \preveg{} on the other hand,
there is no absolute type of a temperature rise.  We can only say
whether an event is a temperature rise with respect to some scale or
other.  If we choose some non-standard scale like the one that
reverses plus and minus temperatures as we suggested above then what
we normally call a fall in temperature will in fact be a rise in
temperature \textit{according to that scale}.  You are in principle
allowed to choose whatever scale you like, though if you are using the
type in a communicative situation you had better make clear to your
interlocutor what scale you are using and perhaps also why you are
using this scale as opposed to one of the standardly accepted ones.
Like the parametric contents we introduced in
Chapter~\ref{ch:propnames}, the dependent types introduce a
presupposition-like component to communicative situations.  We are
assuming the existence of some scale in the context.

Why do we characterize the domain of the function in \preveg{} in
terms of records containing a scale rather than just scales as in
\nexteg{}?
\begin{ex} 
$\lambda \sigma$:(\textit{AmbTempFrame}
    $\rightarrow$ \textit{Real}) .\\  
\hspace*{2em}\record{\tfield{e}{\textit{AmbTempFrame}$^2$}\\
        \tfield{c$_{\mathrm{rise}}$}{$\sigma$(e[0]) $<$ $\sigma$(e[1])}} 
\end{ex} 
The intuitive reason is that we want to think of the arguments to such
functions as being contexts, that is situations (frames) modelled as
records.  The scale will normally be only one of many informational
components which can be provided by the context and the use of a
record type allows for there to be more components present.  In
practical terms of developing an analysis it is useful to use a record
type to characterize the domain even if we have only isolated one
parameter since if further analysis should show that additional
parameters are relevant this will mean that we can add fields to the
domain type thereby restricting the domain of the function rather than
giving it a radically different type.

And indeed in this case we will now show that there is at least one
more relevant parameter that needs to be taken account of before we
have anything like a reasonable account of the type of temperature
rise events.  In (\ref{ex:AmbTempFrame}) we specified that an ambient
temperature frame relates a real number (``the temperature'') to a
spatial location.  And now we are saying that a temperature rise is a
string of two such frames where the temperature is higher in the
second frame.  But we have not said anything about how the locations
in the two frames should be related.  For example, suppose I have a
string of two temperature frames where the location in the first is
London and the location in the second is Marrakesh.  Does that
constitute a rise in temperature (assuming that the temperature in the
second frame is higher than the one in the first)?  Certainly not a
temperature rise in London, nor in Marrakesh.  If you want to talk
about a temperature rise in a particular location then both frames
have to have that location and we need a way of expressing that
restriction.  Of course, you can talk about temperature rises which
take place as you move from one place to another and which therefore
seem to involve distinct locations.  However, it seems that even in
these cases something has to be kept constant between the two frames.
One might analyse it in terms of a constant path to which both
locations have to belong or as a constant relative location such as
the place where a particular person (or car, or airplane) is.  You
cannot just pick two arbitrary temperature frames without holding
something constant which ties them together.  We will deal here with
the simple case where the location is kept constant.\footnote{Although
  in astronomical terms, of course, even a location like London is a
  relative location, that is, where London is according to the
  rotation of the earth and its orbit around the sun.  Thus the simple
  cases are not really different from the cases apparently involving
  paths.}  We will say that the background information for judging an
event as a temperature rise has to
include not only a scale but also a location which is held constant in
the two frames.  This is expressed in \nexteg{}.
\begin{ex} 
$\lambda r$:\smallrecord{\smalltfield{fix}{\smallrecord{\smalltfield{loc}{\textit{Loc}}}}\\
                         \smalltfield{scale}{(\textit{AmbTempFrame}
                             $\rightarrow$ \textit{Real})}} .\\  
\hspace*{2em}\record{\tfield{e}{(\textit{AmbTempFrame}\d{$\wedge$}\smallrecord{\smallmfield{loc}{$r$.fix.loc}{\textit{Loc}}})$^2$}\\
        \tfield{c$_{\mathrm{rise}}$}{$r$.scale(e[0]) $<$ $r$.scale(e[1])}} 
\label{ex:temprisefixloc}
\end{ex} 
Here the `fix'-field in the context is required to be a record which
provides a location.  One reason for making the `fix'-field a record
rather than simply a location is that we will soon see an example
where more than one parameter needs to be fixed.  It will also help us
ultimately in characterizing a general type for a rising event (not
just a rise in temperature) if we can refer to the type in the
`fix'-field as \textit{Rec} (``record'') rather than to list a
disjunction of all the various types of the parameters that can be
held constant in different cases.

The temperature rise event itself is now required to be a string of
two frames which belong to a subtype of \textit{AmbTempFrame}, namely
where the `loc'-field has been made manifest and is specified to have
the value specified for `loc' in the `fix'-field.  Here we are using
the record in the `fix'-field of the argument to the function to
partially specify the type \textit{AmbTempFrame} by fixing values for
some of its fields.  One can think of the `fix'-record as playing the
role of a partial
assignment of values to fields in the type.  To emphasize this
important role and to facilitate making general statements without
having to name the particular fields involved, we shall use the
operation of specification or anchoring introduced on p.~\pageref{pg:anchoring}ff. This operation maps a record
type, $T$, and a record, $r$ to the result of specifying $T$ with $r$,
which we notate as $T\parallel r$.
% \begin{ex} 
% \smallrecord{\smalltfield{$\ell_1$}{$T_1$}\\
%              \smalltfield{$\ell_2$}{$T_2$}\\
%              \smalltfield{$\ell_3$}{$T_3$}}$\parallel$
% \smallrecord{\field{$\ell_2$}{$a$}\\
%              \field{$\ell_3$}{$b$}\\
%              \field{$\ell_4$}{$c$}} = 
% \smallrecord{\smalltfield{$\ell_1$}{$T_1$}\\
%              \smallmfield{$\ell_2$}{$a$}{$T_2$}\\
%              \smallmfield{$\ell_3$}{$b$}{$T_3$}}
%
% provided that $a:T_2$ and $b:T_3$ 
% \end{ex} 
% In a case where for example $a:T_2$ but not $b:T_3$ we would have
% \nexteg{}.
% \begin{ex} 
% \smallrecord{\smalltfield{$\ell_1$}{$T_1$}\\
%              \smalltfield{$\ell_2$}{$T_2$}\\
%              \smalltfield{$\ell_3$}{$T_3$}}$\parallel$
% \smallrecord{\field{$\ell_2$}{$a$}\\
%              \field{$\ell_3$}{$b$}\\
%              \field{$\ell_4$}{$c$}} = 
% \smallrecord{\smalltfield{$\ell_1$}{$T_1$}\\
%              \smallmfield{$\ell_2$}{$a$}{$T_2$}\\
%              \smalltfield{$\ell_3$}{$T_3$}}
%
% 
% \end{ex}   
% The result \preveg{} would also have obtained if $T_3$ had not been a
% type but a pair consisting of a dependent type and a sequence of
% paths, that is, the kind of thing which in our standard abbreviation
% we represent as a predicate with a label as argument such as
% `walk($\ell_1$)'.  A precise definition of this operation is given in
% Appendix~\ref{app:specrec}.
Using this notation we can rewrite (\ref{ex:temprisefixloc}) as
\nexteg{}.
\begin{ex} 
$\lambda r$:\smallrecord{\smalltfield{fix}{\smallrecord{\smalltfield{loc}{\textit{Loc}}}}\\
                         \smalltfield{scale}{(\textit{AmbTempFrame}
                             $\rightarrow$ \textit{Real})}} .\\  
\hspace*{2em}\record{\tfield{e}{(\textit{AmbTempFrame}$\parallel$$r$.fix)$^2$}\\
        \tfield{c$_{\mathrm{rise}}$}{$r$.scale(e[0]) $<$ $r$.scale(e[1])}} 
\label{ex:temprisespecfix}
\end{ex}
We will call \preveg{} \textit{TempRiseEvent} and its domain type
\textit{TempRiseEventCntxt}.  This means that given a record, $c$, of
type \textit{TempRiseEventCntxt} we can make a judgement such as that
given in \nexteg{}.
\begin{ex} 
$e$ : \textit{TempRiseEvent}$(c)$ 
\end{ex} 
That is, we judge that $e$ is a temperature rising event according to
the context $c$.  

This is still a very simple theory of what a temperature rise event
may be but it will be sufficient for our current purposes.  How might
we use this to specify a content for the intransitive verb
\textit{rise} in a sentence like \textit{the temperature is rising}?
First we define a predicated `rise' which takes two arguments which
are both records.  Thus the arity of `rise' is $\langle$\textit{Rec},
\textit{Rec}$\rangle$.  The first record can be an ambient temperature
frame and the second a temperature rise event context.  We specify
a witness condition associated with `rise' in \nexteg{}.
\begin{ex} 
  $e$ : rise$(r,c)$ if
  \begin{quote}
    $r$ : \textit{AmbTempFrame},\\
    $c$ : \textit{TempRiseEventCntxt} and \\
    $e$ : \textit{TempRiseEvent}$(c)$ \d{$\wedge$}
    \smallrecord{
      \smalltfield{e}{\smallrecord{
          \smallmfield{t$_0$}{$r$}{\textit{AmbTempFrame}}}}}
  \end{quote}
  
\end{ex} 
Note first that in \preveg{} we use `if' rather than `iff'.  This is
not the only witness condition which we will associate with `rise'; it
represents a sufficient but not necessary condition.  Note also that
the third condition specifies that $e$ is a temperature rise event and
that the first item in the string of two ambient temperature frames
thus specified is the temperature frame $r$, that is, the first
argument to the predicate.  (Recall that a string $e_1e_2$ is modelled
as the record in \nexteg{}.)
\begin{ex} 
  \record{
    \field{t$_0$}{$e_1$}\\
    \field{t$_1$}{$e_2$}}
\end{ex} 
The intuition is, then, that when we predicate `rise'  of an
ambient temperature frame, we are saying that it is the initial frame
in a temperature rising event.  We can use this predicate in the
characterization of a parametric content for the verb \textit{rise},
given in \nexteg{}. 
\begin{ex} 
  $\ulcorner\lambda c$:\smallrecord{
    \smalltfield{$\mathfrak{c}$}{\textit{TempRiseEventCntxt}}} .
  $\lambda r$:\smallrecord{\smalltfield{x}{\textit{Rec}}} . \record{
    \tfield{e}{rise($r$, $c.\mathfrak{c}$)}}$\urcorner$
\end{ex} 
However, the verb \textit{rise} can be used to talk about other kinds
of rising events than temperature rises and we will need different
parametric contents for other cases.  

We move on
now to price rise events.  We will take \nexteg{} to be the type of
price frames, \textit{PriceFrame}.
\begin{ex} 
\record{\tfield{x}{\textit{Real}} \\
        \tfield{loc}{\textit{Loc}} \\
        \tfield{commodity}{\textit{Ind}} \\
        \tfield{e}{price(commodity, loc, x)}} 
\end{ex} 
The fields represented here are based on a much stripped down version
of the FrameNet frame \texttt{Commerce\_scenario} where our
`commdodity'-field corresponds to the frame element called `goods' and
the `x'-field corresponds to the frame element `money'. A price rise
is a string of two price frames where the value in the `x'-field is
higher in the second.  Here, as in the case of a temperature rise, we need to keep
the location constant.  It does not make sense to say that a price
rise has taken place if we compare a price in Marrakesh with a price
in London, even though the price in London may be higher.  In the case
of price we also need to keep the commodity constant, something that
does not figure at all in ambient temperature.  We cannot say that a
price rise has taken place if we have the price of tomatoes in the
first frame and the price of oranges in the second frame.  Thus,
following the model of (\ref{ex:temprisespecfix}), we can characterize
the dependent type of price rises as \nexteg{}.
\begin{ex} 
$\lambda
r$:\smallrecord{\smalltfield{fix}{\smallrecord{\smalltfield{loc}{\textit{Loc}}\\
                                               \smalltfield{commodity}{\textit{Ind}}}}\\
                         \smalltfield{scale}{(\textit{PriceFrame}
                             $\rightarrow$ \textit{Real})}} .\\  
\hspace*{2em}\record{\tfield{e}{(\textit{PriceFrame}$\parallel$$r$.fix)$^2$}\\
        \tfield{c$_{\mathrm{rise}}$}{$r$.scale(e[0]) $<$ $r$.scale(e[1])}}  
    \end{ex}
We call \preveg{} \textit{PriceRiseEvent} and its domain type
\textit{PriceRiseEventCntxt}.  We can add a new witness condition
associated with `rise'.
\begin{ex} 
  $e$ : rise$(r,c)$ if
  \begin{quote}
    $r$ : \textit{PriceFrame},\\
    $c$ : \textit{PriceRiseEventCntxt} and \\
    $e$ : \textit{PriceRiseEvent}$(c)$ \d{$\wedge$}
    \smallrecord{
      \smalltfield{e}{\smallrecord{
          \smallmfield{t$_0$}{$r$}{\textit{PriceFrame}}}}}
  \end{quote}

\end{ex}

We can construct a parametric content for the verb \textit{rise} which
exploits this witness condition.  This is given in \nexteg{}.
\begin{ex} 
  $\ulcorner\lambda c$:\smallrecord{
    \smalltfield{$\mathfrak{c}$}{\textit{PriceRiseEventCntxt}}} .
  $\lambda r$:\smallrecord{\smalltfield{x}{\textit{Rec}}} . \record{
    \tfield{e}{rise($r$, $c.\mathfrak{c}$)}}$\urcorner$
\end{ex} 

Finally we consider a third kind of rising event discussed in
\cite{Cooper2012} based on the example in \nexteg{}.
\begin{ex} 
As they get to deck, they see the Inquisitor, calling out to a Titan
in the seas. The giant Titan rises through the waves, shrieking at the
Inquisitor.

\medskip

\hfill {\footnotesize
  \url{http://en.wikipedia.org/wiki/Risen_(video_game)}}

\hfill {\footnotesize accessed 4th February, 2010} 
\end{ex}   
Here what needs to be kept constant in the rising event is the Titan.
What needs to change between the two frames in the event is the height
of the location of the Titan.  Thus in this example the
location is \textit{not} kept constant. In order to analyze this we
can use location frames of the type \textit{LocFrame} as given in
\nexteg{}.
\begin{ex} 
\record{\tfield{x}{\textit{Ind}} \\
        \tfield{loc}{\textit{Loc}} \\
        \tfield{e}{at(x, loc)}} 
\end{ex} 
The dependent type, \textit{LocRiseEvent}, with domain type \textit{LocRiseEventCntxt}, for a rise in location event is \nexteg{}.
\begin{ex} 
$\lambda
r$:\smallrecord{\smalltfield{fix}{\smallrecord{\smalltfield{x}{\textit{Ind}}
                                               }}\\
                         \smalltfield{scale}{(\textit{LocFrame}
                             $\rightarrow$ \textit{Real})}} .\\  
\hspace*{2em}\record{\tfield{e}{(\textit{LocFrame}$\parallel$$r$.fix)$^2$}\\
        \tfield{c$_{\mathrm{rise}}$}{$r$.scale(e[0]) $<$ $r$.scale(e[1])}}  
\end{ex} 
Here the obvious scale function does not simply return the value of a
field in the location frame.  What is needed is a scale based on the
height of the location.  One way to do this would be to characterize
the type of locations, \textit{Loc}, as the type of points in three-dimensional
Euclidean space.  That is, we consider \textit{Loc} to be an
abbreviation for \nexteg{}.
\begin{ex} 
\record{\tfield{x-coord}{\textit{Real}}\\
        \tfield{y-coord}{\textit{Real}}\\
        \tfield{z-coord}{\textit{Real}}}
\end{ex} 
Each of the fields in \preveg{} corresponds to a coordinate in
Euclidean space.  A more adequate treatment would be to consider
locations as regions in Euclidean space but we will not pursue that
here.  Treating \textit{Loc} as \preveg{} means that we can
characterize the scale function, $\zeta_{\mathrm{height}}$, as returning the height of the
location in the location frame, as in \nexteg{}.
\begin{ex} 
$\lambda r$:\textit{LocFrame} . $r$.loc.z-coord 
\end{ex} 
If we wish to restrict the dependent type of rising events to vertical
rises we can fix the $x$ and $y$-coordinates of the location as in
\nexteg{}.
\begin{ex} 
$\lambda
r$:\smallrecord{\smalltfield{fix}{\smallrecord{\smalltfield{x}{\textit{Ind}}\\
                                               \smalltfield{loc}{\smallrecord{\smalltfield{x-coord}{\textit{Real}}\\
                                                                              \smalltfield{y-coord}{\textit{Real}}}}
                                               }}\\
                         \smalltfield{scale}{(\textit{LocFrame}
                             $\rightarrow$ \textit{Real})}} .\\  
\hspace*{2em}\record{\tfield{e}{(\textit{LocFrame}$\parallel$$r$.fix)$^2$}\\
        \tfield{c$_{\mathrm{rise}}$}{$r$.scale(e[0]) $<$ $r$.scale(e[1])}}  
    \end{ex}

We can now add a new witness condition associated with `rise', given
in \nexteg{}.
\begin{ex} 
  $e$ : rise$(r,c)$ if
  \begin{quote}
    $r$ : \textit{LocFrame},\\
    $c$ : \textit{LocRiseEventCntxt} and \\
    $e$ : \textit{LocRiseEvent}$(c)$ \d{$\wedge$}
    \smallrecord{
      \smalltfield{e}{\smallrecord{
          \smallmfield{t$_0$}{$r$}{\textit{LocFrame}}}}}
  \end{quote}
  
\end{ex}
We can use this predicate to create a parametric content for the
intransitive verb \textit{rise}, as in \nexteg{}.
\begin{ex} 
  $\ulcorner\lambda c$:\smallrecord{
    \smalltfield{$\mathfrak{c}$}{\textit{LocRiseEventCntxt}}} .
  $\lambda r$:\smallrecord{\smalltfield{x}{\textit{Rec}}} . \record{
    \tfield{e}{rise($r$.x, $c.\mathfrak{c}$)}}$\urcorner$
\end{ex}

We have now characterized three kinds of rising events.  In
\cite{Cooper2010,Cooper2012}  we argued that there is in principle no
limit to the different kinds of rising events which can be referred to
in natural language and that new types are created on the fly as the
need arises.  The formulation in those works did not allow us to
express what all these particular meanings have in common.  We were
only able to say that the various meanings seem to have some kind of
family resemblance. Now that we have abstracted out scales and
parameters to be fixed we have an opportunity to formulate something
more general.  There are two things that vary across the different
dependent types that we have characterized for risings.  One is the
frame type being considered and the other is the type of the record which
contains the parameters held constant in the rising event. If we
abstract over both of these we have a characterization of rising
events in general. This is given in \nexteg{}.
\begin{ex} 
$\lambda
r$:\smallrecord{\smalltfield{frame\_type}{\textit{RecType}}\\
                \smalltfield{fix\_type}{\textit{RecType}}\\
\smalltfield{fix}{fix\_type}\\
                         \smalltfield{scale}{(frame\_type
                             $\rightarrow$ \textit{Real})}} .\\  
\hspace*{2em}\record{\tfield{e}{($r$.frame\_type$\parallel$$r$.fix)$^2$}\\
        \tfield{c$_{\mathrm{rise}}$}{$r$.scale(e[0]) $<$ $r$.scale(e[1])}}  
\end{ex}

\preveg{} is so general (virtually everything of content has been
parametrized) that it may be hard to see it as being used in the
characterization of the meaning of \textit{rise}.  What seems
important for characterizing the meanings of \textit{rise} that a
speaker has acquired is precisely the collection of frame types, and
associated fix types and scales which an agent has developed through
experience.  \preveg{} seems to be relevant to a kind of meta-meaning
which specifies what kind of contents can be associated with the word
\textit{rise}.  In this sense it seems related to the notion of
\textit{meaning potential}, a term which has its origins in the work
of \cite{Halliday1977} where meanings are spoken of informally as
being ``created by the social system'' and charaterized as
``integrated systems of meaning potential'' (p. 199).  The notion is much
discussed in more recent literature, for example, \cite{Linell2009}, 
where meaning potential is discussed in the following terms: ``Lexical
meaning potentials are (partly) open meaning resources, where actual
meanings can only emerge in specific, situated interactions''
(p. 330).  The parametric contents for \textit{rise} that we have
presented here (and included in the summary of resources in
Section~\ref{sec:summresch5}) are examples of what Linell is calling here ``actual meanings''.

\section{Frames and common nouns}
\label{sec:compsemPartee}

A central aspect of our analysis of the Partee puzzle is that the
contents of common nouns are functions that take frames, that is
records, as arguments.  Nevertheless, we make a distinction between
individual level predicates like `dog' whose arity is
$\langle\textit{Ind}\rangle$ and frame level predicates like
`temperature' whose arity is $\langle\textit{Rec}\rangle$.  Leaving
aside for now the need for parametric contents, the content for
associated with an utterance event of type ``dog'' would be
(\ref{ex:non-paradog}a) repeated here as \nexteg{a}.  This is contrasted with
the content for an utterance of type ``temperature'' given in
\nexteg{b}.
\begin{ex} 
\begin{subex} 
 
\item $\lambda
  r$:\smallrecord{\smalltfield{x}{\textit{Ind}}} . \record{\tfield{e}{dog($r$.x)}}  
 
\item $\lambda
  r$:\smallrecord{\smalltfield{x}{\textit{Rec}}} . \record{\tfield{e}{temperature($r$.x)}} 
 
\end{subex} 
\label{ex:dogtemperaturecont}   
\end{ex} 
We make an exactly similar distinction between individual level and
frame level verb phrases.  In \nexteg{} we present contents which can
be associated with utterances of type ``run'' and ``rise''
respectively.
\begin{ex} 
\begin{subex} 
 
\item $\lambda
  r$:\smallrecord{\smalltfield{x}{\textit{Ind}}}
  . \record{\tfield{e}{run($r$.x)}} 
 
\item $\lambda
  r$:\smallrecord{\smalltfield{x}{\textit{Rec}}} . \record{\tfield{e}{rise($r$.x)}} 
 
\end{subex} 
   
\end{ex} 
The types which we associate with the individual level and frame level properties in
(\ref{ex:dogtemperaturecont}) and \preveg{} are given in \nexteg{}.
\begin{ex} 
\begin{subex} 
 
\item (\smallrecord{\smalltfield{x}{\textit{Ind}}} $\rightarrow$ \textit{RecType})
 
\item (\smallrecord{\smalltfield{x}{\textit{Rec}}} $\rightarrow$ \textit{RecType}) 
 
\end{subex} 
\label{ex:pptytypes}   
\end{ex} 
While these types are distinct, they are nevertheless related in that
they both have the same range type and the domain types of \preveg{a}
and \preveg{b} are both record types requiring a field with the label `x'.  Up until now we have
used \textit{Ppty} (``property'') to designate \preveg{a}.  Now we
might be more specific and designate it as \textit{IndPpty}
(``property of individuals'') and use \textit{FramePpty} (``property
of frames'') to designate \preveg{b}. % In our previous treatment of the
% temperature puzzle both individual level and frame level properties
% were of the type \preveg{a} because we treated numbers as individuals,
% that is, as being of type \textit{Ind}.  On this view
% \textit{AmbTempFrame} can be
% defined as \nexteg{a} rather than our current proposal repeated in
% \nexteg{b}.
% \begin{ex} 
% \begin{subex} 
 
% \item  \record{\tfield{x}{\textit{Ind}} \\
%         \tfield{loc}{\textit{Loc}} \\
%         \tfield{e}{temp(loc, x)}}
 
% \item \record{\tfield{x}{\textit{Real}} \\
%         \tfield{loc}{\textit{Loc}} \\
%         \tfield{e}{temp(loc, x)}} 
 
% \end{subex} 
   
% \end{ex} 
% Choosing \preveg{a} rather than \preveg{b} would mean that the
% distinction between individual level and frame level properties would
% not be one of the type of properties as such (since they would both be
% of type (\ref{ex:pptytypes}a)) but rather in the arity of the
% predicate used within the record type that they returned, that is, for
% example, $\langle$\textit{Ind}$\rangle$ for `dog' and
% $\langle$\textit{Rec}$\rangle$ for `temperature'.  This represents an
% appealing feature of using record types with subtyping, namely that fine-grained type
% distinctions can be introduced by predicates within record types which
% all belong to the same type \textit{RecType}. For a compositional
% semantics this means  that fine grained type distinctions associated
% with lexical items need not be reflected in the types of the contents
% of the phrases in which those lexical items occur.  This is in
% significant contrast to the simple type theory used by Montague where
% there was not subtyping and any type distinction introduced in a
% constituent would be reflected as a type distinction in higher level
% phrases.  To exploit this feature of the type system here we would
% have to treat (real) numbers as individuals.  This would not
% necessarily mean abandoning the type \textit{Real} but it would mean
% stipulating that \textit{Real} is a subtype of \textit{Ind}.  For
% example, we could let \textit{AmbTempFrame} be \preveg{a} but require
% that the predicate `temp' used in this type have arity
% \mbox{$\langle$\textit{Loc}, \textit{Real}$\rangle$}.  This, together with the
% requirement \textit{Real}$\sqsubseteq$\textit{Ind}, would mean that
% \preveg{a} and \preveg{b} would be equivalent in the sense that they
% would have the same set of witnesses.

% The alternative sketched above where numbers are treated as
% individuals has much to commend it.  But nevertheless we have not
% chosen it here for a number of intuitive and practical reasons:
% \begin{enumerate} 
 
% \item There is a fairly robust intuition that numbers are not, in
%   fact, individuals. 
 
% \item The proposed solution involves stipulating a subtype relation
%   between basic types.  While this is not ruled out in TTR, we would
%   like to keep it to a minimum and not use it where there is an
%   alternative of analyzing the subtyping in terms of record types.
%   Using record types you can see (and compute) whether one type is a
%   subtype of another whereas subtyping between basic types is not
%   explicit in the representation of the types.

% \item There are types in TTR of which both types in
%   (\ref{ex:pptytypes}) are subtypes and these types are candidates for
%   the general type \textit{Ppty}, i.e. properties in general as
%   opposed to properties of particular types of objects. 
 
% \end{enumerate} 
% The last point here requires some explanation.
Given that TTR has join
(disjunctive) types (Appendix~\ref{app:jointypes}) we always have the
option of forming the join of those types which we want to represent
types of properties.  Thus, given the two types of properties we have
seen so far we can form the join type in \nexteg{}.
\begin{ex} 
((\smallrecord{\smalltfield{x}{\textit{Ind}}} $\rightarrow$
\textit{RecType}) $\vee$ (\smallrecord{\smalltfield{x}{\textit{Rec}}} $\rightarrow$
\textit{RecType}))
\label{ex:indorrecppty} 
\end{ex} 
If there are more types of properties we wish to add to the general
type of properties we can form a larger join type to include them.  We
can always form a join type based on any finite collection of types.
Using join types in this way we can create a type which has all the
witnesses of any finite collection of types.  We cannot, however, express a type
corresponding to an infinite set of types in this way.  In addition
using join types in this way does not make
explicit any relationship between the various types in the collection,
in this case that all the types are function types
whose range type is \textit{RecType} and whose domain type is a
record type with an `x'-field.  In order to deal with this
kind of case, we will use the same technique as we used for parametric
contents in Chapter~\ref{ch:propnames}.  We will first define the type
\textit{xType} as a basic type of record types that have a `x' among its
labels, as specified in \nexteg{}.
\begin{ex} 
$T$ : \textit{xType} iff $T$ : \textit{RecType} and $\text{x}\in\mathrm{labels}(T)$ 
\end{ex} 
  We will treat properties as a
pair (that is, a record with two fields) consisting of a type (labelled with `bg') and a function
(labelled with `fg') corresponding to what we have up to now
been calling a property.  \nexteg{a} is an example of the new kind of
property which we normally represent as \nexteg{b} and \nexteg{c} is the new definition of the type,
\textit{Ppty}, of properties.\footnote{% Many type
  % theorists find types such as (\ref{ex:pptyT}) problematic since they
  % involve quantification over subtypes.  If you are one of those and
  % you do not mind there only being a finite number of property types then
  % you can retract to the kind of join type in (\ref{ex:indorrecppty}).
  % If you are a type theorist who also finds this kind of type
  % problematic (and there are many) or you want to have a type
  % corresponding to an infinite variety of properties then you can
  % think of \textit{Ppty} as being a function of two arguments which is
  % defined when the second argument is a subtype of
  % \smallrecord{\smalltfield{x}{$T$}}, where $T$ is the first argument.  Thus
  % \textit{Ppty}($T$)($T'$) would yield the type
  % ($T'$ $\rightarrow$
  % \textit{RecType}).
  For a similar kind of case, though a different approach to treating
  it, see the discussion in \cite{GinzburgCooperFernando2014}, p. 93,
  of the type of Austinian questions.  We have also treated this kind
  of case in terms of a limited kind of polymorphism, for example, in \cite{GinzburgCooper2014}.}
\begin{ex} 
\begin{subex} 
 
\item \record{\field{bg}{\smallrecord{\smalltfield{x}{\textit{Ind}}}\\
              \field{fg}{$\lambda
                r$:\smallrecord{\smalltfield{x}{\textit{Ind}}}
                . \smallrecord{\smalltfield{e}{dog($r$.x)}}}}}

          
        \item $\ulcorner\lambda
                r$:\smallrecord{\smalltfield{x}{\textit{Ind}}}
                . \smallrecord{\smalltfield{e}{dog($r$.x)}}$\urcorner$
 
\item \record{\tfield{bg}{\textit{xType}}\\
              \tfield{fg}(bg$\rightarrow$\textit{RecType})} 
 
\end{subex} 
\label{ex:dog-Ppty}   
\end{ex}
If $P$ is a property as in \preveg{a,b}, we will use $P(r)$ to
represent $P.\text{fg}(r)$ (the result of applying the function $P$.fg
to $r$).
\begin{shaded}
  Actually, things are a little more complicated than this.  Using
  this polymorphic type for properties introduces a complication when
  we define functions on properties. This is illustrated in the
  example in \nexteg{} where we assume that $a$:\textit{Ind}, $p$
  is a predicate whose arity is $\langle\textit{Rec}\rangle$ and that
  the types \textit{Ind} and \textit{Rec} preclude each other, that
  is, nothing can be of both type \textit{Ind} and type \textit{Rec}.
  \begin{ex} 
\begin{subex} 
 
\item $\lambda P$:\textit{Ppty} . $P$(\smallrecord{\field{x}{$a$}})
  ($\ulcorner\lambda r$:\smallrecord{\smalltfield{x}{\textit{Rec}}} . \record{\tfield{e}{$p$($r$.x)}}$\urcorner$) 
 
\item $\ulcorner\lambda r$:\smallrecord{\smalltfield{x}{\textit{Rec}}}
  . \record{\tfield{e}{$p$($r$.x)}}$\urcorner$ (\smallrecord{\field{x}{$a$}}) 
 
\end{subex} 
   
\end{ex}
In \preveg{a} the argument $\ulcorner\lambda
r$:\smallrecord{\smalltfield{x}{\textit{Rec}}}
. \record{\tfield{e}{$p$($r$.x)}}$\urcorner$ seems appropriate for the
function $\lambda P$:\textit{Ppty} . $P$(\smallrecord{\field{x}{$a$}})
since according to our new definition the argument is indeed of type
\textit{Ppty}.  However, it appears that the result of
$\beta$-conversion given in \preveg{b} leads to a situation where the
argument \smallrecord{\tfield{x}{$a$}} is not appropriate to the
function $\lambda r$:\smallrecord{\smalltfield{x}{\textit{Rec}}}
. \record{\tfield{e}{$p$($r$.x)}} because $a$ is an individual and not
a record.  This appears to lead to a contradictory situation where
\preveg{a} is both well-formed and ill-formed because by
$\beta$-conversion \preveg{a and b} represent the same object.  We
avoid this problem by saying that the application of a property, $P$,
to an argument, $a$, if $P$.fg($a$) in the case where $a:P$.bg but
that otherwise it returns a type that is necessarily empty.  We
introduce a distinguished type, `$\bot$', called ``bottom'', which is
necessarily empty.  We can achieve this, for example, by giving it the
witness condition in \nexteg{}.
\begin{ex} 
$a:\bot$ iff $a\not=a$ 
\end{ex} 
We can then define property application more accurately than we did
above as in \nexteg{}.
\begin{ex} 
  If $P$ : \textit{Ppty} and $a$ is some object, then
  \begin{enumerate} 
 
\item $P(a)=P.\text{fg}(a)$ if $a:P.\text{bg}$ 
 
\item otherwise $P(a)=\bot$ 
 
\end{enumerate} 
\label{ex:property-application}  
\end{ex} 

An advantage of this approach to properties is that providing
inappropriate arguments to properties will result in something that is
necessarily false but not ill-formed while its negation will also be
well-formed and true.  Thus while \nexteg{a} seems strange because
individual numbers do not rise \nexteg{b} seems true for the same reason.
\begin{ex} 
\begin{subex} 
 
\item 90 rises 
 
\item 90 does not rise 
 
\end{subex} 
   
\end{ex} 
  
\end{shaded}  


Now that we have generalized our definition of property in this way,
we need also to adjust our definition of property extension.  The definition we gave in
Chapter~\ref{ch:gram}, example (\ref{ex:propext}), is repeated in
\nexteg{}.\footnote{Recall that the notation $\down{T}$ is defined by
\begin{quote}  
$\down{T} = \{a\mid a:T\}$
\end{quote} 
}
\begin{ex} 
$\downP{P}$ = $\{a\mid\exists r[r:$\smallrecord{\smalltfield{x}{\textit{Ind}}}
$\wedge\ r.\mathrm{x}=a\wedge\down{P(r)}\not=\emptyset]\}$ 
\end{ex} 
This definition is based on the assumption that all properties are of
type (\smallrecord{\smalltfield{x}{\textit{Ind}}} $\rightarrow$
\textit{RecType}).  Now we need to modify it so that we will have
a notion of property extension for our new definition of
\textit{Ppty}.  This is done in \nexteg{}.\label{pg:property-extension-rev}
\begin{ex} 
If $T$ : \textit{xType} and $P$ : ($T\rightarrow\textit{RecType}$) then $\downP{P}$ = $\{a\mid\exists r[r:T$
$\wedge\ r.\mathrm{x}=a\wedge\down{P(r)}\not=\emptyset]\}$  
\end{ex}

  

% TTR
% introduces a restricted kind of polymorphism in function types which
% allows for certain kinds of infinite joins of function types
% (Appendix~\ref{app:funtypes}).  Given a type, $T$, we can
% give the type to which a function belongs just in case its domain type
% is a subtype of $T$ and its range is included among the witnesses of
% \textit{RecType}.  This type is given in \nexteg{}.
% \begin{ex} 
% % $\displaystyle{\bigvee_{T:\textit{Type},
% %     T'\sqsubseteq\text{\smallrecord{\smalltfield{x}{$T$}}}}}(T'\rightarrow
% % \textit{RecType})$ 

% $\displaystyle{\bigvee_{T'\sqsubseteq[\text{x}:T]}}(T'\rightarrow
% \textit{RecType})$ 
% \label{ex:pptyT}
% \end{ex}
% We can denote \preveg{} as \textit{Ppty}($T$) and let \textit{Ppty}
% represent the function, a dependent type, in \nexteg{}.
% \begin{ex} 
% $\lambda T$:\textit{Type} .  $\displaystyle{\bigvee_{T'\sqsubseteq[\text{x}:T]}}(T'\rightarrow \textit{RecType})$
% \end{ex}
% Then we can say that $P$ is a property just in case for some type,
% $T$, $P:\textit{Ppty}(T)$ and it is convenient to let the notation `$P$
% : \textit{Ppty}' designate this.  (In general, we could let $a:T$,
% where $T$ is a dependent type, mean ``there is some $b$ such that $a:T(b)$''.)  In this way
% we are saying that a property is a function whose domain is
% characterized by some record type with an 'x'-field and whose result for any
% object in its domain will be a record type.


% Suppose that a situation $e$ is of type temperature($r$).  What does
% that tell us about $r$? Given what we have seen so far we might expect
% that $r$ is an ambient temperature frame, that is, \mbox{$r$ :
% \textit{AmbTempFrame}}.  We might be tempted to express this as a
% constraint on assignments to types, that is, a constraint on
% possibilities in the sense of Appendix~\ref{app:modal}.  This might be
% expressed as in \nexteg{}.
% \begin{ex} 
% We restrict attention to those assignments to types (possibilities)
% such that for any situations $e$ and $r$, if $e$ : temperature($r$) then
% $r$ : \textit{AmbTempFrame}. 
% \end{ex} 
% Intuitively \preveg{} is similar to Montague's
% (\citeyear{Montague1973}) constraints on the interpretations of
% intensional logic to which we should restrict attention.  These
% constraints are standardly referred to as \textit{meaning postulates}
% in the literature although Montague himself did not give them this
% label. \preveg{} is fine if we are only concerned with ambient
% temperature or wish the predicate `temperature' to relate only to
% ambient temperature.  In general, however, we must take account of the
% fact that there are other kinds of temperature such as the temperature
% of objects, substances and human bodies.  If we choose to have
% separate predicates for all of these then \preveg{} is a possible
% constraint to have.  On the other hand, if we want to have a single
% predicate that covers all of these cases then \preveg{} will have to
% be modified.  One thing we might be tempted to do in this case is turn
% the implication around.  Instead of saying ``If it's a temperature
% then it's an ambient temperature frame'' as in \preveg{} we say ``If
% it's an ambient temperature frame then it's a temperature''.  This
% might look like \nexteg{}.
% \begin{ex} 
% We restrict attention to those assignments to types (possibilities)
% such that for any situation $r$, if \mbox{$r$ :
% \textit{AmbTempFrame}} then there is a situation $e$, such that $e$ :
% temperature($r$). 
% \label{ex:ambtemptopos-prose}
% \end{ex}       
% Note that it is important here that we changed the quantification over
% $e$ to existential quantification with scope over the consequent of
% the conditional.  It would have been wrong to have universal
% quantification as in \nexteg{}.
% \begin{ex} 
% For any situations $r$ and $e$, if \mbox{$r$ :
% \textit{AmbTempFrame}} then $e$ : temperature($r$) 
% \end{ex} 
% \preveg{} would require that every situation would have to be of the
% type temperature($r$) if the antecedent holds and this would have the
% unintuitive consequence that every situation would be a proof object
% for the temperature in every available location.  This would be
% particularly unwieldy if we wish to consider uncountably many
% locations as would be natural when considering geometric spaces.

% One advantage of (\ref{ex:ambtemptopos-prose}) is that it fits well
% into the kind of cognitive approach to semantics that we are trying to
% promote where we focus on how an agent with limited knowledge will use
% language rather than on a complete mathematical treatment of how
% a language considered as an abstract object relates to the world at
% large as Montague did. (\ref{ex:ambtemptopos-prose}) expresses one way
% in which a situation can be considered to be a temperature situation.
% It leaves open whether there are other kinds of situations which can
% be considered as temperature situations.  Consider an agent acquiring
% the concept of temperature, that is, what the temperature predicate can
% be applied to.  Such an agent may first become aware of the relevance
% of ambient
% temperature as expressed by (\ref{ex:ambtemptopos-prose}) and then
% subsequently add similar constraints on the same predicate for, say,
% the temperature of food, body temperature and so on.  By adding
% constraints to its resources in this way, an agent can incrementally
% build up an increasingly rich appreciation of a concept like
% temperature.  Another advantage of this is that an agent which has a
% collection of such constraints as resources can focus on some subset
% of those constraints or one particular constraint in a given context.
% Thus in Montague's example \textit{The temperature is 90�} we
% know that the temperature that is being referred to is ambient
% temperature.  

% This suggests that the appropriate notion for these
% constraints is that of \textit{topos} as discussed by
% \cite{Breitholtz2014a}.  According to Breitholtz a topos can be
% modelled as a function returning a type, that is, a dependent type
% similar to those we have used as update functions in this work.  Thus
% we can replace the prose statement (\ref{ex:ambtemptopos-prose}) with
% the function in \nexteg{a} which is associated with the licensing
% condition \nexteg{b}.
% \begin{ex} 
% \begin{subex} 
 
% \item $\lambda r$:\smallrecord{\smalltfield{x}{\textit{AmbTempFrame}}} . \record{\tfield{e}{temperature($r$.x)}} 
 
% \item If $f : (T\rightarrow Type)$ is a topos and $r$ is a record
%   (situation or frame) then for any
%   agent $A$, $r :_A T$ licenses $:_A f(r)$
 
% \end{subex} 
   
% \end{ex} 
% \preveg{}  absorbs the prose statement in
% (\ref{ex:ambtemptopos-prose}) into our theory of dependent types
% (which we assume can be implemented in memory) and our general theory
% of action which is supervenient on the type system.  The licensing
% condition \preveg{b} says that a topos will license an agent which
% judges a situation to be of the domain type of the topos to make a
% judgement that there is something of the type which the topos returns
% for that situation.  Different agents will have different topoi in
% their resources and may choose to act or not on the basis of a
% particular judgement and topos.  Different collections of the topoi
% available in an agent's resources may be activated in different
% circumstances.  And, of course, the collection of resources is dynamic
% in the sense that an agent may be learning new topoi or adjusting old
% ones depending on input from the environment.  Thus while topoi can
% be used to replace Montague's meaning postulates they represent a much
% more flexible tool which can be used to model the reasoning mechanisms
% available to an agent at a given time.

% Another advantage of \preveg{a} is that it is also the right kind of
% object to be used as a more specified content of \textit{temperature}.
% It represents a restriction of the function in
% (\ref{ex:dogtemperaturecont}b) obtained by replacing its domain type
% with a subtype.  This is a natural restriction of
% (\ref{ex:dogtemperaturecont}b) given that we have \preveg{a} as a resource.
% While it might seem intuitive that a particular utterance of the noun
% \textit{temperature} might be restricted to ambient temperature,
% something that might seem puzzling for a formulation of compositional
% semantics is that this restriction is passed on to the verb as well
% although intuitively obvious.

In characterizing the content of the noun \textit{temperature} we used
the predicate `temperature'.  This predicate has the arity
$\langle$\textit{Rec}$\rangle$.  What is the relationship between this
predicate and the type \textit{AmbTempFrame}, which is a type of
records.  An intuition that seems reasonable if we are treating
\textit{temperature} as a frame level noun is that one kind of frame
that could count as a temperature is an ambient temperature frame.  We
can express this by the witness condition in \nexteg{}.
\begin{ex} 
  $s$ : temperature$(r)$ if $r$ : \textit{AmbTempFrame} and $s=r$
  \label{ex:temperature-witcond}
\end{ex}

This witness condition corresponds to the inference rule in \nexteg{}.
\begin{ex} 
  \begin{prooftree}
    \hypo{r:\textit{AmbTempFrame}}
    \infer1{r:\text{temperature}(r)}
  \end{prooftree}
  
\end{ex} 
 Note that in (\ref{ex:temperature-witcond}) we again use \textit{if}
 rather than \textit{iff}.  This represents one way in which a record
 can be regarded as a temperature.  This allows for there to be other
 ways to be regarded as a temperature.  Note also that we are talking
 of judgements of the form $a:p(a)$ where the witness for the type is
 identical with the argument to the predicate.  This might appear to
 be introducing some kind of non-wellfoundedness.  However, this is
 not the case.  We can think of the type $p(a)$ in such a case as
 corresponding to a singleton type $T_a$, that is, it is either
 witnessed by $a$ or nothing.

When we say \textit{the temperature is
  rising} we are talking about an event which is a temperature rise,
not a price rise or any other kind of rise.  Somehow we have to
coordinate the frame which is chosen in connection with the
interpretation of temperature with the frame which is chosen in
connection with the interpretation of \textit{rise}.  The solution to
this that we wish to propose rests on the treatment of generalized
quantifiers proposed in \cite{Cooper2011,Cooper2013}. 

\section{Definite descriptions as dynamic generalized quantifiers}
\label{sec:dgqs}

In Chapter~\ref{ch:gram} we showed how to treat indefinite
descriptions (consisting of an indefinite article and a common noun
phrase) as generalized quantifiers.  We will now do something similar
for definite descriptions (consisting of a definite article and a
common noun phrase). We will then show how to modify this static
interpretation of generalized quantifiers so that it becomes a dynamic
treatment as presented in \cite{Cooper2011}.  We will see that the
dynamic treatment accounts for how the frame associated with the noun
is passed to the verb.

We will treat the definite article \textit{the} as introducing a
uniqueness condition.  We say that a property is unique in a situation
just in case its property extension in that situation is a singleton
set.  We will make this precise by introducing a predicate `unique'
whose arity is $\langle\textit{Ppty}\rangle$.  We characterize a
witness condition associated with this predicate using the notion of
restricted type introduced in Chapter~\ref{ch:gram},
pp.~\pageref{pg:restfld}\textit{ff}, but adjusted to take account of
the new kind of property as in \nexteg{}.
\begin{ex} 
  If $P$ is a property, then $P\!\restriction\!r$ is
  \begin{quote}
    \record{
      \field{bg}{$P$.bg}\\
      \field{fg}{$P$.fg$\restriction\!r$}}
  \end{quote}
  
\end{ex} 
  

The witness condition for `unique' is given in \nexteg{}.
\begin{ex} 
If $P$:\textit{Ppty} and $s$:\textit{Rec}, 

\hspace*{1em} then $s:\textrm{unique}(P)$ iff
  $\mid\!\downP{P\!\restriction\!s}\!\mid = 1$  
\end{ex}
\preveg{} says that a situation, $s$, is of the type `unique($P$)'
(where $P$ is a property) just in case there is exactly one object
which has the property $P$ in some component of $s$.  We could
introduce a generalized quantifier predicate, `the', with arity
$\langle\textit{Ppty},\textit{Ppty}\rangle$, associated with the
witness condition in \nexteg{}.
\begin{ex} 
If $P,Q$:\textit{Ppty} then, 
\begin{quote}
 $s:\mathrm{the}(P,Q)$ iff
 $s$ : unique($P$)
  and
  $\downP{P\!\!\restriction\!\!s}\subseteq\downP{Q\!\!\restriction\!\!s}$
\end{quote}

\end{ex} 
This corresponds to the Russellian definite description analysis as
used by Montague (see \citealp{DowtyWallPeters1981} for an explanation of
this) except that here the uniqueness is restricted to a particular
situation.  This means that we could talk about ``the dog'' without
requiring that there is exactly one dog in the universe.

It is well-known that the uniqueness condition in
the Russellian treatment of definite descriptions used by Montague is not quite right
for natural language.  (For a detailed discussion of the issues
involved see \citealp{Elbourne2012}.)  We can, for example, use the
definite description \textit{the dog} even though there are several
dogs.  It is not a simple matter of restricting ourselves to a
particular situation that we are describing since we may be describing
a situation with several dogs but still refer to some particular dog
in the situation as \textit{the dog}.  Such examples are discussed in
\cite{Cooper1996a} citing \nexteg{} from \cite{McCawley1979}.
\begin{ex} 
The dog had a fight with another dog yesterday 
\end{ex} 
Our solution to this is to in effect introduce resource situations
\citep{BarwisePerry1983,Cooper1996a}.  (A similar proposal is made by
\citealp{Elbourne2012}.) We follow the analysis in \cite{Cooper2013b}
and exploit the fact that properties can be restricted to a particular
situation by introducing a restricted field in the foreground as in \nexteg{}.
\begin{ex}
$\ulcorner\lambda
  r$:\smallrecord{\smalltfield{x}{\textit{Ind}}}. \smallrecord{\smallrfield{e}{$s$}{dog($r$.x)}}$\urcorner$ 
\end{ex} 
For the restricted field notation \smallrecord{\smallrfield{e}{$s$}{dog($r$.x)}} see
Chapter~\ref{ch:gram}, p.~\pageref{pg:restfld}.  \preveg{} can be glossed as ``the property of being a dog in $s$''.
We will abbreviate this as `dog$'$$\restriction$$s$' where we use
`dog$'$' to abbreviate the property without the restricted field.  These
abbreviations are represented in \nexteg{}.
\begin{ex} 
\begin{subex} 
 
\item dog$'$ abbreviates 
$\ulcorner\lambda
  r$:\smallrecord{\smalltfield{x}{\textit{Ind}}}. \smallrecord{\smalltfield{e}{dog($r$.x)}}$\urcorner$ 
 
\item dog$'$$\restriction$$s$ abbreviates 
$\ulcorner\lambda
  r$:\smallrecord{\smalltfield{x}{\textit{Ind}}}. \smallrecord{\smallrfield{e}{$s$}{dog($r$.x)}}$\urcorner$ 

\end{subex} 
\label{ex:dogprime}    
\end{ex} 
\begin{shaded}
In general we can say that if $p$ is a predicate with arity $\langle
T\rangle$ then $p'$ represents \nexteg{} % and $p'\!\restriction\!s$
% represents \nexteg{b}.
\begin{ex} 
%\begin{subex} 
 
%\item
  $\ulcorner\lambda
  r$:\smallrecord{\smalltfield{x}{$T$}} . \record{\tfield{e}{$p$($r$.x)}}$\urcorner$ 
 
% \item $\ulcorner\lambda
%   r$:\smallrecord{\smalltfield{x}{$T$}} . \record{\rfield{e}{$s$}{$p$($r$.x)}}$\urcorner$  
 
% \end{subex} 
   
\end{ex} 
  
\end{shaded}


We could introduce a function `SemDefArt' on the model of `SemIndefArt'
which was defined in Chapter~\ref{ch:gram}, example
(\ref{ex:semindefart}), although modified to accommodate parametric contents.  % Given the new definition of properties as
% pairs of a type and a function (labelled `bg' and `fg', respectively)
% we have to specify that the arguments to the quantifier relation are
% the functions (i.e. `fg').
This is given in \nexteg{}.
\begin{ex}
  $\ulcorner\lambda c$:\textit{Rec} . \\
\hspace*{1em}$\lambda Q$:\textit{Ppty} . \\
\hspace*{2em} $\lambda P$:\textit{Ppty}
. \record{\mfield{restr}{$Q$}{\textit{Ppty}} \\
          \mfield{scope}{$P$}{\textit{Ppty}} \\
          \tfield{e}{the(restr, scope)}}$\urcorner$ 
\end{ex} 
% This is exactly the same as `SemIndefArt' except that instead of the
% predicate `exist' we have `the'.  The constraint on `the' which
% relates it to classical generalized quantifier theory is a refinement
% of the constraint given for `exist' in Chapter~\ref{ch:gram}, example
% (\ref{ex:constraintexist}).  This is given in \nexteg{}.
% \begin{ex} 
% $\down{\mathrm{the}(P,Q)}\not=\emptyset$ iff
% $\mid\downP{P}\mid\ =1$ and $\downP{P}\cap\downP{Q}\not=\emptyset$ 
% \end{ex} 
% \preveg{} is like the constraint on `exist' except that it adds the
% additional requirement that the property extension of the first
% argument to `the' has cardinality exactly one.  This replicates
% Montague's Russellian treatment of the definite article.  We could
% equivalently define this constraint as \nexteg{}.  
% \begin{ex} 
% $\down{\mathrm{the}(P,Q)}\not=\emptyset$ iff
% $\mid\downP{P}\mid\ =1$ and $\downP{P}\subseteq\downP{Q}$
% \label{ex:theconstruniv} 
% \end{ex}   
% Since we require
% that there be exactly one object which has $P$ it does not make a
% difference whether we require that there is some object which has $P$
% which also has $Q$ or that every object that has $P$ also has
% $Q$. \preveg{} is the way the constraint is stated in
% \cite{Cooper2013b}. 



  
% Our strategy for doing this will build on a proposal in
% \cite{Cooper2011} which had to do with different ways of counting
% books (e.g. by physical volume, information content, exact text and so
% on).  The technique was to relativize the notion of property extension
% to different fields in frames. For example, the number of physical
% volumes may be greater than the number of information contents, since
% several volumes might have the same content (i.e. are different copies
% of the \textit{same} book).  Further the number of information
% contents might be smaller than the number of exact texts if, for
% example, we count different translations as having the same
% information contents.  Whether we count different translations as
% being different books or different translations of the same book
% depends on what is the most expedient way of counting in the situation
% at hand.  In \cite{Cooper2011} we only proposed ways of counting by
% different constituents in frames (that is, fields in records with a
% particular label) but we did not propose a way of counting by the
% frames themselves.  This we will add in the definitions given in
% \nexteg{}.
% \begin{ex} 
% \begin{subex} 
 
% \item If $P$:\textit{Ppty}, then
%   \mbox{$\downP{P} = \{r\mid r:P.\mathrm{bg} \wedge \down{P.\mathrm{fg}(r)}\not=\emptyset\}$} 
 
% \item If $P$:\textit{Ppty} and $\ell$ is a label in $P.\mathrm{bg}$, then $\downPl{P}{\ell} =
%   \{a\mid\exists r[r:P.\mathrm{bg}\wedge
%   r.\ell=a \wedge \down{P.\mathrm{fg}(r)}\not=\emptyset]\}$ 
 
% \end{subex} 
   
% \end{ex} 
% This means that given a particular property we may have the option of
% computing its property extension as a set of frames or as a set of
% particular constituents in those frames.  In
% Section~\ref{sec:passengers} we will suggest that this corresponds to
% a regular variation in content in natural language.  For now, we need
% to adjust (\ref{ex:theconstruniv}) to accommodate the variation.

% We need to adjust (\ref{ex:theconstruniv}) so that it is a conditional
% rather than a biconditional since now the notations $\downP{P}$ and
% $\downP{Q}$ will refer to the frame property extensions of $P$ and $Q$
% and this constraint will represent one of several ways in which the
% type can be instantiated.  The new version is given in \nexteg{a}.
% The possibilities which allow for different labels to be chosen are
% given in \nexteg{b}.
% \begin{ex} 
% \begin{subex} 
 
% \item If $P,Q$:\textit{Ppty}, then 

%  \hspace*{1em}$\down{\mathrm{the}(P,Q)}\not=\emptyset$ if  $\mid\downP{P}\mid\ =1$
%   and $\downP{P}\subseteq\downP{Q}$
 
% \item If $P,Q$:\textit{Ppty} and $\ell$ is a label in both
%   $P.\mathrm{bg}$ and $Q.\mathrm{bg}$,  then 

% \hspace*{1em}$\down{\mathrm{the}(P,Q)}\not=\emptyset$ if
% $\mid\downPl{P}{\ell}\mid\ =1$
%   and $\downPl{P}{\ell}\subseteq\downPl{Q}{\ell}$ 
 
% \end{subex} 
% \label{ex:therelconstrs}   
% \end{ex} 
% We can meet the constraint in (\ref{ex:theconstruniv}) by defining the witness condition for
% the($P$, $Q$) as in \nexteg{}, using the definition of restricted
% properties introduced in Chapter~\ref{ch:gram}, example~(\ref{ex:restricteddog}) on pp.~\pageref{ex:restricteddog}\textit{ff}.
% \begin{ex} 
% % \begin{subex} 
 
% % \item
%   If $P,Q$:\textit{Ppty} then, 

%  \hspace*{1em}$s:\mathrm{the}(P,Q)$ iff
%  $\mid\!\downP{P\!\!\restriction\!\!s}\!\mid\ = 1$
%   and $\downP{P\!\!\restriction\!\!s}\subseteq\downP{Q\!\!\restriction\!\!s}$
 
% % \item If $P,Q$:\textit{Ppty} and $\ell$ is a label in both
% %   $P.\mathrm{bg}$ and $Q.\mathrm{bg}$ then, 

% % \hspace*{1em}$a:\mathrm{the}(P,Q)$ if
% % $\downPl{P}{\ell}=\{a\}$
% %   and $\downPl{P}{\ell}\subseteq\downPl{Q}{\ell}$ 
 
% % \end{subex} 
% \label{ex:thewitcond}   
% \end{ex} 
% This gives the predicate `the' the flavour of Russell's
% \rotiota-operator.\footnote{An alternative is to maintain the intuition
%   that witnesses for ptypes are situation-like (i.e. records) and let
%   the witnesses be represented as \smallrecord{\field{x}{$r$}} and \smallrecord{\field{x}{$a$}} respectively.} 
%  The constraints and witness conditions in
% (\ref{ex:therelconstrs}) and \preveg{} should be viewed as different
% resources available to the language user which can be used separately
% on different occasions as appropriate.  They correspond intuitively to
% different ways of counting.  Thus we would want a definite description
% to be able to fail or succeed with respect to a particular way of
% counting rather than to succeed just in case there is some way of
% counting with respect to which it succeeds.



% In \cite{Cooper2013b} we introduced a predicate `unique' which takes two arguments, a
% property and a record (situation).  That is, its arity is
% $\langle$\textit{Ppty}, \textit{Rec}$\rangle$.  
% We required that the constraint in \nexteg{} hold of `unique'. % As with
% % our characterization of the constraints on the predicate `the' we give
% % alternative contraints which can be employed depending on the way of
% % counting that is being employed.
% \begin{ex}
% % \begin{subex} 
% % \item
%   If $P$:\textit{Ppty}, $T$ is a type and $s$:$T$, 

% \hspace*{1em} then $\down{\textrm{unique}(P,s)} \not=\emptyset$ iff
%   $\mid\downP{P\!\restriction\!s}\mid\ =1$ 

% % \item If $P$:\textit{Ppty}, $T$ is a type,
% %   $s$:$T$ and $\ell$ is a label in $P.\mathrm{bg}$,

% % \hspace*{1em} then $\down{\textrm{unique}(P,s)} \not=\emptyset$ if
% %   $\mid\downPl{P\!\restriction\!s}{\ell}\mid\ =1$

% % \end{subex}
% \label{ex:uniqueconstraint} 
% \end{ex} 

% The constraint in (\ref{ex:uniqueconstraint}) expresses that uniqueness holds between a
% property and a situation just in case the result of restricting the
% property to that situation is a property whose property extension is a
% singleton set.  

% Here we will do things slightly differently.  We introduce a one-place
% predicate `unique' whose arity is $\langle$\textit{Ppty}$\rangle$ and
% which is associated with the witness condition in \nexteg{}.
% \begin{ex} 
% % \begin{subex} 
% % \item
  

% % \item If $P$:\textit{Ppty} and $\ell$ is a label in $P$.bg,

% % \hspace*{1em} then $a:\down{\textrm{unique}(P,s)}$ if
% %   $\downPl{P\!\restriction\!s}{\ell} =\{a\}$

% % \end{subex} 
% \end{ex}

The reason that we need a uniqueness predicate of this kind has to do
with the nature of our type theory.  The typing mechanism allows us to
say for example what is given in \nexteg{}.
\begin{ex} 
$s$ : dog($a$) 
\end{ex} 
One way to paraphrase \preveg{} is ``$a$ is a dog in $s$''.  It says
that $s$ is of type `dog($a$)' but does not rule out that $s$ can be
of other types as well including possibly `dog($b$)' where $b$ is
distinct from $a$.  We do not have a way of saying that `dog($a$)' is
the only type to which $s$ belongs. This would correspond to
Schubert's (\citeyear{Schubert2000})\label{pg:Schubert-characterize} notion of characterizing a
situation, that is, in our terms, presenting an exhaustive list of
types to which it belongs, which given that we have meet types (Appendix~\ref{app:meettypes}),
corresponds to providing a single type to which it belongs such that there is no
other type to which it belongs.  We have made this choice because it would be very hard if
not impossible to guarantee that anything belongs to just one type in
the kind of type system we have introduced.  Consider, for example,
join types. Given our definition of join types in
Appendix~\ref{app:jointypes} if any object $a$ is of some type $T$ it
will also be of type $(T\vee T')$ for any type $T'$.  Introduction
of this classical kind of disjunction into the system makes it
difficult to define a useful notion of a type that completely
characterizes an object or situation in the way that Schubert
wants.\footnote{Schubert's argument for needing the notion of
  characterization has to do with defining a causation relation
  between events.  It seems to me that an analysis of causality must
  involve a type of the causing event.  Thus in addition to a
  two-place cause relation between two events, ``$e_1$ caused
  $e_2$'', we need a three-place cause relation between two events and
  a type of the first event, ``$e_1$ caused $e_2$ in virtue of the fact that
  $e_1:T$''. Thus, to take an example that Schubert discusses,
  \textit{John's singing in the shower caused Mary to wake up in
    virtue of the fact that it was a singing event} but not \textit{John's singing in the shower caused Mary to wake up in
    virtue of the fact that it was an event in the shower}.  Allowing
  types to be arguments to predicates in the way that we do provides a
different solution to the problem that Schubert presents.} 


Introducing the predicate `unique' in the way that we have allows us
to place a constraint on the types to which a situation belongs
without having to give a complete characterization of all the types to
which it belongs.  Defining it as a predicate whose argument is a
property means that its argument, the
property, involves a type.  A property is a function which returns a type.
Technically, we call it a dependent type.  In
Chapter~\ref{ch:intensional} we will suggest that allowing types or
dependent types as arguments to predicates is a characteristic of
evolutionary higher organisms (at least humans).  It seems intuitive that the kind of
uniqueness involved in the semantics of definite descriptions should
belong to this higher kind of reasoning.  We can imagine simple
organisms (perhaps even as simple as an amoeba) which respond to situations of certain types in certain
ways, for example, eating  behaviour when confronted with a situation in which an item of food is present.
However, it seems unintuitive that such a simple organism would be
programmed to engage in eating behaviour when exactly one item of food
is present and not otherwise.

Despite the success of our analysis so far in reducing the uniqueness
condition to a particular situation rather than applying it to the
whole universe, there is another problem with the use of Russellian
definite descriptions which it does not address.  This is that the
uniqueness is treated as part of the assertion, that is, as something
that is at issue.   This means that we should be able to deny an
utterance of a sentence like \nexteg{a} or answer the question
\nexteg{b} with something like
\nexteg{c}.
\begin{ex} 
\begin{subex} 
 
\item The dog barked.
  
\item Did the dog bark?
 
\item No, we don't have a dog. 
 
\end{subex} 
   
\end{ex}
If \preveg{c} is at all a possible response to \preveg{a or b} then it
feels like the denial of a presupposition.  When a definite
description is used in dialogue it seems that there is an assumption
that the interlocutor will be able to identify a relevant situation
in which there is a unique dog in a similar way as is suggested in our proposal for
the identification of referents of uses of proper names in
Chapter~\ref{ch:propnames}, that is, looking first on the dialogue
gameboard, then in long term memory and if nothing is to be found then
accommodating a situation in which there is a unique dog.
  
We shall use uniqueness to create a presuppositional account of
definite descriptions using the techniques for parametric contents
which we developed in Chapter~\ref{ch:propnames}.  The presupposition
type (a version of that proposed in \citealp{Cooper2013b} adjusted to
the new one-place predicate `unique') is given in \nexteg{}.
\begin{ex} 
 \smallrecord{\smalltfield{e}{unique(dog$'$)}}
\label{ex:bgthedog}
\end{ex} 
This is the type that, according to the techniques developed in
Chapter~\ref{ch:propnames}, will need to be matched against an agent's
resources (gameboard or long term memory) or, if a match is not
available, will need to be accommodated into the agent's gameboard.
It requires there to be a situation which has exactly one dog in it.  Satisfying the uniqueness
presupposition on this view is not so much a question of determining
the way the world is (i.e. whether the dog is in some objective sense
unique) as determining how the agent has carved up the world into
situations.

\preveg{} will, then, be the background of the
parametric content of the noun-phrase \textit{the dog}. Three
different options for this parametric content present themselves, as
in \nexteg{}.
\begin{ex} 
\begin{subex} 
 
\item $\ulcorner\lambda
  c$:\smallrecord{
    \smalltfield{$\mathfrak{c}$}{\smallrecord{
        \smalltfield{e}{unique(dog$'$)}}}} . \\
\hspace*{2em}$\lambda P$:\textit{Ppty} .
\record{
  \mfield{restr}{dog$'$$\restriction$$c.\mathfrak{c}$.e}{\textit{Ppty}}\\
  \mfield{scope}{$P$}{\textit{Ppty}}\\
  \tfield{e}{the(restr, scope)}}$\urcorner$ 
 
\item $\ulcorner\lambda
  c$:\smallrecord{
    \smalltfield{$\mathfrak{c}$}{\smallrecord{\smalltfield{e}{unique(dog$'$)}}}} . \\
\hspace*{2em}$\lambda P$:\textit{Ppty} .
\record{
  \mfield{restr}{dog$'$$\restriction$$c.\mathfrak{c}$.e}{\textit{Ppty}}\\
  \mfield{scope}{$P$}{\textit{Ppty}}\\
  \tfield{e}{exist(restr, scope)}}$\urcorner$ 

\item $\ulcorner\lambda
  c$:\smallrecord{
    \smalltfield{$\mathfrak{c}$}{\smallrecord{\smalltfield{e}{unique(dog$'$)}}}} . \\
\hspace*{2em}$\lambda P$:\textit{Ppty} .
\record{
  \mfield{restr}{dog$'$$\restriction$$c.\mathfrak{c}$.e}{\textit{Ppty}}\\
  \mfield{scope}{$P$}{\textit{Ppty}}\\
  \tfield{e}{every(restr, scope)}}$\urcorner$  
 
\end{subex} 
\label{ex:threethes}   
\end{ex}

  % Since we are relating the
% generalized quantifier predicates to the classical set relations given
% in \cite{BarwiseCooper1981} and we are requiring uniqueness by the
% presupposition, it will not make a difference in terms of which
% objects are required to have which properties whether we choose
% \preveg{a} (using the predicate constraint in
% (\ref{ex:theconstruniv})), \preveg{b} (using the predicate constraint
% in \nexteg{}, which repeats Chapter~\ref{ch:gram},
% example~(\ref{ex:constraintexist}))
% \begin{ex} 
% % \begin{subex} 
 
% % \item
%   If 
%   $P,Q$:\textit{Ppty} then 

%  \hspace*{1em}$\down{\mathrm{exist}(P,Q)}\not=\emptyset$ iff  $\downP{P}\cap\downP{Q}\not=\emptyset$
 
% % \item If $P,Q$:\textit{Ppty} and
% %   $\ell$ is a label in both $P$.bg and $Q$.bg then 

% % \hspace*{1em}$\down{\mathrm{exist}(P,Q)}\not=\emptyset$ if
% % $\downPl{P}{\ell}\cap\downPl{Q}{\ell}\not=\emptyset$ 
 
% % \end{subex} 
   
% \end{ex} 
% or (\ref{ex:threethes}c) using the constraint in \nexteg{}.
% \begin{ex} 
% % \begin{subex} 
 
% % \item
%   If $P,Q$:\textit{Ppty} then 

%  \hspace*{1em}$\down{\mathrm{every}(P,Q)}\not=\emptyset$ iff $\downP{P}\subseteq\downP{Q}$
 
% % \item If $P,Q$:\textit{Ppty} and
% %   $\ell$ is a label in both $P$.bg and $Q$.bg then 

% % \hspace*{1em}$\down{\mathrm{every}(P,Q)}\not=\emptyset$ if
% %   $\downPl{P}{\ell}\subseteq\downPl{Q}{\ell}$ 
 
% % \end{subex} 
   
% \end{ex} 
It does not make much difference which of these you choose for the
analysis of singular definite descriptions.  In \cite{Cooper2013b} we
chose the option corresponding to (\ref{ex:threethes}c), which offers
some vague hope of being able to draw a parallel with plural
definites.  Note that choosing (\ref{ex:threethes}b) or
(\ref{ex:threethes}c) eliminates the need for the predicate `the'.
Here we will choose \preveg{c}.

From the perspective of compositional semantics it is important that
the common noun \textit{dog} in these examples plays a role twice in
the parametric content of the noun phrase \textit{the dog}:  once in
determining the type of the context and once as the first argument to
the quantifier.  In order to achieve the contribution to the context
we will need to treat the content of \textit{the} not as a parametric
content but as a function from properties (corresponding to the common
noun) to a parametric content.  The content of \textit{the}, `SemDefArt', is given
in \nexteg{}.
\begin{ex} 
  $\lambda Q$:\textit{Ppty} . \\
  \hspace{1em}$\ulcorner\lambda c$:\smallrecord{
    \smalltfield{$\mathfrak{c}$}{\smallrecord{
        \smalltfield{e}{unique($Q$)}}}} . \\
  \hspace{2em}$\lambda P$:\textit{Ppty} . \\
  \hspace{3em}\record{
    \mfield{restr}{$Q\!\restriction\!c.\mathfrak{c}$.e}{\textit{Ppty}}\\
    \mfield{scope}{$P$}{\textit{Ppty}}\\
    \tfield{e}{every(restr, scope)}}$\urcorner$
\end{ex}
For the sake of consistency in how determiners are combined with nouns
we shall adjust the definition of the
`SemIndefArt' so that it too is a dependent parametric content of the
same form as `SemDefArt' even
though it does not introduce a presupposition that depends on the
following noun.  `SemIndefArt' is defined as \nexteg{}.
\begin{ex} 
% $\lambda Q$:\textit{PPpty} . 
% \smallrecord{\field{bg}{\smallrecord{\smalltfield{f}{\textit{Rec}}\\
%                                      \smalltfield{a}{$Q$.bg}}}\\
%              \field{fg}{$\lambda r$:\smallrecord{\smalltfield{f}{\textit{Rec}}\\
%                                      \smalltfield{a}{$Q$.bg}} . 
% $\lambda P$:\textit{Ppty} . 
% \smallrecord{\smallmfield{restr}{$Q$.fg($r$.a)}{\textit{Ppty}}\\
%              \smallmfield{scope}{$P$}{\textit{Ppty}}\\
%              \smalltfield{e}{exist(restr, scope)}}}}
  $\lambda Q$:\textit{Ppty} . \\
  \hspace{1em}$\ulcorner\lambda c$:\textit{Rec} . \\
  \hspace{2em}$\lambda P$:\textit{Ppty} . \\
  \hspace{3em}\record{
    \mfield{restr}{$Q$}{\textit{Ppty}}\\
    \mfield{scope}{$P$}{\textit{Ppty}}\\
    \tfield{e}{exist(restr, scope)}}$\urcorner$
\end{ex} 


How should these contents be combined with the content of a common noun
to form the content of the noun-phrase?  Let us refer to \preveg{} as
\textbf{the} and use \textbf{dog} to refer to the parametric content
associated with \textit{dog} given in \nexteg{}, where `dog$'$'
represents the property of being a dog as we defined earlier.
\begin{ex} 
  $\lambda c$:\smallrecord{
    \smalltfield{$\mathfrak{c}$}{\textit{Rec}}\\
    \smalltfield{$\mathfrak{s}$}{\textit{Rec}}} . dog$'$ 
\end{ex} 
(This assumes that the content for \textit{dog} does not depend on the
context.)  We can derive a parametric content for \textit{the dog} as
indicated in \nexteg{}.



\begin{ex}
  $\lambda c$:\record{
    \tfield{$\mathfrak{c}$}{\record{
        \tfield{s}{\textbf{dog}.bg}\\
        \tfield{f}{\textbf{the}(\textbf{dog}(s)).bg}\\
        \mfield{a}{s.$\mathfrak{c}$}{\textbf{dog}.bg.$\mathfrak{c}$}}}\\
    \mfield{$\mathfrak{s}$}{$\mathfrak{c}$.s.$\mathfrak{s}$}{\textbf{dog}.bg.$\mathfrak{s}$}}
  . [\textbf{the}]$_{\mathfrak{c}\leadsto\mathfrak{c}.\text{f}}$([\textbf{dog}]$_{\mathfrak{c}\leadsto\mathfrak{c}.\text{a}}$($c$))($c$)
  
  % $\lambda c$:\smallrecord{
  %   \smalltfield{f}{(\textbf{the}(\textbf{dog}(a))).bg}\\
  %   \smalltfield{a}{\textbf{dog}.bg}} . \textbf{the}(\textbf{dog}($c$.a))($c$.f)
\end{ex}



In order to achieve this we need a variant of the operation
`ContForwardApp' defined on p.~\pageref{ex:ContForwardApp@} which
combines contents of phrases based on forward application. We will
call new variant `ContForwardApp$_{\text{@\!@}}$' since it makes use of a
combination operation we designate by `@\!@' whereas the normal version
uses the combination operation `@', whose current version is defined
on p.\pageref{ex:s-combination}.  `ContForwardApp$_{\text{@\!@}}$' is given in
\nexteg{a} and `@\!@' is defined in \nexteg{b}.
\begin{ex} 
\begin{subex} 
 
% \item   If $T_{\text{arg}}$ and $T_{\text{res}}$ are types, then
%   ContForwardApp($T_{\text{arg}}$, $T_{\text{res}}$) is
% \begin{quote}
%   $\lambda
% u$:\smallrecord{\smalltfield{cont}{$(T_{\text{arg}}\rightarrow
%     T_{\text{res}})$}}$^{\frown}$\smallrecord{\smalltfield{cont}{$T_{\text{arg}}$}} . \\
% \hspace*{2em}\smallrecord{\smallmfield{cont}{$u$[0].cont@$u$[1].cont}{$T_{\text{res}}$}}

%\end{quote} 
 
\item If $T_{\text{arg}}$ and $T_{\text{res}}$ are types, then
  ContForwardApp$_{\text{@\!@}}$($T_{\text{arg}}$, $T_{\text{res}}$) is
\begin{quote}
  $\lambda
u$:\smallrecord{\smalltfield{cont}{$(T_{\text{arg}}\rightarrow
    T_{\text{res}})$}}$^{\frown}$\smallrecord{\smalltfield{cont}{$T_{\text{arg}}$}} . \\
\hspace*{2em}\smallrecord{\smallmfield{cont}{$u$[0].cont@\!@$u$[1].cont}{$T_{\text{res}}$}}

\end{quote}

\item If $\alpha$ : ($T_1\rightarrow$ \smallrecord{\smalltfield{bg}{\textit{RecType}}\\
                           \smalltfield{fg}{(bg$\rightarrow T_2$)}}) 
                         and $\beta$ : \smallrecord{\smalltfield{bg}{\smallrecord{
                               \smalltfield{$\mathfrak{c}$}{\textit{RecType}}\\
                               \smalltfield{$\mathfrak{s}$}{\textit{RecType}}}}\\
                           \smalltfield{fg}{(bg$\rightarrow T_1$)}}
                         then the \textit{combination of $\alpha$ and
    $\beta$  based on functional application}, $\alpha\text{@\!@}\beta$, is
  \begin{quote}
    $\ulcorner\lambda c$:\record{
      \tfield{$\mathfrak{c}$}{\record{
          \tfield{s}{$\beta$.bg}\\
          \tfield{f}{$\alpha(\beta(s))$.bg}\\
          \mfield{a}{s.$\mathfrak{c}$}{$\beta$.bg.$\mathfrak{c}$}}}}
      . \\*[\baselineskip]
      \hspace*{10em}$[\alpha]_{\mathfrak{c}\leadsto\mathfrak{c}.\text{f}}([\beta]_{\mathfrak{c}\leadsto\mathfrak{c}.\text{a}}(c))(c)\urcorner$
    \end{quote}
    
 
\end{subex} 
   
\end{ex}

  
  





What we have characterized so far is a static treatment of generalized
quantifiers.  Dynamic generalized quantifiers as presented in
\cite{Cooper2011} involving changing the
constraint on the quantifier predicate so that the information
represented by the first argument to the quantifier predicate is passed on as a
restriction to the second argument of the predicate.  What we mean by
the information associated with a property $P$ is essentially the
merge of the domain type and the range type of the dependent type
which is the foreground of the property.  Consider the example
`dog$'$' repeated in \nexteg{}.
\begin{ex} 
$\lambda r$:\smallrecord{\smalltfield{x}{\textit{Ind}}} . \record{\tfield{e}{dog($r$.x)}} 
\end{ex} 
Informally the information associated with this which is to be passed
on to the second argument of the quantifier is that we are talking
about some individual which is a dog.  This is represented by the
record type in \nexteg{}.
\begin{ex} 
  \record{
    \tfield{x}{\textit{Ind}}\\
    \tfield{e}{dog(x)}}
  \label{ex:dog-fptype}
\end{ex} 
Intuitively, the type in \preveg{} can be obtained from `dog$'$' by
merging the domain type with the result of relabelling the range type
so that `$r$.x' is replaced by `x', that is, \nexteg{}.
\begin{ex} 
\record{\tfield{e}{dog(x)}}
\end{ex} 
% (Recall that the merge operation
% introduced on p.~\pageref{pg:merge} is defined on labelled sets so that
% it will apply to objects like this that are not well-formed types.
% However, in the precise definition below we will not use this
% ill-formed type.)
We call (\ref{ex:dog-fptype}) a \textit{fixed point type} for `dog$'$'
because for any record $r$ of this type, $r$ : dog$'$($r$).  We use
`$\mathcal{F}$(dog$'$)' to designate (\ref{ex:dog-fptype}).
\begin{shaded}
  Defining $\mathcal{F}$ precisely is a little tricky.  If $T$ is a
  record type and $r$ is a record of any type, we will use
  $T_{r.\pi\leadsto\pi}$ to designate the type like $T$ except that
  for any $\pi\in\mathrm{paths}(T)$ any occurrence of $r.\pi$ is
  replaced by $\pi$.  For example, suppose $T$ is \nexteg{} (using
  official notation for dependent fields to be absolutely clear).
  \begin{ex} 
    \record{
      \tfield{x}{\textit{Ind}}\\
      \tfield{e}{$\langle\lambda v$:\textit{Ind} . dog($v$), $\langle r.\text{x}\rangle\rangle$}}
  \end{ex}
  Then $T_{r.\pi\leadsto\pi}$ will be \nexteg{}.
  \begin{ex} 
    \record{
      \tfield{x}{\textit{Ind}}\\
      \tfield{e}{$\langle\lambda v$:\textit{Ind} . dog($v$), $\langle\text{x}\rangle\rangle$}}
\end{ex} 
We then characterize $\mathcal{F}$ as in \nexteg{}, repeated in the
Appendix on p.~\pageref{pg:fixedpointtype}.
\begin{ex} 
If $\mathcal{T}$ is a dependent record type of the form $\lambda
r\!:\!T_1\ .\ T_2\dep{r}$ where $T_1$ is a record type and for any
$r$,
$\mathrm{paths}(T_1)\cap\mathrm{paths}(\mathcal{T}(r))=\emptyset$,
then
\begin{quote}
  $\mathcal{F}(\mathcal{T})$ is that type $T$ such that for any
  $r^*:T_1$, $\lambda r\!:\!T_1\ .\
  (T_1\text{\d{$\wedge$}}T_2)_{r.\pi\leadsto\pi}(r^*)=T$
\end{quote}
\label{ex:fixedpointtype}
\end{ex}
We illustrate this with an example.  Suppose that $\lambda
r\!:\!T_1\ .\ T_2\dep{r}$ is
`dog$'$', that is, \nexteg{}.
\begin{ex} 
$\lambda r$:\smallrecord{\smalltfield{x}{\textit{Ind}}} . \record{\tfield{e}{dog($r$.x)}} 
\end{ex}
Then $\lambda r\!:\!T_1\ .\ T_1\text{\d{$\wedge$}}T_2\dep{r}$ is
\nexteg{}.
\begin{ex} 
  $\lambda r$:\smallrecord{\smalltfield{x}{\textit{Ind}}} . \record{
    \tfield{x}{\textit{Ind}}\\
    \tfield{e}{dog($r$.x)}} 
\end{ex} 
Removing any dependence on $r$ we obtain $\lambda r\!:\!T_1\ .\
(T_1\text{\d{$\wedge$}}T_2)_{r.\pi\leadsto\pi}$, that is, \nexteg{}.
\begin{ex} 
 $\lambda r$:\smallrecord{\smalltfield{x}{\textit{Ind}}} . \record{
    \tfield{x}{\textit{Ind}}\\
    \tfield{e}{dog(x)}}
\end{ex} 
Note that since we have removed all dependencies on $r$, \preveg{} is
a constant function, that is, for any record of type
\smallrecord{\smalltfield{x}{\textit{Ind}}} it will return the type
\nexteg{}.
\begin{ex} 
\record{
    \tfield{x}{\textit{Ind}}\\
    \tfield{e}{dog(x)}} 
\end{ex} 
\preveg{} is, then, $\mathcal{F}$(dog$'$).  
  

The importance of the constraint that there are no shared paths in the
domain type and the resulting type of the function to which we apply
$\mathcal{F}$ is made clear by the abstract example in \nexteg{}.
Suppose that we did not have this restriction and allowed
$\mathcal{F}$ to apply to a function like \nexteg{a}.  Then the putative result
would be \nexteg{b} which is not well-formed since we do not allow
dependent fields which depend on themselves.
\begin{ex} 
\begin{subex} 
 
\item $\lambda r$:\smallrecord{\smalltfield{$\ell$}{$T_1$}} . \record{\tfield{$\ell$}{$T_2\dep{r.\ell}$}} 
 
\item \record{\tfield{$\ell$}{$T_1$\d{$\wedge$}$T_2\dep{\ell}$}} 
 
\end{subex} 
   
\end{ex} 

If we feel that we need to construct a fixed point type for a
dependent type where there are shared paths, then we can achieve the
effect we need by first relabelling the dependent type so that there
are no shared paths.

\end{shaded}


We shall use the fixed point type of the first argument to restrict
the dependent type which is 
the second argument.  We define the restriction of a function by a
type as in \nexteg{}.
\begin{ex} 
If $f$ is a function $\lambda v\!:\!T_1\ .\ \phi$, then the
\textit{restriction of} $f$ \textit{by a type} $T_2$, % \footnotemark
$f|_{T_2}$, 
is $\lambda v\!:\!(T_1$\d{$\wedge$}$T_2)\ .\ \phi$ 
\label{ex:function-restriction} 
\end{ex} 
% \footnotetext{Note that we do not require that $T_2$ is a subtype of
%   $T_1$.}
We can extend this notation to properties as in \nexteg{}.
\begin{ex} 
If $P$:\textit{Ppty}, then $P|_T$ is the property
\begin{quote}
  \record{
    \field{bg}{$P$.bg\d{$\wedge$}$T$}\\
    \field{fg}{$P$.fg$|_T$}} 
\end{quote}
that is,
\begin{quote}
  $\ulcorner P$.fg$|_T\urcorner$
\end{quote}

    
   
\end{ex} 
  
We can then define dynamic versions of the constraints on
the quantifier predicates and their witness conditions as in \nexteg{}.  % Note that in \preveg{} if
% a label $\ell$ is in $T$ then it will also be a label in $P|_T$.bg.
% Thus in \nexteg{b,d} we only need to choose a label in the background
% of the first argument to the quantifier.
\begin{ex} 
\begin{subex} 
 
% \item If $P,Q$:\textit{Ppty} then 

%  \hspace*{1em}$\down{\mathrm{exist}(P,Q)}\not=\emptyset$ iff  $\downP{P}\cap\downP{Q|_{\mathcal{F}(P.\mathrm{fg})}}\not=\emptyset$ 
 
% % \item If $P,Q$:\textit{Ppty} and $\ell$ is a label in $P$.bg  then 

% % \hspace*{1em}$\down{\mathrm{exist}(P,Q)}\not=\emptyset$ if
% % $\downPl{P}{\ell}\cap\downPl{Q|_{\mathcal{F}(P.\mathrm{fg})}}{\ell}\not=\emptyset$ 

% \item If $P,Q$:\textit{Ppty} then 

%  \hspace*{1em}$\down{\mathrm{every}(P,Q)}\not=\emptyset$ iff $\downP{P}\subseteq\downP{Q|_{\mathcal{F}(P.\mathrm{fg})}}$

% % \item If $P,Q$:\textit{Ppty} and $\ell$ is a label in $P$.bg then 

% % \hspace*{1em}$\down{\mathrm{every}(P,Q)}\not=\emptyset$ if
% %   $\downPl{P}{\ell}\subseteq\downPl{Q|_{\mathcal{F}(P.\mathrm{fg})}}{\ell}$ 

\item If $P,Q$:\textit{Ppty} then

\hspace*{1em} $s$ : exist($P$,$Q$) iff
$\downP{P\!\restriction\!s}\cap\downP{Q|_{\mathcal{F}(P.\mathrm{fg})}\!\restriction\!s}\not=\emptyset$

\item If $P,Q$:\textit{Ppty} then

\hspace*{1em} $s$ : every($P$,$Q$) iff $\downP{P\!\restriction\!s}\subseteq\downP{Q|_{\mathcal{F}(P.\mathrm{fg})}\!\restriction\!s}$
\label{subex:dynqconstr-every}  
 
\end{subex} 
\label{ex:dynqconstr}   
\end{ex} 
The original motivation for treating generalized quantifiers
dynamically was to be able to treat the kind of ``donkey-anaphora''
binding that occurs in sentences like \textit{every farmer who owns a
  donkey likes it}.  Our version of dynamic generalized quantifiers
essentially replicates the treatment in \cite{Chierchia1995}, though
in our own terms.  A similar analysis of generalized quantifiers,
exploiting contexts in type theory, is given in
\cite{Fernando2001}. In order to see how our strategy here will
facilitate the treatment of donkey anaphora we will have to wait until
we have a treatment of anaphora in Chapter~\ref{ch:quant}.    The
basic strategy is to exploit the conservativity of generalized
quantifiers and treat \textit{every farmer who owns a donkey likes it}
as \textit{every farmer who owns a donkey is a farmer who owns a
  donkey and likes it}.  This is achieved by restricting the second
argument of the quantifier predicate in the manner indicated in \preveg{}.

For present purposes the advantage of dynamicizing the generalized
quantifiers is that if the first argument property is restricted to be
a property of ambient temperature then that restriction will be passed
on to the second argument.  Let us look in detail at how this will
happen.  Consider the type in \nexteg{}.
\begin{ex}

every(
$\ulcorner\lambda r$:\smallrecord{\smalltfield{x}{\textit{AmbTempFrame}}} . 
            \smallrecord{\smallrfield{e}{$s$}{temperature($r$.x)}}$\urcorner$, \\ 
\hspace*{2.4em} 
$\ulcorner\lambda r$:\smallrecord{\smalltfield{x}{\textit{Rec}}} . 
            \smallrecord{\smalltfield{e}{rise($r$.x)}}$\urcorner$)

\label{ex:thetemprises}
\end{ex} 
% This type might arise as a result of determining the content of
% \textit{the temperature rises} using the parametric content for
% \textit{the temperature} in \nexteg{} (based on (\ref{ex:threethes})).
% \begin{ex} 
% \mbox{$\lambda r$:\smallrecord{\smalltfield{e}{unique(
% \smallrecord{\field{bg}{\textit{AmbTempFrame}}\\
%         \field{fg}{$\lambda
%                            r$:\smallrecord{\smalltfield{x}{\textit{AmbTempFrame}}}
%                            . \smallrecord{\smalltfield{e}{temperature($r$.x)}}}})}} .} \\
% \hspace*{2em}$\lambda P$:\textit{Ppty}
% . \smallrecord{\smalltfield{e}{every(
% \smallrecord{\field{bg}{\textit{AmbTempFrame}}\\
%         \field{fg}{$\lambda
%     r_1$:\smallrecord{\smalltfield{x}{\textit{AmbTempFrame}}}
%     . \smallrecord{\smallrfield{e}{$r$.e}{temperature($r_1$.x)}}}}, $P$)}} 
% \label{ex:thetemp}
% \end{ex} 
The result of applying $\mathcal{F}$ to the foreground of the first argument of (\ref{ex:thetemprises}) in
order to obtain a fixed point type is given in \nexteg{}.
\begin{ex} 
\record{\tfield{x}{\textit{AmbTempFrame}}\\
        \rfield{e}{$s$}{temperature(x)}} 
\end{ex} 
The condition on `every' in
(\ref{ex:dynqconstr}\ref{subex:dynqconstr-every}) requires that we
compare the first argument to `every' with the result of restricting
the second argument with \preveg{}.  The foreground of this is given in \nexteg{a},
which is identical with \nexteg{b} (by the definition of restriction)
and \nexteg{c} (by the definition of merge) and to 
\nexteg{d} (by the definition of merge\footnote{For this step we need
  to take the version of merge in Appendix~\ref{app:merge} which
  contains the two additional clauses taking account of subtypes.} because \textit{AmbTempFrame} is a subtype of \textit{Rec}).
\begin{ex} 
\begin{subex} 
 
\item $\lambda r$:\smallrecord{\smalltfield{x}{\textit{Rec}}} . 
            \smallrecord{\smalltfield{e}{rise($r$.x)}}$|_\text{\smallrecord{\smalltfield{x}{\textit{AmbTempFrame}}\\
        \smallrfield{e}{$s$}{temperature(x)}}}$ 
 
\item $\lambda r$:\smallrecord{\smalltfield{x}{\textit{Rec}}}\d{$\wedge$}\smallrecord{\smalltfield{x}{\textit{AmbTempFrame}}\\
                               \smallrfield{e}{$s$}{temperature(x)}} . 
            \smallrecord{\smalltfield{e}{rise($r$.x)}} 

\item $\lambda
  r$:\smallrecord{\smalltfield{x}{\textit{Rec}\d{$\wedge$}\textit{AmbTempFrame}}\\
                  \smallrfield{e}{$s$}{temperature(x)}} . 
            \smallrecord{\smalltfield{e}{rise($r$.x)}}

\item $\lambda r$:\smallrecord{\smalltfield{x}{\textit{AmbTempFrame}}\\
                               \smallrfield{e}{$s$}{temperature(x)}} . 
            \smallrecord{\smalltfield{e}{rise($r$.x)}} 
 
\end{subex} 
\label{ex:riserestrambtemp}  
\end{ex} 
Thus intuitively by choosing to restrict the first argument property
to ambient temperature frames we are also restricting the second
argument property to ambient temperature frames. 

This technique for dynamic quantifiers also has an important
consequence if we try to combine frame level and individual level
properties.  Suppose for example that we are trying to compute the
witness condition for \textit{the temperature runs} where
\textit{runs} corresponds to the content given in
(\ref{ex:dogtemperaturecont}a).  Then we will have \nexteg{}
as the foreground of the second argument property.
\begin{ex} 
\begin{subex} 
 
\item $\lambda r$:\smallrecord{\smalltfield{x}{\textit{Ind}}} . 
            \smallrecord{\smalltfield{e}{run($r$.x)}}$|_\text{\smallrecord{\smalltfield{x}{\textit{AmbTempFrame}}\\
        \smallrfield{e}{$s$}{temperature(x)}}}$ 
 
\item $\lambda r$:\smallrecord{\smalltfield{x}{\textit{Ind}}}\d{$\wedge$}\smallrecord{\smalltfield{x}{\textit{AmbTempFrame}}\\
                               \smallrfield{e}{$s$}{temperature(x)}} . 
            \smallrecord{\smalltfield{e}{run($r$.x)}}

\item  $\lambda
  r$:\smallrecord{\smalltfield{x}{\textit{Ind}\d{$\wedge$}\textit{AmbTempFrame}}\\
                  \smallrfield{e}{$s$}{temperature(x)}} . 
            \smallrecord{\smalltfield{e}{run($r$.x)}}

\item $\lambda
  r$:\smallrecord{\smalltfield{x}{\textit{Ind}$\wedge$\textit{AmbTempFrame}}\\
                  \smallrfield{e}{$s$}{temperature(x)}} . 
            \smallrecord{\smalltfield{e}{run($r$.x)}}


 
\end{subex} 
\label{ex:runrestrictedambtemp}   
\end{ex}
Here since neither \textit{Ind} nor \textit{AmbTempFrame} are a
subtype of the other the final step of merging represented in
\preveg{d} is the meet type (without the dot!) whose components are
the two types which were merged.   The property represented in
\preveg{} is thus necessarily empty (that is,
its property extension is the empty set no matter what we assign to
the basic types), if we have the assumption that
individuals are non-records.  This would be a way of requiring that
the content of \textit{runs} be coerced to something which could hold
for temperature frames in order to prevent the sentence from being
anomalous.  Similarly, if we wish to find a content for \textit{the
  dog rises} then we have to associate \textit{rises} with an
individual property or alternatively associate \textit{dog} with a
frame property.

What then should be the content of \textit{is ninety}?  An obvious
modification to the treatment of \textit{be} in Chapter~\ref{ch:gram}, substituting the type \textit{Real} for the type \textit{Ind},
would lead to the property in \nexteg{}.
\begin{ex} 
$\ulcorner\lambda r$:\smallrecord{\smalltfield{x}{\textit{Real}}}
. \record{\mfield{x}{$r$.x, 90}{\textit{Ind}}\\
          \tfield{e}{be(x)}}$\urcorner$ 
\end{ex} 
This property might be the content you need if you are treating a sentence like
\textit{2 times 45 is 90}.  However, if we use this content with
\textit{the temperature} we will run into a similar problem as that
represented in (\ref{ex:runrestrictedambtemp}).  This is spelled out in \nexteg{}
\begin{ex} 
\begin{subex} 
 
\item $\lambda r$:\smallrecord{\smalltfield{x}{\textit{Real}}} . 
            \smallrecord{\smallmfield{x}{$r$.x, 90}{\textit{Ind}}\\
                         \smalltfield{e}{be(x)}}$|_\text{\smallrecord{\smalltfield{x}{\textit{AmbTempFrame}}\\
        \smallrfield{e}{$s$}{temperature(x)}}}$ 
 
\item $\lambda r$:\smallrecord{\smalltfield{x}{\textit{Real}}}\d{$\wedge$}\smallrecord{\smalltfield{x}{\textit{AmbTempFrame}}\\
                               \smallrfield{e}{$s$}{temperature(x)}} . 
            \smallrecord{\smallmfield{x}{$r$.x, 90}{\textit{Ind}}\\
                         \smalltfield{e}{be(x)}} 

\item $\lambda
  r$:\smallrecord{\smalltfield{x}{\textit{Real}\d{$\wedge$}\textit{AmbTempFrame}}\\
                  \smallrfield{e}{$s$}{temperature(x)}} . 
            \smallrecord{\smallmfield{x}{$r$.x, 90}{\textit{Ind}} \\
                         \smalltfield{e}{be(x)}}

\item $\lambda
  r$:\smallrecord{\smalltfield{x}{\textit{Real}$\wedge$\textit{AmbTempFrame}}\\
                  \smallrfield{e}{$s$}{temperature(x)}} . 
            \smallrecord{\smallmfield{x}{$r$.x, 90}{\textit{Ind}}\\
                         \smalltfield{e}{be(x)}}


 
\end{subex} 
   
\end{ex}  
Assuming that real numbers are not records, we have the same problem
as we had in (\ref{ex:runrestrictedambtemp}) in that the property
turns out to be necessarily empty.  What we need instead is a property of frames (records) that will make
reference to a scale, $\zeta$, of the kind we defined for \textit{AmbTempFrame}
in (\ref{ex:scaleambtempframe}), for example, a property with the
foreground given in \nexteg{}.
\begin{ex} 
$\lambda r$:\smallrecord{\smalltfield{x}{\textit{Rec}}} . 
            \smallrecord{\smallmfield{x}{$\zeta$($r$.x), 90}{\textit{Ind}}\\
                         \smalltfield{e}{be(x)}}
\label{ex:isninety} 
\end{ex} 
If $\zeta$ is fixed to be the scale in (\ref{ex:scaleambtempframe})
then \preveg{} is identical with \nexteg{}.
\begin{ex} 
$\lambda r$:\smallrecord{\smalltfield{x}{\textit{Rec}}} . 
            \smallrecord{\smallmfield{x}{$r$.x.x, 90}{\textit{Ind}}\\
                         \smalltfield{e}{be(x)}}
\end{ex}          
That is, what is checked for being identical with 90 is the `x'-field of the temperature frame which is in the
`x'-field of the argument to the property.  If we choose this property
as the content for \textit{is ninety} then the restriction of the
property as second argument to the quantifier will give a property as
result which is not necessarily empty.  This property is shown in \nexteg{}.
\begin{ex} 
\begin{subex} 
 
\item $\ulcorner\lambda r$:\smallrecord{\smalltfield{x}{\textit{Rec}}} . 
            \smallrecord{\smallmfield{x}{$\zeta$($r$.x), 90}{\textit{Ind}}\\
                         \smalltfield{e}{be(x)}}$|_{\text{\smallrecord{\smalltfield{x}{\textit{AmbTempFrame}}\\
        \smallrfield{e}{$s$}{temperature(x)}}}}\urcorner$ 
 
\item $\ulcorner\lambda r$:\smallrecord{\smalltfield{x}{\textit{Rec}}}\d{$\wedge$}\smallrecord{\smalltfield{x}{\textit{AmbTempFrame}}\\
                               \smallmfield{e}{$s$}{temperature(x)}} . 
            \smallrecord{\smallmfield{x}{$\zeta$($r$.x), 90}{\textit{Ind}}\\
                         \smalltfield{e}{be(x)}}$\urcorner$ 

\item $\ulcorner\lambda
  r$:\smallrecord{\smalltfield{x}{\textit{Rec}\d{$\wedge$}\textit{AmbTempFrame}}\\
                  \smallrfield{e}{$s$}{temperature(x)}} . 
            \smallrecord{\smallmfield{x}{$\zeta$($r$.x), 90}{\textit{Ind}}\\
                         \smalltfield{e}{be(x)}}$\urcorner$

\item $\ulcorner\lambda
  r$:\smallrecord{\smalltfield{x}{\textit{AmbTempFrame}}\\
                  \smallrfield{e}{$s$}{temperature(x)}} . 
            \smallrecord{\smallmfield{x}{$\zeta$($r$.x), 90}{\textit{Ind}}\\
                         \smalltfield{e}{be(x)}}$\urcorner$
 
\end{subex}  
\end{ex} 
Now, as in (\ref{ex:riserestrambtemp}), \preveg{d} is equivalent
 to \preveg{c} in the sense that exactly the same objects will
have the properties.  This is because \textit{AmbTempFrame} is a
subtype of \textit{Rec}.  In the functions in \preveg{} there are two
parameters which will need to be determined by context in
compositional semantics, that is, will need to be found by matching
the domain type of a parametric content against an agent's resources.
These are the resource situation, $s$, and the scale, $\zeta$.



\section{Individual vs. frame level nouns}
\label{sec:indvsframe}

We have made a distinction between individual level nouns like
\textit{dog} and frame level nouns like \textit{temperature}, differentiating
their contents as in (\ref{ex:dogtemperaturecont}) and motivating the
distinction with the Partee puzzle.  Now consider \nexteg{}.
\begin{ex} 
\begin{subex} 
 
\item The dog is nine 
 
\item The dog is getting older/aging

\item Nine is getting older/aging 
 
\end{subex} 
   
\end{ex} 
We have the same intuitions about \preveg{} as we do about the
original temperature puzzle.  We cannot conclude \preveg{c} from
\preveg{a,b}.  Does this mean that \textit{dog} is a frame level noun
after all?  Certainly, if we think of frames as being like entries in
relational databases, it would be natural to think of age (or
information allowing us to compute age such as date of birth) as being
a natural field in a dog-frame.\footnote{Curiously, it does not seem
  to figure in FrameNet for \textit{dog} (as of 2nd March, 2015).  The
  noun \textit{dog} is associated with the frame Animals which
  inherits from the frame Biological entity.  But in neither of these
  frames is there a frame element corresponding to age or date of
  birth.  There is a frame Age but this does not seem to be related to
  Animals or Biological entity.} 

Our strategy to deal with this will be to say that contents of individual level
nouns can be coerced to frame level contents, whereas the contents of
frame level nouns cannot be coerced ``down'' to individual level
contents.  Thus in addition to \nexteg{a} we have \nexteg{b}.
\begin{ex} 
\begin{subex} 
 
\item $\ulcorner\lambda c$:\textit{Rec} . $\ulcorner\lambda
  r$:\smallrecord{\smalltfield{x}{\textit{Ind}}} . \record{\tfield{e}{dog($r$.x)}}$\urcorner\urcorner$  
 
\item  $\ulcorner\lambda c$:\textit{Rec} . $\ulcorner\lambda
  r$:\smallrecord{\smalltfield{x}{\textit{Rec}}} . \record{\tfield{e}{dog\_frame($r$.x)}}$\urcorner\urcorner$ 
 
\end{subex} 
 \label{ex:dog-dogframe}  
\end{ex} 
The predicate `dog\_frame' is related to the predicate `dog' by the
constraint in \nexteg{}.
\begin{ex} 
$e$ : dog\_frame($r$) implies $r$ : \smallrecord{\smalltfield{x}{\textit{Ind}}\\\smalltfield{e}{dog(x)}} 
\end{ex} 
There are several different kinds of dog frames
with additional information about a dog which an agent may acquire or
focus on.  Here we will consider just
frames which contain a field labelled `age' as exemplified in \nexteg{}.
\begin{ex} 
\smallrecord{\smalltfield{x}{\textit{Ind}}\\
                      \smalltfield{e}{dog(x)}\\
                      \smalltfield{age}{\textit{Real}}\\
                      \smalltfield{c$_{\mathrm{age}}$}{age\_of(x,age)}}
                     
\label{ex:dogframe}
\end{ex} 
An age scale, $\zeta_{\mathrm{age}}$, for individuals can then be
defined as the function in \nexteg{}.
\begin{ex} 
$\zeta_{\mathrm{age}}$ = 
$\lambda r$:\smallrecord{\smalltfield{x}{\textit{Ind}}\\
                         \smalltfield{age}{\textit{Real}}\\
                         \smalltfield{c$_{\mathrm{age}}$}{age\_of(x,age)}}
                       . $r$.age
  
\label{ex:agescale}
\end{ex} 
The content for \textit{is nine} in \textit{the dog is nine} is
then like (\ref{ex:isninety}) with $\zeta$ set to
$\zeta_{\mathrm{age}}$ and 9 replacing 90. Thus \textit{be} followed
by a numeral can be coerced to a content depending on some scale which
is available as a resource.  

We can think of the sentence \textit{the dog is nine} as involving two
coercions:  one coercing the content of \textit{dog} to a frame level
property and the other coercing the  content of \textit{be} to a
function which when applied to a number will return a frame level
property depending on an available scale.  Such coercions do not appear to be
universally available in languages.  For example, in German it is
preferable to say \textit{die Temperatur ist 35 Grad} ``the
temperature is 35 degrees'' rather than \textit{\#die Temperatur ist
  35} ``the temperature is 35''.  Similarly \textit{der Hund ist neun
  Jahre alt} ``the dog is nine years old'' is preferred over
\textit{\#der Hund ist neun} ``the dog is nine''.  We will return to
the matter of coercion or creation of new contents in Section~\ref{sec:passengers}.    

%  We now
% turn our attention to the formulation of the compositional semantics.

% \section{Defining a compositional semantics for the Partee
%   puzzle}

% We will now make precise the resources that are needed in order to
% account for the data expressed in \nexteg{}.
% \begin{ex} 
% \begin{subex} 
 
% \item a/the dog runs 
 
% \item the dog is nine

% \item the temperature is ninety

% \item the temperature rises 
 
% \end{subex} 
   
% \end{ex} 
  
% We start with the determiners.  The background type
% (``presupposition'') introduced by \textit{the}
% (\ref{ex:bgthedog}) depends on the common noun. This means that the
% contents of 
% \textit{the} and \textit{dog} cannot be combined by the S-combinator
% strategy for the combination of parametric contents that was
% introduced in Chapter~\ref{ch:propnames}, defined in
% Appendix~\ref{app:signcomb}.  This combination method passes up the
% background requirements of the two daughters without modifying them as
% depicted graphically in \nexteg{}.
% \newsavebox{\m}
% \newsavebox{\done}
% \newsavebox{\dtwo}
% \sbox{\m}{\smallrecord{\field{bg}{\smallrecord{\field{f}{$T_{F_\mathrm{bg}}$}\\
%                                             \field{a}{$T_{A_\mathrm{bg}}$}}}\\
%                     \field{fg}{$\lambda r$:\smallrecord{\field{f}{$T_{F_\mathrm{bg}}$}\\
%                                             \field{a}{$T_{A_\mathrm{bg}}$}}
%                                           . $F_{\mathrm{fg}}(r.\mathrm{f})(A_{\mathrm{fg}}(r.\mathrm{a}))$}} }
% \sbox{\done}{\smallrecord{\field{bg}{$T_{F_\mathrm{bg}}$}\\
%                              \field{fg}{$F_{\mathrm{fg}}$}}}
% \sbox{\dtwo}{\smallrecord{\field{bg}{$T_{A_\mathrm{bg}}$}\\
%                               \field{fg}{$A_{\mathrm{fg}}$}}}
% \begin{ex} 
% \Tree [.$X$\\\usebox{\m}
%                 $F$\\\usebox{\done} 
%                 $A$\\\usebox{\dtwo} ] 
% \label{ex:scombtree}
% \end{ex} 
    
% Instead what we need is \nexteg{} where $\mathfrak{F}$ is a function
% from parametric contents to parametric contents and $\mathcal{A}$ is a
% parametric content.
% \begin{ex} 
% \Tree [.$X$\\$\mathfrak{F}(\mathcal{A})$ $F$\\$\mathfrak{F}$ $A$\\$\mathcal{A}$ ] 
% \end{ex} 
% Here the function $\mathfrak{F}$ has to do the S-combinator like work
% which was achieved by the combination method in
% (\ref{ex:scombtree}). In \nexteg{} we show a schematic version of the
% function and on the node $X$ we show the schematic result of applying
% the function to the argument.
% \sbox{\m}{\smallrecord{\field{bg}{\smallrecord{\smalltfield{f}{$T_{F_\mathrm{bg}}(A_{\mathrm{fg}})$}\\
%                                                \smalltfield{a}{$T_{A_\mathrm{bg}}$}}}\\
%                        \field{fg}{$\lambda r$:\smallrecord{\smalltfield{f}{$T_{F_\mathrm{bg}}(A_{\mathrm{fg}})$}\\
%                                                \smalltfield{a}{$T_{A_\mathrm{bg}}$}}
%                                              . $F_{\mathrm{fn}}(A_{\mathrm{fg}}(r.\mathrm{a}))$}}}
% \sbox{\done}{$\lambda
%   p$:\smallrecord{\smalltfield{bg}{\textit{RecType}}\\
%                   \smalltfield{f}{$T_{A_\mathrm{fg}}$}} . 
% \smallrecord{\field{bg}{\smallrecord{\smalltfield{f}{$T_{F_\mathrm{bg}}(p.\mathrm{fg})$}\\
%                                      \smalltfield{a}{$p$.bg}}}\\
%              \field{fg}{$\lambda r$:\smallrecord{\smalltfield{f}{$T_{F_\mathrm{bg}}(p.\mathrm{fg})$}\\
%                                      \smalltfield{a}{$p$.bg}}
%                                    . $F_{\mathrm{fn}}(p.\mathrm{fg}(r.\mathrm{a}))$}}
% }
% \begin{ex} 
% \Tree [.$X$\\\usebox{\m}
%                 $F$\\\usebox{\done} 
%                 $A$\\\usebox{\dtwo} ] 

% \end{ex} 
% We will call a function from parametric contents to parametric
% contents a \textit{dependent parametric content}.  That is, it depends
% on another parametric content to yield a parametric content. The
% content of definite articles, SemDefArt, will be defined as an
% instance of the schema under $F$ in \preveg{}.  The definition is
% given in \nexteg{a} whose type is \nexteg{b}.
% \begin{ex}
% \begin{subex} 
% \item $\lambda Q$:\textit{PPpty} . 
% \smallrecord{\field{bg}{\smallrecord{\smalltfield{f}{\smallrecord{\smalltfield{e}{unique($Q$.fg($\Uparrow$a))}}}\\
%                                      \smalltfield{a}{$Q$.bg}}}\\
%              \field{fg}{$\lambda r$:\smallrecord{\smalltfield{f}{\smallrecord{\smalltfield{e}{unique($Q$.fg($\Uparrow$a))}}}\\
%                                      \smalltfield{a}{$Q$.bg}} . 
% $\lambda P$:\textit{Ppty} . 
% \smallrecord{\smallmfield{restr}{$Q$.fg($r$.a)}{\textit{Ppty}}\\
%              \smallmfield{scope}{$P$}{\textit{Ppty}}\\
%              \smalltfield{e}{every(restr, scope)}}}}

%      \item (\textit{PPpty}$\to$\textit{PQuant})
% \end{subex}
% \end{ex} 


% The lexical resource we need to include in the English lexicon is
% \nexteg{a} where Lex$_{\mathrm{DefArt}}$ is a universal resource
% defined in \nexteg{b}. (Lex is defined in
% Appendix~\ref{app:lexuniversal}.)
% \begin{ex} 
% \begin{subex} 
 
% \item Lex$_{\mathrm{DefArt}}$(``the'') 
 
% \item Lex$_{\mathrm{DefArt}}$($T_{\mathrm{phon}}$), where
%   $T_{\mathrm{phon}}$ is a phonological type, is defined as\\ 
% Lex($T_{\mathrm{phon}}$, \textit{Det}) \d{$\wedge$} \smallrecord{\smallmfield{cnt}{SemDefArt}{(\textit{PPpty}$\to$\textit{PQuant})}} 
 
% \end{subex} 
   
% \end{ex} 
 


% The lexical resource we need to include in the English lexicon is
% \nexteg{a} where Lex$_{\mathrm{IndefArt}}$ is a universal resource
% defined in \nexteg{b}. 
% \begin{ex} 
% \begin{subex} 
 
% \item Lex$_{\mathrm{IndefArt}}$(``a'') 
 
% \item Lex$_{\mathrm{IndefArt}}$($T_{\mathrm{phon}}$), where
%   $T_{\mathrm{phon}}$ is a phonological type, is defined as\\ 
% Lex($T_{\mathrm{phon}}$, \textit{Det}) \d{$\wedge$} \smallrecord{\smallmfield{cnt}{SemIndefArt}{(\textit{PPpty}$\to$\textit{PQuant})}} 
 
% \end{subex} 
   
% \end{ex} 


% We now move on to common nouns.  We define SemCommonNoun($p$,
% $T_{\mathrm{arg}}$, $T_{\mathrm{restr}}$, $T_{\mathrm{bg}}$),
% where $p$ is a predicate with arity $\langle T_{\mathrm{arg}}\rangle$,
% $T_{\mathrm{restr}}\sqsubseteq T_{\mathrm{arg}}$ and
% $T_{\mathrm{bg}}$ is a record type representing the background requirements, as \nexteg{}.
% \begin{ex} 
% \record{\field{bg}{$T_{\mathrm{bg}}$}\\
%         \field{fg}{$\lambda c$:$T_{\mathrm{bg}}$ . \record{\field{bg}{$T_{\mathrm{restr}}$}\\
%                            \field{fg}{$\lambda r$:\smallrecord{\smalltfield{x}{$T_{\mathrm{restr}}$}} . 
%                       \smallrecord{\smalltfield{e}{$p$($r$.x)}}}}}} 
% \label{ex:semcommonnoun}
% \end{ex} 
% The lexical resources for common nouns we will include in the English
% lexicon are \nexteg{a} where Lex$_{\mathrm{CommonNoun}}$ is a
% universal resource defined in \nexteg{b}.



% \begin{ex} 
% \begin{subex} 
 
% \item Lex$_{\mathrm{CommonNoun}}$(``dog'', dog, \textit{Ind},
%   \textit{Ind}, \textit{Rec})\\
%       Lex$_{\mathrm{CommonNoun}}$(``temperature'', temperature,
%       \textit{Rec}, \textit{Rec}, \textit{Rec}) 
 
% \item Lex$_{\mathrm{CommonNoun}}$($T_{\mathrm{phon}}$, $p$,
%   $T_{\mathrm{arg}}$, $T_{\mathrm{restr}}$, $T_{\mathrm{bg}}$),
%   where $T_{\mathrm{phon}}$ is a phonological type, $p$ is a predicate
%   with arity $\langle T_{\mathrm{arg}}\rangle$, $T_{\mathrm{restr}}\sqsubseteq T_{\mathrm{arg}}$ and $T_{\mathrm{bg}}$ is a record type, is defined as\\
% Lex($T_{\mathrm{phon}}$, $N$) \d{$\wedge$} 
% \smallrecord{\smallmfield{cnt}{SemCommonNoun($p$, $T_{\mathrm{arg}}$,
%     $T_{\mathrm{restr}}$, $T_{\mathrm{bg}}$)}{\textit{PPpty}}} 
 
% \end{subex} 
% \label{ex:lexdogtemp}   
% \end{ex}



We can think of the common noun sign types like (\ref{ex:dog-dogframe}a) as
unmodulated in something like the sense of modulation discussed by
\cite{Recanati2010} in that the restriction type yielding the type of
the domain of the property is identical with the type that represents
the arity of the predicate.  We will see later a way to modulate the content
of the noun by choosing a subtype of the type of the predicate
argument as the domain type of the property. 

Here we discuss an operation `CommonNounIndToFrame' which is defined
on individual level common noun sign types and ``raises'' them to
frame level common noun sign types.  In order to facilitate this we
first introduce a function $\mathrm{FrameType}$ which maps predicates
with arity $\langle\textit{Ind}\rangle$ to a type of frames.  The way
$\mathrm{FrameType}$ is defined depends on a particular agent at a
particular time, that is, $\mathrm{FrameType}(p)$ represents the type
of frames that an agent associates with the predicate $p$.  A general
constraint on $\mathrm{FrameType}$ is that for any $p$ in its domain
\nexteg{} must hold.
\begin{ex} 
  $\mathrm{FrameType}(p)\sqsubseteq$ \record{
    \tfield{x}{\textit{Ind}}\\
    \tfield{e}{$p$(x)}}
\end{ex} 
Using this we can derive a new predicate `$p$\_frame' from any
predicate $p$ for which $\mathrm{FrameType}$ is defined.  This is
characterized in \nexteg{}.
\begin{ex} 
\begin{subex} 
 
\item If $p$ is a predicate in the domain of $\mathrm{FrameType}$,
  then $p$\_frame is a predicate with arity $\langle\textit{Rec}\rangle$. 
 
\item $e:p\_\text{frame}(r)$ iff $r:\mathrm{FrameType}(p)$ and $e=r$ 
 
\end{subex} 
   
\end{ex} 
  


`CommonNounIndToFrame' is defined in \nexteg{}.

\begin{ex}
If $T_{\mathrm{phon}}$ is a phonological type, $p$ is a predicate with
arity $\langle\textit{Ind}\rangle$ and
$T_\mathrm{bg}$ is a record type (the ``background type'' or
``presupposition'') then \\ 
\mbox{CommonNounIndToFrame(Lex$_{\mathrm{CommonNoun}}$($T_{\mathrm{phon}}$,
  $T_\mathrm{bg}$, $p$))} =
\begin{quote}
  Lex$_{\mathrm{CommonNoun}}$($T_{\mathrm{phon}}$,
  $T_\mathrm{bg}$, $p$\_frame)
\end{quote}

\label{ex:CommonNounIndToFrame} 
\end{ex} 
This operation is a universal resource which may or may not be used by
individual languages.  Given the discussion in
Section~\ref{sec:indvsframe}, we suggest that it is used productively
in English but not in German, for example.  This gives us a way of
generating new lexical resources from already existing resources.
Similarly, we can think of $p$\_frame as being the result of applying
a ``raising'' operation to the predicate $p$. % where the new predicate
% is associated with the general constraint expressed in \nexteg{}.
% \begin{ex} 
% If $p$ is a predicate with arity $\langle$\textit{Ind}$\rangle$, then
% for any $e$ and $r$,  
% \begin{quote}
% $e$ : $p$\_frame($r$) implies $r$ : 
% \smallrecord{\smalltfield{x}{\textit{Ind}}\\
%              \smalltfield{e}{$p$(x)}}
% \end{quote}
% \label{ex:predframe}
% \end{ex}

Another way to generate new lexical resources from basic common noun
sign types is to restrict the domain of the common noun by some type
(perhaps related to a topos as suggested in
Section~\ref{sec:compsemPartee}).  This is formulated in \nexteg{}.


\begin{ex} 
If $T_{\mathrm{phon}}$ is a phonological type, $p$ is a predicate,
$T_{\mathrm{bg}}$ and $T_{\mathrm{res}}$ are record types and $\Sigma$
is Lex$_{\mathrm{CommonNoun}}$($T_{\mathrm{phon}}$, $T_{\mathrm{bg}}$,
$p$), then RestrictCommonNoun($\Sigma$, $T_{\mathrm{res}}$) is
\begin{quote}
$\Sigma$
\fbox{\d{$\wedge$}} \record{
  \mfield{cont}{$\ulcorner\lambda c\!:\!T_{\mathrm{bg}}\ .\
    \ulcorner\text{SemCommonNoun}(T_{\mathrm{bg}},p)(c)\!\mid_{T_{\mathrm{res}}}\urcorner\urcorner$}{\textit{PPpty}}}
\end{quote}
  
\end{ex} 
This will enable us, for example,  to restrict the basic lexical entry
for \textit{temperature}, $\Sigma_{\text{``temperature''}}$ (repeated in \nexteg{a}) to obtain the
additional lexical resource \nexteg{b}.
\begin{ex} 
\begin{subex} 
 
\item $\Sigma_{\text{``temperature''}}$ = Lex$_{\mathrm{CommonNoun}}$(``temperature'', \textit{Rec}, temperature)
       
 
\item RestrictCommonNoun($\Sigma_{\text{``temperature''}}$, \textit{AmbTempFrame})
 
\end{subex} 
   
\end{ex}
This gives us a content for \textit{temperature} which is restricted
to ambient temperature.

We can combine restriction coercion with frame coercion.  While frame
coercion gives us a general frame level property of records we can
restrict the frame to be of a certain type corresponding to a
particular type of frame that we have as a resource.  For example,
suppose that we have resource which is a frame type for dog frames, \textit{DogFrame} as introduced in
(\ref{ex:dogframe}) repeated in \nexteg{}.
\begin{ex} 
\smallrecord{\smalltfield{x}{\textit{Ind}}\\
                      \smalltfield{e}{dog(x)}\\
                      \smalltfield{age}{\textit{Real}}\\
                      \smalltfield{c$_{\mathrm{age}}$}{age\_of(x,age)}} 
\end{ex} 
We can use \textit{DogFrame} to restrict the result of coercing our
frame level dog sign type, that is there can be a two-step coercion
from the basic lexical entry in \nexteg{a}, $\Sigma_{\text{``dog''}}$, as represented
in \nexteg{a} to \nexteg{b}.
\begin{ex}
  \begin{subex} 
 
\item  $\Sigma_{\text{``dog''}}$ = Lex$_{\text{CommonNoun}}$(``dog'', \textit{Rec}, dog)
 
\item
  RestrictCommonNoun(CommonNounIndToFrame($\Sigma_{\text{``dog''}}$), \textit{DogFrame}) 
 
\end{subex} 
  
% Lex$_{\mathrm{CommonNoun}}$(``dog'', dog, \textit{Ind},
%   \textit{Ind}, \textit{Rec}) $\leadsto$ Lex$_{\mathrm{CommonNoun}}$(``dog'', dog\_frame, \textit{Rec},
%   \textit{Rec}, \textit{Rec}) $\leadsto$ Lex$_{\mathrm{CommonNoun}}$(``dog'', dog\_frame, \textit{Rec},
%   \textit{DogFrame}, \textit{Rec})
\end{ex} 



% We treat intransitive verbs in a parallel fashion to common nouns.
% Thus 
% \begin{quote}
% SemIntransVerb($p$,
% $T_{\mathrm{arg}}$, $T_{\mathrm{restr}}$, $T_{\mathrm{bg}}$)
% \end{quote}
% where $p$ is a predicate with arity $\langle T_{\mathrm{arg}}\rangle$,
% $T_{\mathrm{restr}}\sqsubseteq T_{\mathrm{arg}}$ and
% $T_{\mathrm{bg}}$ is the record type (\ref{ex:semcommonnoun}) as for
% common nouns.  We define  Lex$_{\mathrm{IntransVerb}}$ similarly to
% Lex$_{\mathrm{CommonNoun}}$ in \nexteg{}.
% \begin{ex} 
% Lex$_{\mathrm{IntransVerb}}$($T_{\mathrm{phon}}$, $p$,
%   $T_{\mathrm{arg}}$, $T_{\mathrm{restr}}$, $T_{\mathrm{bg}}$),
%   where $T_{\mathrm{phon}}$ is a phonological type, $p$ is a predicate
%   with arity $\langle T_{\mathrm{arg}}\rangle$, $T_{\mathrm{restr}}\sqsubseteq T_{\mathrm{arg}}$ and $T_{\mathrm{bg}}$ is a record type, is defined as\\
% Lex($T_{\mathrm{phon}}$, \textit{VP}) \d{$\wedge$} 
% \smallrecord{\smallmfield{cnt}{SemIntransVerb($p$, $T_{\mathrm{arg}}$,
%     $T_{\mathrm{restr}}$, $T_{\mathrm{bg}}$)}{\textit{PPpty}}} 
% \end{ex} 
% The basic lexical resources that we need for \textit{runs} and
% \textit{rises} are given in \nexteg{}.
% \begin{ex} 
% Lex$_{\mathrm{IntransVerb}}$(``runs'', run, \textit{Ind},
%   \textit{Ind}, \textit{Rec})\\
%       Lex$_{\mathrm{IntransVerb}}$(``rises'', rise,
%       \textit{Rec}, \textit{Rec}, \textit{Rec})  
% \end{ex} 


We need a treatment of \textit{is} which will allow it to
combine with numerals like \textit{nine} and \textit{ninety} to form a
frame level predicate as indicated in (\ref{ex:isninety}).  We start
from a parametrized version of the definition of SemBe which we
introduced in Chapter~\ref{ch:gram}, also adjusted for our new
treatment of properties.  This is given in
\nexteg{}. 
\begin{ex}
$\ulcorner\lambda c:T_{\text{bg}}$ . \\
        \hspace*{1em}$\lambda\mathcal{Q}$:\textit{Quant} . \\
\hspace*{2em} $\ulcorner\lambda r_1$:\smallrecord{\tfield{x}{\textit{Ind}}}
. \\
\hspace*{3em} $\mathcal{Q}$($\ulcorner\lambda
r_2$:\smallrecord{\tfield{x}{\textit{Ind}}}
. \record{\mfield{x}{$r_1$.x, $r_2$.x}{\textit{Ind}}\\
  \tfield{e}{be(x)}}$\urcorner$)$\urcorner\urcorner$ 
% $\lambda r$:\textit{Rec} . \\
% \hspace*{1em}$\lambda\mathcal{Q}$:\textit{Quant} . \\
% \hspace*{2em}
% \record{\field{bg}{\textit{Ind}}\\
%              \field{fg}{$\lambda r_1$:\smallrecord{\smalltfield{x}{\textit{Ind}}}
% . \\
% & & \hspace*{1em}$\mathcal{Q}$(\record{\field{bg}{\textit{Ind}}\\
%                            \field{fg}{$\lambda
% r_2$:\smallrecord{\smalltfield{x}{\textit{Ind}}}
% . \record{\mfield{x}{$r_1$.x,$r_2$.x}{\textit{Ind}}\\
%           \tfield{e}{be(x)}}
% }
% }
% )}}
 
\end{ex} 
Here the context represented by the first argument to the function,
$c$, does not contribute anything to the final content of \textit{be}
which represents straightforward equality.  In this chapter we want to
allow equality not only between individuals but also objects of other
types.  We can do this by letting the content of \textit{be} be
parametric on a type including the label `x' introduced in the context as in \nexteg{}.
\begin{ex} 
  $\ulcorner\lambda c$:\smallrecord{
    \smalltfield{$\mathfrak{c}$}{\smallrecord{
        \smalltfield{ty}{\textit{Type}}}}} . \\
        \hspace*{1em}$\lambda\mathcal{Q}$:\textit{Quant} . \\
        \hspace*{2em} $\ulcorner\lambda r_1$:\smallrecord{
          \smalltfield{x}{$c.\mathfrak{c}$.ty}}
. \\
\hspace*{3em} $\mathcal{Q}$($\ulcorner\lambda
r_2$:\smallrecord{
  \smalltfield{x}{$c.\mathfrak{c}$.ty}}
. \record{\mfield{x}{$r_1$.x, $r_2$.x}{$c.\mathfrak{c}$.ty}\\
  \tfield{e}{be(x)}}$\urcorner$)$\urcorner\urcorner$ 
\end{ex}
This gives us a content for \textit{be} which will express identity
for objects in any type.  Thus it will be appropriate for both
sentences in \nexteg{} where \nexteg{a} expresses identity for
individuals and \nexteg{b} expresses identity for numbers.
\begin{ex} 
\begin{subex} 
 
\item The dog is Fido 
 
\item The number is nine 
 
\end{subex} 
   
\end{ex} 
We will call this `SemBe$_{\text{ID}}$'.  We need a slightly different
content in order to deal with \nexteg{}.
\begin{ex}
  \begin{subex} 
 
\item The dog is nine 
 
\item The temperature is ninety 
 
\end{subex} 
  
 
\end{ex} 
Here we need to introduce a scale into the context and make sure that
the second argument to \textit{be} is a number.  This is given in
\nexteg{}.
\begin{ex} 
$\ulcorner\lambda c$:\smallrecord{
    \smalltfield{$\mathfrak{c}$}{\smallrecord{
        \smalltfield{ty}{\textit{Type}}\\
      \smalltfield{sc}{(ty$\rightarrow$\textit{Real})}}}} . \\
        \hspace*{1em}$\lambda\mathcal{Q}$:\textit{Quant} . \\
        \hspace*{2em} $\ulcorner\lambda r_1$:\smallrecord{
          \smalltfield{x}{$c.\mathfrak{c}$.ty}}
. \\
\hspace*{3em} $\mathcal{Q}$($\ulcorner\lambda
r_2$:\smallrecord{
  \smalltfield{x}{\textit{Real}}}
. \record{\mfield{x}{$c.\mathfrak{c}$.sc($r_1$.x), $r_2$.x}{\textit{Real}}\\
  \tfield{e}{be(x)}}$\urcorner$)$\urcorner\urcorner$  
\end{ex} 
We will refer to this as `SemBe$_{\text{scalar}}$'.

We now have two similar contents for \textit{be}.  It would be a
simple matter to define `SemBe$_{\text{scalar}}$' as the result of a
coercion operation applied to `SemBe$_{\text{ID}}$' if we wanted to.
\cite{Montague1973} treated both kinds of \textit{be} as identical and
used individual concepts instead of individuals in order to account
for the Partee puzzle.  Most following work avoids this use of
individual concepts and instead assumes that the content of
\textit{be} is different in the two cases.  The analysis we have
presented is a way of capturing both intuitions.  We have two contents
associated with \textit{be} but we have made explicit that they are
closely related and both basically have \textit{be} express an
identity relation, corresponding to Montague's intuition.  Note that
technically both sentences in \nexteg{} are ambiguous according to
this analysis.
\begin{ex} 
\begin{subex} 
 
\item The dog is nine 
 
\item The temperature is ninety 
 
\end{subex} 
   
\end{ex} 
They each have a nonsense necessarily false reading in which,
respectively, the dog is identical with the number nine and the
temperature is identical with the number ninety.  The derivation of
both of these sentences involves a property application which results
in `$\bot$' (see (\ref{ex:property-application}) on
p.~\pageref{ex:property-application}).  It is natural to assume that
speakers in general disprefer the content `$\bot$'.  Note that this
does not necessarily mean that speakers disprefer all contradictory
readings.  There are other necessarily empty types.  For example,
suppose that $T_1$ and $T_2$ preclude each other, that is, no object
can be a witness for both types.  Then $T_1\wedge T_2$ is a
necessarily empty type but is distinct from `$\bot$'.

% We will thus
% give SemBe two arguments and represent \preveg{} as
% SemBe(\textit{Ind}, \textit{Rec}).  In general we will define
% SemBe($T_{\mathrm{arg}}$, $T_{\mathrm{bg}}$) to be \nexteg{}.
% \begin{ex}
% $\lambda r$:$T_{\mathrm{bg}}$ . \\
% \hspace*{1em}$\lambda\mathcal{Q}$:\textit{Quant} . \\
% \hspace*{2em}
% \record{\field{bg}{$T_{\mathrm{arg}}$}\\
%              \field{fg}{$\lambda r_1$:\smallrecord{\smalltfield{x}{$T_{\mathrm{arg}}$}}
% . \\
% & & \hspace*{1em}$\mathcal{Q}$(\record{\field{bg}{$T_{\mathrm{arg}}$}\\
%                            \field{fg}{$\lambda
% r_2$:\smallrecord{\smalltfield{x}{$T_{\mathrm{arg}}$}}
% . \record{\mfield{x}{$r_1$.x, $r_2$.x}{$T_{\mathrm{arg}}$}\\
%           \tfield{e}{be(x)}}
% }
% }
% )}}
 


% \end{ex} 
% We will say that this holds if $T_{\mathrm{bg}}$ does not require a
% scale, more precisely if $T_{\mathrm{bg}}$ is not a subtype of 
% \smallrecord{\smalltfield{sc}{($T_{\mathrm{arg}}\to\textit{Real}$)}}. If
% $T_{\mathrm{bg}}\sqsubseteq$
% \smallrecord{\smalltfield{sc}{($T_{\mathrm{arg}}\to\textit{Real}$)}}
% then SemBe($T_{\mathrm{arg}}$, $T_{\mathrm{bg}}$) is \nexteg{}.
% \begin{ex}
% $\lambda r$:$T_{\mathrm{bg}}$ . \\
% \hspace*{1em}$\lambda\mathcal{Q}$:\textit{Quant} . \\
% \hspace*{2em}
% \record{\field{bg}{$T_{\mathrm{arg}}$}\\
%              \field{fg}{$\lambda r_1$:\smallrecord{\smalltfield{x}{$T_{\mathrm{arg}}$}}
% . \\
% & & \hspace*{1em}$\mathcal{Q}$(\record{\field{bg}{\smallrecord{\smalltfield{x}{\textit{Real}}}}\\
%                            \field{fg}{$\lambda
% r_2$:\smallrecord{\smalltfield{x}{\textit{Real}}}
% . \record{\mfield{x}{$r$.sc($r_1$.x), $r_2$.x}{\textit{Real}}\\
%           \tfield{e}{be(x)}}
% }
% }
% )}}
 


% \end{ex}


% Using the scale $\zeta_{\mathrm{age}}$ from (\ref{ex:agescale}),
% repeated as \nexteg{b}, we can
% use the domain type of that function, which we refer to as
% \textit{AgeFrame} (as specified in \nexteg{a}) as $T_{\mathrm{arg}}$ and construct a meaning for
% \textit{be} SemBe(\textit{AgeFrame},
% \smallrecord{\smalltfield{sc}{(\textit{AgeFrame}$\to\textit{Real}$)}}),
% given in \nexteg{c}.
% \begin{ex} 
% \begin{subex}

% \item \textit{AgeFrame} =  \smallrecord{\smalltfield{x}{\textit{Ind}}\\
%                          \smalltfield{age}{\textit{Real}}\\
%                          \smalltfield{c$_{\mathrm{age}}$}{age\_of(x,age)}}
 
% \item $\zeta_{\mathrm{age}}$ = 
% $\lambda r$:\textit{AgeFrame}
%                        . $r$.age 
 
% \item $\lambda r$:\smallrecord{\smalltfield{sc}{(\textit{AgeFrame}$\to\textit{Real}$)}}
%   . \\
% \hspace*{1em}$\lambda\mathcal{Q}$:\textit{Quant} . \\
% \hspace*{2em}
% \record{\field{bg}{\textit{AgeFrame}}\\
%              \field{fg}{$\lambda r_1$:\smallrecord{\smalltfield{x}{AgeFrame}}
% . \\
% & & \hspace*{1em}$\mathcal{Q}$(\record{\field{bg}{\smallrecord{\smalltfield{x}{\textit{Real}}}}\\
%                            \field{fg}{$\lambda
% r_2$:\smallrecord{\smalltfield{x}{\textit{Real}}}
% . \record{\mfield{x}{$r$.sc($r_1$.x), $r_2$.x}{\textit{Real}}\\
%           \tfield{e}{be(x)}}
% }
% }
% )}} 
 
% \end{subex} 
   
% \end{ex} 
% The idea is that \preveg{c} will be created as a resource on the basis
% of the existence of \preveg{b} which will in turn rely on the fact
% that \preveg{a} is a resource.  \preveg{c} is available as a meaning
% of \textit{be} in English but not for German.

To complete the picture we need to account for \textit{nine} and
\textit{ninety}.  We will treat these as logically proper names of
real numbers.  Thus we will not treat them as introducing
presuppositions in the manner in which we suggested in
Chapter~\ref{ch:propnames} but rather in the Montague-like manner
which we used for proper names in Chapter~\ref{ch:gram}, except that
we now adjust it to take account of parametric contents and the new
definition of \textit{Ppty} (property).
% \begin{ex} 
% \record{\tfield{bg}{\textit{Type}}\\
%         \tfield{fg}{(\smallrecord{\smalltfield{x}{bg}}$\to$\textit{RecType})}} 
% \end{ex} 
% If $T$ is a type we let \textit{Ppty}($T$) represent the partial
% specification of \textit{Ppty} presented in \nexteg{}.
% \begin{ex} 
% \record{\mfield{bg}{$T$}{\textit{Type}}\\
%         \tfield{fg}{(\smallrecord{\smalltfield{x}{bg}}$\to$\textit{RecType})}} 
% \end{ex}        
In $n$ is a (real) number, then SemNumeral($n$) (the content for a
number expression such as \textit{nine}) is as given in
\nexteg{}.
\begin{ex} 
$\ulcorner\lambda c$:\textit{Rec} . $\lambda P$:\textit{Ppty} . $P$(\smallrecord{\field{x}{$n$}})$\urcorner$ 
\end{ex} 
Then we can define Lex$_{\mathrm{numeral}}$ as an operation which
takes a phonological type $T_{\mathrm{phon}}$ and a (real) number $n$
and returns the sign type \nexteg{}.
\begin{ex} 
Lex$_{\mathrm{numeral}}$($T_{\mathrm{phon}}$, $n$) = \\
\hspace*{1em}Lex($T_{\mathrm{phon}}$, \textit{NP}) \d{$\wedge$} \smallrecord{\smallmfield{cnt}{SemNumeral($n$)}{\textit{PQuant}}}  
\end{ex} 
The two sign types that we need as resources for our small fragment are given in
\nexteg{}.
\begin{ex} 
\begin{subex} 
 
\item Lex$_{\mathrm{numeral}}$(``nine'', 9) 
 
\item Lex$_{\mathrm{numeral}}$(``ninety'', 90)  
 
\end{subex} 
   
\end{ex}

% Since we are now using two methods of semantic combination for
% contents, simple function application for \textit{Det N} constructions
% and our variant of S-combination for other constructions we need to
% use the variant of CntForwardApp from Chapter~\ref{ch:gram} for the
% former and the variant from Chapter~\ref{ch:propnames} for the latter.
% We will here call the Chapter~\ref{ch:gram} version CntForwardApp as
% before and rename the Chapter~\ref{ch:propnames} version to CntSForwardApp.  
% Details are given in Appendices~\ref{app:signcomb} and \ref{app:interpps}.


  

\section{Passengers and ships}
\label{sec:passengers}
\cite{Gupta1980} points out examples such as \nexteg{}.
\begin{ex} 
\begin{subex} 
 
\item National Airlines served at least two million passengers in 1975 
 
\item Every passenger is a person

\item National Airlines served at least two million persons in 1975 
 
\end{subex} 
\label{ex:NA2mill}   
\end{ex} 
His claim is that we cannot conclude \preveg{c} from \preveg{a,b}.
There is a reading of \preveg{a} where what is being counted is not
passengers as individual people but passenger events, events of people
taking flights, where possibly the same people are involved in several
flights.  Gupta claims that it is the only reading that this
sentence has.  While it is certainly the preferred reading for this
sentence (say, in the context of National Airlines' annual report or
advertizing campaign), I think the sentence also has a reading where
individuals are being counted.  Consider \nexteg{}.
\begin{ex} 
National Airlines served at least two million passengers in 1975.
Each one of them signed the petition.  
\end{ex} 
While \preveg{} could mean that a number of passengers signed the
petition several times our knowledge that people normally only sign a
given petition once makes a reading where there are two million
distinct individuals involved more likely.  Similarly, while
(\ref{ex:NA2mill}c) seems to prefer the individual reading where there
are two million distinct individuals it is not impossible to get an
event reading here.  \cite{Krifka1990} makes a similar point.
Gupta's analysis of such examples involves individual concepts and is
therefore reminiscent of the functional concepts used by
\cite{Lobner1979,Lobner1981} to analyze the Partee puzzle.

\cite{Carlson1982} makes a similar point about Gupta's examples in
that nouns which appear to normally point to individual related
readings can in the right context get the event related readings.  One
of his examples is a traffic engineer's report as in \nexteg{}.
\begin{ex}
Of the 1,000 cars
using Elm St. over the past 49 hours, only 12 cars made noise in
excess of EPA recommended limits.
\end{ex}
It is easy to interpret this in terms of 1,000 and 12 car events
rather than individual cars.  Carlson's suggestion is to use his
notion of \textit{individual stage}, what he describes intuitively as
``things-at-a-time''.  \cite{Krifka1990} remarks that ``Carlson's
notion of a stage serves basically to reconstruct events''.  While
this is not literally correct, the intuition is nevertheless right.
Carlson was writing at a time when times and time intervals were used
to attempt to capture phenomena that in more modern semantics would be
analyzed in terms of events or situations.  Thus Carlson's notion of
stage is related to a frame-theoretic approach which associates an
individual with an event.  

Consider the noun \textit{passenger}.  It would be natural to assume
that passengers are associated with journey events.  FrameNet\footnote{As of 13th May 2015.}
does not have an entry for \textit{passenger}.  The closest relevant
frame appears to be TRAVEL which has frame elements for traveller,
source, goal, path, direction, mode of transport, among others. The
FrameNet lexical entry for \textit{journey} is associated with this
frame.  Let us take the type \textit{TravelFrame} to be the stripped
down version of the travel frame type in \nexteg{a}.  Then we could
take the type \textit{PassengerFrame} to be \nexteg{b}.



\begin{ex} 
\begin{subex} 
 
\item \record{\tfield{traveller}{\textit{Ind}}\\
              \tfield{source}{\textit{Loc}}\\
              \tfield{goal}{\textit{Loc}}}
 
\item \record{\tfield{x}{\textit{Ind}}\\
              \tfield{e}{passenger(x)}\\
              \tfield{journey}{\textit{TravelFrame}}\\
              \tfield{c$_{\mathrm{travel}}$}{take\_journey(x, journey)}}
 
\end{subex} 
   
\end{ex}

Here `take\_journey' is a predicate with arity
$\langle\textit{Ind},\textit{TravelFrame}\rangle$ with the witness
condition 
in \nexteg{}.
\begin{ex} 
  If $a$:\textit{Ind} and $e$:\textit{TravelFrame}, then
  \begin{quote}
    $s$ : 
    take\_journey($a$, $e$) iff $s=e$ and $e$.traveller = $a$
  \end{quote}
  
\end{ex}

Let us suppose that the basic lexical entry for \textit{passenger} is
\nexteg{}, where `passenger' is a predicate with arity $\langle\textit{Ind}\rangle$.  % This will mean that its (parametric) content is
% \nexteg{b} (with vacuous dependence on the context).
\begin{ex} 
 Lex$_{\mathrm{CommonNoun}}$(``passenger'', \textit{Rec}, passenger) 
 
% \item \record{\field{bg}{\textit{Rec}}\\
%               \field{fg}{$\lambda c$:\textit{Rec} . 
%                             \record{\field{bg}{\textit{Ind}}\\
%                                     \field{fg}{$\lambda
%                                       r$:\smallrecord{\smalltfield{x}{\textit{Ind}}}
%                                       . passenger($r$.x)}}}} 
 
% \end{subex} 
   
\end{ex} 
% That is, its non-parametric content is a property of individuals.
% Given the coercion CommonNounIndToFrame we have defined we can coerce
% this lexical item to \nexteg{a} which means that its parametric
% content will be \nexteg{b}.
% \begin{ex} 
% \begin{subex} 
 
% \item \mbox{Lex$_{\mathrm{CommonNoun}}$(``passenger'', passenger\_frame,
%   \textit{Rec}, \textit{Rec}, \textit{Rec})} 
 
% \item \record{\field{bg}{\textit{Rec}}\\
%               \field{fg}{$\lambda c$:\textit{Rec} . 
%                             \record{\field{bg}{\textit{Rec}}\\
%                                     \field{fg}{$\lambda
%                                       r$:\smallrecord{\smalltfield{x}{\textit{Rec}}}
%                                       . passenger\_frame($r$.x)}}}} 
 
% \end{subex} 
   
% \end{ex}
Let us call \preveg{} $\Sigma_{\text{``passenger''}}$.  Just as we did
for \textit{dog} we can introduce a coerced version of the lexical
entry for \textit{passenger} as in \nexteg{}, using the type \textit{PassengerFrame}.
\begin{ex} 
RestrictCommonNoun(CommonNounIndToFrame($\Sigma_{\text{``passenger''}}$),
\textit{PassengerFrame})
\label{ex:passengerrestricted}
\end{ex} 
  
% This means that the non-parametric content is a property of frames.  An agent who has the frame type \textit{PassengerFrame} available as a
% resource can use it to restrict the domain of the property using the
% coercion RestrictCommonNouns.  This produces \nexteg{a} which means
% that its parametric content will be \nexteg{b}.
% \begin{ex} 
% \begin{subex} 
 
% \item \mbox{Lex$_{\mathrm{CommonNoun}}$(``passenger'', passenger\_frame,
%   \textit{Rec}, \textit{PassengerFrame}, \textit{Rec})} 
 
% \item \record{\field{bg}{\textit{Rec}}\\
%               \field{fg}{$\lambda c$:\textit{Rec} . 
%                             \smallrecord{\field{bg}{\textit{PassengerFrame}}\\
%                                     \field{fg}{$\lambda
%                                       r$:\smallrecord{\smalltfield{x}{\textit{PassengerFrame}}}
%                                       . \smallrecord{\smalltfield{e}{passenger\_frame($r$.x)}}}}}} 
 
% \end{subex} 
% \label{ex:passengerrestricted}   
% \end{ex}
This means that the non-parametric content will now be a property of
passenger frames of type \textit{PassengerFrame}.  This introduces not
only a passenger but also a journey, an event in which in which the
passenger is the traveller.

It seems that we have now done something which \cite{Krifka1990}
explicitly warned us against.  At the end of his discussion of
Carlson's analysis he comes to the conclusion that it is wrong to look
for an explanation of event-related readings of these sentences in
terms of a noun ambiguity.  One of Krifka's examples is \nexteg{}
(which gives the title to his paper).
\begin{ex} 
Four thousand ships passed through the lock 
\end{ex} 
This can either mean that four thousand distinct ships passed through
the lock or that the there were four thousand
ship-passing-through-the-lock events a number of which involved the
same ships.  The problem he sees is that if we treat \textit{ship} as
being ambiguous between denoting individual ships or ship stages in
Carlson's sense then there will be too many stages which pass through
the lock.  For example, suppose that a particular ship passes through
the lock twice.  This gives us two stages of the ship which pass
through the lock.  But then, Krifka claims, there will be a third
stage, the sum of the first two, which also passes through the lock.
It is not clear to me that this is an insuperable problem for the
stage analysis.  We need to count stages that pass through the lock
exactly once.  Let us see how the frame analysis fares.

We will start with a singular example in order to avoid the additional
problems offered by the plural.  Consider \nexteg{}.
\begin{ex} 
Every passenger gets a hot meal 
\end{ex} 
Suppose that an airline has this as part of its advertizing campaign.
Smith, a frequent traveller,  takes a flight with the airline and as
expected gets a hot meal.  A few weeks later she takes another flight
with the same airline and does not get a hot meal.  She sues the
airline for false advertizing.  At the hearing, her lawyer argues,
citing \cite{Gupta1980}, that the advertizing campaign claims that
every passenger gets a hot meal on every flight they take.  The lawyer for the
airline company argues, citing \cite{Krifka1990}, that the sentence in
question is ambiguous between an individual and an event reading, that
the airline had intended the individual reading and thus the
requirements of the advertizing campaign had been met by the meal that
Smith was served on the first flight.  Smith's lawyer then calls an
expert witness, a linguist who quickly crowdsources a survey of native
speakers' interpretations of the sentence in the context of the
campaign and discovers that there is an overwhelming preference for
the meal-on-every-flight reading.  (The small percentage of
respondents who preferred the individual reading over the event
reading gave their occupation as professional logician.)  Smith wins
the case and receives an additional hot meal.

What is important for us at the moment is the fact that there is an
event reading of this sentence.  We will return to the matter of
preferred readings below. We will treat the content of \textit{every}
on the model of the content of the indefinite article, except that the
quantifier relation will be `every' instead of `exist'.  Thus we will
define SemUniversal on the model of SemIndefArt.\footnote{We leave to one side the
  issue of whether \textit{every} should introduce a background
  constraint that there are at least three objects which have the
  property associated with the noun.}  This is given in \nexteg{}.
\begin{ex} 
% $\lambda Q$:\textit{PPpty} . 
% \smallrecord{\field{bg}{\smallrecord{\smalltfield{f}{\textit{Rec}}\\
%                                      \smalltfield{a}{$Q$.bg}}}\\
%              \field{fg}{$\lambda r$:\smallrecord{\smalltfield{f}{\textit{Rec}}\\
%                                      \smalltfield{a}{$Q$.bg}} . 
% $\lambda P$:\textit{Ppty} . 
% \smallrecord{\smallmfield{restr}{$Q$.fg($r$.a)}{\textit{Ppty}}\\
%              \smallmfield{scope}{$P$}{\textit{Ppty}}\\
%              \smalltfield{e}{every(restr, scope)}}}}
  $\lambda Q$:\textit{Ppty} . \\
  \hspace*{1em}$\ulcorner\lambda c$:\textit{Rec} . \\
  \hspace*{2em}$\lambda P$:\textit{Ppty} . \\
  \hspace*{3em}\record{
    \mfield{restr}{$Q$}{\textit{Ppty}}\\
    \mfield{scope}{$P$}{\textit{Ppty}}\\
    \tfield{e}{every(restr, scope)}}$\urcorner$
\end{ex} 
   


If we use the content associated with
\textit{passenger} in (\ref{ex:passengerrestricted}) the
content associated with \textit{every passenger} will
be \nexteg{}.
\begin{ex}
$\ulcorner\lambda c$:\textit{Rec} . \\
  \hspace*{1em}$\lambda P$:\textit{Ppty} . \\
  \hspace*{2em}\record{
    \mfield{restr}{$\ulcorner\lambda                                    r$:\smallrecord{\smalltfield{x}{\textit{PassengerFrame}}} . passenger\_frame($r$.x)$\urcorner$}{\textit{Ppty}}\\
    \mfield{scope}{$P$}{\textit{Ppty}}\\
    \tfield{e}{every(restr, scope)}}$\urcorner$  
% $\lambda P$:\textit{Ppty} . 
% \smallrecord{\smallmfield{restr}{\smallrecord{\field{bg}{\textit{PassengerFrame}}\\
%                                     \field{fg}{$\lambda
%                                    r$:\smallrecord{\smalltfield{x}{\textit{PassengerFrame}}}
%                                       . passenger\_frame($r$.x)}}}{\textit{Ppty}}\\
%              \smallmfield{scope}{$P$}{\textit{Ppty}}\\
%              \smalltfield{e}{every(restr, scope)}} 
\label{ex:conteverypassenger}
\end{ex} 
In order to simplify matters let us treat \textit{gets a hot meal} as
if it were an intransitive verb corresponding to a single predicate
`get\_a\_hot\_meal'.  This is a predicate whose arity is
$\langle$\textit{Ind}$\rangle$.  It is individuals, not frames
(situations), that get hot meals.  Thus the content of
\textit{gets a hot meal} will be \nexteg{}.
\begin{ex} 
$\ulcorner\lambda c$:\textit{Rec} . $\ulcorner\lambda r$:\smallrecord{\smalltfield{x}{\textit{Ind}}} . 
                 \record{\tfield{e}{get\_a\_hot\_meal($r$.x)}}$\urcorner\urcorner$ 
\label{ex:contgetahotmeal}
\end{ex} 
While \preveg{} is the right type of argument for
(\ref{ex:conteverypassenger}) since it is a property it will lead us
eventually into problems because there is nothing which is both a
passenger frame and an individual for the reasons discussed in
Section~\ref{sec:dgqs}.  What we need is a coercion which will obtain
a frame level intransitive verb to match the frame level noun.  This
would be a coercion IntransVerbIndToFrame exactly parallel to
CommonNounIndToFrame defined in (\ref{ex:CommonNounIndToFrame}).  Thus
IntransVerbIndToFrame is defined as in \nexteg{}.



\begin{ex}
  If $T_{\mathrm{phon}}$ is a phonological type, $p$ is a predicate with
arity $\langle\textit{Ind}\rangle$ and
$T_\mathrm{bg}$ is a record type (the ``background type'' or
``presupposition'') then \\ 
\mbox{IntransVerbIndToFrame(Lex$_{\mathrm{IntransVerb}}$($T_{\mathrm{phon}}$,
  $T_\mathrm{bg}$, $p$))} =
\begin{quote}
  Lex$_{\mathrm{IntransVerb}}$($T_{\mathrm{phon}}$,
  $T_\mathrm{bg}$, $p$\_frame)
\end{quote}
% If $T_{\mathrm{phon}}$ is a phonological type, $p$ is a predicate and
% $T_\mathrm{bg}$ is a record type (the ``background type'' or
% ``presupposition'') then \\ 
% \mbox{IntransVerbIndToFrame(Lex$_{\mathrm{IntransVerb}}$($T_{\mathrm{phon}}$,
% $p$, \textit{Ind}, \textit{Ind}, $T_\mathrm{bg}$))} = Lex$_{\mathrm{IntransVerb}}$($T_{\mathrm{phon}}$,
% $p$\_frame, \textit{Rec}, \textit{Rec}, $T_\mathrm{bg}$)
\label{ex:IntransVerbIndToFrame} 
\end{ex} 
Thus the new non-parametric content derived for
\textit{get\_a\_hot\_meal} will be \nexteg{}.
\begin{ex} 
$\ulcorner\lambda c$:\smallrecord{Rec} . $\ulcorner\lambda r$:\smallrecord{\smalltfield{x}{\textit{Rec}}} . 
                 \record{\tfield{e}{get\_a\_hot\_meal\_frame($r$.x)}}$\urcorner\urcorner$
\label{ex:contgetahotmealframe}  
\end{ex}

% Recall that if $p$ is a predicate of individuals then $p$\_frame is a
% predicate of frames that contain an individual of which $p$ holds (as
% required in (\ref{ex:predframe}).  This means that an argument, $r$, to
% `get\_a\_hot\_meal\_frame' which makes the type
% `get\_a\_hot\_meal\_frame($r$)' non-empty
A ``get\_a\_hot\_meal'' frame will, according to our characterization
of predicates of the form `$p$\_frame'  be of type \nexteg{}.
\begin{ex} 
\record{\tfield{x}{\textit{Ind}}\\
        \tfield{e}{get\_a\_hot\_meal(x)}} 
\end{ex} 
Intuitively the `every'
relation holding between the two frame-level coerced individual
properties corresponding to \textit{passenger} and
\textit{get\_a\_hot\_meal} will mean ``every frame (situation)
containing an individual in the `x'-field who is a passenger taking a
journey will be a frame where the individual in the `x'-field gets a
hot meal''.  Or, more formally, \nexteg{}.
\begin{ex} 
every $r$ of type \record{\tfield{x}{\textit{Ind}}\\
              \tfield{e}{passenger(x)}\\
              \tfield{journey}{\textit{TravelFrame}}\\
              \tfield{c$_{\mathrm{travel}}$}{take\_journey(x,
                journey)}} 
is of type  \record{\tfield{x}{\textit{Ind}}\\
        \tfield{e}{get\_a\_hot\_meal(x)}} 
\end{ex} 
This means that every frame of type \textit{PassengerFrame} will be of
type \nexteg{a}, that is \nexteg{b} which is
      identical with \nexteg{c}.
\begin{ex} 
\begin{subex}

\item \textit{PassengerFrame}\d{$\wedge$}\smallrecord{\smalltfield{x}{\textit{Ind}}\\
        \smalltfield{e}{get\_a\_hot\_meal(x)}} 
 
\item  \record{\tfield{x}{\textit{Ind}}\\
              \tfield{e}{passenger(x)}\\
              \tfield{journey}{\textit{TravelFrame}}\\
              \tfield{c$_{\mathrm{travel}}$}{take\_journey(x,
                journey)}} \\ \d{$\wedge$}
\record{\tfield{x}{\textit{Ind}}\\
        \tfield{e}{get\_a\_hot\_meal(x)}}
 
\item  \record{\tfield{x}{\textit{Ind}}\\
              \tfield{e}{passenger(x)$\wedge$get\_a\_hot\_meal(x)}\\
              \tfield{journey}{\textit{TravelFrame}}\\
              \tfield{c$_{\mathrm{travel}}$}{take\_journey(x,
                journey)}}
 
\end{subex} 
   
\end{ex} 
     

Thus even though we have coerced to a frame-level reading
it is still the passengers (i.e. individuals) in the frames who are getting
the hot meal not the situation which is the frame.

Things go less well with cardinality quantifiers, however.  Consider
\textit{2000 passengers get a hot meal} which corresponds to
\nexteg{}.
\begin{ex} 
2000 $r$ of type \record{\tfield{x}{\textit{Ind}}\\
              \tfield{e}{passenger(x)}\\
              \tfield{journey}{\textit{TravelFrame}}\\
              \tfield{c$_{\mathrm{travel}}$}{take\_journey(x,
                journey)}} 
are of type  \record{\tfield{x}{\textit{Ind}}\\
        \tfield{e}{get\_a\_hot\_meal(x)}} 
\end{ex} 
The problem is not exactly the same as the problem which Krifka
foresaw with the summing of stages although it is intuitively
related.  It has to do with the way we have set up subtyping with
record types.  Given a record of a type we can always add a new field
to the record and obtain a distinct record of the same type.
Trivially the field we add could contain an object already occurring
in a field in the original record.  As we
are assuming that the set of labels is countably infinite if there is
one record of a given type there will be infinitely many records of
the same type.  We illustrate this with an abstract example in
\nexteg{}.
\begin{ex} 
\begin{subex} 
 
\item \record{\tfield{$\ell_1$}{$T_1$}\\
              \tfield{$\ell_2$}{$T_2$}} 
 
\item \record{\field{$\ell_1$}{$a$}\\
              \field{$\ell_2$}{$b$}}

\item \record{\field{$\ell_1$}{$a$}\\
              \field{$\ell_2$}{$b$}\\
              \field{$\ell_3$}{$a$}}

\item \record{\field{$\ell_1$}{$a$}\\
              \field{$\ell_2$}{$b$}\\
              \field{$\ell_3$}{$a$}\\
              \field{$\ell_4$}{$a$}}

\item \ldots
 
\end{subex} 
   
\end{ex} 
If \preveg{b} is of type \preveg{a} (i.e. $a:T_1$ and $b:T_2$), then so are \preveg{c} and
\preveg{d} and so on as we successively ``grow'' the record without
changing the fields that make the records a witness for the type and
without necessarily adding anything new in the new fields. If records
model events (situations) then this corresponds to the intuition that given any event there will always be a
larger event of which it is a part.  For example, if I wash my hands that
is part of an event in which I wash my hands and stand at the
washbasin.  This is in turn part of an event in which I wash my hands,
stand at the washbasin and breathe and so on.  We want this to be true
but still there is the robust intuition that we are only talking about
one event of washing my hands here which is part of infinitely many
larger events.

Fortunately, this problem is easy to fix.  We use the notion of one
record being a proper part of another, that is the set of paths of the
first is a proper subset of the set of paths of the second and the
objects at the end of the total paths of the first are the same as
those at the end of the corresponding total path in the second.
\begin{shaded}
Technically this is given by
the definition in \nexteg{} (repeated in Appendix~\ref{app:settypes}).  % Recall that records are
% essentially sets of fields (Appendix~\ref{app:rec}).  As a first
% approximation we can say that a
% record, $r_1$, is a \textit{proper part of a record}, $r_2$, $r_1<r_2$, just in
% case $r_1$ is a proper subset of $r_2$. This definition is not quite
% sufficient, however, since records can contain records and we wish the
% proper part of relation to be recursive.  Consider \nexteg{}.  We
% would like to say that \nexteg{a} is a proper part of \nexteg{b} even
% though \nexteg{a} is not a proper subset of \nexteg{b}.
% %We can then define a notion of
% % being a \textit{minimal witness} for a record type as in \nexteg{}.
% % \begin{ex} 
% % If $T$ is a record type, $r$ is a \textit{minimal witness} for $T$,
% % $r:_{\mathrm{min}}T$, iff $r:T$ and there is no $r'<\varphi(r)$ such
% % that $r':\varphi(T)$. 
% % \end{ex} 
% % The use of flattening (represented by $\varphi$, see
% % Appendix~\ref{app:rectypes}) is to get the minimality to recurse
% % inside the records.  For example, \nexteg{b} is not a minimal witness
% % for \nexteg{a} (where the $T_i$ are non-record types) even if $a:T_1$,
% % $b:T_2$ and $d:T_3$ because of \nexteg{c}.
% \begin{ex} 
% \begin{subex} 
 
% % \item \record{\tfield{$\ell_1$}{\record{\tfield{$\ell_1$}{$T_1$}\\
% %                                         \tfield{$\ell_2$}{$T_2$}}}\\
% %               \tfield{$\ell_3$}{$T_3$}}

% \item \record{\field{$\ell_1$}{\record{\field{$\ell_1$}{$a$}\\
%                                        \field{$\ell_2$}{$b$}
%                                        }}\\
%               \field{$\ell_3$}{$d$}}
 
% \item \record{\field{$\ell_1$}{\record{\field{$\ell_1$}{$a$}\\
%                                        \field{$\ell_2$}{$b$}\\
%                                        \field{$\ell_3$}{$c$}}}\\
%               \field{$\ell_3$}{$d$}} 

% % \item \record{\field{$\ell_1.\ell_1$}{$a$}\\
% %               \field{$\ell_1.\ell_2$}{$b$}\\
% %               \field{$\ell_1.\ell_3$}{$c$}\\
% %               \field{$\ell_3$}{$d$}}

% % is not a minimal record of type

% % \record{\tfield{$\ell_1.\ell_1$}{$T_1$}\\
% %         \tfield{$\ell_1.\ell_2$}{$T_2$}\\
% %         \tfield{$\ell_3$}{$T_3$}}
 
% \end{subex} 
   
% \end{ex}
% We achieve this by defining the proper part of relation in terms of
% the flattening operation on records (represented by $\varphi$, see
% Appendix~\ref{app:rectypes}).  Thus if we take the flattenings of the
% records in \preveg{} as in \nexteg{} we see that\nexteg{a} is a proper
% subset of \nexteg{b}.
% \begin{ex} 
% \begin{subex} 
 
% \item \record{\field{$\ell_1.\ell_1$}{$a$}\\
%               \field{$\ell_1.\ell_2$}{$b$}\\
%               \field{$\ell_3$}{$d$}} 
 
% \item \record{\field{$\ell_1.\ell_1$}{$a$}\\
%               \field{$\ell_1.\ell_2$}{$b$}\\
%               \field{$\ell_1.\ell_3$}{$c$}\\
%               \field{$\ell_3$}{$d$}} 
 
% \end{subex} 
   
% \end{ex} 
% We define the proper part of relation in \nexteg{}.
\begin{ex} 
% \begin{subex} 
 
% \item
  $r_1$ \textit{is a proper
    part of} $r_2$, $r_1<r_2$, just in case
  \begin{enumerate}
  \item $r_1$ and $r_2$ are records, 
  \item
    $\mathrm{paths}_{\text{rec}}(r_1)\subset\mathrm{paths}_{\text{rec}}(r_2)$
      and
    \item for all $\pi\in\mathrm{tpaths}(r_1)$, $r_1.\pi=r_2.\pi$
    \end{enumerate}

    
    
 
% \item If $o_1$ and $o_2$ are objects of some type and at least one of
%   them is not of type \textit{Rec}, then $o_1$ is \textit{not} a
%   proper part of $o_2$, $o_1\not<o_2$ 
 
% \end{subex} 
\label{ex:properpart}   
\end{ex}
\end{shaded}
    
% We can generalize the notion of minimal witness to types other than
% record types by adding the clause in \nexteg{}.
% \begin{ex} 
% If $T$ is a type other than a record type, $a:_{\mathrm{min}}T$ iff $a:T$. 
% \end{ex} 



This notion yields a notion of minimal object of a given type which is related to Schubert's
\citeyear{Schubert2000} notion of characterization discussed on 
p.~\pageref{pg:Schubert-characterize}.  It is different from Schubert's notion in
that we do not say that there are no other types to which the
situation belongs but rather that no proper part of the situation is
of the type.  In this way it is related to the notion of minimal
situation discussed by \cite{Kratzer2014} and elsewhere in earlier
work. It is also, of course, related to mereological approaches that
have been used, for example,  in approaches to the analysis of the
plural as in \cite{Krifka1990} and much other literature.  It is this
we will exploit in our analysis of the plural cardinality
quantifiers.  We introduce the notion of a \textit{plurality} as a set of
objects of a given type that does not contain two objects
where one is a proper part of the other. 

\begin{shaded}
We characterize a notion of plurality types as in \nexteg{} (repeated
in Appendix~\ref{app:settypes}). 
\begin{ex}
  A system of complex types {\bf TYPE$_C$} = $\langle${\bf Type}, {\bf BType},
$\langle$\textbf{PType}, {\bf Pred}, \textbf{ArgIndices}, {\it
  Arity\/}$\rangle$, $\langle A,F\rangle$$\rangle$ with set types \textit{has plurality types} if
\begin{enumerate} 
 
\item for any $T \in \textbf{Type}$, $\mathrm{plurality}(T) \in \textbf{Type}$ 
 
\item for any $T \in \textbf{Type}$, 
$A:_{\mathbf{TYPE_C}}\mathrm{plurality}(T)$ iff  
\begin{enumerate} 
 
\item $A :_{\mathbf{TYPE_C}} \mathrm{set}(T)$ 
 
\item if $a\in A$ then for any $b$ such that $a<b$, $b\not\in A$ 
 
\end{enumerate} 
  
   
 
\end{enumerate}

% \begin{subex} 
 
% \item If $T$ is a type, then $\{\!\mid T\mid\!\}$ is also a type (the
%   type of pluralities such that every element of the plurality is of
%   type $T$)
 
% \item $A:\{\!\mid T\mid\!\}$ iff
% \begin{enumerate} 
 
% \item $A : \{T\}$ 
 
% %\item $\mid A\mid\ \geq 2$

% \item if $a\in A$ then for any $b$ such that $a<b$, $b\not\in A$
 
% \end{enumerate} 
  
 
% \end{subex} 
\label{ex:plurality-types}   
\end{ex}

In terms of our informal proof theoretic notation this can be
expressed as \nexteg{}.
\begin{ex}
  For $\Gamma$ a system of complex types with set types,
\begin{subex} 
 
\item
  \begin{prooftree}
    \hypo{\Gamma\vdash T\in\textbf{Type}}
    \infer1{\Gamma\vdash\mathrm{plurality}(T)\in\textbf{Type}}
  \end{prooftree}
  
 
\item
  \begin{prooftree}
    \hypo{\Gamma\vdash A:\mathrm{set}(T)}
    \hypo{[a\in A, a<b]}
    \ellipsis{}{b\not\in A}
    \infer2{\Gamma\vdash A:\mathrm{plurality}(T)}
  \end{prooftree}
  
 
\end{subex} 
   
\end{ex} 
  
\end{shaded}



It might seem natural to
require that a plurality contains at least two objects.  The choice
not to place this requirement on a plurality makes this analysis
number neutral in the sense of \cite{Zweig2008,Zweig2009}.
\cite{Zweig2008} contains a useful overview of some of the variants of
analyses of the plural that have been proposed in the literature,
including the distinction between set-based and sum-based analyses.
In the type theory we have proposed we have sets already available and
a kind of mereology based on the structure of records, as illustrated
in (\ref{ex:properpart}), and we have used a combination of these in
our characterization of plurality.  Whether this proposal would
survive an in-depth investigation of the plural in this framework is
an open question.  In particular the work on mass terms by
\cite{SuttonFilip2017} that we will in any case need an additional
sum-structure.

@@\todo{Get rid of \textit{PlPpty}.  Define SemCommonNounPl etc.?}
    
We propose here a treatment of basic plural
quantification cases involving cardinality quantification that will
allow us to say something about the content of \textit{2000 passengers
  get a hot meal}.  We first distinguish between singular and plural
properties.  The definition of \textit{Ppty} given in
(\ref{ex:dog-Ppty}b) and repeated in \nexteg{a} becomes our definition
of the type of singular properties, \textit{SgPpty}.
\begin{ex} 
\begin{subex} 
 
\item \textit{SgPpty} $\equiv$ \record{\tfield{bg}{\textit{Type}}\\
              \tfield{fg}{(\smallrecord{\smalltfield{x}{bg}}$\rightarrow$\textit{RecType})}}  
 
\item \textit{PlPpty} $\equiv$ \record{\tfield{bg}{\textit{Type}}\\
              \tfield{fg}{(\smallrecord{\smalltfield{x}{$\{\!\mid$bg$\mid\!\}$}}$\rightarrow$\textit{RecType})}}

\item  \textit{Ppty} $\equiv$ \textit{SgPpty}$\vee$\textit{PlPpty}
 
\end{subex} 
   
\end{ex} 
The type of plural properties, \textit{PlPpty}, given in \preveg{b},
requires the foreground of the property to be a function which has as
its domain records whose `x'-field contains a plurality of objects of
the background type.  We then redefine \textit{Ppty} in \preveg{c} as
being the type of objects which are either singular or plural
properties.

Note that according to \preveg{} there is no constraint on the kinds
of types which can be the backgrounds of either singular or plural
properties.  Thus a singular property could have a plurality type as
its background.  This could be used for nouns like \textit{committee}
which seems to represent a property of a plurality of people.  Note
that we can create a plurality type of any type including plurality
types so we have an infinite hierarchy of plurality types.  This can
be seen in examples like \textit{league} (of baseball teams) where
each element in the league, i.e. a team, is itself a plurality of
baseball players.  These kinds of examples were noted in the earliest
literature on treating the plural in Montague semantics as problematic
for Montague's approach since they appeared to involve an infinite
hierarchy of types which would have to correspond to an infinite
hierarchy of syntactic categories \citep{Bennett1974}.   Montague's type
system lacked the option of creating a single type corresponding to
the infinite hierarchy in the way we are doing here.  Plural
properties too can have plurality types as their backgrounds and this
could correspond to plural nouns like \textit{committees} and
\textit{leagues}.  In terms of the domain of the foreground function
of properties there is largely an overlap between singular and plural
properties.  Singular properties allow for both pluralities and
non-pluralities whereas plural properties allow only for pluralities.
It may seem strange from a formal point of view not to make a
non-overlapping division between the two, but this does seem to
correspond to the way the plural works in natural
languages.\footnote{It is not currently clear how \textit{pluralia
    tantum} such as \textit{scissors} and \textit{trousers} fit into
  this story.}

We consider cardinality quantifiers such as \textit{two} and
\textit{two thousand} to correspond to predicates whose arity is
$\langle$\textit{PlPpty}, \textit{PlPpty}$\rangle$.  If $\nu$ is a
natural number (that is, an object of type \textit{Nat}), let $\nu_{\mathrm{pred}}$ be such a predicate
corresponding to $\nu$.  We also introduce a predicate `card'
(``cardinality'') with arities $\langle\{T\},\textit{Card}\rangle$ for
any type, $T$, where \textit{Card} is the type of cardinal numbers
(the natural numbers together with the transfinite cardinals,
$\aleph_0, \aleph_1, \ldots$).  
This predicate obeys the constraint in \nexteg{}.
\begin{ex} 
$\down{\mathrm{card}(X,\nu)}\not=\emptyset$ iff $\mid\!X\!\mid\ =\nu$
\end{ex} 
The cardinality predicates obey the constraint in \nexteg{}.
\begin{ex} 
$\down{\nu_{\mathrm{pred}}(P,Q)}\not=\emptyset$ iff 
\begin{quote}
$\down{\mathcal{F}(Q.\mathrm{fg}\!\mid_{\mathcal{F}(P.\mathrm{fg})})\text{\d{$\wedge$}\smallrecord{\smalltfield{x}{$\{\!\mid\!
        P$.bg$\mid\!\}$}\\
                \smalltfield{c}{card(x, $\nu$)}}}}\not=\emptyset$
\end{quote}
\label{ex:cardconstr} 
\end{ex} 
Let us take this definition through our example \textit{2000
  passengers get a hot meal}.  The relevant type corresponding to the
content of this sentence is given in \nexteg{}.
\begin{ex} 
2000$_{\mathrm{pred}}$(\smallrecord{\field{bg}{\textit{PassengerFrame}}\\
                                    \field{fg}{$\lambda
                                      r$:\smallrecord{\smalltfield{x}{$\{\!\mid$\textit{PassengerFrame}$\mid\!\}$}}
                                      . \smallrecord{\smalltfield{e}{passenger\_frame\_pl($r$.x)}}}}, 
\hspace*{4em}\smallrecord{\field{bg}{\textit{Rec}}\\
             \field{fg}{$\lambda
               r$:\smallrecord{\smalltfield{x}{$\{\!\mid$\textit{Rec}$\mid\!\}$}} . 
                 \smallrecord{\smalltfield{e}{get\_a\_hot\_meal\_frame\_pl($r$.x)}}}}) 
\label{ex:2kpassgahm}
\end{ex} 
The first argument to `2000$_{\mathrm{pred}}$' is a pluralized version
of the property in the restriction field in (\ref{ex:conteverypassenger}). The
second argument is a pluralized version of the property in
(\ref{ex:contgetahotmealframe}). \preveg{} makes use of a
pluralization operation, `\_pl' on predicates which can be introduced
as in \nexteg{}.
\begin{ex} 
If $p$ is a predicate with arity $\langle T\rangle$, then $p$\_pl is a
predicate with arity $\langle\{\!\mid T\mid\!\}\rangle$ 
\end{ex} 
The cases we are considering here are distributive plurals, that is,
the constraint in \nexteg{} holds.
\begin{ex} 
$\down{p\mathrm{\_pl}(A)}\not=\emptyset$ iff $a\in A$ implies $\down{p(a)}\not=\emptyset$ 
\end{ex} 
Let us instantiate (\ref{ex:cardconstr}) bit by bit with
(\ref{ex:2kpassgahm}).  We first compute the fixed point type of the
foreground of the first argument.  That is, \nexteg{a} which is
identical to \nexteg{b}.
\begin{ex} 
\begin{subex} 
 
\item $\mathcal{F}$($\lambda
                                      r$:\smallrecord{\smalltfield{x}{$\{\!\mid$\textit{PassengerFrame}$\mid\!\}$}}
                                      . \\
\hspace*{5em}\smallrecord{\smalltfield{e}{passenger\_frame\_pl($r$.x)}}) 
 
\item
  \record{\tfield{x}{$\{\!\mid$\textit{PassengerFrame}$\mid\!\}$}\\
          \tfield{e}{passenger\_frame\_pl(x)}} 
 
\end{subex} 
   
\end{ex} 
Then we compute the result of restricting the second argument to the
quantifier predicate by \preveg{b}.  This is given in \nexteg{a} which
is identical to \nexteg{b}.
\begin{ex} 
\begin{subex} 
 
\item  $\lambda
               r$:\smallrecord{\smalltfield{x}{$\{\!\mid$\textit{Rec}$\mid\!\}$}} . 
                 \smallrecord{\smalltfield{e}{get\_a\_hot\_meal\_frame\_pl($r$.x)}}$\!\mid_{\text{\smallrecord{\smalltfield{x}{$\{\!\mid$\textit{PassengerFrame}$\mid\!\}$}\\
          \smalltfield{e}{passenger\_frame\_pl(x)}}}}$
 
\item $\lambda
               r$:\smallrecord{\smalltfield{x}{$\{\!\mid$\textit{PassengerFrame}$\mid\!\}$}\\
          \smalltfield{e}{passenger\_frame\_pl(x)}} . \\
\hspace*{5em}                 \smallrecord{\smalltfield{e}{get\_a\_hot\_meal\_frame\_pl($r$.x)}} 
 
\end{subex} 
   
\end{ex} 
We then compute the fixed point type of \preveg{b}, given in
\nexteg{a} which is identical with \nexteg{b}.
\begin{ex} 
\begin{subex} 
 
\item $\mathcal{F}$($\lambda
               r$:\smallrecord{\smalltfield{x}{$\{\!\mid$\textit{PassengerFrame}$\mid\!\}$}\\
          \smalltfield{e}{passenger\_frame\_pl(x)}} . \\
\hspace*{5em}                 \smallrecord{\smalltfield{e}{get\_a\_hot\_meal\_frame\_pl($r$.x)}}) 
 
\item  \smallrecord{\smalltfield{x}{$\{\!\mid$\textit{PassengerFrame}$\mid\!\}$}\\
          \smalltfield{e}{passenger\_frame\_pl(x)$\wedge$get\_a\_hot\_meal\_frame\_pl(x)}}
 
\end{subex} 
   
\end{ex} 
The final step specified in (\ref{ex:cardconstr}) involves merging
\preveg{b} with \nexteg{a}, that is, \nexteg{b} which is identical
with \nexteg{c}.
\begin{ex} 
\begin{subex} 
 
\item
  \record{\tfield{x}{$\{\!\mid$\textit{PassengerFrame}$\mid\!\}$}\\
          \tfield{c}{card(x,2000)}}
 
\item \smallrecord{\smalltfield{x}{$\{\!\mid$\textit{PassengerFrame}$\mid\!\}$}\\
          \smalltfield{e}{passenger\_frame\_pl(x)$\wedge$get\_a\_hot\_meal\_frame\_pl(x)}}
 \d{$\wedge$} 
\smallrecord{\smalltfield{x}{$\{\!\mid$\textit{PassengerFrame}$\mid\!\}$}\\
          \smalltfield{c}{card(x,2000)}}

\item \smallrecord{\smalltfield{x}{$\{\!\mid$\textit{PassengerFrame}$\mid\!\}$}\\
          \smalltfield{e}{passenger\_frame\_pl(x)$\wedge$get\_a\_hot\_meal\_frame\_pl(x)}\\
          \smalltfield{c}{card(x,2000)}} 
 
\end{subex} 
   
\end{ex} 
It is thus the type \preveg{c} which is required to be non-empty by
the content of an utterance of \textit{2000 passengers get a hot
  meal}.  This means that it is required that there is a plurality of
passenger frames where the passenger gets a hot meal and this
plurality has the cardinality 2000, or slightly more colloquially,
there are 2000 separate events of a passenger getting a hot meal.

Requiring that there is a plurality with cardinality 2000 is to say
that there are at least 2000 objects meeting whatever conditions are
invoked.  It does not rule out the possibility of there being a larger
plurality of objects meeting the same conditions.  If we want to
express \textit{exactly $\nu$} we can use the condition in \nexteg{}.
\begin{ex} 
 $\down{\mathrm{exactly}\_\nu_{\mathrm{pred}}(P,Q)}\not=\emptyset$ iff 
\begin{enumerate}
\item $\down{\nu_{\mathrm{pred}}(P,Q)}\not=\emptyset$
\item $\down{\mathrm{at\_most}\_\nu_{\mathrm{pred}}(P,Q)}\not=\emptyset$
\end{enumerate}
\end{ex}
where `at\_most\_$\nu$' obeys the constraint in \nexteg{}. 
\begin{ex} 
 $\down{\mathrm{at\_most}\_\nu_{\mathrm{pred}}(P,Q)}\not=\emptyset$ iff 
\begin{quote}
$\down{\mathcal{F}(Q.\mathrm{fg}\!\mid_{\mathcal{F}(P.\mathrm{fg})})\text{\d{$\wedge$}\smallrecord{\smalltfield{x}{$\{\!\mid\!
        P$.bg$\mid\!\}$}\\
                \smalltfield{n}{\textit{Nat}}\\
                \smalltfield{c$_{\mathrm{n}}$}{n $>\nu$}\\
                \smalltfield{c}{card(x, n)}}}}=\emptyset$
\end{quote}
\end{ex} 
  
            
              


% \section{Comparison with other work on frames in semantics}

\section{Summary of resources introduced}
\label{sec:summresch5}

% This summary does not include the resources for chart processing
% introduced in Section~\ref{sec:chart}.

Items that are new since Chapter~\ref{ch:propnames} are marked
``\textbf{New!}'' and items that have been revised since
Chapter~\ref{ch:propnames} are marked ``\textbf{Revised!}''.  % We have
% included some items for completeness which were not explicitly
% introduced in the text.

\subsection{Universal grammar resources} 

\subsubsection{Types} 

\begin{description}

  \item[\textnormal{\textit{Loc}} Revised!] --- \record{\tfield{x-coord}{\textit{Real}}\\
        \tfield{y-coord}{\textit{Real}}\\
        \tfield{z-coord}{\textit{Real}}}

  
\item[\textnormal{\textit{Phon}}] --- a basic type

  $e$ : \textit{Phon} iff $e$ is a phonological event
  
\item[\textnormal{\textit{SEvent}}] --- \record{\tfield{e-loc}{\textit{Loc}} \\
        \tfield{sp}{\textit{Ind}} \\
        \tfield{au}{\textit{Ind}} \\
        \tfield{e}{\textit{Phon}} \\
        \tfield{c$_{\mathrm{loc}}$}{loc(e,e-loc)} \\
        \tfield{c$_{\mathrm{sp}}$}{speaker(e,sp)} \\
        \tfield{c$_{\mathrm{au}}$}{audience(e,au)}} (as in
      Chapter~\ref{ch:infex})
      
    % \item[\textnormal{\textit{IndPpty}} New!] ---
    %   (\smallrecord{\smalltfield{x}{\textit{Ind}}}$\rightarrow$\textit{RecType})

      
    % \item[\textnormal{\textit{FramePpty}} New!] ---
    %   (\smallrecord{\smalltfield{x}{\textit{Rec}}}$\rightarrow$\textit{RecType})
      
    \item[\textnormal{\textit{xType}} New!] --- a basic type

      $T$ : \textit{xType} iff $T$ : \textit{RecType} and $\text{x}\in\mathrm{labels}(T)$

      \item[\textnormal{\textit{Ppty}} Revised!] ---
        \record{\tfield{bg}{\textit{xType}}\\
              \tfield{fg}(bg$\rightarrow$\textit{RecType})}
        
      \item[\textnormal{\textit{PPpty}}] --- \record{\tfield{bg}{\textit{RecType}} \\
        \tfield{fg}{(bg$\rightarrow$\textit{Ppty})}}
        
      \item[\textnormal{\textit{Quant}}] ---
        (\textit{Ppty}$\rightarrow$\textit{RecType})
        
      \item[\textnormal{\textit{PQuant}}] --- \record{\tfield{bg}{\textit{RecType}} \\
          \tfield{fg}{(bg$\rightarrow$\textit{Quant})}}
        
      \item[\textnormal{\textit{QuantDet}}] ---
        (\textit{Ppty}$\rightarrow$\textit{Quant})
        
      \item[\textnormal{\textit{PQuantDet}}] ---
        \record{
          \tfield{bg}{\textit{RecType}}\\
          \tfield{fg}{(bg$\rightarrow$\textit{QuantDet})}}
        
      \item[\textnormal{\textit{PRecType}}] ---
        \record{
          \tfield{bg}{\textit{RecType}}\\
          \tfield{fg}{(bg$\rightarrow$\textit{RecType})}}
          

    \item[\textnormal{\textit{Cont}}] --- \textit{PRecType}$\vee$\textit{PPpty}$\vee$\textit{PQuant}$\vee$\textit{PQuantDet}

      
    \item[\textnormal{\textit{Cat}}] --- a basic type

      s, np, det, n, v, vp : \textit{Cat}

    
    \item[\textnormal{\textit{Syn}}] ---  \record{\tfield{cat}{\textit{Cat}} \\
        \tfield{daughters}{\textit{Sign}$^*$}} 
 


  
    \item[\textnormal{\textit{Sign}}] ---  a basic type

      $\sigma$ : \textit{Sign} iff $\sigma$ :
      \record{\tfield{s-event}{\textit{SEvent}} \\
         \tfield{syn}{\textit{Syn}} \\
        \tfield{cont}{\textit{Cont}}} 

  
\item[\textnormal{\textit{SignType}}] --- a basic type

  $T:\textit{SignType}$ iff $T\sqsubseteq\textit{Sign}$ 

  
\item[\textnormal{\textit{S}}] --- (\textit{Sign} \d{$\wedge$}
  \smallrecord{\smalltfield{syn}{\smallrecord{\smallmfield{cat}{s}{\textit{Cat}}}}})
  
\item[\textnormal{\textit{NP}}] --- (\textit{Sign} \d{$\wedge$}
  \smallrecord{\smalltfield{syn}{\smallrecord{\smallmfield{cat}{np}{\textit{Cat}}}}})

  
\item[\textnormal{\textit{Det}}] --- (\textit{Sign} \d{$\wedge$}
  \smallrecord{\smalltfield{syn}{\smallrecord{\smallmfield{cat}{det}{\textit{Cat}}}}})
  
\item[\textnormal{\textit{N}}] --- (\textit{Sign} \d{$\wedge$}
  \smallrecord{\smalltfield{syn}{\smallrecord{\smallmfield{cat}{n}{\textit{Cat}}}}})
  
\item[\textnormal{\textit{V}}] --- (\textit{Sign} \d{$\wedge$}
  \smallrecord{\smalltfield{syn}{\smallrecord{\smallmfield{cat}{v}{\textit{Cat}}}}})
  
\item[\textnormal{\textit{VP}}] --- (\textit{Sign} \d{$\wedge$}
  \smallrecord{\smalltfield{syn}{\smallrecord{\smallmfield{cat}{vp}{\textit{Cat}}}}})

  
\item[\textnormal{\textit{NoDaughters}}] ---
  \smallrecord{\smalltfield{syn}{\smallrecord{\smallmfield{daughters}{$\varepsilon$}{\textit{Sign}$^*$}}}}

  
\item[\textnormal{\textit{Real}} New!] --- a basic type

  $n$ : \textit{Real} iff $n$ is a real number
 

  

\item[\textnormal{\textit{AmbTempFrame}} New!] --- \record{\tfield{x}{\textit{Real}} \\
        \tfield{loc}{\textit{Loc}} \\
        \tfield{e}{temp(loc, x)}}

      
    \item[\textnormal{\textit{TempRiseEventCntxt}} New!] ---
      \record{
        \tfield{fix}{\record{
            \tfield{loc}{\textit{Loc}}}}\\
        \tfield{scale}{(\textit{AmbTempFrame}
          $\rightarrow$ \textit{Real})}}
      
    \item[\textnormal{\textit{TempRiseEvent}} New!] ---
      
      $\lambda r$:\textit{TempRiseEventCntxt} .\\  
\hspace*{2em}\record{\tfield{e}{(\textit{AmbTempFrame}$\parallel$$r$.fix)$^2$}\\
        \tfield{c$_{\mathrm{rise}}$}{$r$.scale(e[0]) $<$
          $r$.scale(e[1])}}
      
    \item[\textnormal{\textit{PriceFrame}} New!] --- \record{\tfield{x}{\textit{Real}} \\
        \tfield{loc}{\textit{Loc}} \\
        \tfield{commodity}{\textit{Ind}} \\
        \tfield{e}{price(commodity, loc, x)}}
      
    \item[\textnormal{\textit{PriceRiseEventCntxt}} New!] --- \record{
        \tfield{fix}{\record{
            \tfield{loc}{\textit{Loc}}\\
            \tfield{commodity}{\textit{Ind}}}}\\
        \tfield{scale}{(\textit{PriceFrame}
                             $\rightarrow$ \textit{Real})}}
 
  \item[\textnormal{\textit{PriceRiseEvent}} New!] --- \mbox{}

   $\lambda
r$:\textit{TempRiseEventCntxt} .\\  
\hspace*{2em}\record{\tfield{e}{(\textit{PriceFrame}$\parallel$$r$.fix)$^2$}\\
        \tfield{c$_{\mathrm{rise}}$}{$r$.scale(e[0]) $<$
          $r$.scale(e[1])}}
      
    \item[\textnormal{\textit{LocFrame}} New!] --- \record{\tfield{x}{\textit{Ind}} \\
        \tfield{loc}{\textit{Loc}} \\
        \tfield{e}{at(x, loc)}}
      
    \item[\textnormal{\textit{LocRiseEventCntxt}} New!] ---
      \record{\tfield{fix}{\record{\tfield{x}{\textit{Ind}}
                                               }}\\
                         \tfield{scale}{(\textit{LocFrame}
                           $\rightarrow$ \textit{Real})}}
                       
 \item[\textnormal{\textit{LocRiseEvent}} New!] ---

   $\lambda
r$:\textit{LocRiseEventCntxt} .\\  
\hspace*{2em}\record{\tfield{e}{(\textit{LocFrame}$\parallel$$r$.fix)$^2$}\\
        \tfield{c$_{\mathrm{rise}}$}{$r$.scale(e[0]) $<$ $r$.scale(e[1])}}

                       


\end{description}

 \subsubsection{Predicates} 

\begin{description}

 \item[with arity \textnormal{$\langle$\textit{Phon},
    \textit{Loc}$\rangle$}] \mbox{}
  
    \begin{description}

    \item[\textnormal{loc}] --- $e$ : loc($u$, $l$) iff $u$ is located
      at $l$ in $e$

    \end{description}
    
  \item[with arity \textnormal{$\langle$\textit{Phon},
      \textit{Ind}$\rangle$}] \mbox{}

    \begin{description}

    \item[\textnormal{speaker}] --- $e$ : speaker($u$, $a$) iff $u$
        is the speaker of $u$ in $e$

    \item[\textnormal{audience}] --- $e$ : audience($u$, $a$) iff
        $u$ is the audience of $u$ in $e$

    \end{description}
      
  \item[with arity \textnormal{$\langle$\textit{Ppty}$\rangle$}]
    \mbox{}

    \begin{description}
      
    \item[\textnormal{unique} New!] --- $s:\textrm{unique}(P)$ iff
      $\mid\!\downP{P\!\restriction\!s}\!\mid = 1$

    \end{description}
    

  
\item[with arity \textnormal{$\langle\textit{Ppty},\textit{Ppty}\rangle$}] \mbox{}

  \begin{description}
    
  \item[\textnormal{exist} Revised!] --- % $e$ : exist($P$,$Q$) iff
    % $\downP{P}\cap\downP{Q\!\restriction\! e}\not=\emptyset$
    $s$ : exist($P$,$Q$) iff
    $\downP{P\!\restriction\!s}\cap\downP{Q|_{\mathcal{F}(P.\mathrm{fg})}\!\restriction\!s}\not=\emptyset$
    
  \item[\textnormal{every} New!] --- $s$ : every($P$,$Q$) iff $\downP{P\!\restriction\!s}\subseteq\downP{Q|_{\mathcal{F}(P.\mathrm{fg})}\!\restriction\!s}$

  \end{description}
\item[with arity \textnormal{\{$\langle T\rangle\mid T$ is a type\}}]
  \mbox{}

  \begin{description}
    
  \item[\textnormal{be}] --- $e:\text{be}(a)$ iff $a\varepsilon e$

  \end{description}

  
\item[with arity
  \textnormal{$\langle\textit{Loc},\textit{Real}\rangle$}] \mbox{}
  \begin{description}
    
  \item[\textnormal{temp} New!] --- $e:\text{temp}(l,n)$ iff $n$ is
    the temperature at $l$ in $e$.

    
  

  \end{description}
  
\item[with arity
  \textnormal{$\langle\textit{Real},\textit{Real}\rangle$}] \mbox{}

  \begin{description}
    
  \item[\textnormal{less-than} New!] --- $e$ : less-than($n$, $m$) iff $n\varepsilon e$, $m\varepsilon e$ and $n<m$

  \end{description}

  


\end{description}

\subsubsection{Scales}

\begin{description}
  
\item[\textnormal{$\zeta_{\text{temp}}$} New!] ---
  $\lambda r$:\textit{AmbTempFrame} . $r$.x : (\textit{AmbTempFrame}
  $\rightarrow$ \textit{Real})

  
\item[\textnormal{$\zeta_{\mathrm{height}}$} New!] ---
  $\lambda r$:\textit{LocFrame} . $r$.loc.z-coord : (\textit{LocFrame}
  $\rightarrow$ \textit{Real})

  
\item[\textnormal{$\zeta_{\text{age}}$} New!] ---  
$\lambda r$:\smallrecord{\smalltfield{x}{\textit{Ind}}\\
                         \smalltfield{age}{\textit{Real}}\\
                         \smalltfield{c$_{\mathrm{age}}$}{age\_of(x,age)}}
                       . $r$.age : (\smallrecord{\smalltfield{x}{\textit{Ind}}\\
                         \smalltfield{age}{\textit{Real}}\\
                         \smalltfield{c$_{\mathrm{age}}$}{age\_of(x,age)}}
                       $\rightarrow$ \textit{Real})

\end{description}



\subsubsection{Lexicon} 
\begin{description}
\item[\textnormal{Lex}] \mbox{}

  If $T_{\mathrm{phon}}$ is a phonological type (that is,
$T_{\mathrm{phon}}\sqsubseteq\textit{Phon}$) and $T_{\mathrm{sign}}$
is a sign type (that is, $T_{\mathrm{sign}}\sqsubseteq\textit{Sign}$), then we shall use
Lex($T_{\mathrm{phon}}$, $T_{\mathrm{sign}}$) to represent
\begin{quote}
(($T_{\mathrm{sign}}$ \d{$\wedge$}
\smallrecord{\smalltfield{s-event}{\smallrecord{\smalltfield{e}{$T_{\mathrm{phon}}$}}}})
\d{$\wedge$} \textit{NoDaughters})
\end{quote}

% \item[\textnormal{SemCommonNoun($p$)}] \mbox{}

%   If $p$ is a predicate with arity $\langle\textit{Ind}\rangle$, then SemCommonNoun($p$) is
%   \begin{quote}
%     $\lambda c$:\textit{Rec} . $\lambda
% r$:\smallrecord{\smalltfield{x}{\textit{Ind}}}
% . \record{\tfield{e}{$p$($r$.x)}}
% \end{quote}

% \item[\textnormal{Lex$_{\mathrm{CommonNoun}}$($T_{\mathrm{phon}}$,
%     $p$)}] \mbox{}

%   If $T_{\mathrm{phon}}$ is a phonological type and $p$ is a
%   predicate with arity $\langle\textit{Ind}\rangle$, then Lex$_{\mathrm{CommonNoun}}$($T_{\mathrm{phon}}$,
%   $p$) is
%   \begin{quote}
%     Lex($T_{\mathrm{phon}}$, \textit{N}) \d{$\wedge$}
%     \smallrecord{\smallmfield{cont}{SemCommonNoun($p$)}{\textit{PPpty}}}
%   \end{quote}

\item[\textnormal{SemCommonNoun($T_{\mathrm{bg}}$, $p$)} Revised!]
  \mbox{}

    If $p$ is a predicate with arity $\langle\textit{Ind}\rangle$ and
    $T_{\mathrm{bg}}$ is a type (of context), then
    SemCommonNoun($T_{\mathrm{bg}}$, $p$) is
  \begin{quote}
    $\ulcorner\lambda c$:$T_{\mathrm{bg}}$ . $\ulcorner\lambda
r$:\smallrecord{\smalltfield{x}{\textit{Ind}}}
. \record{\tfield{e}{$p$($r$.x)}}$\urcorner\urcorner$
\end{quote}

If $p$ is a predicate with arity $\langle\textit{Rec}\rangle$ and $T_{\mathrm{bg}}$ is a type (of context), then
    SemCommonNoun($T_{\mathrm{bg}}$, $p$) is
  \begin{quote}
    $\ulcorner\lambda c$:$T_{\mathrm{bg}}$ . $\ulcorner\lambda
r$:\smallrecord{\smalltfield{x}{\textit{Rec}}}
. \record{\tfield{e}{$p$($r$.x)}}$\urcorner\urcorner$
\end{quote}
  
%     \todo{Not sure this is used}If $p$ is a predicate with arity $\langle\textit{Rec},
%     \textit{Rec}\rangle$ and $T_{\mathrm{bg}}\sqsubseteq$\smallrecord{\smalltfield{$\mathfrak{c}$}{\textit{Rec}}} is a type (of context), then
%     SemCommonNoun($T_{\mathrm{bg}}$, $p$) is
%   \begin{quote}
%     $\lambda c$:$T_{\mathrm{bg}}$ . $\lambda
% r$:\textit{Rec}
% . \record{\tfield{e}{$p$($r$, $c.\mathfrak{c}$)}}
% \end{quote}

\item[\textnormal{Lex$_{\mathrm{CommonNoun}}$($T_{\mathrm{phon}}$,
    $T_{\mathrm{bg}}$, $p$)} Revised!] \mbox{}

    % If $T_{\mathrm{phon}}$ is a phonological type, $p$ is a
  % predicate with arity $\langle\textit{Ind}\rangle$ and
  % $T_{\mathrm{bg}}$ is a type (of context), then
  % Lex$_{\mathrm{CommonNoun}}$($T_{\mathrm{phon}}$, $T_{\mathrm{bg}}$,
  % $p$) is
  % \begin{quote}
  %   Lex($T_{\mathrm{phon}}$, \textit{N}) \d{$\wedge$}
  %   \smallrecord{\smallmfield{cont}{SemCommonNoun($T_{\mathrm{bg}}$,
  %       $p$)}{\textit{PPpty}}}
  % \end{quote}


  If $T_{\mathrm{phon}}$ is a phonological type, $p$ is a
  predicate with arity $\langle\textit{Ind}\rangle$ or $\langle\textit{Rec}\rangle$ and
  $T_{\mathrm{bg}}$ is a type (of context), then
  Lex$_{\mathrm{CommonNoun}}$($T_{\mathrm{phon}}$, $T_{\mathrm{bg}}$,
  $p$) is
  \begin{quote}
    Lex($T_{\mathrm{phon}}$, \textit{N}) \d{$\wedge$}
    \smallrecord{\smallmfield{cont}{SemCommonNoun($T_{\mathrm{bg}}$,
        $p$)}{\textit{PPpty}}}
  \end{quote}
  
\item[\textnormal{SemPropName($T_{\text{phon}}$)}] \mbox{}

  If $T_{\text{phon}}$ is a phonological type, then SemPropName($T_{\text{phon}}$) is
  \begin{quote}
    $\ulcorner\lambda c$: \smallrecord{
      \smalltfield{$\mathfrak{c}$}{\smallrecord{\smalltfield{x}{\textit{Ind}}\\
                         \smalltfield{e}{named(x, $T_{\mathrm{phon}}$)}}}} . $\lambda
                       P$:\textit{Ppty} . $P(c.\mathfrak{c})\urcorner$
  \end{quote}
  
\item[\textnormal{Lex$_{\mathrm{PropName}}$($T_{\mathrm{phon}}$
    )}] \mbox{}

  If $T_{\mathrm{phon}}$ is a phonological type,

  then Lex$_{\mathrm{PropName}}$($T_{\mathrm{phon}}$) is
  \begin{quote}
    Lex($T_{\mathrm{phon}}$, \textit{NP}) \d{$\wedge$}
\smallrecord{\smallmfield{cnt}{SemPropName($T_{\mathrm{phon}}$)}{\textit{PQuant}}}
\end{quote}

\item[\textnormal{SemPron} ] \mbox{}

$\ulcorner\lambda c$:\smallrecord{
    \smalltfield{$\mathfrak{s}$}{\smallrecord{
        \smalltfield{x$_0$}{\textit{Ind}}}}} . $\lambda
  P$:\textit{Ppty}
  . $P$(\smallrecord{\field{x}{$c.\mathfrak{s}$.x$_0$}})$\urcorner$

\item[\textnormal{LexPron($T_{\text{phon}}$)}] \mbox{}

If $T_{\text{phon}}$ is a phonological type, then
LexPron($T_{\text{phon}}$) is
\begin{quote}
Lex($T_{\mathrm{phon}}$, \textit{NP}) \d{$\wedge$}
\smallrecord{\smallmfield{cnt}{SemPron}{\textit{PQuant}}}
\end{quote}

\item[\textnormal{SemNumeral($n$)} New!] \mbox{}

  If $n$ is a real number, then SemNumeral($n$) is
  \begin{quote}
    $\ulcorner\lambda c$:\textit{Rec} . $\lambda P$:\textit{Ppty}
    . $P$(\smallrecord{\field{x}{$n$}})$\urcorner$
  \end{quote}

  
\item[\textnormal{Lex$_{\mathrm{numeral}}$($T_{\mathrm{phon}}$, $n$)}
    New!] \mbox{}

    If $T_{\mathrm{phon}}$ is a phonological type and $n$ is a real
    number, then Lex$_{\mathrm{numeral}}$($T_{\mathrm{phon}}$, $n$) is
    \begin{quote}
      Lex($T_{\mathrm{phon}}$, \textit{NP}) \d{$\wedge$}
      \smallrecord{\smallmfield{cnt}{SemNumeral($n$)}{\textit{PQuant}}}
    \end{quote}
    

  
\item[\textnormal{SemIndefArt} Revised!] \mbox{}

  % $\ulcorner\lambda c$:\textit{Rec} . \\
% \hspace*{1em}$\lambda Q$:\textit{Ppty} . \\
% \hspace*{2em} $\lambda P$:\textit{Ppty}
% . \record{\mfield{restr}{$Q$}{\textit{Ppty}} \\
%           \mfield{scope}{$P$}{\textit{Ppty}} \\
%           \tfield{e}{exist(restr, scope)}}$\urcorner$

  $\lambda Q$:\textit{Ppty} . \\
  \hspace*{1em}$\ulcorner\lambda c$:\textit{Rec} . \\
  \hspace*{2em}$\lambda P$:\textit{Ppty} . \\
  \hspace*{3em}\record{
    \mfield{restr}{$Q$}{\textit{Ppty}}\\
    \mfield{scope}{$P$}{\textit{Ppty}}\\
    \tfield{e}{exist(restr, scope)}}$\urcorner$

        
      \item[\textnormal{Lex$_{\mathrm{IndefArt}}$($T_{\mathrm{Phon}}$)}
        Revised!]
        \mbox{}

        If $T_{\mathrm{Phon}}$ is a phonological type, then
        Lex$_{\mathrm{IndefArt}}$($T_{\mathrm{Phon}}$) is
        \begin{quote}
          Lex($T_{\mathrm{Phon}}$, \textit{Det}) \d{$\wedge$}
          \smallrecord{\smallmfield{cont}{SemIndefArt}{(\textit{Ppty}$\rightarrow$\textit{PQuant})}}
        \end{quote}

        
      \item[\textnormal{SemUniversal} New!] \mbox{}

        $\lambda Q$:\textit{Ppty} . \\
  \hspace*{1em}$\ulcorner\lambda c$:\textit{Rec} . \\
  \hspace*{2em}$\lambda P$:\textit{Ppty} . \\
  \hspace*{3em}\record{
    \mfield{restr}{$Q$}{\textit{Ppty}}\\
    \mfield{scope}{$P$}{\textit{Ppty}}\\
    \tfield{e}{every(restr, scope)}}$\urcorner$

  
\item[\textnormal{Lex$_{\mathrm{Universal}}$($T_{\mathrm{Phon}}$)}
        New!]
        \mbox{}

        If $T_{\mathrm{Phon}}$ is a phonological type, then
        Lex$_{\mathrm{Universal}}$($T_{\mathrm{Phon}}$) is
        \begin{quote}
          Lex($T_{\mathrm{Phon}}$, \textit{Det}) \d{$\wedge$}
          \smallrecord{\smallmfield{cont}{SemUniversal}{(\textit{Ppty}$\rightarrow$\textit{PQuant})}}
        \end{quote}

        \item[\textnormal{SemDefArt} New!] \mbox{}

 %  $\lambda c$:\textit{Rec} . \\
% \hspace*{1em}$\lambda Q$:\textit{Ppty} . \\
% \hspace*{2em} $\lambda P$:\textit{Ppty}
% . \record{\mfield{restr}{$Q$}{\textit{Ppty}} \\
%           \mfield{scope}{$P$}{\textit{Ppty}} \\
%           \tfield{e}{the(restr, scope)}}

          $\lambda Q$:\textit{Ppty} . \\
  \hspace*{1em}$\ulcorner\lambda c$:\smallrecord{
    \smalltfield{$\mathfrak{c}$}{\smallrecord{
        \smalltfield{e}{unique($Q$)}}}} . \\
  \hspace*{2em}$\lambda P$:\textit{Ppty} . \\
  \hspace*{3em}\record{
    \mfield{restr}{$Q\!\restriction\!c.\mathfrak{c}$.e}{\textit{Ppty}}\\
    \mfield{scope}{$P$}{\textit{Ppty}}\\
    \tfield{e}{every(restr, scope)}}$\urcorner$

        
      \item[\textnormal{Lex$_{\mathrm{DefArt}}$($T_{\mathrm{Phon}}$)} New!]
        \mbox{}

        If $T_{\mathrm{Phon}}$ is a phonological type, then
        Lex$_{\mathrm{IndefArt}}$($T_{\mathrm{Phon}}$) is
        \begin{quote}
          Lex($T_{\mathrm{Phon}}$, \textit{Det}) \d{$\wedge$}
          \smallrecord{\smallmfield{cont}{SemDefArt}{(\textit{Ppty}$\rightarrow$\textit{PQuant})}}
        \end{quote}

      \item[\textnormal{SemIntransVerb($T_{\mathrm{bg}}$, $p$)} Revised!]
        \mbox{}

        If $T_{\text{bg}}$ is a record type (for context) and $p$ is a
        predicate with arity $\langle\textit{Ind}\rangle$, then SemIntransVerb($T_{\mathrm{bg}}$, $p$) is
        \begin{quote}
          $\ulcorner\lambda c$:$T_{\mathrm{bg}}$ . $\ulcorner\lambda
          r$:\smallrecord{\smalltfield{x}{\textit{Ind}}}
          . \record{\tfield{e}{$p$($r$.x)}}$\urcorner\urcorner$
        \end{quote}

        If $T_{\text{bg}}\sqsubseteq$\smallrecord{\smalltfield{$\mathfrak{c}$}{\textit{Rec}}} is a record type (for context) and $p$ is a
        predicate with arity $\langle\textit{Rec}, \textit{Rec}\rangle$, then SemIntransVerb($T_{\mathrm{bg}}$, $p$) is
        \begin{quote}
          $\ulcorner\lambda c$:$T_{\mathrm{bg}}$ . $\ulcorner\lambda
          r$:\smallrecord{\smalltfield{x}{\textit{Rec}}}
          . \record{\tfield{e}{$p$($r$.x, $c.\mathfrak{c}$)}}$\urcorner\urcorner$
        \end{quote}

        
      \item[\textnormal{Lex$_{\mathrm{IntransVerb}}$($T_{\mathrm{phon}}$,
          $T_{\mathrm{bg}}$, $p$)} Revised!] \mbox{}

        % If $T_{\mathrm{phon}}$ is a phonological type,
        % $T_{\mathrm{bg}}$ a record type (for context) and $p$ is a
        % predicate with arity $\langle\textit{Ind}\rangle$, then Lex$_{\mathrm{IntransVerb}}$($T_{\mathrm{phon}}$,
        % $T_{\mathrm{bg}}$, $p$) is
        % \begin{quote}
        %   Lex($T_{\mathrm{phon}}$, \textit{V$_i$}) \d{$\wedge$}
        %   \smallrecord{\smallmfield{cnt}{SemIntransVerb($T_{\mathrm{bg}}$, $p$)}{\textit{PPpty}}}
        % \end{quote}

        If $T_{\mathrm{phon}}$ is a phonological type,
        $T_{\mathrm{bg}}\sqsubseteq$\smallrecord{\smalltfield{$\mathfrak{c}$}{\textit{Rec}}} a record type (for context) and $p$ is a
        predicate with arity $\langle\textit{Ind}\rangle$ or $\langle\textit{Rec}, \textit{Rec}\rangle$, then Lex$_{\mathrm{IntransVerb}}$($T_{\mathrm{phon}}$,
        $T_{\mathrm{bg}}$, $p$) is
        \begin{quote}
          Lex($T_{\mathrm{phon}}$, \textit{V$_i$}) \d{$\wedge$}
          \smallrecord{\smallmfield{cnt}{SemIntransVerb($T_{\mathrm{bg}}$, $p$)}{\textit{PPpty}}}
        \end{quote}

        
      \item[\textnormal{SemTransVerb($T_{\mathrm{bg}}, p$)}] \mbox{}

        If $T_{\text{bg}}$ is a record type (for context) and $p$ is a
        predicate with arity $\langle\textit{Ind},\textit{Ind}\rangle$, then SemTransVerb($T_{\mathrm{bg}}$, $p$) is
        \begin{quote}
          $\ulcorner\lambda c$:$T_{\mathrm{bg}}$ . $\lambda
          \mathcal{Q}$:\textit{Quant} . $\lambda
          r_1$:\smallrecord{\smalltfield{x}{\textit{Ind}}} . $\mathcal{Q}(\lambda r_2$:\smallrecord{\smalltfield{x}{\textit{Ind}}}
          . \record{\tfield{e}{$p$($r_1$.x, $r_2$.x)}}$)\urcorner$
        \end{quote}
        
\item[\textnormal{Lex$_{\mathrm{TransVerb}}$($T_{\mathrm{phon}}$,
          $T_{\mathrm{bg}}$, $p$)}] \mbox{}

        If $T_{\mathrm{phon}}$ is a phonological type,
        $T_{\mathrm{bg}}$ a record type (for context) and $p$ is a
        predicate with arity $\langle\textit{Ind},\textit{Ind}\rangle$, then Lex$_{\mathrm{TransVerb}}$($T_{\mathrm{phon}}$,
        $T_{\mathrm{bg}}$, $p$) is
        \begin{quote}
          Lex($T_{\mathrm{phon}}$, \textit{V$_t$}) \d{$\wedge$}
          \smallrecord{\smallmfield{cnt}{SemTransVerb($T_{\mathrm{bg}}$, $p$)}{\textit{PPpty}}}
        \end{quote}

        

        
      \item[\textnormal{SemBe} Revised!] \mbox{}

        % $\lambda c:T_{\text{bg}}$ . \\
%         \hspace*{1em}$\lambda\mathcal{Q}$:\textit{Quant} . \\
% \hspace*{2em} $\ulcorner\lambda r_1$:\smallrecord{\tfield{x}{\textit{Ind}}}
% . \\
% \hspace*{3em} $\mathcal{Q}$($\ulcorner\lambda
% r_2$:\smallrecord{\tfield{x}{\textit{Ind}}}
% . \record{\mfield{x}{$r_2$.x, $r_1$.x}{\textit{Ind}}\\
% \tfield{e}{be(x)}}$\urcorner$)$\urcorner$

        \begin{description}
          
        \item[\textnormal{SemBe$_{\text{ID}}$}] \mbox{}

          $\ulcorner\lambda c$:\smallrecord{
    \smalltfield{$\mathfrak{c}$}{\smallrecord{
        \smalltfield{ty}{\textit{Type}}}}} . \\
        \hspace*{1em}$\lambda\mathcal{Q}$:\textit{Quant} . \\
        \hspace*{2em} $\ulcorner\lambda r_1$:\smallrecord{
          \smalltfield{x}{$c.\mathfrak{c}$.ty}}
. \\
\hspace*{3em} $\mathcal{Q}$($\ulcorner\lambda
r_2$:\smallrecord{
  \smalltfield{x}{$c.\mathfrak{c}$.ty}}
. \record{\mfield{x}{$r_1$.x, $r_2$.x}{$c.\mathfrak{c}$.ty}\\
  \tfield{e}{be(x)}}$\urcorner$)$\urcorner\urcorner$

\item[\textnormal{SemBe$_{\text{scalar}}$}] \mbox{}

  $\ulcorner\lambda c$:\smallrecord{
    \smalltfield{$\mathfrak{c}$}{\smallrecord{
        \smalltfield{ty}{\textit{Type}}\\
      \smalltfield{sc}{(ty$\rightarrow$\textit{Real})}}}} . \\
        \hspace*{1em}$\lambda\mathcal{Q}$:\textit{Quant} . \\
        \hspace*{2em} $\ulcorner\lambda r_1$:\smallrecord{
          \smalltfield{x}{$c.\mathfrak{c}$.ty}}
. \\
\hspace*{3em} $\mathcal{Q}$($\ulcorner\lambda
r_2$:\smallrecord{
  \smalltfield{x}{\textit{Real}}}
. \record{\mfield{x}{$c.\mathfrak{c}$.sc($r_1$.x), $r_2$.x}{\textit{Real}}\\
  \tfield{e}{be(x)}}$\urcorner$)$\urcorner\urcorner$

\end{description}


\item[\textnormal{Lex$_{\mathrm{be}}$($T_{\mathrm{Phon}}$)} Revised!] \mbox{}

  If $T_{\mathrm{Phon}}$ is a phonological type, then
  Lex$_{\mathrm{be}_{\text{ID}}}$($T_{\mathrm{Phon}}$) is
\begin{quote}
  Lex($T_{\mathrm{Phon}}$,
\textit{V}) \d{$\wedge$}
\smallrecord{\smallmfield{cont}{SemBe$_{\text{ID}}$}{(\textit{Quant}$\rightarrow$\textit{Ppty})}}
\end{quote}

If $T_{\mathrm{Phon}}$ is a phonological type, then
  Lex$_{\mathrm{be}_{\text{scalar}}}$($T_{\mathrm{Phon}}$) is
\begin{quote}
  Lex($T_{\mathrm{Phon}}$,
\textit{V}) \d{$\wedge$}
\smallrecord{\smallmfield{cont}{SemBe$_{\text{scalar}}$}{(\textit{Quant}$\rightarrow$\textit{Ppty})}}
\end{quote}

\item[\textnormal{$\mathrm{FrameType}(p)$} New!] \mbox{}

  $\mathrm{FrameType}$ is a partial function on predicates, $p$, with
  arity $\langle\textit{Ind}\rangle$ which can be defined for
  particular agents and particular times, which obeys the constraint:
  \begin{quote}
    $\mathrm{FrameType}(p)\sqsubseteq$ \record{
    \tfield{x}{\textit{Ind}}\\
    \tfield{e}{$p$(x)}}
\end{quote}

\item[\textnormal{$p$\_frame} New!] \mbox{}

  \begin{enumerate} 
 
\item If $p$ is a predicate in the domain of $\mathrm{FrameType}$,
  then $p$\_frame is a predicate with arity $\langle\textit{Rec}\rangle$. 
 
\item $e:p\_\text{frame}(r)$ iff $r:\mathrm{FrameType}(p)$ and $e=r$ 
 
\end{enumerate}

\item[\textnormal{CommonNounIndToFrame} New!] \mbox{}

  If $T_{\mathrm{phon}}$ is a phonological type, $p$ is a predicate with
arity $\langle\textit{Ind}\rangle$ and
$T_\mathrm{bg}$ is a record type (the ``background type'' or
``presupposition'') then \\ 
\mbox{CommonNounIndToFrame(Lex$_{\mathrm{CommonNoun}}$($T_{\mathrm{phon}}$,
  $T_\mathrm{bg}$, $p$))} =
\begin{quote}
  Lex$_{\mathrm{CommonNoun}}$($T_{\mathrm{phon}}$,
  $T_\mathrm{bg}$, $p$\_frame)
\end{quote}

\item[\textnormal{RestrictCommonNoun} New!] \mbox{}

  If $T_{\mathrm{phon}}$ is a phonological type, $p$ is a predicate,
$T_{\mathrm{bg}}$ and $T_{\mathrm{res}}$ are record types and $\Sigma$
is Lex$_{\mathrm{CommonNoun}}$($T_{\mathrm{phon}}$, $T_{\mathrm{bg}}$,
$p$), then RestrictCommonNoun($\Sigma$, $T_{\mathrm{res}}$) is
\begin{quote}
$\Sigma$
\fbox{\d{$\wedge$}} \record{
  \mfield{cont}{$\ulcorner\lambda c\!:\!T_{\mathrm{bg}}\ .\
    \ulcorner\text{SemCommonNoun}(T_{\mathrm{bg}},p)(c)\!\mid_{T_{\mathrm{res}}}\urcorner\urcorner$}{\textit{PPpty}}}
\end{quote}
  
        

  
  
\end{description}


\subsubsection{Constituent structure} 
\begin{description}
\item[\textnormal{RuleDaughters($T_{\text{daughters}}$,
$T_{\text{mother}}$)}] \mbox{}

If $T_{\text{mother}}$ is a sign type and $T_{\text{daughters}}$ is a
type of strings of signs then
\begin{quote}
RuleDaughters($T_{\text{daughters}}$,
$T_{\text{mother}}$)
\end{quote}
is
\begin{quote}
  $\lambda u\! :\! T_{\text{daughters}}$\ . $T_{\text{mother}}$ \d{$\wedge$} \smallrecord{\smalltfield{syn}{\smallrecord{\smallmfield{daughters}{$u$}{$T_{\text{daughters}}$}}}}
\end{quote}

\item[\textnormal{ConcatPhon}] \mbox{}

  $\lambda
u$:\smallrecord{\smalltfield{s-event}{\smallrecord{\smalltfield{e}{\textit{Phon}}}}}$^+$\
. \\
\hspace*{1em}\record{\tfield{s-event}{\record{\mfield{e}{concat$_i$($u[i]$.s-event.e)}{\textit{Phon}}}}}

\item[\textnormal{$T_{\text{mother}}\longrightarrow T_{\text{daughter}_1},\ldots
    T_{\text{daughter}_n}$}] \mbox{}

  If $T_{\text{mother}}$ is a sign type and
  $T_{\text{daughter}_1},\ldots T_{\text{daughter}_n}$ are sign types,
  then
  \begin{quote}
    $T_{\text{mother}}\longrightarrow T_{\text{daughter}_1}\ldots
    T_{\text{daughter}_n}$
  \end{quote}
  represents
  \begin{quote}
RuleDaughters($T_{\text{mother}}$,
${T_{\text{daughter}_1}}^\frown\ldots^\frown T_{\text{daughter}_n}$)\d{\d{$\wedge$}}ConcatPhon 
\end{quote}

\item[\textnormal{$\alpha\text{@}\beta$}] \mbox{}

  If $\alpha$ : \smallrecord{\smalltfield{bg}{\textit{RecType}}\\
                           \smalltfield{fg}{(bg$\rightarrow$($T_1\rightarrow
                             T_2$))}} 
and $\beta$ : \smallrecord{\smalltfield{bg}{\textit{RecType}}\\
                           \smalltfield{fg}{(bg$\rightarrow T_1$)}}
                         then the \textit{combination of $\alpha$ and
    $\beta$  based on functional application}, $\alpha\text{@}\beta$, is
\begin{quote}
  $\ulcorner\lambda c$:$[\alpha.\text{bg}]_{\mathfrak{c}\leadsto\mathfrak{c}.\text{f}}$
      \d{$\wedge$}$\mathrm{incr}_{\mathfrak{s}.\text{x}}([\beta.\text{bg}]_{\mathfrak{c}\leadsto\mathfrak{c}.\text{a}},\alpha.\text{bg})$
      . \\
      \hspace*{2em}
$[\alpha]_{\mathfrak{c}\leadsto\mathfrak{c}.\text{f}}(c)(\mathrm{incr}_{\mathfrak{s}.\text{x}}([\beta.\text{fg}]_{\mathfrak{c}\leadsto\mathfrak{c}.\text{a}},\alpha.\text{bg})(c))\urcorner$
      
\end{quote}


\item[\textnormal{ContForwardApp($T_{\text{arg}}$, $T_{\text{res}}$)}] \mbox{}

  If $T_{\text{arg}}$ and $T_{\text{res}}$ are types, then
  ContForwardApp($T_{\text{arg}}$, $T_{\text{res}}$) is
\begin{quote}
  $\lambda
u$:\smallrecord{\smalltfield{cont}{$(T_{\text{arg}}\rightarrow
    T_{\text{res}})$}}$^{\frown}$\smallrecord{\smalltfield{cont}{$T_{\text{arg}}$}} . \\
\hspace*{2em}\smallrecord{\smallmfield{cont}{$u$[0].cont@$u$[1].cont}{$T_{\text{res}}$}}

\end{quote}

\item[\textnormal{$T_{\text{mother}}\longrightarrow
    T_{\text{daughter}_1}\ T_{\text{daughter}_2}\ \mid\
    T_{\text{daughter}_1}'(T_{\text{daughter}_2}':T_{\text{arg}}):T_{\text{res}}$}]
  \mbox{}

  If $T_{\text{mother}}$, $T_{\text{daughter}_1}$ and
  $T_{\text{daughter}_2}$ are sign types and $T_{\text{arg}}$ and
  $T_{\text{res}}$ are content types, then
  \begin{quote}
    $T_{\text{mother}}\longrightarrow
    T_{\text{daughter}_1}\ T_{\text{daughter}_2}\ \mid\
    T_{\text{daughter}_1}'(T_{\text{daughter}_2}':T_{\text{arg}}):T_{\text{res}}$
  \end{quote}
  is
  \begin{quote}
  $T_{\text{mother}}\longrightarrow
    T_{\text{daughter}_1}\ T_{\text{daughter}_2}$ \d{\d{$\wedge$}}
    ContForwardApp($T_{\text{arg}}$, $T_{\text{res}}$)
  \end{quote}

  
\item[\textnormal{$\alpha\text{@\!@}\beta$} New!] \mbox{}

  If $\alpha$ : ($T_1\rightarrow$ \smallrecord{\smalltfield{bg}{\textit{RecType}}\\
                           \smalltfield{fg}{(bg$\rightarrow T_2$)}}) 
                         and $\beta$ : \smallrecord{\smalltfield{bg}{\smallrecord{
                               \smalltfield{$\mathfrak{c}$}{\textit{RecType}}\\
                               \smalltfield{$\mathfrak{s}$}{\textit{RecType}}}}\\
                           \smalltfield{fg}{(bg$\rightarrow T_1$)}}
                         then the \textit{combination of $\alpha$ and
    $\beta$  based on functional application}, $\alpha\text{@\!@}\beta$, is
  \begin{quote}
    $\ulcorner\lambda c$:\record{
      \tfield{$\mathfrak{c}$}{\record{
          \tfield{s}{$\beta$.bg}\\
          \tfield{f}{$\alpha(\beta(s))$.bg}\\
          \mfield{a}{s.$\mathfrak{c}$}{$\beta$.bg.$\mathfrak{c}$}}}}
      . \\*[\baselineskip]
      \hspace*{10em}$[\alpha]_{\mathfrak{c}\leadsto\mathfrak{c}.\text{f}}([\beta]_{\mathfrak{c}\leadsto\mathfrak{c}.\text{a}}(c))(c)\urcorner$
    \end{quote}
    
\item[\textnormal{ContForwardApp$_{\text{@\!@}}$($T_{\text{arg}}$,
    $T_{\text{res}}$)} New!] \mbox{}

  If $T_{\text{arg}}$ and $T_{\text{res}}$ are types, then
  ContForwardApp$_{\text{@\!@}}$($T_{\text{arg}}$, $T_{\text{res}}$) is
\begin{quote}
  $\lambda
u$:\smallrecord{\smalltfield{cont}{$(T_{\text{arg}}\rightarrow
    T_{\text{res}})$}}$^{\frown}$\smallrecord{\smalltfield{cont}{$T_{\text{arg}}$}} . \\
\hspace*{2em}\smallrecord{\smallmfield{cont}{$u$[0].cont@\!@$u$[1].cont}{$T_{\text{res}}$}}

\end{quote}
  
\item[\textnormal{$T_{\text{mother}}\longrightarrow
    T_{\text{daughter}_1}\ T_{\text{daughter}_2}\ \mid\
    T_{\text{daughter}_1}'(_{\text{@\!@}}T_{\text{daughter}_2}':T_{\text{arg}}):T_{\text{res}}$}
  New!]
  \mbox{}

  If $T_{\text{mother}}$, $T_{\text{daughter}_1}$ and
  $T_{\text{daughter}_2}$ are sign types and $T_{\text{arg}}$ and
  $T_{\text{res}}$ are content types, then
  \begin{quote}
    $T_{\text{mother}}\longrightarrow
    T_{\text{daughter}_1}\ T_{\text{daughter}_2}\ \mid\
    T_{\text{daughter}_1}'(_{\text{@\!@}}T_{\text{daughter}_2}':T_{\text{arg}}):T_{\text{res}}$
  \end{quote}
  is
  \begin{quote}
  $T_{\text{mother}}\longrightarrow
    T_{\text{daughter}_1}\ T_{\text{daughter}_2}$ \d{\d{$\wedge$}}
    ContForwardApp$_{\text{@\!@}}$($T_{\text{arg}}$, $T_{\text{res}}$)
  \end{quote}  

\end{description}


\subsubsection{Action rules} (as in Chapter~\ref{ch:gram})

% % \begin{description}

  




% % \item[\textnormal{\textsc{LexRes}}] \mbox{}

% %   \begin{prooftree}
% %     \hypo{\text{Lex}(T,C) \text{ resource}_A}
% %     \hypo{u:_A T}
% %     \infer[enth]2{:_A(\text{Lex}(T,C)\text{\d{$\wedge$}\smallrecord{\smalltfield{s-event}{\smallrecord{\smallmfield{e}{$u$}{$T$}}}}})}
% %   \end{prooftree}


% %   \end{description}



\subsection{Universal speech act resources} (as in Chapter~\ref{ch:infex})

% \subsubsection{Types}

% \begin{description}
  
% \item[\textnormal{\textit{Assertion}}] --- \record{
%   \tfield{s-event}{\textit{SEvent}}\\
%   \tfield{cont}{\textit{RecType}}\\
%   \tfield{illoc}{assert(s-event, cont)}
% }
% \item[\textnormal{\textit{Query}}] --- \record{
%   \tfield{s-event}{\textit{SEvent}}\\
%   \tfield{cont}{\textit{Question}}\\
%   \tfield{illoc}{query(s-event, cont)}
% }

% \item[\textnormal{\textit{Command}}] --- \record{
%   \tfield{s-event}{\textit{SEvent}}\\
%   \tfield{cont}{\textit{RecType}}\\
%   \tfield{illoc}{command(s-event, cont)}
% }

% \item[\textnormal{\textit{Acknowledgement}}] --- \record{
%   \tfield{s-event}{\textit{SEvent}}\\
%   \tfield{cont}{\textit{RecType}}\\
%   \tfield{illoc}{acknowledge(s-event, cont)}
% }

% \item[\textnormal{\textit{AssertionType}}] --- a basic type

%   $T:\textit{AssertionType}$ iff $T\sqsubseteq \textit{Assertion}$

  
% \item[\textnormal{\textit{QueryType}}] --- a basic type

%   $T:\textit{QueryType}$ iff $T\sqsubseteq \textit{Query}$

  
% \item[\textnormal{\textit{CommandType}}] --- a basic type

%   $T:\textit{CommandType}$ iff $T\sqsubseteq \textit{Command}$


 
% \item[\textnormal{\textit{AcknowledgementType}}] --- a basic type

%   $T:\textit{AcknowledgementType}$ iff $T\sqsubseteq \textit{Acknowledgement}$

% \end{description}

\subsection{Universal dialogue resources} (as in Chapter~\ref{ch:propnames})

% \subsubsection{Types}

% \begin{description}
% \item[\textnormal{\textit{InfoState}}] --- \record{
%     \tfield{private}{
%       \record{
%         \tfield{agenda}{$\mathrm{list}(\textit{RecType})$}}} \\
%     \tfield{shared}{
%       \record{
%         \tfield{latest-utterance}{\textit{Sign}$^*$}\\
%         \tfield{commitments}{\textit{RecType}}}}}
  
% \item[\textnormal{\textit{InitInfoState}}] --- \record{
%     \tfield{private}{
%       \record{
%         \mfield{agenda}{[ ]}{$\mathrm{list}(\textit{RecType})$}}} \\
%     \tfield{shared}{
%       \record{
%         \mfield{latest-utterance}{$\varepsilon$}{\textit{Sign}$^*$}\\
%         \mfield{commitments}{\textit{Rec}}{\textit{RecType}}}}}  

  
% \item[\textnormal{\textit{GameBoard}} New!] --- a basic type

%   $T$ : \textit{GameBoard} iff $T\sqsubseteq$ \textit{InfoState}
  
% \item[\textnormal{\textit{TotalInfoState}} New!] --- \record{\tfield{ltm}{\textit{RecType}} \\
%         \tfield{gb}{(ltm$\rightarrow$\textit{GameBoard})}}


% \end{description}

% \subsubsection{Update functions and action rules}

% %(as in Chapter~\ref{ch:infex})

% \begin{description}
% \item[\textnormal{f$_{\textsc{PlanAckAss}}$}] $\lambda r$:\textit{InfoState} . \\
%   \hspace*{4em}$\lambda u$:\textit{Assertion} . \\
%   \hspace*{6em}
%   \smallrecord{
%     \smalltfield{private}{\smallrecord{
%         \smalltfield{agenda}{\smallrecord{
%             \smalltfield{fst}{\smallrecord{
%                 \smalltfield{s-event}{\textit{SEvent} \d{$\wedge$} \smallrecord{
%                     \smallmfield{sp}{$u$.s-event.au}{\textit{Ind}}\\
%                     \smallmfield{au}{$u$.s-event.sp}{\textit{Ind}}}}\\
%                 \smallmfield{cont}{$u$.cont}{\textit{Cont}}\\
%                 \smalltfield{illoc}{acknowledge(s-event, cont)}}}\\
%             \smallmfield{rst}{$r$.private.agenda}{$\mathrm{list}(\textit{RecType})$}}}}}\\
%     \smalltfield{shared}{\smallrecord{
%         \smallmfield{latest-utterance}{$u$}{\textit{Assertion}}}}}
  
% \item[\textnormal{\textsc{PlanAckAss}}]
%   \begin{prooftree}
%     \hypo{s_{i,A}:_A T_{\mathrm{curr}}}
%     \hypo{T_{\mathrm{curr}}\sqsubseteq\mathrm{domtype}(\text{f}_{\textsc{PlanAckAss}})}
%     \hypo{u^*:_A T_{\mathrm{utt}}}
%     \hypo{T_{\mathrm{utt}}\sqsubseteq\textit{Assertion}}
%     \infer[enth]4{s_{i+1,A}:_A
%       T_{\mathrm{curr}}\text{\fbox{\d{$\wedge$}}}(\text{f}_{\textsc{PlanAccAss}}(s_{i,A})(u^*)\text{\d{$\wedge$}}\text{\smallrecord{\smalltfield{shared}{\smallrecord{
%               \smalltfield{latest-utterance}{$T_{\mathrm{utt}}$}}}}})}
%   \end{prooftree}

  
% \item[\textnormal{f$_{\textsc{IntegAck}}$}]
%   $\lambda r$:\textit{InfoState} . \\
%   \hspace*{2em}$\lambda u$:\textit{Acknowledgement} . \\
%   \hspace*{4em}\smallrecord{
%     \smalltfield{shared}{\smallrecord{
%         \smallmfield{commitments}{\smallrecord{
%             \smalltfield{prev}{$r$.shared.commitments}}\d{$\wedge$}$u$.cont}{\textit{RecType}}\\
%         \smallmfield{latest-utterance}{$u$}{\textit{Acknowledgement}}}}}

  
% \item[\textnormal{\textsc{IntegAck}}]
%    \begin{prooftree}
%     \hypo{s_{i,a}:_A T_{\mathrm{curr}}}
%     \hypo{T_{\mathrm{curr}}\sqsubseteq\mathrm{domtype}(\text{f}_{\mathrm{IntegAck}})}
%     \hypo{u^*:_A T_{\mathrm{utt}}}
%     \hypo{T_{\mathrm{utt}}\sqsubseteq\textit{Acknowledgement}}
%     \infer[enth]4{s_{i+1,A}:_A
%       T_{\mathrm{curr}}\text{\fbox{\d{$\wedge$}}}(\text{f}_{\textsc{IntegAck}}(s_{i,A})(u^*)\text{
%         \d{$\wedge$} \smallrecord{\smalltfield{shared}{\smallrecord{
%              \smalltfield{latest-utterance}{$T_{\mathrm{utt}}$}}}}})} 
%  \end{prooftree}

 
% \item[\textnormal{\textsc{ExecTopAgenda}}]
%   \begin{prooftree}
%     \hypo{s_{i,A}:_A\textit{InfoState}\text{\d{$\wedge$}
%         \smallrecord{\smalltfield{private}{\smallrecord{
%               \smalltfield{agenda}{\smallrecord{
%                   \smalltfield{fst}{\textit{RecType}}\\
%                   \smalltfield{rst}{$\mathrm{list}(\textit{RecType})$}}}}}}}}
%     \infer[enth]1{:_A
%       s_{i,A}.\text{private}.\text{agenda}.\text{fst}!}
%   \end{prooftree}

  
% \item[\textnormal{\textsc{DowndateAgenda}}]
%   \mbox{}

%   \hspace*{-2.5em}
%   \begin{prooftree}
%     \hypo{s_{i,A}:_A T_{\mathrm{curr}}}
%     \hypo{T_{\mathrm{curr}}\sqsubseteq\text{
%         \smallrecord{
%           \smalltfield{private}{\smallrecord{
%               \smalltfield{agenda}{\smallrecord{
%                   \smalltfield{fst}{\textit{RecType}}\\
%                   \smalltfield{rst}{$\mathrm{list}(\textit{RecType})$}}}}}}}}
%     \hypo{u^*:_A s_{i,A}.\text{private}.\text{agenda}.\text{fst}}
%     \infer[enth]3{s_{i+1,A}:_A
%       T_{\mathrm{curr}}\text{\fbox{\d{$\wedge$}}
%         \smallrecord{
%           \smalltfield{private}{\smallrecord{
%               \smallmfield{agenda}{$s_{i,A}$.private.agenda.rst}{$\mathrm{list}(\textit{RecType})$}}}}}}
%   \end{prooftree}
  
% \item[\textnormal{\textsc{AccGB}} New!] See p.~\pageref{ex:AccGBfinal}.

  
% \item[\textnormal{\textsc{AccLTM}} New!] See p.~\pageref{ex:AccLTM}.
  
% \item[\textnormal{\textsc{AccNM}} New!] See p.~\pageref{ex:AccNM}.
  
% \item[Control regime for accommodation New!] \mbox{}

%   \begin{subex} 
 
% \item if there is a labelling, $\eta$, such that
% $s_{i,A}^{\text{tot}}.\text{gb}(s_{i,A}^{\text{tot}}).\text{ltm}.\text{shared}.\text{commitments}\sqsubseteq[u^*.\text{cont}.\text{bg}]_\eta$,
%   then use \text{AccGB} with $\eta$ 
 
% \item else if there is a labelling, $\eta$, such that
%   $s_{i,A}^{\text{tot}}.\text{ltm}\sqsubseteq[u^*.\text{cont}.\text{bg}]_\eta$
%   then use \text{AccLTM} with $\eta$

  
% \item else use \textsc{AccNM}
 
% \end{subex} 

% \end{description}

\subsection{English resources} 

\subsubsection{Types and predicates} 

\begin{description}

  
\item[Basic phonological  types for words] \mbox{}

  \{``Dudamel'', ``is'', ``a'', ``conductor'', ``Beethoven'',
  ``composer'', ``Uchida'', ``pianist'', ``aha'', ``ok'', ``leaves''
  , ``hugs'', ``dog'' \textbf{New!}, ``nine'' \textbf{New!},
  ``ninety'' \textbf{New!}\}

%\item [Witnesses for basic types] \mbox{}

%  \begin{description}
% \item[\textnormal{\textit{Ind}}] --- dudamel, beethoven, uchida :
%   \textit{Ind} \textbf{No longer necessary for interpretation of
%     proper names!}
%   \end{description}

  
\item[Predicates] \mbox{}
  
  \begin{description}
  
  \item[with arity \textnormal{$\langle\textit{Ind}\rangle$}]
    \{conductor, composer, pianist, leave,  dog \textbf{New!},
    passenger \textbf{New!}\}
    
  \item[with arity
    \textnormal{$\langle\textit{Ind},\textit{Ind}\rangle$}] \{hug\}

    \item[with arity
  \textnormal{$\langle\textit{Rec},\textit{Rec}\rangle$}] --- \{rise
  \textbf{New!}\}

      $e$ : rise$(r,c)$ if
  \begin{quote}
    $r$ : \textit{AmbTempFrame},\\
    $c$ : \textit{TempRiseEventCntxt} and \\
    $e$ : \textit{TempRiseEvent}$(c)$ \d{$\wedge$}
    \smallrecord{
      \smalltfield{e}{\smallrecord{
          \smallmfield{t$_0$}{$r$}{\textit{AmbTempFrame}}}}}
  \end{quote}
  \textbf{or} if
  \begin{quote}
    $r$ : \textit{PriceFrame},\\
    $c$ : \textit{PriceRiseEventCntxt} and \\
    $e$ : \textit{PriceRiseEvent}$(c)$ \d{$\wedge$}
    \smallrecord{
      \smalltfield{e}{\smallrecord{
          \smallmfield{t$_0$}{$r$}{\textit{PriceFrame}}}}}
  \end{quote}
  \textbf{or} if
  \begin{quote}
    $r$ : \textit{LocFrame},\\
    $c$ : \textit{LocRiseEventCntxt} and \\
    $e$ : \textit{LocRiseEvent}$(c)$ \d{$\wedge$}
    \smallrecord{
      \smalltfield{e}{\smallrecord{
          \smallmfield{t$_0$}{$r$}{\textit{LocFrame}}}}}
  \end{quote}

  

  
\item[with arity \textnormal{$\langle\textit{Rec}\rangle$}] ---
  \{temperature \textbf{New!}\}
  
$e$ : temperature$(r)$ if 
  \begin{quote}
    $r$ : \textit{AmbTempFrame} and $e=r$
  \end{quote}

  
\item[with arity
  \textnormal{$\langle\textit{Ind},\textit{TravelFrame}\rangle$}] ---
  \{take\_journey \textbf{New!}\}

  $s$ : take\_journey($a$, $e$) iff $s=e$ and $e$.traveller = $a$
 
\end{description}

\item[Frame types] \mbox{}

  \begin{description}

    
  \item[\textnormal{\textit{DogFrame}} New!] ---
    \record{
      \tfield{x}{\textit{Ind}}\\
      \tfield{e}{dog(x)}\\
      \tfield{age}{\textit{Real}}\\
      \tfield{c$_{\mathrm{age}}$}{age\_of(x,age)}}
                    
  \item[\textnormal{\textit{TravelFrame}} New!] ---
    \record{
      \tfield{traveller}{\textit{Ind}}\\
      \tfield{source}{\textit{Loc}}\\
      \tfield{goal}{\textit{Loc}}}

    
  \item[\textnormal{\textit{PassengerFrame}} New!] ---
    \record{\tfield{x}{\textit{Ind}}\\
              \tfield{e}{passenger(x)}\\
              \tfield{journey}{\textit{TravelFrame}}\\
              \tfield{c$_{\mathrm{travel}}$}{take\_journey(x, journey)}}


  \end{description}
  
  

\end{description}

\subsubsection{Grammar} 

\begin{description}



\item[Lexical sign types] \mbox{}

\begin{tabbing}
    \{\=Lex$_{\mathrm{PropName}}$(``Dudamel''), \\
    \> Lex$_{\mathrm{PropName}}$(``Beethoven''), \\
    \> Lex$_{\mathrm{numeral}}$(``nine'', 9) \textbf{New!}, \\
    \> Lex$_{\mathrm{numeral}}$(``ninety'', 90) \textbf{New!}, \\ 
    \> Lex$_{\mathrm{IndefArt}}$(``a''), \\
    \> Lex$_{\mathrm{Universal}}$(``every'') \textbf{New!}, \\
\> Lex$_{\mathrm{DefArt}}$(``the'') \textbf{New!},\\
\> Lex$_{\text{CommonNoun}}$(``composer'', \textit{Rec}, composer) \textbf{Revised!},\\
\> Lex$_{\text{CommonNoun}}$(``conductor'', \textit{Rec}, conductor) \textbf{Revised!}, \\
\> Lex$_{\text{CommonNoun}}$(``dog'', \textit{Rec}, dog) (= $\Sigma_{\text{``dog''}}$)
\textbf{New!}, \\
\> RestrictCommonNoun(CommonNounIndToFrame($\Sigma_{\text{``dog''}}$),
\textit{DogFrame}) \textbf{New!},\\
\> Lex$_{\mathrm{CommonNoun}}$(``passenger'', \textit{Rec}, passenger) ($=\Sigma_{\text{``passenger''}}$)
\textbf{New!}, \\
\> RestrictCommonNoun(CommonNounIndToFrame($\Sigma_{\text{``passenger''}}$),
\textit{PassengerFrame}) \textbf{New!}, \\
\> Lex$_{\text{CommonNoun}}$(``temperature'', \textit{Rec},
temperature) (= $\Sigma_{\text{``temperature''}}$) \textbf{New!}, \\
\> RestrictCommonNoun($\Sigma_{\text{``temperature''}}$,
\textit{AmbTempFrame}) \textbf{New!},\\
\> Lex$_{\mathrm{IntransVerb}}$(``leave'', \textit{Rec}, leave), \\
\> Lex$_{\mathrm{IntransVerb}}$(``run'', \textit{Rec}, run),
\textbf{New!}, \\
\> Lex$_{\mathrm{IntransVerb}}$(``rise'', \smallrecord{\smalltfield{$\mathfrak{c}$}{\textit{TempRiseEventCntxt}}},
rise) \textbf{New!}, \\
\> Lex$_{\mathrm{IntransVerb}}$(``rise'', \smallrecord{\smalltfield{$\mathfrak{c}$}{\textit{PriceRiseEventCntxt}}},
rise) \textbf{New!}, \\
\> Lex$_{\mathrm{IntransVerb}}$(``rise'', \smallrecord{\smalltfield{$\mathfrak{c}$}{\textit{LocRiseEventCntxt}}},
rise) \textbf{New!}, \\
\> Lex$_{\mathrm{TransVerb}}$(``hug'', \textit{Rec}, hug), \\
\> Lex$_{\mathrm{be}_{\text{ID}}}$(``is'') \textbf{Revised!}, \\
\> Lex$_{\mathrm{be}_{\text{scalar}}}$(``is'') \textbf{Revised!}, \\
\> Lex(``ok'', \textit{S}),  \\
\> Lex(``aha'', \textit{S})  \}
\end{tabbing}

\item[Constituent structure rule components] \mbox{} 
  \begin{description}

    \item[\textnormal{CnstrIsA}] \mbox{}

  $\lambda
u$:\textit{V}\d{$\wedge$}\smallrecord{\smalltfield{s-event}{\smallrecord{\smalltfield{e}{``is''}}}}$^{\frown}$\textit{NP}\d{$\wedge$}\smallrecord{\smalltfield{syn}{\smallrecord{\smalltfield{daughters}{\textit{Det}\d{$\wedge$}\smallrecord{\smalltfield{s-event}{\smallrecord{\smalltfield{e}{``a''}}}} \\
                                                                    \hspace*{5em}$^{\frown}$
\textit{N}\d{$\wedge$}\smallrecord{\smalltfield{cont}{\textit{Ppty}}}}
}}}. \\
\hspace*{1em} \textit{VP}\d{$\wedge$}\smallrecord{\smallmfield{cont}{$u$[2].syn.daughters[2].cont}{\textit{Ppty}}} 

  \end{description}
  


\item[Constituent structure rules] \mbox{}
  
  \begin{tabbing}
    \{\=\textit{S} $\longrightarrow$ \textit{NP VP} $\mid$ \textit{NP}$'$(\textit{VP}$'$:\textit{Ppty}):\textit{RecType},\\
    \>\textit{NP} $\longrightarrow$ \textit{Det N} $\mid$
    \textit{Det}$'$($_{\text{@\!@}}$\textit{N}$'$:\textit{Ppty}):\textit{Quant}
    \textbf{Revised!},\\
    \>\textit{VP} $\longrightarrow$ \textit{V} \textit{NP}
    \d{\d{$\wedge$}} CnstrIsA, \\
    \>\textit{VP} $\longrightarrow$ \textit{V} \textit{NP} $\mid$
    \textit{V}$'$(\textit{NP}$'$:\textit{Quant}):\textit{Ppty}\}

  \end{tabbing}

\end{description}
  


\section{Conclusion}
In this chapter we have proposed an analysis of frames as records
which model situations (including events) and we have suggested that
frame types (record types) are important in both the analysis of the
Partee puzzle concerning rising temperatures and prices and in the
analysis of quantification which involves counting events rather than
individuals likes passengers or ships passing through a lock.

Our original inspiration for frames comes from the work of
\cite{Fillmore1982,Fillmore1985} and work on FrameNet
(\url{https://framenet.icsi.berkeley.edu}).  An important aspect of
our approach to frames is that we treat them as first class objects.
That is, they can be arguments to predicates and can be quantified
over.  While this is important, it is not surprising once we decide that
frames are in fact situations (here modelled by records) or situation
types (here modelled by record types).  The distinction between frames
and frame types is not made in the literature deriving from Fillmore's
work but it seems to be an important distinction to draw if we wish to
apply the notion of frame to the kind of examples we have discussed in
this chapter.

The proposal that we have made for solving the Partee puzzle is
closely related to the work of \cite{Lobner2014,Lobnerinprep} whose
inspiration is from the work of
\cite{Barsalou1992a,Barsalou1992,Barsalou1999} rather than Fillmore.
Barsalou's approach embedded in a theory of cognition based on
perception and a conception of cognition as dynamic, that is, a system
in a constant state of flux \citep{PrinzBarsalou2014}, seems much in
agreement with what we are proposing in this book.  Barsalou's
(\citeyear{Barsalou1999}) characterization of basic frame properties
constituting a frame as: ``(1) predicates, (2) attribute-value
bindings, (3) constraints, and (4) recursion'' seem to have a strong
family resemblance with our record types.  Our proposal for
incorporating frames into natural language semantics is, however,
different from L�bner's in that he sees the introduction of a
psychological approach based on frames as a reason to abandon a formal
semantic approach whereas we see type theory as a way of combining the
insights we have gained from model theoretic semantics with a
psychologically oriented approach.  

Our approach to frames has much in common with that of
\cite{KallmeyerOsswald2013} who use feature structures to characterize
their semantic domain.  We have purposely used record types in a way
that makes them correspond both to feature structures and discourse
representation structures which allows us to relate our approach to
more traditional model theoretic semantics at the same time as being
able to merge record types corresponding to unification in
feature-based systems.  However, our record types are included in a
richer system of types including function types facilitates a
treatment of quantification and binding which is not available in a
system which treats feature structures as a semantic
domain.\footnote{It is possible to code up a notation for
  quantification in feature structures but that is not the same as
  giving a semantics for it.}

%Fernando (Muskens)

 

%%% Local Variables: 
%%% mode: latex
%%% TeX-master: "ttl"
%%% End: 
