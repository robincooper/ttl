\chapter{Type theory with records}
\label{app:ttr}

Unless otherwise stated this is the version of TTR presented in 
\cite{Cooper2012}.

\section{Underlying set theory}
\label{app:sets}
In previous statements of this system such as \cite{Cooper2012} we
tacitly assumed a standard underlying set theory such as ZF
(Zermelo-Fraenkel) with urelements (as formulated for example in \citealp{Suppes1960}).  This is what we take to be the
common or garden working set theory which is familiar from the core
literature on formal semantics deriving from Montague's original work
\citep{Montague1974}.  When we introduced complex
objects and types
other than records and record types we were not explicit about exactly which
structured set-theoretic object they represented.  The reason for this
was that, except in the case of records and record types, it did not
seem important exactly how you code structured objects in the set
theory and a detailed exposition would seem to provide another level
of complication over and above an already complicated story.  

In this
version we will take advantage of the freedom provided by an appendix
and spell out a set theoretic coding for all of our structured
objects.  We will use \textit{labelled sets} to model our structured
objects which is what we use for records and record types as well.  We
will assume that our set theory comes equipped with a set of
\textit{urelements} (objects which are not sets but which can be
members of sets) which is partitioned into two countable subsets or
\textit{urelements proper} (intuitively ``real atomic objects'') and
\textit{labels} (intuitively ``objects that are used to label real
objects, either atomic or sets'').  A \textit{labelled set} is a set
of ordered pairs whose first member is a label and whose second
element is either an urelement proper or a set (possibly a labelled
set), such that no more than one ordered pair can contain any
particular label as its first member.  This means that a labelled set
is the traditional set theoretic construction of an extensional
function from a set of labels onto some set.  Suppose that we have a
set

$\{a,b,c,d\}$

and that $\ell_0,\ell_1,\ell_2,\ell_3$ are labels.  Then examples of
labelled sets would be

$\{\langle\ell_0,a\rangle,\langle\ell_1,b\rangle,\langle\ell_2,c\rangle,\langle\ell_3,d\rangle\}$

and

$\{\langle\ell_0,\{\langle\ell_0,a\rangle,\langle\ell_1,b\rangle\}\rangle,\langle\ell_2,c\rangle,\langle\ell_3,d\rangle\}$

Labelled sets where we identify particular distinguished labels will
always give us enough structure to model the structured objects that
we need and define operations on them as required by the type theory.


\section{Basic types}
\label{sec:basic}

A {\it system of basic types\/} is a pair:
\begin{display}
{\bf TYPE$_B$} = $\langle${\bf Type}, $A$$\rangle$
\end{display}
where:
\begin{enumerate} 
 
\item \textbf{Type} is a non-empty set 
 
\item $A$ is a function whose domain is \textbf{Type}

\item for any $T\in\textbf{Type}$, $A(T)$ is a set disjoint from
  \textbf{Type}

\item for any $T\in\textbf{Type}$, $a:_{\mathbf{TYPE_B}}T$ iff $a\in A(T)$
 
\end{enumerate}

A \textit{modal system of basic types}\footnote{This definition was not present in \cite{Cooper2012}.} is a family of
pairs:
\begin{display}
\textbf{TYPE$_{\mathit{MB}}$} = $\langle${\bf Type},
$A\rangle_{A\in\mathcal{A}}$
\end{display}
where:
\begin{enumerate} 
 
\item $\mathcal{A}$ is a set of functions with domain \textbf{Type} 
 
\item for each $A\in\mathcal{A}$, $\langle${\bf Type}, $A\rangle$ is a
  system of basic types
 
\end{enumerate} 

This enables us to define some simple modal notions:

If {\bf TYPE$_{\mathit{MB}}$} = $\langle${\bf Type},
$A\rangle_{A\in\mathcal{A}}$
is a modal system of basic types, we shall use the notation {\bf
  TYPE$_{\mathit{MB}_A}$} (where $A\in\mathcal{A}$) to refer to that
system of basic types in {\bf TYPE$_{\mathit{MB}}$} whose type assignment is
$A$.    Then:  
\begin{enumerate} 
 
\item for any $T_1,T_2\in\textbf{Type}$, $T_1$ \textit{is
    (necessarily) equivalent
    to} $T_2$ \textit{in} {\bf TYPE$_{\mathit{MB}}$},
  $T_1\approx_{\mathbf{TYPE_{\mathit{MB}}}}T_2$,  iff for all
  $A\in\mathcal{A}$, $\{a\mid a:_{\mathbf{TYPE}_{\mathit{MB}_A}}T_1\}=\{a\mid a:_{\mathbf{TYPE}_{\mathit{MB}_A}}T_2\}$
  
 
\item for any $T_1,T_2\in\textbf{Type}$, $T_1$ \textit{is a subtype of} $T_2$ \textit{in} {\bf TYPE$_{\mathit{MB}}$},
  $T_1\sqsubseteq_{\mathbf{TYPE_{\mathit{MB}}}}T_2$,  iff for all
  $A\in\mathcal{A}$, $\{a\mid a:_{\mathbf{TYPE}_{\mathit{MB}_A}}T_1\}\subseteq\{a\mid a:_{\mathbf{TYPE}_{\mathit{MB}_A}}T_2\}$

\item for any $T\in\textbf{Type}$, $T$ \textit{is necessary in} {\bf TYPE$_{\mathit{MB}}$}  iff for all
  $A\in\mathcal{A}$, \\ $\{a\mid a:_{\mathbf{TYPE}_{\mathit{MB}_A}}T\}\not=\emptyset$

\item for any $T\in\textbf{Type}$, $T$ \textit{is possible in} {\bf TYPE$_{\mathit{MB}}$}  iff for some
  $A\in\mathcal{A}$, \\ $\{a\mid a:_{\mathbf{TYPE}_{\mathit{MB}_A}}T\}\not=\emptyset$
 
\end{enumerate}  

\section{Complex types}

\subsection{Predicates}
\label{app:predicates}
We start by introducing the notion of a predicate signature.

A \textit{predicate signature} 
is a triple
\begin{display}
$\langle$\textbf{Pred}, \textbf{ArgIndices}, \textit{Arity}$\rangle$
\end{display}
where:
\begin{enumerate} 
 
\item \textbf{Pred} is a set (of predicates)

\item \textbf{ArgIndices} is a set (of indices for predicate
  arguments, normally types)
 
\item \textit{Arity} is a function with domain \textbf{Pred} and range
  included in the set of finite sequences of members of \textbf{ArgIndices}. 
 
\end{enumerate}

\medskip

A \textit{polymorphic predicate signature} 
is a triple 
\begin{display}
$\langle$\textbf{Pred}, \textbf{ArgIndices}, \textit{Arity}$\rangle$
\end{display}
where:
\begin{enumerate} 
 
\item \textbf{Pred} is a set (of predicates)

\item \textbf{ArgIndices} is a set (of indices for predicate
  arguments, normally types)
 
\item \textit{Arity} is a function with domain \textbf{Pred} and range
  included in the powerset of the set of finite sequences of members
  of \textbf{ArgIndices}. 
 
\end{enumerate}
 

\subsection{Systems of complex types}
\label{app:comptypes}

A {\it system of complex types\/} is a quadruple:
\begin{display}
{\bf TYPE$_C$} = $\langle${\bf Type}, {\bf BType},
$\langle$\textbf{PType}, {\bf Pred}, \textbf{ArgIndices}, {\it Arity\/}$\rangle$, $\langle A,F\rangle$$\rangle$
\end{display}
where:  
\begin{enumerate} 
 
\item $\langle$\textbf{BType}, $A$$\rangle$ is a system of basic types 
 
\item \textbf{BType}$\subseteq$\textbf{Type}

\item for any $T\in\textbf{Type}$, if $a:_{\langle\mathbf{BType},
    A\rangle}T$ then $a:_{\mathbf{TYPE_C}}T$

\item \label{cl:predtypes}$\langle${\bf Pred}, \textbf{ArgIndices},
  {\it Arity\/}$\rangle$ is a (polymorphic) predicate
  signature

\item\hspace*{-1ex}\footnote{This clause has been modified since
    \cite{Cooper2012} where it was a conditional rather than a biconditional.} %\sloppy  
  $P(a_1,\ldots a_n)\in\textbf{PType}$ iff $P\in\textbf{Pred}$, $T_1\in \mathbf{Type},\ldots,T_n\in
  \mathbf{Type}$, \textit{Arity}($P$)=$\langle
  T_1,\ldots,T_n\rangle$  ($\langle
  T_1,\ldots,T_n\rangle$$\in$\textit{Arity}($P$)) and $a_1:_{\mathbf{TYPE_C}}T_1,\ldots,a_n:_{\mathbf{TYPE_C}}T_n$

% If $P\in\textbf{Pred}$, $T_1\in \mathbf{Type},\ldots,T_n\in
%   \mathbf{Type}$, \textit{Arity}($P$)=$\langle
%   T_1,\ldots,T_n\rangle$  ($\langle
%   T_1,\ldots,T_n\rangle$$\in$\textit{Arity}($P$)) and $a_1:_{\mathbf{TYPE_C}}T_1,\ldots,a_n:_{\mathbf{TYPE_C}}T_n$ then
%   $P(a_1,\ldots a_n)\in\textbf{PType}$



\item \textbf{PType}$\subseteq$\textbf{Type}

\item for any $T\in\textbf{PType}$, $F(T)$ is a set disjoint from \textbf{Type}

\item for any $T\in\textbf{PType}$, $a:_{\mathbf{TYPE_C}}T$ iff $a\in F(T)$
 
\end{enumerate}

We call the pair $\langle A,F\rangle$ in a complex system of types the
\textit{model} because of its similarity to first order models in
providing values for the basic types and the ptypes constructed from
predicates and arguments.  It is this pair which connects the system
of types to the non-type theoretical world of objects and situations.

In \cite{Cooper2012} we did not define exactly what object is
represented by $P(a_1,\ldots a_n)$.  Here we will specify it to be the
labelled set

$\{\langle\mathrm{pred},P\rangle,\langle\mathrm{arg}_1,a_1\rangle,\ldots,\langle\mathrm{arg}_n,a_n\rangle\}$

where `pred', `arg$_i$' are reserved labels (not used except as
required here).

What are the objects which belong to these types?  The intuition is
that, for example,

$e$ : run($a$)

means that $e$ is an event or situation where the individual $a$ is
running.  There are two competing intuitions about what $e$ could be.
One is that it is a ``part of the world'', a non-set (urelement).  That is, from
the perspective of set theory and type theory it is an unstructured
atom.  The other intuition we have is that it is a structured object which
contains $a$ as a component and in which a running activity is going
on which involves smaller events such as picking feet up off the
ground, spending certain time in each step cycle with neither foot
touching the ground and so on.  We want to allow for both of these
intuitions.  That is, a witness for a ptype can be a non-set
corresponding to our notion of an event of a certain type.  Or it can
be the kind of labelled set which we call a record.  That is $e$ does
not only belong to the type `run($a$)' but also a record type which
characterizes in more detail the structure of the event.  We will
argue in the text that both intuitions are important and that
observers of the world shift between type theories where certain
ptypes are regarded as types of non-sets and type theories where those
ptypes are types of records.

%\section{Complex types in record types}

\section{Function types}
\label{app:funtypes}

A system of complex types {\bf TYPE$_C$} = $\langle${\bf Type}, {\bf BType},
$\langle$\textbf{PType}, {\bf Pred}, \textbf{ArgIndices}, {\it Arity\/}$\rangle$, $\langle A,F\rangle$$\rangle$ \textit{has function types} if
\begin{enumerate} 
 
\item for any $T_1,T_2 \in \textbf{Type}$, $(T_1\rightarrow T_2) \in \textbf{Type}$ 
 
\item for any $T_1,T_2 \in \textbf{Type}$, $f:_{\mathbf{TYPE_C}}(T_1\rightarrow T_2)$ iff
  $f$ is a function whose domain is $\{a\mid
  a:_{\mathbf{TYPE_C}}T_1\}$ and whose range is included in $\{a\mid a:_{\mathbf{TYPE_C}}T_2\}$ 
 
\end{enumerate}

In \cite{Cooper2012} we did not specify exactly what object is
represented by a function type $(T_1\rightarrow T_2)$.  Here we
specify it to be the labelled set

$\{\langle\mathrm{dmn},T_1\rangle,\langle\mathrm{rng},T_2\rangle\}$

where `dmn' (``domain'') and `rng' (``range'') are reserved labels.

We also introduce a limited kind of polymorphism is function types
which we did not have in \cite{Cooper2012}.


A system of complex types {\bf TYPE$_C$} = $\langle${\bf Type}, {\bf BType},
$\langle$\textbf{PType}, {\bf Pred}, \textbf{ArgIndices}, {\it
  Arity\/}$\rangle$, $\langle A,F\rangle$$\rangle$ \textit{has
  polymorphic function types} if
\begin{enumerate} 
 
\item for any $T_1,T_2 \in \textbf{Type}$,
  $\displaystyle{\bigvee_{T\sqsubseteq T_1}}(T\rightarrow T_2) \in \textbf{Type}$ 
 
\item for any $T_1,T_2 \in \textbf{Type}$,
  $f:_{\mathbf{TYPE_C}}\displaystyle{\bigvee_{T\sqsubseteq
      T_1}}(T\rightarrow T_2)$ iff there is some type $T'$ such that
  $f:(T'\rightarrow T_2)$ and $T'\sqsubseteq T_1$
   
 
\end{enumerate}

We
specify the type $\displaystyle{\bigvee_{T\sqsubseteq T_1}}(T\rightarrow T_2)$   to be the labelled set

$\{\langle\mathrm{polydmn},T_1\rangle,\langle\mathrm{rng},T_2\rangle\}$

where `polydmn' (``polymorphic domain'') and `rng' (``range'') are
reserved labels (`rng' being the same reserved label that was used for
non-polymorphic function types).


In \cite{Cooper2012} we also left it open exactly what kind of object a
function is and assumed there was some theory of functions which would
allow us to characterize them in terms of their domain and range.  In
a classical set theoretic setting where functions are modelled
extensionally as sets of ordered pairs, little more needs to be said.
Ideally, we want a notion of function
that is more like a program or a procedure.  That is, functions can be
intensional in the sense that two distinct functions can correspond to
the same set of ordered pairs. However, it seems that for the purposes
at hand the standard extensional notion of function as a set of
ordered pairs is sufficient and a lot more straightforward to handle
than a more intensional notion given the set-theoretic basis on which
we are building.  There appears to be sufficient intensionality
introduced by our notion of type and the of-type relation.  For this
reason, we will model functions here as sets of ordered pairs in the
classical set-theoretic way.  Ultimately, we suspect that a more
computational and intensional notion of function should be
substituted, but at this point it is unclear what consequences this
might have for the rest of the system.

This choice of modelling functions as sets of ordered pairs means that
$f:_{\mathbf{TYPE_C}}(T_1\rightarrow T_2)$ iff $f\subseteq \{a\mid
  a:_{\mathbf{TYPE_C}}T_1\} \times \{a\mid a:_{\mathbf{TYPE_C}}T_2\}$
  such that if $b \in \{a\mid
  a:_{\mathbf{TYPE_C}}T_1\}$ then there is exactly one $c$, such that
  $\langle b,c\rangle\in f$.  We shall say that in this case the
  result of applying the function $f$ to $b$, in symbols, $f(b)$, is
  $c$.

We introduce a notation for functions which is borrowed from the
$\lambda$-calculus as used by \cite{Montague1973}.  Let $O[v]$ be the
notation for some object of our type theory which uses the variable
$v$ and let $T$ be a type.  Then the function

$\lambda v:T\ .\ O[v]$ 

is to be the function

$\{\langle v,O[v]\rangle\mid v:T\}$

(Here we suppress the subscript $\mathbf{TYPE_C}$ on the `:'.)  For
example, the function

$\lambda v$:\textit{Ind} . run($v$)

is the set of ordered pairs

$\{\langle v,\mathrm{run}(v)\rangle\mid v:\mathit{Ind}\}$

Recall that `run($v$)' is itself a representation for the labelled
set

$\{\langle \mathrm{pred},\mathrm{run}\rangle, \langle
\mathrm{arg}_1,v\rangle\}$

Note that if $f$ is the function $\lambda v$:\textit{Ind} . run($v$)
and $a$:\textit{Ind} then $f(a)$ (the result of applying $f$ to $a$)
is `run($a$)'.  Our definition of function-argument application
guarantees what is called $\beta$-equivalence in the
$\lambda$-calculus.  When we discuss record types as arguments to
functions we will need to introduce one slight complication to our
notion of function application.  We will introduce that complication
when we discuss record types.


\section{List types}
\label{app:listtypes}

List types were not included in \cite{Cooper2012}.

A system of complex types {\bf TYPE$_C$} = $\langle${\bf Type}, {\bf BType},
$\langle$\textbf{PType}, {\bf Pred}, \textbf{ArgIndices}, {\it Arity\/}$\rangle$, $\langle A,F\rangle$$\rangle$ \textit{has list types} if
\begin{enumerate} 
 
\item for any $T \in \textbf{Type}$, $[T] \in \textbf{Type}$ 
 
\item for any $T \in \textbf{Type}$, 
\begin{enumerate} 
 
\item $a\mid L:_{\mathbf{TYPE_C}}[T]$ iff
  $a:_{\mathbf{TYPE_C}}T$ and $L:_{\mathbf{TYPE_C}}[T]$ 
 
\item $[\ ]:_{\mathbf{TYPE_C}}[T]$ 
 
\end{enumerate}

\end{enumerate}

A system of complex types {\bf TYPE$_C$} = $\langle${\bf Type}, {\bf BType},
$\langle$\textbf{PType}, {\bf Pred}, \textbf{ArgIndices}, {\it
  Arity\/}$\rangle$, $\langle A,F\rangle$$\rangle$ \textit{has
  non-empty list types} if
\begin{enumerate} 
 
\item for any $T \in \textbf{Type}$, $_{\mathit{ne}}[T] \in \textbf{Type}$ 
 
\item for any $T \in \textbf{Type}$, 
\begin{enumerate} 
 
\item $a\mid L:_{\mathbf{TYPE_C}} {_{\mathit{ne}}[T]}$ iff
  $a:_{\mathbf{TYPE_C}}T$ and $L:_{\mathbf{TYPE_C}} {_{\mathit{ne}}[T]}$ 
 
\item $[a]:_{\mathbf{TYPE_C}} {_{\mathit{ne}}[T]}$ iff $a:T$
 
\end{enumerate}

\end{enumerate}

If $a\mid L :_{\mathbf{TYPE_C}} {_{\mathit{ne}}[T]}$ for some system of
complex types {\bf TYPE$_C$} and type $T$, then we use fst($a\mid L$)
to refer to $a$ and rst($a\mid L$) to refer to $L$.

In contrast to \cite{Cooper2012} we here make it explcit that $[T]$
represents $\{\langle\mathrm{lst},T\rangle\}$ and
${_{\mathit{ne}}[T]}$ represents $\{\langle\mathrm{nelst},T\rangle\}$
where `lst' and `nelst' are reserved labels. 

Lists are a common data structure used in computer science but they
are not normally defined in basic set theory, although it is
straightforward to define them in terms of sets.  In \cite{Cooper2012}
we did not specify an encoding of lists in terms of sets.  Here we
will use an encoding with labelled sets using the reserved labels
`fst' and `rst' for the first member of the list and the remainder
(``rest'') of
the list respectively.  We let the empty list, [], be the empty set,
$\emptyset$.\footnote{If it is important to distinguish the empty list
  from the empty set we could use an additional reserved label,
  e.g. `lst',  and
  have the empty list be the labelled set $\{\langle
  \mathrm{lst},\emptyset\rangle\}$.}  If $L$ is a list then $a\mid L$
is to be the labelled set $\{\langle\mathrm{fst},a\rangle,\langle\mathrm{rst},L\rangle\}$.

\section{Set types}
\label{app:settypes}

Set types were not included in \cite{Cooper2012}.

A system of complex types {\bf TYPE$_C$} = $\langle${\bf Type}, {\bf BType},
$\langle$\textbf{PType}, {\bf Pred}, \textbf{ArgIndices}, {\it Arity\/}$\rangle$, $\langle A,F\rangle$$\rangle$ \textit{has set types} if
\begin{enumerate} 
 
\item for any $T \in \textbf{Type}$, $\{T\} \in \textbf{Type}$ 
 
\item for any $T \in \textbf{Type}$, 
$A:_{\mathbf{TYPE_C}}\{T\}$ iff  $A$ is a set and 
  for all $a\in A$, $a:_{\mathbf{TYPE_C}}T$
   
 
\end{enumerate}

We let $\{T\}$ represent the labelled set
$\{\langle\mathrm{set},T\rangle\}$ where `set' is a reserved
label.

We also introduce a special kind of set type known as a plurality
type.  The idea here is that a plurality is a set that does not contain any two
objects such that one is a proper part of the other.  The notion of
proper part is characterized by:
\begin{enumerate}
\item If $r_1$ and $r_2$ are records then $r_1$ \textit{is a proper
    part of} $r_2$, $r_1<r_2$, just in case $\varphi(r_1)\subset\varphi(r_2)$. 
 
\item If $o_1$ and $o_2$ are objects of some type and at least one of
  them is not of type \textit{Rec}, then $o_1$ is \textit{not} a
  proper part of $o_2$, $o_1\not<o_2$ 
\end{enumerate}

A system of complex types {\bf TYPE$_C$} = $\langle${\bf Type}, {\bf BType},
$\langle$\textbf{PType}, {\bf Pred}, \textbf{ArgIndices}, {\it Arity\/}$\rangle$, $\langle A,F\rangle$$\rangle$ \textit{has plurality types} if
\begin{enumerate} 
 
\item for any $T \in \textbf{Type}$, $\{\!\mid T\mid\!\} \in \textbf{Type}$ 
 
\item for any $T \in \textbf{Type}$, 
$A:_{\mathbf{TYPE_C}}\{\!\mid T\mid\!\}$ iff  
\begin{enumerate} 
 
\item $A :_{\mathbf{TYPE_C}} \{T\}$ 
 
\item if $a\in A$ then for any $b$ such that $a<b$, $b\not\in A$ 
 
\end{enumerate} 
  
   
 
\end{enumerate}

We let $\{\!\mid T\mid\!\}$ represent the labelled set
$\{\langle\mathrm{plurality},T\rangle\}$ where `plurality' is a reserved
label. 

\section{Singleton types}
\label{app:singletontypes}

Singleton types were not included in the formal definition in \cite{Cooper2012}.

A system of complex types {\bf TYPE$_C$} = $\langle${\bf Type}, {\bf BType},
$\langle$\textbf{PType}, {\bf Pred}, \textbf{ArgIndices}, {\it Arity\/}$\rangle$, $\langle A,F\rangle$$\rangle$ \textit{has singleton types} if
\begin{enumerate} 
 
\item for any $T \in \textbf{Type}$ and $a:_{\mathbf{TYPE_C}}T$, $T_a \in \textbf{Type}$ 
 
\item for any $T \in \textbf{Type}$ and $a:_{\mathbf{TYPE_C}}T$, 
$b:_{\mathbf{TYPE_C}}T_a$ iff  $a=b$
   
 
\end{enumerate}

These clauses are those presented in \cite{Cooper2012}. A more general
version of these clauses seems useful for the uses we wish to make of
singleton types, for example, the restriction of properties discussed
in Appendix~\ref{app:gramrulesuniv}.  The more general version allows
singleton types to be created using an object of any type but will
guarantee that the type is empty if the object is not of the type
being restricted: 
\begin{enumerate} 
 
\item for any $T,T' \in \textbf{Type}$ and $a:_{\mathbf{TYPE_C}}T'$, $T_a \in \textbf{Type}$ 
 
\item for any $T,T' \in \textbf{Type}$ and $a:_{\mathbf{TYPE_C}}T'$, 
$b:_{\mathbf{TYPE_C}}T_a$ iff  $b:_{\mathbf{TYPE_C}}T$ and $a=b$
   
 
\end{enumerate}
As we now allow singleton types that are empty (because the object
used to restrict them is not of the required type) it may seem that
the name ``singleton type'' is a misnomer.  The cases of empty types
are those where we have failed to define a singleton type.

Note that these definitions allow the formation of singleton types
from singleton types.  We sometimes refer to these as
\textit{multiple} singleton types and notate them as $T_{a,b,\ldots}$
rather than the typographically unfortunate ${{T_a}_b}_{\ldots}$.
Following the definition above an object $c$ will be of type $T_{a,b}$
just in case $c:T$ and $a=b=c$.

We let $T_a$ represent the labelled set
$\{\langle\mathrm{singleton},T,a\rangle\}$ where `singleton' is a reserved
label.
 
\section{Join types}
\label{app:jointypes}


A system of complex types {\bf TYPE$_C$} = $\langle${\bf Type}, {\bf BType},
$\langle$\textbf{PType}, {\bf Pred}, \textbf{ArgIndices}, {\it
  Arity\/}$\rangle$, $\langle A,F\rangle$$\rangle$ \textit{has join
  types} if 

\begin{enumerate} 
 
\item for any $T_1,T_2 \in \mathbf{Type}$, $(T_1\vee T_2) \in \mathbf{Type}$ 
 
\item for any $T_1,T_2 \in \mathbf{Type}$, $a:_{\mathbf{TYPE_C}}(T_1\vee T_2)$ iff
  $a:_{\mathbf{TYPE_C}}T_1$ or $a:_{\mathbf{TYPE_C}}T_2$ 
 
\end{enumerate}

Here, but not in \cite{Cooper2012}, we specify that $(T_1\vee T_2)$
represents the labelled set
$\{\langle\mathrm{disj}_1,T_1\rangle,\langle\mathrm{disj_2},T_2\rangle\}$
where `disj$_1$' and `disj$_2$' are reserved labels (``disjunct'').

We add generalized join types which were not present in
\cite{Cooper2012}.  A system of complex types {\bf TYPE$_C$} = $\langle${\bf Type}, {\bf BType},
$\langle$\textbf{PType}, {\bf Pred}, \textbf{ArgIndices}, {\it
  Arity\/}$\rangle$, $\langle A,F\rangle$$\rangle$ \textit{has
  generalized join
  types} if 

\begin{enumerate} 
 
\item for any finite set of types, $\mathbb{T}$, such that $\mathbb{T}
  \subseteq\mathbf{Type}$, $\bigvee\mathbb{T} \in \mathbf{Type}$ 
 
\item for any finite $\mathbb{T} \subseteq\mathbf{Type}$, $a:_{\mathbf{TYPE_C}}\bigvee\mathbb{T}$ iff
  $a:_{\mathbf{TYPE_C}}T$ for some $T\in\mathbb{T}$
 
\end{enumerate}

We specify that $\bigvee\mathbb{T}$
represents the labelled set
$\{\langle\mathrm{disj},\mathbb{T}\rangle\}$
where `disj' is a reserved label (``disjunction'').

\section{Meet types}
\label{app:meettypes}

A system of complex types \textbf{TYPE}$_C$ = $\langle${\bf Type}, {\bf BType},
$\langle$\textbf{PType}, {\bf Pred}, \textbf{ArgIndices}, {\it
  Arity\/}$\rangle$, $\langle A,F\rangle$$\rangle$ \textit{has meet types} if
\begin{enumerate} 
 
\item for any $T_1,T_2 \in \mathbf{Type}$, $(T_1\wedge T_2) \in \mathbf{Type}$ 
 
\item for any $T_1,T_2 \in \mathbf{Type}$, $a:_{\mathbf{TYPE_C}}(T_1\wedge T_2)$ iff
  $a:_{\mathbf{TYPE_C}}T_1$ and $a:_{\mathbf{TYPE_C}}T_2$ 

\end{enumerate}

Here, but not in \cite{Cooper2012}, we specify that $(T_1\wedge T_2)$
represents the labelled set
$\{\langle\mathrm{conj}_1,T_1\rangle,\langle\mathrm{conj_2},T_2\rangle\}$
where `conj$_1$' and `conj$_2$' are reserved labels (``conjunct'').

We add generalized meet types which were not present in
\cite{Cooper2012}.  A system of complex types {\bf TYPE$_C$} = $\langle${\bf Type}, {\bf BType},
$\langle$\textbf{PType}, {\bf Pred}, \textbf{ArgIndices}, {\it
  Arity\/}$\rangle$, $\langle A,F\rangle$$\rangle$ \textit{has
  generalized meet
  types} if 

\begin{enumerate} 
 
\item for any finite set of types, $\mathbb{T}$, such that $\mathbb{T}
  \subseteq\mathbf{Type}$, $\bigwedge\mathbb{T} \in \mathbf{Type}$ 
 
\item for any finite $\mathbb{T} \subseteq\mathbf{Type}$, $a:_{\mathbf{TYPE_C}}\bigwedge\mathbb{T}$ iff
  $a:_{\mathbf{TYPE_C}}T$ for all $T\in\mathbb{T}$
 
\end{enumerate}

We specify that $\bigwedge\mathbb{T}$
represents the labelled set
$\{\langle\mathrm{conj},\mathbb{T}\rangle\}$
where `conj' is a reserved label (``conjunction'').


\section{Models and modal systems of types}
\label{app:modal}



A modal system of complex types
provides a collection of models, $\mathcal{M}$, so that we can talk about properties of the whole collection of
type assignments provided by the various models $M\in\mathcal{M}$.

A \textit{modal system of complex types based on $\mathcal{M}$} is a
family of quadruples\footnote{This definition has been modified since
  \cite{Cooper2012} to make \textbf{PType} and \textbf{Type} be
  relativized to the model $M$.}:
\begin{display}
{\bf TYPE$_{\mathit{MC}}$} = $\langle${\bf Type}$_M$, {\bf BType},
$\langle$\textbf{PType}$_M$, {\bf Pred}, \textbf{ArgIndices}, {\it Arity\/}$\rangle, M\rangle_{M\in\mathcal{M}}$
\end{display}
where for each $M\in \mathcal{M}$, $\langle${\bf Type}$_M$, {\bf BType},
$\langle$\textbf{PType}$_M$, {\bf Pred}, \textbf{ArgIndices}, {\it Arity\/}$\rangle, M\rangle$ is
a system of complex types.

This enables us to define modal notions:

If {\bf TYPE$_{\mathit{MC}}$} = $\langle${\bf Type}$_M$, {\bf BType},
$\langle$\textbf{PType}$_M$, {\bf Pred}, \textbf{ArgIndices}, {\it Arity\/}$\rangle, M\rangle_{M\in\mathcal{M}}$
is a modal system of complex types based
on $\mathcal{M}$, we shall use the notation {\bf
  TYPE$_{\mathit{MC}_M}$} (where $M\in\mathcal{M}$) to refer to that
system of complex types in {\bf TYPE$_{\mathit{MC}}$} whose model is
$M$.  Let \textbf{Type}$_{\mathit{MC}_{\mathit{restr}}}$ be
  $\bigcap\limits_{M\in\mathcal{M}}\!\textbf{Type}_M$, the
  ``restrictive'' set of
  types which occur in all possibilities,  and \textbf{Type}$_{\mathit{MC}_{\mathit{incl}}}$ be
  $\bigcup\limits_{M\in\mathcal{M}}\!\textbf{Type}_M$, the
  ``inclusive'' set of
  types which occur in at least one possibility.  Then we can define
  modal notions either restrictively or inclusively (indicated by the
  subscripts $r$ and $i$ respectively):

\paragraph{restrictive modal notions}  
\begin{enumerate} 
 
\item for any $T_1,T_2\in\textbf{Type}_{\mathit{MC}_{\mathit{restr}}}$, $T_1$ \textit{is
    (necessarily) equivalent$_r$
    to} $T_2$ \textit{in} {\bf TYPE$_{\mathit{MC}}$},
  $T_1\approx_{\mathbf{TYPE_{\mathit{MC}}}}T_2$,  iff for all
  $M\in\mathcal{M}$, $\{a\mid a:_{\mathbf{TYPE}_{\mathit{MC}_M}}T_1\}=\{a\mid a:_{\mathbf{TYPE}_{\mathit{MC}_M}}T_2\}$
  
 
\item for any $T_1,T_2\in\textbf{Type}_{\mathit{MC}_{\mathit{restr}}}$, $T_1$ \textit{is a subtype$_r$ of} $T_2$ \textit{in} {\bf TYPE$_{\mathit{MC}}$},
  $T_1\sqsubseteq_{\mathbf{TYPE_{\mathit{MC}}}}T_2$,  iff for all
  $M\in\mathcal{M}$, $\{a\mid a:_{\mathbf{TYPE}_{\mathit{MC}_M}}T_1\}\subseteq\{a\mid a:_{\mathbf{TYPE}_{\mathit{MC}_M}}T_2\}$

\item for any $T\in\textbf{Type}_{\mathit{MC}_{\mathit{restr}}}$, $T$ \textit{is necessary$_r$ in} {\bf TYPE$_{\mathit{MC}}$}  iff for all
  $M\in\mathcal{M}$, \\ $\{a\mid a:_{\mathbf{TYPE}_{\mathit{MC}_M}}T\}\not=\emptyset$

\item for any $T\in\textbf{Type}_{\mathit{MC}_{\mathit{restr}}}$, $T$ \textit{is possible$_r$ in} {\bf TYPE$_{\mathit{MC}}$}  iff for some
  $M\in\mathcal{M}$, \\ $\{a\mid a:_{\mathbf{TYPE}_{\mathit{MC}_M}}T\}\not=\emptyset$
 
\end{enumerate}

\paragraph{inclusive modal notions}
\begin{enumerate} 
 
\item for any $T_1,T_2\in\textbf{Type}_{\mathit{MC}_{\mathit{incl}}}$, $T_1$ \textit{is
    (necessarily) equivalent$_i$
    to} $T_2$ \textit{in} {\bf TYPE$_{\mathit{MC}}$},
  $T_1\approx_{\mathbf{TYPE_{\mathit{MC}}}}T_2$,  iff for all
  $M\in\mathcal{M}$, if $T_1$ and $T_2$ are members of
  \textbf{Type}$_M$, then $\{a\mid a:_{\mathbf{TYPE}_{\mathit{MC}_M}}T_1\}=\{a\mid a:_{\mathbf{TYPE}_{\mathit{MC}_M}}T_2\}$
  
 
\item for any $T_1,T_2\in\textbf{Type}_{\mathit{MC}_{\mathit{incl}}}$, $T_1$ \textit{is a subtype$_i$ of} $T_2$ \textit{in} {\bf TYPE$_{\mathit{MC}}$},
  $T_1\sqsubseteq_{\mathbf{TYPE_{\mathit{MC}}}}T_2$,  iff for all
  $M\in\mathcal{M}$, if $T_1$ and $T_2$ are members of
  \textbf{Type}$_M$, then $\{a\mid a:_{\mathbf{TYPE}_{\mathit{MC}_M}}T_1\}\subseteq\{a\mid a:_{\mathbf{TYPE}_{\mathit{MC}_M}}T_2\}$

\item for any $T\in\textbf{Type}_{\mathit{MC}_{\mathit{incl}}}$, $T$ \textit{is necessary$_i$ in} {\bf TYPE$_{\mathit{MC}}$}  iff for all
  $M\in\mathcal{M}$, if $T\in$\textbf{Type}$_M$, then \\ $\{a\mid a:_{\mathbf{TYPE}_{\mathit{MC}_M}}T\}\not=\emptyset$

\item for any $T\in\textbf{Type}_{\mathit{MC}_{\mathit{incl}}}$, $T$ \textit{is possible$_i$ in} {\bf TYPE$_{\mathit{MC}}$}  iff for some
  $M\in\mathcal{M}$, if $T\in$\textbf{Type}$_M$, then\\ $\{a\mid a:_{\mathbf{TYPE}_{\mathit{MC}_M}}T\}\not=\emptyset$
 
\end{enumerate}

It is easy to see that if any of the restrictive definitions holds for
given types in a particular system then the corresponding inclusive
definition will also hold for those types in that system.






\section{The type \textit{Type} and stratification}

\label{app:int}

% An intensional type system is one in which the types themselves become
% objects of a type. We introduce a distinguished type \textit{Type} to
% which all the members of the set \textbf{Type} belong.  Things are a
% little more complicated than this, though, since we want \textit{Type}
% itself to be a type and therefore it should belong to the set
% \textbf{Type}.  This would mean that \textit{Type} belongs to itself,
% i.e. \textit{Type}:\textit{Type}.  Allowing types to belong to
% themselves puts us in danger of creating a situation in which
% Russell's paradox arises.  If some members of \textbf{Type} belong to
% themselves then we should be able to talk of the set of types which do
% not belong to themselves,
% $\{T\in\mathbf{Type}\mid T\not\ : T\}$.  Suppose that some model
% assigns this set to $T'$.  Then the question arises whether $T'$
% belongs to itself and we can show that if $T':T'$ then $T'\not\ :T'$
% and if $T'\not\ :T'$ then $T':T'$.  

% In order to avoid this problem we
% will \textit{stratify} (or \textit{ramify}) our type system by introducing types of
% different \textit{orders}.   A type system of order 0 will be a
% system of complex types in the way we have defined it.  The
% set of types, \textbf{Type}$^1$ of a type
% system of order 1 based on this system will contain in addition to
% everything in the original type system a type, \textit{Type}$^1$, to
% which all the types of order 0, members of the set \textbf{Type}$^0$, belong.  In general for all the natural
% numbers $n$, \textit{Type}$^{n+1}$ will be a type to which all the
% types in \textbf{Type}$^n$ belong.  But there may be more additional
% types included in the higher sets of types.  Suppose, for example,
% that we want to introduce a predicate $P$ expressing a relationship
% between individuals and types.  (This will be our basic strategy for
% the treatment of attitude predicates such as \textit{believe} and
% \textit{know}.)  Then \textit{Arity}($P$) might be $\langle
% \mathit{Ind},\mathit{Type}^n\rangle$.  In systems of any order less
% than $n$, $P$ will not be able to be used to construct a type because
% clause~\ref{cl:predtypes} in our definition of systems of complex
% types requires that the types assigned to the arguments be types in
% the system.  However, in systems of order $n$ or greater the required
% type will be present and the predicate will form a type.

% This avoids the risk of running into Russell's paradox but it
% introduces another problem which it is best we deal with straight
% away.  We will illustrate the problem by creating a small example.
% Suppose that we have a system of complex types which includes the type
% \textit{Ind} (``individuals'') to which the objects $a$, $b$ and $c$
% belong.  Suppose further that we have three predicates
% \textit{run},\textit{know} and \textit{believe} and that
% \textit{Arity}(\textit{run})=$\langle\mathit{Ind}\rangle$ and
% \textit{Arity}(\textit{know})=\textit{Arity}(\textit{believe})=$\langle\mathit{Ind},\mathit{Type}^1\rangle$. 
% The set \textbf{Type}$^0$ will contain the types \textit{run}($a$),
% \textit{run}($b$) and \textit{run}($c$) but no types constructed with
% \textit{know} and \textit{believe}.  The set \textbf{Type}$^1$ will
% contain types
% such as \textit{believe}($a$, \textit{run}($a$)) and
% \textit{know}($c$, \textit{run}($b$)) in addition, since \textit{run}($a$),
% \textit{run}($b$) and \textit{run}($c$), being members of
% \textbf{Type}$^0$ will belong to the type \textit{Type}$^1$.  The set
% \textbf{Type}$^2$ will not get any additional types constructed with
% predicates since the arity of the predicates restricts the second
% argument to be of \textit{Type}$^1$.  But suppose we want to express
% that $a$ believes that $b$ knows that $c$ runs, that is we want to
% construct the type \textit{believe}($a$, \textit{know}($b$,
% \textit{run}($c$))).  Perhaps we could solve this by saying that the
% arity of \textit{know} and \textit{believe} is
% $\langle\mathit{Ind},\mathit{Type}^2\rangle$.  But now
% \textbf{Type}$^1$ will not contain any types constructed with these
% predicates and \textbf{Type}$^2$ will again only contain types such as
% \textit{know}($c$, \textit{run}($b$)).

% In order to solve this problem we need to introduce a limited amount
% of \textit{polymorphism} into our arities and assign these predicates
% the arity $\langle\mathit{Ind},\mathit{Type}^n\rangle_{n>0}$ (that is,
% the set of sequences $\langle\mathit{Ind},\mathit{Type}^n\rangle$
% where $n$ is a natural number greater than 0).  Predicates with this
% arity will be able to take arguments of any type \textit{Type}$^n$ where
% $n>0$.  We will say that the predicates \textit{know} and
% \textit{believe} have this arity.  Now it will be the case that
% \textit{run}($c$):\textit{Type}$^1$, \textit{know}($b$,
% \textit{run}($c$)):\textit{Type}$^2$, \textit{believe}($a$, \textit{know}($b$,
% \textit{run}($c$))):\textit{Type}$^3$ and so on.

% \bigskip

An {\it intensional system of complex types\/} is a family of
quadruples indexed by the natural numbers:
\begin{display}
{\bf TYPE$_\mathit{IC}$} = $\langle${\bf Type}$^n$, {\bf BType},
$\langle$\textbf{PType}$^n$, {\bf Pred}, \textbf{ArgIndices}, {\it
  Arity\/}$\rangle$, $\langle A,F^n\rangle$$\rangle_{n\in\mathit{Nat}}$
\end{display}
where (using $\mathbf{TYPE}_{\mathit{IC}_n}$ to refer to the quadruple
indexed by $n$):
\begin{enumerate} 
 
\item for each $n$,$\langle${\bf Type}$^n$, {\bf BType},
$\langle$\textbf{PType}$^n$, {\bf Pred}, \textbf{ArgIndices}, {\it
  Arity\/}$\rangle$, $\langle A,F^n\rangle$$\rangle$ is a 
system of complex types  
 
\item for each $n$, $\mathbf{Type}^n\subseteq\mathbf{Type}^{n+1}$ and
  $\mathbf{PType}^n\subseteq\mathbf{PType}^{n+1}$

\item for each $n$, if $T\in\mathbf{PType}^n$ and $p\in F^n(T)$ then
  $p\in F^{n+1}(T)$

\item for each $n>0$, $\mathit{Type}^n\in\mathbf{Type}^n$

\item for each $n>0$,
  $T:_{\mathbf{TYPE}_{\mathit{IC}_n}}\mathit{Type}^n$ iff $T\in\mathbf{Type}^{n-1}$
 
\end{enumerate}

Here, but not in \cite{Cooper2012}, we make explicit that
\textit{Type} is a distinguished urelement and that \textit{Type}$^n$
represents the labelled set
$\{\langle\mathrm{ord},n\rangle,\langle\mathrm{typ},\mathit{Type}\rangle\}$
where `ord' and `typ' are reserved labels (``order'', ``type'').

\label{pg:dep-fn-types}An intensional system of complex types \textbf{TYPE}$_\mathit{IC}$,
\begin{display}
{\bf TYPE$_\mathit{IC}$} = $\langle${\bf Type}$^n$, {\bf BType},
$\langle$\textbf{PType}$^n$, {\bf Pred}, \textbf{ArgIndices}, {\it
  Arity\/}$\rangle$, $\langle A,F^n\rangle$$\rangle_{n\in\mathit{Nat}}$
\end{display} 
\textit{has dependent function types} if
\begin{enumerate} 
 
\item for any $n>0$, $T \in \mathbf{Type}^n$ and
  $\mathcal{F}:_{\mathbf{TYPE}_{\mathit{IC}_n}}(T\rightarrow\mathit{Type^n})$,
  $((a:T)\rightarrow \mathcal{F}(a)) \in \mathbf{Type}^n$ 
 
\item for each $n>0$,
$f:_{\mathbf{TYPE}_{\mathit{IC}_n}}((a:T)\rightarrow \mathcal{F}(a))$
iff $f$ is a function whose domain is $\{a\mid
a:_{\mathbf{TYPE}_{\mathit{IC}_n}}T\}$ and such that for any $a$ in the
domain of $f$, $f(a):_{\mathbf{TYPE}_{\mathit{IC}_n}}\mathcal{F}(a)$.
   
 
\end{enumerate} 
We might say that on this view dependent function types are
``semi-intensional'' in that they depend on there being a type of
types for their definition but they do not introduce types as
arguments to predicates and do not involve the definition of orders of
types in terms of the types of the next lower order.

Here, in contrast to \cite{Cooper2012}, we make explicit that
$((a:T)\rightarrow \mathcal{F}(a))$ represents the labelled set
$\{\langle\mathrm{dmn},T\rangle,\langle\mathrm{deprng},\mathcal{F}\rangle\}$
where `dmn' as before for function types is a reserved label
corresponding to ``domain'' and `deprng' is a reserved label
corresponding to ``dependent range''.

Putting the definition of a modal type system and an intensional type
system together we obtain:\footnote{This explicit definition was not
  present in \cite{Cooper2012}.}

An {\it intensional modal system of complex types based on $\mathfrak{M}$\/} is a family,
indexed by the natural numbers, of families of quadruples indexed by
members of $\mathfrak{M}$:
\begin{display}
{\bf TYPE$_\mathit{IMC}$} = $\langle${\bf Type}$^n$, {\bf BType},
$\langle$\textbf{PType}$^n$, {\bf Pred}, \textbf{ArgIndices}, {\it
  Arity\/}$\rangle$, $\mathscr{M}_n$$\rangle_{\mathscr{M}\in\mathfrak{M},n\in\mathit{Nat}}$
\end{display}
where:
\begin{enumerate} 
 
\item for each $n$, $\langle${\bf Type}$^n$, {\bf BType},
$\langle$\textbf{PType}$^n$, {\bf Pred}, \textbf{ArgIndices}, {\it
  Arity\/}$\rangle$, $\mathscr{M}_n$$\rangle_{\mathscr{M}\in\mathfrak{M}}$ is a modal
  system of complex types based on $\{\mathscr{M}_n\mid \mathscr{M}\in\mathfrak{M}\}$ 
 
\item for each $\mathscr{M}\in\mathfrak{M}$, $\langle${\bf Type}$^n$, {\bf BType},
$\langle$\textbf{PType}$^n$, {\bf Pred}, \textbf{ArgIndices}, {\it
  Arity\/}$\rangle$, $\mathscr{M}_n$$\rangle_{n\in\mathit{Nat}}$ is an
  intensional system of complex types
 
\end{enumerate} 
 


\section{Records and Record types}
\label{app:rectypes}

In this section we will define what it means for a system of complex
types to have record types.  The objects of record types, that is,
records, are themselves structured mathematical objects of a
particular kind and we will
start by characterizing them.

\subsection{Records}

A {\it record\/} is a finite set of ordered pairs (called {\it fields\/})
which is the graph of a function.  If $r$ is a record and $\langle\ell,v\rangle$
is a field in $r$ we call $\ell$ a {\it label\/} and $v$ a {\it value\/} in $r$
and we use $r.\ell$ to denote $v$.  $r.\ell$ is called a \textit{path}
in $r$.  This means that a record is a labelled set as introduced in Appendix~\ref{app:sets}.

We will use a tabular format to
represent records.  A record 
$\{\langle\ell_1,v_1\rangle,\ldots,\langle\ell_n,v_n\rangle\}$ is displayed as
\begin{display}
\record{\field{$\ell_1$}{$v_1$} \\
        \ldots \\
        \field{$\ell_n$}{$v_n$}}
\end{display}

A value may itself be a record and paths may extend into embedded
records.  A record which contains records as values is called a {\it
  complex record\/} and otherwise a record is {\it simple\/}.  Values which are not records are called {\it
  leaves\/}.  Consider a record $r$ \label{pg:record}
\begin{display}
\record{\field{$f$}{\record{\field{$f$}{\record{\field{$\mathit{ff}$}{$a$}\\
                                                \field{$\mathit{gg}$}{$b$}}}\\
                            \field{$g$}{$c$}}}\\
        \field{$g$}{\record{\field{$h$}{\record{\field{$g$}{$a$}\\
                                                \field{$h$}{$d$}}}}}}
\end{display}
Among the paths in $r$ are $r.f$, $r.g.h$ and $r.f.f.\mathit{ff}$ which denote,
respectively, 
\begin{display}
\record{\field{$f$}{\record{\field{$\mathit{ff}$}{$a$}\\
                                                \field{$\mathit{gg}$}{$b$}}}\\
                            \field{$g$}{$c$}}
\end{display}
\begin{display} 
\record{\field{$g$}{$a$}\\
                                                \field{$h$}{$d$}}
\end{display}
and $a$.  We will make a distinction between \textit{absolute paths},
such as those we have already mentioned, which consist of a record
followed by a series of labels connected by dots and \textit{relative
  paths} which are just a series of labels connected by dots,
e.g. $g.h$.  Relative paths are useful when we wish to refer to
similar paths in different records.  We will use \textit{path} to
refer to either absolute or relative paths when it is clear from the
context which is meant.  The set of leaves of $r$, also known as its {\it
  extension\/} (those objects other than labels which it contains), is
$\{a,b,c,d\}$.  The bag (or multiset) of leaves of $r$, also known as
its {\it multiset extension\/}, is $\{a,a,b,c,d\}$.  A record may be
regarded as a way of labelling and structuring its extension.

An object, $a$, is a \textit{component} of a record, $r$, in symbols,
$a\varepsilon r$, just in case there is some path, $\pi$, in $r$ such
that $r.\pi=a$.  Thus the record, $r$, above has the following
components: $r.f$, $r.f.f$, $r.f.f.\mathit{ff}$,
$r.f.f.\mathit{gg}$, $r.f.g$, $r.g$, $r.g.h$, $r.g.h.g$ and
$r.g.h.h$. An object, $a$, is \textit{present} in a record, $r$, in
symbols, $a\underline{\varepsilon} r$, just in case either $a=r$ or
$a\varepsilon r$.

Two
records are {\it (multiset) extensionally equivalent\/} if they have
the same (multiset) extension.  Two important, though trivial, facts
  about records are:
\begin{quote}
{\it Flattening.\/} \label{pg:flattening} For any record $r$, there is a multiset
extensionally equivalent simple record.  We can define an operation of flattening on records which
will always produce an equivalent simple record.  In the case of our
example, the result of flattening is
\begin{display}
\record{\field{$f.f.\mathit{ff}$}{$a$} \\
        \field{$f.f.\mathit{gg}$}{$b$} \\
        \field{$f.g$}{$c$} \\
        \field{$g.h.g$}{$a$} \\
        \field{$g.h.h$}{$d$}}
\end{display}
assuming the flattening operation uses paths from the original record
in a rather obvious way to create unique labels for the new record.

{\it Relabelling.\/}  \label{pg:relabelling}For any record $r$, if $\pi_1.\ell.\pi_2$ is a
path $\pi$ in $r$, and $\pi_1.\ell'.\pi{_2}'$ is {\it not\/} a path in
$r$ (for any $\pi{_2}'$),
then substituting $\ell'$ for the occurrence of $\ell$ in $\pi$ results in
a record which is multiset equivalent to $r$.  We could, for example,
substitute $k$ for the second occurrence of $g$ in the path $g.h.g$ in our example record.

\record{\field{$f$}{\record{\field{$f$}{\record{\field{$\mathit{ff}$}{$a$}\\
                                                \field{$\mathit{gg}$}{$b$}}}\\
                            \field{$g$}{$c$}}}\\
        \field{$g$}{\record{\field{$h$}{\record{\field{$k$}{$a$}\\
                                                \field{$h$}{$d$}}}}}}

\end{quote}

We will first look in more detail at flattening.  Flattened records
have \textit{complex labels} (or \textit{paths}) of the form
$\ell_1.\ell_2.\ldots.\ell_n$.  If $L$ is a set of labels, the set of
\textit{complex labels}, $L_\pi$, \textit{based on} $L$ is the minimal
set such that:
\begin{enumerate} 
 
\item if $\ell\in L$, then $\ell\in L_\pi$ 
 
\item if $\pi\in L_\pi$ and $\ell\in L$, then $\pi.\ell\in L_\pi$ 
 
\end{enumerate}

Recall that a record, $r$, is a set of fields, $f$, where each $f$, is
an ordered pair, $\langle\ell,v\rangle$ where $\ell$ is a (complex)
label and $v$ is the value of the field which may itself be a record
or some other object.  If $r$ is a record, then $\varphi(r)$,
\textit{the flattening of} $r$, is
\begin{quote}
$\displaystyle{\bigcup_{f\in r}}\mathrm{FlattenField}(f)$
\end{quote}
where $\mathrm{FlattenField}(\langle\ell,v\rangle)$ is
\begin{quote}
$\varphi(\{\langle\ell.\ell_1,v_1\rangle,
           \langle\ell.\ell_2,v_2\rangle,
           \ldots,
           \langle\ell.\ell_n,v_n\rangle\})$
\end{quote}
if $v$ is the record
$\{\langle\ell_1,v_1\rangle,
   \langle\ell_2,v_2\rangle,
   \ldots,
   \langle\ell_n,v_n\rangle\}$
and
$\langle\ell,v\rangle$
otherwise.

We can show that if $r$ is a record then $\varphi(r)$ is a record.  It
follows trivially from the requirement that records are graphs of
functions from labels to values.  If $r$ is a record
$\{\langle\ell_1,v_1\rangle,\ldots,\langle\ell_n,v_n\rangle\}$, then
the \textit{set of labels of} $r$, $\mathrm{labels}(r)$, is
$\{\ell_1,\ldots,\ell_n\}$.  For example, if $r$ is the record
\begin{quote}
\record{\field{x}{\record{\field{z}{$a$}\\
                          \field{w}{$b$}}}\\
        \field{y}{$c$}}
\end{quote}
then $\mathrm{labels}(r)$ is $\{\mathrm{x},\mathrm{y}\}$. That is, `labels' picks out
the ``highest level'' of labels in a complex record.
We use $\mathrm{tpaths}(r)$ to represent the \textit{set of total
  paths} of $r$, $\mathrm{labels}(\varphi(r))$.

If $L$ is a set of labels,
then the set of \textit{complex labels} (or \textit{paths}), $L_+$,
based on $L$ is the minimal set such that
\begin{enumerate} 
 
\item if $\ell\in L$, then $\ell\in L_+$ 
 
\item if $\pi\in L_+$ and $\ell\in L$, then $\pi.\ell\in L_+$ 
 
\end{enumerate}

If $r$ is a record such that $\mathrm{labels}(r)\subseteq L_+$, then
the \textit{inverse flattening} (or \textit{expansion}) \textit{of}
$r$, $\varphi^-(r)$, is the minimal set such that
\begin{enumerate} 
 
\item if $\langle\ell,v\rangle\in r$ and $\ell\in L$, then $\langle\ell,v\rangle\in\varphi^-(r)$ 
 
\item if $\ell\in L$, $\pi\in L_+$ and $\langle\ell.\pi,v\rangle\in
  r$, then 
$\langle\ell,\varphi^-(\{\langle\pi',v'\rangle\mid\langle\ell.\pi',v'\rangle\in r\})\rangle\in\varphi^-(r)$
 
\end{enumerate}

\paragraph{Claim:} If $r$ is a record, then $\varphi^-(\varphi(r))=r$.

The \textit{set of total paths of} $r$, $\mathrm{tpaths}(r)$, is
$\mathrm{labels}(\varphi(r))$.

If $r$ is a record and $\mathrm{tpaths}(r)=\Pi$ then the \textit{set
  of paths of} $r$, $\mathrm{paths}(r)$, is the minimal set such that
\begin{enumerate} 
 
\item If $\pi\in\Pi$, then $\pi\in\mathrm{paths(r)}$ 
 
\item If $\ell$ is a label (in $L$) and
  $\pi.\ell\in\mathrm{paths}(r)$, then $\pi\in\mathrm{paths}(r)$ 
 
\end{enumerate} 

If $\Pi$ is a set of paths (in $L_+$), then the \textit{set of longest
  paths in} $\Pi$, $\mathrm{longest}(\Pi)$ is such that
$\pi\in\mathrm{longest}(\Pi)$ iff $\pi\in\Pi$ and there is no $\pi'$
such that $\pi.\pi'\in\Pi$.

\paragraph{Claim:}
$\mathrm{longest}(\mathrm{paths}(r))=\mathrm{tpaths}(r)$

If $R$ is a set of records, then the \textit{set of longest common
  paths of} $R$, $\mathrm{lcpaths}(R)$, is
\begin{quote}
$\mathrm{longest}(\displaystyle{\bigcap_{r\in R}}\mathrm{paths}(r))$
\end{quote} 

Consider the records
\begin{quote}
$r_1$ = \record{\field{x}{\record{\field{x}{$a$}\\
                                  \field{y}{$b$}}}\\
                \field{y}{\record{\field{x}{$c$}\\
                                  \field{y}{$d$}}}}

$r_2$ = \record{\field{x}{\record{\field{z}{$e$}\\
                                  \field{w}{$f$}}}\\
                \field{y}{\record{\field{x}{$g$}\\
                                  \field{y}{$h$}}}}
\end{quote}
The following hold true:
\begin{quote}
$\mathrm{paths}(r_1) = \{\mathrm{x},\mathrm{x.x},\mathrm{x.y},\mathrm{y},\mathrm{y.x},\mathrm{y.y}\}$\\
$\mathrm{paths}(r_2) = \{\mathrm{x},\mathrm{x.z},\mathrm{x.w},\mathrm{y},\mathrm{y.x},\mathrm{y.y}\}$\\
$\mathrm{lcpaths}(\{r_1,r_2\}) = \{\mathrm{x},\mathrm{y.x},\mathrm{y.y}\}$
\end{quote} 
The notion of longest common paths for a set of records will be
important when we discuss paths in record types below.  [Is this still true?] 
  

% If $r$ is the record 
% \begin{quote}
% $\left\{\begin{array}{l}
%           \langle\ell_1.\pi_{1,1},v_{1,1}\rangle,
%           \langle\ell_1.\pi_{1,2},v_{1,2}\rangle,
%           \ldots,
%           \langle\ell_1.\pi_{1,m_1},v_{1,m_1}\rangle,\\
%           \langle\ell_2.\pi_{2,1},v_{2,1}\rangle,
%           \langle\ell_2.\pi_{2,2},v_{2,2}\rangle,
%           \ldots,
%           \langle\ell_2.\pi_{2,m_2},v_{2,m_2}\rangle,\\
%           \vdots\\
%           \langle\ell_n.\pi_{n,1},v_{n,1}\rangle,
%           \langle\ell_n.\pi_{n,1},v_{n,1}\rangle,
%           \ldots,
%           \langle\ell_n.\pi_{n,m_n},v_{n,m_n}\rangle
%           \end{array}
% \right\}$
% \end{quote}
% then $\varphi^-(r)$, the \textit{inverse
%   flattening} (or \textit{expansion}) \textit{of} $r$, is
% \begin{quote}
% $\left\{\begin{array}{l}
%           \langle\ell_1,\varphi^-(\{\langle\pi_{1,1},v_{1,1}\rangle,
%                                     \langle\pi_{1,2},v_{1,2}\rangle,
%                                     \ldots,
%                                     \langle\pi_{1,m_1},v_{1,m_1}\rangle\})
%                  \rangle,\\
%           \langle\ell_2,\varphi^-(\{\langle\pi_{2,1},v_{2,1}\rangle,
%                                     \langle\pi_{2,2},v_{2,2}\rangle,
%                                     \ldots,
%                                     \langle\pi_{2,m_2},v_{2,m_2}\rangle\})
%                  \rangle,\\
%           \vdots \\
%           \langle\ell_n,\varphi^-(\{\langle\pi_{n,1},v_{n,1}\rangle,
%                                     \langle\pi_{n,2},v_{n,2}\rangle,
%                                     \ldots,
%                                     \langle\pi_{n,m_n},v_{n,m_n}\rangle\})
%                  \rangle
%         \end{array}\right\}$
% \end{quote}

We now turn our attention to relabelling.   
An $L$\textit{-relabelling} is a one-one function whose domain and range
are included in $L_+$. We will standardly represent a relabelling,
$\eta$, whose domain is $\{\pi_1,\ldots,\pi_n\}$ and which is defined
by $\eta(\pi_1)=\pi'_1,\ldots,\eta(\pi_n)=\pi'_n$ as:
\begin{quote}
$\pi_1\leadsto\pi'_1$\\
\vdots \\
$\pi_n\leadsto\pi'_n$
\end{quote} 



If $r$ is a record
$\{\langle\ell_1,v_1\rangle,\ldots,\langle\ell_n,v_n\rangle\}$ and
$\eta$ is a relabelling, the \textit{application of $\eta$ to $r$},
$\eta*r$ is
$\{\langle\ell'_1,v_1\rangle,\ldots,\langle\ell'_n,v_n\rangle\}$ where
for each $i$ such that $1\leq i\leq n$, $\ell'_i$ is $\eta(\ell_i)$ if
$\ell_i$ is in the domain of $\eta$ and $\ell_i$ otherwise.  Suppose
$\eta$ is the relabelling $\mathrm{x}\leadsto\mathrm{xx},
\mathrm{z}\leadsto\mathrm{zz}$ and $r$ is as above.  Then $\eta*r$ is
\begin{quote}
\record{\field{xx}{\record{\field{z}{$a$}\\
                          \field{w}{$b$}}}\\
        \field{y}{$c$}}
\end{quote}
That is, the label `x' is changed, but nothing else.

We say that a relabelling, $\eta$, is \textit{limited} to a record,
$r$, just in case $\mathrm{dom}(\eta)\subseteq\mathrm{labels}(r)$.\label{pg:limited}

We are, however, mainly interested in replacing total paths in a
record which is why we have defined relabellings on the set of paths,
$L_+$ rather than the set of labels, $L$.  The \textit{result of
  relabelling a record, $r$, with a relabelling, $\eta$}, in symbols,
$[r]_\eta$, is $\varphi^-(\eta*\varphi(r))$.  We say that $\eta$ is a
\textit{proper relabelling}\label{pg:proper-relabelling} for $r$ just in case $[r]_\eta$ is a
record.

Suppose $r$ is the record
\begin{quote}
\record{\field{x}{\record{\field{x}{$a$}\\
                          \field{y}{$b$}}}\\
        \field{y}{\record{\field{x}{$c$}}}}
\end{quote}
and $\eta$ is the relabelling
\begin{quote}
x.x $\leadsto$ y.z\\
x.y $\leadsto$ x\\
\end{quote}
$\varphi(r)$ is the flattened record
\begin{quote}
\record{\field{x.x}{$a$}\\
        \field{x.y}{$b$}\\
        \field{y.x}{$c$}}
\end{quote}
$\eta*\varphi(r)$ is
\begin{quote}
\record{\field{y.z}{$a$}\\
        \field{x}{$b$}\\
        \field{y.x}{$c$}}
\end{quote}
Then $[r]_\eta$ is the expansion of this record, that is, the record
\begin{quote}
\record{\field{x}{$b$}\\
        \field{y}{\record{\field{x}{$c$}\\
                          \field{z}{$a$}}}}
\end{quote}
and $\eta$ is a proper relabelling for $r$.  

Now let us consider a case where the relabelling is not proper.
Suppose we start with the same record, $r$, as above but take a new
relabelling, $\eta'$:
\begin{quote}
x.x $\leadsto$ x
\end{quote}
Now $\eta'*\varphi(r)$ is the record
\begin{quote}
\record{\field{x}{$a$}\\
        \field{x.y}{$b$}\\
        \field{y.x}{$c$}}
\end{quote}
However, the expansion $\varphi^-(\eta'*\varphi(r))$ is the set of
ordered pairs:
\begin{quote}
$\{\langle\mathrm{x},a\rangle,
   \langle\mathrm{x},\{\langle\mathrm{y},b\rangle\}\rangle,
   \langle\mathrm{y},\{\langle\mathrm{x},c\rangle\}\rangle\}$
\end{quote}
which is not a record since it is not the graph of a function:  it has
two ordered pairs whose first member is `x'.  We can read the problem
already from the flattened record since it has a label, `x', which is a
proper initial segment of another of its labels, `x.y'.  Therefore
$\eta'$ is not a proper relabelling for $r$.  Note that we cannot read
the improperness off $\eta'$ alone.  $\eta'$ is a proper relabelling
for the record $r'$:
\begin{quote}
\record{\field{x}{\record{\field{x}{$a$}}}\\
        \field{y}{\record{\field{x}{$c$}}}}
\end{quote}
$[r']_{\eta'}$ is the record:
\begin{quote}
\record{\field{x}{$a$}\\
        \field{y}{\record{\field{x}{$c$}}}}
\end{quote}



We use $\eta^-$ to represent the inverse of the labelling $\eta$.
That is, for any $\pi$, $\pi'$, $\pi\leadsto\pi'$ is in $\eta$ just in
case $\pi'\leadsto\pi$ is in $\eta^-$. 
\paragraph{Claim:} If
$\eta$ is a proper relabelling for $r$ limited to $\varphi(r)$, then $\eta^-$ is a proper
relabelling for $[r]_\eta$ and $[[r]_\eta]_{\eta^-}=r$.  

The
requirement that $\eta$ is limited to $\varphi(r)$ is important here
since it prevents cases where $\eta^-$ would change labelling not
changed by $\eta$.  Consider an example where $\eta$ is 
\begin{quote}
x $\leadsto$ z\\
x.x $\leadsto$ y
\end{quote}
and $r$ is
\begin{quote}
\record{\field{x}{\record{\field{x}{$a$}}}\\
        \field{z}{$b$}}
\end{quote}
Note that in this case $\eta$ is \textit{not} limited to $\varphi(r)$ whose
labels are `x.x' and `z' but do not include `x'.  $\eta$ is
proper for $r$ and $[r]_\eta$ is
\begin{quote}
\record{\field{y}{$a$}\\
        \field{z}{$b$}}
\end{quote}
Now consider $\eta^-$ which is
\begin{quote}
z $\leadsto$ x\\
y $\leadsto$ x.x
\end{quote}
The relabelling of $[r]_\eta$ by $\eta^-$ will be the set of ordered
pairs
\begin{quote}
$\{\langle\mathrm{x},b\rangle,\langle\mathrm{x},\{\langle\mathrm{x},a\rangle\}\rangle\}$
\end{quote}
which is not a record (since `x' occurs twice as a label).  Hence in
this case $\eta^-$ is not a proper relabelling for $[r]_\eta$ and
$[[r]_\eta]_{\eta^-}$ is not identical with $r$.  The reason for this
is that $\eta^-$ relabels a field that was not relabelled by $\eta$.
Hence it is doing more than simply ``undoing'' the relabelling
performed by $\eta$.  This state of affairs will be prevented by
requiring that $\eta$ is limited to $r$, for example by removing `x
$\leadsto$ z' from $\eta$.  Only paths changed by $\eta$ will be
changed back by $\eta^-$ and our claim will hold true.


\subsection{Record types}

A (proper) record \textit{type} is a labelled set 
where the objects labelled are types or, in some cases, certain
kinds of mathematical objects which can be used to construct types.

A record $r$ is \textit{well-typed} with respect to a system of types
\textbf{TYPE} with set of types \textbf{Type} and a set of labels $L$
iff for each field $\langle\ell,a\rangle\in r$, $\ell\in L$ and
either $a:_\mathbf{TYPE}T$ for some $T\in\mathbf{Type}$ or $a$ is
itself a record which is well-typed with respect to \textbf{TYPE} and $L$.

A system of complex types \textbf{TYPE}$_C$ = $\langle${\bf Type}, {\bf BType},
$\langle$\textbf{PType}, {\bf Pred}, \textbf{ArgIndices}, {\it
  Arity\/}$\rangle$, $\langle A,F\rangle$$\rangle$ \textit{has record
  types based on $\langle L, \mathbf{RType}\rangle$}, where $L$ is a countably infinite set (of labels)
and \textbf{RType} $\subseteq$ \textbf{Type}, where \textbf{RType} is
defined by:
\begin{enumerate} 
 
\item $\mathit{Rec}\in\mathbf{RType}$

\item $r:_{\mathbf{TYPE}_C}\mathit{Rec}$ iff $r$ is a well-typed record with
  respect to \textbf{TYPE$_C$} and $L$.

\item $\mathit{ERec}\in\mathbf{RType}$

\item $r:_{\mathbf{TYPE}_C}\mathit{ERec}$ iff $r=\emptyset$

\item if $\ell\in L$ and $T\in\mathbf{Type}$, then
  $\{\langle\ell,T\rangle\}\in\mathbf{RType}$.

\item $r:_{\mathbf{TYPE}_C}\{\langle\ell,T\rangle\}$ iff
  $r:_{\mathbf{TYPE}_C}\mathit{Rec}$, $\langle\ell,a\rangle\in r$ and
  $a:_{\mathbf{TYPE}_C}T$.

\item if $R\in\mathbf{RType}-\{\mathit{Rec},\mathit{ERec}\}$, $\ell\in L$, $\ell$ does not occur as a
  label in $R$ (i.e. there is no field $\langle\ell',T'\rangle$ in $R$
  such that $\ell'=\ell$), then
  $R\cup\{\langle\ell,T\rangle\}\in\mathbf{RType}$.

\item $r:_{\mathbf{TYPE}_C}R\cup\{\langle\ell,T\rangle\}$ iff
  $r:_{\mathbf{TYPE}_C}R$, $\langle\ell,a\rangle\in r$ and $a:_{\mathbf{TYPE}_C}T$.
 
\end{enumerate}

We say that $T$ is a \textit{proper} record type if it is a non-empty
set of fields.\footnote{This terminology was not introduced in \cite{Cooper2012}.}\label{pg:proprectype} 

This gives us non-dependent record types in a system of complex types.
We can extend this to intensional systems of complex types (with
stratification).  

% We can introduce a limited
% dependency into a system of complex types with record types by
% requiring that the values associated with certain fields are the
% results of applying an operation to the values of other fields. We
% first define an operation signature based on a system of complex
% types.

% An \textit{operation signature} based on  \textbf{TYPE}$_C$ = $\langle${\bf Type}, {\bf BType},
% $\langle$\textbf{PType}, {\bf Pred}, \textbf{ArgIndices}, {\it
%   Arity\/}$\rangle$, $\langle A,F\rangle$$\rangle$ is a set
% $\mathcal{O}$ of
% quadruples $\langle O, \langle T_1,\ldots,T_n\rangle, T, o\rangle$
% such that
% for any $\omega$ and $\omega'$ in $\mathcal{O}$, if
%   $\omega_1=\omega'_1$ (that is, the first elements of the tuples
%   are identical) then $\omega=\omega'$ (that is, $\mathcal{O}$
%   defines a function on the first elements of the tuples).  
%  Each quadruple is such that
% \begin{enumerate} 
 
% \item $O$ is a symbol (``operator symbol'') 
 
% \item for some $n$, $\langle T_1,\ldots,T_n\rangle$ is a sequence of
%   types in \textbf{Type} (``the arity of the operator'')

% \item $T$ is a type in \textbf{Type} (``the type of the range of the operator'')

% \item $o$ is an operator with domain $\{a\mid a:_{\mathbf{TYPE}_C}
%   T_1\}\times\ldots\times\{a\mid a:_{\mathbf{TYPE}_C} T_n\}$ and range
%   included in $\{a\mid a:_{\mathbf{TYPE}_C} T\}$
 
% \end{enumerate} 

% \sloppy{
% A system of complex types \textbf{TYPE}$_C$ = $\langle${\bf Type}, {\bf BType},
% $\langle$\textbf{PType}, {\bf Pred}, \textbf{ArgIndices}, {\it
%   Arity\/}$\rangle$, $\langle A,F\rangle$$\rangle$ with record
%   types based on $\langle L, \mathbf{RType}\rangle$ \textit{has
%   operator dependent record types based on $\mathcal{O}$}, where
% $\mathcal{O}$ is an operation signature based on \textbf{TYPE}$_C$ if}
% \begin{enumerate} 
 
% \item  if $R$ is a member of {\bf RType}, $\ell\in L$ not occurring as a label in $R$, 
%   $R.\pi_1,\ldots,R.\pi_n$ are paths in $R$  and for some
%   $\omega\in\mathcal{O}$, $\omega_1$ is the operation symbol $O$ and
%   $\omega_2$ is an $n$-place sequence
%    $\langle
%   T_1,\ldots,T_n\rangle$ (that is $O$ is an $n$-place operation symbol
%   with arity $\langle
%   T_1,\ldots,T_n\rangle$) such that if $r:R$ then
%     $r.\pi_1:_{\mathbf{TYPE}_C}T_1,\ldots, r.\pi_n:_{\mathbf{TYPE}_C}T_n$ and $T$ is the range type $\omega_3$, then $R \cup \{\langle\ell, T_{O(\pi_1,\ldots,\pi_n)}\rangle\}$ is a
%     member of {\bf RType}.
    
    

 
% \item Using $o$ to represent the operator $\omega_4$, $r :_{\mathbf{TYPE}_C} R \cup \{\langle\ell, T_{O(\pi_1,\ldots,\pi_n)}\rangle\}$ iff $r :_{\mathbf{TYPE}_C}
%     R$, $\langle\ell,a\rangle$ is a field in $r$ and $a :_{\mathbf{TYPE}_C} T_{o(r.\pi_1,\ldots,r.\pi_n)}$. 
 
% \end{enumerate} 
  
    



An {\it intensional system of complex types\/} {\bf TYPE$_\mathit{IC}$} = $\langle${\bf Type}$^n$, {\bf BType},
$\langle$\textbf{PType}$^n$, {\bf Pred}, \textbf{ArgIndices}, {\it
  Arity\/}$\rangle$, $\langle
A,F^n\rangle$$\rangle_{n\in\mathit{Nat}}$ \textit{has record types
  based on} $\langle L, \mathbf{RType}^n\rangle_{n\in\mathit{Nat}}$ if
for each $n$,  $\langle${\bf Type}$^n$, {\bf BType},
$\langle$\textbf{PType}$^n$, {\bf Pred}, \textbf{ArgIndices}, {\it
  Arity\/}$\rangle$, $\langle
A,F^n\rangle$$\rangle$ has record types based on $\langle
L,\mathbf{RType}^n\rangle$ and
\begin{enumerate} 
 
\item for each $n$, \textbf{RType}$^n$ $\subseteq$ \textbf{RType}$^{n+1}$ 
 
\item for each $n>0$, \textit{RecType}$^n$ $\in$ \textbf{RType}$^n$

\item for each $n>0$,
  $T:_{\mathbf{TYPE}_{\mathit{IC}_n}}\mathit{RecType}^n$ iff $T\in\mathbf{RType}^{n-1}$
 
\end{enumerate} 

Here, but not in \cite{Cooper2012}, we make explicit that
\textit{RecType} is treated in a similar manner to 
\textit{Type}, that is, it is a distinguished urelement and \textit{RecType}$^n$
represents the labelled set
$\{\langle\mathrm{ord},n\rangle,\langle\mathrm{typ},\mathit{RecType}\rangle\}$
where `ord' and `typ' are reserved labels (``order'', ``type'').


Intensional type systems 
may in addition contain \textit{dependent} record types.  

An {\it intensional system of complex types\/} {\bf TYPE$_\mathit{IC}$} = $\langle${\bf Type}$^n$, {\bf BType},
$\langle$\textbf{PType}$^n$, {\bf Pred}, \textbf{ArgIndices}, {\it
  Arity\/}$\rangle$, $\langle
A,F^n\rangle$$\rangle_{n\in\mathit{Nat}}$ \textit{has dependent record types
  based on} $\langle L, \mathbf{RType}^n\rangle_{n\in\mathit{Nat}}$,
if it has records types based on $\langle L,
\mathbf{RType}^n\rangle_{n\in\mathit{Nat}}$ and for each $n>0$
\begin{enumerate} 
 
\item if $R$ is a member of {\bf RType}$^n$, $\ell\in L$  not occurring as a label in $R$, $T_1,\ldots,T_m\in\mathbf{Type}^n$, $R.\pi_1,\ldots,R.\pi_m$ are paths in
    $R$  and
    ${\cal F}$ is a function of type $((a_1:T_1)\rightarrow\ldots\rightarrow((a_m:T_m)\rightarrow\mathit{Type}^n)\ldots)$, then $R \cup \{\langle\ell, \langle{\cal F}, \langle\pi_1,\ldots,\pi_m\rangle\rangle\rangle\}\in\mathbf{RType}^n$.
    


 
\item $r :_{\mathbf{TYPE}_{\mathit{IC}_n}}R\cup\{\langle\ell, \langle{\cal F}, \langle\pi_1,\ldots,\pi_m\rangle\rangle\rangle\}$ iff $r :_{\mathbf{TYPE}_{\mathit{IC}_n}} R$, $\langle\ell, a\rangle$ is a field
in $r$,  $r.\pi_1:_{\mathbf{TYPE}_{\mathit{IC}_n}}T_1,\ldots,r.\pi_m:_{\mathbf{TYPE}_{\mathit{IC}_n}}T_m$ and  $a :_{\mathbf{TYPE}_{\mathit{IC}_n}} {\cal F}(r.\pi_1,\ldots,r.\pi_m)$.  
 
\end{enumerate} 




We represent a record type $\{\langle\ell_1,T_1\rangle,\ldots,\langle\ell_n,T_n\rangle\}$
graphically as
\begin{display}
\record{\tfield{$\ell_1$}{$T_1$} \\
        \ldots \\
        \tfield{$\ell_n$}{$T_n$}}
\end{display}
In the case of a manifest field, that is, one containing a singleton
type, as in $\langle\ell,T_a\rangle$, we display this
\begin{display}
\record{\mfield{$\ell$}{$a$}{$T$}}
\end{display}
In the case of a multiple singleton type (a singleton type formed from
a singleton type) as in $\langle\ell,T_{a,b,\ldots}\rangle$, we display
\begin{display}
\record{\mfield{$\ell$}{$a,b,\ldots$}{$T$}}
\end{display}


In the case of dependent record types
we sometimes use a convenient notation representing e.g. 
\begin{display}
\mbox{$\langle\lambda u \lambda v\ 
\textrm{love}(u,v), \langle\pi_1,\pi_2\rangle\rangle$}
\end{display}
 as 
\begin{display}
love($\pi_1,\pi_2$)
\end{display}

Our systems now allow both function types and dependent record types
and allow dependent record types to be arguments to functions.  We
have to be careful when considering what the result of applying a
function to a dependent record type should be.  Consider the following
simple example:
\begin{display}
$\lambda
v_0:$\textit{RecType} . \smallrecord{\smalltfield{c$_0$}{$v_0$}}
\end{display}
What should be the result of applying this function to the record type
\begin{display}
\record{\tfield{x}{\textit{Ind}} \\
        \tfield{c$_1$}{$\langle\lambda v_1:$\textit{Ind} . dog($v_1$),
          $\langle$x$\rangle\rangle$}}
\end{display}
Given normal assumptions about function application the result would
be
\begin{display}
\record{\tfield{c$_0$}{\record{\tfield{x}{\textit{Ind}} \\
        \tfield{c$_1$}{$\langle\lambda v_1:$\textit{Ind} . dog($v_1$),
          $\langle$x$\rangle\rangle$}}}} (incorrect!)
\end{display}
but this would be incorrect.  In fact it is not a well-formed record
type since x is not a path in it.  Instead the result should be 
\begin{display}
\record{\tfield{c$_0$}{\record{\tfield{x}{\textit{Ind}} \\
        \tfield{c$_1$}{$\langle\lambda v_1:$\textit{Ind} . dog($v_1$),
          $\langle$c$_0$.x$\rangle\rangle$}}}}
\end{display}
where the path from the top of the record type is specified.  However,
in the abbreviatory notation we write just `x' when the label is used
as an argument and interpret this as the path from the top of the
record type to the field labelled `x' in the local record type.  Thus
we will write
\begin{display}
\record{\tfield{x}{\textit{Ind}} \\
        \tfield{c$_1$}{dog(x)}}
\end{display}
(where the `x' in `dog(x)' signifies the path consisting of just the
single label `x') and
\begin{display}
\record{\tfield{c$_0$}{\record{\tfield{x}{\textit{Ind}} \\
        \tfield{c$_1$}{dog(x)}}}}
\end{display}
(where the `x' in `dog(x)' signifies the path from the top of the
record type down to `x' in the local
record type, that is, `c$_0$.x').\footnote{This convention of
  representing the path from the top of the record type to the
  ``local'' field by the final label on the path is new since \cite{Cooper2012}.}

Note
that this adjustment of paths is only required when a record type is being
substituted into a position that lies on a path within a resulting
record type.  It will not, for example, apply in a case where a record
type is to be substituted for an argument to a predicate such as when
applying the function
\begin{display}
$\lambda v_0:$\textit{RecType} . \smallrecord{\tfield{c$_0$}{appear($v_0$)}}
\end{display}
to
\begin{display}
\record{\tfield{x}{\textit{Ind}} \\
        \tfield{c$_1$}{$\langle\lambda v:$\textit{Ind} . dog($v$),
          $\langle$x$\rangle\rangle$} \\
        \tfield{c$_2$}{$\langle\lambda v:$\textit{Ind} . approach($v$),
          $\langle$x$\rangle\rangle$}}
\end{display}
where the position of $v_0$ is in an ``intensional context'', that is,
as the argument to a predicate and there is no path to this position
in the record type resulting from applying the function.  Here the
result of the application is
\begin{display}
\record{\tfield{c$_0$}{appear(
\record{\tfield{x}{\textit{Ind}} \\
        \tfield{c$_1$}{$\langle\lambda v:$\textit{Ind} . dog($v$),
          $\langle$x$\rangle\rangle$} \\
        \tfield{c$_2$}{$\langle\lambda v:$\textit{Ind} . approach($v$),
          $\langle$x$\rangle\rangle$}}
)}}
\end{display}
with no adjustment necessary to the paths representing the
dependencies.\footnote{This record corresponds to the interpretation
  of \textit{it appears that a dog is approaching}.}  (Note that `c$_0$.x' is not a path in this record
type.) 

Suppose that we wish to represent a type which requires that there is
some dog such that it appears to be approaching (that is a \textit{de
  re} reading).  In the abbreviatory notation we might be tempted to
write
\begin{display}
\record{\tfield{x}{\textit{Ind}} \\
        \tfield{c$_1$}{dog(x)} \\
        \tfield{c$_0$}{appear(\record{\tfield{c$_2$}{approach(x)}})}}
      (incorrect!)
\end{display}
corresponding to
\begin{display}
\record{\tfield{x}{\textit{Ind}} \\
        \tfield{c$_1$}{$\langle\lambda v$:\textit{Ind} . dog($v$),
          $\langle$x$\rangle\rangle$} \\
        \tfield{c$_0$}{appear(\record{\tfield{c$_2$}{$\langle\lambda
              v$:\textit{Ind} . approach($v$),
              $\langle$x$\rangle\rangle$}})}} (incorrect!)
\end{display}
This is, however, incorrect since it refers to a path `x' in the type
which is the argument to `appear' which does not exist.  Instead we
need to refer to the path `x' in the record type containing the field
labelled `c$_0$':
\begin{display}
 \record{\tfield{x}{\textit{Ind}} \\
        \tfield{c$_1$}{$\langle\lambda v$:\textit{Ind} . dog($v$),
          $\langle$x$\rangle\rangle$} \\
        \tfield{c$_0$}{$\langle\lambda v$:\textit{Ind} . appear(\record{\tfield{c$_2$}{approach($v$)}}),
              $\langle$x$\rangle\rangle$}}
\end{display}
In the abbreviatory notation we will use `$\Uparrow$' to indicate that
the path referred to is in the ``next higher'' record
type\footnote{This notation is new since \cite{Cooper2012}.}:
\begin{display}
\record{\tfield{x}{\textit{Ind}} \\
        \tfield{c$_1$}{dog(x)} \\
        \tfield{c$_0$}{appear(\record{\tfield{c$_2$}{approach($\Uparrow$x)}})}}
\end{display}

These matters arise as a result of our choice of using paths to
represent dependencies in record types (rather than, for example,
introducing additional unique identifiers to keep track of the
positions within a record type as has been suggested by Thierry Coquand).  It seems like a matter of implementation rather than a matter
of substance and it is straightforward to define a path-aware notion
of substitution which can be used in the definition of what it means
to apply a TTR function to an argument.  If $f$ is a function
represented by $\lambda v:T\ .\ \varphi$ and $\alpha$ is the representation
of an object of type $T$, then the result of applying $f$ to $\alpha$,
$f(\alpha)$, is represented by Subst($\alpha$,$v$,$\varphi$,$\emptyset$),
that is, the result of substituting $\alpha$ for $v$ in $\varphi$ with
respect to the empty path where for arbitrary
$\alpha,v,\varphi,\pi$,
Subst($\alpha$,$v$,$\varphi$,$\pi$) is defined as 
\begin{enumerate} 
 
\item extend-paths($\alpha$,$\pi$), if $\varphi$ is $v$ 
 
\item $\varphi$, if $\varphi$ is of the form $\lambda v:T\ .\ \zeta$, for some
  $T$ and $\zeta$ (i.e. don't do any substitution if $v$ is bound
  within $\varphi$)

\item $\lambda u:T$ . Subst($\alpha$,$v$,$\zeta$,$\pi$), if $\varphi$ is
  of the form $\lambda u:T . \zeta$ and $u$ is not $v$.

\item
  \record{\tfield{$\ell_1$}{Subst($\alpha$,$v$,$T_1$,$\pi.\ell_1$)} \\
          \ldots \\
          \tfield{$\ell_n$}{Subst($\alpha$,$v$,$T_n$,$\pi.\ell_n$)}},
        if $\varphi$ is \record{\tfield{$\ell_1$}{$T_1$} \\
                             \ldots \\ 
                             \tfield{$\ell_n$}{$T_n$}}

\item $P$(Subst($\alpha$,$v$,$\beta_1$,$\pi$),\ldots,Subst($\alpha$,$v$,$\beta_n$,$\pi$)),
  if $\alpha$ is $P(\beta_1,\ldots,\beta_n)$ for some predicate $P$

% \item $O$(Subst($\alpha$,$v$,$\beta_1$,$\pi$),\ldots,Subst($\alpha$,$v$,$\beta_n$,$\pi$)),
%   if $\alpha$ is $O(\beta_1,\ldots,\beta_n)$ for some operator symbol $O$

\item $\varphi$ otherwise
 
\end{enumerate} 
 
extend-paths($\alpha$,$\pi$) is
\begin{enumerate} 
 
\item $\langle f, \langle \pi.\pi_1,\ldots,\pi.\pi_n\rangle\rangle$,
  if $\alpha$ is $\langle f, \langle \pi_1,\ldots,\pi_n\rangle\rangle$ 
 
\item \record{\tfield{$\ell_1$}{extend-paths($T_1,\pi$)} \\
              \ldots \\
              \tfield{$\ell_n$}{extend-paths($T_n,\pi$)}} if $\alpha$
            is \record{\tfield{$\ell_1$}{$T_1$} \\
                       \ldots \\
                       \tfield{$\ell_n$}{$T_n$}}

\item
  $P$(extend-paths($\beta_1,\pi$),\ldots,extend-paths($\beta_n,\pi$)),
  if $\alpha$ is $P(\beta_1,\ldots,\beta_n)$ for some predicate $P$

% \item $O$(extend-paths($\beta_1,\pi$),\ldots,extend-paths($\beta_n,\pi$)),
%   if $\alpha$ is $O(\beta_1,\ldots,\beta_n)$ for some operator symbol $O$

\item $\alpha$, otherwise
 
\end{enumerate} 


\section{Merges of record types}
\label{app:merge}

If $T_1$ and $T_2$ are record types then there will always be a record
type (not a meet) 
$T_3$ which is necessarily equivalent to $T_1\wedge T_2$.
Let us consider some examples:

\smallrecord{\smalltfield{f}{$T_1$}}$\wedge$\smallrecord{\smalltfield{g}{$T_2$}}
$\approx$ 
\smallrecord{\smalltfield{f}{$T_1$}\\
        \smalltfield{g}{$T_2$}}

\smallrecord{\smalltfield{f}{$T_1$}}$\wedge$\smallrecord{\smalltfield{f}{$T_2$}}
$\approx$ 
\smallrecord{\smalltfield{f}{$T_1\wedge T_2$}}

\noindent We
define a function $\mu$ which maps meets of record types to an
equivalent record type, record types to equivalent types where meets
in their values have been simplified by $\mu$ and any other types to themselves:
\begin{enumerate} 
 
\item If for some $T_1$, $T_2$, $T=T_1\wedge T_2$ then
  $\mu(T)=\mu'(\mu(T_1)\wedge\mu(T_2))$. 
 
\item If $T$ is a record type then $\mu(T)$ is $T'$ such that for any
  $\ell$,$v$, $\langle\ell,\mu(v)\rangle\in T'$ iff
  $\langle\ell,v\rangle\in T$.

\item Otherwise $\mu(T)=T$.
 
\end{enumerate} 

$\mu'(T_1\wedge T_2)$ is defined by:
\begin{enumerate} 
 
\item if $T_1$ and $T_2$ are record types, then $\mu'(T_1\wedge
  T_2)=T_3$ such that
\begin{enumerate} 
 
\item for any $\ell,v_1,v_2$, if $\langle\ell,v_1\rangle\in T_1$ and
  $\langle\ell,v_2\rangle\in T_2$, then 

\begin{enumerate} 
 
\item if $v_1$ and $v_2$ are $\langle\lambda u_1:T'_1\ldots\lambda
  u_i:T'_i(\phi),\langle\pi_1\ldots\pi_i\rangle\rangle$ and $\langle\lambda u'_1:T''_1\ldots\lambda
  u'_k:T''_k(\psi),\langle\pi'_1\ldots\pi'_k\rangle\rangle$
  respectively, then $\langle\lambda u_1:T'_1\ldots\lambda
  u_i:T'_i,\lambda u'_1:T''_1\ldots\lambda
  u'_k:T''_k(\mu(\phi\wedge\psi)), \langle\pi_1\ldots\pi_i,\pi'_1\ldots\pi'_k\rangle\rangle\in T_3$
 
\item if $v_1$ is $\langle\lambda u_1:T'_1\ldots\lambda
  u_i:T'_i(\phi),\langle\pi_1\ldots\pi_i\rangle\rangle$ and $v_2$ is a
  type (i.e. not of the form $\langle f,\Pi\rangle$ for some function
  $f$ and sequence of paths $\Pi$), then $\langle\lambda u_1:T'_1\ldots\lambda
  u_i:T'_i(\mu(\phi\wedge
  v_2)),\langle\pi_1\ldots\pi_i\rangle\rangle\in T_3$

\item if $v_2$ is $\langle\lambda u'_1:T''_1\ldots\lambda
  u'_k:T''_k(\psi),\langle\pi'_1\ldots\pi'_k\rangle\rangle$ and $v_1$
  is a type, then $\langle\lambda u'_1:T''_1\ldots\lambda
  u'_k:T''_k(\mu(v_1\wedge\psi)),\langle\pi'_1\ldots\pi'_k\rangle\rangle\in
  T_3$

\item otherwise $\langle\ell,\mu(v_1\wedge
  v_2)\rangle\in T_3$ 
 
\end{enumerate} 
  


 
\item for any $\ell,v_1$, if $\langle\ell,v_1\rangle\in T_1$ and there
  is no $v_2$ such that $\langle\ell,v_2\rangle\in T_2$, then
  $\langle\ell,v_1\rangle\in T_3$

\item for any $\ell,v_2$, if $\langle\ell,v_2\rangle\in T_2$ and there
  is no $v_1$ such that $\langle\ell,v_1\rangle\in T_1$, then
  $\langle\ell,v_2\rangle\in T_3$
 
\end{enumerate} 

\item if $T_1$ is \textit{Rec} and $T_2$ is a record type, then
  $\mu'(T_1\wedge T_2)=T_2$

\item if $T_1$ is a record type  and $T_2$ is \textit{Rec}, then
  $\mu'(T_1\wedge T_2)=T_1$

\item If $T_1$ is $[T_1']$ ($\{T_1'\}$, $\{\!\mid T_1'\mid\!\}$) and
  $T_2$ is $[T_2']$ ($\{T_2'\}$, $\{\!\mid T_2'\mid\!\}$), then
  $\mu'(T_1\wedge T_2)=[\mu(T_1'\wedge T_2')]$ ($\{\mu(T_1'\wedge
  T_2')\}$, $\{\!\mid\mu(T_1'\wedge T_2')\mid\!\}$)
   
 
\item \label{item:mergeotherwise} Otherwise $\mu'(T_1\wedge T_2)=T_1\wedge T_2$ 
 
\end{enumerate} 

$T_1$ \d{$\wedge$} $T_2$ is used to represent $\mu(T_1\wedge T_2)$.
We call  $T_1$ \d{$\wedge$} $T_2$ the \textit{merge} of $T_1$ and $T_2$.   

The following two clauses could be added at
the beginning of the definition of $\mu$ (after providing a
characterization of the subtype relation, $\sqsubseteq$).
\begin{enumerate} 
 
\item if for some $T_1,T_2$, $T=T_1\wedge T_2$ and $T_1\sqsubseteq T_2$
  then $\mu(T)=T_1$ 
 
\item if for some $T_1,T_2$, $T=T_1\wedge T_2$ and $T_2\sqsubseteq T_1$
  then $\mu(T)=T_2$ 
 
\end{enumerate} 

The current first clause would then hold in case neither of the
conditions of these two clauses are met.  The definition without these
additional clauses only accounts for simplification of meets which
have to do with merges of record types whereas the definition with the
additional clauses would in addition have the effect, for example, that
$\mu(T\wedge T_a)=T_a$ and $\mu(T_1\wedge(T_1\vee T_2))= T_1$
(provided that we have an appropriate definition of $\sqsubseteq$)
whereas the current definition without the additional clauses means
that $\mu$ 
leaves these types unchanged.

We define also a notion of \textit{asymmetric merge} of $T_1$ and
$T_2$ which is defined by a function exactly like $\mu$ except that
clause~\ref{item:mergeotherwise} of the definition of $\mu'$ is replaced by 
\begin{enumerate} 
 
\item[\ref{item:mergeotherwise}$'$.] Otherwise $\mu'(T_1\wedge T_2)=T_2$ 
 
\end{enumerate} 
  
We use $T_1$ \fbox{\d{$\wedge$}} $T_2$  to represent the asymmetric
merge of $T_1$ and $T_2$.  

These definitions do not in general avoid
the formation of ill-formed record types since they allow record types
to be replaced with non-record types within a record type thus
potentially removing paths that might be included in dependent type
fields elsewhere in the resulting type.  However, if merging is
restricted to either two record types or two non-record types this
problem should not occur since all paths from both types will be
preserved.  In the case of asymmetric merges we can allow the
replacement of non-record types by record types without risk.  Note
that our definition of dependent record types in \ref{app:rectypes}
allows for dependencies to fields that have conflicting types.  Such
record types will be well-formed though will not have any witnesses.

\paragraph{Merging functions which return types}

$\lambda r\!:\!T_1\ .\ T_2(r)$ \d{\d{$\wedge$}} $\lambda r\!:\!T_3\ .\ T_4(r)$
denotes the function $\lambda r\!:\!T_1$\d{$\wedge$}$T_3\ .\
T_2(r)$\d{$\wedge$}$T_4(r)$.

\paragraph{Constructing fixed point types for functions which return
  types}
\label{pg:fixedpointtype}

If, for
some type $T_1$, $f:(T_1\rightarrow\textit{Type})$ then $\mathcal{F}(f)$ is a \textit{fixed
point type} for $f$, that is $a:\mathcal{F}(f)$ implies $a:f(a)$.
$\mathcal{F}$ is defined by
\begin{quote}
$\mathcal{F}(\lambda r\!:\!T_1\ .\ T_2(r)) = T_1$\d{$\wedge$}$T'$

where $T'$ is like $T_2(r)$ except that any path $r.\pi$ is
replaced by $\pi$.
\end{quote}

Strictly speaking this definition is not quite correct since $T'$ may
not be a type because there may be a path occurring as an argument to
a predicate which is not introduced in $T'$.  A more correct, though
less perspicuous, definition would be
\begin{quote}
$\mathcal{F}(\lambda r\!:\!T_1\ .\ T_2(r)) =
[(\lambda r\!:\!T_1\ .\ T_1$\d{$\wedge$}$T_2(r))(r^*)]^{-r^*}$

where $r^*$ is a record of type $T_1$ such that there is no path occurring as an
argument to a predicate in $T_1$ or $T_2(r)$ of the form $r^*.\pi$ for
any $\pi$ and $[T]^{-r}$ is the result of replacing any path of the
form $r.\pi$, for any $\pi$, occurring as an argument to a predicate
in $T$, with $\pi$.
\end{quote}

This definition, however, fails to capture that $\mathcal{F}$ must be
defined as a partial function which is undefined on functions meeting
a certain condition.  This is taken account of in the following final
definition:
\begin{quote}
$\mathcal{F}$ is a partial function on functions such that if, for
some type $T_1$, $f:(T_1\rightarrow \textit{Type})$, $f=\lambda
r\!:\!T_1\ .\ T_2(r)$, $g$ is the function
\begin{quote}
$\lambda r\!:\!T_1\ .\ T_1$\d{$\wedge$}$T_2(r)$
\end{quote}
and for any $r_1,r_2:T_1$
\begin{quote}
$[g(r_1)]^{-r_1} = [g(r_2)]^{-r_2}$
\end{quote}
then if $r^*:T_1$,
\begin{quote}
$\mathcal{F}(f) = [g(r*)]^{-r^*}$
\end{quote}

For any function $f$ not covered by the above, $\mathcal{F}(f)$ is undefined.
\end{quote}
Let us take some concrete examples based on the discussion in
Chapter~\ref{ch:commonnouns}, Section~\ref{sec:dgqs}.  Suppose that
$f$ is the function
\begin{quote}
$\lambda r$:\smallrecord{\smalltfield{x}{\textit{Ind}}}
  . \record{\tfield{e}{dog($r$.x)}}
\end{quote}
Then $g$ will be
\begin{quote}
$\lambda r$:\smallrecord{\smalltfield{x}{\textit{Ind}}}
  . \smallrecord{\smalltfield{x}{\textit{Ind}}} \d{$\wedge$} \smallrecord{\smalltfield{e}{dog($r$.x)}}
\end{quote}
That is,
\begin{quote}
$\lambda r$:\smallrecord{\smalltfield{x}{\textit{Ind}}}
. \record{\tfield{x}{\textit{Ind}}\\
          \tfield{e}{dog($r$.x)}}
\end{quote}
If we now compute $[g(r^*)]^{-r^*}$ for some
$r^*$:\smallrecord{\smalltfield{x}{\textit{Ind}}}, that is we apply
$g$ to $r^*$ and then remove $r^*$ from all path-names that begin with
it we will obtain
\begin{quote}
\record{\tfield{x}{\textit{Ind}}\\
          \tfield{e}{dog(x)}}
\end{quote}
We would have obtained the same result no matter which record we chose
to be $r^*$, since $r$ only occurs in $g$ at the head of path names.
Now consider $f$ to be a variant of what we propose to be the content
of \textit{temperature} in Chapter~\ref{ch:commonnouns} (in fact,
similar to a
variant that we have proposed in previous work):
\begin{quote}
$\lambda r$:\textit{Rec} . \record{\tfield{e}{temperature($r$)}}
\end{quote}
Now $g$ will be
\begin{quote}
$\lambda r$:\textit{Rec} . \textit{Rec} \d{$\wedge$} \smallrecord{\smalltfield{e}{temperature($r$)}}
\end{quote}
That is,
\begin{quote}
$\lambda r$:\textit{Rec} . \record{\tfield{e}{temperature($r$)}}
\end{quote}
which happens to be identical with $f$.  If we apply this to a record
$r^*$ we will obtain
\begin{quote}
\record{\tfield{e}{temperature($r^*$)}}
\end{quote}
The result of removing all occurrences of $r^*$ at the head of a path
will be identical since $r^*$ occurs here not as the head of a path
but as an argument to a predicate. Consequently if we choose a
different record for $r^*$ the result will be a different type.  Thus
$\mathcal{F}$ is not defined on this function.  This is intuitively
correct since defining fixed points for this function would involve a
kind of non-well foundedness which we have not allowed in TTR.

The moral of this tale is that if you wish to define a dependent type
(that is, a function returning a type), $\lambda r\!:\!T_1\ .\ T_2(r)$,
for which you will be able to compute a fixed point type, make sure
that $T_2$ only depends on $r$ in that $r$ may be the head of path
names in $T_2(r)$.  Normally, you will also want to ensure that $T_1$
and $T_2$ do not share any labels in order to avoid unwanted clashes
when $T_1$ and $T_2$ are merged.
 
% \paragraph{Constructing backgrounding types corresponding to functions which return
%   types}
% \label{pg:backgroundingtype}
% We introduced backgrounding types for dependent types (functions which
% return types).  These are like the fixed point types generated by
% $\mathcal{F}$ except that an additional label `bg' (``background'') is
% placed before the domain type of the function in the resulting type.
% If $f$ is a function returning a type (a dependent type) then we will
% represent the corresponding backgrounding type as $\mathcal{B}(f)$.
% Here is an example:
% \begin{quote}
% $\mathcal{B}$($\lambda r$:\smallrecord{\smalltfield{x}{\textit{Ind}}\\
%                                        \smalltfield{e}{female(x)}} . 
% \smallrecord{\smalltfield{e}{leave($r$.x)}}) = 
% \record{\tfield{bg}{\record{\tfield{x}{\textit{Ind}}\\
%                             \tfield{e}{female(x)}}}\\
%         \tfield{e}{leave(bg.x)}}
% \end{quote}

% More generally, if $f:(T_1\rightarrow\textit{Type})$ then $\mathcal{B}(f)$ is the
% \textit{backgrounding type} for $f$.
% $\mathcal{B}$ is defined by
% \begin{quote}
% $\mathcal{B}(\lambda r\!:\!T_1\ .\ T_2(r))$ = \smallrecord{\smalltfield{bg}{$T_1$}}\d{$\wedge$}$T'$

% where $T'$ is like $T_2(r)$ except that any path $r.\pi$ is
% replaced by bg.$\pi$.
% \end{quote}

% In an exactly parallel way to the definition of the function
% $\mathcal{F}$ which creates fixed point types, this definition is strictly speaking not quite correct since $T'$ may
% not be a type because there may be a path occurring as an argument to
% a predicate which is not introduced in $T'$.  A more correct, though
% less perspicuous, definition parallel to that for $\mathcal{F}$, would be
% \begin{quote}
% $\mathcal{B}(\lambda r\!:\!T_1\ .\ T_2(r))$ =
% [($\lambda r$:$T_1$ . \smallrecord{\smalltfield{bg}{$T_1$}}\d{$\wedge$}$T_2(r)$)($r^*$)]$^{r^*/\mathrm{bg}}$

% where $r^*$ is a record of type \smallrecord{\smalltfield{bg}{$T_1$}} such that there is no path occurring as an
% argument to a predicate in $T_1$ or $T_2(r)$ of the form $r^*.\pi$ for
% any $\pi$ and $[T]^{r^*/\mathrm{bg}}$ is the result of replacing any path of the
% form $r^*.\pi$, for any $\pi$, occurring as an argument to a predicate
% in $T$, with bg.$\pi$.
% \end{quote}

\section{Flattening and relabelling of record types}
\label{app:flattening}

\subsection{Some considerations}
\label{app:flattening-considerations}

We extend the notions of flattening and relabelling of records
discussed in Appendix~\ref{app:rectypes} to types.\footnote{This was
  not made explicit in \cite{Cooper2012}.}  For non-dependent record
types flattening and relabelling are exactly parallel to that of
records since non-dependent record types are sets of ordered pairs of
labels and types.  They are just like records except that the second
members of the pairs are types.  However, in the case of dependent
record types, flattening and relabelling introduces some complications
which need to be taken account of.  We shall start with an informal
discussion of the complications.  

\paragraph{Labels in dependent fields need to be adjusted in
  flattening and unflattening.} First of all we need to
note that labels not only occur as the first members of the ordered
pairs that constitute the record type but also in the paths which
occur in dependent fields.  For the discussion of flattening we need
to make sure that we represent these paths in an appropriate way so
that they are correct in the representation of the flattened type.
Also in relabelling we need to make sure that the relabelling applies
to the paths in the dependent fields as well as the labels of the
record type.  Consider the record type which in abbreviatory notation
we write as:
\begin{quote}
\record{\tfield{x}{\record{\tfield{x}{\textit{Ind}}\\
                           \tfield{e}{dog(x)}}}}
\end{quote}
It would be incorrect to represent the flattening of this type as:
\begin{quote}
\record{\tfield{x.x}{\textit{Ind}}\\
        \tfield{x.e}{dog(x)}}
\end{quote}
The `x' in `dog(x)' does not represent a path in the flattened record
type. Rather the flattened type should be, in abbreviatory notation:
\begin{quote}
\record{\tfield{x.x}{\textit{Ind}}\\
        \tfield{x.e}{dog(x.x)}}
\end{quote}
We need also to guarantee that the unflattening of this type returns
the original unflattened type.  We will, of course, work to achieve
this, not with the abbreviatory notation but with the official
notation, or rather with the set theoretic objects which the official
notation represents.  That is, we need to convert: 
\begin{quote}
\record{\tfield{x}{\record{\tfield{x}{\textit{Ind}}\\
                           \tfield{e}{$\langle\lambda v$:\textit{Ind}
                             .  dog($v$), $\langle$x$\rangle\rangle$}}}}
\end{quote}
to:
\begin{quote}
\record{\tfield{x.x}{\textit{Ind}}\\
        \tfield{x.e}{$\langle\lambda v$:\textit{Ind}
                             .  dog($v$), $\langle$x.x$\rangle\rangle$}}
\end{quote}
and back again.



% or, in abbreviatory notation:
% \begin{quote}
% \record{\tfield{x.x}{\textit{Ind}}\\
%         \tfield{x.e}{dog(x.x)}}
% \end{quote}
\paragraph{Relabelling can require adjustment of dependent fields}
A relabelling such as:
\begin{quote}
x.x $\leadsto$ y
\end{quote}
applied to the flattened type leads to a different problem involved in
expansion.  
The relabelled flattened record would be:
\begin{quote}
\record{\tfield{y}{\textit{Ind}}\\
        \tfield{x.e}{$\langle\lambda v$:\textit{Ind}
                             .  dog($v$), $\langle$y$\rangle\rangle$}}
\end{quote}
% or, in abbreviatory notation:
% \begin{quote}
% \record{\tfield{y}{\textit{Ind}}\\
%         \tfield{x.e}{dog(y)}}
% \end{quote}
Expanding the relabelled flattened type in the way we did for records will result in:
\begin{quote}
\record{\tfield{y}{\textit{Ind}}\\
        \tfield{x}{\record{\tfield{e}{$\langle\lambda v$:\textit{Ind}
                             .  dog($v$), $\langle$y$\rangle\rangle$}}}}
\end{quote}
This is incorrect because the label `y' is not present in 
\begin{quote}
\record{\tfield{e}{$\langle\lambda v$:\textit{Ind}
                             .  dog($v$), $\langle$y$\rangle\rangle$}}
\end{quote}
Rather the correct result should be:
\begin{quote}
\record{\tfield{y}{\textit{Ind}}\\
        \tfield{x}{$\langle\lambda v$:\textit{Ind}
                             . \record{\tfield{e}{dog($v$)}},
                             $\langle$y$\rangle\rangle$}}
\end{quote}
In abbreviatory notation this is:
\begin{quote}
\record{\tfield{y}{\textit{Ind}}\\
        \tfield{x}{\record{\tfield{e}{dog($\Uparrow$y)}}}}
\end{quote}


% This is well-formed when we consider that the `y' in the dependent
% field is a total path beginning at the top of the record.  [Is this
% true?  What happens with ``appears to be approaching''?  Should we
% push the label inside the function?]

\paragraph{Dependencies involving paths which are not total}  
In the examples above the dependencies involve objects at the end of
paths.  It is also possible to have dependencies on records
corresponding to the value of a partial path.  Consider the following
example:
\begin{quote}
\record{\tfield{s}{\record{\tfield{x}{\textit{Ind}}\\
                           \tfield{c}{dog(x)}\\
                           \tfield{e}{run(x)}}}\\
        \tfield{e}{see(sam,s)}}
\end{quote}
which might be a way of representing the type which is the content of
\textit{Sam saw a dog run}.  The official notation for this type is:
\begin{quote}
\record{\tfield{s}{\record{\tfield{x}{\textit{Ind}}\\
                           \tfield{c}{$\langle\lambda v$:\textit{Ind}
                             . dog($v$), $\langle$x$\rangle\rangle$}\\
                           \tfield{e}{$\langle\lambda v$:\textit{Ind}
                             . run($v$),
                             $\langle$x$\rangle\rangle$}}}\\
        \tfield{e}{$\langle\lambda v$:\textit{Rec} . see(sam,$v$), $\langle$s$\rangle\rangle$}}
\end{quote}
Assuming that we have a way of making sure that we have a way of
making sure that paths in dependent fields start from the root of the
type as discussed above, we might think that the flattening of this type is:
\begin{quote}
\record{\tfield{s.x}{\textit{Ind}}\\
        \tfield{s.c}{$\langle\lambda v$:\textit{Ind}
                             . dog($v$), $\langle$s.x$\rangle\rangle$}\\
        \tfield{s.e}{$\langle\lambda v$:\textit{Ind}
                             . run($v$),
                             $\langle$s.x$\rangle\rangle$}\\
        \tfield{e}{$\langle\lambda v$:\textit{Rec} . see(sam,$v$), $\langle$s$\rangle\rangle$}}
\end{quote} 
But this is, of course, not correct since the path `s' in the
`e'-field is not a path in the flattened type.  Instead we will make
the flattened type depend on a record of the type required for the
argument to `see' in the original unflattened type (and also on any
objects on which that type depends):
\begin{quote}
$\lambda r$:\smallrecord{\smalltfield{s}{\smallrecord{\smalltfield{x}{\textit{Ind}}\\
                         \smalltfield{c}{dog(x)}\\
                         \smalltfield{e}{run(x)}}}} . 
\record{\tfield{e}{see(sam, $r$.s)}}
\end{quote}
Call this $\mathcal{T}$.  This is, of course, not a type to which the flattening of a record
belonging to the unflattened type can belong.  Instead we say if $r$
is of the non-flattened type, then there is some $r'$ such that
$\varphi(r):\mathcal{T}(r')$ and there is some flattening, $T_{\varphi}$, of the
domain type of $\mathcal{T}$ such that $\varphi(r):T_{\varphi}$.
% We are now free to apply flattening or relabelling to the domain or
% range of this function without creating the problem that the second
% argument of `see' will not be defined.  Note that there may be several
% dependencies in the flattened type depending on how many partial paths
% occur in the dependent fields of the unflattened type and what the
% dependencies of the types pointed at by those partial paths are.  Thus the
% dependent type which is the flattening may take several arguments.

\subsection{Basic definitions of flattening and
  relabelling of record types}
\label{app:flattening-def}

We start with a definition of what it means for a relabelling to be
limited to a type.  This is given in terms of the notion of a
relabelling being limited to a record (p.~\pageref{pg:limited}).
\begin{quote}
A relabelling, $\eta$, is \textit{limited to} a record type, $T$, just
in case for any record, $r$, $r:T$ implies $\eta$ is limited to
$\varphi(r)$.
\end{quote}
We can then define the notion of relabelling of record types as
follows:
\begin{quote}
If $T_1$ and $T_2$ are record types and $\eta$ is a relabelling
limited to $T_1$, then $T_2$ \textit{is an $\eta$-relabelling of}
$T_1$ just in case for any record, $r$, such that $\eta$ is a proper
relabelling (see p.~\pageref{pg:proper-relabelling}) of $r$, $r:T_1$
implies $[r]_\eta:T_2$.
\end{quote}
Note that this definition does not require a definition of the
flattening of types but relies of the notion of flattening of records
which is necessary to define relabelling of records.  Thus if all we
are interested in is the relabelling of types as in the notions of
matching discussed in Chapter~\ref{ch:intensional}, then we do not
have to address the problem of dependence involving partial paths
discussed in Appendix~\ref{app:flattening-considerations}.

However, for the sake of completeness we give definitions of
flattening and unflattening of record types.  There are two cases to
consider: that where the flattening is a record type and that where
the flattening is a dependent type which returns a record type when
applied to an appropriate argument.  We first give the definitions for
the case where the flattening is a record type.
\begin{quote}
If $T_1$ and $T_2$ are record types, 

$T_2$ \textit{is a flattening of}
$T_1$ just in case for any record, $r$, $r:T_1$ implies
$\varphi(r):T_2$

$T_2$ \textit{is an unflattening of} $T_1$ just in case for any
record, $r$, $r:T_1$ implies $\varphi^-(r):T_2$.

\end{quote}
For the cases where the flattening is a dependent type we have the
following definitions:
\begin{quote}
If $T_1$, $T_2$ and ${T_2}_\varphi$ are record types such that ${T_2}_\varphi$ is a
flattening of $T_2$ and
$\mathcal{T}:(T_2\rightarrow\textit{RecType})$, 

$\mathcal{T}$
\textit{is a flattening of} $T_1$ just in case for any record, $r$,
$r:T_1$ implies that for some record, $r'$, such that  $r':T_2$,
$\varphi(r):(\mathcal{T}(r')\wedge {T_2}_\varphi)$

$T_1$ \textit{is an unflattening of} $\mathcal{T}$ just in case for
any record, $r$, $r:({T_2}_\varphi\wedge\mathcal{T}(\varphi^-(r))$
implies $\varphi^-(r):T_1$
\end{quote}

Note that if $T_1$ ($\mathcal{T}$) is a flattening of $T_2$ if and
only if $T_2$ is an unflattening of $T_1$ ($\mathcal{T}$).



\subsection{Computing flattenings and relabellings of record types}

The definitions in Section~\ref{app:flattening-def} enable us to tell
whether one type is a flattening or relabelling of another but they do
not give us an algorithm for constructing a flattening or relabelling
of a particular type. 

@@


% If $T$ is a type, then $\varphi(T)$, \textit{the flattening of} $T$ is
% \begin{enumerate} 
 
% \item $\displaystyle{\bigcup_{f\in T}}\mathrm{FlattenField}(f)$,  if $T$ is a proper record type (see p.~\pageref{pg:proprectype}) 
 
% \item $T$, otherwise 
 
% \end{enumerate} 
% where $\mathrm{FlattenField}$ is defined as follows:

% $\mathrm{FlattenField}(\langle\ell, T\rangle)$ is
% \begin{enumerate} 
 
% \item $\varphi(\left\{\begin{array}{l}\langle\ell.\ell_1, T_1\rangle,\\
%                                  \langle\ell.\ell_2, T_2\rangle,\\
%                                  \vdots \\
%                                  \langle\ell.\ell_n, T_n\rangle
%                  \end{array}\right\})$, if $T$ is $\left\{\begin{array}{l}\langle\ell_1, T_1\rangle,\\
%                                  \langle\ell_2, T_2\rangle,\\
%                                  \vdots \\
%                                  \langle\ell_n, T_n\rangle
%                  \end{array}\right\}$
 
% \item $\langle\ell, T\rangle$, otherwise 
 
% \end{enumerate} 
    
% Correspondingly, we can define a way of computing the inverse of
% flattening.  

% If $T$ is a type, $\varphi^-(T)$, \textit{the inverse flattening
%   (expansion) of} $T$ is
% \begin{enumerate} 
 
% \item $\left\{\begin{array}{l}\langle\ell_1,
%                                     \varphi^-(\left\{\begin{array}{l}\langle\pi_{1,1},T_{1,1}\rangle\\
%                                                                      \langle\pi_{1,2},T_{1,2}\rangle\\ 
%                                                                      \vdots\\
%                                                                      \langle\pi_{1,m_1},T_{1,m_1}\rangle
%                                                      \end{array}
%                                               \right\}\rangle,\\
%                                \langle\ell_2,
%                                     \varphi^-(\left\{\begin{array}{l}\langle\pi_{2,1},T_{2,1}\rangle\\
%                                                                      \langle\pi_{2,2},T_{2,2}\rangle\\ 
%                                                                      \vdots\\
%                                                                      \langle\pi_{2,m_2},T_{1,m_2}\rangle
%                                                      \end{array}
%                                               \right\}\rangle,\\
%                                \vdots\\
%                                \langle\ell_n,
%                                     \varphi^-(\left\{\begin{array}{l}\langle\pi_{n,1},T_{n,1}\rangle\\
%                                                                      \langle\pi_{n,2},T_{n,2}\rangle\\ 
%                                                                      \vdots\\
%                                                                      \langle\pi_{n,m_n},T_{n,m_n}\rangle
%                                                      \end{array}
%                                               \right\}\rangle\end{array}\right\}$,
%                                             if $T$ is 
% $\left\{\begin{array}{l}\langle\ell_1.\pi_{1,1},T_{1,1}\rangle\\
%                         \langle\ell_1.\pi_{1,2},T_{1,2}\rangle\\
%                         \vdots\\
%                         \langle\ell_1.\pi_{1,m_1},T_{1,m_1}\rangle\\
%                         \langle\ell_2.\pi_{2,1},T_{2,1}\rangle\\
%                         \langle\ell_2.\pi_{2,2},T_{2,2}\rangle\\
%                         \vdots\\
%                         \langle\ell_2.\pi_{2,m_2},T_{2,m_2}\rangle\\
%                         \vdots\\
%                         \langle\ell_n.\pi_{n,1},T_{n,1}\rangle\\
%                         \langle\ell_n.\pi_{n,2},T_{n,2}\rangle\\
%                         \vdots\\
%                         \langle\ell_n.\pi_{n,m_n},T_{1,m_n}\rangle\\
%          \end{array}\right\}$
  
% \item $T$, otherwise 
 
% \end{enumerate}

% The set of \textit{complex labels} (or \textit{paths}), $L_\pi$,  \textit{based
%   on a set of labels}, $L$, is a set such that
% \begin{enumerate} 
 
% \item if $\ell\in L$, then $\ell\in L_\pi$ 
 
% \item if $\pi\in L_\pi$ and $\ell\in L$, then $\pi.\ell\in L_\pi$

% \item nothing else is a member of $L_\pi$
 
% \end{enumerate} 
   


% If $T$ is a type in a system of types based on labels $L$, then a
% \textit{relabelling} for $T$ is a one-one function,
% $\eta$, whose domain is the set of labels of the flattened type
% $\varphi(T)$ and whose range is included in $L_\pi$ and such that for
% no $\pi_1$, $\pi_2$ and $\pi$, $\eta(\pi_2)=\eta(\pi_1).\pi$, that is,
% no new complex label can be assigned by the relabelling which is a
% proper initial part of another complex label assigned by the relabelling.  The result of
% \textit{relabelling} $T$ with $\eta$, a relabelling for $T$, is $\varphi^-([\varphi(T)]_\eta)$
% where for any flattened type $T$, $[T]_\eta$ is the result of replacing every
% occurrence of all the labels $\ell$ in $T$ (including those that occur as arguments to predicates,
% i.e. in dependent fields) with $\eta(\ell)$.

% \textbf{Alternative definition which should be used throughout [????]:}  The result of
% \textit{relabelling} $T$ with $\eta$, a relabelling for $T$, is
% denoted by  $[T]_\eta$ and is defined by $\varphi^-(\{\varphi(T)\}_\eta)$
% where for any flattened type $T$, $\{T\}_\eta$ is the result of replacing every
% occurrence of all the labels $\ell$ in $T$ (including those that occur as arguments to predicates,
% i.e. in dependent fields) with $\eta(\ell)$.

% The fact that a relabelling $\eta$ is one-one means that the inverse
% of $\eta$, $\eta^-$, is also a function which we can use to recover
% the original labelling.  This gives us the opportunity to relabel,
% carry out some operation on the relabelled type and then restore the
% original labelling.  This is, for example, exploited in the discussion
% of accommodation in Chapter~\ref{ch:propnames} on p.~\pageref{pg:inverserelabelling}.

% If fields $f_1$ and $f_2$ have the same label, that is
% $f_1=\langle\ell, T_1\rangle$ and $f_2=\langle\ell, T_2\rangle$ for
% some $\ell$, $T_1$ and $T_2$, then the result of \textit{merging} the two
% \textit{fields} is $\langle\ell,T_1$\d{$\wedge$}$T_2\rangle$. 

\section{Using records to restrict and specify record types}
\label{app:specrec}

(These definitions were not included in \citealp{Cooper2012}.)

If $T$ is a type and $r$ is a record, then $T\!\restriction\!r$ is a
type. $a:T\!\restriction\!r$ iff $a\underline{\varepsilon} r$ (see
Appendix~\ref{app:rectypes}) and $a:T$.

If $T$ is a record type and $r$ is a record, then $T\!\mid\!\restriction\! r$, the
\textit{restriction} of $T$ by
$r$ is the result
of replacing each field, $\langle \ell, T'\rangle$, in $T$ such that
$\ell$ is a label in $r$, with 
\begin{enumerate}
\item  $\langle\ell,T'\!\restriction\!r.\ell\rangle$\footnote{That is,
    in an adaptation of our
graphical notation for manifest fields, \smallrecord{\smalltfield{$\ell$}{$T'$}} is
replaced by \smallrecord{\smallrfield{$\ell$}{$r.\ell$}{$T'$}}}, if
$T'$ is a type
\item $\langle\ell,\langle f',\Pi\rangle\rangle$, if $T'=\langle
  f,\Pi\rangle$ where $f$ is a function and $\Pi$ is a sequence of
  paths of length $n$ and for any
  $a_1,\ldots,a_n$, $f'(a_1)\ldots(a_n)$ is defined iff
  $f(a_1)\ldots(a_n)$ is defined and $f'(a_1)\ldots(a_n) =
  f(a_1)\ldots(a_n)\!\restriction\!r.\ell$
\end{enumerate}

A variant of this notion of restriction is default restriction
which will only require restriction of fields which are not already
restricted. If $T$ is a record type and $r$ is a record, then
$T\drestr r$, the
\textit{default restriction} of $T$ by
$r$ is the result
of replacing each field, $\langle \ell, T'\rangle$, in $T$ such that
$\ell$ is a label in $r$, with 
\begin{enumerate}
\item  $\langle\ell,T'\!\restriction\!r.\ell\rangle$, if
$T'$ is a type but not a restricted type. If $T'$ is a
singleton type then $\langle \ell, T'\rangle$ is replaced by itself.
\item $\langle\ell,\langle f',\Pi\rangle\rangle$, if $T'=\langle
  f,\Pi\rangle$ where $f$ is a function and $\Pi$ is a sequence of
  paths of length $n$ and for any
  $a_1,\ldots,a_n$, $f'(a_1)\ldots(a_n)$ is defined iff
  $f(a_1)\ldots(a_n)$ is defined and $f'(a_1)\ldots(a_n) =
  f(a_1)\ldots(a_n)\!\restriction\!r.\ell$ if $f(a_1)\ldots(a_n)$ is not a
  restricted type. Otherwise, $f'(a_1)\ldots(a_n) =
  f(a_1)\ldots(a_n)$.
\end{enumerate}  


If $T$ is a record type and $r$ is a record, then $T\parallel r$, the
\textit{specification} (or \textit{anchoring}) of $T$ by
$r$\footnote{$r$ could also be referred to as a \textit{partial
    specifier} or \textit{partial assignment} for $T$} is the result
of replacing each field, $\langle \ell, T'\rangle$, in $T$ such that
$\ell$ is a label in $r$, with 
\begin{enumerate}
\item  $\langle\ell,T'_{r.\ell}\rangle$\footnote{That is, in our
standard graphical notation, \smallrecord{\smalltfield{$\ell$}{$T'$}} is
replaced by \smallrecord{\smallmfield{$\ell$}{$r.\ell$}{$T'$}}}, if
$T'$ is a type
\item $\langle\ell,\langle f',\Pi\rangle\rangle$, if $T'=\langle
  f,\Pi\rangle$ where $f$ is a function and $\Pi$ is a sequence of
  paths of length $n$ and for any
  $a_1,\ldots,a_n$, $f'(a_1)\ldots(a_n)$ is defined iff
  $f(a_1)\ldots(a_n)$ is defined and $f'(a_1)\ldots(a_n) =
  f(a_1)\ldots(a_n)_{r.\ell}$
\end{enumerate}

A variant of this notion of specification is default specification
which will only require specification of fields which are not already
specified. If $T$ is a record type and $r$ is a record, then
$T\dspec r$, the
\textit{default specification} (or \textit{default anchoring}) of $T$ by
$r$ is the result
of replacing each field, $\langle \ell, T'\rangle$, in $T$ such that
$\ell$ is a label in $r$, with 
\begin{enumerate}
\item  $\langle\ell,T'_{r.\ell}\rangle$, if
$T'$ is a type but not a singleton type.\footnote{That is, $\langle
  \ell, T'\rangle$ is not already a manifest field.} If $T'$ is a
singleton type then $\langle \ell, T'\rangle$ is replaced by itself.
\item $\langle\ell,\langle f',\Pi\rangle\rangle$, if $T'=\langle
  f,\Pi\rangle$ where $f$ is a function and $\Pi$ is a sequence of
  paths of length $n$ and for any
  $a_1,\ldots,a_n$, $f'(a_1)\ldots(a_n)$ is defined iff
  $f(a_1)\ldots(a_n)$ is defined and $f'(a_1)\ldots(a_n) =
  f(a_1)\ldots(a_n)_{r.\ell}$ if $f(a_1)\ldots(a_n)$ is not a
  singleton type. Otherwise, $f'(a_1)\ldots(a_n) =
  f(a_1)\ldots(a_n)$.
\end{enumerate}  

Types can also be specified by records which have different labels to
the type by using a relabelling.  Thus $T\parallel_\eta r$ is the
result of replacing each field in $T$, $\langle\ell,T'\rangle$,  such
that $\eta(\ell)$ is a label in $r$ and $r.\eta(\ell):T'$, with a
manifest field$\langle\ell,T'_{r.\eta(\ell)}\rangle$.  More exactly we
will define $T\parallel_\eta r$ in terms of flattening, relabelling
and specification by a record:
\begin{quote}
$T\parallel_\eta r$ =
$\varphi^-([[\varphi(T)]_\eta\parallel\varphi(r)]_{\eta^-})$
\end{quote}
Here are two examples:  suppose that % $a:T_1$, $b:T_2$ and 
$\eta$ is a
function with domain $\{\ell_1,\ell_2\}$ such that
$\eta(\ell_1)=\ell_3$ and $\eta(\ell_2)=\ell_4$.  Then:
\begin{quote}
\record{\tfield{$\ell_1$}{$T_1$}\\
        \tfield{$\ell_2$}{$T_2$}} $\parallel_\eta$ 
\record{\field{$\ell_3$}{$a$}\\
        \field{$\ell_4$}{$b$}}
=
\record{\mfield{$\ell_1$}{$a$}{$T_1$}\\
        \mfield{$\ell_2$}{$b$}{$T_2$}}
\end{quote}
Suppose now that % $a$ and $b$ are as above and that 
$\eta$ is a
function with domain $\{\ell_5.\ell_1, \ell_5.\ell_2, \ell_6\}$ (where
$\ell_5.\ell_1$ and $\ell_5.\ell_2$ are complex labels) such that
$\eta(\ell_5.\ell_1)=\ell_7.\ell_3$,
$\eta(\ell_5.\ell_2)=\ell_8.\ell_4$ and $\eta(\ell_6)=\ell_6$.  Then:
\begin{quote}
\record{\tfield{$\ell_5$}{\record{\tfield{$\ell_1$}{$T_1$}\\
                                  \tfield{$\ell_2$}{$T_2$}}}\\
        \tfield{$\ell_6$}{$T_3$}} $\parallel_\eta$
\record{\field{$\ell_7$}{\record{\field{$\ell_3$}{$a$}}}\\
        \field{$\ell_8$}{\record{\field{$\ell_4$}{$b$}\\
                                 \field{$\ell_9$}{$c$}}}} =
\begin{quote}
\record{\tfield{$\ell_5$}{\record{\mfield{$\ell_1$}{$a$}{$T_1$}\\
                                  \mfield{$\ell_2$}{$b$}{$T_2$}}}\\
        \tfield{$\ell_6$}{$T_3$}}
\end{quote}
\end{quote}  

\section{Strings and regular types}
\label{sec:regular}
\label{app:strings}

A \textit{string algebra over a set of objects} $O$ is a pair $\langle S,
^{\frown}\rangle$ where:
\begin{enumerate} 
 
\item $S$ is the closure of $O\cup\{\varepsilon\}$ ($\varepsilon$ is the empty string) under the binary operation `$^{\frown}$' (``concatenation'') 
 
\item for any $s$ in $S$, $\varepsilon^{\frown}s = s^{\frown}\varepsilon = s$

\item for any $s_1,s_2,s_3$ in $S$,
  $(s_1^{\frown}s_2)^{\frown}s_3=s_1^{\frown}(s_2^{\frown}s_3)$.  For
  this reason we normally write $s_1^{\frown}s_2^{\frown}s_3$ or more
  simply $s_1s_2s_3$.
 
\end{enumerate}

The objects in $S$ are called strings.  Strings have length.  $\varepsilon$ has
length 0.  If $s$ is a string in $S$
with length $n$ and $a$ is an object in $O$ then $s^{\frown}a$ has
length $n+1$.  We use $s[n]$ to represent the $n$th element of string
$s$.

In TTR strings are records\footnote{This is new since \cite{Cooper2012}.} with fields labelled by a distinguished
ordered countably infinite set of labels (corresponding to the natural
numbers):  t$_0$,t$_1$,\ldots.  $\varepsilon$ is the empty record,
that is the empty set.  This has length 0.  (Recall that records are sets of ordered
pairs.)  A string with one element $a$ (of some type) is the
record \smallrecord{\field{t$_0$}{$a$}}. If $s$ is a string whose
highest label in the order is t$_n$ ($n\geq 0$) and $a$ is an object of
some type then $s^{\frown}a$ is $s\cup\{\langle t_{n+1},a\rangle\}$.  If
$s$ is a string whose highest label in the order is t$_n$ then the
length of $s$ is $n+1$.  $s[n]$ is defined to be $s$.t$_n$.
Concatenation (`$^{\frown}$') can be extended to include concatenation
of strings of arbitrary length.  If $s$ is a string and $s_n$ is a
string of length $n$, then $s^{\frown}s_n$ is
$s^{\frown}s_n[0]^{\frown}\ldots^{\frown}s_n[n-1]$. 

If $s$ is a string of length $n$ of records such that for each $i$,
$0\leq i<n$, $s[i].\pi$ is a defined path, 
$\displaystyle\mathop{\mathrm{concat}}_{0\leq i<n}(s[i].\pi)$ denotes
$s[0].\pi^{\frown}\ldots^{\frown}s[n-1].\pi$.  We use
concat$_i$($s[i].\pi$) to represent $\displaystyle\mathop{\mathrm{concat}}_{0\leq i<\mathrm{length}(s)}(s[i].\pi)$.

A system of complex types  \textbf{TYPE}$_S$ = $\langle${\bf Type}, {\bf BType},
$\langle$\textbf{PType}, {\bf Pred}, \textbf{ArgIndices}, {\it
  Arity\/}$\rangle$, $\langle A,F\rangle$$\rangle$ with record types
based on $\langle L, \mathbf{RType}\rangle$ \textit{has string types}
if
\begin{enumerate} 
 
\item for each natural number $i$, t$_i\in L$
 
\item \textit{String} $\in$ \textbf{BType}

\item $\emptyset\in$ \textit{String}

\item if $T \in$ \textbf{Type} and $a:_{\mathbf{TYPE}_S}T$ then
  \smallrecord{\field{t$_0$}{a}} : \textit{String}

\item if $s$ :$_{\mathbf{TYPE}_S}$ \textit{String}, t$_n$ is a label in $s$ such that
  there is no $i>n$ where t$_i$ is a label in $s$, $T \in$ \textbf{Type}
  and $a:_{\mathbf{TYPE}_S}T$ then $s\cup\{\langle \mathrm{t}_{n+1},a\rangle\}$ :$_{\mathbf{TYPE}_S}$
  \textit{String}

\item Nothing is of type \textit{String} except as required above.
 
\end{enumerate} 
  
 
We can define types whose elements are strings.  Such types correspond
to regular expressions and we will call them \textit{regular types}.
Here we will define just two kinds of such types:  concatenation types
and Kleene-+ types.

A system of complex types with string types \textbf{TYPE}$_S$ = $\langle${\bf Type}, {\bf BType},
$\langle$\textbf{PType}, {\bf Pred}, \textbf{ArgIndices}, {\it
  Arity\/}$\rangle$, $\langle A,F\rangle$$\rangle$ \textit{has
  concatenation types} if 
\begin{enumerate} 
 
\item 
\begin{enumerate}
\item for any $T_1$, $T_2$ $\in$ {\bf Type},  $({T_1}^\frown\!T_2)$
  $\in$ {\bf Type}

\item for any $T_1$, $T_2$, $T_3$ $\in$ \textbf{Type},
  $({T_1}^{\frown}T_2)^{\frown}T_3 =
  {T_1}^{\frown}({T_2}^{\frown}T_3)$\footnote{This has been added to
    the definition in \cite{Cooper2012} to make associativity explicit.}
\end{enumerate}
 
\item $a :_{\mathbf{TYPE}_S} {T_1}^\frown\!T_2$ iff $a=x^\frown\!y$, $x:_{\mathbf{TYPE}_S}T_1$ and $y:_{\mathbf{TYPE}_S}T_2$ 
 
\end{enumerate} 
\textbf{TYPE}$_S$ \textit{has Kleene-+ types} if
\begin{enumerate} 
 
\item 
for any $T$ $\in$ {\bf Type},  $T^+$ $\in$ {\bf Type} 
 
\item $a:_{\mathbf{TYPE}_S}T^+$ iff $a=x_1^\frown\!\ldots^\frown\!x_n$, $n>0$ and for $i$, $1\leq
i\leq n$, $x_i:_{\mathbf{TYPE}_S}T$ 
 
\end{enumerate} 

\textbf{TYPE}$_S$ \textit{has Kleene-* types} if
\begin{enumerate} 
 
\item 
for any $T$ $\in$ {\bf Type},  $T^*$ $\in$ {\bf Type} 
 
\item $a:_{\mathbf{TYPE}_S}T^*$ iff
  $a=x_1^\frown\!\ldots^\frown\!x_n$, $n\geq 0$ and for $i$, $1\leq
i\leq n$, $x_i:_{\mathbf{TYPE}_S}T$ 
 
\end{enumerate} 

Note that this definition distinguishes and object $a$ and the unit
string consisting of $a$, that is, \smallrecord{\field{$t_0$}{$a$}}.
We will use $\overset{\frown}{a}$ to represent this.

   
Strings are used standardly in formal language theory where strings of
symbols or strings of words are normally considered.  Following
important insights by Tim Fernando
\cite{Fernando2004,Fernando2006,Fernando2008,Fernando2009} we shall be
concerned rather with strings of events.   We use informal notations
like ` ``sam'' ' and ` ``ran'' '  to represent phonological types of
speech events (utterances of \textit{Sam} and \textit{ran}).  Thus `
``sam''$^{\frown}$``ran'' ' is the type of speech events which are
concatenations of an utterance of \textit{Sam} and an utterance of \textit{ran}.

\subsection*{Predicates which relate strings}

We introduce a number of distinguished predicates which are used to
relate strings.  The following predicates all have arity
[\textit{String},\textit{String}]: init, final, final\_align
\begin{description}
\item[init] ``$s_1$ is an initial substring of $s_2$''

If $s_1$ is a string of length $n$ and $s_2$ is a string of any
length, then $s$ : init($s_1$,$s_2$) iff the length of $s_2$ is
greater than or equal to $n$ and for each $i$, $0\leq i<n$,
$s_1[i]=s_2[i]$ and $s=s_2$.

\item[final] ``$s_1$ is an final substring of $s_2$''

If $s_1$ is a string of length $n$ and $s_2$ is a string of 
length $m$, then $s$ : final($s_1$,$s_2$) iff $m$ is
greater than or equal to $n$ and for each $i$, $0\leq i<n$,
$s_1[i]=s_2[(m-n)+i]$ and $s=s_2$.

\item[final\_align] ``$s_1$ is aligned with a final substring of
  $s_2$''

If $s_1$:\textit{Rec}$^+$ is a string of length $n$ and $s_2$:\textit{Rec}$^+$ is a string of 
length $m$, then $s$ : final\_align($s_1$,$s_2$) iff 
\begin{enumerate} 
 
\item $m$ is
greater than or equal to $n$ 
 
\item  $s$ is a string of length $m$

\item for each
$i$, $0\leq i<n$, 

\begin{enumerate} 
 
\item  $s[(m-n)+i]$ :
\smallrecord{\smalltfield{e$_1$}{\textit{Rec}} \\
             \smalltfield{e$_2$}{\textit{Rec}}} 
\item  $s[(m-n)+i]$.e$_1$ =
$s_1[i]$

\item $s[(m-n)+i]$.e$_2$ = $s_2[(m-n)+i]$
 
\end{enumerate} 

\item  otherwise for each
$i$, $0\leq i<m$, $s[i]=s_2[i]$ 
 
\end{enumerate} 


\end{description}

%%% Local Variables: 
%%% mode: latex
%%% TeX-master: "ttl"
%%% End: 
