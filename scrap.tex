We will argue below that it is very important that the complex types
we introduce are structured which enables them to be compared and
modified.  This is what makes it possible to account for how agents
exploit and adapt the resources they have as they create new language
during the course of interaction.  It is not quite enough, however, simply
to have objects with components.  We also need a systematic way of
accessing these components, a system of labelling which will provide
us with handles for the various pieces.  This is where the record
types of TTR come in.  There is a large literature on type theories
with records in computer science, for example,
\cite{Tasistro1997,Betarte1998,BetarteTasistro1998,CoquandPollackTakeyama2004}.
Our notion of record type is closely related to those discussed in
this literature, though (like the rest of TTR) couched in rather
different terms.  For us a record type is a set of fields where each
field is an ordered pair of a label and a type (or a pair consisting
of a dependent type and a sequence of path names corresponding to what
the type is to depend on).  A record belonging to such a type is a set
of fields which includes fields with the same labels as those occurring
in the type.  Each field in the record with a label matching one in
the type must contain an object belonging to the type of the
corresponding field in the type.  This is made precise in
Section~\ref{sec-records}.

It is an important aspect of human cognition that we not only appear
to construct complex cognitive objects out of smaller ones as their
components but that we also have ways of accessing the components and
performing operations like substitutions, deletions and additions.
Cognitive processing also appears to depend on similarity metrics
which require us to compare components.  Thus labelling or the
provision of handles pointing to the components of complex objects is
an important part of a formal theory of human cognition and in TTR it
is the records and record types which do this work for us.

The importance of labelling has been reflected in the use of features
in linguistic theorizing ranging from the early Prague school
\cite{Trubetzkoy1939} to modern feature based grammar
\cite{Sag:Wasow:ea:03}.  It appears in somewhat different form in the
use of discourse referents in the treatment of discourse anaphora in
formal semantics \cite{KampReyle1993}.  In \cite{Cooper2005a} we argue
that record types can be used both to model the feature structures of
feature based grammar and the discourse representation structures of
discourse representation theory.  This is part of a general programme
for developing TTR to be a general type theory which underlies all our
linguistic cognitive processing and the way in which we state the
lexicon and grammar
rules in Section~\ref{sec:sentsem} reflects this.  In fact, what we
would like to see in the future is a single type theory which
underlies all of human cognitive processing and we would suggest that
a step towards this is
the discussion here concerning how accounts of modality and intensionality are
built up on the basis of typing judgements linked to perception.

Working out the formal details of this proposal involves us in
considerable complexity, as will become apparent below.  In contrast
to the statement of TTR in \cite{Cooper2005} we attempt here to
present interacting modules which are not that complex in themselves.
Nevertheless, knowing exactly what you have got when you put
everything together is not an entirely trivial matter.  It would be
nice from a logical point of view if human cognition presented itself
to us in neat separate boxes which we could study independently.  But
this is not the case.  We
are after all involved in the study of a biological system and there
is no reason in principle why our cognitive anatomy should be any
simpler than our physical anatomy with its multiplicity of objects
such as organs, nerves, muscles and arteries and complex dependencies
between them, though all built up on the basis of general principles
of cell structure and DNA.  Compared with what we know about physical
anatomy,  TTR seems quite modest in the number of different kinds of
objects it proposes and the interrelationships between them.  It is
probably going to get more complicated as the research progresses in
the future\ldots
