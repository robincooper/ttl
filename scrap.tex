We will argue below that it is very important that the complex types
we introduce are structured which enables them to be compared and
modified.  This is what makes it possible to account for how agents
exploit and adapt the resources they have as they create new language
during the course of interaction.  It is not quite enough, however, simply
to have objects with components.  We also need a systematic way of
accessing these components, a system of labelling which will provide
us with handles for the various pieces.  This is where the record
types of TTR come in.  There is a large literature on type theories
with records in computer science, for example,
\cite{Tasistro1997,Betarte1998,BetarteTasistro1998,CoquandPollackTakeyama2004}.
Our notion of record type is closely related to those discussed in
this literature, though (like the rest of TTR) couched in rather
different terms.  For us a record type is a set of fields where each
field is an ordered pair of a label and a type (or a pair consisting
of a dependent type and a sequence of path names corresponding to what
the type is to depend on).  A record belonging to such a type is a set
of fields which includes fields with the same labels as those occurring
in the type.  Each field in the record with a label matching one in
the type must contain an object belonging to the type of the
corresponding field in the type.  This is made precise in
Section~\ref{sec-records}.

It is an important aspect of human cognition that we not only appear
to construct complex cognitive objects out of smaller ones as their
components but that we also have ways of accessing the components and
performing operations like substitutions, deletions and additions.
Cognitive processing also appears to depend on similarity metrics
which require us to compare components.  Thus labelling or the
provision of handles pointing to the components of complex objects is
an important part of a formal theory of human cognition and in TTR it
is the records and record types which do this work for us.

The importance of labelling has been reflected in the use of features
in linguistic theorizing ranging from the early Prague school
\cite{Trubetzkoy1939} to modern feature based grammar
\cite{Sag:Wasow:ea:03}.  It appears in somewhat different form in the
use of discourse referents in the treatment of discourse anaphora in
formal semantics \cite{KampReyle1993}.  In \cite{Cooper2005a} we argue
that record types can be used both to model the feature structures of
feature based grammar and the discourse representation structures of
discourse representation theory.  This is part of a general programme
for developing TTR to be a general type theory which underlies all our
linguistic cognitive processing and the way in which we state the
lexicon and grammar
rules in Section~\ref{sec:sentsem} reflects this.  In fact, what we
would like to see in the future is a single type theory which
underlies all of human cognitive processing and we would suggest that
a step towards this is
the discussion here concerning how accounts of modality and intensionality are
built up on the basis of typing judgements linked to perception.

Working out the formal details of this proposal involves us in
considerable complexity, as will become apparent below.  In contrast
to the statement of TTR in \cite{Cooper2005} we attempt here to
present interacting modules which are not that complex in themselves.
Nevertheless, knowing exactly what you have got when you put
everything together is not an entirely trivial matter.  It would be
nice from a logical point of view if human cognition presented itself
to us in neat separate boxes which we could study independently.  But
this is not the case.  We
are after all involved in the study of a biological system and there
is no reason in principle why our cognitive anatomy should be any
simpler than our physical anatomy with its multiplicity of objects
such as organs, nerves, muscles and arteries and complex dependencies
between them, though all built up on the basis of general principles
of cell structure and DNA.  Compared with what we know about physical
anatomy,  TTR seems quite modest in the number of different kinds of
objects it proposes and the interrelationships between them.  It is
probably going to get more complicated as the research progresses in
the future\ldots


We can paraphrase this as ``the property of being a dog which chases a
cat and there is something which it catches''.  In order to obtain the
anaphora we need to align the following two paths in the domain type
of this function: `$\mathfrak{c}.\mathfrak{s}.\text{x}_0$' and
`e$_2$.x'.  This we do in \nexteg{} by creating a manifest field on
the former path.

\begin{ex} 
  $\ulcorner\lambda r$:\smallrecord{
    \smalltfield{x}{\textit{Ind}}\\
    \smalltfield{$\mathfrak{c}$}{\smallrecord{
        \footnotesize{\textit{Cntxt}}\\
        \smalltfield{$\mathfrak{s}$}{\smallrecord{
            %\footnotesize{\textit{Assgnmnt}}\\
            \smallmfield{x$_0$}{$\Uparrow^2$e$_2$.x}{\textit{Ind}}}}\\
        \smalltfield{$\mathfrak{c}$}{\smallrecord{
            \smalltfield{f}{\textit{PropCntxt}}\\
            \smalltfield{a}{\textit{PropCntxt}}}}
}}\\
    \smalltfield{e$_1$}{dog(x)}\\
    \smalltfield{e$_2$}{\smallrecord{
        \smalltfield{x}{$\mathfrak{T}$(cat$'$)}\\
        \smalltfield{e}{chase$^\dagger$($r$.x, x)}}}}
   .
  \record{
    \tfield{e}{catch$^{\dagger}$($r$.x, $r.\mathfrak{c}.\mathfrak{s}.\text{x}_0$)}}$\urcorner$
\end{ex}
We can paraphrase this as ``the property of being a dog which chases a
cat and catches that cat''.

In order to include such properties with aligned paths as the scope of
quantifiers in our interpretations, we will first generalize the
characterization of alignment of paths given on
p.~\pageref{pg:path-alignment-types}f to functions in \nexteg{}.
\begin{ex} 
  If $\varphi=\ulcorner\lambda r\!:\!T\ .\ \psi\urcorner$ and
  $\pi_1,\pi_2\in\mathrm{paths}(T)$, then
  \begin{quote}
    $\varphi_{\pi_1=\pi_2}=\ulcorner\lambda r\!:\!T_{\pi_1=\pi_2}\ .\
    \psi\urcorner$
  \end{quote}
\end{ex} 
We then add two further clauses to the characterization of witnesses
for $\mathfrak{S}(T)$ for content types, $T$, in \nexteg{} with the
new additions boxed.
\begin{ex} 
\begin{subex} 
 
\item If $T:\textit{ContType}$, then $\mathfrak{S}(T)$ is a type 
 
\item The witnesses of $\mathfrak{S}(T)$ are characterized by
  \begin{enumerate} 
 
  \item if $\varphi:T$ then $\varphi:\mathfrak{S}(T)$
    
  \item \fbox{if $\varphi:\mathfrak{S}(T)$ and $\varphi:\textit{PPpty}$,
      then $\mathcal{L}(\varphi):\mathfrak{S}(T)$}
    
  \item \fbox{\begin{minipage}[t]{.9\linewidth}if $\varphi:\mathfrak{S}(T)$,
    $\varphi\sqsubseteq\text{\smallrecord{
        \smallmfield{scope}{$\psi$}{\textit{Ppty}}}}$ and
    $\pi_1,\pi_2\in\mathrm{paths}(\psi.\text{bg})$, then $\varphi[\text{scope}=\psi_{\pi_1=\pi_2}:\textit{Ppty}]:\mathfrak{S}(T)$\end{minipage}}

    
  \item if $\alpha\mathcal{O}\beta:T$, (for
      some combination operation, $\mathcal{O}$) and
      $\alpha\mathcal{O}_{i,j}\beta$ is defined (for some natural
      numbers, $i$ and $j$), then $\alpha\mathcal{O}_{i,j}\beta:\mathfrak{S}(T)$
 
  \item if $\varphi:T$ and $\varphi$ is in the range of `$\mathrm{storage}$', then
    $\mathrm{storage}(\varphi):\mathfrak{S}(T)$

  \item if $\varphi:T$ and `x$_i$' and $\varphi$ are appropriate
    arguments to `$\mathrm{retrieve}$', then
    $\mathrm{retrieve}(\text{x}_i,\varphi):\mathfrak{S}(T)$
    
  \item nothing is a witness for $\mathfrak{S}(T)$ except as required above.
 
  \end{enumerate} 
  
 
\end{subex} 
\label{ex:storage-donkey-type}   
\end{ex}



Overall we can get a parametric content for \textit{no dog which
  chases a cat catches it} which looks like \nexteg{} where
\savebox{\boxone}{\textbf{dog}$^\frown$\textbf{which}$^\frown$\textbf{chases}$^\frown$\textbf{a}$^\frown$\textbf{cat}} \usebox{\boxone}
and \savebox{\boxtwo}{\textbf{catches}$^\frown$\textbf{it}}\usebox{\boxtwo} represent parametric contents
which have not undergone operations introduced by $\mathfrak{S}$.
\begin{ex} 
  $\lambda c$:\textit{Cntxt} . \record{
    \mfield{restr}{\usebox{\boxone}($c$)}{\textit{Ppty}}\\
    \mfield{scope}{$(\mathcal{L}(\text{\usebox{\boxtwo}})(c)|_{\mathcal{F}(\text{\usebox{\boxone}})})_{\mathfrak{c}.\mathfrak{s}.\text{x}_0=\text{e}_2.\text{x}}$}{\textit{Ppty}}\\
    \smalltfield{e}{no(restr, scope)}}
\end{ex} 
Given the semantics we have specified for the determiner \textit{no},
\preveg{} is \nexteg{}.
\begin{ex} 
  $\lambda c$:\textit{Cntxt} . \record{
    \mfield{restr}{\usebox{\boxone}($c$)}{\textit{Ppty}}\\
    \mfield{scope}{$(\mathcal{L}(\text{\usebox{\boxtwo}})(c)|_{\mathcal{F}(\text{\usebox{\boxone}})})_{\mathfrak{c}.\mathfrak{s}.\text{x}_0=\text{e}_2.\text{x}}$}{\textit{Ppty}}\\
    \smalltfield{e}{\record{
        \tfield{X}{every$^w$(restr)}\\
        \tfield{f}{$((x:\mathfrak{T}(X))\rightarrow\neg\mathfrak{P}(\text{scope})\{x\})$}
      }}}
\end{ex} 
We can paraphrase this content as ``for every dog which chases a cat
it's not the case that it's a dog which chases a cat$_i$ and catches
it$_i$'' where the subscript on \textit{cat} and \textit{it} indicates
that \textit{it} is anaphorically related to \textit{a cat}.


% The basic content related to \textit{no} is given in \nexteg{}.
% \begin{ex} 
% $\lambda\mathfrak{s}$:\textit{Rec} $\lambda Q$:\textit{Ppty} $\lambda
% P$:\textit{Ppty} . \record{\tfield{e}{no($Q$,$P|_{\mathcal{F}(Q)}$)}} 
% \end{ex} 
% As usual we abbreviate the content of \textit{dog} as `dog$'$'.  The
% relative pronoun \textit{which} has the content specified for
% \textit{who} in Section~\ref{sec:long-distance}.  (We are simplifying
% by not accounting for the gender distinction between the two.)  We
% repeat this in \nexteg{} and will refer to it here as \textbf{which}.
% \begin{ex} 
% $\lambda\mathfrak{s}$:\smallrecord{\smalltfield{wh}{\textit{Ind}}}
%   . $\lambda P$:\textit{Ppty} . $P$\{$\mathfrak{s}$.wh\} 
% \end{ex} 
% The basic content of an utterance of \textit{chase} is given in
% \nexteg{}.
% \begin{ex} 
% $\lambda\mathfrak{s}$:\textit{Rec} $\lambda
% r_2$:\smallrecord{\smalltfield{x}{\textit{Quant}}} $\lambda
% r_1$:\smallrecord{\smalltfield{x}{\textit{Ind}}}
%   . \record{\tfield{e}{chase($r_1$.x, $r_2$.x)}} 
% \end{ex} 
% The indefinite article $a$ will have the content in \nexteg{}.
% \begin{ex} 
% $\lambda\mathfrak{s}$:\textit{Rec} $\lambda Q$:\textit{Ppty} $\lambda
% P$:\textit{Ppty} . \record{\tfield{e}{exist($Q$, $P|_{\mathcal{F}(Q)}$)}} 
% \end{ex} 
% Thus the phrase \textit{a cat} will have the content in \nexteg{}.
% \begin{ex} 
% $\lambda\mathfrak{s}$:\textit{Rec} $\lambda P$:\textit{Ppty}
% . \record{\tfield{e}{exist(cat$'$, $P|_{\mathcal{F}(\text{cat}')}$)}} 
% \end{ex} 
% We will refer to \preveg{} as \textbf{a$^\frown$cat}.  Given this the
% content of \textit{chase a cat} will be \nexteg{}.
% \begin{ex} 
% $\lambda\mathfrak{s}$:\textit{Rec} $\lambda
% r_1$:\smallrecord{\smalltfield{x}{\textit{Ind}}}
% . \record{\tfield{e}{chase($r_1$.x,
%     \textbf{a$^\frown$cat}($\mathfrak{s}$))}}
% \label{ex:chaseacat-noexport} 
% \end{ex} 
% Given that \textit{chase} is an extensional verb it will obey a
% constraint like (\ref{ex:mp-find}) on p.~\pageref{ex:mp-find}, as
% given in \nexteg{}.
% \begin{ex} 
% $e$ : chase($a$, $Q$) iff $e$ : $Q$($\lambda
% r$:\smallrecord{\smalltfield{x}{\textit{Ind}}} . \smallrecord{\smalltfield{e}{chase$^{\dagger}$($a$,
% $r$.x)}}) 
% \end{ex} 
% This means that we can construe the content to be a function which
% returns an equivalent type to that returned in
% (\ref{ex:chaseacat-noexport}).  This new content is given in
% \nexteg{}.
% \begin{ex} 
% $\lambda s$:\textit{Rec} $\lambda
% r_1$:\smallrecord{\smalltfield{x}{\textit{Ind}}}
%   . \record{\tfield{e}{\textbf{a$^\frown$cat}($\mathfrak{s}$)($\lambda
%       r$:\smallrecord{\smalltfield{x}{\textit{Ind}}}
%       . \smallrecord{\smalltfield{e}{chase$^\dagger$($r_1$.x, $r$.x)}})}} 
% \end{ex} 
% \preveg{} is equivalent to \nexteg{}.
% \begin{ex} 
% $\lambda\mathfrak{s}$:\textit{Rec} $\lambda
% r_1$:\smallrecord{\smalltfield{x}{\textit{Ind}}}
% . \record{\tfield{e}{exist(cat$'$, $\lambda
%     r$:\smallrecord{\smalltfield{x}{\textit{Ind}}\\
%                     \smalltfield{e}{cat(x)}}
%                   . \smallrecord{\smalltfield{e}{chase$^\dagger$($r_1$.x, $r$.x)}})}} 
% \end{ex} 
% In turn, \preveg{} is equivalent to \nexteg{}.
% \begin{ex} 
% $\lambda\mathfrak{s}$:\textit{Rec} $\lambda
% r_1$:\smallrecord{\smalltfield{x}{\textit{Ind}}}
%   . \record{\tfield{e}{\record{\tfield{x}{$\mathfrak{T}$(cat$'$)}\\
%                                \tfield{e}{\record{\tfield{$\mathfrak{c}$}{\smallrecord{\smallmfield{x}{$\Uparrow^2$x}{\textit{Ind}}\\
%                                                    \smalltfield{e}{cat(x)}}}\\
%                    \tfield{e}{\smallrecord{\smalltfield{e}{chase$^\dagger$($r_1$.x,
%                          $\Uparrow\!\mathfrak{c}$.x)}}}}}}}}
% \label{ex:chase-a-cat} 
% \end{ex} 
% We represent \preveg{} as \textbf{chase$^\frown$a$^\frown$cat}.  From
% this we can form a ``sentence with a gap'' interpretation,
% \textbf{chase$^\frown$a$^\frown$cat$_S$}, in the manner described in
% Section~\ref{sec:long-distance}.  This is given in \nexteg{}.
% \begin{ex} 
% $\lambda\mathfrak{s}$:\smallrecord{\smalltfield{wh$_0$}{\textit{Ind}}} . $\lambda
% P$:\textit{Ppty} . $P$\{$\mathfrak{s}$.wh$_0$\}(\textbf{chase$^\frown$a$^\frown$cat}($\mathfrak{s}$)) 
% \end{ex} 
% Unpacking \preveg{} we obtain \nexteg{}.
% \begin{ex} 
% $\lambda\mathfrak{s}$:\smallrecord{\smalltfield{wh$_0$}{\textit{Ind}}}
% . \record{\tfield{e}{\record{\tfield{x}{$\mathfrak{T}$(cat$'$)}\\
%                                \tfield{e}{\record{\tfield{$\mathfrak{c}$}{\smallrecord{\smallmfield{x}{$\Uparrow^2$x}{\textit{Ind}}\\
%                                                    \smalltfield{e}{cat(x)}}}\\
%                    \tfield{e}{\smallrecord{\smalltfield{e}{chase$^\dagger$($\mathfrak{s}$.wh$_0$, $\Uparrow\!\mathfrak{c}$.x)}}}}}}}} 
% \end{ex} 
% Again following the analysis of long-distance dependencies developed
% in Section~\ref{sec:long-distance}, the content of \textit{which
%   chased a cat} is given in \nexteg{}.
% \begin{ex} 
% $\lambda\mathfrak{s}$:\textit{Rec} . \\
% \hspace*{1em} $\lambda
% r_1$:\smallrecord{\smalltfield{x}{\textit{Ind}}}
% . \textbf{which}($\mathfrak{s}\oplus[\text{wh}=r_1.\text{x}$)($\lambda
% r_2$:\smallrecord{\smalltfield{x}{\textit{Ind}}} . \textbf{chase$^\frown$a$^\frown$cat$_S$}($\mathfrak{s}\oplus[\text{wh}_0=r_2.\text{x}]$) 
% \end{ex} 
% Unpacking \preveg{} gives us (\ref{ex:chase-a-cat}).  That is, the content of
% \textit{which chased a cat} is identical with the content of
% \textit{chased a cat}.  For clarity we will represent this content
% also as \textbf{which$^\frown$chase$^\frown$a$^\frown$cat}.
% Again following the proposal in Section~\ref{sec:long-distance} the
% content for \textit{dog which chases a cat} will be
% \nexteg{}. 
% \begin{ex} 
% $\lambda\mathfrak{s}$:\textit{Rec} $\lambda
% r$:\smallrecord{\smalltfield{x}{\textit{Ind}}}
% . \record{\tfield{e$_1$}{dog$'$\{$r$.x\}}\\
%           \tfield{e$_2$}{\textbf{which$^\frown$chase$^\frown$a$^\frown$cat}($\mathfrak{s}$)\{$r$.x\}}} 
% \end{ex} 
% We call this
% \textbf{dog$^\frown$which$^\frown$chase$^\frown$a$^\frown$cat}. Now
% \textit{no dog which chases a cat} will correspond to \nexteg{a} which
% is equivalent to \nexteg{b}.
% \begin{ex} 
% \begin{subex} 
 
% \item $\lambda\mathfrak{s}$:\textit{Rec} $\lambda P$:\textit{Ppty}
%   . \record{\tfield{e}{no(\textbf{dog$^\frown$which$^\frown$chase$^\frown$a$^\frown$cat}($\mathfrak{s}$),
%       $P|_{\mathcal{F}(\textbf{dog}^\frown\textbf{which}^\frown\textbf{chase}^\frown
%           \textbf{a}^\frown\textbf{cat}(\mathfrak{s}))}$)}} 
 
% \item $\lambda\mathfrak{s}$:\textit{Rec} $\lambda P$:\textit{Ppty}
%   . \record{\tfield{e}{\record{\tfield{X}{every$^w$(\textbf{dog$^\frown$which$^\frown$chase$^\frown$a$^\frown$cat}($\mathfrak{s}$))}\\
%                            \tfield{f}{(($x:\mathfrak{T}(\text{X}))\rightarrow\neg\mathfrak{P}(P|_{\mathcal{F}(\textbf{dog$^\frown$which$^\frown$chase$^\frown$a$^\frown$cat}(\mathfrak{s}))})\{\text{x}\}$)}}}} 
 
% \end{subex} 
% \label{ex:no-dog-which-chases-a-cat}   
% \end{ex} 
% Call this
% \textbf{no$^\frown$dog$^\frown$which$^\frown$chase$^\frown$a$^\frown$cat}.
% This is to combine with the content of \textit{catches it}, given in
% \nexteg{}. 
% \begin{ex} 
% $\lambda\mathfrak{s}$:\smallrecord{\smalltfield{x$_0$}{\textit{Ind}}}
% $\lambda r$:\smallrecord{\smalltfield{x}{\textit{Ind}}}
% . \record{\tfield{e}{catch$^\dagger$($r$.x, $\mathfrak{s}$.x$_0$)}} 
% \end{ex}
% We will represent this as \textbf{catch$^\frown$it}. 
% Using the S-combinator strategy to
% (\ref{ex:no-dog-which-chases-a-cat}b) yields \nexteg{a}, equivalently
% \nexteg{b}, which represents a reading of the sentence where
% \textit{it} is not captured and refers to a particular object provided
% by the context.
% \begin{ex}
% \begin{subex} 
 
% \item $\lambda\mathfrak{s}$:\smallrecord{\smalltfield{x$_0$}{\textit{Ind}}} 
%   . \smallrecord{\smalltfield{e}{\record{\tfield{X}{every$^w$(\textbf{dog$^\frown$which$^\frown$chase$^\frown$a$^\frown$cat}($\mathfrak{s}$))}\\
%                            \tfield{f}{(($x:\mathfrak{T}(\text{X}))\rightarrow$\\
% &&$\neg\mathfrak{P}(\lambda r$:\smallrecord{\smalltfield{x}{\textit{Ind}}}
% . \record{\tfield{e}{catch$^\dagger$($r$.x, $\mathfrak{s}$.x$_0$)}}$|_{\mathcal{F}(\textbf{dog$^\frown$which$^\frown$chase$^\frown$a$^\frown$cat}(\mathfrak{s}))})\{\text{x}\}$)}}}} 
 
% \item $\lambda\mathfrak{s}$:\smallrecord{\smalltfield{x$_0$}{\textit{Ind}}} 
%   . \smallrecord{\smalltfield{e}{\record{\tfield{X}{every$^w$(\textbf{dog$^\frown$which$^\frown$chase$^\frown$a$^\frown$cat}($\mathfrak{s}$))}\\
%                            \tfield{f}{(($x:\mathfrak{T}(\text{X}))\rightarrow$\\
% &&\hspace*{2em}$\neg$\smallrecord{\smalltfield{$\mathfrak{c}$}{\smallrecord{\smallmfield{x}{$x$}{\textit{Ind}}\\
%                                                                \smalltfield{e$_1$}{dog$'$\{x\}}\\
%                                                                \smalltfield{e$_2$}{\smallrecord{\smalltfield{e}{\smallrecord{\smalltfield{x}{$\mathfrak{T}$(cat$'$)}\\
%                                                                                                 \smalltfield{e}{\smallrecord{\smalltfield{$\mathfrak{c}$}{\smallrecord{\smallmfield{x}{$\Uparrow^2$x}{\textit{Ind}}\\
%                                                                                                                                                                        \smalltfield{e}{cat(x)}}}\\
%                                                                                                      \smalltfield{e}{\smallrecord{\smalltfield{e}{chase$^\dagger$($\Uparrow^4$x,
%                                                                                                            $\Uparrow\!\mathfrak{c}$.x)}}}}}}}}}}}\\
%                \smalltfield{e}{catch$^\dagger$($x$, $\mathfrak{s}$.x$_0$)}}
% )}}}} 
 
% \end{subex} 
   
 
% \end{ex} 
% In terms of the notation in \preveg{b}, it is simple enough to see
% what needs to be done in order to change it to a case of donkey
% anaphora.  The second argument to `catch$^\dagger$',
% `$\mathfrak{s}$.x$_0$', has to be changed so that it presents a path
% to the cat, that is, `$\mathfrak{c}$.e$_2$.e.x'.  Also the dependence
% of this parametric content on `x$_0$' has to be removed.  Thus the
% domain type for the $\lambda$-abstraction, currently
% `\smallrecord{\smalltfield{x$_0$}{\textit{Ind}}}', should be replaced
% by `\textit{Rec}'.  The result would be \nexteg{}.
% \begin{ex} 
%  $\lambda\mathfrak{s}$:\textit{Rec} 
%   . \smallrecord{\smalltfield{e}{\record{\tfield{X}{every$^w$(\textbf{dog$^\frown$which$^\frown$chase$^\frown$a$^\frown$cat}($\mathfrak{s}$))}\\
%                            \tfield{f}{(($x:\mathfrak{T}(\text{X}))\rightarrow$\\
% &&\hspace*{2em}$\neg$\smallrecord{\smalltfield{$\mathfrak{c}$}{\smallrecord{\smallmfield{x}{$x$}{\textit{Ind}}\\
%                                                                \smalltfield{e$_1$}{dog$'$\{x\}}\\
%                                                                \smalltfield{e$_2$}{\smallrecord{\smalltfield{e}{\smallrecord{\smalltfield{x}{$\mathfrak{T}$(cat$'$)}\\
%                                                                                                 \smalltfield{e}{\smallrecord{\smalltfield{$\mathfrak{c}$}{\smallrecord{\smallmfield{x}{$\Uparrow^2$x}{\textit{Ind}}\\
%                                                                                                                                                                        \smalltfield{e}{cat(x)}}}\\
%                                                                                                      \smalltfield{e}{\smallrecord{\smalltfield{e}{chase$^\dagger$($\Uparrow^4$x,
%                                                                                                            $\Uparrow\!\mathfrak{c}$.x)}}}}}}}}}}}\\
%                \smalltfield{e}{catch$^\dagger$($x$, $\mathfrak{c}$.e$_2$.e.x)}}
% )}}}}
% \label{ex:no-dog-which-chases-a-cat-catches-it-captured} 
% \end{ex} 
% We will achieve this by introducing a notion of restriction on a
% parametric property in which we will allow anaphora to take place.
% Let us first consider a case without anaphora.  Suppose that
% $\mathcal{P}$ is a parametric property of type
% $(T_1\rightarrow(T_2\rightarrow \textit{Type}))$ and that $T$ is a
% type then we define the restriction of $\mathcal{P}$ by $T$,
% $\mathcal{P}|^p_T$, to be \nexteg{}.
% \begin{ex} 
% $\lambda\mathfrak{s}$:$T_1$ $\lambda r$:$T_2$\d{$\wedge$}$T$ . $\mathcal{P}(\mathfrak{s})(r)$ 
% \end{ex} 
% Now suppose that $[\ell,v]$ is a field in $T_1$ and $\pi$ is a path in
% $T_2$ and we want to make an anaphoric association between $\ell$ and
% $\pi$.  We define such a restriction,
% $\mathcal{P}|^p_{T,\ell\leadsto\pi}$ as \nexteg{}.
% \begin{ex} 
% $\lambda\mathfrak{s}$:$T_1\ominus[\ell:v]$ $\lambda
% r$:$T_2$\d{$\wedge$}$T$ . $\mathcal{P}(\mathfrak{s}\oplus[\ell=r.\pi])(r)$ 
% \end{ex}
% To enable the capture of several pronouns, we let $\vec\Pi\subseteq
% T_1\times\text{paths}(T)$ such that $\vec\Pi$ is the graph of a one-one
% function (whose domain is included in the fields of $T_1$ and whose
% range is included in the paths of $T$).  We then define
% $\mathcal{P}|^p_{T,\vec{\Pi}}$ to be \nexteg{}.
% \begin{ex} 
% $\lambda\mathfrak{s}$:$T_1\ominus_{\text{set}}\vec{\Pi}_1$ $\lambda
% r$:$T_2$\d{$\wedge$}$T$ . $\mathcal{P}(\mathfrak{s}\oplus_{\text{set}}\{[\text{label}(\vec{\pi}_1)=r.\vec{\pi}_2]\mid\vec{\pi}\in\vec{\Pi}\})$ 
% \end{ex} 
% If $\vec{\Pi}$ is
% $\{\langle\langle\ell_1,v_1\rangle,\pi_1\rangle,\ldots,\langle\langle\ell_n,v_n\rangle,\pi_n\rangle\}$
% then for convenience we represent $\mathcal{P}|^p_{T,\vec{\Pi}}$ as
% $\mathcal{P}|^p_{T,\ell_1\leadsto\pi_1,\ldots,\ell_n\leadsto\pi_n}$. Using
% this notation,  the content of \textit{no dog which chases a cat catches
%   it} where the pronoun is captured is \nexteg{}.
% \begin{ex} 
% $\lambda\mathfrak{s}$:\textit{Rec}
%   . \smallrecord{\smalltfield{e}{\record{\tfield{X}{every$^w$(\textbf{dog$^\frown$which$^\frown$chase$^\frown$a$^\frown$cat}($\mathfrak{s}$))}\\
%                            \tfield{f}{(($x:\mathfrak{T}(\text{X}))\rightarrow$\\
% &&$\neg\mathfrak{P}$(\textbf{catch$^\frown$it}$|^p_{\mathcal{F}(\textbf{dog$^\frown$which$^\frown$chase$^\frown$a$^\frown$cat}(\mathfrak{s})),\text{x}_0\leadsto\text{e}_2.\text{e}.\text{x}}(\mathfrak{s})$)\{x\})}}}} 
% \end{ex} 
% Unpacking \preveg{} yields
% (\ref{ex:no-dog-which-chases-a-cat-catches-it-captured}).

This treatment of donkey anaphora is essentially similar to that of
\cite{Chierchia1995} in that it associates an existential reading the
pronoun it, that is, it is not the case that there is a cat which the
dog chases that it also catches.  The dog does not catch any of the
cats it chases.  If we use \textit{every} instead of \textit{no} we
get a reading which is paraphrased as ``every dog which chases a cat
is a dog which chases a cat and catches it''.  This is what is known
in the literature as a \textit{weak reading} or a
\textit{$\exists$-reading}. It says that every dog which chases a cat
catches \textit{some} cat that it chases (but not necessarily all).
This reading may intuitively not be appropriate for \textit{every dog
  that chases a cat catches it} which for many speakers would suggest
that the dogs catch all the cats they chase.  However, the weak
reading is important for the examples in \nexteg{}.

% Notice that \preveg{}
% forces what is known in the literature as a \textit{strong} reading
% for the donkey anaphora.  That is, none of the dogs which chase a cat
% catch \textit{any} of the cats which they chase.  That monotone
% decreasing quantifiers force such strong readings was noted, for
% example, by [Chierchia???? Pelletier????].  Monotone increasing
% quantifiers, however, allow \textit{weak} readings.  Examples of
% donkey sentences mentioned in the literature that have naturally weak
% readings are given in \nexteg{}.
\begin{ex} 
\begin{subex} 
 
\item Every person who had a dime put it in the parking
  meter. \citep{PelletierSchubert1989} 
 
\item Every man who has a daughter thinks she is the most beautiful
  girl in the world \citep{Cooper1979} 
 
\end{subex} 
   
\end{ex} 
\preveg{a} does not seem to suggest that anybody who had several dimes
put them all in the meter and \preveg{b} does not seem to
commit a man who has two daughters to believe the contradictory
proposition that they are both the one and only most beautiful girl in
the world.
%
% @@
%
% Let us consider the content for \textit{every person who
%   had a dime put it in the parking meter} given in \nexteg{}.
% \begin{ex} 
%  $\lambda\mathfrak{s}$:\textit{Rec} 
%   . \smallrecord{\smalltfield{e}{\record{\tfield{X}{every$^w$(\textbf{person$^\frown$who$^\frown$have$^\frown$a$^\frown$dime}($\mathfrak{s}$))}\\
%                            \tfield{f}{(($x:\mathfrak{T}(\text{X}))\rightarrow$\\
% &&\hspace*{2em}\smallrecord{\smalltfield{$\mathfrak{c}$}{\smallrecord{\smallmfield{x}{$x$}{\textit{Ind}}\\
%                                                                \smalltfield{e$_1$}{person$'$\{x\}}\\
%                                                                \smalltfield{e$_2$}{\smallrecord{\smalltfield{e}{\smallrecord{\smalltfield{x}{$\mathfrak{T}$(dime$'$)}\\
%                                                                               %                  \smalltfield{e}{\smallrecord{\smalltfield{$\mathfrak{c}$}{\smallrecord{\smallmfield{x}{$\Uparrow^2$x}{\textit{Ind}}\\
%                                                                               %                                                                                         \smalltfield{e}{dime(x)}}}\\
%                                                                               %                       \smalltfield{e}{\smallrecord{\smalltfield{e}{have$^\dagger$($\Uparrow^4$x,
%                                                                                                            $\Uparrow\!\mathfrak{c}$.x)}}}}}}}}}}}\\
%                \smalltfield{e}{put\_in\_the\_meter$^\dagger$($x$, $\mathfrak{c}$.e$_2$.e.x)}}
% )}}}} 
% \end{ex} 
% \preveg{} is exactly similar to
% (\ref{ex:no-dog-which-chases-a-cat-catches-it-captured}) modulo the
% properties and predicates involved except that there is no negation in
% \preveg{}.  \preveg{} requires then that every person who had a dime
% is a person who had a dime and put it in the meter, that is, a weak
% reading which requires only that each relevant person put some dime in
% the meter, not all of their dimes. 
What then do we say about the
original donkey sentences like \textit{every farmer who owns a donkey
  likes it} which were analyzed by Geach and the classical analyses in
DRT and type theory as having a strong reading in which every farmer
who owns a donkey likes any donkey that he owns?  One option is to say
that we only need the weak reading for such sentences as it is
consistent with the stronger reading.  Many speakers feel that it
unclear what the sentence means if some man owns more than one donkey.
% \begin{ex} 
%  $\lambda\mathfrak{s}$:\textit{Rec} 
%   . \smallrecord{\smalltfield{e}{\record{\tfield{X}{every$^w$(\textbf{farmer$^\frown$who$^\frown$owns$^\frown$a$^\frown$donkey}($\mathfrak{s}$))}\\
%                            \tfield{f}{(($x:\mathfrak{T}(\text{X}))\rightarrow$\\
% &&\hspace*{2em}\smallrecord{\smalltfield{$\mathfrak{c}$}{\smallrecord{\smallmfield{x}{$x$}{\textit{Ind}}\\
%                                                                \smalltfield{e$_1$}{farmer$'$\{x\}}\\
%                                                                \smalltfield{e$_2$}{\smallrecord{\smalltfield{e}{\smallrecord{\smalltfield{x}{$\mathfrak{T}$(donkey$'$)}\\
%                                                                                                 \smalltfield{e}{\smallrecord{\smalltfield{$\mathfrak{c}$}{\smallrecord{\smallmfield{x}{$\Uparrow^2$x}{\textit{Ind}}\\
%                                                                                                                                                                        \smalltfield{e}{donkey(x)}}}\\
%                                                                                                      \smalltfield{e}{\smallrecord{\smalltfield{e}{have$^\dagger$($\Uparrow^4$x,
%                                                                                                            $\Uparrow\!\mathfrak{c}$.x)}}}}}}}}}}}\\
%                \smalltfield{e}{like$^\dagger$($x$, $\mathfrak{c}$.e$_2$.e.x)}}
% )}}}} 
% \label{ex:donkey-weak} 
% \end{ex} 
That is,
\textit{every farmer who owns a donkey likes it} requires that for
every donkey owning farmer there is at least one donkey the farmer
owns such that she likes it.  This allows for the farmers to like all
their donkeys but does not require it. (See \citealp{Kanazawa1994} for
a discussion of this.)  In the case of \textit{every dog which chases
  a cat catches it} I have the following intuition:
\begin{quote}
  if there is a dog under consideration which is involved in two
  distinct cat-chasing events (with a single cat) and only succeeds in catching the cat in
  one of the two events, then this seems to make the sentence false;

  if there is a dog under consideration which is involved in a single
  event of chasing two cats and only succeeds in catching one of the
  cats in that event, then it seems that the sentence could still be
  true.
\end{quote}
This suggests an analysis which requires that every relevant
cat-chasing event must involve the catching of at least one of the
cats chased in that event.  Firm judgements concerning such intuitions
are notoriously hard to come by.  \cite{Chierchia1995} argues that the
strong $\forall$-reading is necessary because of examples like
\nexteg{}.
\begin{ex} 
Every man who owned a slave owned his offspring 
\end{ex} 
\preveg{} is a modification of an example in \cite{Heim1990} which is
used to make a different argument.  It seems like a single instance of
somebody owning a slave but not the slave's offspring would be
sufficient to falsify \preveg{}.  Though again, in the case of a slave
who has several offspring one of which is owned by somebody else, it
seems to me that it is unclear whether that is sufficient to falsify
the sentence.

\label{pg:donkey-purification-universal}
One option is to recreate the original Geach reading by
using the variant of the purification operation, `$\mathfrak{P}^\forall$', in
(\ref{ex:purification-universal}), p.~\pageref{ex:purification-universal} which introduces a function on local contexts.
% \begin{ex} 
% If $P$ : \textit{Ppty}, then
% \begin{quote}
% if $P$.bg$^x$ = $P$.bg, then
% \begin{quote}
% $\mathfrak{P}(P)=P$
% \end{quote}
% otherwise:
% \begin{quote}
% $\mathfrak{P}(P)$ is $\ulcorner\lambda r_1$:$P$.bg$^{\text{x}}$
% . ($(r_2\!:\!P.\text{bg}\!\parallel\!\!\text{\smallrecord{\field{x}{$r$.x}}})\rightarrow$
% \record{
%           \tfield{e}{$P(r_2)$}})$\urcorner$
% \end{quote}
% \end{quote}
% %\label{ex:purification-classic}
% \end{ex}
Using `$\mathfrak{P}^\forall$' instead of `$\mathfrak{P}$' will have the
consequence that the content of \textit{every farmer who owns a donkey
  likes it} will be \nexteg{}.
\savebox{\boxone}{\textbf{farmer}$^\frown$\textbf{who}$^\frown$\textbf{owns}$^\frown$\textbf{a}$^\frown$\textbf{donkey}}
\savebox{\boxtwo}{\textbf{likes}$^\frown$\textbf{it}}
\begin{ex} 
  $\lambda c$:\textit{Cntxt} . \record{
    \mfield{restr}{\usebox{\boxone}($c$)}{\textit{Ppty}}\\
    \mfield{scope}{$(\mathcal{L}(\text{\usebox{\boxtwo}})(c)|_{\mathcal{F}(\text{\usebox{\boxone}})})_{\mathfrak{c}.\mathfrak{s}.\text{x}_0=\text{e}_2.\text{x}}$}{\textit{Ppty}}\\
    \smalltfield{e}{every(restr, scope)}}
\end{ex} 

Given the general witness condition associated with `every', \preveg{} is
equivalent to \nexteg{}.
\begin{ex} 
  $\lambda c$:\textit{Cntxt} . \record{
    \mfield{restr}{\usebox{\boxone}($c$)}{\textit{Ppty}}\\
    \mfield{scope}{$(\mathcal{L}(\text{\usebox{\boxtwo}})(c)|_{\mathcal{F}(\text{\usebox{\boxone}})})_{\mathfrak{c}.\mathfrak{s}.\text{x}_0=\text{e}_2.\text{x}}$}{\textit{Ppty}}\\
    \smalltfield{e}{\record{
        \tfield{X}{every$^w$(restr)}\\
        \tfield{f}{$((x:\mathfrak{T}(X))\rightarrow\mathfrak{P}(\text{scope})\{x\})$}
      }}}
\end{ex}
If we use a variant of this witness condition which uses
`$\mathfrak{P}^\forall$' instead of `$\mathfrak{P}$' we have
\nexteg{}.
\begin{ex} 
  $\lambda c$:\textit{Cntxt} . \record{
    \mfield{restr}{\usebox{\boxone}($c$)}{\textit{Ppty}}\\
    \mfield{scope}{$(\mathcal{L}(\text{\usebox{\boxtwo}})(c)|_{\mathcal{F}(\text{\usebox{\boxone}})})_{\mathfrak{c}.\mathfrak{s}.\text{x}_0=\text{e}_2.\text{x}}$}{\textit{Ppty}}\\
    \smalltfield{e}{\record{
        \tfield{X}{every$^w$(restr)}\\
        \tfield{f}{$((x:\mathfrak{T}(X))\rightarrow\mathfrak{P}^\forall(\text{scope})\{x\})$}
      }}}
\end{ex}
Let us check that \preveg{} does in fact give us the strong
$\forall$-reading.  We will do this by showing that the value
associated with the label `scope' will be paraphrasable as ``the
property of being an individual such that if it's a farmer who owns a
donkey, she likes that donkey'', that is, likes every donkey she
owns. We can show this by unpacking the expression representing the
scope in \preveg{}.  `\textbf{likes}$^\frown$\textbf{it}', according to
our treatment of free pronouns and extensional verbs, will be
\nexteg{}.
\begin{ex} 
  $\ulcorner\lambda c$:\smallrecord{
    \footnotesize{Cntxt}\\
    \smalltfield{$\mathfrak{s}$}{\smallrecord{
        %\footnotesize{Assgnmnt}\\
        \smalltfield{x$_0$}{\textit{Ind}}}}} .
  $\ulcorner\lambda r$:\smallrecord{
    \smalltfield{x}{\textit{Ind}}} . \record{
    \tfield{e}{like$^\dagger$($r$.x, $c.\mathfrak{s}$.x$_0$)}}$\urcorner\urcorner$
\end{ex}
Using the definition of the localization operation, $\mathcal{L}$, in
(\ref{ex:localization}), we can see that
`$\mathcal{L}(\textbf{likes}^\frown\textbf{it})$' is \nexteg{}.
\begin{ex} 
$\ulcorner\lambda c$:\textit{Cntxt} . $\ulcorner\lambda
r$:\smallrecord{
  \smalltfield{x}{\textit{Ind}}\\
  \smalltfield{$\mathfrak{c}$}{\smallrecord{
      \footnotesize{\textit{Cntxt}}\\
      \smalltfield{$\mathfrak{s}$}{\smallrecord{
          \smalltfield{x$_0$}{\textit{Ind}}}}}}} . \record{
    \tfield{e}{like$^\dagger$($r$.x, $r.\mathfrak{c}.\mathfrak{s}$.x$_0$)}}$\urcorner\urcorner$
\end{ex} 
Applying \preveg{} to any context, $c$, will obtain \nexteg{}.
\begin{ex} 
 $\ulcorner\lambda
r$:\smallrecord{
  \smalltfield{x}{\textit{Ind}}\\
  \smalltfield{$\mathfrak{c}$}{\smallrecord{
      \footnotesize{\textit{Cntxt}}\\
      \smalltfield{$\mathfrak{s}$}{\smallrecord{
          \smalltfield{x$_0$}{\textit{Ind}}}}}}} . \record{
    \tfield{e}{like$^\dagger$($r$.x, $r.\mathfrak{c}.\mathfrak{s}$.x$_0$)}}$\urcorner$
\end{ex} 
Restricting \preveg{} by
`$\mathcal{F}(\textbf{farmer}^\frown\textbf{who}^\frown\textbf{owns}^\frown\textbf{a}^\frown\textbf{donkey})$',
yields \nexteg{}.
\begin{ex} 
$\ulcorner\lambda r$:\smallrecord{
    \smalltfield{x}{\textit{Ind}}\\
    \smalltfield{$\mathfrak{c}$}{\smallrecord{
        \footnotesize{\textit{Cntxt}}\\
        \smalltfield{$\mathfrak{s}$}{\smallrecord{
            %\footnotesize{\textit{Assgnmnt}}\\
            \smalltfield{x$_0$}{\textit{Ind}}}}\\
        \smalltfield{$\mathfrak{c}$}{\smallrecord{
            \smalltfield{f}{\textit{PropCntxt}}\\
            \smalltfield{a}{\textit{PropCntxt}}}}
}}\\
    \smalltfield{e$_1$}{dog(x)}\\
    \smalltfield{e$_2$}{\smallrecord{
        \smalltfield{x}{$\mathfrak{T}$(cat$'$)}\\
        \smalltfield{e}{chase$^\dagger$($\Uparrow$x, x)}}}}
   .
  \record{
    \tfield{e}{catch$^{\dagger}$($r$.x, $r.\mathfrak{c}.\mathfrak{s}.\text{x}_0$)}}$\urcorner$
\end{ex}



We can paraphrase this as ``the property of being a dog which chases a
cat and there is something which it catches''.  In order to obtain the
anaphora we need to align the following two paths in the domain type
of this function: `$\mathfrak{c}.\mathfrak{s}.\text{x}_0$' and
`e$_2$.x'.  This we do in \nexteg{} by creating a manifest field on
the former path.

\begin{ex} 
  $\ulcorner\lambda r$:\smallrecord{
    \smalltfield{x}{\textit{Ind}}\\
    \smalltfield{$\mathfrak{c}$}{\smallrecord{
        \footnotesize{\textit{Cntxt}}\\
        \smalltfield{$\mathfrak{s}$}{\smallrecord{
            %\footnotesize{\textit{Assgnmnt}}\\
            \smallmfield{x$_0$}{$\Uparrow^2$e$_2$.x}{\textit{Ind}}}}\\
        \smalltfield{$\mathfrak{c}$}{\smallrecord{
            \smalltfield{f}{\textit{PropCntxt}}\\
            \smalltfield{a}{\textit{PropCntxt}}}}
}}\\
    \smalltfield{e$_1$}{dog(x)}\\
    \smalltfield{e$_2$}{\smallrecord{
        \smalltfield{x}{$\mathfrak{T}$(cat$'$)}\\
        \smalltfield{e}{chase$^\dagger$($r$.x, x)}}}}
   .
  \record{
    \tfield{e}{catch$^{\dagger}$($r$.x, $r.\mathfrak{c}.\mathfrak{s}.\text{x}_0$)}}$\urcorner$
\end{ex}
We can paraphrase this as ``the property of being a dog which chases a
cat and catches that cat''.

In order to include such properties with aligned paths as the scope of
quantifiers in our interpretations, we will first generalize the
characterization of alignment of paths given on
p.~\pageref{pg:path-alignment-types}f to functions in \nexteg{}.
\begin{ex} 
  If $\varphi=\ulcorner\lambda r\!:\!T\ .\ \psi\urcorner$ and
  $\pi_1,\pi_2\in\mathrm{paths}(T)$, then
  \begin{quote}
    $\varphi_{\pi_1=\pi_2}=\ulcorner\lambda r\!:\!T_{\pi_1=\pi_2}\ .\
    \psi\urcorner$
  \end{quote}
\end{ex} 
We then add two further clauses to the characterization of witnesses
for $\mathfrak{S}(T)$ for content types, $T$, in \nexteg{} with the
new additions boxed.
\begin{ex} 
\begin{subex} 
 
\item If $T:\textit{ContType}$, then $\mathfrak{S}(T)$ is a type 
 
\item The witnesses of $\mathfrak{S}(T)$ are characterized by
  \begin{enumerate} 
 
  \item if $\varphi:T$ then $\varphi:\mathfrak{S}(T)$
    
  \item \fbox{if $\varphi:\mathfrak{S}(T)$ and $\varphi:\textit{PPpty}$,
      then $\mathcal{L}(\varphi):\mathfrak{S}(T)$}
    
  \item \fbox{\begin{minipage}[t]{.9\linewidth}if $\varphi:\mathfrak{S}(T)$,
    $\varphi\sqsubseteq\text{\smallrecord{
        \smallmfield{scope}{$\psi$}{\textit{Ppty}}}}$ and
    $\pi_1,\pi_2\in\mathrm{paths}(\psi.\text{bg})$, then $\varphi[\text{scope}=\psi_{\pi_1=\pi_2}:\textit{Ppty}]:\mathfrak{S}(T)$\end{minipage}}

    
  \item if $\alpha\mathcal{O}\beta:T$, (for
      some combination operation, $\mathcal{O}$) and
      $\alpha\mathcal{O}_{i,j}\beta$ is defined (for some natural
      numbers, $i$ and $j$), then $\alpha\mathcal{O}_{i,j}\beta:\mathfrak{S}(T)$
 
  \item if $\varphi:T$ and $\varphi$ is in the range of `$\mathrm{storage}$', then
    $\mathrm{storage}(\varphi):\mathfrak{S}(T)$

  \item if $\varphi:T$ and `x$_i$' and $\varphi$ are appropriate
    arguments to `$\mathrm{retrieve}$', then
    $\mathrm{retrieve}(\text{x}_i,\varphi):\mathfrak{S}(T)$
    
  \item nothing is a witness for $\mathfrak{S}(T)$ except as required above.
 
  \end{enumerate} 
  
 
\end{subex} 
\label{ex:storage-donkey-type}   
\end{ex}



Overall we can get a parametric content for \textit{no dog which
  chases a cat catches it} which looks like \nexteg{} where
\savebox{\boxone}{\textbf{dog}$^\frown$\textbf{which}$^\frown$\textbf{chases}$^\frown$\textbf{a}$^\frown$\textbf{cat}} \usebox{\boxone}
and \savebox{\boxtwo}{\textbf{catches}$^\frown$\textbf{it}}\usebox{\boxtwo} represent parametric contents
which have not undergone operations introduced by $\mathfrak{S}$.
\begin{ex} 
  $\lambda c$:\textit{Cntxt} . \record{
    \mfield{restr}{\usebox{\boxone}($c$)}{\textit{Ppty}}\\
    \mfield{scope}{$(\mathcal{L}(\text{\usebox{\boxtwo}})(c)|_{\mathcal{F}(\text{\usebox{\boxone}})})_{\mathfrak{c}.\mathfrak{s}.\text{x}_0=\text{e}_2.\text{x}}$}{\textit{Ppty}}\\
    \smalltfield{e}{no(restr, scope)}}
\end{ex} 
Given the semantics we have specified for the determiner \textit{no},
\preveg{} is \nexteg{}.
\begin{ex} 
  $\lambda c$:\textit{Cntxt} . \record{
    \mfield{restr}{\usebox{\boxone}($c$)}{\textit{Ppty}}\\
    \mfield{scope}{$(\mathcal{L}(\text{\usebox{\boxtwo}})(c)|_{\mathcal{F}(\text{\usebox{\boxone}})})_{\mathfrak{c}.\mathfrak{s}.\text{x}_0=\text{e}_2.\text{x}}$}{\textit{Ppty}}\\
    \smalltfield{e}{\record{
        \tfield{X}{every$^w$(restr)}\\
        \tfield{f}{$((x:\mathfrak{T}(X))\rightarrow\neg\mathfrak{P}(\text{scope})\{x\})$}
      }}}
\end{ex} 
We can paraphrase this content as ``for every dog which chases a cat
it's not the case that it's a dog which chases a cat$_i$ and catches
it$_i$'' where the subscript on \textit{cat} and \textit{it} indicates
that \textit{it} is anaphorically related to \textit{a cat}.


% The basic content related to \textit{no} is given in \nexteg{}.
% \begin{ex} 
% $\lambda\mathfrak{s}$:\textit{Rec} $\lambda Q$:\textit{Ppty} $\lambda
% P$:\textit{Ppty} . \record{\tfield{e}{no($Q$,$P|_{\mathcal{F}(Q)}$)}} 
% \end{ex} 
% As usual we abbreviate the content of \textit{dog} as `dog$'$'.  The
% relative pronoun \textit{which} has the content specified for
% \textit{who} in Section~\ref{sec:long-distance}.  (We are simplifying
% by not accounting for the gender distinction between the two.)  We
% repeat this in \nexteg{} and will refer to it here as \textbf{which}.
% \begin{ex} 
% $\lambda\mathfrak{s}$:\smallrecord{\smalltfield{wh}{\textit{Ind}}}
%   . $\lambda P$:\textit{Ppty} . $P$\{$\mathfrak{s}$.wh\} 
% \end{ex} 
% The basic content of an utterance of \textit{chase} is given in
% \nexteg{}.
% \begin{ex} 
% $\lambda\mathfrak{s}$:\textit{Rec} $\lambda
% r_2$:\smallrecord{\smalltfield{x}{\textit{Quant}}} $\lambda
% r_1$:\smallrecord{\smalltfield{x}{\textit{Ind}}}
%   . \record{\tfield{e}{chase($r_1$.x, $r_2$.x)}} 
% \end{ex} 
% The indefinite article $a$ will have the content in \nexteg{}.
% \begin{ex} 
% $\lambda\mathfrak{s}$:\textit{Rec} $\lambda Q$:\textit{Ppty} $\lambda
% P$:\textit{Ppty} . \record{\tfield{e}{exist($Q$, $P|_{\mathcal{F}(Q)}$)}} 
% \end{ex} 
% Thus the phrase \textit{a cat} will have the content in \nexteg{}.
% \begin{ex} 
% $\lambda\mathfrak{s}$:\textit{Rec} $\lambda P$:\textit{Ppty}
% . \record{\tfield{e}{exist(cat$'$, $P|_{\mathcal{F}(\text{cat}')}$)}} 
% \end{ex} 
% We will refer to \preveg{} as \textbf{a$^\frown$cat}.  Given this the
% content of \textit{chase a cat} will be \nexteg{}.
% \begin{ex} 
% $\lambda\mathfrak{s}$:\textit{Rec} $\lambda
% r_1$:\smallrecord{\smalltfield{x}{\textit{Ind}}}
% . \record{\tfield{e}{chase($r_1$.x,
%     \textbf{a$^\frown$cat}($\mathfrak{s}$))}}
% \label{ex:chaseacat-noexport} 
% \end{ex} 
% Given that \textit{chase} is an extensional verb it will obey a
% constraint like (\ref{ex:mp-find}) on p.~\pageref{ex:mp-find}, as
% given in \nexteg{}.
% \begin{ex} 
% $e$ : chase($a$, $Q$) iff $e$ : $Q$($\lambda
% r$:\smallrecord{\smalltfield{x}{\textit{Ind}}} . \smallrecord{\smalltfield{e}{chase$^{\dagger}$($a$,
% $r$.x)}}) 
% \end{ex} 
% This means that we can construe the content to be a function which
% returns an equivalent type to that returned in
% (\ref{ex:chaseacat-noexport}).  This new content is given in
% \nexteg{}.
% \begin{ex} 
% $\lambda s$:\textit{Rec} $\lambda
% r_1$:\smallrecord{\smalltfield{x}{\textit{Ind}}}
%   . \record{\tfield{e}{\textbf{a$^\frown$cat}($\mathfrak{s}$)($\lambda
%       r$:\smallrecord{\smalltfield{x}{\textit{Ind}}}
%       . \smallrecord{\smalltfield{e}{chase$^\dagger$($r_1$.x, $r$.x)}})}} 
% \end{ex} 
% \preveg{} is equivalent to \nexteg{}.
% \begin{ex} 
% $\lambda\mathfrak{s}$:\textit{Rec} $\lambda
% r_1$:\smallrecord{\smalltfield{x}{\textit{Ind}}}
% . \record{\tfield{e}{exist(cat$'$, $\lambda
%     r$:\smallrecord{\smalltfield{x}{\textit{Ind}}\\
%                     \smalltfield{e}{cat(x)}}
%                   . \smallrecord{\smalltfield{e}{chase$^\dagger$($r_1$.x, $r$.x)}})}} 
% \end{ex} 
% In turn, \preveg{} is equivalent to \nexteg{}.
% \begin{ex} 
% $\lambda\mathfrak{s}$:\textit{Rec} $\lambda
% r_1$:\smallrecord{\smalltfield{x}{\textit{Ind}}}
%   . \record{\tfield{e}{\record{\tfield{x}{$\mathfrak{T}$(cat$'$)}\\
%                                \tfield{e}{\record{\tfield{$\mathfrak{c}$}{\smallrecord{\smallmfield{x}{$\Uparrow^2$x}{\textit{Ind}}\\
%                                                    \smalltfield{e}{cat(x)}}}\\
%                    \tfield{e}{\smallrecord{\smalltfield{e}{chase$^\dagger$($r_1$.x,
%                          $\Uparrow\!\mathfrak{c}$.x)}}}}}}}}
% \label{ex:chase-a-cat} 
% \end{ex} 
% We represent \preveg{} as \textbf{chase$^\frown$a$^\frown$cat}.  From
% this we can form a ``sentence with a gap'' interpretation,
% \textbf{chase$^\frown$a$^\frown$cat$_S$}, in the manner described in
% Section~\ref{sec:long-distance}.  This is given in \nexteg{}.
% \begin{ex} 
% $\lambda\mathfrak{s}$:\smallrecord{\smalltfield{wh$_0$}{\textit{Ind}}} . $\lambda
% P$:\textit{Ppty} . $P$\{$\mathfrak{s}$.wh$_0$\}(\textbf{chase$^\frown$a$^\frown$cat}($\mathfrak{s}$)) 
% \end{ex} 
% Unpacking \preveg{} we obtain \nexteg{}.
% \begin{ex} 
% $\lambda\mathfrak{s}$:\smallrecord{\smalltfield{wh$_0$}{\textit{Ind}}}
% . \record{\tfield{e}{\record{\tfield{x}{$\mathfrak{T}$(cat$'$)}\\
%                                \tfield{e}{\record{\tfield{$\mathfrak{c}$}{\smallrecord{\smallmfield{x}{$\Uparrow^2$x}{\textit{Ind}}\\
%                                                    \smalltfield{e}{cat(x)}}}\\
%                    \tfield{e}{\smallrecord{\smalltfield{e}{chase$^\dagger$($\mathfrak{s}$.wh$_0$, $\Uparrow\!\mathfrak{c}$.x)}}}}}}}} 
% \end{ex} 
% Again following the analysis of long-distance dependencies developed
% in Section~\ref{sec:long-distance}, the content of \textit{which
%   chased a cat} is given in \nexteg{}.
% \begin{ex} 
% $\lambda\mathfrak{s}$:\textit{Rec} . \\
% \hspace*{1em} $\lambda
% r_1$:\smallrecord{\smalltfield{x}{\textit{Ind}}}
% . \textbf{which}($\mathfrak{s}\oplus[\text{wh}=r_1.\text{x}$)($\lambda
% r_2$:\smallrecord{\smalltfield{x}{\textit{Ind}}} . \textbf{chase$^\frown$a$^\frown$cat$_S$}($\mathfrak{s}\oplus[\text{wh}_0=r_2.\text{x}]$) 
% \end{ex} 
% Unpacking \preveg{} gives us (\ref{ex:chase-a-cat}).  That is, the content of
% \textit{which chased a cat} is identical with the content of
% \textit{chased a cat}.  For clarity we will represent this content
% also as \textbf{which$^\frown$chase$^\frown$a$^\frown$cat}.
% Again following the proposal in Section~\ref{sec:long-distance} the
% content for \textit{dog which chases a cat} will be
% \nexteg{}. 
% \begin{ex} 
% $\lambda\mathfrak{s}$:\textit{Rec} $\lambda
% r$:\smallrecord{\smalltfield{x}{\textit{Ind}}}
% . \record{\tfield{e$_1$}{dog$'$\{$r$.x\}}\\
%           \tfield{e$_2$}{\textbf{which$^\frown$chase$^\frown$a$^\frown$cat}($\mathfrak{s}$)\{$r$.x\}}} 
% \end{ex} 
% We call this
% \textbf{dog$^\frown$which$^\frown$chase$^\frown$a$^\frown$cat}. Now
% \textit{no dog which chases a cat} will correspond to \nexteg{a} which
% is equivalent to \nexteg{b}.
% \begin{ex} 
% \begin{subex} 
 
% \item $\lambda\mathfrak{s}$:\textit{Rec} $\lambda P$:\textit{Ppty}
%   . \record{\tfield{e}{no(\textbf{dog$^\frown$which$^\frown$chase$^\frown$a$^\frown$cat}($\mathfrak{s}$),
%       $P|_{\mathcal{F}(\textbf{dog}^\frown\textbf{which}^\frown\textbf{chase}^\frown
%           \textbf{a}^\frown\textbf{cat}(\mathfrak{s}))}$)}} 
 
% \item $\lambda\mathfrak{s}$:\textit{Rec} $\lambda P$:\textit{Ppty}
%   . \record{\tfield{e}{\record{\tfield{X}{every$^w$(\textbf{dog$^\frown$which$^\frown$chase$^\frown$a$^\frown$cat}($\mathfrak{s}$))}\\
%                            \tfield{f}{(($x:\mathfrak{T}(\text{X}))\rightarrow\neg\mathfrak{P}(P|_{\mathcal{F}(\textbf{dog$^\frown$which$^\frown$chase$^\frown$a$^\frown$cat}(\mathfrak{s}))})\{\text{x}\}$)}}}} 
 
% \end{subex} 
% \label{ex:no-dog-which-chases-a-cat}   
% \end{ex} 
% Call this
% \textbf{no$^\frown$dog$^\frown$which$^\frown$chase$^\frown$a$^\frown$cat}.
% This is to combine with the content of \textit{catches it}, given in
% \nexteg{}. 
% \begin{ex} 
% $\lambda\mathfrak{s}$:\smallrecord{\smalltfield{x$_0$}{\textit{Ind}}}
% $\lambda r$:\smallrecord{\smalltfield{x}{\textit{Ind}}}
% . \record{\tfield{e}{catch$^\dagger$($r$.x, $\mathfrak{s}$.x$_0$)}} 
% \end{ex}
% We will represent this as \textbf{catch$^\frown$it}. 
% Using the S-combinator strategy to
% (\ref{ex:no-dog-which-chases-a-cat}b) yields \nexteg{a}, equivalently
% \nexteg{b}, which represents a reading of the sentence where
% \textit{it} is not captured and refers to a particular object provided
% by the context.
% \begin{ex}
% \begin{subex} 
 
% \item $\lambda\mathfrak{s}$:\smallrecord{\smalltfield{x$_0$}{\textit{Ind}}} 
%   . \smallrecord{\smalltfield{e}{\record{\tfield{X}{every$^w$(\textbf{dog$^\frown$which$^\frown$chase$^\frown$a$^\frown$cat}($\mathfrak{s}$))}\\
%                            \tfield{f}{(($x:\mathfrak{T}(\text{X}))\rightarrow$\\
% &&$\neg\mathfrak{P}(\lambda r$:\smallrecord{\smalltfield{x}{\textit{Ind}}}
% . \record{\tfield{e}{catch$^\dagger$($r$.x, $\mathfrak{s}$.x$_0$)}}$|_{\mathcal{F}(\textbf{dog$^\frown$which$^\frown$chase$^\frown$a$^\frown$cat}(\mathfrak{s}))})\{\text{x}\}$)}}}} 
 
% \item $\lambda\mathfrak{s}$:\smallrecord{\smalltfield{x$_0$}{\textit{Ind}}} 
%   . \smallrecord{\smalltfield{e}{\record{\tfield{X}{every$^w$(\textbf{dog$^\frown$which$^\frown$chase$^\frown$a$^\frown$cat}($\mathfrak{s}$))}\\
%                            \tfield{f}{(($x:\mathfrak{T}(\text{X}))\rightarrow$\\
% &&\hspace*{2em}$\neg$\smallrecord{\smalltfield{$\mathfrak{c}$}{\smallrecord{\smallmfield{x}{$x$}{\textit{Ind}}\\
%                                                                \smalltfield{e$_1$}{dog$'$\{x\}}\\
%                                                                \smalltfield{e$_2$}{\smallrecord{\smalltfield{e}{\smallrecord{\smalltfield{x}{$\mathfrak{T}$(cat$'$)}\\
%                                                                                                 \smalltfield{e}{\smallrecord{\smalltfield{$\mathfrak{c}$}{\smallrecord{\smallmfield{x}{$\Uparrow^2$x}{\textit{Ind}}\\
%                                                                                                                                                                        \smalltfield{e}{cat(x)}}}\\
%                                                                                                      \smalltfield{e}{\smallrecord{\smalltfield{e}{chase$^\dagger$($\Uparrow^4$x,
%                                                                                                            $\Uparrow\!\mathfrak{c}$.x)}}}}}}}}}}}\\
%                \smalltfield{e}{catch$^\dagger$($x$, $\mathfrak{s}$.x$_0$)}}
% )}}}} 
 
% \end{subex} 
   
 
% \end{ex} 
% In terms of the notation in \preveg{b}, it is simple enough to see
% what needs to be done in order to change it to a case of donkey
% anaphora.  The second argument to `catch$^\dagger$',
% `$\mathfrak{s}$.x$_0$', has to be changed so that it presents a path
% to the cat, that is, `$\mathfrak{c}$.e$_2$.e.x'.  Also the dependence
% of this parametric content on `x$_0$' has to be removed.  Thus the
% domain type for the $\lambda$-abstraction, currently
% `\smallrecord{\smalltfield{x$_0$}{\textit{Ind}}}', should be replaced
% by `\textit{Rec}'.  The result would be \nexteg{}.
% \begin{ex} 
%  $\lambda\mathfrak{s}$:\textit{Rec} 
%   . \smallrecord{\smalltfield{e}{\record{\tfield{X}{every$^w$(\textbf{dog$^\frown$which$^\frown$chase$^\frown$a$^\frown$cat}($\mathfrak{s}$))}\\
%                            \tfield{f}{(($x:\mathfrak{T}(\text{X}))\rightarrow$\\
% &&\hspace*{2em}$\neg$\smallrecord{\smalltfield{$\mathfrak{c}$}{\smallrecord{\smallmfield{x}{$x$}{\textit{Ind}}\\
%                                                                \smalltfield{e$_1$}{dog$'$\{x\}}\\
%                                                                \smalltfield{e$_2$}{\smallrecord{\smalltfield{e}{\smallrecord{\smalltfield{x}{$\mathfrak{T}$(cat$'$)}\\
%                                                                                                 \smalltfield{e}{\smallrecord{\smalltfield{$\mathfrak{c}$}{\smallrecord{\smallmfield{x}{$\Uparrow^2$x}{\textit{Ind}}\\
%                                                                                                                                                                        \smalltfield{e}{cat(x)}}}\\
%                                                                                                      \smalltfield{e}{\smallrecord{\smalltfield{e}{chase$^\dagger$($\Uparrow^4$x,
%                                                                                                            $\Uparrow\!\mathfrak{c}$.x)}}}}}}}}}}}\\
%                \smalltfield{e}{catch$^\dagger$($x$, $\mathfrak{c}$.e$_2$.e.x)}}
% )}}}}
% \label{ex:no-dog-which-chases-a-cat-catches-it-captured} 
% \end{ex} 
% We will achieve this by introducing a notion of restriction on a
% parametric property in which we will allow anaphora to take place.
% Let us first consider a case without anaphora.  Suppose that
% $\mathcal{P}$ is a parametric property of type
% $(T_1\rightarrow(T_2\rightarrow \textit{Type}))$ and that $T$ is a
% type then we define the restriction of $\mathcal{P}$ by $T$,
% $\mathcal{P}|^p_T$, to be \nexteg{}.
% \begin{ex} 
% $\lambda\mathfrak{s}$:$T_1$ $\lambda r$:$T_2$\d{$\wedge$}$T$ . $\mathcal{P}(\mathfrak{s})(r)$ 
% \end{ex} 
% Now suppose that $[\ell,v]$ is a field in $T_1$ and $\pi$ is a path in
% $T_2$ and we want to make an anaphoric association between $\ell$ and
% $\pi$.  We define such a restriction,
% $\mathcal{P}|^p_{T,\ell\leadsto\pi}$ as \nexteg{}.
% \begin{ex} 
% $\lambda\mathfrak{s}$:$T_1\ominus[\ell:v]$ $\lambda
% r$:$T_2$\d{$\wedge$}$T$ . $\mathcal{P}(\mathfrak{s}\oplus[\ell=r.\pi])(r)$ 
% \end{ex}
% To enable the capture of several pronouns, we let $\vec\Pi\subseteq
% T_1\times\text{paths}(T)$ such that $\vec\Pi$ is the graph of a one-one
% function (whose domain is included in the fields of $T_1$ and whose
% range is included in the paths of $T$).  We then define
% $\mathcal{P}|^p_{T,\vec{\Pi}}$ to be \nexteg{}.
% \begin{ex} 
% $\lambda\mathfrak{s}$:$T_1\ominus_{\text{set}}\vec{\Pi}_1$ $\lambda
% r$:$T_2$\d{$\wedge$}$T$ . $\mathcal{P}(\mathfrak{s}\oplus_{\text{set}}\{[\text{label}(\vec{\pi}_1)=r.\vec{\pi}_2]\mid\vec{\pi}\in\vec{\Pi}\})$ 
% \end{ex} 
% If $\vec{\Pi}$ is
% $\{\langle\langle\ell_1,v_1\rangle,\pi_1\rangle,\ldots,\langle\langle\ell_n,v_n\rangle,\pi_n\rangle\}$
% then for convenience we represent $\mathcal{P}|^p_{T,\vec{\Pi}}$ as
% $\mathcal{P}|^p_{T,\ell_1\leadsto\pi_1,\ldots,\ell_n\leadsto\pi_n}$. Using
% this notation,  the content of \textit{no dog which chases a cat catches
%   it} where the pronoun is captured is \nexteg{}.
% \begin{ex} 
% $\lambda\mathfrak{s}$:\textit{Rec}
%   . \smallrecord{\smalltfield{e}{\record{\tfield{X}{every$^w$(\textbf{dog$^\frown$which$^\frown$chase$^\frown$a$^\frown$cat}($\mathfrak{s}$))}\\
%                            \tfield{f}{(($x:\mathfrak{T}(\text{X}))\rightarrow$\\
% &&$\neg\mathfrak{P}$(\textbf{catch$^\frown$it}$|^p_{\mathcal{F}(\textbf{dog$^\frown$which$^\frown$chase$^\frown$a$^\frown$cat}(\mathfrak{s})),\text{x}_0\leadsto\text{e}_2.\text{e}.\text{x}}(\mathfrak{s})$)\{x\})}}}} 
% \end{ex} 
% Unpacking \preveg{} yields
% (\ref{ex:no-dog-which-chases-a-cat-catches-it-captured}).

This treatment of donkey anaphora is essentially similar to that of
\cite{Chierchia1995} in that it associates an existential reading the
pronoun it, that is, it is not the case that there is a cat which the
dog chases that it also catches.  The dog does not catch any of the
cats it chases.  If we use \textit{every} instead of \textit{no} we
get a reading which is paraphrased as ``every dog which chases a cat
is a dog which chases a cat and catches it''.  This is what is known
in the literature as a \textit{weak reading} or a
\textit{$\exists$-reading}. It says that every dog which chases a cat
catches \textit{some} cat that it chases (but not necessarily all).
This reading may intuitively not be appropriate for \textit{every dog
  that chases a cat catches it} which for many speakers would suggest
that the dogs catch all the cats they chase.  However, the weak
reading is important for the examples in \nexteg{}.

% Notice that \preveg{}
% forces what is known in the literature as a \textit{strong} reading
% for the donkey anaphora.  That is, none of the dogs which chase a cat
% catch \textit{any} of the cats which they chase.  That monotone
% decreasing quantifiers force such strong readings was noted, for
% example, by [Chierchia???? Pelletier????].  Monotone increasing
% quantifiers, however, allow \textit{weak} readings.  Examples of
% donkey sentences mentioned in the literature that have naturally weak
% readings are given in \nexteg{}.
\begin{ex} 
\begin{subex} 
 
\item Every person who had a dime put it in the parking
  meter. \citep{PelletierSchubert1989} 
 
\item Every man who has a daughter thinks she is the most beautiful
  girl in the world \citep{Cooper1979} 
 
\end{subex} 
   
\end{ex} 
\preveg{a} does not seem to suggest that anybody who had several dimes
put them all in the meter and \preveg{b} does not seem to
commit a man who has two daughters to believe the contradictory
proposition that they are both the one and only most beautiful girl in
the world.
%
% @@
%
% Let us consider the content for \textit{every person who
%   had a dime put it in the parking meter} given in \nexteg{}.
% \begin{ex} 
%  $\lambda\mathfrak{s}$:\textit{Rec} 
%   . \smallrecord{\smalltfield{e}{\record{\tfield{X}{every$^w$(\textbf{person$^\frown$who$^\frown$have$^\frown$a$^\frown$dime}($\mathfrak{s}$))}\\
%                            \tfield{f}{(($x:\mathfrak{T}(\text{X}))\rightarrow$\\
% &&\hspace*{2em}\smallrecord{\smalltfield{$\mathfrak{c}$}{\smallrecord{\smallmfield{x}{$x$}{\textit{Ind}}\\
%                                                                \smalltfield{e$_1$}{person$'$\{x\}}\\
%                                                                \smalltfield{e$_2$}{\smallrecord{\smalltfield{e}{\smallrecord{\smalltfield{x}{$\mathfrak{T}$(dime$'$)}\\
%                                                                               %                  \smalltfield{e}{\smallrecord{\smalltfield{$\mathfrak{c}$}{\smallrecord{\smallmfield{x}{$\Uparrow^2$x}{\textit{Ind}}\\
%                                                                               %                                                                                         \smalltfield{e}{dime(x)}}}\\
%                                                                               %                       \smalltfield{e}{\smallrecord{\smalltfield{e}{have$^\dagger$($\Uparrow^4$x,
%                                                                                                            $\Uparrow\!\mathfrak{c}$.x)}}}}}}}}}}}\\
%                \smalltfield{e}{put\_in\_the\_meter$^\dagger$($x$, $\mathfrak{c}$.e$_2$.e.x)}}
% )}}}} 
% \end{ex} 
% \preveg{} is exactly similar to
% (\ref{ex:no-dog-which-chases-a-cat-catches-it-captured}) modulo the
% properties and predicates involved except that there is no negation in
% \preveg{}.  \preveg{} requires then that every person who had a dime
% is a person who had a dime and put it in the meter, that is, a weak
% reading which requires only that each relevant person put some dime in
% the meter, not all of their dimes. 
What then do we say about the
original donkey sentences like \textit{every farmer who owns a donkey
  likes it} which were analyzed by Geach and the classical analyses in
DRT and type theory as having a strong reading in which every farmer
who owns a donkey likes any donkey that he owns?  One option is to say
that we only need the weak reading for such sentences as it is
consistent with the stronger reading.  Many speakers feel that it
unclear what the sentence means if some man owns more than one donkey.
% \begin{ex} 
%  $\lambda\mathfrak{s}$:\textit{Rec} 
%   . \smallrecord{\smalltfield{e}{\record{\tfield{X}{every$^w$(\textbf{farmer$^\frown$who$^\frown$owns$^\frown$a$^\frown$donkey}($\mathfrak{s}$))}\\
%                            \tfield{f}{(($x:\mathfrak{T}(\text{X}))\rightarrow$\\
% &&\hspace*{2em}\smallrecord{\smalltfield{$\mathfrak{c}$}{\smallrecord{\smallmfield{x}{$x$}{\textit{Ind}}\\
%                                                                \smalltfield{e$_1$}{farmer$'$\{x\}}\\
%                                                                \smalltfield{e$_2$}{\smallrecord{\smalltfield{e}{\smallrecord{\smalltfield{x}{$\mathfrak{T}$(donkey$'$)}\\
%                                                                                                 \smalltfield{e}{\smallrecord{\smalltfield{$\mathfrak{c}$}{\smallrecord{\smallmfield{x}{$\Uparrow^2$x}{\textit{Ind}}\\
%                                                                                                                                                                        \smalltfield{e}{donkey(x)}}}\\
%                                                                                                      \smalltfield{e}{\smallrecord{\smalltfield{e}{have$^\dagger$($\Uparrow^4$x,
%                                                                                                            $\Uparrow\!\mathfrak{c}$.x)}}}}}}}}}}}\\
%                \smalltfield{e}{like$^\dagger$($x$, $\mathfrak{c}$.e$_2$.e.x)}}
% )}}}} 
% \label{ex:donkey-weak} 
% \end{ex} 
That is,
\textit{every farmer who owns a donkey likes it} requires that for
every donkey owning farmer there is at least one donkey the farmer
owns such that she likes it.  This allows for the farmers to like all
their donkeys but does not require it. (See \citealp{Kanazawa1994} for
a discussion of this.)  In the case of \textit{every dog which chases
  a cat catches it} I have the following intuition:
\begin{quote}
  if there is a dog under consideration which is involved in two
  distinct cat-chasing events (with a single cat) and only succeeds in catching the cat in
  one of the two events, then this seems to make the sentence false;

  if there is a dog under consideration which is involved in a single
  event of chasing two cats and only succeeds in catching one of the
  cats in that event, then it seems that the sentence could still be
  true.
\end{quote}
This suggests an analysis which requires that every relevant
cat-chasing event must involve the catching of at least one of the
cats chased in that event.  Firm judgements concerning such intuitions
are notoriously hard to come by.  \cite{Chierchia1995} argues that the
strong $\forall$-reading is necessary because of examples like
\nexteg{}.
\begin{ex} 
Every man who owned a slave owned his offspring 
\end{ex} 
\preveg{} is a modification of an example in \cite{Heim1990} which is
used to make a different argument.  It seems like a single instance of
somebody owning a slave but not the slave's offspring would be
sufficient to falsify \preveg{}.  Though again, in the case of a slave
who has several offspring one of which is owned by somebody else, it
seems to me that it is unclear whether that is sufficient to falsify
the sentence.

\label{pg:donkey-purification-universal}
One option is to recreate the original Geach reading by
using the variant of the purification operation, `$\mathfrak{P}^\forall$', in
(\ref{ex:purification-universal}), p.~\pageref{ex:purification-universal} which introduces a function on local contexts.
% \begin{ex} 
% If $P$ : \textit{Ppty}, then
% \begin{quote}
% if $P$.bg$^x$ = $P$.bg, then
% \begin{quote}
% $\mathfrak{P}(P)=P$
% \end{quote}
% otherwise:
% \begin{quote}
% $\mathfrak{P}(P)$ is $\ulcorner\lambda r_1$:$P$.bg$^{\text{x}}$
% . ($(r_2\!:\!P.\text{bg}\!\parallel\!\!\text{\smallrecord{\field{x}{$r$.x}}})\rightarrow$
% \record{
%           \tfield{e}{$P(r_2)$}})$\urcorner$
% \end{quote}
% \end{quote}
% %\label{ex:purification-classic}
% \end{ex}
Using `$\mathfrak{P}^\forall$' instead of `$\mathfrak{P}$' will have the
consequence that the content of \textit{every farmer who owns a donkey
  likes it} will be \nexteg{}.
\savebox{\boxone}{\textbf{farmer}$^\frown$\textbf{who}$^\frown$\textbf{owns}$^\frown$\textbf{a}$^\frown$\textbf{donkey}}
\savebox{\boxtwo}{\textbf{likes}$^\frown$\textbf{it}}
\begin{ex} 
  $\lambda c$:\textit{Cntxt} . \record{
    \mfield{restr}{\usebox{\boxone}($c$)}{\textit{Ppty}}\\
    \mfield{scope}{$(\mathcal{L}(\text{\usebox{\boxtwo}})(c)|_{\mathcal{F}(\text{\usebox{\boxone}})})_{\mathfrak{c}.\mathfrak{s}.\text{x}_0=\text{e}_2.\text{x}}$}{\textit{Ppty}}\\
    \smalltfield{e}{every(restr, scope)}}
\end{ex} 

Given the general witness condition associated with `every', \preveg{} is
equivalent to \nexteg{}.
\begin{ex} 
  $\lambda c$:\textit{Cntxt} . \record{
    \mfield{restr}{\usebox{\boxone}($c$)}{\textit{Ppty}}\\
    \mfield{scope}{$(\mathcal{L}(\text{\usebox{\boxtwo}})(c)|_{\mathcal{F}(\text{\usebox{\boxone}})})_{\mathfrak{c}.\mathfrak{s}.\text{x}_0=\text{e}_2.\text{x}}$}{\textit{Ppty}}\\
    \smalltfield{e}{\record{
        \tfield{X}{every$^w$(restr)}\\
        \tfield{f}{$((x:\mathfrak{T}(X))\rightarrow\mathfrak{P}(\text{scope})\{x\})$}
      }}}
\end{ex}
If we use a variant of this witness condition which uses
`$\mathfrak{P}^\forall$' instead of `$\mathfrak{P}$' we have
\nexteg{}.
\begin{ex} 
  $\lambda c$:\textit{Cntxt} . \record{
    \mfield{restr}{\usebox{\boxone}($c$)}{\textit{Ppty}}\\
    \mfield{scope}{$(\mathcal{L}(\text{\usebox{\boxtwo}})(c)|_{\mathcal{F}(\text{\usebox{\boxone}})})_{\mathfrak{c}.\mathfrak{s}.\text{x}_0=\text{e}_2.\text{x}}$}{\textit{Ppty}}\\
    \smalltfield{e}{\record{
        \tfield{X}{every$^w$(restr)}\\
        \tfield{f}{$((x:\mathfrak{T}(X))\rightarrow\mathfrak{P}^\forall(\text{scope})\{x\})$}
      }}}
\end{ex}
Let us check that \preveg{} does in fact give us the strong
$\forall$-reading.  We will do this by showing that the value
associated with the label `scope' will be paraphrasable as ``the
property of being an individual such that if it's a farmer who owns a
donkey, she likes that donkey'', that is, likes every donkey she
owns. We can show this by unpacking the expression representing the
scope in \preveg{}.  `\textbf{likes}$^\frown$\textbf{it}', according to
our treatment of free pronouns and extensional verbs, will be
\nexteg{}.
\begin{ex} 
  $\ulcorner\lambda c$:\smallrecord{
    \footnotesize{Cntxt}\\
    \smalltfield{$\mathfrak{s}$}{\smallrecord{
        %\footnotesize{Assgnmnt}\\
        \smalltfield{x$_0$}{\textit{Ind}}}}} .
  $\ulcorner\lambda r$:\smallrecord{
    \smalltfield{x}{\textit{Ind}}} . \record{
    \tfield{e}{like$^\dagger$($r$.x, $c.\mathfrak{s}$.x$_0$)}}$\urcorner\urcorner$
\end{ex}
Using the definition of the localization operation, $\mathcal{L}$, in
(\ref{ex:localization}), we can see that
`$\mathcal{L}(\textbf{likes}^\frown\textbf{it})$' is \nexteg{}.
\begin{ex} 
$\ulcorner\lambda c$:\textit{Cntxt} . $\ulcorner\lambda
r$:\smallrecord{
  \smalltfield{x}{\textit{Ind}}\\
  \smalltfield{$\mathfrak{c}$}{\smallrecord{
      \footnotesize{\textit{Cntxt}}\\
      \smalltfield{$\mathfrak{s}$}{\smallrecord{
          \smalltfield{x$_0$}{\textit{Ind}}}}}}} . \record{
    \tfield{e}{like$^\dagger$($r$.x, $r.\mathfrak{c}.\mathfrak{s}$.x$_0$)}}$\urcorner\urcorner$
\end{ex} 
Applying \preveg{} to any context, $c$, will obtain \nexteg{}.
\begin{ex} 
 $\ulcorner\lambda
r$:\smallrecord{
  \smalltfield{x}{\textit{Ind}}\\
  \smalltfield{$\mathfrak{c}$}{\smallrecord{
      \footnotesize{\textit{Cntxt}}\\
      \smalltfield{$\mathfrak{s}$}{\smallrecord{
          \smalltfield{x$_0$}{\textit{Ind}}}}}}} . \record{
    \tfield{e}{like$^\dagger$($r$.x, $r.\mathfrak{c}.\mathfrak{s}$.x$_0$)}}$\urcorner$
\end{ex} 
Restricting \preveg{} by
`$\mathcal{F}(\textbf{farmer}^\frown\textbf{who}^\frown\textbf{owns}^\frown\textbf{a}^\frown\textbf{donkey})$',
yields \nexteg{}.
\begin{ex} 
  $\ulcorner\lambda r$:\smallrecord{
    \smalltfield{x}{\textit{Ind}}\\
    \smalltfield{$\mathfrak{c}$}{\smallrecord{
        \footnotesize{\textit{Cntxt}}\\
        \smalltfield{$\mathfrak{s}$}{\smallrecord{
            %\footnotesize{\textit{Assgnmnt}}\\
            \smallmfield{x$_0$}{$\Uparrow^2$e$_2$.x}{\textit{Ind}}}}\\
        \smalltfield{$\mathfrak{c}$}{\smallrecord{
            \smalltfield{f}{\textit{PropCntxt}}\\
            \smalltfield{a}{\textit{PropCntxt}}}}
}}\\
    \smalltfield{e$_1$}{farmer(x)}\\
    \smalltfield{e$_2$}{\smallrecord{
        \smalltfield{x}{$\mathfrak{T}$(donkey$'$)}\\
        \smalltfield{e}{own$^\dagger$($r$.x, x)}}}}
   .
  \record{
    \tfield{e}{like$^{\dagger}$($r$.x, $r.\mathfrak{c}.\mathfrak{s}.\text{x}_0$)}}$\urcorner$
\end{ex} 
  