\chapter*{Introduction}
\label{ch:intro}
\addcontentsline{toc}{chapter}{Introduction}

As we interact with the world and with each other we need to classify objects
and events, that is, we need to make judgements about what types of
objects and events we are confronted with.  This is an important part
of what is involved in planning the future actions we should carry
out and how we should coordinate with other agents in carrying out
collaborative actions.  This is true of action in general, including
linguistic action.  The aim of this book is to characterize a
notion of type which will cover both linguistic and non-linguistic
action and to lay the foundations for a theory of action based on
these types.  We will argue that a theory of language based on
action allows us to take a perspective on linguistic content which is
centered on interaction in dialogue and that this is importantly different to
the traditional view of natural languages as being essentially similar
to formal languages such as logics developed by philosophers or mathematicians.  At the same time we will argue that the
tremendous technical advances made by the formal language view of
semantics can be
incorporated into the action-based view and that this can lead to
important improvements both of intuitive understanding and empirical
coverage.

Part~I of the book (Chapters~\ref{ch:percint}--\ref{ch:gram}) deals with a theory of types related to perception
and action and shows a way of presenting a theory of grammar within a
theory of action.  Part~II
(Chapters~\ref{ch:propnames}--\ref{ch:quant}) then looks at a number of central issues in
semantics from a dialogical perspective and argues that there are
advantages to looking at some old puzzles from this perspective.

In Chapter~\ref{ch:percint} we introduce a notion of perception of an
object or event as making a judgement that the object is of a type.
In symbols, we write $a:T$ to indicate that object $a$ is of type $T$.
We shall talk interchangeably of an object being of a type or being a
witness for a type.  Our claim
is that we can only perceive something as being of a type, even if
that type is very general (like \textit{PhysicalObject} or
\textit{Event}) -- we cannot perceive it \textit{simpliciter}.  We
present basic notions of the theory of types which will be developed
in the book, TTR, a type theory with
records, which builds to a great extent on ideas taken from the type
theory of Per Martin-L�f although we have made significant changes
both in the general design and aims of the theory and a number of
details which appear to us to be motivated by cognitive and linguistic
considerations.  We also introduce some basic notions of a theory of
action based on these types which will be developed further as the
book progresses.  The overall approach presented here owes much to the
theory of situations and situation semantics presented by Barwise and
Perry in the 1980's.  One of the themes of this book is a
working out of parts of the old situation theory using ideas taken from
Martin-L�f's type theory.

A central notion in TTR is that of \textit{record}.  The term
``record'' is used in computer science for what is often called an
attribute-value matrix (AVM) or feature structure in linguistics.  A
record is a collection of fields consisting of a label (attribute or
feature in the standard linguistic way of talking) and an object of
some kind (which itself can be a record).  An schematic example of a
record is given in \nexteg{}, where the $\ell_i$ are labels and the
$o_i$ are objects.
\begin{ex} 
\record{\field{$\ell_0$}{\record{\field{$\ell_1$}{$o_0$}\\
                                 \field{$\ell_2$}{$o_1$}}}\\
        \field{$\ell_3$}{$o_2$}}
\label{ex:sch-rec} 
\end{ex} 
Records are witnesses for record types which are also collections of
fields. Rather than objects, the fields in a record type contain types.
In the schematic example in \nexteg{} the $T_i$ are types.
\begin{ex} 
\record{\tfield{$\ell_0$}{\record{\tfield{$\ell_1$}{$T_0$}\\
                                 \tfield{$\ell_2$}{$T_1$}}}\\
        \tfield{$\ell_3$}{$T_2$}} 
\end{ex} 
The record (\ref{ex:sch-rec}) will be of the type \preveg{} just in
case the objects are of the types with the same labelling, that is, $o_0:T_0$, $o_1:T_1$ and $o_2:T_2$. Martin-L�f's orginal type
theory did not have records or record types though there have been
many suggestions in the literature on how to add them.  We have
borrowed freely from some of these ideas in TTR although the way we
have developed the notions differs essentially from previous
proposals.  We will use records and record types to model situations
and situation types.   

In Chapter~\ref{ch:infex} we apply the theory of types from
Chapter~\ref{ch:percint} to basic notions of information update in
dialogue.  Here we build on seminal work on dialogue analysis by
Jonathan Ginzburg and also related computational implementation by
Staffan Larsson
leading to the information state update approach to dialogue systems.
We have adapted these ideas in a way that allows us to pursue the
questions of grammar and semantics that we take up in the remainder of
the book.  A central notion here is that of the dialogue gameboard
which we construe as a type of information state representing the
current state of play in the dialogue from the perspective of a
dialogue participant.  It includes the dialogue participant's view of what has been committed to as being
true in the dialogue so far and what questions are currently under discussion. 

In Chapter~\ref{ch:gram} we show how syntax and semantics can be
embedded in the theory of action characterized in
Chapters~\ref{ch:percint} and \ref{ch:infex}.  This is in contrast to
a formal language view where language is seen as a set of analyzed
strings of symbols associated with meanings of some kind.  The
philosophical ground of the action-based approach goes back to the relational theory of
meaning introduced in Barwise and Perry's situation semantics which
focusses on the relation between utterance situations and described
situations.  This was perhaps the first attempt to generalize the
Speech Act Theory developed by Austin and Searle to the concerns of
compositional interpretation of syntactic structure.  \ignore{We think that
placing the old situation semantics project within the theory of types
that we present makes it more convincing and also gives it greater
mathematical precision and detail.}  A more recent theory to which the
ideas in this chapter are related is that of Dynamic Syntax (DS).
While the particular formulations in our approach look rather
different from those in DS the two theories have common aims relating
to the analysis of language as action and an emphasis on the
incremental nature of language which in this chapter we relate to the
building of a chart type.  There is also a common interest in the
treatment of language as a system in flux where an act of speaking can
create a new previously unavailable linguistic resource that can be
reused in future speech events.

The theory of types that we employ gives us two important notions
which will be important in the development of semantics in Part~II.
The first is the notion of \textit{intensionality}.  Types in TTR are
intensional in that the identity of a type is not established in terms
of the set of objects which are of that type.  That is, types are not
\textit{extensional} in the way that sets are in a standard set
theory.  The axiom of extensionality in standard set theory requires
that there cannot be two sets which have the same members.  In contrast, there can be different types which have exactly the same set
of witnesses.  The second notion has to do with the facts that the
types themselves are treated as objects that can enter into relations
and be used to construct new types.  We will call this \textit{first
  class citizenship of types}, though it is related to notions of
\textit{intentionality} (with a ``t'') and \textit{reflection} in
programming languages, that is, the ability not only to carry out
procedures but to reflect on and reason about them.  In our terms, an
important enabling factor for human language is that we not only can
perceive objects and events in terms of types and act on these
perceptions but that we can also
reason about and act on the types themselves, for example, in ascribing them to
other agents as
beliefs or making a plan to achieve a goal by creating an event of a
certain type.  The types become cognitive
\textit{resources} which we can exploit in our communicative
activity.  In Part~II we will look at a number of examples of this.


In Chapter~\ref{ch:propnames} we examine reference by uses of proper names and
occurrences of pronouns which are not bound by quantifiers.  In order
to account for this we need a notion of \textit{parametric content},
which is to say that the content of an utterance depends on a context
belonging to a certain type.  For example, an utterance of the proper
name \textit{Sam} requires a context in which there is an individual
named ``Sam''.  But where in her resources should a dialogue
participant look for such a context?  One obvious place is the
conversational gameboard that we introduced in
Chapter~\ref{ch:infex}.  That is, the dialogue participant should
determine whether there has been reference to somebody of that name
already in the current dialogue according to her gameboard.  Another place is the visual scene, or more
generally the ambient situation which the agent can perceive by
different sense modalities.  This we also represent as a resource
using a type -- that is, the type for which the ambient situation
would be a witness if the agent's perception is correct.  Yet another
place to look is the agent's long term memory (which we will equate
with the agent's beliefs, although one may ultimately wish to make a
distinction).  This resource is also modelled as a type representing
how the world would be if the agent's memory or beliefs are correct.
The fact that we are reasoning about to what extent the context type
associated with the utterance matches the types modelling the agent's
relevant resources enables us to talk about cases where there are
names of non-existent objects (that is, the agent's resource types do
not exactly match the world) or where a single object in the world
corresponds to two objects in the resources or \textit{vice versa} (another way in which
there can be a mismatch between reality and an agent's resources).

In Chapter~\ref{ch:commonnouns} we look at frames associated with
common nouns.  The idea of frames goes back to early work on frame
semantics by Fillmore and also psychological work on frames by
Barsalou.  We will construe frames as situations (modelled as records
in TTR).  We will argue that frame types are an additional kind of
resource which is exploited in natural language semantics.  
A common noun like \textit{dog}, in addition to being
associated with the property of being a dog, can also be associated
with a type of situation (a frame type) which is common for dogs, for
example, where the dog has a name, an age and various other attributes
we commonly attribute to dogs.  We will argue that such a frame can
play an important role in interpreting utterances such as \textit{the
  dog is nine} in the sense of ``the dog is nine years old''.  Some
nouns, such as \textit{temperature}, seem to represent frame level
predicates, following an analysis suggested by Sebastian L�bner in
order to account for the analysis of utterances like \textit{the
  temperature is rising} where it is not the case that some particular
temperature is rising (say, 30\textdegree) but that different
situations (frames) with different temperatures are being compared.
Nouns which normally predicate of individuals can be coerced to
predicate of frames.  An example is the noun \textit{ship} in an
example originally discussed by Manfred Krifka: \textit{four thousand
  ships passed through the lock} which can either mean that four
thousand distinct ships passed through the lock or that there were
four thousand ship-passing-through-the-lock events some of which may
have involved the same ship. We argue that in order to interpret such
examples you need to have as a resource an appropriate frame type
associated with the noun \textit{ship}.

In Chapter~\ref{ch:intensional} we explore phenomena in natural
language which are standardly referred to as \textit{modality} and
\textit{intensionality}. We argue that types as we conceive them are better
placed to deal with these phenomena than possible worlds that are used
in standard formal semantics.  In standard formal semantics
propositions are regarded as sets of possible worlds.  For example,
the proposition corresponding to \textit{a boy hugged a dog} is the
set of all logically possible worlds in which a boy hugged a dog is true.  What we
substitute for this is the type of situations in which a boy hugged a
dog.  At an intuitive level these notions are quite similar.  They
both represent mathematical objects which allow for many different
possibilities as long as the fact that a boy hugged a dog is held
constant across them.  One important difference is that sets of
possible worlds are extensional sets whereas as our types are
intensional.  Thus it is possible for us to have two distinct types
which have exactly the same witnesses.  One pair of such examples we
discuss is \textit{Kim sold Syntactic Structures to Sam} and
\textit{Sam bought Syntactic Structures from Kim}.  Intuitively we
want these to represent different propositions and we argue that they
can yield different truth conditions when embedded under a predicate
like \textit{legal}.  (Under Swedish law, for example, it is illegal to buy sex but
legal to sell sex.) Another pair
involves so-called mathematical propositions which are true in all
possible worlds but which nevertheless we would want to represent
different propositions:  \textit{Two plus two equals four} and
\textit{Fermat's last theorem is true} (as proved by Andrew Wiles).

The chapter begins with a discussion of the problems associated with
possible worlds analyses. We then continue with a discussion of
modality and in particular of how Angelika Kratzer's notions of
conversational background and ideals can be seen with advantage as
resources based on types and the kind of topoi that Ellen Breitholtz
has introduced in the TTR literature.  In the third part of the
chapter we discuss what are traditionally regarded as intensional
constructions involving attitude verbs like \textit{believe} and
intensional verbs like \textit{need} and \textit{want}.  We treat
`believe' as a relation between individuals and types (corresponding
to the content of the embedded sentence).  For an
individual to believe a type it has to be the case that the type
matches (in a way we make precise) the type which models the beliefs
(or long term memory) of the individual, that is the same resource
that was needed in Chapter~\ref{ch:propnames} to get the dialogical
analysis of proper names to work out.  \ignore{Thus resources modelled as
types play an important role in our account of intensionality as well.}

In Chapter~\ref{ch:quant} we look at generalized quantifiers from the
perspective of dialogic interaction.  Traditionally generalized
quantifiers are treated as sets of sets or sets of properties and the
work of Barwise and Cooper on generalized quantifiers built on this
idea.  Barwise and Cooper also introduced the auxiliary notion of witness set
for quantifiers under the heading ``Processing quantified
statements''.  In this chapter we turn things around and make the
characterization of witness sets the primary notion in defining
quantifiers.  This makes it more straightforward to account for the
anaphoric possibilities relating to quantified expressions in
dialogue.
We often use quantified statements in dialogue when we have inadequate
information to determine their truth.  This is particularly true of
determiners like \textit{every} and \textit{most} when talking about
large sets.  We suggest that this phenomenon can be analyzed by
estimating a probability based on the evidence presented in our
cognitive resources (long-term memory or beliefs as discussed in
Chapters~\ref{ch:propnames} and \ref{ch:intensional}).
Finally, we give an account of how TTR types can be used to talk of
content which is underspecified for quantifier scope.  The idea is to
compute content types which have the various available contents which
can be associated with an utterance as witnesses.

There are a number of general themes which are woven together in this book and
which have different emphasis at different points in the text:
\begin{itemize} 
 
\item language as action 
 
\item linguistic content grounded in perception as type judgement

\item language as interaction and coordination

\item language as a system in flux 

\item types, not possible worlds

\item types and reference to non-existent objects

\item types and cognitive resources

\item getting the balance right between language, the external world
  and mental states

\item types as a way of doing underspecification

\item avoiding an intermediate ``semantic'' language such as logical
  form or discourse representation language but rather giving a direct
  interpretation of linguistic events in terms of semantic a semantic
  universe containing structured objects
 
\end{itemize} 
Behind all this is a desire to find a theory of types which can be
used to talk about cognition in general as well as allow us to give a
general account of language which includes many of the insights we
have gained from separate linguistic theories, a foundation for a
formal approach to cognition, if you like.

Why try to do all of this at once?  Would it not have been better to
write individual books and papers on each of these topics in turn?
These are questions that I have asked myself at various points while
writing this book.  It worries me (and it will probably worry you)
despite the fact that I know the answer:  it is important to have a
single approach to language in which all these issues can be addressed
simultaneously.  Taking the issues one at a time is not as
convincing or ultimately as interesting as showing how these different
aspects of language interact in a complex system, giving us a view
of linguistic interpretation which both embraces an action oriented
approach and preserves the insights we have gained from formal
semantics as well as addressing some of the puzzles that it failed to
solve adequately.  

%%% Local Variables:
%%% mode: latex
%%% TeX-master: "ttl"
%%% End:
