\chapter*{Introduction}
\label{ch:intro}

As we interact with the world and with each other we need to classify objects
and events, that is, we need to make judgements about what types of
objects and events we are confronted with.  This is an important part
of what is involved in planning what future actions we should carry
out and how we should coordinate with other agents in carrying out
collaborative actions.  This is true of action in general, including
linguistic action.  The aim of this book is to characterize a
notion of type which will cover both linguistic and non-linguistic
action and to lay the foundations for a theory of action based on
these types.  We will argue that a theory of language based on
action allows us to take a perspective on linguistic content which is
centered on interaction in dialogue and that this is importantly different to
the traditional view of natural languages as being essentially similar
to formal languages.  At the same time we will argue that the
tremendous technical advances made by the formal language view can be
incorporated into the action-based view and that this can lead to
important improvements both of intuitive understanding and empirical
coverage.

Part~I of the book (Chapters~\ref{ch:percint}--\ref{ch:gram}) deals with a theory of types related to perception
and action and shows a way of presenting a theory of grammar within a
theory of action.  Part~II
(Chapters~\ref{ch:propnames}--\ref{ch:quant}) then looks at a number of central issues in
semantics from a dialogical perspective and argues that there are
advantages to looking at some old puzzles from this perspective.

In Chapter~\ref{ch:percint} we introduce a notion of perception of an
object or event as making a judgement that it is of a type.  Our claim
is that we can only perceive something as being of a type, even if
that type is very general (like \textit{PhysicalObject} or
\textit{Event}) -- we cannot perceive it \textit{simpliciter}.  We
present basic notions of the theory of types which will be developed
in the book, TTR, a type theory with
records, which builds to a great extent on ideas taken from the type
theory of Per Martin-L�f although we have made significant changes
both in the general design and aims of the theory and a number of
details which appear to us to be motivated by cognitive and linguistic
considerations.  We also introduce some basic notions of a theory of
action based on these types which will be developed further as the
book progresses.  The overall approach presented here owes much to the
theory of situations and situation semantics presented by Barwise and
Perry in the nineteen eighties.  One of the themes of this book is a
working out of the old situation theory using ideas taken from
Martin-L�f's type theory.

In Chapter~\ref{ch:infex}


%%% Local Variables:
%%% mode: latex
%%% TeX-master: "ttl"
%%% End:
