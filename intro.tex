\chapter*{Introduction}
\label{ch:intro}
\addcontentsline{toc}{chapter}{Introduction}

As we interact with the world and with each other we need to classify objects
and events, that is, we need to make judgements about what types of
objects and events we are confronted with.  This is an important part
of what is involved in planning what future actions we should carry
out and how we should coordinate with other agents in carrying out
collaborative actions.  This is true of action in general, including
linguistic action.  The aim of this book is to characterize a
notion of type which will cover both linguistic and non-linguistic
action and to lay the foundations for a theory of action based on
these types.  We will argue that a theory of language based on
action allows us to take a perspective on linguistic content which is
centered on interaction in dialogue and that this is importantly different to
the traditional view of natural languages as being essentially similar
to formal languages.  At the same time we will argue that the
tremendous technical advances made by the formal language view can be
incorporated into the action-based view and that this can lead to
important improvements both of intuitive understanding and empirical
coverage.

Part~I of the book (Chapters~\ref{ch:percint}--\ref{ch:gram}) deals with a theory of types related to perception
and action and shows a way of presenting a theory of grammar within a
theory of action.  Part~II
(Chapters~\ref{ch:propnames}--\ref{ch:quant}) then looks at a number of central issues in
semantics from a dialogical perspective and argues that there are
advantages to looking at some old puzzles from this perspective.

In Chapter~\ref{ch:percint} we introduce a notion of perception of an
object or event as making a judgement that the object is of a type.
In symbols, we write $a:T$ to indicate that object $a$ is of type $T$.
We shall talk interchangeably of an object being of a type or being a
witness for a type.  Our claim
is that we can only perceive something as being of a type, even if
that type is very general (like \textit{PhysicalObject} or
\textit{Event}) -- we cannot perceive it \textit{simpliciter}.  We
present basic notions of the theory of types which will be developed
in the book, TTR, a type theory with
records, which builds to a great extent on ideas taken from the type
theory of Per Martin-L�f although we have made significant changes
both in the general design and aims of the theory and a number of
details which appear to us to be motivated by cognitive and linguistic
considerations.  We also introduce some basic notions of a theory of
action based on these types which will be developed further as the
book progresses.  The overall approach presented here owes much to the
theory of situations and situation semantics presented by Barwise and
Perry in the nineteen eighties.  One of the themes of this book is a
working out of parts of the old situation theory using ideas taken from
Martin-L�f's type theory.

A central notion in TTR is that of \textit{record}.  The term
``record'' is used in computer science for what is often called an
attribute-value matrix (AVM) or feature structure in linguistics.  A
record is a collection of fields consisting of a label (attribute or
feature in the standard linguistic way of talking) and an object of
some kind (which itself can be a record).  An schematic example of a
record is given in \nexteg{}, where the $\ell_i$ are labels and the
$o_i$ are objects.
\begin{ex} 
\record{\field{$\ell_0$}{\record{\field{$\ell_1$}{$o_0$}\\
                                 \field{$\ell_2$}{$o_1$}}}\\
        \field{$\ell_3$}{$o_2$}}
\label{ex:sch-rec} 
\end{ex} 
Records are witnesses for record types which are also collections of
fields. Rather than objects, the fields in a record type contain types.
In the schematic example in \nexteg{} the $T_i$ are types.
\begin{ex} 
\record{\tfield{$\ell_0$}{\record{\tfield{$\ell_1$}{$T_0$}\\
                                 \tfield{$\ell_2$}{$T_1$}}}\\
        \tfield{$\ell_3$}{$T_2$}} 
\end{ex} 
The record (\ref{ex:sch-rec}) will be of the type \preveg{} just in
case $o_0:T_0$, $o_1:T_1$ and $o_2:T_2$. Martin-L�f's orginal type
theory did not have records or record types though there have been
many suggestions in the literature on how to add them.  We have
borrowed freely from some of these ideas in TTR although the way we
have developed the notions differs essentially from previous
proposals.  We will use records and record types to model situations
and situation types.   

In Chapter~\ref{ch:infex} we apply the theory of types from
Chapter~\ref{ch:percint} to basic notions of information update in
dialogue.  Here we build on seminal work on dialogue analysis by
Jonathan Ginzburg and also related computational implementation by
Staffan Larsson
leading to the information state update approach to dialogue systems.
We have adapted these ideas in a way that allows us to persue the
questions of grammar and semantics that we take up in the remainder of
the book.  A central notion here is that of the dialogue gameboard
which we construe as a type of information state representing the
current state of play in the dialogue from the perspective of a
dialogue participant including what has been committed to as being
true in the dialogue so far and what questions are currently under discussion. 

In Chapter~\ref{ch:gram} we show how syntax and semantics can be
embedded in the theory of action characterized in
Chapters~\ref{ch:percint} and \ref{ch:infex}.  This is in contrast to
a formal language view where language is seen as a set of analyzed
strings of symbols associated with meanings of some kind.  The
philosophical ground of the action-based approach goes back to the relational theory of
meaning introduced in Barwise and Perry's situation semantics which
focusses on the relation between utterance situations and described
situations.  This was perhaps the first attempt to generalize the
Speech Act Theory developed by Austin and Searle to the concerns of
compositional interpretation of syntactic structure.  We think that
placing the old situation semantics project within the theory of types
that we present makes it more convincing and also gives it greater
mathematical precision and detail.  A more recent theory to which the
ideas in this chapter are related is that of Dynamic Syntax (DS).
While the particular formulations in our approach look rather
different from those in DS the two theories have common aims relating
to the analysis of language as action and an emphasis on the
incremental nature of language which in this chapter we relate to the
building of a chart type.  There is also a common interest in the
treatment of language as a system in flux where an act of speaking can
create a new previously unavailable linguistic resource that can be
reused in future speech events.

The theory of types that we employ gives us two important notions
which will be important in the development of semantics in Part~II.
The first is the notion of \textit{intensionality}.  Types in TTR are
intensional in that the identity of a type is not established in terms
of the set of objects which are of that type.  That is, types are not
\textit{extensional} in the way that sets are in a standard set
theory.  The axiom of extensionality in standard set theory requires
there cannot be two sets which have the same members.  In contrast, there can be different types which have exactly the same set
of witnesses.  The second notion has to do with the facts that the
types themselves are treated as objects that can enter into relations
and be used to construct new types.  We will call this \textit{first
  class citizenship of types}, though it is related to notions of
\textit{intentionality} (with a ``t'') and \textit{reflection} in
programming languages, that is, the ability not only to carry out
procedures but to reflect on and reason about them.  In our terms, an
important enabling factor for human language is that we not only can
perceive objects and events in terms of types and act on these
perceptions but that we can also
reason about and act on the types themselves, for example, in ascribing them to
other agents as
beliefs or making a plan to achieve a goal.  The types become cognitive
\textit{resources} which we can exploit in our communicative
activity.  In Part~II we will look at a number of examples of this.


In Chapter~\ref{ch:propnames} we examine reference by uses of proper names and
occurrences of pronouns which are not bound by quantifiers.  In order
to account for this we need a notion of \textit{parametric content},
which is to say that the content of an utterance depends on a context
belonging to a certain type.  For example, an utterance of the proper
name \textit{Sam} requires a context in which there is an individual
named ``Sam''.  But where in her resources should a dialogue
participant look for such a context?  One obvious place is the
conversational gameboard that we introduced in
Chapter~\ref{ch:infex}.  Another place is the visual scene, or more
generally the ambient situation which the agent can perceive by
different sense modalities.  This we also represent as a resource
using a type -- that is, the type for which the ambient situation
would be a witness if the agent's perception is correct.  Yet another
place to look is the agent's long term memory (which we will equate
with the agent's beliefs, although one may ultimately wish to make a
distinction).  This resource is also modelled as a type representing
how the world would be if the agent's memory or beliefs are correct.
The fact that we are reasoning about to what extent the context type
associated with the utterance matches the types modelling the agent's
relevant resources enables us to talk about cases where there are
names of non-existent objects (that is, the agent's resource types do
not exactly match the world) or where a single object in the world
corresponds to two objects in the resources or \textit{vice versa} (another way in which
there can be a mismatch between reality and an agent's resources).

In Chapter~\ref{ch:commonnouns} we look at frames associated with
common nouns.  The idea of frames goes back to early work on frame
semantics by Fillmore and also psychological work on frames by
Barsalou.  We will construe frames as situations (modelled as records
in TTR).  A common noun like \textit{dog}



%%% Local Variables:
%%% mode: latex
%%% TeX-master: "ttl"
%%% End:
