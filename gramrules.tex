\chapter{Grammar rules}
\label{app:gramrules}

\section{Universal resources}
\label{app:gramrulesuniv}

\subsection{Frames}

\textit{AmbTempFrame} (Chapter~\ref{ch:commonnouns})

\record{\tfield{x}{\textit{Real}} \\
        \tfield{loc}{\textit{Loc}} \\
        \tfield{e}{temp(loc, x)}} 
 

\textit{AgeFrame} (Chapter~\ref{ch:commonnouns})

\smallrecord{\smalltfield{x}{\textit{Ind}}\\
                         \smalltfield{age}{\textit{Real}}\\
                         \smalltfield{c$_{\mathrm{age}}$}{age\_of(x,age)}}

\textit{DogFrame} (Chapter~\ref{ch:commonnouns})

\smallrecord{\smalltfield{x}{\textit{Ind}}\\
                      \smalltfield{e}{dog(x)}\\
                      \smalltfield{age}{\textit{Real}}\\
                      \smalltfield{c$_{\mathrm{age}}$}{age\_of(x,age)}} 

\subsection{Scales}

$\zeta_{\mathrm{temp}}$ (Chapter~\ref{ch:commonnouns})

$\lambda r$:\textit{AmbTempFrame} . $r$.x

$\zeta_{\mathrm{age}}$ (Chapter~\ref{ch:commonnouns})

$\lambda r$:\textit{AgeFrame}
                       . $r$.age

\subsection{Signs}

\textit{Sign} (Chapter~\ref{ch:infex})

\record{\tfield{s-event}{\textit{SEvent}} \\
        \tfield{cont}{\textit{Cont}}}

\textit{Sign} (Chapter~\ref{ch:gram})

a recursive type

$\sigma$ : \textit{Sign} iff $\sigma$ :\record{\tfield{s-event}{\textit{SEvent}} \\
         \tfield{syn}{\textit{Syn}} \\
        \tfield{cont}{\textit{Cont}}} 

\textit{SEvent} (Chapter~\ref{ch:infex})

\record{\tfield{e-loc}{\textit{Loc}} \\
        \tfield{sp}{\textit{Ind}} \\
        \tfield{au}{\textit{Ind}} \\
        \tfield{e}{\textit{Phon}} \\
        \tfield{c$_{\mathrm{loc}}$}{loc(e,e-loc)} \\
        \tfield{c$_{\mathrm{sp}}$}{speaker(e,sp)} \\
        \tfield{c$_{\mathrm{au}}$}{audience(e,au)}}

\textit{Phon} (Chapter~\ref{ch:infex})

\textit{Word}$^+$

\textit{Cont} (Chapter~\ref{ch:infex})

\textit{RecType}

\textit{Cont} (Chapter~\ref{ch:gram})

\textit{RecType} $\vee$ \textit{Ppty} $\vee$ \textit{Quant} $\vee$ (\textit{Ppty}$\rightarrow$\textit{Quant})

\textit{Ppty} (Chapter~\ref{ch:gram})

(\smallrecord{\smalltfield{x}{\textit{Ind}}}$\rightarrow$\textit{RecType})

\textit{Ppty} (Chapter~\ref{ch:commonnouns})

\record{\tfield{bg}{\textit{Type}}\\
              \tfield{fg}{(\smallrecord{\smalltfield{x}{bg}}$\rightarrow$\textit{RecType})}} 

\textit{Ppty}($T$), where $T$ is a type (Chapter~\ref{ch:commonnouns})

\record{\mfield{bg}{$T$}{\textit{Type}}\\
              \tfield{fg}{(\smallrecord{\smalltfield{x}{bg}}$\rightarrow$\textit{RecType})}}

% $\displaystyle{\bigvee_{T\sqsubseteq\textit{Rec}}}(T\rightarrow
% \textit{RecType})$

\textit{PPpty} (Chapter~\ref{ch:propnames})

\record{\tfield{bg}{\textit{RecType}}\\
              \tfield{fg}{(bg$\rightarrow$\textit{Ppty})}}

\textit{Abbreviations for properties} (Chapter~\ref{ch:commonnouns}) 

If $p$ is a predicate with arity $\langle T\rangle$ then $p'$
represents the property

\begin{quote}
\record{\field{bg}{\smallrecord{\smalltfield{x}{$T$}}}\\
        \field{fg}{$\lambda r$:\smallrecord{\smalltfield{x}{$T$}}
. \record{\tfield{e}{$p$($r$.x)}}}}
\end{quote}

\bigskip

If $P$ is the property

\begin{quote}
\record{\field{bg}{\smallrecord{\smalltfield{x}{$T_1$}}}\\
        \field{fg}{$\lambda r$:\smallrecord{\smalltfield{x}{$T_1$}}
          . $T_2(r)$}}
\end{quote}

then $P\!\restriction\!s$ represents

\begin{quote}
\record{\field{bg}{\smallrecord{\smalltfield{x}{$T_1$}}}\\
        \field{fg}{$\lambda r$:\smallrecord{\smalltfield{x}{$T_1$}}
          . $T_2(r)\!\drestr$\smallrecord{\field{e}{$s$}}}}
\end{quote}

% If $T_1$, $T_2$ and $T_3$ are types, $T$ is a record type such that $\langle$e,
% $T_2$$\rangle$ $\in$ $T$, $s:T_3$ and $P$ is the property

% $\lambda r$:\smallrecord{\smalltfield{x}{$T_1$}} . $T$

% then $P$$\restriction$$s$ is

% $\lambda r$:\smallrecord{\smalltfield{x}{$T_1$}} . $T'$

% where $T'$ is like $T$ except that $\langle$e,$T_2$$\rangle$ is
% replaced by $\langle$e,${T_2}_s$$\rangle$ (that is, $T' = (T -
% \{\langle\mathrm{e},T_2\rangle\})\cup\{\langle\mathrm{e,{T_2}_s}\rangle\}$).

This uses the definition of $T\drestr r$ in
Appendix\ref{app:specrec}. % Note that these definitions require the second alternative definition
% of singleton types in Appendix~\ref{app:singletontypes}.

\textit{Quant} (Chapter~\ref{ch:gram})

(\textit{Ppty}$\rightarrow$\textit{RecType})

\textit{PQuant} (Chapter~\ref{ch:propnames})

\record{\tfield{bg}{\textit{RecType}} \\
        \tfield{fg}{(bg$\rightarrow$\textit{Quant})}}

%$\displaystyle{\bigvee_{T\sqsubseteq \mathit{Rec}}}(T\rightarrow\textit{Quant})$

\textit{Syn} (Chapter~\ref{ch:gram})

\record{\tfield{cat}{\textit{Cat}} \\
        \tfield{daughters}{\textit{Sign}$^*$}} 

\textit{Cat} (Chapter~\ref{ch:gram})

s, np, det, n, v, vp : \textit{Cat}

\textit{Category sign types:}

\begin{quote}
\textit{S} (Chapter~\ref{ch:gram})

\textit{Sign} \d{$\wedge$}
\smallrecord{\smalltfield{syn}{\smallrecord{\smallmfield{cat}{s}{\textit{Cat}}}}}

\textit{NP} (Chapter~\ref{ch:gram})

\textit{Sign} \d{$\wedge$}
\smallrecord{\smalltfield{syn}{\smallrecord{\smallmfield{cat}{np}{\textit{Cat}}}}}

\textit{Det} (Chapter~\ref{ch:gram})

\textit{Sign} \d{$\wedge$}
\smallrecord{\smalltfield{syn}{\smallrecord{\smallmfield{cat}{det}{\textit{Cat}}}}}

\textit{N} (Chapter~\ref{ch:gram})

\textit{Sign} \d{$\wedge$}
\smallrecord{\smalltfield{syn}{\smallrecord{\smallmfield{cat}{n}{\textit{Cat}}}}}

\textit{V} (Chapter~\ref{ch:gram})

\textit{Sign} \d{$\wedge$}
\smallrecord{\smalltfield{syn}{\smallrecord{\smallmfield{cat}{v}{\textit{Cat}}}}}

\textit{VP} (Chapter~\ref{ch:gram})

\textit{Sign} \d{$\wedge$}
\smallrecord{\smalltfield{syn}{\smallrecord{\smallmfield{cat}{vp}{\textit{Cat}}}}}

\end{quote}

\textit{NoDaughters} (Chapter~\ref{ch:gram})

\smallrecord{\smalltfield{syn}{\smallrecord{\smallmfield{daughters}{$\varepsilon$}{\textit{Sign}$^*$}}}}



\subsection{Sign type construction operations}
\label{app:signtypeconstr}

\subsubsection{Lexicon}
\label{app:lexuniversal}

sign (Chapter~\ref{ch:infex})

If $\sigma$ is a type of speech event and $\kappa$ is a type (of
situation) then \\
sign($\sigma$,$\kappa$)= \smallrecord{\smalltfield{s-event}{\smallrecord{\smalltfield{e}{$\sigma$}} } \\
        \mfield{cont}{\smallrecord{\smalltfield{e}{$\kappa$} \\
                             \smalltfield{c$_{\mathrm{tns}}$}{final\_align($\Uparrow$s-event.e,e)}}}{\textit{RecType}}}

sign$_{\mathit{uc}}$ (Chapter~\ref{ch:infex})

If $\sigma$ is a type of speech event then \\
sign$_{\mathit{uc}}$($\sigma$)= \smallrecord{\smalltfield{s-event}{\smallrecord{\smalltfield{e}{$\sigma$}} } \\
        \smalltfield{cont}{\textit{RecType}}}

Lex (Chapter~\ref{ch:gram})

$\lambda T_1$:\textit{Type}\\
\hspace*{.25em}$\lambda T_2$:\textit{Type} . \\
\hspace*{.5em}\mbox{$T_1$ \d{$\wedge$}
\smallrecord{\smalltfield{s-event}{\smallrecord{\smalltfield{e}{$T_2$}}}}
      \d{$\wedge$} \textit{NoDaughters}}

\textit{Licensing condition associated with lexical resources}
(Chapter~\ref{ch:gram})

If Lex($T$, $C$) is a resource available to agent $A$, then for any
$u$, $u:_A T$ licenses $:_A$ Lex($T$, $C$)
\d{$\wedge$}\smallrecord{\tfield{s-event}{\smallrecord{\smallmfield{e}{$u$}{$T$}}}}

\textit{Universal resources for lexical content construction}

SemCommonNoun($p$), where $p$ is a predicate with arity $\langle$\textit{Ind}$\rangle$ (Chapter~\ref{ch:gram})

$\lambda
r$:\smallrecord{\smalltfield{x}{\textit{Ind}}}
. \record{\tfield{e}{$p$($r$.x)}}

SemCommonNoun($p$,
$T_{\mathrm{arg}}$, $T_{\mathrm{restr}}$, $T_{\mathrm{bg}}$),
where $p$ is a predicate with arity $\langle T_{\mathrm{arg}}\rangle$,
$T_{\mathrm{restr}}\sqsubseteq T_{\mathrm{arg}}$ and
$T_{\mathrm{bg}}$ is a record type representing the background
requirements (Chapter~\ref{ch:commonnouns})

\record{\field{bg}{$T_{\mathrm{bg}}$}\\
        \field{fg}{$\lambda c$:$T_{\mathrm{bg}}$ . \record{\field{bg}{$T_{\mathrm{restr}}$}\\
                           \field{fg}{$\lambda r$:\smallrecord{\smalltfield{x}{$T_{\mathrm{restr}}$}} . 
                      \smallrecord{\smalltfield{e}{$p$($r$.x)}}}}}}

SemIntransVerb($T_{\mathrm{bg}}$, $p$), where $T_{\mathrm{bg}}$, the
``background'' or ``presupposition'' type, is a record type and $p$ is a predicate with arity $\langle$\textit{Ind}$\rangle$ (Chapter~\ref{ch:propnames})

\record{\field{bg}{$T_{\mathrm{bg}}$}\\
        \field{fg}{$\lambda r_1$:$T_{\mathrm{bg}}$ . $\lambda
r_2$:\smallrecord{\smalltfield{x}{\textit{Ind}}}
. \record{\tfield{e}{$p$($r_2$.x)}}}} 

SemIntransVerb($p$,
$T_{\mathrm{arg}}$, $T_{\mathrm{restr}}$, $T_{\mathrm{bg}}$) where $p$ is a predicate with arity $\langle T_{\mathrm{arg}}\rangle$,
$T_{\mathrm{restr}}\sqsubseteq T_{\mathrm{arg}}$ and
$T_{\mathrm{bg}}$ (Chapter~\ref{ch:commonnouns})

\record{\field{bg}{$T_{\mathrm{bg}}$}\\
        \field{fg}{$\lambda c$:$T_{\mathrm{bg}}$ . \record{\field{bg}{$T_{\mathrm{restr}}$}\\
                           \field{fg}{$\lambda r$:\smallrecord{\smalltfield{x}{$T_{\mathrm{restr}}$}} . 
                      \smallrecord{\smalltfield{e}{$p$($r$.x)}}}}}}


SemPropName($a$), where $a$:\textit{Ind} (Chapter~\ref{ch:gram})

$\lambda P$:\textit{Ppty} . $P$(\smallrecord{\field{x}{$a$}}) 

SemPropName($T$), where $T$ is a phonological type
(Chapter~\ref{ch:propnames})

\record{\field{bg}{\smallrecord{\smalltfield{x}{\textit{Ind}}\\
                         \smalltfield{e}{named(x, $T$)}}} \\
        \field{fg}{$\lambda r$: \smallrecord{\smalltfield{x}{\textit{Ind}}\\
                         \smalltfield{e}{named(x, $T$)}} . \\
& & \hspace*{5em}$\lambda P$:\textit{Ppty} . $P(r)$}}

 

SemNumeral($n$), where $n$:\textit{Real}
(Chapter~\ref{ch:commonnouns})

$\lambda r$:\textit{Rec} . \\
\hspace*{1em}$\lambda P$:\textit{Ppty}(\textit{Real}) . \\
\hspace*{2em}$P$.fg(\smallrecord{\field{x}{$n$}})

SemIndefArt (Chapter~\ref{ch:gram})

$\lambda Q$:\textit{Ppty} . \\
\hspace*{1em} $\lambda P$:\textit{Ppty}
. \record{\mfield{restr}{$Q$}{\textit{Ppty}} \\
          \mfield{scope}{$P$}{\textit{Ppty}} \\
          \tfield{e}{exist(restr, scope)}}

SemIndefArt (Chapter~\ref{ch:commonnouns})

$\lambda Q$:\textit{PPpty} . 
\smallrecord{\field{bg}{\smallrecord{\smalltfield{f}{\textit{Rec}}\\
                                     \smalltfield{a}{$Q$.bg}}}\\
             \field{fg}{$\lambda r$:\smallrecord{\smalltfield{f}{\textit{Rec}}\\
                                     \smalltfield{a}{$Q$.bg}} . 
$\lambda P$:\textit{Ppty} . 
\smallrecord{\smallmfield{restr}{$Q$.fg($r$.a)}{\textit{Ppty}}\\
             \smallmfield{scope}{$P$}{\textit{Ppty}}\\
             \smalltfield{e}{exist(restr, scope)}}}}

SemDefArt (Chapter~\ref{ch:commonnouns})

$\lambda Q$:\textit{PPpty} . 
\smallrecord{\field{bg}{\smallrecord{\smalltfield{f}{\smallrecord{\smalltfield{s}{\textit{Rec}}\\
                                                                  \smalltfield{e}{unique($Q$.fg($\Uparrow$a),s)}}}\\
                                     \smalltfield{a}{$Q$.bg}}}\\
             \field{fg}{$\lambda r$:\smallrecord{\smalltfield{f}{\smallrecord{\smalltfield{s}{\textit{Rec}}\\
                                                                  \smalltfield{e}{unique($Q$.fg($\Uparrow$a),s)}}}\\
                                     \smalltfield{a}{$Q$.bg}} . 
$\lambda P$:\textit{Ppty} . 
\smallrecord{\smallmfield{restr}{$Q$.fg($r$.a)}{\textit{Ppty}}\\
             \smallmfield{scope}{$P$}{\textit{Ppty}}\\
             \smalltfield{e}{every(restr, scope)}}}}

SemBe (Chapter~\ref{ch:gram})

% $\lambda\mathcal{Q}$:\textit{Quant} . \\
% \hspace*{1em} $\lambda r_1$:\smallrecord{\tfield{x}{\textit{Ind}}}
% . \\
% \hspace*{2em} $\mathcal{Q}$($\lambda
% r_2$:\smallrecord{\tfield{x}{\textit{Ind}}}
% . \record{\tfield{e}{$r_2$.x = $r_1$.x}})

% $\lambda\mathcal{Q}$:\textit{Quant} . \\
% \hspace*{1em} $\lambda r_1$:\smallrecord{\tfield{x}{\textit{Ind}}}
% . \\
% \hspace*{2em} $\mathcal{Q}$($\lambda
% r_2$:\smallrecord{\tfield{x}{\textit{Ind}}}
% . \record{\mfield{x}{$r_2$.x, $r_1$.x}{\textit{Ind}}})

$\lambda\mathcal{Q}$:\textit{Quant} . \\
\hspace*{1em} $\lambda r_1$:\smallrecord{\tfield{x}{\textit{Ind}}}
. \\
\hspace*{2em} $\mathcal{Q}$($\lambda
r_2$:\smallrecord{\tfield{x}{\textit{Ind}}}
. \record{\mfield{x}{$r_2$.x, $r_1$.x}{\textit{Ind}} \\
          \tfield{e}{be(x)}})


SemBe($T_{\mathrm{arg}}$, $T_{\mathrm{bg}}$)  where $T_{\mathrm{arg}}$
and $T_{\mathrm{bg}}$ are types (Chapter~\ref{ch:commonnouns})

If $T_{\mathrm{bg}}\sqsubseteq$
\smallrecord{\smalltfield{sc}{($T_{\mathrm{arg}}\to\textit{Real}$)}}
then SemBe($T_{\mathrm{arg}}$, $T_{\mathrm{bg}}$) is

% $\lambda r$:$T_{\mathrm{bg}}$ . \\
% \hspace*{1em}$\lambda\mathcal{Q}$:\textit{Quant} . \\
% \hspace*{2em}
% \record{\field{bg}{$T_{\mathrm{arg}}$}\\
%              \field{fg}{$\lambda r_1$:\smallrecord{\smalltfield{x}{$T_{\mathrm{arg}}$}}
% . \\
% & & \hspace*{1em}$\mathcal{Q}$(\record{\field{bg}{\smallrecord{\smalltfield{x}{\textit{Real}}}}\\
%                            \field{fg}{$\lambda
% r_2$:\smallrecord{\smalltfield{x}{\textit{Real}}}
% . \record{\tfield{e}{$r$.sc($r_1$.x) = $r_2$.x}}
% }
% }
% )}}

$\lambda r$:$T_{\mathrm{bg}}$ . \\
\hspace*{1em}$\lambda\mathcal{Q}$:\textit{Quant} . \\
\hspace*{2em}
\record{\field{bg}{$T_{\mathrm{arg}}$}\\
             \field{fg}{$\lambda r_1$:\smallrecord{\smalltfield{x}{$T_{\mathrm{arg}}$}}
. \\
& & \hspace*{1em}$\mathcal{Q}$(\record{\field{bg}{\smallrecord{\smalltfield{x}{\textit{Real}}}}\\
                           \field{fg}{$\lambda
r_2$:\smallrecord{\smalltfield{x}{\textit{Real}}}
. \record{\mfield{x}{$r$.sc($r_1$.x), $r_2$.x}{\textit{Real}}\\
          \tfield{e}{be(x)}}
}
}
)}}


Otherwise, SemBe($T_{\mathrm{arg}}$, $T_{\mathrm{bg}}$) is

% $\lambda r$:$T_{\mathrm{bg}}$ . \\
% \hspace*{1em}$\lambda\mathcal{Q}$:\textit{Quant} . \\
% \hspace*{2em}
% \record{\field{bg}{$T_{\mathrm{arg}}$}\\
%              \field{fg}{$\lambda r_1$:\smallrecord{\smalltfield{x}{$T_{\mathrm{arg}}$}}
% . \\
% & & \hspace*{1em}$\mathcal{Q}$(\record{\field{bg}{$T_{\mathrm{arg}}$}\\
%                            \field{fg}{$\lambda
% r_2$:\smallrecord{\smalltfield{x}{$T_{\mathrm{arg}}$}}
% . \record{\tfield{e}{$r_1$.x = $r_2$.x}}
% }
% }
% )}}

$\lambda r$:$T_{\mathrm{bg}}$ . \\
\hspace*{1em}$\lambda\mathcal{Q}$:\textit{Quant} . \\
\hspace*{2em}
\record{\field{bg}{$T_{\mathrm{arg}}$}\\
             \field{fg}{$\lambda r_1$:\smallrecord{\smalltfield{x}{$T_{\mathrm{arg}}$}}
. \\
& & \hspace*{1em}$\mathcal{Q}$(\record{\field{bg}{$T_{\mathrm{arg}}$}\\
                           \field{fg}{$\lambda
r_2$:\smallrecord{\smalltfield{x}{$T_{\mathrm{arg}}$}}
. \record{\mfield{x}{$r_1$.x, $r_2$.x}{$T_{\mathrm{arg}}$}\\
          \tfield{e}{be(x)}}
}
}
)}}


\textit{Universal resources for associating lexical content with
  phonological types}

Lex$_{\mathrm{CommonNoun}}$($T_{\mathrm{phon}}$, $p$), where
$T_{\mathrm{phon}}$ is a phonological type and $p$ is a predicate with
arity $\langle$\textit{Ind}$\rangle$ (Chapter~\ref{ch:gram})\\
is defined as \\
Lex($T_{\mathrm{phon}}$, \textit{N}) \d{$\wedge$}
\smallrecord{\smallmfield{cont}{SemCommonNoun($p$)}{\textit{Ppty}}}

Lex$_{\mathrm{CommonNoun}}$($T_{\mathrm{phon}}$, $p$,
  $T_{\mathrm{arg}}$, $T_{\mathrm{restr}}$, $T_{\mathrm{bg}}$),
  where $T_{\mathrm{phon}}$ is a phonological type, $p$ is a predicate
  with arity $\langle T_{\mathrm{arg}}\rangle$,
  $T_{\mathrm{restr}}\sqsubseteq T_{\mathrm{arg}}$ and
  $T_{\mathrm{bg}}$ is a record type (Chapter~\ref{ch:commonnouns}) \\
 is defined as\\
Lex($T_{\mathrm{phon}}$, \textit{N}) \d{$\wedge$} 
\smallrecord{\smallmfield{cont}{SemCommonNoun($p$, $T_{\mathrm{arg}}$,
    $T_{\mathrm{restr}}$, $T_{\mathrm{bg}}$)}{\textit{PPpty}}} 

Lex$_{\mathrm{IntransVerb}}$($T_{\mathrm{phon}}$, $T_{\mathrm{bg}}$, $p$), where
$T_{\mathrm{phon}}$ is a phonological type and $p$ is a predicate with
arity $\langle$\textit{Ind}$\rangle$ (Chapter~\ref{ch:propnames})\\
is defined as \\
Lex($T_{\mathrm{phon}}$, \textit{VP}) \d{$\wedge$}
\smallrecord{\smallmfield{cont}{SemIntransVerb($T_{\mathrm{bg}}$,
    $p$)}{\textit{PPpty}}} 

Lex$_{\mathrm{IntransVerb}}$($T_{\mathrm{phon}}$, $p$,
  $T_{\mathrm{arg}}$, $T_{\mathrm{restr}}$, $T_{\mathrm{bg}}$),
  where $T_{\mathrm{phon}}$ is a phonological type, $p$ is a predicate
  with arity $\langle T_{\mathrm{arg}}\rangle$,
  $T_{\mathrm{restr}}\sqsubseteq T_{\mathrm{arg}}$ and
  $T_{\mathrm{bg}}$ is a record type (Chapter~\ref{ch:commonnouns})\\ 
is defined as\\
Lex($T_{\mathrm{phon}}$, \textit{VP}) \d{$\wedge$} 
\smallrecord{\smallmfield{cont}{SemIntransVerb($p$, $T_{\mathrm{arg}}$,
    $T_{\mathrm{restr}}$, $T_{\mathrm{bg}}$)}{\textit{PPpty}}}


Lex$_{\mathrm{PropName}}$($T_{\mathrm{Phon}}$, $a$), where
$T_{\mathrm{Phon}}$ is a phonological type and $a$:\textit{Ind}
(Chapter~\ref{ch:gram}) \\
is defined as \\
Lex($T_{\mathrm{Phon}}$, \textit{NP}) \d{$\wedge$}
\smallrecord{\smallmfield{cont}{SemPropName($a$)}{\textit{Quant}}}

Lex$_{\mathrm{PropName}}$($T_{\mathrm{Phon}}$), where
$T_{\mathrm{Phon}}$ is a phonological type
(Chapter~\ref{ch:propnames}) \\
is defined as \\
Lex($T_{\mathrm{Phon}}$, \textit{NP}) \d{$\wedge$}
\smallrecord{\smallmfield{cont}{SemPropName($T_{\mathrm{Phon}}$)}{\textit{PQuant}}}

Lex$_{\mathrm{numeral}}$, where
$T_{\mathrm{phon}}$ is a phonological type  and $n$ is a (real) number
(Chapter~\ref{ch:commonnouns}) \\
is defined as \\
Lex($T_{\mathrm{phon}}$, \textit{NP}) \d{$\wedge$} \smallrecord{\smallmfield{cont}{SemNumeral($n$)}{\textit{PQuant}}}

Lex$_{\mathrm{IndefArt}}$($T_{\mathrm{Phon}}$), where
$T_{\mathrm{Phon}}$ is a phonological type (Chapter~\ref{ch:gram}) \\
is defined as \\
Lex($T_{\mathrm{Phon}}$, \textit{Det}) \d{$\wedge$}
\smallrecord{\smallmfield{cont}{SemIndefArt}{(\textit{Ppty}$\rightarrow$\textit{Quant})}}

Lex$_{\mathrm{IndefArt}}$($T_{\mathrm{phon}}$), where
  $T_{\mathrm{phon}}$ is a phonological type
  (Chapter~\ref{ch:commonnouns}) \\
 is defined as\\ 
Lex($T_{\mathrm{phon}}$, \textit{Det}) \d{$\wedge$} \smallrecord{\smallmfield{cont}{SemIndefArt}{(\textit{PPpty}$\to$\textit{PQuant})}}

Lex$_{\mathrm{DefArt}}$($T_{\mathrm{phon}}$), where
  $T_{\mathrm{phon}}$ is a phonological type
  (Chapter~\ref{ch:commonnouns})\\
 is defined as\\ 
Lex($T_{\mathrm{phon}}$, \textit{Det}) \d{$\wedge$} \smallrecord{\smallmfield{cont}{SemDefArt}{(\textit{PPpty}$\to$\textit{PQuant})}}

Lex$_{\mathrm{be}}$($T_{\mathrm{Phon}}$), where
$T_{\mathrm{Phon}}$ is a phonological type (Chapter~\ref{ch:gram}) \\
is defined as \\
Lex($T_{\mathrm{Phon}}$,
\textit{V}) \d{$\wedge$}
\smallrecord{\smallmfield{cont}{SemBe}{(\textit{Quant}$\rightarrow$\textit{Ppty})}} 

Lex$_{\mathrm{be}}$($T_{\mathrm{Phon}}$, $T_{\mathrm{arg}}$, $T_{\mathrm{bg}}$), where
$T_{\mathrm{Phon}}$ is a phonological type and $T_{\mathrm{arg}}$ and
$T_{\mathrm{bg}}$ are types  (Chapter~\ref{ch:commonnouns}) \\
is defined as \\
Lex($T_{\mathrm{Phon}}$,
\textit{V}) \d{$\wedge$}
\smallrecord{\smallmfield{cont}{SemBe($T_{\mathrm{arg}}$, $T_{\mathrm{bg}}$)}{(\textit{Quant}$\rightarrow$\textit{PPpty})}} 

\textit{Universal resources for coercing lexical sign types to new
  lexical sign types}

CommonNounIndToFrame (Chapter~\ref{ch:commonnouns})

If $T_{\mathrm{phon}}$ is a phonological type, $p$ is a predicate and
$T_\mathrm{bg}$ is a record type (the ``background type'' or
``presupposition'') then \\ 
\mbox{CommonNounIndToFrame(Lex$_{\mathrm{CommonNoun}}$($T_{\mathrm{phon}}$,
$p$, \textit{Ind}, \textit{Ind}, $T_\mathrm{bg}$))} = \\
Lex$_{\mathrm{CommonNoun}}$($T_{\mathrm{phon}}$,
$p$\_frame, \textit{Rec}, \textit{Rec}, $T_\mathrm{bg}$)

where if $p$ is a predicate with arity $\langle$\textit{Ind}$\rangle$, then
for any $e$ and $r$,  
\begin{quote}
$e$ : $p$\_frame($r$) implies $r$ : 
\smallrecord{\smalltfield{x}{\textit{Ind}}\\
             \smalltfield{e}{$p$(x)}}
\end{quote}

RestrictCommonNoun (Chapter~\ref{ch:commonnouns})

If $T_{\mathrm{phon}}$ is a phonological type, $p$ is a predicate,
$T_{\mathrm{arg}}$ is a type and that arity of $p$ is $\langle
T_{\mathrm{arg}}\rangle$, $T_{\mathrm{restr}}\sqsubseteq
T_{\mathrm{arg}}$, $T_{\mathrm{bg}}$ is a record type and
$T_{\mathrm{mod}}\sqsubseteq T_{\mathrm{restr}}$ then

\mbox{RestrictCommonNoun(Lex$_{\mathrm{CommonNoun}}$($T_{\mathrm{phon}}$,
$p$, $T_{\mathrm{arg}}$, $T_{\mathrm{restr}}$, $T_{\mathrm{bg}}$),
$T_{\mathrm{mod}}$)} = \\
Lex$_{\mathrm{CommonNoun}}$($T_{\mathrm{phon}}$,
$p$, $T_{\mathrm{arg}}$, $T_{\mathrm{mod}}$, $T_{\mathrm{bg}}$)  



\subsubsection{Operations which construct sign combination functions}
\label{app:signcomb}

\textit{Licensing condition associated with sign combination
  functions} (Chapter~\ref{ch:gram})

If $f:(T_1\rightarrow Type)$ is a sign combination function available to agent $A$, then for
any $u$, $u :_A T_1$ licenses $:_A f(u)$


RuleDaughters (Chapter~\ref{ch:gram})

RuleDaughters maps two types to a sign combination function

$\lambda T_1$ : \textit{Type} \\
\hspace*{1em} $\lambda T_2$ : \textit{Type}\ . \\
\hspace*{2em} $\lambda u : T_1$\ . $T_2$ \d{$\wedge$}
\smallrecord{\smalltfield{syn}{\smallrecord{\smallmfield{daughters}{$u$}{$T_1$}}}}

ConcatPhon (Chapter~\ref{ch:gram})

$\lambda
u$:\smallrecord{\smalltfield{s-event}{\smallrecord{\smalltfield{e}{\textit{Phon}}}}}$^+$\
. \\
\hspace*{1em}\record{\tfield{s-event}{\record{\mfield{e}{concat$_i$($u[i]$.s-event.e)}{\textit{Phon}}}}}


Phrase structure rule notation (Chapter~\ref{ch:gram})

If $C,C_1,\ldots,C_n$ are category sign types then,
\begin{quote}
$C \longrightarrow C_1 \ldots C_n$ represents RuleDaughters($C$,
${C_1}^{\frown}\ldots^{\frown}C_n$) \d{\d{$\wedge$}} ConcatPhon
\end{quote}

Combination of parametric contents (Chapter~\ref{ch:propnames})

If $\alpha$ : \smallrecord{\smalltfield{bg}{\textit{RecType}}\\
                           \smalltfield{fg}{(bg$\rightarrow$($T_1\rightarrow
                             T_2$))}} 
and $\beta$ : \smallrecord{\smalltfield{bg}{\textit{RecType}}\\
                           \smalltfield{fg}{(bg$\rightarrow T_1$)}}
                         then the \textit{combination of $\alpha$ and
    $\beta$  based on functional application}, $\alpha\text{@}\beta$, is
\begin{quote}
\record{\field{bg}{\smallrecord{\smalltfield{f}{[$\alpha$.bg]$^{\text{f}.}$}\\
                                \smalltfield{a}{[$\beta$.bg]$^{\text{a}.}$}}}\\
        \field{fg}{$\lambda r$:\smallrecord{\smalltfield{f}{[$\alpha$.bg]$^{\text{f}.}$}\\
                                \smalltfield{a}{[$\beta$.bg]$^{\text{a}.}$}} . $\alpha$.fg($r$.f)($\beta$.fg($r$.a))}}
\end{quote}
where $[T]^\pi$ represents the result of prefixing each path-name
occurring as an argument to a predicate in $T$ with $\pi$.

ContForwardApp (Chapter~\ref{ch:gram})

$\lambda T_1$:\textit{Type} $\lambda T_2$:\textit{Type} . \\
\hspace*{1em}$\lambda
u$:\smallrecord{\smalltfield{cont}{$(T_2\rightarrow
    T_1)$}}$^{\frown}$\smallrecord{\smalltfield{cont}{$T_2$}} . \\
\hspace*{2em}\smallrecord{\smallmfield{cont}{$u$[0].cont($u$[1].cont)}{$T_1$}}

ContForwardApp (Chapter~\ref{ch:propnames})

$\lambda T_1$:\textit{Type} $\lambda T_2$:\textit{Type} . \\
\hspace*{1em}$\lambda
u$:\smallrecord{\smalltfield{cont}{\smallrecord{\smalltfield{bg}{\textit{RecType}}\\
                                               \smalltfield{fg}{(bg$\rightarrow$($T_2\rightarrow T_1$))}}}}$^{\frown}$
   \smallrecord{\smalltfield{cont}{\smallrecord{\smalltfield{bg}{\textit{RecType}}\\
                                               \smalltfield{fg}{(bg$\rightarrow T_2$)}}}} . \\
\hspace*{2em}\smallrecord{\smallmfield{cont}{$u$[0].cont@$u$[1].cont}{\smallrecord{\smalltfield{bg}{\textit{RecType}}\\
                                                                                \smalltfield{fg}{(bg$\rightarrow T_1$)}}}} 

ContForwardApp (Chapter~\ref{ch:commonnouns})

$\lambda T_1$:\textit{Type} $\lambda T_2$:\textit{Type} . \\
\hspace*{1em}$\lambda
u$:\smallrecord{\smalltfield{cont}{$(T_2\rightarrow
    T_1)$}}$^{\frown}$\smallrecord{\smalltfield{cont}{$T_2$}} . \\
\hspace*{2em}\smallrecord{\smallmfield{cont}{$u$[0].cont($u$[1].cont)}{$T_1$}}

ContSForwardApp (Chapter~\ref{ch:commonnouns})

$\lambda T_1$:\textit{Type} $\lambda T_2$:\textit{Type} . \\
\hspace*{1em}$\lambda
u$:\smallrecord{\smalltfield{cont}{\smallrecord{\smalltfield{bg}{\textit{RecType}}\\
                                               \smalltfield{fg}{(bg$\rightarrow$($T_2\rightarrow T_1$))}}}}$^{\frown}$
   \smallrecord{\smalltfield{cont}{\smallrecord{\smalltfield{bg}{\textit{RecType}}\\
                                               \smalltfield{fg}{(bg$\rightarrow T_2$)}}}} . \\
\hspace*{2em}\smallrecord{\smallmfield{cont}{$u$[0].cont@$u$[1].cont}{\smallrecord{\smalltfield{bg}{\textit{RecType}}\\
                                                                                \smalltfield{fg}{(bg$\rightarrow T_1$)}}}} 

\section{English resources}

\subsection{Lexicon}
\label{app:lexeng}

(Chapter~\ref{ch:infex})

sign(``Dudamel is a conductor'', conductor(dudamel)), \\ 
sign(``Beethoven
is a composer'', composer(beethoven)), \\
sign(``Uchida is a
pianist'', pianist(uchida)), \\
sign$_{\mathit{uc}}$(``ok''), \\
sign$_{\mathit{uc}}$(``aha'')

(Chapter~\ref{ch:gram})

Lex(``Dudamel'', \textit{NP}) \\
Lex(``Beethoven'', \textit{NP}) \\
Lex(``a'', \textit{Det}) \\
Lex(``composer'', \textit{N}) \\
Lex(``conductor'', \textit{N}) \\
Lex(``is'', \textit{V}) \\
Lex(``ok'', \textit{S}) \\
Lex(``aha'',\textit{S})

Lex$_{\mathrm{PropName}}$(``Dudamel'', $d$), where $d$:\textit{Ind} \\
Lex$_{\mathrm{PropName}}$(``Beethoven'', $b$), where $b$:\textit{Ind}
\\
Lex$_{\mathrm{CommonNoun}}$(``composer'', composer), where `composer'
is a predicate with arity
$\langle$\smallrecord{\smalltfield{x}{\textit{Ind}}}$\rangle$ \\ 
Lex$_{\mathrm{CommonNoun}}$(``conductor'', conductor), where `conductor'
is a predicate with arity
$\langle$\smallrecord{\smalltfield{x}{\textit{Ind}}}$\rangle$ \\  
Lex$_{\mathrm{IndefArt}}$(``a'') \\
Lex$_{\mathrm{be}}$(``is'')

(Chapter~\ref{ch:propnames})

Lex$_{\mathrm{PropName}}$(``Sam'') \\
Lex$_{\mathrm{IntransVerb}}$(``leave'', \textit{Rec}, leave) 

(Chapter~\ref{ch:commonnouns})

Lex$_{\mathrm{DefArt}}$(``the'') \\
Lex$_{\mathrm{IndefArt}}$(``a'') \\
Lex$_{\mathrm{CommonNoun}}$(``dog'', dog, \textit{Ind},
  \textit{Ind}, \textit{Rec})\\
Lex$_{\mathrm{CommonNoun}}$(``dog'', dog\_frame, \textit{Rec},
  \textit{Rec}, \textit{Rec}) (derived by CommonNounIndToFrame) \\
Lex$_{\mathrm{CommonNoun}}$(``dog'', dog\_frame, \textit{Rec},
  \textit{DogFrame}, \textit{Rec}) (derived by RestrictCommonNoun) \\
Lex$_{\mathrm{CommonNoun}}$(``temperature'', temperature,
      \textit{Rec}, \textit{Rec}, \textit{Rec}) \\
Lex$_{\mathrm{CommonNoun}}$(``temperature'', temperature,
      \textit{Rec}, \textit{AmbTempFrame}, \textit{Rec}) (derived by
      RestrictCommonNoun) \\
Lex$_{\mathrm{IntransVerb}}$(``runs'', run, \textit{Ind},
  \textit{Ind}, \textit{Rec})\\
Lex$_{\mathrm{IntransVerb}}$(``rises'', rise,
      \textit{Rec}, \textit{Rec}, \textit{Rec})  \\
Lex$_{\mathrm{be}}$(``is'', \textit{Ind}, \textit{Rec}) \\
Lex$_{\mathrm{be}}$(``is'', \textit{AgeFrame},
\smallrecord{\smalltfield{sc}{(\textit{AgeFrame}$\to$\textit{Real})}})
\\
Lex$_{\mathrm{be}}$(``is'', \textit{AmbTempFrame},
\smallrecord{\smalltfield{sc}{(\textit{AmbTempFrame}$\to$\textit{Real})}})
\\
Lex$_{\mathrm{numeral}}$(``nine'', 9) \\
Lex$_{\mathrm{numeral}}$(``ninety'', 90)


\subsection{Phrase structure}

(Chapter~\ref{ch:gram})

\textit{S} $\longrightarrow$ \textit{NP} \textit{VP} \\
\textit{NP} $\longrightarrow$ \textit{Det} \textit{N} \\
\textit{VP} $\longrightarrow$ \textit{V} \textit{NP}

\subsection{Non-compositional Constructions}

CnstrIsA (Chapter~\ref{ch:gram})

$\lambda
u$:\textit{V}\d{$\wedge$}\smallrecord{\smalltfield{s-event}{\smallrecord{\smalltfield{e}{``is''}}}}$^{\frown}$\textit{NP}\d{$\wedge$}\smallrecord{\smalltfield{syn}{\smallrecord{\smalltfield{daughters}{\textit{Det}\d{$\wedge$}\smallrecord{\smalltfield{s-event}{\smallrecord{\smalltfield{e}{``a''}}}} \\
                                                                    \hspace*{5em}$^{\frown}$
\textit{N}\d{$\wedge$}\smallrecord{\smalltfield{cont}{\textit{Ppty}}}}
}}}. \\
\hspace*{1em}
\textit{VP}\d{$\wedge$}\smallrecord{\smallmfield{cont}{$u$[2].syn.daughters[2].cont}{\textit{Ppty}}}

\subsection{Interpreted phrase structure}
\label{app:interpps}

(Chapter~\ref{ch:gram})

\textit{S} $\longrightarrow$ \textit{NP} \textit{VP} \d{\d{$\wedge$}}
ContForwardApp(\textit{Ppty}, \textit{RecType}) 

\textit{NP} $\longrightarrow$ \textit{Det} \textit{N} \d{\d{$\wedge$}}
ContForwardApp(\textit{Ppty}, \textit{Quant}) 

\textit{VP} $\longrightarrow$ \textit{V} \textit{NP} \d{\d{$\wedge$}}
CnstrIsA

A more readable abbreviatory notation for these rules is:

\textit{S} $\longrightarrow$ \textit{NP} \textit{VP} $\mid$
\textit{NP}$'$(\textit{VP}$'$)

\textit{NP} $\longrightarrow$ \textit{Det} \textit{N} $\mid$
\textit{Det}$'$(\textit{N}$'$)

\textit{VP} $\longrightarrow$ \lb{\textit{V}}{``is''}
\lb{\textit{NP}}{\lb{\textit{Det}}{``a''} \textit{N}} $\mid$
\textit{N}$'$

Note that this last rule does not correspond to a context-free
phrase-structure rule.

(Chapter~\ref{ch:commonnouns})

\textit{S} $\longrightarrow$ \textit{NP} \textit{VP} \d{\d{$\wedge$}}
ContSForwardApp(\textit{Ppty}, \textit{RecType}) 

\textit{NP} $\longrightarrow$ \textit{Det} \textit{N} \d{\d{$\wedge$}}
ContForwardApp(\textit{PPpty}, \textit{PQuant}) 

\textit{VP} $\longrightarrow$ \textit{V} \textit{NP} \d{\d{$\wedge$}}
ContSForwardApp(\textit{Quant}, \textit{Ppty})

A more readable abbreviatory notation for these rules is:

\textit{S} $\longrightarrow$ \textit{NP} \textit{VP} $\mid$
\textit{NP}$'$@\textit{VP}$'$

\textit{NP} $\longrightarrow$ \textit{Det} \textit{N} $\mid$
\textit{Det}$'$(\textit{N}$'$)

\textit{VP} $\longrightarrow$ \textit{V} \textit{NP} $\mid$
\textit{V}$'$@\textit{NP}$'$

%%% Local Variables: 
%%% mode: latex
%%% TeX-master: "ttl"
%%% End: 
