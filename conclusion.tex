\chapter*{Conclusion}
\label{ch:conclusion}
\addcontentsline{toc}{chapter}{Conclusion}
\markboth{\textit{CONCLUSION}}{}


What view of language have we come to after working through the
details of this book?
There are a number of general themes which are woven together:

\paragraph{Language as action} We see the prototypical form of
language as speech events in the context of dialogue and see other
forms of language such as written text, talking or thinking to
yourself as derivative of this.  Rather than start from a formal view
of language in terms of strings of symbols with associated structure
and try to associate this with linguistic events, we start with a
simple theory of action and show how the insights of the formal view
of language can be expressed in those terms.
 
\paragraph{Linguistic content grounded in perception as type
  judgement}  Language is not an abstract formal construction but
rather a form of communicative behaviour embedded in biological
agents.  As such the nature of the content which those biological
agents can express is conditioned by the physical relationship which
they have with their environment and the way that they
perceive and cognitively represent it.  We have offered
the beginnings of an abstract theory of how perception and linguistic
content can be related using type judgements.  It is important that
such a theory does not limit content to concern only that which can be
perceived.  We have suggested that linguistic agents (and presumably
non-linguistic higher animals) can not only judge things to be of
types but have developed the ability to reflect on and reason with the types themselves.  This
includes the ability to reflect on types that could not possibly have
a witness as well as types realized in the past or which might be
realized in the future or which correspond to a different possibility
from what is actually the case.

\paragraph{Language as interaction and coordination}  If you think of
language as a form of communicative behaviour it comes as no surprise
that interaction and coordination are central to a theory of
language.  In this book we have tried to relate linguistic interaction
and coordination to an approach in terms of a general theory of
action.  The techniques that seem necessary for the successful
coordination of a game of fetch between a dog and a human seem to
underly what is needed for coordination in dialogue.  We have
presented a view on which both involve recognizing (or predicting) the
type of event that is to be jointly realized by the coordinating
agents and update rules in terms of information states involving at
least an agenda and affordances provided to the agents by what has
happened in the event so far.  This involves a view of events as
strings of smaller events which have some similarity to scripts as
they have been used in artificial intelligence.  What we think of as
grammar arises out of these update procedures.  This is in contrast to
previous views of language processing where it is often the case the
linguistic processing such as parsing arises from the pre-existence of
grammatical rules which need to be called upon by a parsing algorithm.

\paragraph{Language as a system in flux}  Given a view of language as
action which involves interaction and coordination it does not come as
a surprise that language is a system in flux that is constantly being
adjusted to fit current needs, both in terms of what content one
wishes to express (the world is constantly presenting us with types of
situations we have not seen before) and in terms of the linguistic
habits and presumed resources available to our interlocutors.  We have
tried to build a theory in which it would be natural to modify the
system on the fly.  For example, our discussion of proper names talks
of proper names as being able to be connected to individuals on the
fly during the course of communication.  Something similar is going on
in our discussion of semantic frames (modelled as record types) which
can, we suggest, be created on the fly.

\paragraph{Types, not possible worlds}  We have used types to do the
kind of work which in other semantic theories is done by possible
worlds.  One type corresponds to uncountably many possible worlds.  It
is more tractable to reason with a small number of types, for example,
when dealing with preference orders in the analysis of modality, than
the transfinite collection of possible worlds that correspond to
them.  Types also give us a better way of modelling propositions than
sets of possible worlds as in many other approaches to linguistic
semantics.  Firstly, a proposition is modelled by a single type rather
than an uncountable number of possible worlds.  Secondly, types
provide us with a finer granularity than sets of possible worlds and this
enables us to distinguish among propositions which are logically
equivalent.  Finally, when the grain of types is too fine, as with
record types that only differ in their labelling, it is
straightforward to use relabelling in order to make this grain irrelevant.

\paragraph{Types and reference to non-existent objects} There are a
number of ways in which expressions in natural languages, notably
proper names and pronouns, can be used to refer to non-existent
objects.  Rather than posit a class of non-existent objects as such in
our semantic universe we have used types in order to achieve this.
Recall that there is no requirement that all types have witnesses.
There are also ``impossible'' or inconsistent types which could not
have witnesses in any possibility.  The labels in record types can be
exploited to achieve the effect of anaphoric reference to non-existent
objects.  This provides us with a parsimonious theory in which there a
types and real existent witnesses for types.  We do not have to in
addition posit possible but non-actual objects or impossible objects
and it gives us a convenient and intuitive way of talking about fiction as
presenting a type (or a series of types) which are not witnessed.

\paragraph{Types and cognitive resources}  Given that we have viewed
language in terms of action by cognitive agents grounded in their
perception of the world, it does not come as a surprise that the
agents' mental states and cognitive resources play a central role in
our account of how language is used in communication.  This represents
a shift from the view given by classical formal semantics which is
often presented in terms of the relationship between language and the
world.  While we have tried to show that the insights gained from 
traditional formal semantics can be incorporated in our
action-based theory, the puzzles that remain in the traditional
approach often involve cases where mental states seem to play a role.
This is particularly obvious in the case of attitudes such as belief.
Our analysis of belief involves matching the content of an utterance
against a type which we judge to represent an aspect of the mental
state of the person whose belief we are reporting.  However, we have
also used mental states in cases where it is not so obvious in
traditional theories that they are necessary, for example, in the
treatment of the use of proper names.  This is natural in a
dialogically oriented approach to semantics where processing an
utterance in a dialogue involves updating an information state which
records the current ``scoreboard'' (using David Lewis's term) of the
dialogue so far.  We have proposed that types can be used in the
modelling of cognitive resources as well as in the modelling of
utterances and their content.  For example, a model of an agent's
belief state is a type which intuitively represents how the world would be if the
agent's beliefs were true.  We suspect that most people's total belief
states are inconsistent types which could not actually be witnessed.



\paragraph{Types as a way of doing underspecification}  Types, as we
know, can have several witnesses.  This means that we can use types to
represent that we do not have information to choose among the
witnesses.  Thus types can be used to do the work of what have been
called ``underspecified representations'' in the linguistic and
computational literature or ``descriptions'' in the computational
literature.  Our exploration of underspecification
Chapter~\ref{ch:underspec} involved relating types of utterances to
types of contents rather than to contents as we had done in the
earlier chapters.  The aim of the work on underspecification in
general in the literature is to make the treatment of ambiguity in
natural language tractable by having a single representation covering
in some cases hundreds or even thousands of contents that might be
associated with a given utterance and being able to refine such a
general representation as the meaning of the utterance is narrowed
down.  Structured types such as record types are well-suited to this
kind of refinement and are thus promising candidates for 
underspecified processing of natural language.  From a
psycholinguistic and philosophical perspective the use of types both
for underspecification and the modelling of beliefs provides us with a
interesting perspective on linguistic processing:  namely, that what
we get from processing an utterance is not a fully specified content
but rather a belief about what kind of content was expressed --- a
belief that may be more or less specific depending on the information
we have available.  Given that we also have beliefs about what the
kind of context we are in, also modelled by types we can begin to see
how content types could be refined by context types thus giving us a
perspective on what is called ``ambiguity resolution'' in the
computational literature.



\paragraph{Direct interpretation} Part of Montague's original
programme for semantics was to show that natural languages can be
interpreted directly without first having to translate them into an
artificial logical representation.  For Montague this was in support
of the slogan ``English as a formal language'' --- the claim that
English can be given the same kind of semantic treatment as the
artificial formal languages created by logicians.  This is appealing
for the kind of theory developed in this book which is an attempt to
say something about the kind of cognition involved in linguistic
communication.  Presumably our linguistic behaviour has evolved in
order to support our communication with each other.  A theory which
says that language first has to be mapped to an artificial logical
language which is then interpreted suggests either that evolution has
failed to give us a language which we can use for communication or
that we, as theorists, have so far failed to find a way of
explaining how the actual speech events as such can convey
information.  In this book we have not introduced an intermediate
language between the natural language and its interpretation.  Rather
we have explored what kind of types corresponding to cognitive
resources are related to speech events.  This seems more convincing as
the basis of a cognitive theory of communication. 
 
\paragraph{Types in a general theory of cognition}  Behind all this is a desire to find a theory of types which can be
used to talk about cognition in general as well as allow us to give a
general account of language which includes many of the insights we
have gained from separate linguistic theories, a foundation for a
formal approach to cognition, if you like.  What has been presented in
this book represents a first step towards such a theory.

Why try to do all of this at once?  Would it not have been better to
write individual books and papers on each of these topics in turn?
These are questions that I have asked myself at various points while
writing this book.  It worries me (and it will probably worry you)
despite the fact that I know the answer:  it is important to have a
single approach to language in which all these issues can be addressed
simultaneously.  Taking the issues one at a time is not as
convincing or ultimately as interesting as showing how these different
aspects of language interact in a complex system, giving us a view
of linguistic interpretation which both embraces an action oriented
approach and preserves the insights we have gained from formal
semantics as well as addressing some of the puzzles that it failed to
solve adequately.  A large part of the motivation for using types in
the way that we have is that the same theory of types can be used
address all of these issues within a single coherent theory.


%%% Local Variables:
%%% mode: latex
%%% TeX-master: "ttl"
%%% End:
