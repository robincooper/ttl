\chapter*{Conclusion}
\label{ch:conclusion}
\addcontentsline{toc}{chapter}{Conclusion}
\markboth{\textit{CONCLUSION}}{}


What view of language have we come to after working through the
details of this book?
There are a number of general themes which are woven together:

\paragraph{Language as action} We see the prototypical form of
language as speech events in the context of dialogue and see other
forms of language such as written text, talking or thinking to
yourself as derivative of this.  Rather than start from a formal view
of language in terms of strings of symbols with associated structure
and try to associate this with linguistic events, we start with a
simple theory of action and show how the insights of the formal view
of language can be expressed in those terms.
 
\paragraph{Linguistic content grounded in perception as type
  judgement}  Language is not an abstract formal construction but
rather a form of communicative behaviour embedded in biological
agents.  As such the nature of the content which those biological
agents can express is conditioned by the physical relationship which
they have with their environment and the way that they
perceive and cognitively represent it.  We have offered
the beginnings of an abstract theory of how perception and linguistic
content can be related using type judgements.  It is important that
such a theory does not limit content to concern only that which can be
perceived.  We have suggested that linguistic agents (and presumably
non-linguistic higher animals) can not only judge things to be of
types but have developed the ability to reflect on and reason with the types themselves.  This
includes the ability to reflect on types that could not possibly have
a witness as well as types realized in the past or which might be
realized in the future or which correspond to a different possibility
from what is actually the case.

\paragraph{Language as interaction and coordination}  If you think of
language as a form of communicative behaviour it comes as no surprise
that interaction and coordination are central to a theory of
language.  In this book we have tried to relate linguistic interaction
and coordination to an approach in terms of a general theory of
action.  The techniques that seem necessary for the successful
coordination of a game of fetch between a dog and a human seem to
underly what is needed for coordination in dialogue.  We have
presented a view on which both involve recognizing (or predicting) the
type of event that is to be jointly realized by the coordinating
agents and update rules in terms of information states involving at
least an agenda and affordances provided to the agents by what has
happened in the event so far.  This involves a view of events as
strings of smaller events which have some similarity to scripts as
they have been used in artificial intelligence.  What we think of as
grammar arises out of these update procedures.  This is in contrast to
previous views of language processing where it is often the case the
linguistic processing such as parsing arises from the pre-existence of
grammatical rules which need to be called upon by a parsing algorithm.

\paragraph{Language as a system in flux}  Given a view of language as
action which involves interaction and coordination it does not come as
a surprise that language is a system in flux that is constantly being
adjusted to fit current needs, both in terms of what content one
wishes to express (the world is constantly presenting us with types of
situations we have not seen before) and in terms of the linguistic
habits and presumed resources available to our interlocutors.  We have
tried to build a theory in which it would be natural to modify the
system on the fly.  For example, our discussion of proper names talks
of proper names as being able to be connected to individuals on the
fly during the course of communication.  Something similar is going on
in our discussion of semantic frames (modelled as record types) which
can, we suggest, be created on the fly.

\paragraph{Types, not possible worlds}  We have used types to do the
kind of work which in other semantic theories is done by possible
worlds.  One type corresponds to uncountably many possible worlds.  It
is more tractable to reason with a small number of types, for example,
when dealing with preference orders in the analysis of modality, than
the transfinite collection of possible worlds that correspond to
them.  Types also give us a better way of modelling propositions than
sets of possible worlds as in many other approaches to linguistic
semantics.  Firstly, a proposition is modelled by a single type rather
than an uncountable number of possible worlds.  Secondly, types
provide us with a finer granularity than sets of possible worlds and this
enables us to distinguish among propositions which are logically
equivalent.  Finally, when the grain of types is too fine, as with
record types that only differ in their labelling, it is
straightforward to use relabelling in order to make this grain irrelevant.

\paragraph{Types and reference to non-existent objects} @@

\paragraph{Types and cognitive resources}

\paragraph{Getting the balance right between language, the external world
  and mental states}

\paragraph{Types as a way of doing underspecification}

\paragraph{Avoiding an intermediate ``semantic'' language such as logical
  form or discourse representation language but rather giving a direct
  interpretation of linguistic events in terms of a semantic
  universe containing structured objects}
 
Behind all this is a desire to find a theory of types which can be
used to talk about cognition in general as well as allow us to give a
general account of language which includes many of the insights we
have gained from separate linguistic theories, a foundation for a
formal approach to cognition, if you like.

Why try to do all of this at once?  Would it not have been better to
write individual books and papers on each of these topics in turn?
These are questions that I have asked myself at various points while
writing this book.  It worries me (and it will probably worry you)
despite the fact that I know the answer:  it is important to have a
single approach to language in which all these issues can be addressed
simultaneously.  Taking the issues one at a time is not as
convincing or ultimately as interesting as showing how these different
aspects of language interact in a complex system, giving us a view
of linguistic interpretation which both embraces an action oriented
approach and preserves the insights we have gained from formal
semantics as well as addressing some of the puzzles that it failed to
solve adequately.  


%%% Local Variables:
%%% mode: latex
%%% TeX-master: "ttl"
%%% End:
