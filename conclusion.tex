\chapter*{Conclusion}
\label{ch:conclusion}
\addcontentsline{toc}{chapter}{Conclusion}

What view of language have we come to after working through the
details of this book?
There are a number of general themes which are woven together:

\paragraph{Language as action} We see the prototypical form of
language as speech events in the context of dialogue and see other
forms of language such as written text, talking or thinking to
yourself as derivative of this.  Rather than start from a formal view
of language in terms of strings of symbols with associated structure
and try to associate this with linguistic events, we start with a
simple theory of action and show how the insights of the formal view
of language can be expressed in those terms.
 
\paragraph{Linguistic content grounded in perception as type
  judgement}  Language is not an abstract formal construction but
rather a form of communicative behaviour embedded in biological
agents.  As such the nature of the content which those biological
agents can express is conditioned by the physical relationship which
they have with their environment and the way that they
perceive and cognitively represent it.  We have offered
the beginnings of an abstract theory of how perception and linguistic
content can be related using type judgements.  It is important that
such a theory does not limit content to concern only that which can be
perceived.  We have suggested that linguistic agents (and presumably
non-linguistic higher animals) can not only judge things to be of
types but have developed the ability to reflect on and reason with the types themselves.  This
includes the ability to reflect on types that could not possibly have
a witness as well as types realized in the past or which might be
realized in the future or which correspond to a different possibility
from what is actually the case.

\paragraph{Language as interaction and coordination}  If you think of
language as a form of communicative behaviour it comes as no surprise
that interaction and coordination are central to a theory of
language.

@@

\paragraph{Language as a system in flux}

\paragraph{Types, not possible worlds}

\paragraph{Types and reference to non-existent objects}

\paragraph{Types and cognitive resources}

\paragraph{Getting the balance right between language, the external world
  and mental states}

\paragraph{Types as a way of doing underspecification}

\paragraph{Avoiding an intermediate ``semantic'' language such as logical
  form or discourse representation language but rather giving a direct
  interpretation of linguistic events in terms of a semantic
  universe containing structured objects}
 
Behind all this is a desire to find a theory of types which can be
used to talk about cognition in general as well as allow us to give a
general account of language which includes many of the insights we
have gained from separate linguistic theories, a foundation for a
formal approach to cognition, if you like.

Why try to do all of this at once?  Would it not have been better to
write individual books and papers on each of these topics in turn?
These are questions that I have asked myself at various points while
writing this book.  It worries me (and it will probably worry you)
despite the fact that I know the answer:  it is important to have a
single approach to language in which all these issues can be addressed
simultaneously.  Taking the issues one at a time is not as
convincing or ultimately as interesting as showing how these different
aspects of language interact in a complex system, giving us a view
of linguistic interpretation which both embraces an action oriented
approach and preserves the insights we have gained from formal
semantics as well as addressing some of the puzzles that it failed to
solve adequately.  


%%% Local Variables:
%%% mode: latex
%%% TeX-master: "ttl"
%%% End:
