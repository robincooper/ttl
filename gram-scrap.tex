%[from infex]



% We shall first characterize an abstract syntax which is (to some
% extent) language neutral since we are interested in characterizing
% multilingual grammar.  The abstract syntax (like tectogrammar) will
% characterize general structural and categorial characteristics without
% commiting to any particular phonological form.  It will also serve as
% a base for deriving semantic content.  Abstract syntax will be refined
% in order to obtain the concrete syntax for a particular language.

% Our syntax will be quite simple in order to deal with our current
% example.  However, we will discuss the kind of modularity that will be
% useful for discussing more complex grammars.

% In abstract syntax lexical items will be represented by record types
% of the form

% \record{\mfield{cat}{$c$}{\textit{Cat}} \\
%         \mfield{id}{$\mathit{id}$}{\textit{Id}}}

% where $c$ is some particular category like \textit{np} or \textit{s}
% and $\mathit{id}$ is a unique identifier associated with the
% particular item.
% We will represent this type as

% \{Lex $\mathit{id}$ $c$\}

% The reason for this abbreviation is not just to save space.  The
% characterization of which record types are associated with lexical
% items is not merely a matter of individual grammars.  Their general
% nature should be characterized in a (language) universal resource
% which formally characterizes our linguistic theory in a language
% independent way.  While we are not making the claim that the
% association of a category and a unique identifier with lexical items
% is any deep linguistic insight it is nevertheless something which is
% more general than the characterization of any one language.  And
% things will get more interesting later.

These record types can be combined to form record types corresponding
to abstract complex expressions.  This is done by functions which are
to phrase structure rules in a context-free syntax except that, like
the immediate dominance rules of generalized phrase structure grammar,
they do not specify word order as there is no phonology associated
with the abstract syntax.  A binary rule with ID $\mathit{id}$ whose mother category is $m$
and whose daughter categories are $d_1$ and $d_2$ is modelled by the
function

\begin{ex}
\label{eg:abstractbinary}
$\lambda r_1$:\record{\mfield{cat}{$d_1$}{\textit{Cat}} \\
                      \tfield{id}{\textit{Id}}} \\
\hspace*{2em} $\lambda r_2$:\record{\mfield{cat}{$d_2$}{\textit{Cat}} \\
                                    \tfield{id}{\textit{Id}}} \\
\hspace*{4em} (\record{\mfield{cat}{$m$}{\textit{Cat}} \\
                       \mfield{id}{$\mathit{id}$($r_1$.id)($r_2$.id)}{\textit{Id}} \\
                       \tfield{daughters}{\record{\mfield{first}{$r_1$}{\record{\mfield{cat}{$d_1$}{\textit{Cat}}}} \\
                                                  \tfield{rest}{\record{\mfield{first}{$r_2$}{\record{\mfield{cat}{$d_1$}{\textit{Cat}}}} \\
                                                                        \mfield{rest}{[]}{[\textit{Rec}]}}}}}})
\end{ex}

Like update functions, functions representing phrase-structure map
objects to a type.  Here the intuition is that if we believe that we
have  two objects of certain types then we can conclude that there is
an object of the type which is returned by the function.  The id-field in the type corresponding to the mother provides a
convenient representation of the function-argument structure
(corresponding to the abstract syntax tree in Ranta's (????) GF).  We
abbreviate this function (for the same reasons as we abbreviated the
lexical entries) as 

\{RuleBinary $\mathit{id}$ $m$ $d_1$ $d_2$\}

Similarly a unary rule with ID $\mathit{id}$ whose mother category is $m$
and whose daughter category is $d$ is modelled by the
function


$\lambda r$:\record{\mfield{cat}{$d$}{\textit{Cat}} \\
                                    \tfield{id}{\textit{Id}}} \\
\hspace*{2em} (\record{\mfield{cat}{$m$}{\textit{Cat}} \\
                       \mfield{id}{$\mathit{id}$($r$.id)}{\textit{Id}} \\
                       \tfield{daughters}{\record{\mfield{first}{$r$}{\record{\mfield{cat}{$d$}{\textit{Cat}}}} \\
                                                  \mfield{rest}{[]}{[\textit{Rec}]}}}})

and is abbreviated as

\{RuleUnary $\mathit{id}$ $m$ $d$\}


\begin{quote}
\{RuleBinary s\_np\_vp s np vp\} \\
\{RuleBinary np\_det\_n np det n\} \\
\{RuleBinary vp\_cop\_np vp cop np\} \\
\{RuleBinary d\_d\_s d d s\} \\
\{RuleUnary d\_s d s\}
\end{quote}
The last two rules are discourse rules which say that a discourse can
either consist of a discourse and a sentence or just a single
sentence.  The idea is that the grammar should give us enough
information to be able to parse a turn as a single unit and that a
turn may contain more than one sentence e.g. \textit{Ok. Dudamel is a
  composer.}

The concrete syntax for English refines these lexical entries and
rules using information from the resource grammar for English.  The
resource grammar provides phonological and morphological information,
including information about word order.  We will treat semantics
separately.  The resource grammar provides methods for refining
whatever abstract syntax has been defined.  In the definition of the
concrete syntax we will normally use abbreviations of the form

\{ResGram.$a$ Abs.$a$\}

where $a$ is some lexical item or rule introduced in the abstract
syntax.  `ResGram.$a$' is the refinement method defined in the
resource grammar for $a$ and `Abs.$a$' is the item defined in the
abstract syntax for $a$.  (Mentioning $a$ twice in the formulation allows us the
possibility of using different names in the resource grammar and
abstract syntax if we wish: \{ResGram.$b$ Abs.$a$\}.)  `ResGram' and
`Abs' are to be thought of as variables representing the particular
resource grammar and abstract syntax in use for the concrete syntax we
are defining.

The resource grammar makes use of a universal grammar resource in
defining the refinements.  Thus, for example, the English resource
grammar defines the proper name `Beethoven' to be associated with the
method

\{LexItemPhon $\alpha$ [beethoven]\}

where $\alpha$ is a record type (that provided by the abstract syntax)
and `[beethoven]' is our mock representation of the phonology -- a
list of words.  The universal grammar resource defines

\{LexItemPhon $\alpha$ $a$\}

to be

$\alpha$\d{$\wedge$}{\record{\mfield{phon}{$a$}{\textit{Phon}}}}

This is not particularly deep linguistic theory but it does allow us
to separate out the language particular fact of the phonology
associated with the proper name and the general theoretical decision
that phonology should be represented in a field labelled ``phon''.
And, as we pointed out before, our universal linguistic theory will
get more interesting as we progress.  In the case of words which
decline or conjugate the resource grammar yields a whole paradigm in
the form of a record.  Thus the English resource grammar defines the
word \textit{composer} to be associated with

\{RegN $\alpha$ composer\}

\{RegN $\alpha$ $w$\}

is defined to be

\record{\field{sg}{\{Singular \{LexItemPhon $\alpha$ [$w$] \}\}} \\
        \field{pl}{\{Plural \{LexItemPhon $\alpha$ [\{AddS $w$\}]\}\}}}

where \{Singular $\beta$\} and \{Plural $\beta$\} are defined in the
universal grammar resource to be respectively:

$\beta$\d{$\wedge$}\record{\tfield{agr}{\record{\mfield{num}{sg}{\textit{Number}}}}}

$\beta$\d{$\wedge$}\record{\tfield{agr}{\record{\mfield{num}{pl}{\textit{Number}}}}}

and \{AddS $w$\} abbreviates a morphophonological method defined in
the English resource grammar which adds an \textit{s}-morpheme to a
lexical item (which in turn makes use of phonological operations
defined in a universal phonology resource).

Each lexical entry in the concrete syntax chooses one element in the
paradigm.  Thus the concrete lexicon for our small grammar is defined
by\footnote{We ignore the fact that the indefinite article should
  actually provide a paradigm consisting of \textit{a} and \textit{an}}:

\{ResGram.Dudamel Abs.Dudamel\} \\
\{ResGram.Beethoven Abs.Beethoven\} \\
\{ResGram.IndefArt Abs.IndefArt\} \\
\{ResGram.composer Abs.composer\}.sg \\
\{ResGram.conductor Abs.conductor\}.sg \\
\{ResGram.be Abs.be\}.pres.third.sg \\
\{ResGram.ok Abs.ok\} \\
\{ResGram.aha Abs.aha\}

The refinement of rules also makes use of methods defined in the
English resource grammar.

\{ResGram.S\_NP\_VP Abs.S\_NP\_VP\} \\
\{ResGram.NP\_Det\_N Abs.NP\_Det\_N\} \\
\{ResGram.VP\_Cop\_NP Abs.VP\_Cop\_NP\} \\
\{ResGram.D\_D\_S Abs.D\_D\_S\} \\
\{ResGram.D\_S Abs.D\_S\}

The binary rules defined in the resource grammar all make use of a
universal resource which adds left-right concatenation of phonology to
a rule.  Left-right concatenation of phonology for a binary rule is expressed by the
function

\begin{ex}
\label{eg:lrconcatphon}
$\lambda r_1$ \record{\tfield{phon}{\textit{Phon}}} \\
\hspace*{2em} $\lambda r_2$ \record{\tfield{phon}{\textit{Phon}}} \\
\hspace*{4em} (\record{\mfield{phon}{$r_1$.phon$^{\frown}r_2$.phon}{\textit{Phon}}})
\end{ex}

where $r_1$.phon$^{\frown}r_2$.phon is the left-right concatenation of the two
phonologies. This function has to be merged with the function obtained
from the abstract grammar.  The merging of two functions,
$\mu(f_1,f_2)$ is defined by:
\begin{quote}
if $f_1$ is $\lambda v_1:T_1(\phi(v_1))$ and $f_2$ is $\lambda v_2:T_2(\psi(v_2))$ then $\mu(f_1,f_2)$ is $\lambda v_1:T_1$\d{$\wedge$}$T_2(\mu(\phi(v_1),\psi(v_1)))$

if $f_1$ and $f_2$ are record types then $\mu(f_1,f_2)$ is
$f_1$\d{$\wedge$}$f_2$

otherwise $\mu(f_1,f_2)$ is undefined

\end{quote}

This means that if we have an abstract rule of the form
(\ref{eg:abstractbinary}), the result of merging it with
(\ref{eg:lrconcatphon}) will be:

$\lambda r_1$:\record{\tfield{phon}{\textit{Phon}} \\
                      \mfield{cat}{$d_1$}{\textit{Cat}} \\
                      \tfield{id}{\textit{Id}}} \\
\hspace*{2em} $\lambda r_2$:\record{\tfield{phon}{\textit{Phon}} \\
                                    \mfield{cat}{$d_2$}{\textit{Cat}} \\
                                    \tfield{id}{\textit{Id}}} \\
\hspace*{4em} (\record{\mfield{phon}{$r_1$.phon$^{\frown}r_2$.phon}{\textit{Phon}} \\
                       \mfield{cat}{$m$}{\textit{Cat}} \\
                       \mfield{id}{$\mathit{id}$($r_1$.id)($r_2$.id)}{\textit{Id}} \\
                       \tfield{daughters}{\record{\mfield{first}{$r_1$}{\record{\mfield{cat}{$d_1$}{\textit{Cat}}}} \\
                                                  \tfield{rest}{\record{\mfield{first}{$r_2$}{\record{\mfield{cat}{$d_1$}{\textit{Cat}}}} \\
                                                                        \mfield{rest}{[]}{[\textit{Rec}]}}}}}})

This is the simplest kind of refinement from a rule of abstract syntax
to a rule of concrete syntax, just adding basic phonological
information to the rule.  Additional refinements for individual rules
can, for example, include constraints on agreement.  The universal
grammar resource provides us with a method for obtaining functions which require the second
daughter to agree with the first daughter on a certain agr-field
labelled $\ell$ associated with type $T$ (e.g. `num' and
\textit{Number} respectively): 



$\lambda r_1$: \smallrecord{\smalltfield{agr}{\smallrecord{\smalltfield{$\ell$}{$T$}}}} \\
\hspace*{2em} $\lambda r_2$: \smallrecord{\smalltfield{agr}{\smallrecord{\smalltfield{$\ell$}{$T$}}}} \\
\hspace*{4em} (\record{\tfield{daughters}{\record{\tfield{first}{\smallrecord{\smalltfield{agr}{\smallrecord{\smalltfield{$\ell$}{$T$}}}}} \\
                                                  \tfield{rest}{\record{\tfield{first}{\record{\tfield{agr}{\smallrecord{\smallmfield{$\ell$}{daughters.first.agr.$\ell$}{$T$}}}}}}}}}})

The English rule for subject-verb agreement will use this twice: once
for number agreement and once for person agreement.  Putting all this
together,  we obtain the following function as the instantiation of
\{ResGram.S\_NP\_VP Abs.S\_NP\_VP\}:

$\lambda r_1$: \record{\tfield{phon}{\textit{Phon}} \\
                      \mfield{cat}{np}{\textit{Cat}} \\
                      \tfield{agr}{\record{\tfield{num}{\textit{Number}} \\
                                           \tfield{pers}{\textit{Person}}}} \\
                      \tfield{id}{\textit{Id}}} \\
\hspace*{2em} $\lambda r_2$: \record{\tfield{phon}{\textit{Phon}} \\
                                    \mfield{cat}{vp}{\textit{Cat}} \\
                                    \tfield{agr}{\record{\tfield{num}{\textit{Number}} \\
                                                         \tfield{pers}{\textit{Person}}}} \\
                                    \tfield{id}{\textit{Id}}} \\
\hspace*{4em} (\smallrecord{\smallmfield{phon}{$r_1$.phon$^{\frown}r_2$.phon}{\textit{Phon}} \\
                       \smallmfield{cat}{s}{\textit{Cat}} \\
                       \smalltfield{daughters}{\smallrecord{\mfield{first}{$r_1$}{\record{\mfield{cat}{np}{\textit{Cat}} \\
                                                                                \tfield{agr}{\record{\tfield{num}{\textit{Number}} \\
                                                                                                     \tfield{pers}{\textit{Person}}}}}} \\
                                                  \tfield{rest}{\record{\mfield{first}{$r_2$}{\record{\mfield{cat}{vp}{\textit{Cat}} \\
                                                                                                      \tfield{agr}{\record{\mfield{num}{daughters.first.agr.num}{\textit{Number}} \\
                                                                                                                           \mfield{pers}{daughters.first.agr.pers}{\textit{Person}}}}}} \\
                                                                        \mfield{rest}{[]}{[\textit{Rec}]}}}}} \\
                       \smallmfield{id}{s\_np\_vp($r_1$.id)($r_2$.id)}{\textit{Id}}})

Now let us consider how to build a Swedish concrete syntax for this
abstract syntax.  The basic idea is that the definition of the Swedish
concrete syntax is the same as the definition of the English concrete
syntax except that now `ResGram' refers to the Swedish resource
grammar rather than the English one, with some adjustments for the
different morphologies (utrum gender and indefinite forms of nouns), represented below.  


\{ResGram.Dudamel Abs.Dudamel\} \\
\{ResGram.Beethoven Abs.Beethoven\} \\
\{ResGram.IndefArt Abs.IndefArt \}.utr \\
\{ResGram.tons�ttare Abs.composer\}.sg.indef \\
\{ResGram.dirigent Abs.conductor\}.sg.indef \\
\{ResGram.vara Abs.be\}.pres \\
\{ResGram.ok Abs.ok\} \\
\{ResGram.jaha Abs.aha\}


There is, however, a slight
complication with this simple picture when it comes to syntactic constructions.  Where English has

\begin{quote}
Dudamel is a conductor
\end{quote}

Swedish has

\begin{quote}
Dudamel \egstack{�r \\ \textit{is}} \egstack{dirigent \\ \textit{conductor}}
\end{quote}

without the indefinite article.  And similarly:

\begin{quote}
Beethoven \egstack{�r \\ \textit{is}} \egstack{tons�ttare \\
  \textit{composer}}
\end{quote}

In order to handle this we use the following rules:

\{ResGram.S\_NP\_VP Abs.S\_NP\_VP\} \\
\{ResGram.NP\_N \{UResAbs.ApplyRule Abs.NP\_Det\_N Abs.indefArt\}\} \\
\{ResGram.VP\_Cop\_NP Abs.VP\_Cop\_NP\} \\
\{ResGram.D\_D\_S Abs.D\_D\_S\} \\
\{ResGram.D\_S Abs.D\_S\}

With one exception these rules are the same as those we used for
English except that now `ResGram' is being used to refer to the
Swedish resource grammar.  The exception is that instead of the rule
that says that NPs may consist of a determiner followed by a noun we
are now using a rule that says that NPs may consist of just a noun.
This means that there is a mismatch between our concrete syntax and
the abstract syntax that we are using which has the rule with a
determiner.  Thus the concrete syntax rule here is not strictly a
refinement of the corresponding abstract syntax rule.  Rather it is a
refinement of the result of applying the abstract syntax rule NP\_Det\_N to the
abstract indefinite article.  The method for obtaining the result of
such an application in given in a universal resource concerned with
the construction and manipulation of abstract syntax representations.
The resulting concrete rule is:

$\lambda r:$\record{\tfield{phon}{\textit{Phon}} \\
                    \mfield{cat}{n}{\textit{Cat}} \\
                    \tfield{agr}{\textit{Rec}} \\
                    \tfield{id}{\textit{Id}}} \\
\hspace*{2em}(\record{\mfield{phon}{$r$.phon}{\textit{Phon}} \\
                      \mfield{cat}{np}{\textit{Cat}} \\
                      \mfield{agr}{daughters.first.agr}{\textit{Rec}} \\
                      \tfield{daughters}{\record{\mfield{first}{$r$}{\record{\mfield{cat}{n}{\textit{Cat}}}} \\
                                                 \mfield{rest}{[]}{\textit{Rec}}}} \\
                      \mfield{id}{np\_det\_n(`IndefArt')($r$.id)}{\textit{Id}}})

Note that the daughters-field reflects the concrete [$_\mathrm{NP}$ N]
structure whereas the id-field reflects the abstract [$_\mathrm{NP}$
Det N] structure with the indefinite article.  There is no particular
motivation for having the [$_\mathrm{NP}$ Det N] structure in abstract
syntax and then relating that to the [$_\mathrm{NP}$ N] in Swedish as
opposed to having the [$_\mathrm{NP}$ N] structure in the abstract
syntax and adding in the determiner in the concrete syntax of
English.  The reason is simply that we built our abstract syntax first
with English in mind.  We imagine that it might be the case that an
agent which first learns English and then learns Swedish would have
the analysis we have presented whereas an agent that learnt Swedish
first would have the alternative solution.  The exact nature of the
abstract syntax is not what is important for the characterization of
linguistic theory.  Linguistic universals lie rather in the nature
of the operations available to the language learner for relating
abstract structures to natural language utterances.  Our theory is not
so much about precisely which structures are associated with natural
language expressions as about what tools are available to an agent for
constructing representations and how these tools will be used given
the agent's previous linguistic experience.

This is, of course, a much simplified and stripped down discussion
which touches on a very complex issue of Swedish concerning when the
indefinite article is or is not to be used in such sentences.  For our
little example we are just taking the option of not using the
indefinite article but there is a great deal of subtlety in its use
which as far as we know is still not completely understood.


So far we have been representing the interpretation of the name
\textit{Dudamel} by something that looks like a logical constant
`Dudamel'.  Proper names in natural languages are context dependent.
Even within a restricted musical domain it is quite likely that
\textit{Strauss} can be used to refer to either Richard Strauss or
Johann Strauss.  Without an appropriate context to identify the person
being referred to we only get the information from a use of a proper
name that some individual with that name is being referred to.  We
know that an assumption is being made that we are in a context of a
certain type, i.e. one which contains an individual bearing the name.
This context type we will model as a record type such as:

\begin{ex}
\label{eg:dudamel-contexttype}
\record{\tfield{x}{\textit{Ind}} \\
        \tfield{c}{named(x, ``Dudamel'')}}
\end{ex}

We will use models to help us formalize the notion of contexts of this
type.  Our models will be very close to the kind of models which are
used for the intepretation of first order logic.  A model will be a
pair $\langle A,F \rangle$ where $A$ is a function which assigns sets
to all our basic types, such as \textit{Ind}, and $F$ is a function
which assigns a set of proof objects (possibly empty) to each
combination of a predicate with appropriate arguments, that is
arguments which correspond to the arity associated with the predicate,
specifying number, order and type of arguments associated with the
predicate.  In order to handle (\ref{eg:dudamel-contexttype}) we can
assume two basic types: \textit{Ind} and \textit{Name}.  A minimal
model to handle (\ref{eg:dudamel-contexttype}) could be one where $A$
is defined by
\begin{ex}
\label{eg:dudamel-model}
$A(\mathit{Ind})=\{d\}$ \\
$A(\mathit{Name})=\{\mathrm{``Dudamel"}\}$
\end{ex}
Assuming that the arity of the predicate `named' is $\langle
\mathit{Ind},\mathit{Named} \rangle$ the only type we can construct
with this predicate based on this model is 
\begin{quote}
named($d$,``Dudamel'')
\end{quote}
This means that we can define $F$ by
\begin{quote}
$F(\mathrm{named}(d,\mathrm{``Dudamel"}))=\{p\}$
\end{quote}
The exact nature of the proof object $p$ is very often not important.  In many
applications we are only interested in whether a given type
constructed with a predicate as assigned the empty set or a non-empty
set (corresponding to the two truth values of classical logic).
However, our type theoretical framework gives us the power to make
fine-grained distinctions among proof objects and to make distinctions
according to the number of proof objects available (since we are
dealing with sets of proof objects).

The models are defined in order to define what objects are of certain
types where this is not defined by the general type theory.  We have
formulated them in this way in order to draw an obvious parallel with
classical first order models.  However, the same effect can be
achieved by adding the following judgements to our type theory
\begin{quote}
$d$ : \textit{Ind} \\
``Dudamel'' : \textit{Name} \\
$p$ : named($d$,``Dudamel'')
\end{quote}
together with a general requirement that nothing else belongs to these
types except as required above.

Now that we have a model we can \textit{instantiate}
(\ref{eg:dudamel-contexttype}) in this model.  The only instantiation
available to it in this model is

\record{\field{x}{$d$} \\
        \field{c}{$p$}}

To instantiate a record type in a model we construct a record with
fields with the same labels and with values which are instantiations
of the types in the record-type fields.  The instantiation of a basic
type or a type constructed with a predicate is one of the objects
assigned to it in the model.  This, then is a record, based on the
model, which represents a
context of the type (\ref{eg:dudamel-contexttype}).  Note that this
record does not occur in the model.  The model represents the world as
it is.  When I go to a concert given by the Gothenburg Symphony
Orchestra conducted by Gustavo Dudamel there are certainly many
mysterious objects the exact nature and ontological status of which is
still quite unclear to us, such as symphonies, performances, musical
interpretations, an orchestra which seems to take on a different
extension for different pieces of music.  Even Dudamel himself, an
individual if ever there was one, is ontologically mysterious.  To
what extent does his, or any other individual's, individuation depend
on our cognition?  What seems clear, though, is that none of these
objects, whatever they are, are records.  They do not come structured
into collections of fields with labels attached.  Our idea is that the
mathematical objects we are calling records model something that
arises in cognition.  The labels give us handles (in the computational
sense) that enable cognitive processes such as compositional
interpretation and anaphora.

This is not the only record of the type (\ref{eg:dudamel-contexttype})
which is yielded by our model.  Even if we tie down the particular
individual which is Dudamel and the particular proof that he is called
by the name ``Dudamel'' there could be other fields in the record and
these would not change the fact that the record is of the type
(\ref{eg:dudamel-contexttype}).  When we determine that for a given
utterance of the name \textit{Dudamel} the referent will be Gustavo
Dudamel, among other things Music Director of the Gothenburg Symphony
Orchestra, we do not thereby fix a specific context.  Rather we refine
the context type from (\ref{eg:dudamel-contexttype}) to

\record{\mfield{x}{$d$}{\textit{Ind}} \\
        \mfield{c}{$p$}{named(x, ``Dudamel'')}}

This is a \textit{fully specified} record type in that all its fields
are manifest, but it is nevertheless not a singleton type because it
has records with additional fields among its members.  

It is important that we model reference resolution in terms of context
type refinement rather than context type instantiation.  Suppose that
we are interpreting something containing two occurrences of proper
names, say \textit{Dudamel} and \textit{J�rvi}.  At a certain point in
the interpretation process we may have determined the referent of
\textit{Dudamel} but not that of \textit{J�rvi} (possibly Neeme J�rvi,
principal conductor emeritus of the Gothenburg Symphony Orchestra,
possibly one of his children Paavo, Kristjan (both conductors) and
Maarika (flautist) and so on).  We can model this partial resolution
by the partially specified record type

\begin{ex}
\label{eg:DudamelJaervi-partialres}
\record{\mfield{x$_1$}{$d$}{\textit{Ind}} \\
        \mfield{c$_1$}{$p$}{named(x$_1$, ``Dudamel'')} \\
        \tfield{x$_2$}{\textit{Ind}} \\
        \tfield{c$_2$}{named($x_2$, ``J�rvi'')}}
\end{ex}

Refining a type with respect to a model is like instantiating a type
with respect to that model except that instead of constructing a
record with appropriate values from the model we construct a new type
with singleton types for the values in the way the we have shown
above.  If no appropriate value is available from the model for any
field in the type then the result of the refinement is the original
type.  This means that the refinement of the type

\begin{ex}
\label{eg:DudamelJaervi-type}
\record{\tfield{x$_1$}{\textit{Ind}} \\
        \tfield{c$_1$}{named(x$_1$, ``Dudamel'')} \\
        \tfield{x$_2$}{\textit{Ind}} \\
        \tfield{c$_2$}{named($x_2$, ``J�rvi'')}}
\end{ex}

in the model (\ref{eg:dudamel-model}) is (\ref{eg:DudamelJaervi-type})
itself.  It cannot be refined further in (\ref{eg:dudamel-model}) since there is no individual
named J�rvi in the model.  How, then, can we go about obtaining
partial specifications such as (\ref{eg:DudamelJaervi-partialres})?  Note
that while (\ref{eg:DudamelJaervi-type}) cannot be refined further in
(\ref{eg:dudamel-model}), (\ref{eg:dudamel-contexttype}) can.
(\ref{eg:DudamelJaervi-partialres}) could be defined for example as

\record{\tfield{x$_1$}{\textit{Ind}} \\
        \tfield{c$_1$}{named(x$_1$,
          ``Dudamel'')}}$_{(\ref{eg:dudamel-model})}$ \d{$\wedge$} \record{
                                                                  \tfield{x$_2$}{\textit{Ind}} \\
                                                                  \tfield{c$_2$}{named($x_2$,
                                                                    ``J�rvi'')}}$_{(\ref{eg:dudamel-model})}$

where we use $T_M$ to represent a refinement of $T$ in model $M$.  The
idea is that context type refinement will be carried out individually
on natural language constituents such as noun-phrases and the results
assembled by simplifying meets (i.e. \d{$\wedge$}).  Note that it is
\textit{not} in general the case that
${T_1}_M$\d{$\wedge$}${T_2}_M$=($T_1$\d{$\wedge$}$T_2$)$_M$.

We are being a bit loose here in referring to $T_M$ as \textit{a}
refinement of $T$ in $M$.  There could, of course, be several objects
in $M$ fulfilling the constraints of $T$ and thus several different ways of
refining it.  One way to make refinement in a model functional is to
only look at unique refinements and to say that $T_M=T$ in cases where
there is more than one possible way of doing the refinement.  Another
way is to work with sets of refinements  and a third way is to have a choice function
which picks out one particular refinement when there are many
(possibly like an $\epsilon$ operator).  


[from ddl-final]

In order to illustrate how the content we have given for \textit{ran}
in section~\ref{sec:TTR-verbs} figures in a grammar and compositional semantics we shall
define a toy grammar which covers the sentence \textit{Sam ran}. 

We will present our grammar in terms of signs (in a similar sense to
HPSG, see, for example, \cite{Sag:Wasow:ea:03}).  Our signs will be
records of type \textit{Sign} which for present purposes we will take
to be the type:
\begin{display}
\record{\tfield{s-event}{\textit{SEvent}}
                               \\
        \tfield{synsem}{\record{\tfield{cat}{\textit{Cat}}\\
                                \tfield{cnt}{\textit{Cnt}}}}}
\end{display}
\noindent We shall spell out the nature of the types \textit{SEvent},
\textit{Cat} and \textit{Cnt} below.  A sign 
has two main components, one corresponding to the physical nature of
the speech event (`s-event') and the other to its interpretation
(syntax and semantics, `synsem', using the label which is
well-established in HPSG).

\textit{SEvent} is the type 
\begin{quote}
\record{\tfield{phon}{\textit{Phon}}\\
                                 \tfield{s-time}{\record{\tfield{start}{\textit{Time}}
                                     \\
                                                         \tfield{end}{\textit{Time}}}}\\
                                 \tfield{utt$_\mathrm{at}$}{$\langle
                                   \lambda v_1:$\textit{Str}$(\lambda
                                   v_2:$\textit{Time}($\lambda
                                   v_3:$\textit{Time}(uttered\_at($v_1$,$v_2$,$v_3$)))), \\
                                   & &
                                   \hspace*{2em}$\langle$s-event.phon,
                                   s-event.s-time.start,
                                   s-event.s-time.end$\rangle\rangle$}}
\end{quote}

   In the s-event component the
phon-field represents the phonology of an expression. Here we will
take phonology
as a string of word utterances although in a complete treatment of spoken
language we would need phonological and phonetic attributes.  That is
we take \textit{Phon} to be \textit{Wrd}$^+$ where \textit{Wrd} (the
type of word utterances) is defined in the lexicon.  The
s-time (``speech time'') field represents the starting and ending time
for the utterance.  We assume the existence of a predicate
`uttered\_at' with arity
$\langle\mathit{Phon},\mathit{Time},\mathit{Time}\rangle$.  An object of type
`uttered\_at($a$,$t_1$,$t_2$)' could be an event where $a$ is uttered
beginning at $t_1$ and ending at $t_2$ or a corresponding hypothesis
produced by a speech recognizer with time-stamps, depending on the application of the
theory.  In a more complete treatment we would need additional
information about the physical nature of the speech event, such as the
identity of the speaker and where it took place.  

In the synsem component
the cat-field introduces a category for the phrase.  For present
purposes we will require that the following hold of the type
\textit{Cat}:
\begin{quote}
s, np, vp, n$_{\mathrm{prop}}$, v$_{\mathrm{i}}$ : \textit{Cat}
\end{quote}
The objects of type \textit{Cat} (s, np, vp etc.) are regarded as
convenient abstract objects which are used to categorize classes of
speech events.

The cnt-field represents the content or interpretation of the
utterance.  
Since the content types become rather long we will introduce
abbreviations to make them readable:
\begin{quote}
\textit{Ppty}, ``property'' is to be
\smallrecord{\smalltfield{x}{\textit{Ind}}}$\rightarrow$\textit{RecType}
  \\
\textit{Quant}, ``quantifier'' is to be
\textit{Ppty}$\rightarrow$\textit{RecType} %\\
% if $T$ is a type then $^\sigma T$ is to be
% \textit{Seq}$_{\mathit{Ind}}\rightarrow T$ 
\end{quote}
We only use a small finite number of function types for content
types and thus we are able to define the type \textit{Cnt} for present
purposes as
\begin{quote}
\textit{RecType}$\vee$(\textit{Ppty}$\vee$\textit{Quant})
% \ignore{\smallrecord{\smalltfield{x}{\textit{Ind}}} $\vee$\\(}$^\sigma$\textit{RecType}
% $\vee$\\($^\sigma$\textit{Prop} $\vee$\\($^\sigma$\textit{Quant}
% %$\vee$\\($^\sigma$(\textit{Prop}$\rightarrow$\textit{Prop})
% $\vee$\\($^\sigma$(\textit{Quant}$\rightarrow$\textit{Prop})
% $\vee$\\($^\sigma$(\textit{RecType}$\rightarrow$\textit{Prop})
% $\vee$\\($^\sigma$(\textit{Prop}$\rightarrow$\textit{Quant}) 
% %$\vee$\\$^\sigma$(\textit{Quant}$\rightarrow$(\textit{Prop}$\rightarrow$\textit{Prop}))
% )))))\ignore{))}
\end{quote}
This makes use of join types which are defined in a similar way to
meet types:
\textbf{TYPE}$_C$ = $\langle${\bf Type}, {\bf BType},
$\langle$\textbf{PType}, {\bf Pred}, \textbf{ArgIndices}, {\it
  Arity\/}$\rangle$, $\langle A,F\rangle$$\rangle$ \textit{has join types} if
\begin{enumerate} 
 
\item for any $T_1,T_2 \in \mathbf{Type}$, $(T_1\vee T_2) \in \mathbf{Type}$ 
 
\item for any $T_1,T_2 \in \mathbf{Type}$, $a:_{\mathbf{TYPE_C}}(T_1\vee T_2)$ iff
  $a:_{\mathbf{TYPE_C}}T_1$ or $a:_{\mathbf{TYPE_C}}T_2$ 
 
\end{enumerate} 
% This means that we can define \textit{Sign}, the type of signs, to be
% \begin{display}
% \record{\tfield{s-event}{\record{\tfield{phon}{\textit{Str}}\\
%                                  \tfield{s-time}{\record{\tfield{start}{\textit{Time}}
%                                      \\
%                                                          \tfield{end}{\textit{Time}}}}\\
%                                  \tfield{utt$_\mathrm{at}$}{$\langle
%                                    \lambda v_1:$\textit{Str}$(\lambda
%                                    v_2:$\textit{Time}($\lambda
%                                    v_3:$\textit{Time}(uttered\_at($v_1$,$v_2$,$v_3$)))), \\
%                                    & &
%                                    \hspace*{2em}$\langle$s-event.phon,
%                                    s-event.s-time.start, s-event.s-time.end$\rangle\rangle$}}}
%                                \\
%         \tfield{synsem}{\record{\tfield{cat}{\textit{Cat}}\\
%                                 \tfield{cnt}{\textit{CntType}}}}}

% \end{display}

We will present first the lexicon and then rules for
combining phrases.  

\begin{ex} 
If $T_1$ and $T_2$ are types and
$f:(T_1\rightarrow(T_2\rightarrow\mathit{Type}))$ is a chart update
function, $A$ is an agent, $s_i$ is $A$'s current information state and
$s_i:_A T_i$ where $T_i\sqsubseteq T_1$, then an event $e:_A T_2$ (and
$e:T_2$) licenses $s_{i+1}:_A T_i\oplus f(s_i)(e)$
\end{ex} 
Here we use `$\oplus$' as a variant of asymmetric merge
(`\fbox{\d{$\wedge$}}').  It is defined by the clauses in \nexteg{}.
\begin{ex} 
\begin{subex} 
 
\item $(T_1^{\frown}\ldots{^{\frown}T_n})\oplus
  ({T'_1}^{\frown}\ldots{^{\frown}T'_m}) = \\
 T_1^{\frown}\ldots{^{\frown}(T_n}$\d{$\wedge$}$T'_1)^{\frown}\ldots{^{\frown}T'_m}$ 
 
\item If either $T_1$ or $T_2$ is not a string type then $T_1\oplus
  T_2 = T_1$\fbox{\d{$\wedge$}}$T_2$ 
 
\end{subex} 
   
\end{ex} 

%%% Local Variables: 
%%% mode: latex
%%% TeX-master: t
%%% End: 
