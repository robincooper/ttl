\chapter{Intensional logic}
\label{app:intlog}

\section{Interpreting intensional logic in TTR}

Montague's (\citeyear{Montague1974}, ch. 8\footnote{'The Proper
  Treatment of Quantification in Ordinary English'}, PTQ) intensional logic
extends first order logic in four ways:
\begin{enumerate} 
 
\item it adds a simply typed lambda calculus -- predicates are
  reconstructed as function types and arguments to predicates can
  be of any type 
 
\item it allows quantification over variables of any type (making it
  into a higher order logic)

\item it adds modal and tense operators

\item types corresponding to intensions are introduced together with
  logical expressions which denote intensions as well as an operator
  which maps an intension-denoting expression to the corresponding
  extension and another operator which maps an expression, $\alpha$,
  to an expression denoting the intension of $\alpha$.
 
\end{enumerate} 
Montague's system of types is defined by the following recursive
definition:
\begin{enumerate} 
 
\item $e$ and $t$ are types (entities and truth values respectively) 
 
\item if $T_1$ and $T_2$ are types then $\langle T_1,T_2\rangle$ is a
  type (the type of functions from objects of type $T_1$ to objects of
  type $T_2$)

\item if $T$ is a type then $\langle s,T\rangle$ is a type (the type
  of intensions corresponding to type $T$, functions from possible
  worlds and times to objects of type $T$)
 
\end{enumerate}
We define a function $\tau$ that maps Montague's types to types in
TTR:
\begin{enumerate} 
 
\item $\tau(e)=$\smallrecord{\smalltfield{x}{\textit{Ind}}}\footnote{Note that
    `x' is a particular label, not a variable over labels.}

$\tau(t)=$\textit{RecType} 
 
\item $\tau(\langle T_1,T_2\rangle)=\tau(T_1)\rightarrow\tau(T_2)$

\item $\tau(\langle s,T\rangle)=\tau(T)$
 
\end{enumerate} 
Note that Montague's type $e$ (entities) corresponds to a type of records
containing an entity.  The fact that we have a whole structure here
with potentially extra fields will become important in our treatment
of Montague's examples involving individual concepts such as
temperatures and prices. In place of truth values we have record
types.  These are the intensional objects which will play the role of
Montague's propositions. We rely on the structure and intensionality
associated with these types rather than on the intensional types that Montague
introduces which we collapse with the corresponding non-intensional type.    

We will use the definition of the syntax of intensional logic given in PTQ with an
infinite set of constants $C_T$ and a countably infinite set of
variables $V_T$ for each Montague type $T$.  We shall use a particular
set of labels $L$ and an
intensional modal system of complex types based on a collection of
stratified models, $\mathfrak{M}$, {\bf TYPE$_\mathit{IMC}$} = $\langle${\bf Type}$^n$, {\bf BType},
$\langle$\textbf{PType}$^n$, {\bf Pred}, \textbf{ArgIndices}, {\it
  Arity\/}$\rangle$,
$\mathscr{M}_n$$\rangle_{\mathscr{M}\in\mathfrak{M},n\in\mathit{Nat}}$,
where for each $\mathscr{M}\in\mathfrak{M}$, $\langle${\bf Type}$^n$, {\bf BType},
$\langle$\textbf{PType}$^n$, {\bf Pred}, \textbf{ArgIndices}, {\it
  Arity\/}$\rangle$, $\mathscr{M}_n$$\rangle_{n\in\mathit{Nat}}$ is an
  intensional system of complex types with dependent record types
  based on $\langle L,
\mathbf{RType}^n\rangle_{n\in\mathit{Nat}}$ for 
\textbf{RType}$^n$ as required by the definition of systems with
dependent record types. We place the following conditions on this
system:
\begin{enumerate} 
 
\item \textit{Ind}, \textit{Time} $\in$ \textbf{BType} (the types of
  individuals and points of time) 

\item for all $\mathscr{M},\mathscr{M'}\in\mathfrak{M}$,
  $A_{\mathscr{M}}(\mathit{Time})=A_{\mathscr{M'}}(\mathit{Time})$ and
  $A_{\mathscr{M}}(\mathit{Time})$ is totally ordered and dense using
  the natural ``earlier than'' relation which we will represent by
  $\leq$ and its strict version $<$.

\item the set of labels $L$ includes the distinguished label
 e-time (event time) 
 
\item \textbf{Pred} contains distinguished predicates
  corresponding to the tense and modal operators of intensional logic
  and equality:
\begin{itemize} 
 
\item poss (``possible'') with arity
  $\langle\mathit{Type}^n\rangle_{n>0}$\footnote{Montague did not
    include the possibility operator in his definition of intensional logic.} 
 
\item nec (``necessary'') with arity $\langle\mathit{Type}^n\rangle_{n>0}$

\item fut (``future'') with arity
  $\langle\mathit{RecType}^n,\mathit{Time}\rangle_{n>0}$

\item past (``past'') with arity
  $\langle\mathit{RecType}^n,\mathit{Time}\rangle_{n>0}$

\item eq (``equals'') with arity $\langle
  T,T\rangle_{T\in\bigcup_{n\in\mathit{Nat}}\mathbf{Type^n}}$

\item pre (``precedes'') with arity $\langle \mathit{Time},\mathit{Time}\rangle$
 
\end{itemize}
We place the following constraints on these predicates:
\begin{itemize} 
 
\item
  $a:_{\mathbf{TYPE}_{\mathit{IMC}_{\mathscr{M}_n}}}\mathrm{poss}(T)$
  iff for some $\mathscr{M}'\in\mathfrak{M}$, $a:_{\mathbf{TYPE}_{\mathit{IMC}_{{\mathscr{M}'}_n}}}T$ 
 
\item $a:_{\mathbf{TYPE}_{\mathit{IMC}_{\mathscr{M}_n}}}\mathrm{nec}(T)$
  iff for all $\mathscr{M}'\in\mathfrak{M}$,
  $a:_{\mathbf{TYPE}_{\mathit{IMC}_{{\mathscr{M}'}_n}}}T$\footnote{Montague
    interpreted the necessity operator in intensional logic as
    ``necessarily always'' quantifying over times in addition to
    possibilities represented by possible worlds. We could define a
    combined modal and tense predicate nec-alw with arity $\langle\mathit{RecType}^n,\mathit{Time}\rangle_{n>0}$.}

\item $a:_{\mathbf{TYPE}_{\mathit{IMC}_{\mathscr{M}_n}}}\mathrm{fut}(T,t)$
  iff \\ 
  $a:_{\mathbf{TYPE}_{\mathit{IMC}_{\mathscr{M}_n}}}T$\d{$\wedge$}
\record{\tfield{e-time}{\textit{Time}} \\
        \tfield{c$_{\mathrm{tns}}$}{$\langle\lambda
          v:$\textit{Time}(pre($t,v$)), $\langle$ e-time $\rangle\rangle$}}

\ignore{for some
  $t':_{\mathbf{TYPE}_{\mathit{IMC}_{\mathscr{M}_n}}}\mathit{Time}$
  such that $t<t'$,($t'$/\smallrecord{\smalltfield{e-time}{\textit{Time}}})}

\item \label{ilrule:past} $a:_{\mathbf{TYPE}_{\mathit{IMC}_{\mathscr{M}_n}}}\mathrm{past}(T,t)$
  iff \\
 $a:_{\mathbf{TYPE}_{\mathit{IMC}_{\mathscr{M}_n}}}T$\d{$\wedge$}
\record{\tfield{e-time}{\textit{Time}} \\
        \tfield{c$_{\mathrm{tns}}$}{$\langle\lambda
          v:$\textit{Time}(pre($v,t$)), $\langle$ e-time $\rangle\rangle$}}

\ignore{for some
  $t':_{\mathbf{TYPE}_{\mathit{IMC}_{\mathscr{M}_n}}}\mathit{Time}$
  such that $t'<t$,
  $a:_{\mathbf{TYPE}_{\mathit{IMC}_{\mathscr{M}_n}}}T$($t'$/\smallrecord{\smalltfield{e-time}{\textit{Time}}})}

\item
  $a:_{\mathbf{TYPE}_{\mathit{IMC}_{\mathscr{M}_n}}}\mathrm{eq}(b,c)$
  iff $a=b=c$

\item
  $a:_{\mathbf{TYPE}_{\mathit{IMC}_{\mathscr{M}_n}}}\mathrm{pre}(t,t')$
  iff $a=\langle t,t'\rangle$ and $t<t'$
\end{itemize} 
   
  
 
\item for any type of intensional logic, $T$,
  we introduce a function $\kappa_T$ from $C_T$,
  and $\mathfrak{M}$ to a partial function from \textit{Nat} such that for any $c\in C_T$,
  $\mathscr{M}\in\mathfrak{M}$ and $n\in\mathit{Nat}$, 
\begin{enumerate} 
 
\item either
  $\kappa_T(c,\mathscr{M})(n):_{\mathbf{TYPE}_{\mathit{IMC}_{\mathscr{M}_n}}}\tau(T)$
  or $\kappa_T(c,\mathscr{M})(n)$ is undefined
 
\item if $\kappa_T(c,\mathscr{M})(n)=a$ then
  $\kappa_T(c,\mathscr{M})(n+1)=a$ 

\item for $n>0$

\begin{enumerate}
\item if $\kappa_T(c,\mathscr{M})(n)=a$ and
  $a:_{\mathbf{TYPE}_{\mathit{IMC}_{\mathscr{M}_{n-1}}}}\tau(T)$, then
  $\kappa_T(c,\mathscr{M})(n-1)=a$

\item if $\kappa_T(c,\mathscr{M})(n)=a$ and
  it is not the case that $a:_{\mathbf{TYPE}_{\mathit{IMC}_{\mathscr{M}_{n-1}}}}\tau(T)$, then
  $\kappa_T(c,\mathscr{M})(n-1)$ is undefined

\item if $\kappa_T(c,\mathscr{M})(n)$ is undefined, then
  $\kappa_T(c,\mathscr{M})(n-1)$ is undefined

\end{enumerate}



 
\end{enumerate} 

(That is, a constant of type $T$, can be interpreted as different
objects of type $\tau(T)$ in different stratified models -- they may
be ``non-rigid designators''.  Within a single stratified model,
however, they may only correspond to one object and must designate
that object at all levels at which it exists.  A constant is undefined
at levels below which the object exists.)

  
 \item we introduce a one-one function, $\gamma$, from the union of all the sets,
  $V_T$, of variables of all the types $T$ of intensional logic into
  $L$ (a one-one function from variables of intensional logic into the
  set of labels used in record types) 

\item there is a countably infinite proper subset of $L$, $L_c$, whose
  members are represented by $c_i$ where $i$ is a natural number.
  (This is the set of labels used for ``constraints'' corresponding to
  formulae of intensional logic.)  $L_c$, the range of $\gamma$ and
  the set of distinguished labels we have introduced are
  disjoint.
 
\end{enumerate} 
  
We define a function \mng{.}$^{.,.,.,.,.,.}$ which maps an expression,
$\alpha$, of intensional logic to a function from an assignment, $\sigma$, of
countably infinite sequences to TTR types corresponding to intensional
logic types (that is, the range of $\tau$), such that if $T$ is a type
of intensional logic, $\sigma(\tau(T))$ is a sequence of objects of
type $\tau(T)$. (Below we shall refer to $\sigma$ as a type-sequence
assignment.  We shall use the notation $\sigma(a/T_n)$ to represent a
type-sequence assignment like $\sigma$ except that $\sigma(T)(n)=a$.)  The mapping \mng{.}$^{.,.,.,.,.,.}$ is relative to a natural
number, $n$, a path, $\pi$, a function, $g$, from the union of the sets of
intensional logic variables, $V_T$, for intensional logic types $T$ to
natural numbers, a
time, $t$, a stratified model, $\mathscr{M}$, and order, $\omega$.  We define \iltrans{$\alpha$} as
follows\footnote{following the characterization of the semantics of
  intensional logic starting on p. 258 of \cite{Montague1974}}:
\begin{enumerate} 
 
\item \label{ilrule:constant} if $\alpha$ is a constant of intensional logic type $T$, then
  \iltrans{$\alpha$} is that function $f$ such that for any
  type-sequence assignment, $\sigma$, $f(\sigma)=\kappa_T(\alpha,\mathscr{M})(\omega)$. 
 
\item \label{ilrule:variable} if $\alpha$ is a variable of intensional logic type $T$, then
  \iltrans{$\alpha$} is that function $f$ such that for any
  type-sequence assignment, $\sigma$,
  $f(\sigma)=\sigma(\tau(T))_{g(\alpha)}$.

\item \label{ilrule:abstraction} if $\alpha$ is an expression of intensional logic type $T_1$ and
  $u$ is a variable of intensional logic type $T_2$, then
  \iltrans{$\lambda u\alpha$} is that function $f$ such that for any
  type-sequence assignment, $\sigma$, $f(\sigma)=\lambda
  v_n:\tau(T_2)$(\mng{$\alpha$}$^{n+1,\pi,g(n/u),t,\mathscr{M},\omega}(\sigma(v_n/{\tau(T_1)}_n))$.

\item \label{ilrule:application} if $\alpha$ is an expression of intensional logic type $\langle
  T_1,T_2\rangle$ and $\beta$ is an expression of intensional logic
  type $T_1$, then \iltrans{$\alpha(\beta)$} is that function $f$ such
  that for any type-sequence assignment, $\sigma$, $f(\sigma)=$\iltrans{$\alpha$}($\sigma$)(\iltrans{$\beta$}($\sigma$)).

\item if $\alpha$ and $\beta$ are expressions of intensional logic type
  $T$, then \iltrans{$\alpha=\beta$} is that function $f$ such that
  for any type-sequence assignment, $\sigma$, $f(\sigma)$=
  \record{\tfield{c$_n$}{eq(\mng{$\alpha$}$^{n+1,\pi,g,t,\mathscr{M},\omega}$($\sigma$),\mng{$\beta$}$^{n+1,\pi,g,t,\mathscr{M},\omega}$($\sigma$))}}

\item 
\begin{enumerate} 
 
\item \label{ilrule:negation} if $\phi$ is a formula of intensional logic and
  $\alpha=\neg\phi$, then \iltrans{$\alpha$} is that function
  $f$ such that for any type-sequence assignment, $\sigma$, $f(\sigma)=$\record{\tfield{c$_n$}{\mng{$\phi$}$^{n+1,\pi,g,t,\mathscr{M},\omega}$($\sigma$) $\rightarrow$
  $\bot$}}. 
 
\item if $\phi$ and $\psi$ are formulae of first order logic and
  $\alpha=\phi\wedge\psi$, then \iltrans{$\alpha$} is that function
  $f$ such that for any type-sequence assignment, $\sigma$,
  $f(\sigma)=$\record{\tfield{c$_n$}{\mng{$\phi$}}$^{n+1,\pi,g,t,\mathscr{M},\omega}$($\sigma$) \d{$\wedge$}
  \mng{$\psi$}$^{n+2,\pi,g,t,\mathscr{M},\omega}$($\sigma$)}. 

\item if $\phi$ and $\psi$ are formulae of intensional logic and
  $\alpha=\phi\vee\psi$, then \iltrans{$\alpha$} is that function
  $f$ such that for any type-sequence assignment, $\sigma$,
  $f(\sigma)=$\record{\tfield{c$_n$}{\mng{$\phi$}$^{n+1,\pi,g,t,\mathscr{M},\omega}$($\sigma$) $\vee$
  \mng{$\psi$}$^{n+2,\pi,g,t,\mathscr{M},\omega}$($\sigma$)}}.

\item if $\phi$ and $\psi$ are formulae of intensional logic and
  $\alpha=\phi\rightarrow\psi$, then \iltrans{$\alpha$} is that function
  $f$ such that for any type-sequence assignment, $\sigma$,
  $f(\sigma)=$\record{\tfield{c$_n$}{\mng{$\phi$}$^{n+1,\pi,g,t,\mathscr{M},\omega}$($\sigma$) $\rightarrow$
  \mng{$\psi$}$^{n+2,\pi,g,t,\mathscr{M},\omega}$($\sigma$)}}.

\item if $\phi$ and $\psi$ are formulae of intensional logic and
  $\alpha=\phi\leftrightarrow\psi$, then \iltrans{$\alpha$} is that function
  $f$ such that for any type-sequence assignment, $\sigma$,
  $f(\sigma)=$\record{\tfield{c$_n$}{\mng{$\phi$}$^{n+1,\pi,g,t,\mathscr{M},\omega}$($\sigma$) $\rightarrow$
  \mng{$\psi$}$^{n+2,\pi,g,t,\mathscr{M},\omega}$($\sigma$)} \\
                      \tfield{c$_{n+3}$}{\mng{$\psi$}$^{n+4,\pi,g,t,\mathscr{M},\omega}$($\sigma$) $\rightarrow$
  \mng{$\phi$}$^{n+5,\pi,g,t,\mathscr{M},\omega}$($\sigma$)}}. 
 
\end{enumerate} 

\item 
\begin{enumerate} 
 
\item if $\phi$ is a formula of intensional logic and $u$ is a variable of intensional
  logic of type $T$ and $\alpha=\exists u\phi$, then \iltrans{$\alpha$} is that function
  $f$ such that for any type-sequence assignment, $\sigma$,
  $f(\sigma)=$
\record{\tfield{$\gamma(u)$}{$T$} \\
        \tfield{c$_n$}{$\langle\lambda v_n:T$
          (\mng{$\phi$}$^{n+1,\pi.c_n,g(n/u),t,\mathscr{M},\omega}$($\sigma(v_n/T_n)$)),
          $\langle\pi.\gamma(u)\rangle\rangle$}}. 
 
\item if $\phi$ is a formula of intensional logic and $u$ is a
  variable of type $T$ of intensional
  logic and $\alpha=\forall u\phi$, then \iltrans{$\alpha$} is that function
  $f$ such that for any type-sequence assignment, $\sigma$,
  $f(\sigma)=$
\smallrecord{\smalltfield{c$_n$}{($r_n$:\smallrecord{\smalltfield{$\gamma(u)$}{$T$}})
    $\rightarrow$ $\lambda r_{n+1}:$\smallrecord{\smalltfield{$\gamma(u)$}{$T$}}(\mng{$\phi$}$^{n+2,\emptyset,g(n+1/u),t,\mathscr{M},\omega}$($\sigma(r_{n+1}.\gamma(u)/T_{n+1})$))($r_n$)}}. 
 
\end{enumerate} 

\item 
\begin{enumerate} 
 
\item if $\phi$ is a formula of intensional logic and
  $\alpha=\Box\phi$ then \iltrans{$\alpha$} is that function $f$ such
  that for any type-sequence assignment, $\sigma$, $f(\sigma)$ is \record{\tfield{c$_n$}{nec(\mng{$\phi$}$^{n+1,\emptyset,g,t,\mathscr{M},\omega}$($\sigma$))}}
 
\item if $\phi$ is a formula of intensional logic and
  $\alpha=\Diamond\phi$\footnote{Montague does not include
    $\Diamond\phi$ as an expression in his intensional logic but we
    treat it here for the sake of completeness.} then
  \iltrans{$\alpha$} is that function $f$ such that for any
  type-sequence assignment, $\sigma$, $f(\sigma)$ is 
  \record{\tfield{c$_n$}{poss(\mng{$\phi$}$^{n+1,\emptyset,g,t,\mathscr{M},\omega}$($\sigma$))}}

\item if $\phi$ is a formula of intensional logic and
  $\alpha=W\phi$ then \iltrans{$\alpha$} is that function
  $f$ such that for any type-sequence assignment, $\sigma$,
  $f(\sigma)$ is
  \record{\tfield{c$_n$}{fut(\mng{$\phi$}$^{n+1,\emptyset,g,t,\mathscr{M},\omega}$($\sigma$),$t$)}} 

\item \label{ilrule:H} if $\phi$ is a formula of intensional logic and
  $\alpha=H\phi$ then \iltrans{$\alpha$} is that function $f$ such
  that for any type-sequence assignment, $\sigma$, $f(\sigma)$ is
  \record{\tfield{c$_n$}{past(\mng{$\phi$}$^{n+1,\emptyset,g,t,\mathscr{M},\omega}$($\sigma$),$t$)}}
 
\end{enumerate} 

\item if $\alpha$ is an expression of intensional logic, then
  \iltrans{$\hat{\ }\alpha$} is \iltrans{$\alpha$}.

\item if $\alpha$ is an expression of intensional logic of type
  $\langle s,T\rangle$, then \iltrans{$\check{\ }\alpha$} is \iltrans{$\alpha$}.
 
\end{enumerate} 

The \textit{TTR interpretation}, \mng{$\alpha$}$^{g,t,\mathscr{M},\omega}$, \textit{of an
  expression}, $\alpha$, \textit{of intensional logic with respect to an
  assignment}, $g$, \textit{of natural numbers to variables, a
  moment of time}, $t$, a stratified model, $\mathscr{M}$, and an
order, $\omega$,
is \mng{$\alpha$}$^{0,\emptyset,g,t,\mathscr{M},\omega}$.

If $\alpha$ is a closed expression of intensional logic,
then for any type-sequence assignments, $\sigma$ and $\sigma'$ and any
assignments to variables $g$ and $g'$,
\mng{$\alpha$}$^{g,t,\mathscr{M},\omega}$($\sigma$)=\mng{$\alpha$}$^{g,t,\mathscr{M},\omega}$($\sigma'$) and
\mng{$\alpha$}$^{g,t,\mathscr{M},\omega}$($\sigma$)=\mng{$\alpha$}$^{g',t,\mathscr{M},\omega}$($\sigma$).
Therefore we say that the \textit{TTR interpretation},
\mng{$\alpha$}$^{t,\mathscr{M},\omega}$ \textit{of a closed expression}, $\alpha$, \textit{of
  intensional logic with respect to a time}, $t$, a stratified model,
$\mathscr{M}$, and order, $\omega$, is
\mng{$\alpha$}$^{g,t,\mathscr{M},\omega}$($\sigma$) for some arbitrarily chosen $g$ and $\sigma$.


\section{Examples of intensional logic interpretations}

Montague (\citeyear{Montague1974}, p.259) notes that an expression,
$\gamma$ of intensional logic type $\langle T,t\rangle$ will denote
the characteristic function of a set of objects of type $T$ and that
if $\alpha$ is an expression of type $T$, then the formula 
$\gamma(\alpha)$ can be used to assert that the denotation of $\alpha$
is a member of the set represented by $\gamma$.  We have replaced $t$
with the type \textit{RecType}.  Record types play the role of
intuitive propositions for us.  They are ``true'' if there is
something of the type and ``false'' if they are empty.  In using types
as intuitive propositions we are following the ``propositions as types
principle'' which we have imported from Martin-L�f type theory
\cite{Martin-Loef1984,NordstromPeterssonSmith1990}.  (For a useful
brief account of the origins of the propositions as types principle
see \cite{Ranta1994}, p. 39ff.)  In choosing to define predicates as
functions from objects to propositions we are following the strategy
used by, for example, \cite{Thomason1980} and it is similar to work by
proponents of property theory,
e.g. \cite{ChierchiaTurner1988,FoxLappin2005}, where sentences are taken to correspond
directly to propositions rather than truth-values.  Despite the fact
that we have increased the intensionality of our interpretations by
replacing Montague's truth-values with types, we can regain the
characteristic functions of sets associated with these types.  If $f$
is a function of type $T\rightarrow\mathit{RecType}$ then we can
define a corresponding characteristic function, $f_c$, of a subset of objects of
type $T$ by 
\begin{quote}
if $a:T$, then $f_c(a)=1$ if $f(a)$ is a non-empty type and $f_c(a)=0$
otherwise
\end{quote}
If $f$ is a function of type
\smallrecord{\smalltfield{x}{\textit{Ind}}}$\rightarrow$\textit{RecType},
  then we can in addition define a characteristic function, $f_e$, of a set of
  individuals corresponding to Montague's original function of type
  $\langle e,t\rangle$:
\begin{quote}
if $a:\mathit{Ind}$, then $f_e(a)=1$ if $f$(\smallrecord{\field{x}{$a$}}) is a non-empty type and $f_e(a)=0$
otherwise
\end{quote}
We can make similar definitions for recovering $n$-ary relations as
sets of $n$-ary tuples from functions which yield a record type when
supplied with all their arguments.  For example, is $f$ is a function
of type
\smallrecord{\smalltfield{x}{\textit{Ind}}}$\rightarrow$(\smallrecord{\smalltfield{x}{\textit{Ind}}}$\rightarrow$\textit{RecType})
by defining a function $f_e$ such that
\begin{quote}
if $a,b:\mathit{Ind}$, then $f_e(\langle a,b\rangle)=1$ if $f$(\smallrecord{\field{x}{$b$}})(\smallrecord{\field{x}{$a$}}) is a non-empty type and $f_e(\langle a,b\rangle)=0$
otherwise
\end{quote}


\section{Examples of inference in intensional logic}

\paragraph{$\beta$-equivalence}
In intensional logic we have
\begin{display}
$\lambda u\alpha(a)=\alpha(a/u)$
\end{display}
as long as $u$ does not occur freely within the scope of `$\hat{\ }$'
in $\alpha$ and no free variables in $a$ become bound within $\alpha(a/u)$. Since for any intensional logic expression, $\beta$,
\iltrans{$\beta$}=\iltrans{$\hat{\ }\beta$} the presence of `$\hat{\
}$' in $\alpha$ will not create an exception to $\beta$-equivalence
under the TTR interpretation.  In TTR we have \iltrans{$\lambda
  u\alpha(a)$} as record type relettering equivalent to
\iltrans{$\alpha(a/u)$} and in a number of cases where the TTR
interpretation does not depend on $n$ we have exact equivalence.  It
is perhaps instructive to go through two particular examples to see why
this equivalence holds.  We first take a case of exact equivalence.
Here $x$ is to be a variable of type $e$, $\mathrm{run}$ a constant of
type $\langle e,t\rangle$ and $a$ a constant of type $e$.  Then,
\begin{quote}
\mng{$\lambda
  x[\mathrm{run}(x)](a)$}$^{0,\emptyset,g,t,\mathscr{M},\omega}$($\sigma$)
\\
= \mng{$\lambda
  x[\mathrm{run}(x)]$}$^{0,\emptyset,g,t,\mathscr{M},\omega}$($\sigma$)(\mng{$a$}$^{0,\emptyset,g,t,\mathscr{M},\omega}$($\sigma$))
(rule~\ref{ilrule:application}) \\
= $\lambda
v_0:$\smallrecord{\smalltfield{x}{\textit{Ind}}}(\mng{run($x$)}$^{1,\emptyset,g(0/x),t,\mathscr{M},\omega}$($\sigma$($v_0$/\smallrecord{\smalltfield{x}{\textit{Ind}}}$_0$)))(\mng{$a$}$^{0,\emptyset,g,t,\mathscr{M},\omega}$($\sigma$))
(rule~\ref{ilrule:abstraction}) \\
= $\lambda
v_0:$\smallrecord{\smalltfield{x}{\textit{Ind}}}($\kappa_{\langle
  e,t\rangle}(\mathrm{run},\mathscr{M})(\omega)(v_0)$)($\kappa_e(a,\mathscr{M})(\omega)$)
(rules~\ref{ilrule:application},~\ref{ilrule:variable},~\ref{ilrule:constant})
\\
= $\kappa_{\langle
  e,t\rangle}(\mathrm{run},\mathscr{M})(\omega)(\kappa_e(a,\mathscr{M})(\omega))$
(function application) \\
= \mng{run($a$)}$^{0,\emptyset,g,t,\mathscr{M},\omega}$ (rules~\ref{ilrule:application},~\ref{ilrule:constant})
\end{quote}
In a similar way we can show that \mng{$\lambda
  x[\neg\mathrm{run}(x)](a)$}$^{0,\emptyset,g,t,\mathscr{M},\omega}$($\sigma$)
is \smallrecord{\smalltfield{c$_1$}{$\kappa_{\langle
  e,t\rangle}(\mathrm{run},\mathscr{M})(\omega)(\kappa_e(a,\mathscr{M})(\omega))\rightarrow
\bot$}}
whereas \mng{$\neg$run($a$)}$^{0,\emptyset,g,t,\mathscr{M},\omega}$($\sigma$)
is \smallrecord{\smalltfield{c$_0$}{$\kappa_{\langle
  e,t\rangle}(\mathrm{run},\mathscr{M})(\omega)(\kappa_e(a,\mathscr{M})(\omega))\rightarrow
\bot$}} using rule~\ref{ilrule:negation}.  That is, in this case we
have relettering rather than exact equivalence because negation
introduces the labels c$_1$ and c$_0$ respectively into the
interpretation.

To complete the discussion of these examples we will consider what
reasonable assignments to the two constants might be.  Let us say that
for any $\mathscr{M}$ and $\omega$, 
\begin{quote}
$\kappa_{\langle
  e,t\rangle}(\mathrm{run},\mathscr{M})(\omega)$=$\lambda
r:$\smallrecord{\smalltfield{x}{\textit{Ind}}}(\smallrecord{\smalltfield{c$_\mathit{run}$}{run$'$($r$.x)}}) \\
  $\kappa_e(a,\mathscr{M})(\omega)$=\smallrecord{\field{x}{$a'$}}.
\end{quote}
Then we have \mng{$\lambda
  x[\mathrm{run}(x)](a)$}$^{0,\emptyset,g,t,\mathscr{M},\omega}$($\sigma$)=\mng{run($a$)}$^{0,\emptyset,g,t,\mathscr{M},\omega}$($\sigma$)=\smallrecord{\smalltfield{c$_\mathit{run}$}{run$'$($a'$)}}.
Given that the ultimate contribution of the individual constant $a$ to
the TTR interpretation is the individual $a'$ why have we made the
type $e$ of intensional logic correspond to the TTR type
\smallrecord{\smalltfield{x}{\textit{Ind}}} rather than simply
\textit{Ind}?  The reason is that some predicates create types that
depend on the record as a whole, not just the x-field.  Consider, for
example, a noun such as \textit{price}.  This has a whole frame
associated \label{pg:frame} with it in the sense of frame semantics
\cite{Fillmore1982,Fillmore1985}.  In the FrameNet database
(\url{http://framenet.icsi.berkeley.edu}, accessed 25th Aug, 2009)
\textit{price} has four core frame elements or roles:  Buyer, Goods,
Money and Seller.  For our present purposes we will not represent all
of these roles but we will add a role for something like Reichenbach's (\citeyear{Reichenbach1947})
event time (labelled with the distinguished label `e-time').  Thus
price frames can be represented as records of the type:
\begin{display}
\record{\tfield{x}{\textit{Ind}} \\
        \tfield{e-time}{\textit{Time}} \\
        \tfield{commodity}{\textit{Ind}} \\
        \tfield{c$_{\mathit{price}}$}{$\langle\lambda
          v:$\textit{Ind}(price($v$)),$\langle\mathrm{x}\rangle\rangle$}
        \\
        \tfield{c$_{\mathit{price\_of\_at}}$}{$\langle\lambda
          v_1:$\textit{Ind}($\lambda v_2:$\textit{Ind}($\lambda
          v_3:$\textit{Time}(price\_of\_at($v_1$,$v_2$,$v_3$)))),
          $\langle$x,commodity,e-time$\rangle\rangle$}}

\end{display}

Consider now Montague's example \label{pg:temppuzzle} \textit{a price rises}.  Montague's
  examples involving price and temperature result from a puzzle
  presented to him by Barbara Partee.  The puzzle is, for example, how
  to prevent the following inference:
\begin{display}
\begin{tabular}{l}
The temperature/price is 90 \\
The temperature/price is rising \\
\hline
90 is rising
\end{tabular}
\end{display}
Montague's solution to this is to say that \textit{rise} is a
predicate of individual concepts (that is, intensions of type $\langle
s,e\rangle$) which does not ultimately cash out as a predicate of
individuals.  We say, instead, that \textit{rise} is a predicate of
frames which does not ultimately cash out as a predicate of an
individual occuring in the frame.  Thus, for any $\mathscr{M}$ and
$\omega$
\begin{quote}
$\kappa_{\langle
  e,t\rangle}(\mathrm{rise},\mathscr{M})(\omega)$=$\lambda
r:$\smallrecord{\smalltfield{x}{\textit{Ind}}}(\smallrecord{\smalltfield{c$_\mathit{rise}$}{rise$'$($r$)}})
\end{quote}
This will prevent the offending inference.  But we can say more about what kind of objects might be of the type rise$'$($r$) if $r$ is a
price frame?  A simple suggestion would be a pair of frames of the
price frame type whose value for the commodity field is the same but
where the e-time of the second frame is later than the e-time of the
first frame and the value in the x-field of the second frame is higher (on the scale which
we use to compare prices) than that in the first.  Thus \textit{rise}
characterizes an event construed as a series of frames.  This view of
events is taken from the notion of Fernando situation strings
developed in a series of papers by Fernando, including
\cite{Fernando2006,Fernando2009}.  Note that different kinds of frames
will require rather different accounts of what series of frames
constitute a ``rising''.  If $r$ is a temperature frame then there
will not be a commodity field but there should be a field for spatial
location.  If the temperature is rising then it is important to have
two frames in which the location is the same but the time is later in
the second frame and the temperature is higher.  The idea that
predicates can take different interpretations depending on their
arguments is central to generative lexicon theory \label{pg:gl}
\cite{Pustejovsky1995,Pustejovsky2006} and much other literature.  Nevertheless, the different
ways of rising of price and temperature are more closely related to
each other (as is reflected in their both being characterized in
FrameNet as related to the Change\_position\_on\_a\_scale frame) than to,
for example, the rising of cakes (possibly related to FrameNet's
Dough\_rising) or balloons (FrameNet's Motion\_directional frame).  Frames characterized in
terms of record types will give us important tools for developing
simple similarity metrics and spaces of interpretations which will
become important later on when we talk about innovation in dialogue
and learning.  We will return to these examples in future sections.

\paragraph{Tense operators}
In Montague's intensional logic tense operators are interpreted as
Priorean \cite{Prior1957,Prior1967,Prior2003}.  Our interpretation of
them is instead Reichenbachian \cite{Reichenbach1947}.  That is, we
take the operators as providing information about an event time,
rather than as shifting the time of evaluation.  This means that the
logic of these operators is essentially different from the Montague's
original interpretation.  For example,
\begin{display}
$WH\phi$
\end{display}
in Montague's version would be interpreted as being true if there is a
time, $t$, in the future such that there is some time previous to $t$
at which $\phi$ is true.  Our interpretation in contrast will require
that there is an event time which is both in the future and in the
past -- a contradiction, given that we are assuming non-circular
time. Let us see why this is so.
\begin{quote}
\iltrans{$WH\phi$}($\sigma$) \\
= 
\record{\tfield{c$_n$}{fut(\mng{$H\phi$}$^{n+1,\emptyset,g,t,\mathscr{M},\omega}$($\sigma$),$t$)}}
(rule~\ref{ilrule:H}) \\
=
\record{\tfield{c$_n$}{fut(\record{\tfield{c$_{n+1}$}{past(\mng{$\phi$}$^{n+2,\emptyset,g,t,\mathscr{M},\omega}$($\sigma$),$t$)}}, $t$)}}
(rule~\ref{ilrule:H}) \\
$\approx$
(\mng{$\phi$}$^{n+2,\emptyset,g,t,\mathscr{M},\omega}$($\sigma$)
\d{$\wedge$} \record{\tfield{e-time}{\textit{Time}} \\
        \tfield{c$_{\mathit{tns}}$}{$\langle\lambda
          v:$\textit{Time}(pre($v,t$)), $\langle$ e-time
          $\rangle\rangle$}}) \d{$\wedge$} \record{\tfield{e-time}{\textit{Time}} \\
        \tfield{c$_{\mathit{tns}}$}{$\langle\lambda
          v:$\textit{Time}(pre($t,v$)), $\langle$ e-time
          $\rangle\rangle$}} (system condition \ref{ilrule:past})

Suppose that
\mng{$\phi$}$^{n+2,\emptyset,g,t,\mathscr{M},\omega}$($\sigma$) is
\record{\tfield{c$_{n+2}$}{$T$}}.  \\
Then \iltrans{$WH\phi$}($\sigma$) $\approx$
\record{\tfield{e-time}{\textit{Time}} \\
        \tfield{c$_{\mathit{tns}}$}{$\langle\lambda
          v:$\textit{Time}(pre($v,t$)$\wedge$pre($t,v$)), $\langle$ e-time
          $\rangle\rangle$} \\
        \tfield{c$_{n+2}$}{$T$}} (by the definition of \d{$\wedge$})
\end{quote}

This means that we would need to use something other than a
combination of $W$ and $H$ to treat examples of the future perfect
such as \textit{The boy will have left}.  Following Reichenbach's
analysis we would need to introduce operators which place constraints
on reference times.

The notion of frame discussed above becomes important in the treatment
of tense here.  Suppose that the intensional logic predicate `run'
corresponds to $\lambda
r:$\smallrecord{\smalltfield{x}{\textit{Ind}}}(\smallrecord{\smalltfield{c$_\mathit{run}$}{run$'$($r$.x)}})
as suggested above and that j is of type \textit{Ind}.  This would
mean that \iltrans{run(j)}($\sigma$)$\approx$
\begin{display}
\record{\tfield{e-time}{\textit{Time}} \\
        \tfield{c$_{\mathit{tns}}$}{$\langle\lambda
          v:$\textit{Time}(pre($v,t$)), $\langle$ e-time
          $\rangle\rangle$} \\
        \tfield{c$_{\mathit{run}}$}{run$'$(j)}}
\end{display}
Here the event time is required to be earlier than $t$ but this is in
no way related to the running event.  According to FrameNet
\textit{run} belongs to, among a number of others, the Self\_motion
frame which has a role (allbeit non-core\footnote{According to
  \cite{RuppenhoferEllsworthPetruckJohnsonScheffczyk2006} Time is
  always treated as a non-core frame element since it does not introduce an
  ``additional, independent or distinct'' event -- this despite the
  fact that it appears to represent a ``conceptually necessary
  component'' of a frame as is required for core frame elements.}) Time (``The time when the
motion occurs'') corresponding to our e-time, together with many other
roles (frame elements).  There are two ways to introduce dependence on
e-time.  The first corresponds to what \cite{Dowty1989} calls ``the
ordered-argument theory of predicates and arguments'':
\begin{display}
\record{\tfield{e-time}{\textit{Time}} \\
        \tfield{c$_{\mathit{tns}}$}{$\langle\lambda
          v:$\textit{Time}(pre($v,t$)), $\langle$e-time$\rangle\rangle$} \\
        \tfield{c$_{\mathit{run}}$}{$\langle\lambda
          v:$\textit{Time}(run$'$(j,$v$)), $\langle$e-time$\rangle\rangle$}}
\end{display}
This involves changing the arity of run$'$ to $\langle$\textit{Ind},\textit{Time}$\rangle$.
The second corresponds to a Davidsonian \cite{Davidson1967} (or
neo-Davidsonian \cite{Dowty1989}) approach to the treatment of time
arguments for event predicates:
\begin{display}
\record{\tfield{e-time}{\textit{Time}} \\
        \tfield{c$_{\mathit{tns}}$}{$\langle\lambda
          v:$\textit{Time}(pre($v,t$)), $\langle$e-time$\rangle\rangle$} \\
        \tfield{c$_{\mathit{run}}$}{run$'$(j)} \\
        \tfield{c$_{\mathit{at}}$}{$\langle\lambda
          v_1:$\textit{Sit}($\lambda
          v_2:$\textit{Time}(at($v_1,v_2$))), $\langle$c$_{\mathit{run}}$,e-time$\rangle\rangle$}}
\end{display}
This second option would involve introducing an additional basic type
\textit{Sit} (``situation'') and requiring that all types constructed
with predicates are a subtype of \textit{Sit}.  This corresponds to
Ranta's (\citeyear{Ranta1994}, p. 54) linking of the Davidsonian
events to the propositions as types principle and also to the notion
of situation type developed in situation theory
\cite{BarwisePerry1983,Barwise1989}.  My inclination is to use the
first option for arguments that are conceptually necessary.  
It is impossible to imagine a running event which does not take place
at some time.  The second option with (neo-)Davidsonian arguments can
then be used for conceptually optional arguments.  Notice that this is
not the same as syntactic optionality of arguments in natural
languages where arguments can be omitted despite the fact that they
are conceptually necessary.  According to FrameNet\footnote{Accessed
  7th Sept 2009.} the Self\_motion frame to which \textit{run} belongs
has six core roles (not including Time which is non-core) including
frame elements like Area and Path.  We shall not represent all of
these but just include a time role as an extra argument to the
predicate run$'$.  Thus we can amend $\kappa_{\langle
  e,t\rangle}(\mathrm{run},\mathscr{M})(\omega)$ to be
\begin{display}
$\lambda
r:$\smallrecord{\smalltfield{x}{\textit{Ind}}}(\record{
\tfield{e-time}{\textit{Time}} \\
\tfield{c$_\mathit{run}$}{$\langle\lambda v:$\textit{Time}(run$'$($r$.x,$v$)),$\langle$e-time$\rangle\rangle$}})
\end{display}
Note that this is a non-tensed interpretation.  When applied to an
argument it will yield a type which requires there to be some time at
which the running is taking place but does not relate that time to the
present or moment of evaluation.  Taking this option means that,
unlike the Priorean tense operators that Montague uses, lack of a
tense operator does not correspond to the present tense.  Thus in
order to obtain present tense interpretations we would need an
additional tense operator.  We will return to these issues in the next
section where we interpret English.
