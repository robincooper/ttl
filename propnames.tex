\chapter{Reference and mental states}
\label{ch:propnames}
\setcounter{equation}{0}

\section{Montague's PTQ as a semantic benchmark}

In this chapter\footnote{An version of some of the material in this
  chapter has appeared in \cite{Cooper2017a}.} and the following chapters we will extend the linguistic coverage of the toy
grammar we presented in Chapter~\ref{ch:gram}.  We will take
Montague's PTQ \citep{Montague1973,Montague1974} as providing a benchmark of linguistic phenomena that
need to be covered and try to cover a sizeable part of what Montague covered, although we will add a few things which are
obviously closely related to Montague's original benchmark and which
have been treated subsequently in the literature.

For many of the phenomena we discuss we will first present a treatment
which is as close as possible to Montague's original treatment and
then present a treatment which exploits the advantages of the approach
we are proposing in this book as well as more recent developments
since Montague's original work.  Our aim is to show that we have
something to say about all these phenomena in an overall consistent
framework, that is, to show that we can cover a significant part of the benchmark using the
tools we are proposing and in many cases say something new concerning
a dialogical approach to these phenomena.  In doing this within the
space of a single book we will not be able to cover all the aspects of
these phenomena which have been studied in the literature following
after Montague.  We hope, however, to show that it is a fruitful line
of research to add a rich type theoretic perspective and a dialogical
approach to current work in linguistic semantics.   
 



\section{Montague's treatment of proper names and a sign-based approach}

The treatment of proper names that we presented in
Chapter~\ref{ch:gram}, encapsulated in the definition of SemPropName
and Lex$_{\mathrm{PropName}}$, is an adaptation of Montague's
original treatment in that it has the content of a proper name utterance
as a quantifier generated from an individual.  The essence of
Montague's treatment was that if we have a proper name \textit{Sam}
whose denotation is based on an individual `sam', then the denotation
of \textit{Sam} is the characteristic function of the set of
properties possessed by the individual concept of `sam'.  Montague
modelled individual concepts as functions from possible worlds to
individuals.  Using more or less Montague's logical notation, the denotation of
\textit{Sam} would be represented by \nexteg{}.
\begin{ex} 
$\lambda P.P\{[\,\up\mathrm{sam}]\}$ 
\label{ex:MontagueSam}
\end{ex} 
Here $[\,\up\mathrm{sam}]$ represents the individual concept of `sam',
that is,
that function, $f$, on the set of possible worlds such that for any
world $w$, $f(w)=\mathrm{sam}$.  The reason that Montague used the
individual concept (and the associated special notion of application
involved in applying a property to an individual concept represented
by the `\{\}'-brackets) was to treat what is known as the Partee-puzzle
concerning temperature and price which we will discuss in Chapter~\ref{ch:commonnouns}.  Many
subsequent researchers came to the conclusion that Montague's treatment of
this puzzle was not the correct one and that the individual concept
was not necessary in the treatment of proper names.  Thus \preveg{}
could be simplified to \nexteg{}.
\begin{ex} 
$\lambda P.P(\mathrm{sam})$ 
\end{ex} 
The content that we assigned to an utterance of \textit{Sam} in
Chapter~\ref{ch:gram} is represented in \nexteg{}.
\begin{ex} 
$\lambda P$:\textit{Ppty}.$P$(\smallrecord{\field{x}{sam}}) 
\end{ex} 
The reason that we have chosen to characterize properties as having
records as their domain rather than individuals, has to do with our
treatment of the Partee puzzle as we will explain in Chapter~\ref{ch:commonnouns}.  Thus the
reason that we have the record \smallrecord{\field{x}{sam}} as the
argument to the property rather than an individual as in \nexteg{} is for the same reason as
Montague introduced an individual concept.
\begin{ex} 
$\lambda P$:\textit{Ppty}.$P$(sam) 
\end{ex} 
The treatment of proper names we presented in Chapter~\ref{ch:gram}
has an important advantage over
Montague's original.  For Montague, (\ref{ex:MontagueSam}) is the
result of applying an interpretation function to the linguistic
expression \textit{Sam} and a number of indices for the
interpretation, $\mathfrak{A}$, a possible world, $i$, a time, $j$,
and an assignment to variables, $g$.  This is represented in
\nexteg{}.
\begin{ex} 
\mng{Sam}$^{\mathfrak{A},i,j,g}$ = $\lambda P.P\{[\,\up\mathrm{sam}]\}$ 
\end{ex} 
This requires that the English expression \textit{Sam} is always
associated with the same individual `sam' with respect to
$\mathfrak{A}$ and any $i,j,g$ related to $\mathfrak{A}$.  This seems to go against the obvious fact that
more than one individual can have the name \textit{Sam}.  It does not
work to say that a different individual can be associated with
\textit{Sam} when it is evaluated with respect to different
parameters.  $g$ is irrelevant since it is defined as an assignment to
variables and the English expression \textit{Sam} is not (associated
with) a variable --- it cannot be bound by a quantifier.\footnote{This
  claim has been called into question by later research.  See
  \cite{Maier2009} for discussion.}  A strategy
which involves varying the
possible world and time to get a different individual associated with
\textit{Sam} would be defeated by the fact that there are many people
called Sam in the actual world right now as well as having the
unintuitive consequence that \textit{Sam might be Sam} would be true if it is true
that Sam might be somebody else called Sam and \textit{Sam will be
  Sam} could be true if somebody called Sam now is somebody else
called Sam at a future time.  We might try saying that associating a
different individual with Sam involves a different interpretation,
$\mathfrak{A}'$, of the language.  This has some intuitive appeal and
we will discuss a variant of it in Section~\ref{sec:Paderewski} in
relation to a proposal by \cite{Ludlow2014}. But
it will come to grief when we need to talk about two people named Sam
in the same sentence unless we allow a switch in interpretation
mid-sentence.  While allowing interpretation to change mid-sentence
may be an attractive option for other reasons it is not an option that
is available on Montague's account of meaning.  The normal assumption
is that in cases where two individuals have the same name the language
contains two expressions which are pronounced the same, for example,
\textit{Sam}$_1$ and \textit{Sam}$_2$.  This would make the treatment
of proper names somewhat like Montague's treatment of pronouns in that
they have silent numerical subscripts attached to them.  How many
\textit{Sam}$_i$ should the language contain? One for each person
named Sam, now, in the past and future and who could be named Sam in
some non-actual world?  If we follow the strategy with variables we
would introduce countably many \textit{Sam}$_i$ so that we would
always have enough.  But with assignments to variables we can always
assign individuals to more that one variable without this causing a
problem.  But the consequence of doing this with proper names would be
to say that an individual can have many names that are pronounced the
same.  (Sam says, ``My name is Sam'', not ``My names are Sam''.) Similarly no two individuals would have the same name, although
they would be able to have distinct names which are pronounced the
same.  This would mean that the interpretation of \textit{have the
  same name} would have to mean ``have names which are pronounced the
same''.  This might cause difficulties distinguishing between a case
where we have two people named Sam and a case where people really do
have distinct names which are pronounced the same such as \textit{Ann}
and \textit{Anne} (unless you want to count this as a case of spelling
the same name differently).

In contrast the analysis of proper names we presented in
Chapter~\ref{ch:gram} is sign-based.  It allows several sign types to
share the same phonology but be associated with different contents.
Treating the language in terms of signs eliminates the need for
arbitrary indexing of proper names.  It also allows us to individuate
names in a sensible way.  One way to individuate names is by the
phonologies occurring in proper name sign types.  Thus if we have two
proper name sign types with the same phonology but contents associated
with different individuals, then we have two individuals with the same
name.  Note that this proposal would make \textit{Ann} and
\textit{Anne} different spellings of the same name since they are both
associated with the same phonological type.  How we individuate names
can be different in different contexts if we follow the kind of
proposal for counting discussed by \cite{Cooper2011}.  We could, for
example, introduce a field into lexical sign types for an
orthographical type and allow the individuation of names by either
phonology or orthography or a combination of both depending on what is
most useful to the purpose at hand.

Using signs in this way seems to give us a clear, if rather simple,
advantage over Montague's formal language approach, even though we
have so far essentiually just transplanted Montague's analysis of
proper names into our variant of a sign-based approach.  However,
there is a remaining question within sign-based approaches which is
a kind of correlate to the need on Montague's approach to create many
different names \textit{Sam}$_i$.  We are tempted to think of a
``language'' as being defined as a collection of sign types.  Thus a
person who knows English will know sign types which pair the
phonological type ``Sam'' with various individuals who are called
Sam.  The problem with this is that different speakers of English will
know different people named Sam and thus technically we would have to
say that they speak different languages.  This may well be a coherent
technical notion of language.  In the terminology of
Chapter~\ref{ch:gram} we would say that the two agents indeed have
different linguistic resources available to them.  But there is also a
resource which the two agents share, even if they do not have any
overlap in the people named Sam that they are aware of.  This is the
knowledge that \textit{Sam} is a proper name in English and can be
used to name individuals.  Arguably it is this knowledge which is
constitutive of English, rather than the knowledge of who is actually
called Sam, important though that might be for performing adequately
in linguistic situations.  In Chapter~\ref{ch:gram} we introduced sign
type contruction operations and in particular
`Lex$_{\mathrm{PropName}}$' which maps a phonological type and an
individual to an appropiate proper name sign type.  We called this a universal resource
since it represents the general knowledge that utterances can be used
to name individuals.  In the English resources we defined there we
named sign types such as `Lex$_{\mathrm{PropName}}$(``Sam'', sam)',
where we specify both the phonological type and the individual
associated with it.  But, given the power of functional abstraction,
we can identify \nexteg{} as an English resource where the
phonological type is specified but not the particular individual.
\begin{ex} 
$\lambda x$:\textit{Ind} . Lex$_{\mathrm{PropName}}$(``Sam'', $x$)
\label{ex:lexpropnameSamResource} 
\end{ex} 
Saying that an agent has this function available as an English resource could
be argued to encode the fact that the agent has the knowledge that
\textit{Sam} is a proper name in English.  An agent who has this
resource has a recipe for constructing an appropriate sign type in
their resources whenever they meet somebody called Sam.  Knowing that
\textit{Sam} is a proper name in English is not a matter of knowing
who is called Sam but rather a matter of knowing what to do
linguistically when you encounter somebody called Sam.  Thus while we
have so far just taken over Montague's original analysis of proper
names we have given ourselves the opportunity to recast it in terms of
a theory which enables agents to update their linguistic resources as
they become aware of new facts about the world.

\section{Proper names and communication}

However, what we have done so far tells us little about the
communicative processes associated with utterances of proper names.
In \cite{Cooper2013b} we pointed out that this kind of analysis does not give us any way of placing the requirement
on the interlocutor's gameboard that there already be a person named
Sam available in order to integrate the new information onto the
gameboard.  As \cite{Ginzburg2012} points out, the successful use
of a proper name to refer to an individual $a$ requires that the name
be publically known as a name for $a$.  We will follow the analysis of
\cite{Cooper2013b} in parametrizing the content.  A
\textit{parametric content} is a function which maps a context to a
content.  As such it relates to Montague's technical notion of
  \textit{meaning} in his paper `Universal Grammar'
  \citep{Montague1970,Montague1974} where he regarded meaning as a
  function from possible worlds and contexts of use to
  denotations.\footnote{See Section~4 (Semantics: Theory of Reference)
    of `Universal Grammar'.}
  This also corresponds to the notion of \textit{character} in
  \cite{Kaplan1978}.

We will take a context to be a situation
modelled as a record.  A simple proposal for a parametric content for
a proper name might be \nexteg{}.
\begin{ex} 
$\lambda r$:\smallrecord{\smalltfield{x}{\textit{Ind}}} . \\
\hspace*{1em}$\lambda P$:\textit{Ppty} . $P(r)$
\label{ex:paramcontpropname} 
\end{ex} 
This would allow any record with an individual labelled `x' to be
mapped to a proper name content.  Recall that the label `x' is picked
up by the notion of property that we defined in Chapter~\ref{ch:gram}
as being of type
(\smallrecord{\smalltfield{x}{\textit{Ind}}}$\rightarrow$\textit{RecType}),
an example being \nexteg{}. 
\begin{ex} 
$\lambda r$:\smallrecord{\smalltfield{x}{\textit{Ind}}} . \smallrecord{\smalltfield{e}{run($r$.x)}} 
\end{ex} 
Associating the phonological type ``Sam'' with
(\ref{ex:paramcontpropname}) would essentially be a way of
encapsulating in the interpretation of \textit{Sam} what is expressed
by (\ref{ex:lexpropnameSamResource}) --- namely, that potentially any
individual can be called Sam.  However, we want the parametric content of
\textit{Sam} to be
more restrictive than this.  It is going to be the tool that we use to
help us identify an appropriate referent when we are confronted with
an utterance of type ``Sam''.  The obvious constraint that we should
place is that the referent is indeed named Sam.  Thus we can restrict
(\ref{ex:paramcontpropname}) so that it is an appropriate parametric
content for \textit{Sam} rather than something that appears to be a
parametric content appropriate to proper names in general.  The
modification is given in \nexteg{}.
\begin{ex} 
$\lambda r$:\smallrecord{\smalltfield{x}{\textit{Ind}}\\
                         \smalltfield{e}{named(x, ``Sam'')}} . \\
\hspace*{5em}$\lambda P$:\textit{Ppty} . $P(r)$ 
\label{ex:parametricSam}
\end{ex} 
This is closely related to treatments of proper
names that were proposed earlier in situation semantics
\citep{GawronPeters1990,Cooper1991,BarwiseCooper1993}.  A more recent
close relation is Maier's (\citeyear{Maier2009}) proposal for the
treatment of proper names in terms of layered discourse representation
theory (LDRT).  Maier points out in a useful overview of the history
of semantic treatments of proper names that this view of proper
names is a hybrid of the descriptivist and referential approaches:
it uses a description like ``named Sam'' to provide a presuppositional
restriction on the kind of referent which can be assigned to the
proper name.  \preveg{}
maps a context in which there is an individual named Sam to a proper
name content based on that individual.  Care has to be taken with the
predicate `named' on this kind of analysis.  It is important that it
not be too restrictive, for example, requiring the legal registering
of the name.  It may be sufficient that someone at some point has
called the individual by the name.  The exact conditions under which a
situation may be of a type constructed with this predicate will vary
depending on the needs associated with the conversation at hand.  We
may, for example, take a stricter view of what it means to have a
certain name if we are talking in a court of law than if we are trying
to attract somebody's attention to avoid an accident on a
mountainside.  This flexibility of meaning ``in flux'' has been
discussed in \cite{CooperKempson2008,Cooper2012,Ludlow2014,GinzburgCooper2014,KrachtKlein2014} among many other
places and we will return to it several times in the following chapters.

An alternative to the use of parametric contents is to use parametric
signs.  This could be formulated as in \nexteg{} where
Lex$_{\mathrm{PropName}}$ is the function for associating lexical
content with phonological types that was introduced in
Chapter~\ref{ch:gram}.
\begin{ex} 
$\lambda r$:\smallrecord{\smalltfield{x}{\textit{Ind}}\\
                         \smalltfield{e}{named(x, ``Sam'')}} . \\
\hspace*{5em}Lex$_{\mathrm{PropName}}$(``Sam'', $r$.x) 
\end{ex} 
Intuitively, \preveg{} says that given a situation in which there is
an individual named by the phonological type ``Sam'' we can construct
a sign type in which the phonological type ``Sam'' is associated with
that individual.  From the point of view of the formal semantics
tradition \preveg{} is a much more radical proposal than
(\ref{ex:parametricSam}).  The function (\ref{ex:parametricSam}) is a
close relative of Montague's \textit{meaning} and Kaplan's \textit{character}.
It is a function from contexts to contents, although our theory of
what contexts and contents are differs from both Montague's and
Kaplan's proposals.  The function in \preveg{}, however, is something
that creates a kind of linguistic resource on the basis of a context.
That is, given a context in which `sam' is named by ``Sam'' we derive
the information that linguistic signs can be used which associate
``Sam'' with `sam'.  If we did not know this before we are extending
the collection of linguistic resources we have available.  We suspect
that both parametric contents and parametric sign types could be of
importance for a theory of linguistic interpretation and learning.
For now, we will work with the less radical notion of parametric
content.  

Parametric contents as we have presented them so far are problematic
for compositional semantics because the domain type of the
function (representing the ``presupposition'') which is the parametric
content varies from case to case depending on what the intuitive
presupposition of the phrase is.  According to our rules it will
always be some subtype of \textit{RecType} (since we are thinking of
contexts as records/situations) and we can therefore assign them to
the type of partial functions from records to quantifiers,
$(\textit{Rec}\rightharpoonup\textit{Quant})$. However, while types of
partial functions as we have defined them are useful for gathering together functions of
different types into a single class they are not useful in a setting
where we need to guarantee that a given function is provided with an
argument which is in its domain.
%but it would not be possible to state
% a single type of parametric content for proper names or other
% syntactic categories.

For this reason we will say that a parametric content is a pair
(construed as a record with two fields) containing a type and a
function whose domain type is that type.  We can create such a
parametric content by using a redefined version of `SemPropName' which
we introduced in Chapter~\ref{ch:gram}.  Whereas the version from
Chapter~\ref{ch:gram} took an individual as argument and created the
content of a name of that individual, the new version will take a
phonological type as argument and create a parametric content
requiring an individual named by that phonological type.  The new
version is given in \nexteg{}.
\begin{ex} 
SemPropName($T$), where $T$ is a phonological type,
is

\record{\field{bg}{\smallrecord{\smalltfield{x}{\textit{Ind}}\\
                         \smalltfield{e}{named(x, $T$)}}} \\
        \field{fg}{$\lambda r$: \smallrecord{\smalltfield{x}{\textit{Ind}}\\
                         \smalltfield{e}{named(x, $T$)}} . \\
& & \hspace*{5em}$\lambda P$:\textit{Ppty} . $P(r)$}}
\label{ex:SemPropName-chpropnames} 
\end{ex}
Here the field labelled `bg' (``background'') contains a record type
and the field labelled `fg' (``foreground'') is a function whose
domain type is the background record type.  From now on we will mean
records of this kind by \textit{parametric content}.

While it is useful to think of parametric contents as records of this
kind the record notation in \preveg{} is clumsy and repetitive and we
will need to talk a good deal about different parametric contents.  We
will therefore use a more concise notation for such records as given
in \nexteg{}.
\begin{ex} 
  $\ulcorner\lambda v\!:\!T\ .\ \varphi\urcorner$ represents the record
  \begin{quote}
    \record{\field{bg}{$T$}\\
      \field{fg}{$\lambda v\!:\!T\ .\ \varphi$}}
  \end{quote}
\label{ex:cornerquote-notation} 
\end{ex}
This means that we can write (\ref{ex:SemPropName-chpropnames}) more
economically as \nexteg{}
\begin{ex} 
SemPropName($T$), where $T$ is a phonological type,
is
\begin{quote}
$\ulcorner\lambda r$: \smallrecord{\smalltfield{x}{\textit{Ind}}\\
                         \smalltfield{e}{named(x, $T$)}} . $\lambda
                       P$:\textit{Ppty} . $P(r)\urcorner$
                     \end{quote}
                     
\label{ex:SemPropName-chpropnames-econ} 
\end{ex}
  


The type of a parametric content of proper names is \nexteg{}.
\begin{ex}
\record{\tfield{bg}{\textit{RecType}} \\
        \tfield{fg}{(bg$\rightarrow$\textit{Quant})}}
% $\displaystyle{\bigvee_{T\sqsubseteq
%     \mathit{Rec}}}(T\rightarrow\textit{Quant})$
\end{ex}
That is, the foreground is a function from
records of the  background type (modelling contexts) to quantifiers.  We will refer to this type as
\textit{PQuant} (``parametric quantifiers'').  The universal resource
Lex$_{\mathrm{PropName}}$ for associating proper name content with
phonological types, creating a sign type for a proper name, will now be
redefined so that it only takes a phonological type as argument as in
\nexteg{}.

\begin{ex} 
Lex$_{\mathrm{PropName}}$($T_{\mathrm{Phon}}$), where
$T_{\mathrm{Phon}}$ is a phonological type,
\\
is defined as \\
Lex($T_{\mathrm{Phon}}$, \textit{NP}) \d{$\wedge$}
\smallrecord{\smallmfield{cnt}{SemPropName($T_{\mathrm{Phon}}$)}{\textit{PQuant}}} 
\end{ex} 
Note that the phonological type plays a dual role here.  It figures
once as determining the phonology of the sign and again as determining
the presupposition associated with the parametric content.   
  

There are two main questions that need to be answered about parametric
contents.  One concerns how the compositional semantics works and the
other concerns the nature of contexts and how you compute with them.
We will take the compositionality issue first.  Let us assume that all
signs provide us with a parametric content rather than a content.  In
those cases where there is no constraint on what the context must be
we will use a trivial parametric content, that is, one that maps any
context (modelled as a record) to the same content.  Thus, for
example, if we wish to represent a theory in which the intransitive
verb \textit{leave} does not place any restrictions on the context, we
could represent its parametric content as \nexteg{a} which is of the
type for parametric properties (\textit{PPpty}) given in \nexteg{b}.
\begin{ex}
\begin{subex}
\item $\ulcorner\lambda r_1$:\textit{Rec}.$\lambda r_2$:\smallrecord{\smalltfield{x}{\textit{Ind}}}.
\smallrecord{\smalltfield{e}{leave($r_2$.x)}}$\urcorner$

\item \record{\tfield{bg}{\textit{RecType}}\\
              \tfield{fg}{(bg$\rightarrow$\textit{Ppty})}}
\end{subex}
\label{ex:parametricLeave}

\end{ex}
The foreground of this parametric property will map any context $r_1$ to the function \nexteg{} which does not depend in
any way on $r_1$.
\begin{ex} 
$\lambda r_2$:\smallrecord{\smalltfield{x}{\textit{Ind}}}.
\smallrecord{\smalltfield{e}{leave($r_2$.x)}} 
\end{ex} 
  
Such a content could be introduced by a resource for
lexical content construction `SemIntransVerb' as characterized in
\nexteg{}, where $T_{\mathrm{bg}}$, the
``background'' or ``presupposition'' type, is a record type and $p$ is a predicate with arity $\langle$\textit{Ind}$\rangle$.
\begin{ex} 
  SemIntransVerb($T_{\mathrm{bg}}$, $p$) is
  \begin{quote}
$\ulcorner\lambda r_1$:$T_{\mathrm{bg}}$ . $\lambda
r_2$:\smallrecord{\smalltfield{x}{\textit{Ind}}}
. \record{\tfield{e}{$p$($r_2$.x)}}$\urcorner$
\end{quote}

\end{ex}


 
Note that if \preveg{} is the only way of constructing parametric content for
lexical intransitive verbs, then although it is possible to place
restrictions on the context by choosing a non-trivial record type
(something other than \textit{Rec}) for $T_{\mathrm{bg}}$ this will
not have any effect on the property returned as the content.  As we
are not here concerned with presuppositions introduced by lexical
intransitive verbs we will leave open whether it is necessary to
change this.  `SemIntransVerb' will be used by the universal resource
`Lex$_{\mathrm{IntransVerb}}$' defined in \nexteg{}, where
$T_{\mathrm{phon}}$ is a phonological type and $p$ is a predicate with
arity $\langle$\textit{Ind}$\rangle$.
\begin{ex} 
Lex$_{\mathrm{IntransVerb}}$($T_{\mathrm{phon}}$, $T_{\mathrm{bg}}$, $p$)\\
is defined as \\
Lex($T_{\mathrm{phon}}$, \textit{N}) \d{$\wedge$}
\smallrecord{\smallmfield{cnt}{SemIntransVerb($T_{\mathrm{bg}}$, $p$)}{\textit{PPpty}}}  
\end{ex} 
This means that the English resource corresponding to the lexical
entry for \textit{leave} can be defined as \nexteg{}.
\begin{ex} 
Lex$_{\mathrm{IntransVerb}}$(``leave'', \textit{Rec}, leave) 
\end{ex} 
       

A standard strategy for dealing with compositional semantics when
using parametric contents is to use a version of what is known in
combinatorial logic as the S-combinator.  In its $\lambda$-calculus
version this is \nexteg{}.
\begin{ex} 
$\lambda z\ .\ \alpha(z)(\beta(z))$ 
\end{ex} 
Our version of the S-combinator including different type requirements
on the context arising from the function and the argument will be
\nexteg{}.
\begin{ex}
If $\alpha$ : \smallrecord{\smalltfield{bg}{\textit{RecType}}\\
                           \smalltfield{fg}{(bg$\rightarrow$($T_1\rightarrow
                             T_2$))}} 
and $\beta$ : \smallrecord{\smalltfield{bg}{\textit{RecType}}\\
                           \smalltfield{fg}{(bg$\rightarrow T_1$)}}
                         then the \textit{combination of $\alpha$ and
    $\beta$  based on functional application}, $\alpha\text{@}\beta$, is
\begin{quote}
$\ulcorner\lambda r$:\smallrecord{\smalltfield{f}{$\alpha$.bg}\\
                                \smalltfield{a}{$\beta$.bg}} . $\alpha$.fg($r$.f)($\beta$.fg($r$.a))$\urcorner$
\end{quote}
% if $\alpha : (T_1 \rightarrow (T_2 \rightarrow T_3))$ and 
% \hspace*{1em}$\beta
%   : (T_4\rightarrow T_2)$ then the \textit{combination of $\alpha$ and
%     $\beta$  based on functional application} is 
% \begin{quote}
% $\lambda r$:\smallrecord{\smalltfield{f}{$T_1$} \\
%                          \smalltfield{a}{$T_4$}}. $\alpha(r.\textrm{f})(\beta(r.\textrm{a}))$
% \end{quote} 
\end{ex}



Note that in the background for the result we have kept the
backgrounds of $\alpha$ and $\beta$
separated in their own fields labelled `f' (``function'') and `a'
(``argument'').% \footnote{While textually this statement of the combination
  % will be correct, we need to take account of the fact that the
  % abbreviatory notation for labels in argument positions to predicates
  % now represent path-names in $\alpha$.bg and $\beta$.bg to which the labels `f'
  % and `a' have been prefixed respectively.  To be precise we could
  % notate this as $[\alpha.\text{bg}]^{f.}$ and $[\beta.\text{bg}]^{a.}$.}
This means that we avoid an unwanted clash of labels
if $\alpha$.bg and $\beta$.bg should happen to share
labels.

This new method of combination for parametric contents means that we
also have to adjust the sign combination operation CntForwardApp
(``forward application of contents'') used in the definition of
interpreted phrase structure rules (see Chapter~\ref{ch:gram},
example~(\ref{ex:ContForwardApp})).  The new version using `@' rather
than straightforward application is given in \nexteg{}.
\begin{ex} 
$\lambda T_1$:\textit{Type} $\lambda T_2$:\textit{Type} . \\
\hspace*{1em}$\lambda
u$:\smallrecord{\smalltfield{cont}{\smallrecord{\smalltfield{bg}{\textit{RecType}}\\
                                               \smalltfield{fg}{(bg$\rightarrow$($T_2\rightarrow T_1$))}}}}$^{\frown}$
   \smallrecord{\smalltfield{cont}{\smallrecord{\smalltfield{bg}{\textit{RecType}}\\
                                               \smalltfield{fg}{(bg$\rightarrow T_2$)}}}} . \\
\hspace*{2em}\smallrecord{\smallmfield{cont}{$u$[0].cont@$u$[1].cont}{\smallrecord{\smalltfield{bg}{\textit{RecType}}\\
                                                                                \smalltfield{fg}{(bg$\rightarrow T_1$)}}}}  
\end{ex} 
  

We can use \preveg{} to combine the contents (\ref{ex:parametricSam}) and
(\ref{ex:parametricLeave}).  The result is given in \nexteg{} where we can
show by successive applications of $\beta$-reduction that
\nexteg{a--d} are all identical.



\begin{ex} 
\begin{subex} 
 
\item $\ulcorner\lambda
  r_1$:\smallrecord{\smalltfield{f}{\smallrecord{\smalltfield{x}{\textit{Ind}}\\
                                               \smalltfield{e}{named(x,
                                                 ``Sam'')}}}\\
                  \smalltfield{a}{\textit{Rec}}} . \\
\hspace*{2em}($\lambda r_2$:\smallrecord{\smalltfield{x}{\textit{Ind}}\\
                         \smalltfield{e}{named(x, ``Sam'')}} . $\lambda P$:\textit{Ppty} . $P(r_2)$)($r_1$.f)\\
\hspace*{4em}($\lambda r_3$:\textit{Rec}.$\lambda r_4$:\smallrecord{\smalltfield{x}{\textit{Ind}}}.
\smallrecord{\smalltfield{e}{leave($r_4$.x)}}($r_1$.a))$\urcorner$ 
 
\item  $\ulcorner\lambda
  r_1$:\smallrecord{\smalltfield{f}{\smallrecord{\smalltfield{x}{\textit{Ind}}\\
                                               \smalltfield{e}{named(x,
                                                 ``Sam'')}}}\\
                  \smalltfield{a}{\textit{Rec}}} . \\
\hspace*{2em}$\lambda P$:\textit{Ppty} . $P$($r_1$.f)\\
\hspace*{4em}($\lambda r_4$:\smallrecord{\smalltfield{x}{\textit{Ind}}}.
\smallrecord{\smalltfield{e}{leave($r_4$.x)}})$\urcorner$ 

\item $\ulcorner\lambda
  r_1$:\smallrecord{\smalltfield{f}{\smallrecord{\smalltfield{x}{\textit{Ind}}\\
                                               \smalltfield{e}{named(x,
                                                 ``Sam'')}}}\\
                  \smalltfield{a}{\textit{Rec}}} . \\
\hspace*{2em}$\lambda r_4$:\smallrecord{\smalltfield{x}{\textit{Ind}}}.
\smallrecord{\smalltfield{e}{leave($r_4$.x)}}($r_1$.f)$\urcorner$

\item $\ulcorner\lambda
  r_1$:\smallrecord{\smalltfield{f}{\smallrecord{\smalltfield{x}{\textit{Ind}}\\
                                               \smalltfield{e}{named(x,
                                                 ``Sam'')}}}\\
                  \smalltfield{a}{\textit{Rec}}} . \\
\hspace*{2em}
\smallrecord{\smalltfield{e}{leave($r_1$.f.x)}}$\urcorner$
 
 
\end{subex} 
\label{ex:paramcontSamLeft}   
\end{ex}

\preveg{} represents the parametric content of \textit{Sam leaves}.
Given a situation containing an individual, $a$, named by ``Sam'' the
function which is its foreground
returns a type of situation in which $a$ leaves.  As usual this type
can play the role of a ``proposition''.  It  is ``true''
if there is a situation of the type and ``false'' if there is no
situation of the type.

The background of the parametric content, that is the domain type
of it foreground,
is to be thought of as placing a constraint on the context.  The idea
is that you can only get to the non-parametric content if you have an
appropriate situation available.  The background of the parametric
content is a type which represents a kind of \textit{presupposition}.  We shall treat
presuppositions as constraints on the resources available to dialogue
participants.  In Chapter~\ref{ch:infex} we introduced the notion of a
dialogue gameboard as a type of dialogue information state.  The most
obvious place to look for the referent of an utterance of a proper
name is in the shared commitments represented on the gameboard
representing what has been committed to in the dialogue so far.  If an
individual named Sam has already been introduced in the dialogue, then
a subsequent utterance of \textit{Sam} in that dialogue is most likely
to refer to that individual unless there is an explicit indication to
the contrary.  The shared commitments on an agent's dialogue gameboard
represent information that is particularly \textit{salient} to the
agent.  The notion of salience in semantics was first introduced by
\cite{Lewis1979} in connection with the analysis of definite
descriptions.  As Lewis says, ``There are various ways for something
to gain salience. Some have to do with the course of conversation,
others do not.''  We wish to suggest that a way of gaining salience in
a conversation is by figuring in the shared commitments on the
gameboard.  (\citealp{Ginzburg2012}, argues that being on shared
commitments, or FACTS in his terminology, is not always sufficient to
indicate salience.)

A reasonable strategy, then, is to look at the shared
commitments on the dialogue gameboard first and then look elsewhere if
that fails.  We will first explore what we need to do to match the
background type of a parametric content against the type which models the
shared commitments of the dialogue and then we will discuss what needs
to be done if there is not a successful match with the shared
commitments. In Chapter~\ref{ch:infex}  we treated the gameboard as a
record type.  In Chapter~\ref{ch:infex}, example 
(\ref{ex:gameboardDudamelBeethovenUchida}), for instance, the shared commitments
were represented as the type \nexteg{}.
\begin{ex} 
\smallrecord{\smalltfield{prev}{\smallrecord{\smalltfield{prev}{\smallrecord{\smalltfield{prev}{\textit{Rec}}
    \\
                                                                                                       \smalltfield{e}{conductor(dudamel)}}}
                                                                   \\
                                                                       \smalltfield{e}{composer(beethoven)}}}
                                   \\
                                       \smalltfield{e}{pianist(uchida)}} 
\end{ex} 
Recall that with each successive updating of the shared commitments
the record type representing the previous state of shared commitments
was embedded under the label `prev' (``previous'').  This prevented
label clash and also kept a record of the order in which information
was introduced.  As \cite{Lewis1979} observed, information introduced
later in the dialogue tends to be more salient than information
introduced earlier.  Thus keeping track of the order also gives us one
measure of relative salience.

In Chapter~\ref{ch:infex} we were using the Montague treatment of
proper names that did not introduce the naming predicate.  In this
chapter we will work towards shared commitments where the naming
associated with proper names is made explicit, as in \nexteg{}.
\begin{ex} 
\smallrecord{\smalltfield{prev}{\smallrecord{\smalltfield{prev}{\smallrecord{\smalltfield{prev}{\textit{Rec}}\\
                                                                             \smalltfield{bg}{\smallrecord{\smalltfield{x}{\textit{Ind}}\\
                                                                                                           \smalltfield{e}{named(x, ``Dudamel'')}}}\\
                                                                              \smalltfield{fg}{\smallrecord{\smalltfield{e}{conductor($\Uparrow$bg.x)}}}}}
                                                                   \\
                                              \smalltfield{bg}{\smallrecord{\smalltfield{x}{\textit{Ind}}\\
                                                                            \smalltfield{e}{named(x, ``Beethoven'')}}}\\
                                              \smalltfield{fg}{\smallrecord{\smalltfield{e}{composer($\Uparrow$bg.x)}}}}}
                                   \\
              \smalltfield{bg}{\smallrecord{\smalltfield{x}{\textit{Ind}}\\
                                            \smalltfield{e}{named(x, ``Uchida'')}}}\\
              \smalltfield{fg}{\smallrecord{\smalltfield{e}{pianist($\Uparrow$bg.x)}}}}
\label{ex:gameboardDudamelBeethovenUchidaBg}  
\end{ex} 
Here we are using the label `bg' to represent background information
in the manner suggested by \cite{Larsson2010} and we see also that
this labelling
corresponds to our use of `bg' and `fg' in parametric contents.  Note that in this
version of the shared commitments we have lost the connection with the
actual individuals `dudamel', `beethoven' and `uchida'.  This can be
seen as an advantage if we are representing the information state of
an agent in the kind of situation described in Chapter~\ref{ch:infex}.
If we simply inform an agent with no previous knowledge of Dudamel that Dudamel
is a conductor, then the information that this agent will get is that
there is somebody named Dudamel who is a conductor.  There will be no
connection to a particular individual of whom the agent is aware.  If
this is not the case, we can reinstate the connection to the
individuals by using manifest fields to anchor the information as in
\nexteg{}.
\begin{ex} 
\smallrecord{\smalltfield{prev}{\smallrecord{\smalltfield{prev}{\smallrecord{\smalltfield{prev}{\textit{Rec}}\\
                                                                             \smalltfield{bg}{\smallrecord{\smallmfield{x}{dudamel}{\textit{Ind}}\\
                                                                                                           \smalltfield{e}{named(x, ``Dudamel'')}}}\\
                                                                              \smalltfield{fg}{\smallrecord{\smalltfield{e}{conductor($\Uparrow$bg.x)}}}}}
                                                                   \\
                                              \smalltfield{bg}{\smallrecord{\smallmfield{x}{beethoven}{\textit{Ind}}\\
                                                                            \smalltfield{e}{named(x, ``Beethoven'')}}}\\
                                              \smalltfield{fg}{\smallrecord{\smalltfield{e}{composer($\Uparrow$bg.x)}}}}}
                                   \\
              \smalltfield{bg}{\smallrecord{\smallmfield{x}{uchida}{\textit{Ind}}\\
                                            \smalltfield{e}{named(x, ``Uchida'')}}}\\
              \smalltfield{fg}{\smallrecord{\smalltfield{e}{pianist($\Uparrow$bg.x)}}}}   
\end{ex} 
The `bg'-fields in (\ref{ex:gameboardDudamelBeethovenUchidaBg}) can
be thought of as corresponding to the internal anchors of
\cite{Kamp1990,KampGenabithReyle2011}.  The use of manifest fields in
\preveg{} would then correspond to the association of what they call external anchors
with those internal anchors.

The task we have before us is to try to match the domain type of the function
in (\ref{ex:paramcontSamLeft}), that is, the type which is the
background of the paramertric content, repeated in \nexteg{}, against the types of shared commitments
in (\ref{ex:gameboardDudamelBeethovenUchidaBg}) or \preveg{}.
\begin{ex} 
\smallrecord{\smalltfield{f}{\smallrecord{\smalltfield{x}{\textit{Ind}}\\
                                               \smalltfield{e}{named(x,
                                                 ``Sam'')}}}\\
                  \smalltfield{a}{\textit{Rec}}}
\label{ex:namedSam} 
\end{ex} 
Intuitively, this attempt at matching should fail since there is no
commitment to an individual named Sam in the shared commitments.
Suppose now that we add to
(\ref{ex:gameboardDudamelBeethovenUchidaBg}) as in \nexteg{}.
\begin{ex} 
\smallrecord{\smalltfield{prev}{\smallrecord{\smalltfield{prev}{\smallrecord{\smalltfield{prev}{\smallrecord{\smalltfield{prev}{\textit{Rec}}\\
                                                                             \smalltfield{bg}{\smallrecord{\smalltfield{x}{\textit{Ind}}\\
                                                                                                           \smalltfield{e}{named(x, ``Dudamel'')}}}\\
                                                                              \smalltfield{fg}{\smallrecord{\smalltfield{e}{conductor($\Uparrow$bg.x)}}}}}
                                                                   \\
                                              \smalltfield{bg}{\smallrecord{\smalltfield{x}{\textit{Ind}}\\
                                                                            \smalltfield{e}{named(x, ``Beethoven'')}}}\\
                                              \smalltfield{fg}{\smallrecord{\smalltfield{e}{composer($\Uparrow$bg.x)}}}}}
                                   \\
              \smalltfield{bg}{\smallrecord{\smalltfield{x}{\textit{Ind}}\\
                                            \smalltfield{e}{named(x, ``Uchida'')}}}\\
              \smalltfield{fg}{\smallrecord{\smalltfield{e}{pianist($\Uparrow$bg.x)}}}}}\\
     \smalltfield{bg}{\smallrecord{\smalltfield{x}{\textit{Ind}}\\
                                   \smalltfield{e}{named(x,
                                     ``Sam'')}}}\\
     \smalltfield{fg}{\smallrecord{\smalltfield{e}{singer($\Uparrow$bg.x)}}}} 
\label{ex:sharedcommsnonexpanded}
\end{ex} 
Intuitively, this should enable a match since this does commit to an
individual named Sam.  However, there is not a direct formal
relationship between (\ref{ex:namedSam}) and \preveg{} corresponding
to this intuition. We will use relabelling of record
types (see Appendix~\ref{app:relabelling-rectypes}) in order to
capture the relationship.  
% First recall that
% \preveg{} is an abbreviated form of \nexteg{} where we have expanded
% the paths of the labels which are used as arguments to predicates.  (We
% use $\ell^n$ for $\ell.\ell.\ldots.\ell$ where the label $\ell$ occurs $n$ times.) 
% \begin{ex} 
% \smallrecord{\smalltfield{prev}{\smallrecord{\smalltfield{prev}{\smallrecord{\smalltfield{prev}{\smallrecord{\smalltfield{prev}{\textit{Rec}}\\
%                                                                              \smalltfield{bg}{\smallrecord{\smalltfield{x}{\textit{Ind}}\\
%                                                                                                            \smalltfield{e}{named(prev$^3$.bg.x, ``Dudamel'')}}}\\
%                                                                               \smalltfield{fg}{\smallrecord{\smalltfield{e}{conductor(prev$^3$.bg.x)}}}}}
%                                                                    \\
%                                               \smalltfield{bg}{\smallrecord{\smalltfield{x}{\textit{Ind}}\\
%                                                                             \smalltfield{e}{named(prev$^2$.bg.x, ``Beethoven'')}}}\\
%                                               \smalltfield{fg}{\smallrecord{\smalltfield{e}{composer(prev$^2$.bg.x)}}}}}
%                                    \\
%               \smalltfield{bg}{\smallrecord{\smalltfield{x}{\textit{Ind}}\\
%                                             \smalltfield{e}{named(prev.bg.x, ``Uchida'')}}}\\
%               \smalltfield{fg}{\smallrecord{\smalltfield{e}{pianist(prev.bg.x)}}}}}\\
%      \smalltfield{bg}{\smallrecord{\smalltfield{x}{\textit{Ind}}\\
%                                    \smalltfield{e}{named(bg.x,
%                                      ``Sam'')}}}\\
%      \smalltfield{fg}{\smallrecord{\smalltfield{e}{singer(bg.x)}}}}
% \label{ex:nonflatsharedcomms} 
% \end{ex}
% The result of flattening \preveg{} will be a new type \nexteg{} where each path
% has been replaced by a single complex label consisting of the sequence
% of labels on the path (which we represent using the normal
% dot-notation for paths).
% \begin{ex} 
% \record{\tfield{prev$^4$}{\textit{Rec}} \\
%         \tfield{prev$^3$.bg.x}{\textit{Ind}} \\
%         \tfield{prev$^3$.bg.e}{named(prev$^3$.bg.x, ``Dudamel'')} \\
%         \tfield{prev$^3$.fg.e}{conductor(prev$^3$.bg.x)} \\
%         \tfield{prev$^2$.bg.x}{\textit{Ind}} \\
%         \tfield{prev$^2$.bg.e}{named(prev$^2$.bg.x, ``Beethoven'')} \\
%         \tfield{prev$^2$.fg.e}{composer(prev$^2$.bg.x)} \\
%         \tfield{prev.bg.x}{\textit{Ind}} \\
%         \tfield{prev.bg.e}{named(prev.bg.x, ``Uchida'')} \\
%         \tfield{prev.fg.e}{pianist(prev.bg.x)} \\
%         \tfield{bg.x}{\textit{Ind}} \\
%         \tfield{bg.e}{named(bg.x, ``Sam'')} \\
%         \tfield{fg.e}{singer(bg.x)}}
% \label{ex:flatsharedcomms}
% \end{ex} 
% While (\ref{ex:nonflatsharedcomms}) and \preveg{} are distinct record types which do
% not share any witnesses there is nevertheless a strong equivalence
% between them in that for any record which is of the type
% (\ref{ex:nonflatsharedcomms}) there is a multiset extensionally
% equivalent record (see Appendix~\ref{app:rectypes}) of type \preveg{} and \textit{vice
%   versa}.  There is a one-one mapping between the two types which
% preserves multiset extension.  Intuitively, this means that the two
% types represent the same basic commitments about the world, namely
% Dudamel is a conducor, Beethoven is a composer, Uchida is a pianist
% and Sam is a singer.  The difference between the two types involves
% the structure they impose on this world.  In the case of \preveg{} we
% have one big situation in which all of these facts hold and in
% (\ref{ex:nonflatsharedcomms}) we have a situation which is made up of
% several smaller situations for each of the individuals involved. Note,
% however, that because we have used the complex labels representing the
% paths we are able to recreate that structure from the flattened type
% in \preveg{}.
% Note also that we can still read off the relative salience of the
% various individuals and facts by checking the number of occurrences
% of `prev' in the label.  

% In the type of the potential new information state that we are hoping
% to create (\ref{ex:nonflatsharedcomms}) would be embedded under the
% label `prev' showing that it is the type representing shared
% commitments in the previous information state. Thus the actual
% flattened type we want to relate the background of the parametric
% content to is \nexteg{}.
% \begin{ex} 
% \record{\tfield{prev$^5$}{\textit{Rec}} \\
%         \tfield{prev$^4$.bg.x}{\textit{Ind}} \\
%         \tfield{prev$^4$.bg.e}{named(prev$^3$.bg.x, ``Dudamel'')} \\
%         \tfield{prev$^4$.fg.e}{conductor(prev$^3$.bg.x)} \\
%         \tfield{prev$^3$.bg.x}{\textit{Ind}} \\
%         \tfield{prev$^3$.bg.e}{named(prev$^2$.bg.x, ``Beethoven'')} \\
%         \tfield{prev$^3$.fg.e}{composer(prev$^2$.bg.x)} \\
%         \tfield{prev$^2$.bg.x}{\textit{Ind}} \\
%         \tfield{prev$^2$.bg.e}{named(prev.bg.x, ``Uchida'')} \\
%         \tfield{prev$^2$.fg.e}{pianist(prev.bg.x)} \\
%         \tfield{prev.bg.x}{\textit{Ind}} \\
%         \tfield{prev.bg.e}{named(bg.x, ``Sam'')} \\
%         \tfield{prev.fg.e}{singer(bg.x)}}
% \label{ex:prevflatsharedcomms}
% \end{ex} 

% We can also flatten the type we are trying to match, that is
% (\ref{ex:namedSam}).  The result is \nexteg{}.
% \begin{ex} 
% \record{\tfield{f.x}{\textit{Ind}} \\
%         \tfield{f.e}{named(f.x, ``Sam'')} \\
%         \tfield{a}{\textit{Rec}}}
%       \end{ex}

In order to match (\ref{ex:namedSam})  against \preveg{}  we look
for a relabelling, $\eta$, of (\ref{ex:namedSam}) that would make
\preveg{} be a subtype of (\ref{ex:namedSam}).  Such a
relabelling is given in \nexteg{a} which we will write as \nexteg{b}.and the result of applying it to
(\ref{ex:namedSam}) is given in \nexteg{c}.
\begin{ex}
\begin{subex}
\item $\eta$ is a function with domain \{f.x,f.e,a\} such that
\begin{quote}
$\eta$(f.x) = bg.x\\
$\eta$(f.e) = bg.e\\
$\eta$(a) = prev$^4$
\end{quote}
where prev$^4$ stands for prev.prev.prev.prev
\item f.x $\leadsto$ bg.x\\
f.e $\leadsto$ bg.e\\
a $\leadsto$ prev$^4$
 
\item 
\record{\tfield{bg}{\record{\tfield{x}{\textit{Ind}}\\
                            \tfield{e}{named(x, ``Sam'')}}}\\
        \tfield{prev}{\record{\tfield{prev}{\record{\tfield{prev}{\record{\tfield{prev}{\textit{Rec}}}}}}}}}

% \record{\tfield{prev.bg.x}{\textit{Ind}} \\
%         \tfield{prev.bg.e}{named(bg.x, ``Sam'')} \\
%         \tfield{prev$^5$}{\textit{Rec}}}
\end{subex}
\end{ex}
This means, then, that any situation which is of the type required by
the shared commitments would, modulo the relabelling, be of the type
which is the background of the parametric content under consideration,
spelled out in \nexteg{}.
\begin{ex}
$\ulcorner\lambda
  r$:\smallrecord{\smalltfield{f}{\smallrecord{\smalltfield{x}{\textit{Ind}}\\
                                               \smalltfield{e}{named(x,
                                                 ``Sam'')}}}\\
                  \smalltfield{a}{\textit{Rec}}} . 
\smallrecord{\smalltfield{e}{leave($r$.f.x)}}$\urcorner$ 
\label{ex:pcont-samleft} 
\end{ex}
The background of the parametric content is
being used as a presupposition which is being matched against the
hearer's current information state.

\begin{shaded}
In order to give a more precise characterization of relabelling we
will need two auxiliary notions.  If $T$ is a record type, we will use
$\mathrm{dpaths}(T)$ to represent the dependent paths of $T$, that is,
the set of paths which occur in any field within $T$ in a dependent
field, that is any path $\pi$ that occurs in $\Pi$ in some dependent
field $[\ell:\langle f,\Pi\rangle]$ anywhere within $T$.  We also use
a notion of \textit{initial subpath}.  To characterize this we will
use the notation given in \nexteg{}.
\begin{ex} 
If $\pi_1$ is $\ell_1.\ \ldots\ .\ell_n$ and $\pi_2$ is
$\ell_{n+1}.\ \ldots\ .\ell_m$ (where all $\ell_i$ are labels), then
$\pi_1.\pi_2$ represents $\ell_1.\ \ldots\ .\ell_n.\ell_{n+1}.\
\ldots\ .\ell_m$ 
\end{ex}
We can now characterize the notion of initial subpath as in \nexteg{}.
\begin{ex}
  \begin{subex}
  \item $\pi_1$ is an \textit{initial subpath} of $\pi_2$,
    $\pi_1\leq\pi_2$, just in case either $\pi_1=\pi_2$ or there is
    some $\pi$ such that $\pi_2=\pi_1.\pi$.
    
  \item $\pi_1$ is a \textit{proper initial subpath} of $\pi_2$,
    $\pi_1<\pi_2$, just in case there is
    some $\pi$ such that $\pi_2=\pi_1.\pi$.
  \end{subex}
  
\end{ex}

If $\mathcal{L}$ is a set of labels, we use $\mathcal{L}^\pi$ to
represent the set of potential paths $\ell_1.\ \ldots\ .\ell_n$ where
$n$ is a natural number and each $\ell_i$ is a member of $\mathcal{L}$. The notion of a relabelling of a record type can now be characterized
as in \nexteg{}.
\begin{ex} 
A \textit{relabelling, $\eta$, of a record type $T$ relative to a set of
  labels $\mathcal{L}$ on which $T$ is based} is a one-one function whose
domain is included in $\mathrm{paths(T)}$ and whose range is included
in $\mathcal{L}^\pi$ such that
\begin{enumerate} 
 
\item if $\pi_1,\pi_2\in\mathrm{dom}(\eta)$ then $\pi_1<\pi_2$ iff $\eta(\pi_1)<\eta(\pi_2)$ 
 
\item if $\pi\in\mathrm{dpaths}(T)$, then $\pi\in\mathrm{dom}(\eta)$

  
\item if $\pi\in\mathrm{paths}(T)-\mathrm{dom}(\eta)$, then it is not
  the case for any $\pi'\in\mathrm{dom}(\eta)$ that $\eta(\pi')=\pi$
 
\end{enumerate} 
  
\end{ex} 
Note that we allow $\eta(\pi)=\pi$.  Thus while clause~2 of \preveg{} requires each path
mentioned in a dependent field of the record type
(i.e. $\mathrm{dpaths}(T)$) is in the domain of
the relabelling, we do not require that the relabelling actually
change that path.  It is, however, important that we guarantee that
there is a path in the relabelled type which the dependent field can
refer to.  This becomes important when we have paths in dependent
fields which are not total.  Consider \nexteg{} which could
intuitively be a type corresponding to \textit{Sam saw a dog run}.
\begin{ex} 
\record{\tfield{s}{\record{\tfield{x}{\textit{Ind}}\\
                           \tfield{c}{dog(x)}\\
                           \tfield{e}{run(x)}}}\\
        \tfield{e}{see(sam,s)}} 
\end{ex} 
Suppose that we tried to apply the relabelling in \nexteg{} to this.
\begin{ex} 
  s.x $\leadsto$ x\\
  s.c $\leadsto$ c\\
  s.e $\leadsto$ e$_1$
\end{ex} 
We might expect this relabelling to result in the type in \nexteg{}
which is ill-formed because the `e'-field depends on a path `s' which
is not present.
\begin{ex} 
  \record{
    \tfield{x}{\textit{Ind}}\\
    \tfield{c}{dog(x)}\\
    \tfield{e$_1$}{run(x)}\\
    \tfield{e}{see(sam,s)}}
    
\end{ex} 
This relabelling is prevented by the requirement in clause~2 that all
paths mentioned in dependent fields (the ``dpaths'') be assigned something by the
relabelling.  In this example the dpaths are `s' and `s.x'
(represented just by `x' in the abbreviatory notation).  Furthermore,
it is required that $\eta(\text{s})<\eta(\text{s.x})$ by clause~1.
That is, in intuitive terms, we must provide a path to what Sam sees
in the relabelled type.  In \nexteg{} is an example of a relabelling
which respects the requirements.
\begin{ex} 
  s $\leadsto$ sit\\
  s.x $\leadsto$ sit.id.x\\
  s.c $\leadsto$ sit.id.e\\
  s.e $\leadsto$ sit.e
\end{ex} 
This results in \nexteg{}.
\begin{ex} 
  \record{
    \tfield{sit}{\record{
        \tfield{id}{\record{
            \tfield{x}{\textit{Ind}}\\
            \tfield{e}{dog(x)}}}\\
        \tfield{e}{run(x)}}}\\
    \tfield{e}{see(sam,sit)}}
\end{ex} 

We represent the result of relabelling a type $T$ with a relabelling,
$\eta$, for $T$, as $[T]_\eta$.  We characterize this in terms of the
unique identifier notation for record types introduced in
Chapter~\ref{ch:infex}, p.~\pageref{pg:unique-identifier-notation}.
We will use three additional notions based on this notation as given
in \nexteg{}.
\begin{ex} 
\begin{subex} 
 
\item A \textit{unique identifier segment} is a path $\ell_1.\ \ldots\
  .{\ell_n}_{\text{\fbox{$i$}}}$ such that only
  ${\ell_n}_{\text{\fbox{$i$}}}$ has a unique identifier subscript in $\ell_1,\ldots,{\ell_n}_{\text{\fbox{$i$}}}$
 
\item A \textit{unique identifier segmentation} is a path
  ${\pi_1}_{\text{\fbox{$i$}}}.{\pi_2}_{\text{\fbox{j}}}.\ \ldots\
  .{\pi_n}_{\text{\fbox{$k$}}}$ or ${\pi_1}_{\text{\fbox{$i$}}}.{\pi_2}_{\text{\fbox{j}}}.\ \ldots\
  .{\pi_n}$ where ${\pi_1}_{\text{\fbox{$i$}}},{\pi_2}_{\text{\fbox{j}}},\ldots,
  {\pi_n}_{\text{\fbox{$k$}}}$ or ${\pi_1}_{\text{\fbox{$i$}}},{\pi_2}_{\text{\fbox{j}}},\ldots,
  {\pi_{n-1}}_{\text{\fbox{$k$}}}$ are unique identifier segments

  
\item The \textit{result of removing the prefix $\pi_1$ from a path
  $\pi_1.\pi_2$}, in symbols $\pi_1.\pi_2\backslash\pi_1$, is $\pi_2$
 
\end{subex} 
   
\end{ex}
We can now characterize $[T]_\eta$ as in \nexteg{}.
\begin{ex} 
The \textit{result of relabelling a type $T$ with a relabelling,
  $\eta$, for $T$}, $[T]_\eta$, is defined by carrying out the
following on the unique identifier notation of $T$.
\begin{enumerate} 
 
\item if $\pi\in\mathrm{dom}(\eta)$ and $\pi$ (that is, without unique
  identifiers) is represented as a path in $T$, then replace $\pi$
  with $\eta(\pi)$ 
 
\item if ${\pi_1}_{\text{\fbox{$i$}}}.{\pi_2}_{\text{\fbox{j}}}.\ \ldots\
  .{\pi_n}_{\text{\fbox{$k$}}}$ is a unique identifier
  segmentation in $\mathrm{dom}(\eta)$ then replace it with
  $\eta(\pi_1)_{\text{\fbox{$i$}}}.(\eta(\pi_1.\pi_2)\backslash\eta(\pi_1))_{\text{\fbox{$j$}}}.\
  \ldots\ .(\eta(\pi_1.\pi_2.\ \ldots\
  .\pi_n)\backslash\eta(\pi_1.\pi_2.\ \ldots\
  .\pi_{n-1}))_{\text{\fbox{$k$}}}$

  
\item if ${\pi_1}_{\text{\fbox{$i$}}}.{\pi_2}_{\text{\fbox{j}}}.\ \ldots\
  .{\pi_n}$ is a unique identifier segmentation in
  $\mathrm{dom}(\eta)$ then replace it with $\eta(\pi_1)_{\text{\fbox{$i$}}}.(\eta(\pi_1.\pi_2)\backslash\eta(\pi_1))_{\text{\fbox{$j$}}}.\
  \ldots\ .\eta(\pi_1.\pi_2.\ \ldots\
  .\pi_n)\backslash\eta(\pi_1.\pi_2.\ \ldots\
  .\pi_{n-1})$

\end{enumerate} 
  
\end{ex}

Note that using this definition in terms of unique identifier notation
will automatically account for cases in which the scope of the
dependent field has to be adjusted in the relabelling.  Consider the
record type in \nexteg{a} and the relabelling in \nexteg{b} which
should result in \nexteg{c}.
\begin{ex} 
\begin{subex} 
 
\item \record{\tfield{x}{\record{\tfield{x}{\textit{Ind}}\\
                           \tfield{e}{$\langle\lambda v$:\textit{Ind}
                             .  dog($v$), $\langle$x$\rangle\rangle$}}}} 
 
\item x.x $\leadsto$ y

                     
\item \record{\tfield{y}{\textit{Ind}}\\
        \tfield{x}{$\langle\lambda v$:\textit{Ind}
                             . \record{\tfield{e}{dog($v$)}},
                             $\langle$y$\rangle\rangle$}}
 
\end{subex} 
   
\end{ex} 
The unique identifier notation for \preveg{a} and \preveg{c} are
\nexteg{a} and \nexteg{b} respectively and \nexteg{b} is the result of
relabelling \nexteg{a} with \preveg{b} according to the definition we
have given.
\begin{ex} 
\begin{subex} 
 
\item
  \record{
    \tfield{x}{\record{
        \tfield{x$_{\text{\fbox{0}}}$}{\textit{Ind}}\\
        \tfield{e}{dog(\fbox{0})}}}}
 
  \item
    \record{
      \tfield{y$_{\text{\fbox{0}}}$}{\textit{Ind}}\\
      \tfield{x}{\record{
          \tfield{e}{dog(\fbox{0})}}}}
 
\end{subex} 
   
\end{ex} 
To understand why this works, see the characterization of unique
identifier notation in Chapter~\ref{ch:infex}, p.~\pageref{pg:unique-identifier-notation}.  


  
  
  

\end{shaded}



Given that we have now found a match, how can we go about updating the
shared commitments with the new information represented by the
parametric content?  % The technique for doing this will involve the
% notion of a \textit{fixed point type}.  Given a function, $f$, that returns
% a type, we can always ask the question:  is there an object, $a$, such
% that $a:f(a)$.  That is, $a$ is of the type which is the result of
% applying $f$ to $a$ itself.  This means that $a$ is a \textit{fixed
%   point} for $f$.  This raises a further question: can we characterize
% a type $T$ such that any $a$ which is of type $T$ will be a fixed
% point for $f$?  In this case $T$ will be called a \textit{fixed point
%   type} for $f$.  Consider the example of the foreground of the parametric content of
% \textit{Sam left} in (\ref{ex:paramcontSamLeft}) repeated as
% \nexteg{a}.  A fixed point type of this function is given in
% \nexteg{b}.
% \begin{ex} 
% \begin{subex} 
 
% \item $\lambda
%   r_1$:\smallrecord{\smalltfield{f}{\smallrecord{\smalltfield{x}{\textit{Ind}}\\
%                                                \smalltfield{e}{named(x,
%                                                  ``Sam'')}}}\\
%                   \smalltfield{a}{\textit{Rec}}} . 
% %\hspace*{2em}
% \smallrecord{\smalltfield{e}{leave($r_1$.f.x)}} 
 
% \item \smallrecord{\smalltfield{f}{\smallrecord{\smalltfield{x}{\textit{Ind}}\\
%                                                \smalltfield{e}{named(x,
%                                                  ``Sam'')}}}\\
%                   \smalltfield{a}{\textit{Rec}} \\
%                   \smalltfield{e}{leave(f.x)}} 
 
% \end{subex} 
% \label{ex:parcontfixedpoint}   
% \end{ex} 
% Any record of type \preveg{b} will be a fixed point for
% \preveg{a}. \preveg{b} can be obtained by applying the fixed point
% type construction function, $\mathcal{F}$, to \preveg{a}.  This
% function is defined explicitly in Appendix~\ref{app:merge},
% p.~\pageref{pg:fixedpointtype}. 

If we are updating (\ref{ex:sharedcommsnonexpanded}) with the parametric
content (\ref{ex:pcont-samleft}a) then the result should be \nexteg{} where
(\ref{ex:sharedcommsnonexpanded}) has been embedded under the label
`prev' and the new information provided by the parametric content has
been added at the top level of the new type, suitably relabelled so as
to pick out the individual named Sam which has been previously introduced.
\begin{ex}

\smallrecord{\smalltfield{prev}{\smallrecord{\smalltfield{prev}{\smallrecord{\smalltfield{prev}{\smallrecord{\smalltfield{prev}{\smallrecord{\smalltfield{prev}{\textit{Rec}}\\
                                                                             \smalltfield{bg}{\smallrecord{\smalltfield{x}{\textit{Ind}}\\
                                                                                                           \smalltfield{e}{named(x, ``Dudamel'')}}}\\
                                                                              \smalltfield{fg}{\smallrecord{\smalltfield{e}{conductor($\Uparrow$bg.x)}}}}}
                                                                   \\
                                              \smalltfield{bg}{\smallrecord{\smalltfield{x}{\textit{Ind}}\\
                                                                            \smalltfield{e}{named(x, ``Beethoven'')}}}\\
                                              \smalltfield{fg}{\smallrecord{\smalltfield{e}{composer($\Uparrow$bg.x)}}}}}
                                   \\
              \smalltfield{bg}{\smallrecord{\smalltfield{x}{\textit{Ind}}\\
                                            \smalltfield{e}{named(x, ``Uchida'')}}}\\
              \smalltfield{fg}{\smallrecord{\smalltfield{e}{pianist($\Uparrow$bg.x)}}}}}\\
     \smalltfield{bg}{\smallrecord{\smalltfield{x}{\textit{Ind}}\\
                                   \smalltfield{e}{named(x,
                                     ``Sam'')}}}\\
     \smalltfield{fg}{\smallrecord{\smalltfield{e}{singer($\Uparrow$bg.x)}}}}}
 \\
\smalltfield{bg}{\smallrecord{\smalltfield{f}{\smallrecord{\smallmfield{x}{$\Uparrow^2$prev.bg.x}{\textit{Ind}}\\
                                               \smallmfield{e}{$\Uparrow^2$prev.bg.e}{named(x,
                                                 ``Sam'')}}}\\
                  \smallmfield{a}{$\Uparrow$prev$^5$}{\textit{Rec}}}}\\
\smalltfield{fg}{\smallrecord{\smalltfield{e}{leave($\Uparrow$bg.f.x)}}}} 
\end{ex}
Note that this both achieves a link to a previous mention of Sam and
simultaneously ensures that Sam is the most salient individual in
shared commitments in virtue of the new mention.


We can achieve this update by the following method.  Suppose that $T_{\mathrm{comm}}$ is the
type representing shared commitments that we wish to update with a
parametric content given in \nexteg{}.
\begin{ex}
\record{\field{bg}{$T_{\mathrm{bg}}$}\\
        \field{fg}{$f$}}
\end{ex}
where $f:(T_{\mathrm{bg}}\rightarrow\textit{RecType})$.  % The
% first thing to do is embed  $T_{\mathrm{comm}}$ under the label `prev', obtaining
% \smallrecord{\smalltfield{prev}{$T_{\mathrm{comm}}$}}.
% Let
% $r_{\mathrm{prev}}$ be a record of this type.  We need to
% consider the flattened version of this type, that is,
% $\varphi$(\smallrecord{\smalltfield{prev}{$T_{\mathrm{comm}}$}}). 
We
need to find a relabelling, $\eta$, of 
$T_{\mathrm{bg}}$ such that $T_{\mathrm{comm}}\sqsubseteq[T_{\mathrm{bg}}]_\eta$.
% $\varphi$(\smallrecord{\smalltfield{prev}{$T_{\mathrm{comm}}$}})
% $\sqsubseteq$ $[\varphi(T_{\mathrm{bg}})]_\eta$, that is, the
% flattened version of
% \smallrecord{\smalltfield{prev}{$T_{\mathrm{comm}}$}} is a subtype of
% result of relabelling the flattened version of $T_{\mathrm{bg}}$ with
% $\eta$.
Suppose that we have a record, $r_{\mathrm{comm}}$, of type
$T_{\mathrm{comm}}$.  We can use this together with $\eta$ to anchor
$T_{\mathrm{bg}}$.  The result of the anchoring is notated as 
$T_{\mathrm{bg}}\parallel_\eta r_{\mathrm{comm}}$. The operation
$T \parallel_\eta r$ % (defined explicitly in
% Appendix~\ref{app:specrec})
replaces fields in $T$, \smallrecord{\smalltfield{$\ell$}{$T'$}}, such
that $\ell$ is in the
domain of the relabelling, $\eta$, and for which $\eta$ returns a
path, $\pi$,
in $r$ such that $r.\pi:T'$ with a manifest
field \smallrecord{\smallmfield{$\ell$}{$r.\pi$}{$T'$}}. For example,
suppose that $T_{\mathrm{bg}}$ is \nexteg{a},
$r_{\mathrm{comm}}:T_{\mathrm{comm}}$ and $\eta$ is \nexteg{b}. Then
\mbox{$T_{\mathrm{bg}}\parallel_\eta r_{\mathrm{comm}}$} is \nexteg{c}.
\begin{ex} 
\begin{subex} 
 
\item \record{\tfield{x}{\textit{Ind}}\\
              \tfield{e}{named(x, ``Sam'')}} 
 
\item x $\leadsto$ bg.x\\
      e $\leadsto$ bg.e

\item \record{\mfield{x}{$r_{\mathrm{comm}}$.bg.x}{\textit{Ind}}\\
              \mfield{e}{$r_{\mathrm{comm}}$.bg.e}{named(x, ``Sam'')}}
 
\end{subex} 
   
\end{ex} 
  

In the type of the updated shared
commitments the background will be the background of the parametric
content anchored to the previous shared commitments and the foreground
will be the result of applying the function which is the foreground of
the parametric content to this background.
The updated type of the shared commitments
will thus be that given in \nexteg{}.   
\begin{ex} 
\record{\tfield{prev}{$T_{\mathrm{comm}}$}\\
             \tfield{bg}{$T_{\mathrm{bg}}$ $\parallel_\eta$
               prev}\\
             \tfield{fg}{$f$(bg)}}
         \end{ex}



\begin{shaded}
  % anchoring types with records
  Recall that a dependent field in a record type normally represented
  as $[\ell=a:T]$ is in actual fact $[\ell:T_a]$ where $T_a$ is a
  singleton type.  The intuition in defining $T\parallel r$ is 
  that for any path, $\pi$, in $T$ leading to a type $T.\pi$, $T.\pi$ is
  replaced by the singleton type constructed from $T.\pi$ and $r.\pi$.
  This is made precise in \nexteg{}.
  \begin{ex} 
  If $T$ is a record type and $r$ is a record, then $T\parallel r$,  the
\textit{specification (or anchoring) of $T$ by
$r$} is a type, $T'$, like $T$ except that if $\pi$ is a path in both
$T$ and $r$,  
\begin{enumerate}
\item  if
$T.\pi$ is a type, then $T'.\pi$ is $(T.\pi)_{r.\pi}$ (that is, if
$\ell$ is the last label in $\pi$ then $[\ell:T.\pi]$ is replaced by
$[\ell=r.\pi:T.\pi]$ at the end of $\pi$ in $T$)

\item if $T.\pi$ is $\langle f,\langle\pi_1,\ldots,\pi_n\rangle\rangle$, then
  $T'.\pi$ is $\langle f',\langle\pi_1,\ldots,\pi_n\rangle\rangle$ where for any
  $a_1,\ldots,a_n$, $f'(a_1)\ldots(a_n)$ is defined iff
  $f(a_1)\ldots(a_n)$ is defined and $f'(a_1)\ldots(a_n) =
  (f(a_1)\ldots(a_n))_{r.\pi}$
\end{enumerate}
\end{ex}
Note that according to this definition is the type at the end of path
$\pi$ is already a singleton type, that is, the field is already
manifest, this will add an additional identity constraint.  Recall
that $c:(T_a)_b$ iff $c:T_a$ and $c=b$.  Thus $c:(T_a)_b$ requires $a=b=c$.

A variant of this notion of specification is default specification
which will only require specification of fields which are not already
specified, that is, manifest fields will remain unchanged. This
variant is given in \nexteg{}.
  \begin{ex} 
  If $T$ is a record type and $r$ is a record, then $T\dspec r$,  the
\textit{default specification (or anchoring) of $T$ by
$r$} is a type, $T'$, like $T$ except that if $\pi$ is a path in both
$T$ and $r$,  
\begin{enumerate}
\item  if
$T.\pi$ is a type but not a singleton type, then $T'.\pi$ is $(T.\pi)_{r.\pi}$ 

\item if $T.\pi$ is $\langle f,\langle\pi_1,\ldots,\pi_n\rangle\rangle$, then
  $T'.\pi$ is $\langle f',\langle\pi_1,\ldots,\pi_n\rangle\rangle$ where for any
  $a_1,\ldots,a_n$, $f'(a_1)\ldots(a_n)$ is defined iff
  $f(a_1)\ldots(a_n)$ is defined  and $f'(a_1)\ldots(a_n) =
  (f(a_1)\ldots(a_n))_{r.\pi}$ if $f(a_1)\ldots(a_n)$ is not a
  singleton type and otherwise $f'(a_1)\ldots(a_n) = f(a_1)\ldots(a_n)$ 
\end{enumerate}
\end{ex}

% If $T$ is a record type and $r$ is a record, then
% $T\dspec r$, the
% \textit{default specification} (or \textit{default anchoring}) of $T$ by
% $r$ is the result
% of replacing each field, $\langle \ell, T'\rangle$, in $T$ such that
% $\ell$ is a label in $r$, with 
% \begin{enumerate}
% \item  $\langle\ell,T'_{r.\ell}\rangle$, if
% $T'$ is a type but not a singleton type. (That is, $\langle
%   \ell, T'\rangle$ is not already a manifest field.) If $T'$ is a
% singleton type then $\langle \ell, T'\rangle$ is replaced by itself.
% \item $\langle\ell,\langle f',\Pi\rangle\rangle$, if $T'=\langle
%   f,\Pi\rangle$ where $f$ is a function and $\Pi$ is a sequence of
%   paths of length $n$ and for any
%   $a_1,\ldots,a_n$, $f'(a_1)\ldots(a_n)$ is defined iff
%   $f(a_1)\ldots(a_n)$ is defined and $f'(a_1)\ldots(a_n) =
%   f(a_1)\ldots(a_n)_{r.\ell}$ if $f(a_1)\ldots(a_n)$ is not a
%   singleton type. Otherwise, $f'(a_1)\ldots(a_n) =
%   f(a_1)\ldots(a_n)$.
% \end{enumerate}  

Types can also be specified by records which have different paths to
the type by using a relabelling.  Thus $T\parallel_\eta r$ anchors the
value of a path, $\pi$, in $T$ with the value of $\eta(\pi)$ in
$r$. % is the
% result of replacing each field in $T$, $\langle\ell,T'\rangle$,  such
% that $\eta(\ell)$ is a label in $r$ and $r.\eta(\ell):T'$, with a
% manifest field$\langle\ell,T'_{r.\eta(\ell)}\rangle$.  More exactly we
% will define $T\parallel_\eta r$ in terms of flattening, relabelling
% and specification by a record:
% \begin{quote}
% $T\parallel_\eta r$ =
% $\varphi^-([[\varphi(T)]_\eta\parallel\varphi(r)]_{\eta^-})$
% \end{quote}
This is defined in \nexteg{}.
\begin{ex} 
 If $T$ is a record type, $r$ is a record and $\eta$ is a relabelling
 of $T$ whose range is included in $\mathrm{paths}(r)$, then $T\parallel_\eta r$,  the
\textit{specification (or anchoring) of $T$ by
$r$ relative to $\eta$} is a type, $T'$, like $T$ except that if $\pi$
is in the domain of $\eta$,  
\begin{enumerate}
\item  if
$T.\pi$ is a type, then $T'.\pi$ is $(T.\pi)_{r.\eta(\pi)}$ 

\item if $T.\pi$ is $\langle f,\langle\pi_1,\ldots,\pi_n\rangle\rangle$, then
  $T'.\pi$ is $\langle f',\langle\pi_1,\ldots,\pi_n\rangle\rangle$ where for any
  $a_1,\ldots,a_n$, $f'(a_1)\ldots(a_n)$ is defined iff
  $f(a_1)\ldots(a_n)$ is defined and $f'(a_1)\ldots(a_n) =
  (f(a_1)\ldots(a_n))_{r.\eta(\pi)}$
\end{enumerate} 
\end{ex} 
In \nexteg{} we provide two simple examples.
\begin{ex} 
\begin{subex} 
 
\item let $\eta_1$ be characterized by
  \begin{quote}
    $\ell_1\leadsto\ell_3$\\
    $\ell_2\leadsto\ell_4$
  \end{quote}
  
 
\item then \record{\tfield{$\ell_1$}{$T_1$}\\
        \tfield{$\ell_2$}{$T_2$}} $\parallel_{\eta_1}$ 
\record{\field{$\ell_3$}{$a$}\\
        \field{$\ell_4$}{$b$}}
=
\record{\mfield{$\ell_1$}{$a$}{$T_1$}\\
  \mfield{$\ell_2$}{$b$}{$T_2$}}
\item let $\eta_2$ be characterized by
  \begin{quote}
    $\ell_5.\ell_1\leadsto\ell_7.\ell_3$\\
    $\ell_5.\ell_2\leadsto\ell_8.\ell_4$
  \end{quote}


  
\item then \record{\tfield{$\ell_5$}{\record{\tfield{$\ell_1$}{$T_1$}\\
                                  \tfield{$\ell_2$}{$T_2$}}}\\
        \tfield{$\ell_6$}{$T_3$}} $\parallel_{\eta_2}$
\record{\field{$\ell_7$}{\record{\field{$\ell_3$}{$a$}}}\\
        \field{$\ell_8$}{\record{\field{$\ell_4$}{$b$}\\
                                 \field{$\ell_9$}{$c$}}}} =
\begin{quote}
\record{\tfield{$\ell_5$}{\record{\mfield{$\ell_1$}{$a$}{$T_1$}\\
                                  \mfield{$\ell_2$}{$b$}{$T_2$}}}\\
        \tfield{$\ell_6$}{$T_3$}}
\end{quote}
 
\end{subex} 
   
\end{ex} 
  
  
% Here are two examples:  suppose that % $a:T_1$, $b:T_2$ and 
% $\eta$ is a
% function with domain $\{\ell_1,\ell_2\}$ such that
% $\eta(\ell_1)=\ell_3$ and $\eta(\ell_2)=\ell_4$.  Then:
% \begin{quote}
% \record{\tfield{$\ell_1$}{$T_1$}\\
%         \tfield{$\ell_2$}{$T_2$}} $\parallel_\eta$ 
% \record{\field{$\ell_3$}{$a$}\\
%         \field{$\ell_4$}{$b$}}
% =
% \record{\mfield{$\ell_1$}{$a$}{$T_1$}\\
%         \mfield{$\ell_2$}{$b$}{$T_2$}}
% \end{quote}
% Suppose now that % $a$ and $b$ are as above and that 
% $\eta$ is a
% function with domain $\{\ell_5.\ell_1, \ell_5.\ell_2, \ell_6\}$ (where
% $\ell_5.\ell_1$ and $\ell_5.\ell_2$ are complex labels) such that
% $\eta(\ell_5.\ell_1)=\ell_7.\ell_3$,
% $\eta(\ell_5.\ell_2)=\ell_8.\ell_4$ and $\eta(\ell_6)=\ell_6$.  Then:
% \begin{quote}
% \record{\tfield{$\ell_5$}{\record{\tfield{$\ell_1$}{$T_1$}\\
%                                   \tfield{$\ell_2$}{$T_2$}}}\\
%         \tfield{$\ell_6$}{$T_3$}} $\parallel_\eta$
% \record{\field{$\ell_7$}{\record{\field{$\ell_3$}{$a$}}}\\
%         \field{$\ell_8$}{\record{\field{$\ell_4$}{$b$}\\
%                                  \field{$\ell_9$}{$c$}}}} =
% \begin{quote}
% \record{\tfield{$\ell_5$}{\record{\mfield{$\ell_1$}{$a$}{$T_1$}\\
%                                   \mfield{$\ell_2$}{$b$}{$T_2$}}}\\
%         \tfield{$\ell_6$}{$T_3$}}
% \end{quote}
% \end{quote}  

\end{shaded}

           
We can tie all this together in a single action rule, given in
\nexteg{}, a preliminary version which we will revise slightly later
in the light of other accommodation functions which we will introduce
in Section~\ref{sec:accommodation}.
\begin{ex} 
\textsc{AccGB} -- preliminary version

% $\lambda T$:\textit{GameBoard} . \\
% \hspace*{1em} $\lambda
% f$:\smallrecord{\smalltfield{bg}{\textit{RecType}}\\
%                 \smalltfield{fg}{(bg$\rightarrow$\textit{RecType})}}
%               . \\
% \hspace*{2em} $\lambda r$:$T$ . \\ 
% \hspace*{3em} $T$ \fbox{\d{$\wedge$}} 
% \smallrecord{\smalltfield{shared}{\smallrecord{\smallmfield{commitments}{\smallrecord{\smalltfield{prev}{$r$.shared.commitments}\\
%                                                                                       \smalltfield{bg}{$f$.bg
%                                                                                         $\parallel_\eta$
%                                                                                         prev}\\
%                                                                                         \smalltfield{fg}{$f$.fg(bg)}}}{\textit{RecType}}}}}
\hspace*{-2.5em}\begin{prooftree}[separation=.5em]
  \hypo{s_{i,A}:_A T_{\text{inf}}}
  \hypo{T_{\text{inf}}\sqsubseteq\text{\smallrecord{
        \smalltfield{shared}{\smallrecord{
            \smalltfield{commitments}{\textit{RecType}}}}}}}
  \hypo{u^*.\text{cont}:_A T_{\text{utt}}}
  \hypo{T_{\text{utt}}\sqsubseteq\text{\smallrecord{
        \smalltfield{bg}{\textit{RecType}}\\
        \smalltfield{fg}{(bg$\rightarrow$\textit{RecType})}}}}
  \infer[enth]4[$\eta$]{s_{i+1,A}:_A T_{\text{inf}} \text{\fbox{\d{$\wedge$}}
      \smallrecord{
        \smalltfield{shared}{\smallrecord{
            \smallmfield{commitments}{\smallrecord{
                \smalltfield{prev}{$s_{i,A}$.shared.commitments}\\
                \smalltfield{bg}{$u^*$.cont.bg$\parallel_\eta$prev}\\
                \smalltfield{fg}{$u^*$.cont.fg(bg)}}}{\textit{RecType}}}}}}}
\end{prooftree}
where $\eta$ is a relabelling for $u^*$.cont.bg with range included in $s_{i,A}$.shared.commitments
\label{ex:AccGBprelim} 
\end{ex}

This action rule requires that an agent, $A$, judges her current
information state, $s_{i,A}$, to be of type, $T$, which includes
shared commitments.  Furthermore, $A$ judges the content of the
current utterance, $u^*$, to be a parametric content which will yield
a record type when applied to a context of the appropriate background
type.  Given an appropriate relabelling, $\eta$, of the background type of the
parametric content which associates it with elements of the shared
commitments in the current information state, $A$ is licensed
(afforded) to make the judgement that her next information state is
like the current one except that the shared commitments have been
changed so that the current shared commitments are marked as previous,
the background of the parametric content anchored to those previous
shared commitments constrains a background situation under the label
`bg' and a foreground situation is added, under the label `fg', resulting from applying the
(foreground of) the parametric content to the background thus
introduced.  \preveg{} abstracts away from the problem of exactly how
to choose an appropriate relabelling, $\eta$.  % takes a game-board, $T$, (recall that a gameboard is a type of
% an information state which in turn is a record) and a parametric
% content and returns a function that will map an information state of
% type $T$ to a new type which is the result of an asymmetric merge of
% $T$ with a type that will replace the type representing shared
% commitments according to $T$ with a new type where the old shared
% commitments is labelled with `prev'  and new `bg' and `fg' fields are
% added as described above.

  

%  $\eta$ in general will not be defined
% on all the labels of $\mathcal{F}(f)$ since its set of labels will be
% a superset of the set of labels of $T_{\mathrm{bg}}$.  What we need is
% a relabelling of $\mathcal{F}(f)$, $\eta'$ which is an extension of
% $\eta$, that is, the domain of $\eta$ is a subset of the domain of
% $\eta'$ and  for any label, $\ell$, in the domain of $\eta$,
% $\eta'(\ell)=\eta(\ell)$.\label{pg:etaprime}  We can now compute the merge of the two
% flattened types we have obtained, that is,
% $\varphi$(\smallrecord{\smalltfield{prev}{$T_{\mathrm{comm}}$}})\d{$\wedge$}$[\varphi(\mathcal{F}(f))]_{\eta'}$.
% Finally, we can unflatten the resulting merge, obtaining the final
% result of the update,
% $\varphi^-$($\varphi$(\smallrecord{\smalltfield{prev}{$T_{\mathrm{comm}}$}})\d{$\wedge$}$[\varphi(\mathcal{F}(f))]_{\eta'}$).

\section{Proper names, salience and accommodation}
\label{sec:accommodation}

What we have presented so far enables us to find a match for
presuppositions introduced by a parametric content when such a match
is present in shared commitments.  Suppose there is more than one such
match.  In that case there will be a choice of relabellings $\eta$.
In this case we may wish to choose the relabelling that corresponds to
a match with the most salient match in terms of recency of
introduction into the shared commitments.  Technically, this means
that we choose the relabelling which introduces paths with the least
number of occurrences of `prev'.  Note that the most recent match may
be anchored to a match that was introduced earlier in the manner we
have just described.  There may be other factors than recency which
contribute to salience, for example, the kinds of factors that are
discussed in centering theory \citep{JoshiWeinstein1981,GroszJoshiWeinstein1983,GroszJoshiWeinstein1995,WalkerJoshiPrince1998,PoesioStevensonEugenioHitzeman2004}.  We will leave it to future work to give a
more detailed account of saliency in the current framework.

What happens when there is no match for \textit{Sam} in the shared
commitments?  Here we need some kind of accommodation in order to use
the parametric content to update the gameboard.  There are two kinds
of accommodation we will consider.  The first is where the agent knows
of a person named Sam independently of the current conversation.  That
is, a match for \textit{Sam} can be found in the agent's resources
corresponding to long term memory.  We will not attempt a detailed
account of the stucture of long term memory.  We assume that it is
complex and constantly in flux not only in terms of new information
being added but also in terms of what is salient in the old
information, depending on which part of the memory is being focussed
on at any particular time.  Here we will content ourselves with a
simple model of long term memory as a record type of a similar kind to
that we have proposed for shared commitments.  This means that the
techniques we need for matching will be the same as those discussed
above.  In reality the notion of salience with respect to long term
memory will be a good deal more complicated than salience with respect
to the shared commitments on the dialogue gameboard.  You have to take
into account not only recency but also likelihood based on other
knowledge that it is this particular Sam that is being referred to.
For example, if you believe that your interlocutor could not possibly
know of the Sam in your memory who is otherwise the most likely
candidate you should not choose that Sam as a match.  Choosing an
appropriate match involves a great deal of world knowledge and common
sense.  We will ignore these matters and concentrate our attention on
what needs to be done if we find a suitable match.  The idea is that
if you have failed to find a match in shared commitments on the
gameboard but you do find a match in long term memory, then you need
to load the item from long term memory into the shared commitments on
your gameboard.  This is what will constitute accommodation in this
case.

We will introduce the notion of a \textit{total information state} \citep[\textit{cf.}][]{Larsson2002}
which includes a record type corresponding to long term memory,
represented by the `ltm'-field in \nexteg{} and a dialogue gameboard,
represented by the `gb'-field in \nexteg{}.  Up until now we have
thought of the gameboard as a record type.  Now, however, we want to
be able to make links from the gameboard to long term memory and we
will achieve this by making the gameboard be a dependent type which
maps records (situations) of the type representing long term memory to
the record type representing the gameboard.  Thus a total information state
will be of the type \nexteg{}.
\begin{ex} 
\record{\tfield{ltm}{\textit{RecType}} \\
        \tfield{gb}{(ltm$\rightarrow$\textit{GameBoard})}} 
\end{ex} 
Here we use \textit{GameBoard} as the type of types which are a
subtype of \textit{InfoState} (for example, as defined most recently
in Chapter~\ref{ch:gram}, p.\pageref{ex:InfoState-chgram}), that is, a gameboard is a type of
information states.  Formally, this is expressed as in \nexteg{}.
\begin{ex} 
$T$ : \textit{GameBoard} iff $T\sqsubseteq$ \textit{InfoState}
\label{ex:GameBoard}
\end{ex} 

  
An example of a type corresponding to long term memory is given in
\nexteg{}.
\begin{ex} 
\smallrecord{\smalltfield{id$_0$}{\textit{Rec}}\\
             \smalltfield{id$_1$}{\smallrecord{\smalltfield{x}{\textit{Ind}}\\
                                               \smalltfield{e}{named(x,
                                                 ``Dudamel'')}}}\\ 
             \smalltfield{id$_2$}{\smallrecord{\smalltfield{e}{conductor($\Uparrow$id$_1$.x)}}}\\
             \smalltfield{id$_3$}{\smallrecord{\smalltfield{x}{\textit{Ind}}\\
                                               \smalltfield{e}{named(x,
                                                 ``Beethoven'')}}}\\
             \smalltfield{id$_4$}{\smallrecord{\smalltfield{e}{composer($\Uparrow$id$_3$.x)}}}\\
             \smalltfield{id$_5$}{\smallrecord{\smalltfield{x}{\textit{Ind}}\\
                                               \smalltfield{e}{named(x,
                                                 ``Uchida'')}}}\\
             \smalltfield{id$_6$}{\smallrecord{\smalltfield{x}{pianist($\Uparrow$id$_5$.x)}}}\\
             \smalltfield{id$_7$}{\smallrecord{\smalltfield{x}{\textit{Ind}}\\
                                               \smalltfield{e}{named(x,
                                                 ``Sam'')}}}\\
             \smalltfield{id$_8$}{\smallrecord{\smalltfield{e}{singer($\Uparrow$id$_7$.x)}}}}
\label{ex:ltm} 
\end{ex} 
\preveg{} is one way of putting the information in shared commitments represented by
(\ref{ex:sharedcommsnonexpanded}) into a type corresponding to long term
memory.  We are assuming that in long term memory information is
indexed by unique identifiers modelled here by the labels `id$_n$' (of
which we assume there is a countably infinite stock, one for each
natural number, $n$).  It is important that in long term memory paths
are persistent under updating, that is, the old paths do not change
when we add information to long term memory.  This is in contrast to
the kind of updating we proposed for the gameboard, adding the label
`prev' to the path for the old gameboard.  This meant that all paths
within the old gameboard were adjusted by an update.  When we link from the
gameboard to long term memory we want to make sure that the link uses
a persistent path which will still be correct if the long term memory
should get updated.  When long term memory is updated we prefix the
path to the new information with 
the identifier `id$_{i+1}$', where $i$ is the highest index on an
`id'-label in the long term memory type we are updating.  (This is the
same technique we used for `e'-labels in our treatment of chart parsing
in Chapter~\ref{ch:gram}.) The way of achieving the link is
illustrated schematically in \nexteg{} where we use $M$ to represent
the long term memory \preveg{} and leave out all irrelevant details of
the gameboard.
\begin{ex} 
\scalebox{.85}{\smallrecord{\smallmfield{ltm}{$M$}{\textit{RecType}}\\
        \smallmfield{gb}{$\lambda r$:ltm . \smallrecord{\ldots\\
                                              \smalltfield{shared}{\smallrecord{\smallmfield{commitments}{\smallrecord{\ldots\\
                                                                                                   \smalltfield{bg}{\smallrecord{\smallmfield{x}{$r$.id$_7$.x}{\textit{Ind}}\\
                                                                                                                                 \smallmfield{e}{$r$.id$_7$.e}{named(x,``Sam'')}}}\\
                                                                                                   \smalltfield{fg}{\smallrecord{\smalltfield{e}{leave($\Uparrow$bg.x)}}}
                                                                                                   }}{\textit{RecType}}}}\\
                                                                                             \ldots}}{(ltm$\rightarrow$\textit{RecType})}}
                                                                                       }
\label{ex:tisSamLeaves}
\end{ex} 
The intuition expressed in \preveg{} is as follows:  given a
situation, $r$,
of the type represented by our long term memory, that is one in which
a particular appropriate individual is labelled by `id$_7$', the gameboard will
be a type of information state where the background of the parametric
content used to update the shared commitments is anchored to `id$_7$'.  Two agents are aligned in their shared
commitments to the extent that we can find an equivalence between the
two types which represent their respective view of the shared
commitments obtained by applying their respective functions labelled `gb' to a
situation of their respective memory types. 


The link represented by the dependence on the long term memory type corresponds
to what \cite{Kamp1990,KampGenabithReyle2011} call an internal
anchor.  We are representing here how individual roles in an agent's
view of shared commitments can be anchored in that agent's long term
memory.  In a more complete treatment we could in addition make the
gameboard depend on a type for the current visual scene and also types
for other sensory input.  Our use of dependent types and Kamp
\textit{et al.}'s use of internal anchors allow us to link different
components of cognitive structure.  Cognitive structure can also be
linked to objects in the external world, giving rise to what Kamp
\textit{et al.} call external anchors.  Our manifest fields can be
used to correspond to their external anchors.  Suppose, for example,
that we have an individual `sam' who is named Sam.  We can use a
manifest field to restrict the long term memory type (\ref{ex:ltm}) so that any record
(``situation'') of that type has `sam' in the `id$_7$.x'-field.  This
is represented in \nexteg{} where for convenience we have omitted all but the
`id$_7$'-field in (\ref{ex:ltm}).
\begin{ex} 
\smallrecord{\ldots\\
             \smalltfield{id$_7$}{\smallrecord{\smallmfield{x}{sam}{\textit{Ind}}\\
                                               \smalltfield{e}{named(x,
                                                 ``Sam'')}}}\\
             \ldots}
\end{ex} 
If $M$ in (\ref{ex:tisSamLeaves}) is the type \preveg{} then for any
$r:M$, it will be the case that $r$.id$_7$.x will be `sam'.  Thus the
shared commitment is that `sam' leaves.  Given that manifest fields
can occur in any record type, this kind of external anchoring is not
restricted to long term memory but could also be directly in the
gameboard if that is desired. 

Let us now consider how the update of a gameboard dependent on long
term memory can be carried out when there is a match between the
parametric content used for updating and an item in long term memory.
Suppose that agent $A$'s current total information state, $s_{i,A}^{\text{tot}}$, is of the type in \nexteg{}.
\begin{ex} 
\record{\tfield{ltm}{\textit{RecType}}\\
        \mfield{gb}{$\lambda r$:ltm . $T_{\mathrm{gb}}\dep{r}$}{(ltm$\rightarrow$\textit{RecType})}}
\end{ex} 
and that we wish to update this with the parametric content, $f$, given in \nexteg{} (where
$T_{\mathrm{bg}}\sqsubseteq$ \smallrecord{\smalltfield{x}{\textit{Ind}}}).
\begin{ex}
\record{\field{bg}{$T_{\mathrm{bg}}$}\\ 
        \field{fg}{$\lambda r$:$T_{\mathrm{bg}}$ . $T_{\mathrm{upd}}\dep{r}$}}
\end{ex} 
In order to find a match between $f$.bg, that is, $T_{\mathrm{bg}}$ and $s_{i,A}^{\text{tot}}$.ltm
(that is, to ascertain that the presupposition associated with the
parametric content is met by the long term memory of the current total
information state) we need to find a relabelling, $\eta$, of $T_{\mathrm{bg}}$ such that
\nexteg{} holds.
\begin{ex} 
$s_{i,A}^{\text{tot}}$.ltm $\sqsubseteq$ $[T_{\mathrm{bg}}]_\eta$ 
\end{ex} 
Then we can derive \nexteg{} as a type of the updated total
information state.
\begin{ex} 
\smallrecord{\smallmfield{ltm}{$s_{i,A}^{\text{tot}}$.ltm}{\textit{RecType}}\\
        \smallmfield{gb}{$\lambda r$:ltm
          . ($T_{\mathrm{gb}}\dep{r}$\fbox{\d{$\wedge$}}\\\hspace*{5em}\smallrecord{\smalltfield{shared}{\smallrecord{\smallmfield{commitments}{\smallrecord{\smalltfield{prev}{$s_{i,A}^{\text{tot}}$.gb.shared.commitments}\\
                                                                                                                           \smalltfield{bg}{$T_{\mathrm{bg}}\parallel_\eta
                                                                                                                             r$}\\
                                                                                                                           \smalltfield{fg}{$f$(bg)}}}{\textit{RecType}}}}})\\\hspace*{25em}}{(ltm$\rightarrow$\textit{RecType})}}

 
\end{ex}
% Here the notation $T_{\mathrm{bg}}\parallel_\eta r$ represents the
% specification or anchoring of the type $T_{\mathrm{bg}}$ by the record $r$
% according to the relabelling $\eta$.  That is, we replace fields in
% $T_{\mathrm{bg}}$ with manifest fields according to the matches we
% have in the `ltm'-field.
Thus, for example, if $T_{\mathrm{bg}}$ is
(\ref{ex:namedSam}), repeated as \nexteg{a}, $r$ is a record
representing long term memory of type (\ref{ex:ltm}), repeated as
\nexteg{b}, and $\eta$ is the relabelling in \nexteg{c}, then
$T_{\mathrm{bg}}\parallel_\eta r$ is \nexteg{d}.
\begin{ex} 
\begin{subex} 
 
\item \smallrecord{\smalltfield{f}{\smallrecord{\smalltfield{x}{\textit{Ind}}\\
                                               \smalltfield{e}{named(x,
                                                 ``Sam'')}}}\\
                  \smalltfield{a}{\textit{Rec}}} 
 
\item \smallrecord{\smalltfield{id$_0$}{\textit{Rec}}\\
             \smalltfield{id$_1$}{\smallrecord{\smalltfield{x}{\textit{Ind}}\\
                                               \smalltfield{e}{named(x,
                                                 ``Dudamel'')}}}\\ 
             \smalltfield{id$_2$}{\smallrecord{\smalltfield{e}{conductor(id$_1$.x)}}}\\
             \smalltfield{id$_3$}{\smallrecord{\smalltfield{x}{\textit{Ind}}\\
                                               \smalltfield{e}{named(x,
                                                 ``Beethoven'')}}}\\
             \smalltfield{id$_4$}{\smallrecord{\smalltfield{e}{composer(id$_3$.x)}}}\\
             \smalltfield{id$_5$}{\smallrecord{\smalltfield{x}{\textit{Ind}}\\
                                               \smalltfield{e}{named(x,
                                                 ``Uchida'')}}}\\
             \smalltfield{id$_6$}{\smallrecord{\smalltfield{x}{pianist(id$_5$.x)}}}\\
             \smalltfield{id$_7$}{\smallrecord{\smalltfield{x}{\textit{Ind}}\\
                                               \smalltfield{e}{named(x,
                                                 ``Sam'')}}}\\
             \smalltfield{id$_8$}{\smallrecord{\smalltfield{e}{singer(id$_7$.x)}}}}

\item  f.x $\leadsto$ id$_7$.x\\
       f.e $\leadsto$ id$_7$.e\\
       a $\leadsto$ id$_0$

\item \smallrecord{\smalltfield{f}{\smallrecord{\smallmfield{x}{$r$.id$_7$.x}{\textit{Ind}}\\
                                               \smallmfield{e}{$r$.id$_7$.e}{named(x,
                                                 ``Sam'')}}}\\
                  \smallmfield{a}{$r$.id$_0$}{\textit{Rec}}} 
 
\end{subex} 
   
\end{ex} 
% A precise and general definition of this notation is in Appendix~\ref{app:specrec}.   


 
We can now put all this together as the action rule in \nexteg{},
which we call \textsc{AccLTM} (``accommodate match with long term
memory'').  

\begin{sidewaysfigure}
  \begin{ex}
    
    % \hspace*{-4em}
    \scalebox{.85}{
  \begin{prooftree}[separation=.5em]
    \hypo{s_{i,A}^{\text{tot}}:T_{\text{tot}}}
    \hypo{T_{\text{tot}}\sqsubseteq\text{\smallrecord{\
          \smalltfield{ltm}{\textit{RecType}}\\
          \smalltfield{gb}{ltm$\rightarrow$\textit{GameBoard}}}}}
    \hypo{s_{i,A}^{\text{tot}}.\text{gb}(s_{i,A}^{\text{tot}}.\text{ltm})\sqsubseteq\text{\smallrecord{
          \smalltfield{shared}{\smallrecord{
              \smalltfield{commitments}{\textit{RecType}}}}}}}
    \hypo{u^*.\text{cont}:_A T_{\text{utt}}}
    \hypo{T_{\text{utt}}\sqsubseteq\text{\smallrecord{
          \smalltfield{bg}{\textit{RecType}}\\
          \smalltfield{fg}{(bg$\rightarrow$\textit{RecType})}}}}
    \infer[enth]5[$\eta$]{s_{i+1,A}^{\text{tot}}:_A\text{\smallrecord{
          \smallmfield{ltm}{$s_{i,A}^{\text{tot}}$.ltm}{\textit{RecType}}\\
          \smallmfield{gb}{$\lambda r$:ltm
            . $s_{i,A}^{\text{tot}}$.gb$(r)$\fbox{\d{$\wedge$}}\smallrecord{
              \smalltfield{shared}{\smallrecord{
                  \smallmfield{commitments}{\smallrecord{
                      \smalltfield{prev}{$(s_{i,A}^{\text{tot}}.\text{gb}(r))$.shared.commitments}\\
                      \smalltfield{bg}{$u^*$.cont.bg$\parallel_\eta
                        r$}\\
                      \smalltfield{fg}{$u^*$.cont.fg$($bg$)$}}}{\textit{RecType}}}}}}{$($ltm$\rightarrow$\textit{GameBoard}$)$}}}}
  \end{prooftree}
  }

  where $\eta$ is a relabelling for $u^*$.cont.bg with range included
  in $s_{i,A}^{\text{tot}}$.ltm
% \textbf{AccLTM}($\eta$) =\\ 
% $\lambda r$:\record{\tfield{ltm}{\textit{RecType}}\\
%         \tfield{gb}{(ltm$\rightarrow$\textit{GameBoard})}}\\
% \hspace*{1em}$\lambda
% f$: \record{\tfield{bg}{\textit{RecType}}\\
%             \tfield{fg}{(bg$\rightarrow$\textit{RecType})}} . \\
% \hspace*{2em} \smallrecord{\smallmfield{ltm}{$r$.ltm}{\textit{RecType}}\\
%         \smallmfield{gb}{$\lambda r_1$:ltm
%           . (($r$.gb)($r_1$)\fbox{\d{$\wedge$}}\\\hspace*{3em}\smallrecord{\smalltfield{shared}{\smallrecord{\smallmfield{commitments}{
% \smallrecord{\smalltfield{prev}{($r$.gb)($\Uparrow^3$ltm).shared.commitments}\\
%              \smalltfield{bg}{$f$.bg $\parallel_\eta r_1$}\\
%              \smalltfield{fg}{$f$.fg(bg)}
% }}{\textit{RecType}}}}})\\\hspace*{25em}}{(ltm$\rightarrow$\textit{GameBoard})}}
\label{ex:AccLTM}
\end{ex}
\end{sidewaysfigure}
\label{pg:inverserelabelling} 
%Here \textit{GameBoard} is as defined in (\ref{ex:GameBoard}).
     
We have used accommodation from long term memory to represent the kind
of accommodation where the agent has a resource which provides a
match.  In a more complete treatment we could use this technique for
accommodation from other available resources such as the visual scene.

We now turn our attention to accommodation where there is no
appropriate match
with other resources.  This corresponds to the case where the hearer
does not know any appropriate person named Sam but merely adds that
there is a person named Sam to the shared dialogue commitments.
The first step in this update is to create a type from the parametric
content under consideration so that we can merge it with
\smallrecord{\smalltfield{prev}{$T$}}, where $T$ is the type
representing the current shared commitments.  
% The type we create is
% one where the domain type of the function which is the parametric
% content is embedded under the label `bg' (for ``background'') and the
% result is merged with the type which the function returns,
% appropriately relabelled.  In \nexteg{a} we repeat the parametric
% content given in (\ref{ex:parcontfixedpoint}) and in \nexteg{b} we
% give the corresponding backgrounding type. 
Suppose we are considering the parametric content, $\xi$, given in
\nexteg{a}. (Recall the corner quotes notation introduced on p.~\pageref{ex:cornerquote-notation}.) Then the type we will create from $\xi$ is defined as in
\nexteg{b} which is identical with \nexteg{c}.
\begin{ex} 
\begin{subex} 
 
\item $\xi$ = $\ulcorner\lambda
  r$:\smallrecord{\smalltfield{f}{\smallrecord{\smalltfield{x}{\textit{Ind}}\\
                                               \smalltfield{e}{named(x,
                                                 ``Sam'')}}}\\
                  \smalltfield{a}{\textit{Rec}}} . 
\smallrecord{\smalltfield{e}{leave($r$.f.x)}}$\urcorner$ % \record{\field{bg}{\smallrecord{\smalltfield{f}{\smallrecord{\smalltfield{x}{\textit{Ind}}\\
%                                                \smalltfield{e}{named(x,
%                                                  ``Sam'')}}}\\
%                   \smalltfield{a}{\textit{Rec}}}}\\
%               \field{fg}{$\lambda
%   r$:\smallrecord{\smalltfield{f}{\smallrecord{\smalltfield{x}{\textit{Ind}}\\
%                                                \smalltfield{e}{named(x,
%                                                  ``Sam'')}}}\\
%                   \smalltfield{a}{\textit{Rec}}} . 
% %\hspace*{2em}
% \smallrecord{\smalltfield{e}{leave($r$.f.x)}}}} 

\item \record{\tfield{bg}{$\xi$.bg}\\
              \tfield{fg}{\record{\tfield{e}{$\xi$.fg(bg)}}}}
 
\item \smallrecord{\smalltfield{bg}{\smallrecord{\smalltfield{f}{\smallrecord{\smalltfield{x}{\textit{Ind}}\\
                                               \smalltfield{e}{named(x,
                                                 ``Sam'')}}}\\
                                                 \smalltfield{a}{\textit{Rec}}}} \\
                  \smalltfield{fg}{\smallrecord{\smalltfield{e}{leave(bg.f.x)}}}} 
 
\end{subex} 
   
\end{ex}
% An explicit definition of how to obtain the backgrounding type, $\mathcal{B}(f)$, from
% a function, $f$, is given in Appendix~\ref{app:comptypes},
% p.~\pageref{pg:backgroundingtype}.
Suppose now that the current shared
commitments are given by the type in \nexteg{}.
\begin{ex} 
\smallrecord{\smalltfield{prev}{\smallrecord{\smalltfield{prev}{\smallrecord{\smalltfield{prev}{\textit{Rec}}\\
                                                                             \smalltfield{bg}{\smallrecord{\smalltfield{x}{\textit{Ind}}\\
                                                                                                           \smalltfield{e}{named(x, ``Dudamel'')}}}\\
                                                                              \smalltfield{fg}{\smallrecord{\smalltfield{e}{conductor($\Uparrow$bg.x)}}}}}
                                                                   \\
                                              \smalltfield{bg}{\smallrecord{\smalltfield{x}{\textit{Ind}}\\
                                                                            \smalltfield{fg}{\smallrecord{\smalltfield{e}{named(x, ``Beethoven'')}}}}}\\
                                              \smalltfield{fg}{\smallrecord{\smalltfield{e}{composer($\Uparrow$bg.x)}}}}}
                                   \\
              \smalltfield{bg}{\smallrecord{\smalltfield{x}{\textit{Ind}}\\
                                            \smalltfield{e}{named(x, ``Uchida'')}}}\\
              \smalltfield{fg}{\smallrecord{\smalltfield{e}{pianist($\Uparrow$bg.x)}}}}
       
\end{ex} 
Then the new shared commitments will be \nexteg{a} which is
\nexteg{b}.
\begin{ex} 
\begin{subex} 
 
\item  \smallrecord{\smalltfield{prev}{\smallrecord{\smalltfield{prev}{\smallrecord{\smalltfield{prev}{\smallrecord{\smalltfield{prev}{\textit{Rec}}\\
                                                                             \smalltfield{bg}{\smallrecord{\smalltfield{x}{\textit{Ind}}\\
                                                                                                           \smalltfield{e}{named(x, ``Dudamel'')}}}\\
                                                                              \smalltfield{fg}{\smallrecord{\smalltfield{e}{conductor($\Uparrow$bg.x)}}}}}
                                                                   \\
                                              \smalltfield{bg}{\smallrecord{\smalltfield{x}{\textit{Ind}}\\
                                                                            \smalltfield{e}{named(x, ``Beethoven'')}}}\\
                                              \smalltfield{fg}{\smallrecord{\smalltfield{e}{composer($\Uparrow$bg.x)}}}}}
                                   \\
              \smalltfield{bg}{\smallrecord{\smalltfield{x}{\textit{Ind}}\\
                                            \smalltfield{e}{named(x, ``Uchida'')}}}\\
              \smalltfield{fg}{\smallrecord{\smalltfield{e}{pianist($\Uparrow$bg.x)}}}}}} \d{$\wedge$} \smallrecord{\smalltfield{bg}{\smallrecord{\smalltfield{f}{\smallrecord{\smalltfield{x}{\textit{Ind}}\\
                                               \smalltfield{e}{named(x,
                                                 ``Sam'')}}}\\
                                                 \smalltfield{a}{\textit{Rec}}}} \\
                  \smalltfield{fg}{\smallrecord{\smalltfield{e}{leave($\Uparrow$bg.f.x)}}}}
 
\item \smallrecord{\smalltfield{prev}{\smallrecord{\smalltfield{prev}{\smallrecord{\smalltfield{prev}{\smallrecord{\smalltfield{prev}{\textit{Rec}}\\
                                                                             \smalltfield{bg}{\smallrecord{\smalltfield{x}{\textit{Ind}}\\
                                                                                                           \smalltfield{e}{named(x, ``Dudamel'')}}}\\
                                                                              \smalltfield{fg}{\smallrecord{\smalltfield{e}{conductor($\Uparrow$bg.x)}}}}}
                                                                   \\
                                              \smalltfield{bg}{\smallrecord{\smalltfield{x}{\textit{Ind}}\\
                                                                            \smalltfield{e}{named(x, ``Beethoven'')}}}\\
                                              \smalltfield{fg}{\smallrecord{\smalltfield{e}{composer($\Uparrow$bg.x)}}}}}
                                   \\
              \smalltfield{bg}{\smallrecord{\smalltfield{x}{\textit{Ind}}\\
                                            \smalltfield{e}{named(x, ``Uchida'')}}}\\
              \smalltfield{fg}{\smallrecord{\smalltfield{e}{pianist($\Uparrow$bg.x)}}}}}\\
\smalltfield{bg}{\smallrecord{\smalltfield{f}{\smallrecord{\smalltfield{x}{\textit{Ind}}\\
                                               \smalltfield{e}{named(x,
                                                 ``Sam'')}}}\\
                                                 \smalltfield{a}{\textit{Rec}}}} \\
                  \smalltfield{fg}{\smallrecord{\smalltfield{e}{leave($\Uparrow$bg.f.x)}}}}
 
\end{subex} 
   
\end{ex} 
We can now put this together as the action rule in \nexteg{},
which we call \textsc{AccNM} (``accommodate no match'').  This is the same as \textsc{AccLTM} in (\ref{ex:AccLTM}) except that
in the update for shared commitments 
there is no anchoring to long term memory.

  \begin{sidewaysfigure}
  \begin{ex}
    
    % \hspace*{-4em}
    \scalebox{.85}{
  \begin{prooftree}[separation=.5em]
    \hypo{s_{i,A}^{\text{tot}}:T_{\text{tot}}}
    \hypo{T_{\text{tot}}\sqsubseteq\text{\smallrecord{\
          \smalltfield{ltm}{\textit{RecType}}\\
          \smalltfield{gb}{ltm$\rightarrow$\textit{GameBoard}}}}}
    \hypo{s_{i,A}^{\text{tot}}.\text{gb}(s_{i,A}^{\text{tot}}.\text{ltm})\sqsubseteq\text{\smallrecord{
          \smalltfield{shared}{\smallrecord{
              \smalltfield{commitments}{\textit{RecType}}}}}}}
    \hypo{u^*.\text{cont}:_A T_{\text{utt}}}
    \hypo{T_{\text{utt}}\sqsubseteq\text{\smallrecord{
          \smalltfield{bg}{\textit{RecType}}\\
          \smalltfield{fg}{(bg$\rightarrow$\textit{RecType})}}}}
    \infer[enth]5{s_{i+1,A}^{\text{tot}}:_A\text{\smallrecord{
          \smallmfield{ltm}{$s_{i,A}^{\text{tot}}$.ltm}{\textit{RecType}}\\
          \smallmfield{gb}{$\lambda r$:ltm
            . $s_{i,A}^{\text{tot}}$.gb$(r)$\fbox{\d{$\wedge$}}\smallrecord{
              \smalltfield{shared}{\smallrecord{
                  \smallmfield{commitments}{\smallrecord{
                      \smalltfield{prev}{$(s_{i,A}^{\text{tot}}.\text{gb}(r))$.shared.commitments}\\
                      \smalltfield{bg}{$u^*$.cont.bg}\\
                      \smalltfield{fg}{$u^*$.cont.fg$($bg$)$}}}{\textit{RecType}}}}}}{$($ltm$\rightarrow$\textit{GameBoard}$)$}}}}
  \end{prooftree}
  }

  % where $\eta$ is a relabelling for $u^*$.cont.bg with range included
  % in $s_{i,A}^{\text{tot}}$.ltm
\end{ex}
\end{sidewaysfigure}

% \textbf{AccNM} =\\
% $\lambda r$:\record{\tfield{ltm}{\textit{RecType}}\\
%         \tfield{gb}{(ltm$\rightarrow$\textit{GameBoard})}}\\
% \hspace*{1em}$\lambda
% f$: \record{\tfield{bg}{\textit{RecType}}\\
%             \tfield{fg}{(bg$\rightarrow$\textit{RecType})}} . \\
% \hspace*{2em} \smallrecord{\smallmfield{ltm}{$r$.ltm}{\textit{RecType}}\\
%         \smallmfield{gb}{$\lambda r_1$:ltm
%           . (($r$.gb)($r_1$)\fbox{\d{$\wedge$}}\\\hspace*{3em}\smallrecord{\smalltfield{shared}{\smallrecord{\smallmfield{commitments}{
% \smallrecord{\smalltfield{prev}{($r$.gb)($\Uparrow^3$ltm).shared.commitments}\\
%              \smalltfield{bg}{$f$.bg}\\
%              \smalltfield{fg}{$f$.fg(bg)}
% }}{\textit{RecType}}}}})\\\hspace*{25em}}{(ltm$\rightarrow$\textit{GameBoard})}} 
% $\lambda r$:\record{\tfield{ltm}{\textit{RecType}}\\
%                     \tfield{gb}{(ltm$\rightarrow$\textit{GameBoard})}}\\
% \hspace*{1em}$\lambda
% f$:$\displaystyle{\bigvee_{T\sqsubseteq [\mathrm{x}:\mathit{Ind}]}}(T\rightarrow \mathit{RecType})$ . \\
% \hspace*{2em} \smallrecord{\smallmfield{ltm}{$r$.ltm}{\textit{RecType}}\\
%                            \smallmfield{gb}{$\lambda r_1$:ltm
%           . (($r$.gb)($r_1$)\fbox{\d{$\wedge$}}\\\hspace*{5em}\smallrecord{\smalltfield{shared}{\smallrecord{\smallmfield{commitments}{\smallrecord{\smalltfield{prev}{($r$.gb)($\Uparrow^3$ltm).shared.commitments}}\d{$\wedge$}$\mathcal{B}(f)$}{\textit{RecType}}}}})\\\hspace*{25em}}{(ltm$\rightarrow$\textit{GameBoard})}}
%\end{ex}




We can now adjust the preliminary version of \textsc{AccGB} given in
(\ref{ex:AccGBprelim}) which was the action rule for cases where there is a
match on the gameboard so that it matches the general format of action
rules for
total information states.  This is given in \nexteg{}.% Assuming that we have relabellings $\eta$ and $\eta'$
% as defined on p.~% \pageref{pg:etaprime}
% we can formulate the update
% function in \nexteg{}, which we call \textbf{AccGB}($\eta'$)
% (``accommodate match on gameboard'').
\begin{sidewaysfigure}
\begin{ex} 
  \textsc{AccGB} -- final version
  
\hspace*{-2.5em}\scalebox{.85}{\begin{prooftree}[separation=.5em]
  \hypo{s_{i,A}^{\text{tot}}:_A T_{\text{tot}}}
  \hypo{T_{\text{tot}}\sqsubseteq\text{\smallrecord{\
          \smalltfield{ltm}{\textit{RecType}}\\
          \smalltfield{gb}{ltm$\rightarrow$\textit{GameBoard}}}}}
    \hypo{s_{i,A}^{\text{tot}}.\text{gb}(s_{i,A}^{\text{tot}}.\text{ltm})\sqsubseteq\text{\smallrecord{
          \smalltfield{shared}{\smallrecord{
              \smalltfield{commitments}{\textit{RecType}}}}}}}
  \hypo{u^*.\text{cont}:_A T_{\text{utt}}}
  \hypo{T_{\text{utt}}\sqsubseteq\text{\smallrecord{
        \smalltfield{bg}{\textit{RecType}}\\
        \smalltfield{fg}{(bg$\rightarrow$\textit{RecType})}}}}
  \infer[enth]5[$\eta$]{s_{i+1,A}^{\text{tot}}:_A \text{\smallrecord{
          \smallmfield{ltm}{$s_{i,A}^{\text{tot}}$.ltm}{\textit{RecType}}\\
          \smallmfield{gb}{$\lambda r$:ltm
            . $s_{i,A}^{\text{tot}}$.gb$(r)$\fbox{\d{$\wedge$}}\smallrecord{
              \smalltfield{shared}{\smallrecord{
                  \smallmfield{commitments}{\smallrecord{
                      \smalltfield{prev}{$(s_{i,A}^{\text{tot}}.\text{gb}(r))$.shared.commitments}\\
                      \smalltfield{bg}{$u^*$.cont.bg$\parallel_\eta(s_{i,A}^{\text{tot}}.\text{gb}(r))$.shared.commitments}\\
                      \smalltfield{fg}{$u^*$.cont.fg$($bg$)$}}}{\textit{RecType}}}}}}{$($ltm$\rightarrow$\textit{GameBoard}$)$}}}}
  \end{prooftree}
  }
where $\eta$ is a relabelling for $u^*$.cont.bg with range included in $(s_{i,A}^{\text{tot}}.\text{gb}(s_{i,A}^{\text{tot}}.\text{ltm}))$.shared.commitments
\label{ex:AccGBfinal} 
\end{ex}
\end{sidewaysfigure}

% \begin{ex} 
% %\scalebox{.9}{
% \textbf{AccGB}($\eta$) -- final version


% \hspace*{-3em}$\lambda r$:\record{\tfield{ltm}{\textit{RecType}}\\
%         \tfield{gb}{(ltm$\rightarrow$\textit{GameBoard})}} . \\
% \hspace*{-2em} $\lambda
% f$:\smallrecord{\smalltfield{bg}{\textit{RecType}}\\
%                 \smalltfield{fg}{(bg$\rightarrow$\textit{RecType})}}
%               . \\
% \hspace*{-1em} \smallrecord{\smallmfield{ltm}{$r$.ltm}{\textit{RecType}}\\
%         \smallmfield{gb}{$\lambda r_1$:ltm
%           . $r$.gb($r_1$)\fbox{\d{$\wedge$}} 
% \smallrecord{\smalltfield{shared}{\smallrecord{\smallmfield{commitments}{\smallrecord{\smalltfield{prev}{$r$.gb.shared.commitments}\\
%                                                                                       \smalltfield{bg}{$f$.bg
%                                                                                         $\parallel_\eta$
%                                                                                         prev}\\
%                                                                                       \smalltfield{fg}{$f$.fg(bg)}}}{\textit{RecType}}}}}}{\\\hspace*{30em}(ltm$\rightarrow$\textit{RecType})}}
% %} 
% \end{ex}  

% \scalebox{.9}
% {\begin{ex}
% \textbf{AccGB}($\eta'$) =\\ 
% \hspace*{-1em}$\lambda r$:\record{\tfield{ltm}{\textit{RecType}}\\
%                     \tfield{gb}{(ltm$\rightarrow$\textit{GameBoard})}}\\
% \hspace*{-.7em}$\lambda
% f$:$\displaystyle{\bigvee_{T\sqsubseteq [\mathrm{x}:\mathit{Ind}]}}(T\rightarrow \mathit{RecType})$ . \\
% \hspace*{-.6em} \smallrecord{\smallmfield{ltm}{$r$.ltm}{\textit{RecType}}\\
%                            \smallmfield{gb}{$\lambda r_1$:ltm
%           . (($r$.gb)($r_1$)\fbox{\d{$\wedge$}}\\\hspace*{5em}\smallrecord{\smalltfield{shared}{\smallrecord{\smallmfield{commitments}{$\varphi^-$($\varphi$(\smallrecord{\smalltfield{prev}{($r$.gb)($\Uparrow^3$ltm).shared.commitments}})\d{$\wedge$}$[\varphi(\mathcal{F}(f))]_{\eta'}$)\\\hspace*{25em}}{\textit{RecType}}}}})\\\hspace*{25em}}{(ltm$\rightarrow$\textit{GameBoard})}}

%         \end{ex}}



The three action rules for accommodation that we have defined do not
specify how we choose the relabelling $\eta$.  They are governed by
the regime in \nexteg{}.
\begin{ex} 
\begin{subex} 
 
\item if there is a labelling, $\eta$, such that
$s_{i,A}^{\text{tot}}.\text{gb}(s_{i,A}^{\text{tot}}).\text{ltm}.\text{shared}.\text{commitments}\sqsubseteq[u^*.\text{cont}.\text{bg}]_\eta$,
  then use \text{AccGB} with $\eta$ 
 
\item else if there is a labelling, $\eta$, such that
  $s_{i,A}^{\text{tot}}.\text{ltm}\sqsubseteq[u^*.\text{cont}.\text{bg}]_\eta$
  then use \text{AccLTM} with $\eta$

  
\item else use \textsc{AccNM}
 
\end{subex} 
   
\end{ex} 
  
% \begin{ex} 
% If $A$ is an agent, $s_i$ is $A$'s current information state, $f$ is a
% parametric content of type $T_f$ such that
% \begin{quote}
% $T_f$ $\sqsubseteq$ \record{\tfield{bg}{\textit{RecType}}\\
%         \tfield{fg}{(bg$\rightarrow$\textit{RecType})}}
% \end{quote}
%  and
% $s_i:_A T_i$ for some $T_i$ such that 
% \begin{quote}
% $T_i$ $\sqsubseteq$ \smallrecord{\smalltfield{ltm}{\textit{RecType}}\\
%                                  \smalltfield{gb}{\smallrecord{\smalltfield{shared}{\smallrecord{\smalltfield{commitments}{\textit{RecType}}\\
%                                                                                                  \smalltfield{latest-move}{\smallrecord{\smallmfield{cont}{$f$}{$T_f$}}}}}}}}
% \end{quote}
% then \\
% \begin{quote}
% if there is some $\eta$ which is a relabelling of $f$.bg such that 
% \begin{quote}
% \mbox{$s_i$.gb.shared.commitments $\sqsubseteq$ $[f.\mathrm{bg}]_\eta$}
% \end{quote}
% % and there is some $\eta'$ which is a labelling of $\mathcal{F}(f)$ such that $\eta'$ is an extension of
% % $\eta$,\\[.25\baselineskip]  
% then $s_{i+1} :_A
% T_i$\fbox{\d{$\wedge$}}\textbf{AccGB}($\eta$)($s_i$)($f$) is
% licensed\\[\baselineskip]
% else if there is some $\eta$ which is a relabelling of
% $f$.bg such that $s_i$.ltm
% $\sqsubseteq$ $[f.\mathrm{bg}]_\eta$\\[.25\baselineskip]
% then $s_{i+1} :_A
% T_i$\fbox{\d{$\wedge$}}\textbf{AccLTM}($\eta$)($s_i$)($f$) is
% licensed\\[\baselineskip] 
% else $s_{i+1} :_A
% T_i$\fbox{\d{$\wedge$}}\textbf{AccNM}($s_i$)($f$) is
% licensed
% \end{quote} 
% \label{ex:propname-acc-lic}
% \end{ex}



This account of accommodation for proper names where a new item is
allowed to be created in memory when attempts at matching have failed
is similar to a proposal by \cite{GrooteLebedeva2010} to treat
accommodation as error handling when a match has failed to be found.
Our information states can be thought of as corresponding to their
environment which they consider to be not simply a list of individuals
but individuals with their properties, thus providing objects similar
to those like the record types which can be found in our information
states.  One difference between the two proposals, apart from the
obvious fact that our aim here has been to embed the theory in a more
general theory of dialogue, is that
\citeauthor{GrooteLebedeva2010} use a selection function to select the
matches thus apparently assuming an algorithm which yields a unique
result.  We, on the other hand, talk in terms of matches being licensed
and thereby allow for the possibility of non-deterministic
selection. What we have in common, though, is that in order to account
for the way accommodation is carried out we both add an additional
layer to a type
theory based  semantics and talk in procedural terms of
actions to be carried out:  we with our action rules and \citeauthor{GrooteLebedeva2010} with their error handling mechanism.
     
\section{Paderewski}
\label{sec:Paderewski}

\cite{Kripke1979} discusses the case of Peter who hears about a
pianist called Paderewski.  Later, in a different
context, he learns of a Polish national leader and Prime Minister
called Paderewski.  In reality there was a single (remarkable) man
called Paderewski who was both a famous concert pianist and a
distinguished statesman.  But Peter does not realize this and thinks
that he has learned about two distinct people, both named Paderewski.
Thus, in our terms, Peter's long term memory might be a subtype of
\nexteg{} for some natural numbers $i$, $j$, $k$ and $l$.
\begin{ex} 
\smallrecord{\smalltfield{id$_i$}{\smallrecord{\smalltfield{x}{\textit{Ind}}\\
                                               \smalltfield{e}{named(x,
                                                 ``Paderewski'')}}}\\
             \smalltfield{id$_j$}{\smallrecord{\smalltfield{e}{pianist($\Uparrow$id$_i$.x)}}}\\
             \smalltfield{id$_k$}{\smallrecord{\smalltfield{x}{\textit{Ind}}\\
                                               \smalltfield{e}{named(x,
                                                 ``Paderewski'')}}}\\
             \smalltfield{id$_l$}{\smallrecord{\smalltfield{e}{statesman($\Uparrow$id$_k$.x)}}}}
\label{ex:paderewski-unlinked}
\end{ex} 
\preveg{} technically allows for the two Paderewskis to be the same
individual but if there is nothing in Peter's long term memory that
requires them to be the same individual we will count that as
corresponding to his view of them as distinct.  If Peter were in this
state and asked whether the pianist Paderewski and the statesman
Paderewski were the same person Peter might reply, ``Well, I wouldn't
have thought so, but I suppose they could be the same person.  I don't
know.''  On being told that the two Paderewskis are in fact the same
person he might update his long term memory by carrying out the 
merge in \nexteg{a}, that is, his long term memory would now be
\nexteg{b}.
\begin{ex} 
\begin{subex} 
 
\item \smallrecord{\smalltfield{id$_i$}{\smallrecord{\smalltfield{x}{\textit{Ind}}\\
                                               \smalltfield{e}{named(x,
                                                 ``Paderewski'')}}}\\
             \smalltfield{id$_j$}{\smallrecord{\smalltfield{e}{pianist($\Uparrow$id$_i$.x)}}}\\
             \smalltfield{id$_k$}{\smallrecord{\smalltfield{x}{\textit{Ind}}\\
                                               \smalltfield{e}{named(x,
                                                 ``Paderewski'')}}}\\
             \smalltfield{id$_l$}{\smallrecord{\smalltfield{e}{statesman($\Uparrow$id$_k$.x)}}}}\d{$\wedge$}\smallrecord{\smalltfield{id$_i$}{\smallrecord{\smalltfield{x}{\textit{Ind}}}}\\
                        \smalltfield{id$_k$}{\smallrecord{\smallmfield{x}{$\Uparrow$id$_i$.x}{\textit{Ind}}}}} 
 
\item \smallrecord{\smalltfield{id$_i$}{\smallrecord{\smalltfield{x}{\textit{Ind}}\\
                                               \smalltfield{e}{named(x,
                                                 ``Paderewski'')}}}\\
             \smalltfield{id$_j$}{\smallrecord{\smalltfield{e}{pianist($\Uparrow$id$_i$.x)}}}\\
             \smalltfield{id$_k$}{\smallrecord{\smallmfield{x}{$\Uparrow$id$_i$.x}{\textit{Ind}}\\
                                               \smalltfield{e}{named(x,
                                                 ``Paderewski'')}}}\\
             \smalltfield{id$_l$}{\smallrecord{\smalltfield{e}{statesman($\Uparrow$id$_k$.x)}}}} 
 
\end{subex} 
\label{eg:Paderewski-linked}   
\end{ex} 
Eventually, his long term memory may be restructured to the type in
\nexteg{} which is set equivalent to that in \preveg{}, though not
multiset equivalent to it since in any record of this type the
individual named Paderewski will only occur once, not twice as in
\preveg{}.
\begin{ex} 
\smallrecord{\smalltfield{id$_i$}{\smallrecord{\smalltfield{x}{\textit{Ind}}\\
                                               \smalltfield{e}{named(x,
                                                 ``Paderewski'')}}}\\
             \smalltfield{id$_j$}{\smallrecord{\smalltfield{e}{pianist($\Uparrow$id$_i$.x)}}}\\
             
             \smalltfield{id$_l$}{\smallrecord{\smalltfield{e}{statesman($\Uparrow$id$_i$.x)}}}}  
\end{ex} 
We might think of the two types (\ref{eg:Paderewski-linked}b) and \preveg{} as
representing two subtly different states of mind which Peter could be
in.  In (\ref{eg:Paderewski-linked}b) he has two concepts of Paderewski, one concept associated with him
being a pianist and perhaps other associated properties, such as
practicing hard, wearing tails when he is performing, and so on and
the other concept where he is a statesman, and perhaps associated with
other properties such as being a dynamic national leader, a driver of
hard political bargains or whatever.  In \preveg{} he has a
single concept of Paderewski including all he knows about him. The
first state is perhaps a natural one to be in after just learning that
the two Paderewskis are in fact the same, before you have fully
assimilated the identity.  It is harder to discover contradictions
between the two concepts here since it will only be the manifest field
linking the two concepts which will reveal the contradiction.
Suppose, for example, Peter's concept of the statesman Paderewski has
him always late for appointments and pressed for time whereas his
concept of the pianist Paderewski has him never late for appointments
and not pressed for time.  There is no contradiction in the state when
Peter believes there to be two Paderewskis.  Checking for the
inconsistency in the two concept state involves reasoning about the
identity expressed by the manifest field.  One could imagine a simple
consistency checker that does not do this -- logically inadequate, of
course, but human perhaps.  The single concept state could, however,
involve a direct conflict between type and its negation which, one
imagines, even the simplest of consistency checkers would find.  Thus
if Peter finds himself in such a state he might need to refine the
properties that he was ascribing to the two Paderewskis in order to
make the unified concept of the single Paderewski consistent, for
example, by modifying the properties to be always late for political
meetings and pressed for time in his political life but never late to
a musical event and not pressed for time in concerts.

Note that the link that we have expressed between the two concepts in (\ref{eg:Paderewski-linked}b) 
does not involve anything like an external anchor. An alternative
offered us by the type theory to represent that the two Paderewskis
are identical is \nexteg{}, where we are using $p$ to represent the
individual Paderewski.
\begin{ex} 
\smallrecord{\smalltfield{id$_i$}{\smallrecord{\smallmfield{x}{$p$}{\textit{Ind}}\\
                                               \smalltfield{e}{named(x,
                                                 ``Paderewski'')}}}\\
             \smalltfield{id$_j$}{\smallrecord{\smalltfield{e}{pianist(id$_i$.x)}}}\\
             \smalltfield{id$_k$}{\smallrecord{\smallmfield{x}{$p$}{\textit{Ind}}\\
                                               \smalltfield{e}{named(x,
                                                 ``Paderewski'')}}}\\
             \smalltfield{id$_l$}{\smallrecord{\smalltfield{e}{statesman(id$_k$.x)}}}}  
\end{ex} 
Here the link between Peter's two concepts goes through the world
since both his Paderewski concepts are linked to the individual
$p$. If an agent's long term memory is a subtype of \preveg{}, then
\textit{Ind}$_p$ figures in the long term memory type (recall that the
manifest field \smallrecord{\smallmfield{x}{$p$}{\textit{Ind}}} is a
notation for \smallrecord{\smalltfield{x}{\textit{Ind}$_p$}}, where
\textit{Ind}$_p$ is a type whose only witness is $p$ (see
Appendix~\ref{app:singletontypes})).  We take this to mean that the
agent has a direct way of identifying Paderewski but that he has not
in this case become conscious of the identity of the object involved
in different perceptions of Paderewski.\footnote{One could choose to
  interpret such types differently in cognitive terms.}  The situation
could be that Peter observes Paderewski on the concert platform in
tails and then sees him later in the parliament building.  His
observations are connected to the same individual although without him
realizing that he has observed the same Paderewski twice.  Thus the
situation is similar to that decribed for \textit{Hesperus} and
\textit{Phosphorus} in \cite{Frege1892}.  In Frege's case the agent
was visually aware of the planet Venus on different occasions, conceived of as the Evening
Star (Hesperus) and the Morning Star (Phosphorus) without being aware
that the same heavenly body was being observed in the morning as in
the evening.  The difference between Frege's example and that
represented by \preveg{} is that in Frege's case two different proper
names were associated with the different observations of the same
individual whereas here the same proper name is being used for the
same individual, though without awareness that the proper name is
being associated with the same individual on both occasions.    

\cite{Ludlow2014} discusses Kripke's Paderewski and
argues that the reason that proper names can be used to refer to
different individuals can be due to the fact that our lexicons are
dynamic and that we use different microlanguages on different
occasions.  In this discussion he is building on previous work by
\cite{LarsonLudlow1993} although in that work the emphasis is on
interpreted logical forms (pairs of abstract syntactic representations
and semantic values such as truth values for sentences) rather than on
local microlanguages constructed for use in a particular situation as
argued for on the basis of a number of different kinds of examples in
\cite{Ludlow2014}.  In general the idea of local microlanguages being
constructed on the fly during the course of dialogues and for the
purposes at hand is something for which I have a great deal of
sympathy and have argued for in the past
\citep{CooperRanta2008,LarssonCooper2009,Cooper2010,Cooper2012}.  And
indeed \cite{Ludlow2014} is right to argue that proper names provide
support for this view of language.  The argument is straightforward in
the case of proper names and does not involve the kinds of subtleties
of meaning variation which can lead some people to suspicion of this view in the
case of other words. If somebody says to me at a party, ``I'd like to
introduce you to my friend Sam'' and indeed I have never met Sam
before, I can, as a competent speaker of English, immediately form an
association between the phonological type ``Sam'' and the individual
to whom I have been introduced.  It is obviously not part of my
competence as a speaker of English to know all of the individuals in
the universe named Sam.  Our competence lies rather in our ability to
make the connection between the phonological type (a name) and an
individual as the need arises.  The competence involves a
\textit{dynamic} process of acquiring a linguistic coupling of a
speech event type with another part of the world and not a \textit{static}
knowledge of all the available couplings.  Once I have added this
pairing, modelled in our terms as a sign type, to my resources, I have in a technical
sense modified my language.\footnote{In my case the resource is quite
  likely to disappear again shortly afterwards.  People vary in their
  ability to remember names.}  An advantage of sign-based approaches of
the kind we are proposing is that you do not have to resort to
subscripts in some logical language in order to distinguish between
pairings of the same phonological type with different individuals.
This is a trap which \cite{LarsonLudlow1993} fall into when they claim
that there are two (or more) names in such cases distinguished by
subscripts in logical form.  A disadvantage of this analysis is that
no two individuals could have the same name in logical form and thus
we would have to use something else to analyze sentences like \nexteg{}.
\begin{ex} 
My wife's sister, one of my graduate students and our
  neighbour all have the same name: Karin 
\end{ex} 
\preveg{} describes a confusing situation
which I have to contend with on a daily basis.  If the logical form
theory with subscripts were correct this sentence would be necessarily
false and one might have expected that the
natural way to describe this situation would rather have been
\nexteg{}.
\begin{ex} 
My wife's sister, one of my graduate students and our neighbour all
have similar names in that they are pronounced ``Karin'' 
\end{ex} 
\preveg{}, according to my intuitions, is not a natural way of
describing the situation.  This suggests to me that one would need
something in addition to, or in place of, a logical form with subscripts to explain how
speakers of natural languages individuate names.

One interpretation of Ludlow's proposal is that when a proper name is
used to refer to different individuals, different microlanguages
are used for the references to the different individuals.  Thus when
Elisabet says \textit{Karin} and means her sister, she is using a
slightly different language than when she says \textit{Karin} and means
our neighbour. While I am much in sympathy with the idea of different
microlanguages in general it seems to me that such a proposal could
not be quite right.  Consider dialogues like \nexteg{}, a kind of
dialogue which is not infrequent in our house.
\begin{ex} 
\begin{tabular}[t]{ll}
Elisabet: & Karin called \\
Robin: & Karin? \\
Elisabet: & My sister
\end{tabular} 
\end{ex} 
My utterance in \preveg{} is an example of what is called a
clarification request in the dialogue
literature \citep[and much other
literature]{GinzburgCooper2004,Ginzburg2012}.  According to that
literature one of the uses of a clarification request such as
\textit{Karin?} is to ask for further identification of the referent
of the use of the proper name in the previous dialogue turn.  It might
initially seem tempting to regard such a request as being in effect a
request for (partial) identification of the microlanguage Elisabet is
talking.  But if we take that route then we have to ask ourselves what
language the clarification request itself is in.  Assuming that we
have three variants of microlanguages available, one where
\textit{Karin} refers to Elisabet's sister, one where it refers to our
neighbour and one where it refers to my graduate student, then if the
request is in any of those languages the answer to the question is
selfevident and it is hard to see why I would ask it.  And in
particular if I was thinking of Karin, my graduate student, I might be
justified in saying that Elisabet's answer was wrong.  This, of
course, is not at all what is going on.  It seems that the
clarification request is part of a microlanguage in which
\textit{Karin} can be used to refer to any of the three and I am
interested in finding out which was meant here.  This is the kind of
option that might be offered by our sign-based approach where a single
(micro)language can contain several different signs with the same
phonology but with different contents.  The exact treatment of this
needs, of course, an account of questions and clarification questions
in particular which we will not undertake here.

One can understand, however, why the idea of a single referent for
a proper name in a single microlanguage might seem attractive.  When
\cite{Kripke1979} introduces the puzzle about Peter and Paderewski he
is careful to point out the circumstances under which Peter came to
the conclusion that there were two Paderewskis.  Peter first learns
the name Paderewski in connection with the famous pianist.  Then:
``Later, in a different circle, Peter learns of someone called
`Paderewski' who was a Polish nationalist leader and prime minister.''
Kripke's example would not have been at all as convincing if Peter had
learned about Paderewski, the pianist and Paderewski, the statesman
from the same person in the same conversation.  \cite{Ludlow2014} makes a similar
point in criticising Kripke's construction of the apparent
contradiction that Peter believes, namely that Paderewski both is and
is not a pianist.  ``The fallacy involves the conjunction of two
sentences that have the appearance of contradicting each other\ldots
but they do not contradict because they come from different
microlanguages.'' (p. 148).  The fact of the matter is that we do tend
to use proper names to refer uniquely within the same dialogue, all
other things being equal.  Suppose we are involved in a conversation about
pianists and have been, say, comparing the relative merits of
Paderewski and Ashkenazy, and at some point I say \nexteg{}
\begin{ex} 
Paderewski was a
leading statesman in Poland 
\end{ex} 
You would naturally infer that I was
talking about the same Paderewski, unless I explicitly point out that
I intend to refer to a different person with the same name.   It
is, of course, possible to refer to two different people with the same
name within the same dialogue and even within the same sentence, even
though it may lead to confusion.  The assumption is normally, though,
that within the space of a dialogue a name will refer to a unique
individual unless it is explicitly stated otherwise.  One way of being
explicit is to say something like \nexteg{}
\begin{ex} 
I know another person named
Paderewski 
\end{ex} 
If both dialogue participants are aware of the two
people with the same name it is possible to use the names together in
a construction which normally requires different intended referents as
in \nexteg{}.\footnote{I am
  grateful to Anders Tolland and Stellan Petersson for calling my
  attention to the fact that the Churchlands are often referred to as
  ``Churchland and Churchland''.}
\begin{ex} 

Churchland and Churchland think that replacement of symbol
manipulation computer-like devices\ldots with connectionist machines
hold (\textit{sic}) great promise 

\hfill\citep[][p. 21]{Globus1995} 
\end{ex} 
Two people named John engaged in conversation with a third person can
refer to each other with the name \textit{John} when addressing the
third person without risk of confusion as in \nexteg{}
\begin{ex} 
\begin{tabular}[t]{ll}
John E: & I remember John as an inspiring professor when I was a
student \\
John P: & Well, I remember John as an extremely bright student \\
Third person: & I didn't realize you'd known each other that long
\end{tabular} 
\end{ex} 
When addressing a person you can always use their name as a vocative
even if the message you wish to convey involves a person with the same
name as in \nexteg{}.
\begin{ex} 
\begin{tabular}[t]{ll}
A: & John, I'd like to introduce you to my good friend John \\
B: & Glad to meet you.  Another John, eh?
\end{tabular} 
\end{ex} 
It is conceivable that somebody would want to argue that all of these
cases where the same name is used twice to refer to different people
are examples of code-switching between microlanguages within the space
of a dialogue or sentence.  Since code-switching does take place even
between different languages like English and Portuguese within single
dialogues and sentences it is hard to say that such an analysis is
impossible.  However, given that a sign-based analysis of proper names
does not require these examples to be cases of code-switching perhaps
the onus is on the proponent of the code-switching analysis to
motivate this more complex analysis. 

Puzzles about proper names and reference such as the Paderewski puzzle
and Frege's (\citeyear{Frege1892}) original puzzle about
\textit{Hesperus} and \textit{Phosphorus} are standardly presented as
puzzles about belief reports.  Indeed the matters we have discussed in
this section do give rise to puzzles in belief reports and we will
return to this later.  However, we would like to claim that
the discussion here shows that the basis of these puzzles does not lie in
the analysis of belief reports \textit{per se} but in the nature of
communication in dialogue and the resulting organization of memory.
While these phenomena seem puzzling from a Fregean or Montagovian
formal language perspective, from the point of view of a dialogic
approach employing a sign-based analysis they seem to be a natural
consequence of the way that communication takes place and knowledge
gets stored.

\section{The interpretation of unbound pronouns}
\label{sec:unbound}
So far in this chapter we have talked about reference associated with
uses of proper names and seen that mental states play
an important role in explaining how reference works.  The sign-based
approach to proper names that we have taken is essentially different
from the traditional approach to proper names as being something like
constants in a logic, since it allows proper name phonologies to be
paired with different referents in different utterances.  In this way
our proper names are behaving more like free variables whose referents
are determined by a context.  So called deictic or unbound uses of
pronouns are often related to free variables in logic.  The following
question now arises:  how does our treatment of proper names relate to
our treatment of unbound pronouns?  In this section we will give a
treatment of unbound pronouns which will also lay a basis for the
treatment of binding in Chapter~\ref{ch:quant}.

We will consider how to recreate a simple interpretation of pronouns
ranging over individuals, first treating them in a similar way to free
variables in logic.  The central idea is to use records as \textit{pronominal
  contexts} which correspond to partial assignments to variables in
standard logical treatments.  Consider first a simple sentence with a deictic
pronoun as in \nexteg{}.
\begin{ex} 
he left 
\end{ex} 
In our initial pass we will ignore matters
of gender to make things simpler.  The content of \preveg{} is a type
which depends on a context (a situation) which provides a value of the
pronoun \textit{he}. Thus it will have a parametric content which is a
function from a context assigning a value to the pronoun to a
type.  In order to do this we will make use of \textit{structured
  contexts}.

The
contexts that we have used so far are records of general types corresponding to presuppositions
\citep{BeaverGeurts2014} or content not \textit{under discussion}
\citep{Ginzburg1994,Ginzburg2012}  or not \textit{at issue}
\citep{Potts2005}.  For example, \nexteg{} puts a constraint on the
context (expresses a presupposition) that there is an individual named
Sam.
\begin{ex} 
\record{\tfield{x}{\textit{Ind}}\\
              \tfield{e}{named(x, ``Sam'')}} 
\end{ex} 
In the treatment of unbound pronouns we will use records that are much
more restricted in nature and which correspond to partial assignments to 
variables.  Their labels will all be of the form `x$_i$' where $i$ is
a natural number and the record types that we will use will be
resticted to using a restricted number of non-dependent types. For the
current account of singular unbound pronouns we will restrict
ourselves to the type \textit{Ind}.  Thus \nexteg{} is an example of
the kind of record type we will use.
\begin{ex} 
\record{\tfield{x$_0$}{\textit{Ind}}\\
              \tfield{x$_1$}{\textit{Ind}}} 
\end{ex} 
  
We will regard these different kinds of
context as separate components of the context as they may be treated differently by compositional
semantics, for example, in terms of their projection properties which
regulates to what extent contexts required by lower constituents are
also required by higher constituents, as discussed, for example, by
\cite{Potts2005}.  we will have parametric
contents which look like \nexteg{}.
\begin{ex} 
$\lambda c$:\smallrecord{\smalltfield{$\mathfrak{s}$}{\smallrecord{\smalltfield{x$_0$}{\textit{Ind}}\\
              \smalltfield{x$_1$}{\textit{Ind}}}}\\
                \smalltfield{$\mathfrak{c}$}{\smallrecord{\smalltfield{x}{\textit{Ind}}\\
              \smalltfield{e}{named(x, ``Sam'')}}}\\
\ldots} . $\varphi\dep{c}$ 
\end{ex}

We use the label `$\mathfrak{s}$' for the component in the context which
assigns entities to occurrences of pronouns and the label
`$\mathfrak{c}$' for the component which corresponds to
presuppositions.  We leave open the possibility of adding further
components to the context later.

@@

We will use the variable
`$\mathfrak{s}$' to represent such contexts.  Thus the foreground of a parametric content for
\textit{he} could be that given in \nexteg{a}, for \textit{left}
(ignoring tense) that given in \nexteg{b} and their combination, using a variant
of S-combination which we will discuss as we progress, is represented
in \nexteg{c}.
\begin{ex} 
\begin{subex} 
 
\item $\lambda\mathfrak{s}$:\smallrecord{\smalltfield{x$_0$}{\textit{Ind}}} . 
        $\lambda P$:\textit{Ppty} . $P$(\smallrecord{\field{x}{$\mathfrak{s}$.x$_0$}})
 
\item $\lambda\mathfrak{s}$:\textit{Rec} . 
        $\lambda r$:\smallrecord{\smalltfield{x}{\textit{Ind}}} . 
              \record{\tfield{e}{leave($r$.x)}}

\item $\lambda\mathfrak{s}$:\smallrecord{\smalltfield{x$_0$}{\textit{Ind}}}
  . 
         \record{\tfield{e}{leave($\mathfrak{s}$.x$_0$)}}
 
\end{subex} 
   
\end{ex}
We follow Montague's strategy in allowing the content of \textit{he}
to be \nexteg{} for any natural number $i$.
\begin{ex}
$\lambda\mathfrak{s}$:\smallrecord{\smalltfield{x$_i$}{\textit{Ind}}} . 
        $\lambda P$:\textit{Ppty}
        . $P$(\smallrecord{\field{x}{$\mathfrak{s}$.x$_i$}})
\label{ex:pronoun-cont}
\end{ex} 
Thus considering the boy and the dog, a
content for \nexteg{a} will be \nexteg{b}.
\begin{ex} 
\begin{subex} 
 
\item he hugged it 
 
\item $\lambda\mathfrak{s}$:\smallrecord{\smalltfield{x$_0$}{\textit{Ind}}\\
                                   \smalltfield{x$_1$}{\textit{Ind}}}
                                 . 
\record{\tfield{e}{hug($\mathfrak{s}$.x$_0$, $\mathfrak{s}$.x$_1$)}} 
 
\end{subex} 
   
\end{ex} 
An advantage of using record types to characterize pronominal contexts
rather than variable assignments is that we can add further
information represented by the pronoun such as gender.   Thus a simple
treatment of gender for \preveg{} might be given by making the content
be \nexteg{}.
\begin{ex} 
$\lambda\mathfrak{s}$:\smallrecord{\smalltfield{x$_0$}{\textit{Ind}}\\
                             \smalltfield{c$_0$}{male(x$_0$)}\\
                             \smalltfield{x$_1$}{\textit{Ind}}\\
                             \smalltfield{c$_1$}{neuter(x$_1$)}}
                                 . 
\record{\tfield{e}{hug($\mathfrak{s}$.x$_0$, $\mathfrak{s}$.x$_1$)}} 
\end{ex} 
As we will see below, there are some complications with this simple
idea when it comes to the interpretation of pronouns which are bound
by quantifiers.  Even for deictic pronouns there are problems
determining which predicates should be used in a semantic treatment of
gender.  Even for a language like English which apparently has
semantic gender (as opposed to grammatical gender like German or
French), neuter can be used for objects which do not have gender (like
tables) and for animals other than humans which do have gender and
which can be referred to with masculine and feminine pronouns.

% \section{Structured contexts\ignore{, binding principles and
%     projection}}\todo{Moved from Ch~\ref{ch:quant}.  Needs rewriting.}
% \label{sec:struc-cntxt}

% The nature of the context that we have been using in this chapter is
% different from that of the contexts we used with parametric contents
% in previous chapters starting with Chapter~\ref{ch:propnames}.  The
% contexts in this chapter are records which constitute partial
% assignments of objects to labels `x$_0$', `x$_1$', \ldots whereas 

% Thus an example of the kind of context type used in this chapter is
% \nexteg{a} whereas an example of the kind of context types used
% previously is \nexteg{b}.
% \begin{ex} 
% \begin{subex} 
 
% \item \record{\tfield{x$_0$}{\textit{Ind}}\\
%               \tfield{x$_1$}{\textit{Ind}}} 
 
% \item  
 
% \end{subex} 
   
% \end{ex} 
% Contexts of types like \preveg{a} correspond to partial assignments to
% variables in logic where the labels `x$_0$', `x$_1$', \ldots
% correspond to a countably infinite set of variables in a logic.
% Context types like those in \preveg{b}  We
% reserve the labels `x$_0$', `x$_1$' for contexts giving assignments to
% pronouns which can be involved in binding of the kind described in
% this chapter.  In principle we could combine both kinds of contexts
% by, for example, merging the two types in \preveg{} as given in
% \nexteg{}.
% \begin{ex} 
% \record{\tfield{x}{\textit{Ind}}\\
%         \tfield{x$_0$}{\textit{Ind}}\\
%         \tfield{x$_1$}{\textit{Ind}}\\
%         \tfield{e}{named(x, ``Sam'')}} 
% \end{ex}

% However, it seems that   We could try to do this by having several layers of
% lambda abstraction for each of the different kinds of contexts.  That
% is, parametric contents taking into account several different kinds of
% contexts may have the kind of form represented in \nexteg{}.
% \begin{ex} 
% $\lambda\mathfrak{s}$:\smallrecord{\smalltfield{x$_0$}{\textit{Ind}}\\
%               \smalltfield{x$_1$}{\textit{Ind}}} $\lambda c$:\smallrecord{\smalltfield{x}{\textit{Ind}}\\
%               \smalltfield{e}{named(x, ``Sam'')}} \ldots\ \  . $\varphi(\mathfrak{s},c,\ldots)$
% \end{ex} 
% where the `\ldots' represent further lambda abstractions for any other
% kinds of contexts that might need to be distinguished and $\varphi$ is
% some content which may depend on all of these contexts.  One
% difficulty with this kind of proposal is that it is difficult to
% motivate what order the lambda abstractions should come in and it
% could become impossible to decide if we want to allow the different
% kinds of contexts to depend on each other.  In addition, it could
% become unnecessarily complicated to write down the compositional
% interpretation rules.  For this we will keep to having a single lambda
% abstraction for a context with several components which can be
% addressed separately when it comes to characterizing projection
% properties.  Thus instead of \preveg{} 
 

[Perhaps we should have introduced structured contexts already in
Ch. \ref{ch:propnames} on proper names, together with pronouns like
\textit{him}, \textit{himself} and simple ``binding'' principles.  Use
`x$_i$' to distinguish proper noun occurrences like we do for pronouns rather
than `f' and `a' as currently??]

\subsection{\textit{Sam likes him/himself}}

The parametric content for \textit{him} is given in \nexteg{}.

\begin{ex} 
$\lambda
c$:\smallrecord{\smalltfield{$\mathfrak{s}$}{\smallrecord{\smalltfield{x$_0$}{\textit{Ind}}}}\\
                \smallmfield{local}{$\mathfrak{T}$(\{$\mathfrak{s}$.x$_0$\})}{\textit{Type}}\\
\smallmfield{refl}{$\mathfrak{T}$($\emptyset$)}{\textit{Type}}} . 
$\lambda P$:\textit{Ppty} . $P$\{c.$\mathfrak{s}$.x$_0$\}
\end{ex} 

The parametric content of the reflexive pronoun \textit{himself} is
the same as \preveg{} except that the values associated with the
labels `local' and `refl' are switched, as shown in \nexteg{}.

\begin{ex} 
$\lambda
c$:\smallrecord{\smalltfield{$\mathfrak{s}$}{\smallrecord{\smalltfield{x$_0$}{\textit{Ind}}}}\\
                \smallmfield{local}{$\mathfrak{T}$($\emptyset$)}{\textit{Type}}\\
\smallmfield{refl}{$\mathfrak{T}$(\{$\mathfrak{s}$.x$_0$\})}{\textit{Type}}}
 . 
$\lambda P$:\textit{Ppty} . $P$\{c.$\mathfrak{s}$.x$_0$\}
\end{ex} 
We will refer to \preveg{} as \textbf{himself}.  The idea is that the `local'-field in the context must be passed up to
the VP unchanged, whereas the `refl'-field must be emptied
(discharged) at the VP-level.  Thus \textit{likes him} has the
parametric content in \nexteg{a} whereas \textit{likes himself} has
the parametric content in \nexteg{b}.
\begin{ex} 
\begin{subex} 
 
\item $\lambda
c$:\smallrecord{\smalltfield{$\mathfrak{s}$}{\smallrecord{\smalltfield{x$_0$}{\textit{Ind}}}}\\
                \smallmfield{local}{$\mathfrak{T}$(\{$\mathfrak{s}$.x$_0$\})}{\textit{Type}}\\
\smallmfield{refl}{$\mathfrak{T}$($\emptyset$)}{\textit{Type}}} . 
$\lambda r$:\smallrecord{\smalltfield{x}{\textit{Ind}}} . \record{\tfield{e}{like($r$.x,
$\lambda P$:\textit{Ppty} . $P$\{$c$.$\mathfrak{s}$.x$_0$\})}} 
 
\item $\lambda
c$:\smallrecord{\smallmfield{local}{$\mathfrak{T}$($\emptyset$)}{\textit{Type}}\\
  \smallmfield{refl}{$\mathfrak{T}$($\emptyset$)}{\textit{Type}}} . 
$\lambda r$:\smallrecord{\smalltfield{x}{\textit{Ind}}} . \record{\tfield{e}{like($r$.x,
  \textbf{himself}(\smallrecord{\field{$\mathfrak{s}$}{\smallrecord{\field{x$_0$}{$r$.x}}}\\
                                \field{local}{$\mathfrak{T}$($\emptyset$)}\\
                                \field{refl}{$\mathfrak{T}$(\{$r$.x\})}}))}}
 
 
\end{subex} 
   
\end{ex}
The idea is that context parameters represented by `$\mathfrak{s}$'
are open for binding unless object supplying that parameter are
required to be of type `$c$.local'.  Reflexive parameters are, however,
required to be of type `$c$.refl' and these must be bound within the VP.
Note that since the reflexive is bound by `$r$.x' in \preveg{b} it is
not represented in `$c$.refl' in the context for the VP.  Thus
\textit{himself} on its own requires an individual to be supplied by
the context to give it a referent.  However, for the content of
\textit{likes himself} no such individual is required from the context
since the property which is the content of the VP is that of ``being
an $x$ such that $x$ likes $x$''.  [Better way to talk about this?].

[Possible revision of treatment of proper names:]
The content of an utterance of \textit{Sam} is given in \nexteg{}.
\begin{ex} 
$\lambda
c$:\smallrecord{\smalltfield{$\mathfrak{c}$}{\smallrecord{\smalltfield{id$_0$}{\smallrecord{\smalltfield{x}{\textit{Ind}}\\
                                                                                           \smalltfield{e}{named(x,
                                                                                             ``Sam'')}}}}}}
                                                                                 . $\lambda
                                                                                 P$:\textit{Ppty}
                                                                                 . $P$($c$.$\mathfrak{c}$.id$_0$) 
\end{ex} 
The content of \textit{Sam likes him} is given in \nexteg{a} and
\textit{Sam likes himself} in \nexteg{b}.
\begin{ex} 
\begin{subex} 
 
\item $\lambda
  c$:\smallrecord{\smalltfield{$\mathfrak{s}$}{\smallrecord{\smalltfield{x$_0$}{\textit{Ind}}}}\\
                  \smallmfield{local}{$\mathfrak{T}$($\emptyset$)}{\textit{Type}}\\
\smallmfield{refl}{$\mathfrak{T}$($\emptyset$)}{\textit{Type}}\\
\smalltfield{$\mathfrak{c}$}{\smallrecord{\smalltfield{id$_0$}{\smallrecord{\smalltfield{x}{\textit{Ind}}\\
                                                                            \smalltfield{e}{named(x,
                                                                              ``Sam'')}}}}}}
                                                                  . 
\record{\tfield{e}{like($c$.$\mathfrak{c}$.id$_0$.x, $\lambda
    P$:\textit{Ppty} . $P$\{$c$.$\mathfrak{s}$.x$_0$\})}}
 
\item $\lambda
  c$:\smallrecord{
\smallmfield{local}{$\mathfrak{T}$($\emptyset$)}{\textit{Type}}\\
\smallmfield{refl}{$\mathfrak{T}$($\emptyset$)}{\textit{Type}}\\
\smalltfield{$\mathfrak{c}$}{\smallrecord{\smalltfield{id$_0$}{\smallrecord{\smalltfield{x}{\textit{Ind}}\\
                                                                            \smalltfield{e}{named(x,
                                                                              ``Sam'')}}}}}}
                                                                  . \\
\hspace*{3em}                                                                   
\record{\tfield{e}{like($c$.$\mathfrak{c}$.id$_0$.x, $\lambda
    P$:\textit{Ppty} . 
$P$\{$c$.$\mathfrak{c}$.id$_0$.x\})}} 
 
\end{subex} 
   
\end{ex} 
    
[This material should possibly be moved to Ch.~\ref{ch:propnames} so we have
structured contexts from the beginning.]  


\section{Long distance dependencies}\todo{Moved from
  Ch~\ref{ch:quant}.  Is this the right place to put this?}
\label{sec:long-distance}

Let us consider how to derive the content of \textit{who Sam hugged}
as a relative clause.  We use \textbf{hug}$_V$ to represent the
foreground of the content of \textit{hug} as a transitive verb,
characterized as in \nexteg{}.
\begin{ex} 
\textbf{hug}$_V$ = $\lambda\mathfrak{s}$:\textit{Rec} . $\lambda
r_2$:\smallrecord{\smalltfield{x}{\textit{Quant}}} . $\lambda
r_1$:\smallrecord{\smalltfield{x}{\textit{Ind}}}
. \smallrecord{\smalltfield{e}{hug($r_1$.x, $r_2$.x)}} 
\end{ex} 
Our theory of syntax in terms of event types means that what have been
called ``gaps'' or ``traces'' in other theories of syntax would
somehow have to correspond to non-events in which nothing happens and
do not have any temporal extent.  This strongly suggests that they do
not exist.  Instead we will take a strategy similar to that pursued,
for example, in combinatory categorial grammar (see, for example,
\citealp{Steedman2012}) where \textit{hug} can also be interpreted as
a verb phrase whose content is ``looking for'' a \textit{wh}-phrase
content.  We will represent the foreground of this content as
\textbf{hug}$_{\mathit{VP}}$ and characterize it in terms of
\textbf{hug}$_V$ as in \nexteg{a} which is identical with \nexteg{b}.
\begin{ex} 
\begin{subex}
\item \textbf{hug}$_{\mathit{VP}}$ = $\lambda\mathfrak{s}$:\smallrecord{\smalltfield{wh$_0$}{\textit{Ind}}}
. \textbf{hug}$_V$($\mathfrak{s}$)($\lambda P$:\textit{Ppty}
. $P$(\smallrecord{\field{x}{$\mathfrak{s}$.wh$_0$}})) 

\item
  $\lambda\mathfrak{s}$:\smallrecord{\smalltfield{wh$_0$}{\textit{Ind}}}
  . $\lambda r_1$:\smallrecord{\smalltfield{x}{\textit{Ind}}}
  . \smallrecord{\smalltfield{e}{hug($r_1$.x, $\lambda P$:\textit{Ppty}
. $P$(\smallrecord{\field{x}{$\mathfrak{s}$.wh$_0$}}))}}
\end{subex}
\end{ex}

Let us now consider the content of \textit{Sam hugged}.  In order to
avoid the additional complexity of combining different kinds of
context which we will undertake in Section~\ref{sec:struc-cntxt} we
will go back to the treatment of proper names in Chapter~\ref{ch:gram}
rather than the treatment in Chapter~\ref{ch:propnames} where proper
names introduce a constraint on the context. Thus we will consider the
foreground of the parametric content of \textit{Sam}, which we will
represent as \textbf{Sam}, to be \nexteg{}
where `sam' is a particular individual named Sam.
\begin{ex} 
\textbf{Sam} = $\lambda\mathfrak{s}$:\textit{Rec} . $\lambda P$:\textit{Ppty} . $P$(\smallrecord{\field{x}{sam}}) 
\end{ex}
The foreground of the content of \textit{Sam hugged}, which we will
represent as \textbf{Sam$^\frown$hugged}, is \nexteg{a} which is
identical with \nexteg{b}.
\begin{ex} 
\begin{subex} 
 
\item \textbf{Sam$^\frown$hugged} =
  $\lambda\mathfrak{s}$:\smallrecord{\smalltfield{wh$_0$}{\textit{Ind}}}
  . \textbf{Sam}($\mathfrak{s}$)(\textbf{hug}$_{\mathit{VP}}$($\mathfrak{s}$)) 
 
\item
  $\lambda\mathfrak{s}$:\smallrecord{\smalltfield{wh$_0$}{\textit{Ind}}}
  . \smallrecord{\smalltfield{e}{hug(sam, $\lambda P$:\textit{Ppty} . $P$(\smallrecord{\field{x}{$\mathfrak{s}$.wh$_0$}}))}}
 
\end{subex} 
   
\end{ex}
The foreground of the content of \textit{who}, which we will represent
as \textbf{who}, is given in \nexteg{}.
\begin{ex} 
\textbf{who} =
$\lambda\mathfrak{s}$:\smallrecord{\smalltfield{wh}{\textit{Ind}}}
. $\lambda P$:\textit{Ppty} . $P$(\smallrecord{\field{x}{$\mathfrak{s}$.wh}}) 
\end{ex} 
In order to be able to characterize the kind of binding involved in
combining \textit{who} with \textit{Sam hugged} we will define an
operation, $\oplus$, which adds a field to a record or, if the label
of the field being added
is already present in the record, will replace that field with the new
field.  This operation is characterized in \nexteg{}.
\begin{ex} 
If $r$ is a record, $\ell$ is a label and $v$ is some TTR object, then
\\
\hspace*{1em} $r\oplus[\ell=v]$ is \\
\hspace*{2em} $r\cup\{\langle\ell,v\rangle\}$, if there is no $v'$
such that $\langle\ell,v'\rangle\in r$\\
\hspace*{2em}
$(r-\{\langle\ell,r.\ell\rangle\})\cup\{\langle\ell,v\rangle\}$, otherwise 
\end{ex} 
We are using records here to do the work of partial variable
assignments in logic and the $\oplus$-operation carries out the kind
of modification which is performed on variable assignments when
defining binding in the logical treatment.  This now gives us what we
need to characterize the foreground of the content of the relative
clause \textit{who Sam hugged}, which we will represent as
\textbf{who}$^\frown$\textbf{Sam}$^\frown$\textbf{hugged}, as in
\nexteg{a} which is identical with \nexteg{b}.  \nexteg{b} is in turn
equivalent to \nexteg{c} because of the constraint on the extensional
predicate `hug' similar to that for `find' given in  (\ref{ex:mp-find})
on p.~\pageref{ex:mp-find}. 
\begin{ex} 
\begin{subex} 
 
\item \textbf{who}$^\frown$\textbf{Sam}$^\frown$\textbf{hugged} = \\
  \hspace*{1em}$\lambda\mathfrak{s}$:\textit{Rec} . \\
\hspace*{2em}$\lambda
  r_1$:\smallrecord{\smalltfield{x}{\textit{Ind}}}
  . \\
\hspace*{3em}\textbf{who}($\mathfrak{s}\oplus[\text{wh}=r_1.\text{x}]$)
($\lambda
  r_2$:\smallrecord{\smalltfield{x}{\textit{Ind}}} . \textbf{Sam}$^\frown$\textbf{hugged}($\mathfrak{s}\oplus[\text{wh}_0=r_2.\text{x}]$)) 
 
\item $\lambda\mathfrak{s}$:\textit{Rec} . $\lambda
  r_1$:\smallrecord{\smalltfield{x}{\textit{Ind}}}
  . \smallrecord{\smalltfield{e}{hug(sam, $\lambda P$:\textit{Ppty}
      . $P$(\smallrecord{\field{x}{$r_1$.x}}))}}

\item $\lambda\mathfrak{s}$:\textit{Rec} . $\lambda
  r_1$:\smallrecord{\smalltfield{x}{\textit{Ind}}}
  . \smallrecord{\smalltfield{e}{hug$^\dagger$(sam, $r_1$.x))}} 
 
\end{subex} 
   
\end{ex} 
To form the content of a noun modified by a relative clause such as
\textit{child who Sam hugged} we use a record type which requires of
an individual that it has both the property of being a child and being
hugged by Sam as given in \nexteg{}.
\begin{ex} 
$\lambda\mathfrak{s}$:\textit{Rec} $\lambda
r$:\smallrecord{\smalltfield{x}{\textit{Ind}}}
. \record{\tfield{e$_1$}{child$'$\{$r$.x\}}\\
          \tfield{e$_2$}{\textbf{who$^\frown$Sam$^\frown$hugged}($\mathfrak{s}$)\{$r$.x\}}}  
\end{ex} 
[Has the $P\{x\}$ notation already been introduced?]  



\section{Summary}

In this chapter we have looked at the analysis of proper names.  We
started by showing how Montague's analysis of proper names could be
recast in TTR and we showed that there was an advantage in the
sign-based approach that we have adopted in accounting for the fact
that different individuals can have the same name.  Montague's
original analysis did not say anything about the presupposition-like
nature of proper names in that they seem to require interlocutors to
be able to identify appropriate referents for the use of a proper name
from among a number of potential referents which might be available.
We showed how this could be treated by introducing parametric
contents for proper names and we showed how accommodation phenomena
could be accounted for including a simple-minded analysis of salience
analyzed in terms of the information states of agents.  Finally, we
discussed Kripke's puzzle concerning Paderewski and its possible
relation to a theory of microlanguages as discussed recently by
Ludlow.  While in general we find the idea of microlanguages appealing
we suggested that it plays a role in the analysis of proper names in a
rather different way to that suggested by Ludlow.  
  
  


%%% Local Variables: 
%%% mode: latex
%%% TeX-master: "ttl"
%%% End: 
