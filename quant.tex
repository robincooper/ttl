\chapter{Quantification, anaphora and underspecification}
\label{ch:quant}
\setcounter{equation}{0}

\section{The interpretation of unbound pronouns}
We will consider how to recreate a simple interpretation of pronouns
ranging over individuals, first treating them in a similar way to free
variables in logic and then showing how they can be bound by
quantifiers.  The central idea is to use records as \textit{pronominal
  contexts} which correspond to partial assignments to variables in
standard logical treatments.  Consider first a simple sentence with a deictic
pronoun as in \nexteg{}.
\begin{ex} 
he left 
\end{ex} 
In our initial pass we will ignore matters
of gender to make things simpler.  The content of \preveg{} is a type
which depends on a context (a situation) which provides a value of the
pronoun \textit{he}. Thus it will have a parametric content which is a
function from a context assigning a value to the pronoun to a
type.\footnote{We will consider later how to combine types of context
  which assign values to pronouns with other context types which we
  have introduced with parametric contents.}  We will use the variable
`$\mathfrak{s}$' to represent such contexts.  Thus the foreground of a parametric content for
\textit{he} could be that given in \nexteg{a}, for \textit{left}
(ignoring tense) that given in \nexteg{b} and their combination, using a variant
of S-combination which we will discuss as we progress, is represented
in \nexteg{c}.
\begin{ex} 
\begin{subex} 
 
\item $\lambda\mathfrak{s}$:\smallrecord{\smalltfield{x$_0$}{\textit{Ind}}} . 
        $\lambda P$:\textit{Ppty} . $P$(\smallrecord{\field{x}{$\mathfrak{s}$.x$_0$}})
 
\item $\lambda\mathfrak{s}$:\textit{Rec} . 
        $\lambda r$:\smallrecord{\smalltfield{x}{\textit{Ind}}} . 
              \record{\tfield{e}{leave($r$.x)}}

\item $\lambda\mathfrak{s}$:\smallrecord{\smalltfield{x$_0$}{\textit{Ind}}}
  . 
         \record{\tfield{e}{leave($\mathfrak{s}$.x$_0$)}}
 
\end{subex} 
   
\end{ex}
We follow Montague's strategy in allowing the content of \textit{he}
to be \nexteg{} for any natural number $i$.
\begin{ex}
$\lambda\mathfrak{s}$:\smallrecord{\smalltfield{x$_i$}{\textit{Ind}}} . 
        $\lambda P$:\textit{Ppty}
        . $P$(\smallrecord{\field{x}{$\mathfrak{s}$.x$_i$}})
\label{ex:pronoun-cont}
\end{ex} 
Thus considering the boy and the dog, a
content for \nexteg{a} will be \nexteg{b}.
\begin{ex} 
\begin{subex} 
 
\item he hugged it 
 
\item $\lambda\mathfrak{s}$:\smallrecord{\smalltfield{x$_0$}{\textit{Ind}}\\
                                   \smalltfield{x$_1$}{\textit{Ind}}}
                                 . 
\record{\tfield{e}{hug($\mathfrak{s}$.x$_0$, $\mathfrak{s}$.x$_1$)}} 
 
\end{subex} 
   
\end{ex} 
An advantage of using record types to characterize pronominal contexts
rather than variable assignments is that we can add further
information represented by the pronoun such as gender.   Thus a simple
treatment of gender for \preveg{} might be given by making the content
be \nexteg{}.
\begin{ex} 
$\lambda\mathfrak{s}$:\smallrecord{\smalltfield{x$_0$}{\textit{Ind}}\\
                             \smalltfield{c$_0$}{male(x$_0$)}\\
                             \smalltfield{x$_1$}{\textit{Ind}}\\
                             \smalltfield{c$_1$}{neuter(x$_1$)}}
                                 . 
\record{\tfield{e}{hug($\mathfrak{s}$.x$_0$, $\mathfrak{s}$.x$_1$)}} 
\end{ex} 
As we will see below, there are some complications with this simple
idea when it comes to the interpretation of pronouns which are bound
by quantifiers.  Even for deictic pronouns there are problems
determining which predicates should be used in a semantic treatment of
gender.  Even for a language like English which apparently has
semantic gender (as opposed to grammatical gender like German or
French), neuter can be used for objects which do not have gender (like
tables) and for animals other than humans which do have gender and
which can be referred to with masculine and feminine pronouns.

\section{Quantifier scope and underspecification}

Given the kind of interpretation rules we have so far we can obtain a
reading for \nexteg{a} which corresponds to the kind of parametric
content foreground using pronominal contexts
in \nexteg{b}, using the abbreviations `boy$'$' and `dog$'$' as
introduced in Chapter~\ref{ch:commonnouns}.
\begin{ex} 
\begin{subex} 
 
\item a boy hugged every dog 
 
\item $\lambda \mathfrak{s}$:\textit{Rec} . \smallrecord{\smalltfield{e}{exist(boy$'$, $\lambda
      r_1$:\smallrecord{\smalltfield{x}{\textit{Ind}}} . 
\smallrecord{\smalltfield{e}{every(dog$'$, 
$\lambda r_2$: \smallrecord{\smalltfield{x}{\textit{Ind}}} . 
\smallrecord{\smalltfield{e}{hug($r_1$.x, $r_2$.x)}})}})}}
 
\end{subex} 
   
\end{ex} 
This, of course, represents the reading where there is a boy such that
he hugs every dog.  Notice that here we have vacuous abstraction over
pronominal contexts and the constraint on them is that they are a record of
some kind without requiring any particular fields.    

In order to obtain the reading where \textit{every dog} has wide
scope, we follow Montague and pretty much everybody else in basing our
treatment of quantifier scope on the treatment of free pronouns,
though without the contribution of any gender information.  Let us
imagine just for a moment that such a pronoun existed in English and
is written as \textit{it$^*$} with the kind of pronoun interpretations
given in (\ref{ex:pronoun-cont}).  Then the content for \nexteg{a} could be
\nexteg{b}.
\begin{ex} 
\begin{subex} 
 
\item a boy hugged it$^*$
 
\item $\lambda \mathfrak{s}$:\smallrecord{\smalltfield{x$_0$}{\textit{Ind}}} . 
\smallrecord{\smalltfield{e}{exist(boy$'$, $\lambda
    r$:\smallrecord{\smalltfield{x}{\textit{Ind}}}
    . \smallrecord{\smalltfield{e}{hug($r$.x, $\mathfrak{s}$.x$_0$)}})}} 
 
\end{subex} 
   
\end{ex}
Let us further imagine, contrary to fact, that English represented the fact
that a noun phrase has wide scope over a sentence by placing it at the
beginning of a sentence as in \nexteg{a} and giving it an
interpretation where the interpretation of \textit{it$^*$} gets bound
as in \nexteg{b}.
\begin{ex} 
\begin{subex} 
 
\item every dog, a boy hugged it$^*$ 
 
\item $\lambda\mathfrak{s}$:\textit{Rec} . 
\smallrecord{\smalltfield{e}{every(dog$'$,
$\lambda r_1$:\smallrecord{\smalltfield{x$_0$}{\textit{Ind}}} . 
\smallrecord{\smalltfield{e}{exist(boy$'$, $\lambda
    r_2$:\smallrecord{\smalltfield{x}{\textit{Ind}}}
    . \smallrecord{\smalltfield{e}{hug($r_2$.x, $r_1$.x$_0$)}})}})}} 
 
\end{subex} 
\label{ex:edabhi}   
\end{ex} 
The imaginary English expression \preveg{a} corresponds quite closely
to the kind of representation for wide scope readings that are used in
various theories of logical form.  A major difference is that in
logical form there is an index corresponding to the label `x$_0$' that
we use in the interpretation which shows that \textit{every dog} binds
\textit{it$^*$}.  This might be represented something like in
\nexteg{}.
\begin{ex} 
every dog$_{x_0}$, a boy hugged it$^*_{x_0}$ 
\end{ex} 
The imaginary sentence (\ref{ex:edabhi}a) also corresponds closely to
Montague's (\citeyear{Montague1973}) treatment of scope phenomena.
Montague would also index the pronoun and use a quantification rule
with the same index which would replace the pronoun with the
noun-phrase being quantified in.

Neither of these options are open to us since our syntax is defined in
terms of types of utterance situations and signs which relate
utterance situations to contents.  Our realistic strategy does not
allow for the use of additional imaginary utterance structures.  For this reason
we will adapt the kind of storage technique used in \cite{Cooper1983}.
In the original version of storage we moved from assigning a
single content to a syntactic structure to assigning a set of contents
in order to allow for the ambiguous interpretation of a single
syntactic structure.  In our sign-based approach using types the
corresponding move is not such a major change and the result yields a
theory involving underspecified content rather than a set of contents.

To see this consider the type \textit{Sign} introduced in
Chapter~\ref{ch:gram}.  Any object of type \textit{Sign} will be of
the type in \nexteg{}.
\begin{ex} 
\record{\tfield{s-event}{\textit{SEvent}} \\
         \tfield{syn}{\textit{Syn}} \\
        \tfield{cont}{\textit{Cont}}}  
\end{ex} 
The type in \preveg{} is completely underspecified.  Any sign will be
of this type.  We could specify it with respect to content by making
the `cont'-field be a manifest field as in \nexteg{b}, where $c$ is as
given in \nexteg{a}.
\begin{ex} 
\begin{subex} 
 
\item $c$ = \smallrecord{\smalltfield{e}{exist(boy$'$, $\lambda
      r_1$:\smallrecord{\smalltfield{x}{\textit{Ind}}}
        . \smallrecord{\smalltfield{e}{every(dog$'$, $\lambda
            r_2$:\smallrecord{\smalltfield{x}{\textit{Ind}}}
              . \smallrecord{\smalltfield{e}{hug($r_1$.x, $r_2$.x)}})}})}}
 
\item  \record{\tfield{s-event}{\textit{SEvent}} \\
         \tfield{syn}{\textit{Syn}} \\
        \mfield{cont}{$c$}{\textit{Cont}}}  
 
\end{subex} 
   
\end{ex} 
Now recall that the manifest field
\smallrecord{\smallmfield{cont}{$c$}{\textit{Cont}}} is just a
convenient way of writing
\smallrecord{\smalltfield{cont}{\textit{Cont}$_c$}} where
\textit{Cont}$_c$ is a singleton type whose only witness is $c$.  It
is in this sense that the content has been specified to be $c$.
Suppose now that we do not have enough information to fully specify
the content, that is, tie it down to be one particular content, but we
know that it has to be one of either $c$ or $c'$ (as characterized in
\nexteg{a}).  This could be represented by using a join type of two
singleton types, \textit{Cont}$_c$$\vee$\textit{Cont}$_{c'}$, as in
\nexteg{b}.
\begin{ex} 
\begin{subex} 
 
\item $c'$ =  \smallrecord{\smalltfield{e}{every(dog$'$, $\lambda
      r_1$:\smallrecord{\smalltfield{x}{\textit{Ind}}}
        . \smallrecord{\smalltfield{e}{exist(boy$'$, $\lambda
            r_2$:\smallrecord{\smalltfield{x}{\textit{Ind}}}
              . \smallrecord{\smalltfield{e}{hug($r_2$.x, $r_1$.x)}})}})}}
 
\item  \record{\tfield{s-event}{\textit{SEvent}} \\
         \tfield{syn}{\textit{Syn}} \\
        \tfield{cont}{\textit{Cont}$_c$$\vee$\textit{Cont}$_{c'}$}}  
 
\end{subex} 
   
\end{ex} 
\preveg{} is the type of signs whose contents are either $c$ or $c'$.
This is, then, a single type, which corresponds to an ``underspecified
content''.  Of course, the set of witnesses of the join type,
$\{c,c'\}$, corresponds to the set of contents that could be generated
by a storage algorithm.  Our strategy, then, is to devise a way of
computing such types on the basis of compositional interpretation.     
Notice that saying that the content is of type
\textit{Cont}$_c$$\vee$\textit{Cont}$_{c'}$, that is, that it is
either $c$ or $c'$ is distinct from saying that the content is the
disjunction of $c$ and $c'$, that is, if $c$ and $c'$ are types, the
type $c\vee c'$.  The witnesses for $c\vee c'$ will be situations $s$
such that either $s:c$ or $s:c'$ whereas the witness for
\textit{Cont}$_c$$\vee$\textit{Cont}$_{c'}$ will be the contents $c$
and $c'$.  This will make an important different in compositional
interpretation.  For example, if a sentence $S$ has a content of type     
\textit{Cont}$_c$$\vee$\textit{Cont}$_{c'}$ we would want to say that
the sentence \textit{Sam believes $S$} has a content of type
\textit{Cont}$_{believe(sam,c)}$$\vee$\textit{Cont}$_{believe(sam,c')}$
rather than that it has the content `believe(sam,$c\vee c'$)'.

In order to make the relationship between sets of contents and join
types smoother we will use generalized join types
(Appendix~\ref{app:jointypes}).  Suppose we have computed a set of
possible contents, $\{c_1,\ldots,c_n\}$, for a phrase, then the type
of the content for that phrase is
$\bigvee\{\textit{Cont}_{c_1},\ldots,\textit{Cont}_{c_n}\}$.  If
$\mathbb{C}$ represents the set of contents we have computed, we can
use the more convenient notation
$\displaystyle{\bigvee_{c\in\mathbb{C}}}\textit{Cont}_c$ to mean
$\bigvee\{\textit{Cont}_c\mid c\in\mathbb{C}\}$.

We will take content now to be divided into two fields labelled
`unplugged' and `plugged'.  This terminology goes back to
\cite{Bos1996}.  The `unplugged'-field contains a set of unplugged
interpretations consisting of a set of parametric quantifiers
(possibly empty) and a core interpretation consisting of a parametric
content.  Thus we define a type \textit{UInterp} of unplugged
interpretations as in \nexteg{}.
\begin{ex} 
\textit{UInterp} = \record{\tfield{quants}{\{\textit{PQuant}\}}\\
                           \tfield{core}{\textit{PCont}}}
\end{ex} 
The `cont'-field in a sign will now be required to be of the type in
\nexteg{}.
\begin{ex} 
\record{\tfield{unplugged}{\{\textit{UInterp}\}}\\
        \tfield{plugged}{$\displaystyle{\bigvee_{c\in\mathrm{unplugged},\
              c.\mathrm{quants}=\emptyset}\textit{Cont}_{c.\mathrm{core}}}$}}
\end{ex} 
That is, the `unplugged'-field in the content will contain a set of
unplugged interpretations (consisting of a quantifier store, possibly empty, and a core
interpretation) and the `plugged'-field will contain one of the cores
of those unplugged interpretations whose quantifier store is
empty.

Let us consider what this will look like for the example \textit{every
  boy wants a dog} using \textit{every}$'$, \textit{boy}$'$ etc. as
abbreviations for non-parametric contents for these words.  We show
the `cont'-field of the sign type associated with constituents of this
sentence.
\begin{ex} 
\begin{subex} 
 
\item a dog 
 
\item
  \smallrecord{\smallmfield{unplugged}{\{\smallrecord{\field{quants}{$\emptyset$}\\
                                            \field{core}{\smallrecord{\field{bg}{\textit{Rec}}\\
                                                                      \field{fg}{$\lambda\mathfrak{s}$:\textit{Rec}
                                                                        . \textit{a}$'$(\textit{dog}$'$)}}}}, \\
   \hspace*{5em}                            \smallrecord{\field{quants}{\{\smallrecord{\field{bg}{\smallrecord{\smalltfield{x$_0$}{\textit{Ind}}}} \\
                                                                          \field{fg}{$\lambda\mathfrak{s}$:\smallrecord{\smalltfield{x$_0$}{\textit{Ind}}}
                                                                            . \textit{a}$'$(\textit{dog}$'$)}}\}} \\
                                            \field{core}{\smallrecord{\field{bg}{\smallrecord{\smalltfield{x$_0$}{\textit{Ind}}}}
                                                \\
                                                                      \field{fg}{$\lambda\mathfrak{s}$:\smallrecord{\smalltfield{x$_0$}{\textit{Ind}}}
                                                                        . $\lambda
                                                                        P$:\textit{Ppty}
                                                                        . $P(\mathfrak{s}.\textrm{x}_0)$}}}}\}} {\{\textit{UInterp}\}}
                                                            \\
          \smalltfield{plugged}{\textit{Cont}$_{\text{\smallrecord{\field{bg}{\textit{Rec}}\\
                                                                      \field{fg}{$\lambda\mathfrak{s}$:\textit{Rec}
                                                                        . \textit{a}$'$(\textit{dog}$'$)}}}}$}}
 
\end{subex} 
   
\end{ex} 
\begin{ex} 
\begin{subex} 
 
\item want a dog 
 
\item
  \smallrecord{\smallmfield{unplugged}{\{\smallrecord{\field{quants}{$\emptyset$}\\
                                            \field{core}{\smallrecord{\field{bg}{\textit{Rec}}\\
                                                                      \field{fg}{$\lambda\mathfrak{s}$:\textit{Rec}
                                                                        . \textit{want}$'$(\textit{a}$'$(\textit{dog}$'$))}}}}, \\
   \hspace*{5em}                            \smallrecord{\field{quants}{\{\smallrecord{\field{bg}{\smallrecord{\smalltfield{x$_0$}{\textit{Ind}}}} \\
                                                                          \field{fg}{$\lambda\mathfrak{s}$:\smallrecord{\smalltfield{x$_0$}{\textit{Ind}}}
                                                                            . \textit{a}$'$(\textit{dog}$'$)}}\}} \\
                                            \field{core}{\smallrecord{\field{bg}{\smallrecord{\smalltfield{x$_0$}{\textit{Ind}}}}
                                                \\
                                                                      \field{fg}{$\lambda\mathfrak{s}$:\smallrecord{\smalltfield{x$_0$}{\textit{Ind}}}
                                                                        . \textit{want}$'$($\lambda
                                                                        P$:\textit{Ppty}
                                                                        . $P(\mathfrak{s}.\textrm{x}_0)$)}}}}\}} {\{\textit{UInterp}\}}
                                                            \\
          \smalltfield{plugged}{\textit{Cont}$_{\text{\smallrecord{\field{bg}{\textit{Rec}}\\
                                                                      \field{fg}{$\lambda\mathfrak{s}$:\textit{Rec}
                                                                        . \textit{want}(\textit{a}$'$(\textit{dog}$'$))}}}}$}}
 
\end{subex} 
   
\end{ex} 

\begin{ex} 
\begin{subex} 
 
\item every boy 
 
\item
  \smallrecord{\smallmfield{unplugged}{\{\smallrecord{\field{quants}{$\emptyset$}\\
                                            \field{core}{\smallrecord{\field{bg}{\textit{Rec}}\\
                                                                      \field{fg}{$\lambda\mathfrak{s}$:\textit{Rec}
                                                                        . \textit{every}$'$(\textit{boy}$'$)}}}}, \\
   \hspace*{5em}                            \smallrecord{\field{quants}{\{\smallrecord{\field{bg}{\smallrecord{\smalltfield{x$_0$}{\textit{Ind}}}} \\
                                                                          \field{fg}{$\lambda\mathfrak{s}$:\smallrecord{\smalltfield{x$_0$}{\textit{Ind}}}
                                                                            . \textit{every}$'$(\textit{boy}$'$)}}\}} \\
                                            \field{core}{\smallrecord{\field{bg}{\smallrecord{\smalltfield{x$_0$}{\textit{Ind}}}}
                                                \\
                                                                      \field{fg}{$\lambda\mathfrak{s}$:\smallrecord{\smalltfield{x$_0$}{\textit{Ind}}}
                                                                        . $\lambda
                                                                        P$:\textit{Ppty}
                                                                        . $P(\mathfrak{s}.\textrm{x}_0)$}}}}\}} {\{\textit{UInterp}\}}
                                                            \\
          \smalltfield{plugged}{\textit{Cont}$_{\text{\smallrecord{\field{bg}{\textit{Rec}}\\
                                                                      \field{fg}{$\lambda\mathfrak{s}$:\textit{Rec}
                                                                        . \textit{every}$'$(\textit{boy}$'$)}}}}$}}
 
\end{subex} 
   
\end{ex} 
    
\begin{ex} 
\begin{subex} 
 
\item every boy wants a dog 
 
\item
% \newsavebox{\test}
% \sbox\test{\begin{tabular}{l}a,\\b\end{tabular}}
% \usebox{\test}

% \newsavebox{\unplugged}
% \sbox\unplugged{$\left\{ \text{\begin{tabular}{l}\usebox{\unpluggeda},\\
%                                     \usebox{\unpluggedb},\\
%                                     \usebox{\unpluggedc},\\
%                                     \usebox{\unpluggedd},\\
%                                     \usebox{\unpluggede},\\
%                                     \usebox{\unpluggedf},\\
%                                     \usebox{\unpluggedg}
%                   \end{tabular}}\right\}$}

\newsavebox{\plugged}
\sbox\plugged{$\displaystyle{\bigvee_{c\in\mathbb{C}_{\mathrm{plugged}}}\textit{Cont}_c}$}
                              

\smallrecord{\smallmfield{unplugged}{$\mathbb{C}_{\mathrm{unplugged}}$}{\{\textit{UInterp}\}}\\
             \smalltfield{plugged}{\usebox{\plugged}}}

 
\end{subex} 
   
\end{ex} 
where $\mathbb{C}_{\mathrm{unplugged}}$ is the set whose members are
listed in \nexteg{} and $\mathbb{C}_{\mathrm{plugged}}$ is the set
whose members are listed in (\ref{ex:plugged}).
\newsavebox{\unpluggeda}
\sbox\unpluggeda{\smallrecord{\field{quants}{$\emptyset$}\\
                                                       \field{core}{\smallrecord{\field{bg}{\textit{Rec}}\\
                                                                                 \field{fg}{$\lambda\mathfrak{s}$:\textit{Rec}
                                                                         . \textit{every}$'$(\textit{boy}$'$)(\textit{want}$'$(\textit{a}$'$(\textit{dog}$'$)))}}}}}

\newsavebox{\unpluggedb}
\sbox\unpluggedb{\smallrecord{\field{quants}{\{\smallrecord{\field{bg}{\smallrecord{\smalltfield{x$_0$}{\textit{Ind}}}} \\
                                                                           \field{fg}{$\lambda\mathfrak{s}$:\smallrecord{\smalltfield{x$_0$}{\textit{Ind}}}
                                                                            . \textit{every}$'$(\textit{boy}$'$)}}\}} \\
                                            \field{core}{\smallrecord{\field{bg}{\smallrecord{\smalltfield{x$_0$}{\textit{Ind}}}}
                                                \\
                                                                      \field{fg}{\begin{tabular}{l}$\lambda\mathfrak{s}$:\smallrecord{\smalltfield{x$_0$}{\textit{Ind}}}
                                                                        . \\
                                                                                   \hspace*{1em}$\lambda
                                                                        P$:\textit{Ppty}
                                                                        . \\
                                                                                   \hspace*{2em}$P(\mathfrak{s}.\textrm{x}_0)$(\textit{want}$'$(\textit{a}$'$(\textit{dog}$'$)))\end{tabular}}}}}}

\newsavebox{\unpluggedc}
\sbox\unpluggedc{\smallrecord{\field{quants}{\{\begin{tabular}{l}\smallrecord{\field{bg}{\smallrecord{\smalltfield{x$_0$}{\textit{Ind}}}} \\
                                                                          \field{fg}{$\lambda\mathfrak{s}$:\smallrecord{\smalltfield{x$_0$}{\textit{Ind}}}
                                                                            . \textit{every}$'$(\textit{boy}$'$)}}, \\
                                                                        \smallrecord{\field{bg}{\smallrecord{\smalltfield{x$_1$}{\textit{Ind}}}} \\
                                                                          \field{fg}{$\lambda\mathfrak{s}$:\smallrecord{\smalltfield{x$_1$}{\textit{Ind}}}
                                                                            . \textit{a}$'$(\textit{dog}$'$)}}\end{tabular}\}} \\
                                            \field{core}{\smallrecord{\field{bg}{\smallrecord{\smalltfield{x$_0$}{\textit{Ind}}\\
                                                                                              \smalltfield{x$_1$}{\textit{Ind}}}}
                                                \\
                                                                      \field{fg}{\begin{tabular}{l}$\lambda\mathfrak{s}$:\smallrecord{\smalltfield{x$_0$}{\textit{Ind}} \\
                                                                                                                \smalltfield{x$_1$}{\textit{Ind}}}
                                                                        . \\
                                                                                                       \hspace*{1em}$\lambda
                                                                        P$:\textit{Ppty}
                                                                        . \\
                                                                                   \hspace*{2em}$P(\mathfrak{s}.\textrm{x}_0)$(\textit{want}$'$($\lambda
                                                                                   P$:\textit{Ppty}
                                                                                   .
                                                                                   $P(\mathfrak{s}.\textrm{x}_1)$))\end{tabular}}}}}}

\newsavebox{\unpluggedd}
\sbox\unpluggedd{\smallrecord{\field{quants}{\{\smallrecord{\field{bg}{\smallrecord{\smalltfield{x$_1$}{\textit{Ind}}}} \\
                                                                          \field{fg}{$\lambda\mathfrak{s}$:\smallrecord{\smalltfield{x$_1$}{\textit{Ind}}}
                                                                            . \textit{a}$'$(\textit{dog}$'$)}}\}} \\
                                            \field{core}{\smallrecord{\field{bg}{\smallrecord{\smalltfield{x$_1$}{\textit{Ind}}}}
                                                \\
                                                                      \field{fg}{\begin{tabular}{l}$\lambda\mathfrak{s}$:\smallrecord{\smalltfield{x$_1$}{\textit{Ind}}}
                                                                        . \\
                                                                                   \hspace*{1em}\textit{every}$'$(\textit{boy}$'$) \\ 
\hspace*{2em}($\lambda
   r$:\smallrecord{\smalltfield{x$_0$}{\textit{Ind}}} . \\
   \hspace*{3em}$\lambda P$:\textit{Ppty} .  $P$($r$.x$_0$)\\
                                                                                   \hspace*{4em}(\textit{want}$'$\\ \hspace*{5em}($\lambda
                                                                        P$:\textit{Ppty}
                                                                        .
                                                                                
$P(\mathfrak{s}.\textrm{x}_1)$)))\end{tabular}}}}}}

\newsavebox{\unpluggede}
\sbox\unpluggede{\smallrecord{\field{quants}{\{\smallrecord{\field{bg}{\smallrecord{\smalltfield{x$_0$}{\textit{Ind}}}} \\
                                                                          \field{fg}{$\lambda\mathfrak{s}$:\smallrecord{\smalltfield{x$_0$}{\textit{Ind}}}
                                                                            . \textit{every}$'$(\textit{boy}$'$)}}\}} \\
                                            \field{core}{\smallrecord{\field{bg}{\smallrecord{\smalltfield{x$_0$}{\textit{Ind}}}}
                                                \\
                                                                      \field{fg}{\begin{tabular}{l}$\lambda\mathfrak{s}$:\smallrecord{\smalltfield{x$_0$}{\textit{Ind}}}
                                                                        . \\
                                                                                   \hspace*{1em}\textit{a}$'$(\textit{dog}$'$)\\
                                                                                   \hspace*{2em}($\lambda
                                                                        r$:\smallrecord{\smalltfield{x$_1$}{\textit{Ind}}}
                                                                          . \\
                                                                                   \hspace*{3em}$\lambda
                                                                          P$:\textit{Ppty}
                                                                          . \\
                                                                                   \hspace*{4em}$P$($r$.x$_1$)\\
                                                                                   \hspace*{5em}(\textit{want}$'$($\lambda
                                                                        P$:\textit{Ppty}
                                                                        . $P(\mathfrak{s}.\textrm{x}_0)$))})\end{tabular}}}}}

\newsavebox{\unpluggedf}
\sbox\unpluggedf{\smallrecord{\field{quants}{\{\}}\\
                              \field{core}{\smallrecord{\field{bg}{\textit{Rec}}\\
                                                        \field{fg}{\begin{tabular}{l}
                                                                    $\lambda\mathfrak{s}$:\textit{Rec}
                                                                     \\
                                                                    \hspace*{1em}\textit{a}$'$(\textit{dog}$'$)
                                                                      \\
                                                                    \hspace*{2em}($\lambda
                                                                     r_1$:\smallrecord{\smalltfield{x$_1$}{\textit{Ind}}}
                                                                     . \\
                                                                    \hspace*{3em}\textit{every}$'$(\textit{boy}$'$)\\
                                                                    \hspace*{4em}($\lambda
                                                                     r_2$:\smallrecord{\smalltfield{x$_0$}{\textit{Ind}}}
                                                                     . \\
                                                                    \hspace*{5em}$\lambda
                                                                     P$:\textit{Ppty}
                                                                     . $P$($r_1$.x$_0$)\\
                                                                    \hspace*{6em}\textit{wants}$'$($\lambda
                                                                     P$:\textit{Ppty}
                                                                     . $P$($r_1$.x$_1$))))\end{tabular}}}}}}

\newsavebox{\unpluggedg}
\sbox\unpluggedg{\smallrecord{\field{quants}{\{\}}\\
                              \field{core}{\smallrecord{\field{bg}{\textit{Rec}}\\
                                                        \field{fg}{\begin{tabular}{l}
                                                                    $\lambda\mathfrak{s}$:\textit{Rec}
                                                                     \\
                                                                    \hspace*{1em}\textit{every}$'$(\textit{boy}$'$)
                                                                      \\
                                                                    \hspace*{2em}($\lambda
                                                                     r_1$:\smallrecord{\smalltfield{x$_0$}{\textit{Ind}}}
                                                                     . \\
                                                                    \hspace*{3em}\textit{a}$'$(\textit{dog}$'$)\\
                                                                    \hspace*{4em}($\lambda
                                                                     r_2$:\smallrecord{\smalltfield{x$_1$}{\textit{Ind}}}
                                                                     . \\
                                                                    \hspace*{5em}$\lambda
                                                                     P$:\textit{Ppty}
                                                                     . $P$($r_1$.x$_0$)\\
                                                                    \hspace*{6em}\textit{wants}$'$($\lambda
                                                                     P$:\textit{Ppty}
                                                                     . $P$($r_1$.x$_1$))))\end{tabular}}}}}}



\begin{ex} 
\begin{subex} 
 
\item \usebox{\unpluggeda} 
 
\item \usebox{\unpluggedb}

\item \usebox{\unpluggedc}

\item \usebox{\unpluggedd}

\item \usebox{\unpluggede}

\item \usebox{\unpluggedf}

\item \usebox{\unpluggedg} 
 
\end{subex} 
   
\end{ex} 

\begin{ex} 
\begin{subex} 
 
\item \smallrecord{\field{bg}{\textit{Rec}}\\
                   \field{fg}{$\lambda\mathfrak{s}$:\textit{Rec} . \textit{every}$'$(\textit{boy}$'$)(\textit{want}$'$(\textit{a}$'$(\textit{dog}$'$)))}}
                               
\item  \smallrecord{\field{bg}{\textit{Rec}}\\
                    \field{fg}{\begin{tabular}{l}
                                    $\lambda\mathfrak{s}$:\textit{Rec}
                                                                     \\
                                      \hspace*{1em}\textit{a}$'$(\textit{dog}$'$)
                                                                      \\
                                      \hspace*{2em}($\lambda
                                                           r_1$:\smallrecord{\smalltfield{x$_1$}{\textit{Ind}}}
                                                                     . \\
                                      \hspace*{3em}\textit{every}$'$(\textit{boy}$'$)\\
                                      \hspace*{4em}($\lambda
                                                          r_2$:\smallrecord{\smalltfield{x$_0$}{\textit{Ind}}}
                                                                     . \\
                                      \hspace*{5em}$\lambda
                                                        P$:\textit{Ppty}
                                                                     . $P$($r_1$.x$_0$)\\
                                       \hspace*{6em}\textit{wants}$'$($\lambda
                                                                     P$:\textit{Ppty}
                                                                     . $P$($r_1$.x$_1$))))\end{tabular}}}
 
\item  \smallrecord{\field{bg}{\textit{Rec}}\\
                                                        \field{fg}{\begin{tabular}{l}
                                                                    $\lambda\mathfrak{s}$:\textit{Rec}
                                                                     \\
                                                                    \hspace*{1em}\textit{every}$'$(\textit{boy}$'$)
                                                                      \\
                                                                    \hspace*{2em}($\lambda
                                                                     r_1$:\smallrecord{\smalltfield{x$_0$}{\textit{Ind}}}
                                                                     . \\
                                                                    \hspace*{3em}\textit{a}$'$(\textit{dog}$'$)\\
                                                                    \hspace*{4em}($\lambda
                                                                     r_2$:\smallrecord{\smalltfield{x$_1$}{\textit{Ind}}}
                                                                     . \\
                                                                    \hspace*{5em}$\lambda
                                                                     P$:\textit{Ppty}
                                                                     . $P$($r_1$.x$_0$)\\
                                                                    \hspace*{6em}\textit{wants}$'$($\lambda
                                                                     P$:\textit{Ppty}
                                                                     . $P$($r_1$.x$_1$))))\end{tabular}}}
 
\end{subex} 
\label{ex:plugged}   
\end{ex} 
  
We now take a look at what is needed in order achieve the analysis we
have discussed above.  We will express storage and retrieval as
constraints on linguistically allowable sign types.  As a preliminary
we will define a bookkeeping function, max$_x$, which enables us to determine
the maximum $i$ such that `x$_i$' is used as a label in a set of
records, $R$.  This will enable us to store quantifiers using labels
`x$_0$', `x$_1$' and so on in order as in the examples discussed
above. `max$_x$' is defined as in \nexteg{} where $R$ is an arbitrary
set of records.
\begin{ex} 
max$_x$($R$) = 
\begin{quote}
max($\{i\mid\exists r\in R\ \exists v\ \langle \mathrm{x}_i,v\rangle\in r.\mathrm{bg}\}$), if
defined

otherwise: -1 
\end{quote}
\end{ex}
That is, max$_x$($R$) is the maximum $i$ such that `x$_i$' is a label
in the `bg'-field of one of the records in $R$.  If there is no such label max$_x$($R$)
returns -1. Thus in adding to a quantifier store, $R$, (as in the
`quants'-fields in the examples above) we will always add a new
quantifier indexed by max$_x$($R$)+1.  For brevity we will denote this by
incr$_x$($R$), that is, the result of incrementing the maximal x in $R$. 

It will also be useful to define an operation that adds a field
\smallrecord{\field{$\ell$}{$v$}} to a record $r$ if there is no field
with the label $\ell$ or replaces the $\ell$-field in $r$ with
\smallrecord{\field{$\ell$}{$v$}}.  We will represent the new record
thus derived as $r\oplus$\smallrecord{\field{$\ell$}{$v$}}.  This
is defined in \nexteg{}.
\begin{ex} 
$r\oplus$\smallrecord{\field{$\ell$}{$v$}} is \\
\hspace*{1em}$r\cup\{\langle\ell,v\rangle\}$, if there is no $v'$ such
that $\langle\ell,v'\rangle\in\mathfrak{s}$\\
\hspace*{1em}$(r-\{\langle\ell,r.l\rangle\})\cup\{\langle\ell,v\rangle\}$, otherwise 
\end{ex} 
Similarly we define $T\ominus$\smallrecord{\smalltfield{$\ell$}{$v$}}
for record types, $T$, in
\nexteg{}.
\begin{ex} 
$T\ominus$\smallrecord{\smalltfield{$\ell$}{$v$}} is \\ 
\hspace*{1em}$T-\{\langle\ell,v\rangle\}$ 
\end{ex} 
Note that this version of subtraction for record types will only
result in a well-formed record type if there are not fields in $T$
which depend on $\ell$.  In order to rectify this we need to make
$T\ominus$\smallrecord{\smalltfield{$\ell$}{$v$}} remove in addition
all fields which depend on $\ell$.  A field, $f$, is dependent on $\ell$
just in case it is of the form
$\langle\ell',\langle\mathcal{F},\pi\rangle\rangle$ where $\pi$ is a
sequence of paths containing $\ell$. Therefore we can refine the
definition in \preveg{} as that in \nexteg{}.
\begin{ex} 
$T\ominus$\smallrecord{\smalltfield{$\ell$}{$v$}} is \\ 
\hspace*{1em}$T-(\{\langle\ell,v\rangle\}\cup\{f\in T\mid f\
\text{dependent on}\ \ell\})$ 
\end{ex} 
This is technically not enough, since there may still be fields left
which are dependent on the dependent fields we have removed.
Therefore we need to make the definition recursive as in \nexteg{}.
\begin{ex} 
$T\ominus$\smallrecord{\smalltfield{$\ell$}{$v$}} is \\ 
\hspace*{1em}$(\ldots(T-\{\langle\ell,v\rangle\}\ominus
f_0)\ldots)\ominus f_n$ where $\{f_0,\ldots,f_n\}$ is $\{f\in T\mid f\
\text{dependent on}\ \ell\}$ 
\end{ex}   

[Is this enough to cover paths which contain $\ell$?]

We can now define storage as in
\nexteg{}, where $\sigma$ is an arbitrary sign.
\begin{ex} 
\textbf{Storage}

If $\alpha\in\sigma$.cont.unplugged, $\alpha$.core : \textit{PQuant}
and \\ $\alpha$.core.fg $\not=$
$\lambda\mathfrak{s}$:\smallrecord{\smalltfield{x$_i$}{\textit{Ind}}}
  . $\lambda P$:\textit{Ppty} . $P$($\mathfrak{s}$.x$_i$) for any $i$, \\then

\hspace*{1em}\record{\field{quants}{$\alpha$.quants $\cup$ \{
                              \record{\field{bg}{\smallrecord{\smalltfield{x$_{\mathrm{incr}_x(\alpha.\mathrm{quants})}$}{\textit{Ind}}}}\\
        \field{fg}{$\lambda\mathfrak{s}$:\smallrecord{\smalltfield{x$_{\mathrm{incr}_x(\alpha.\mathrm{quants})}$}{\textit{Ind}}}
          . $\alpha$.core.fg($\mathfrak{s}$)}}\}}\\
\field{core}{\record{\field{bg}{\smallrecord{\smalltfield{x$_{\mathrm{incr}_x(\alpha.\mathrm{quants})}$}{\textit{Ind}}}}\\
                     \field{fg}{$\lambda\mathfrak{s}$:\smallrecord{\smalltfield{x$_{\mathrm{incr}_x(\alpha.\mathrm{quants})}$}{\textit{Ind}}}
                       . $\lambda P$:\textit{Ppty}
                       . $P$($\mathfrak{s}$.x$_{\mathrm{incr}_x(\alpha.\mathrm{quants})}$)}}}}

\hspace*{4em}$\in$ $\sigma$.cont.unplugged
                                 
\end{ex} 
If $\sigma$ is a sign, $\sigma$.cont.unplugged is a set of unplugged interpretations.
\preveg{} says that if any
of those is a noun-phrase interpretation, that is, its core is of type
\textit{PQuant}, and is not the result of applying storage, that is,
the foreground of its core is not an interpretation depending on a
context, $\mathfrak{s}$, as introduced by storage, then
$\sigma$.cont.unplugged also contains an unplugged interpretation
where the core has been stored. 

We now characterize retrieval (in a version corresponding to
quantification with scope over sentences, as opposed to verb phrases
or common nouns) in \nexteg{}, where again $\sigma$ is an arbitrary
sign.
\begin{ex} 
\textbf{Retrieval (S)} 

If $\alpha\in\sigma$.cont.unplugged, for some $i$,
$\lambda\mathfrak{s}$:\smallrecord{\smalltfield{x$_i$}{\textit{Ind}}}
. $Q$ $\in$ $\alpha$.quants and $\alpha$.core : ($T$ $\rightarrow$
\textit{Type}), where $T$ $\sqsubseteq$
\smallrecord{\smalltfield{x$_i$}{\textit{Ind}}}, then \\

\hspace*{1em}\record{\field{quants}{$\alpha$.quants $-$
    $\{\lambda\mathfrak{s}$:\smallrecord{\smalltfield{x$_i$}{\textit{Ind}}}
    . $Q\}$} \\
        \field{core}{$\lambda\mathfrak{s}$:$T-$\{\smallrecord{\smalltfield{x$_i$}{\textit{Ind}}}\}
          . $Q$($\lambda
          r$:\smallrecord{\smalltfield{x$_i$}{\textit{Ind}}}
          . $\alpha$.core($\mathfrak{s}\oplus$\smallrecord{\field{x$_i$}{$r$.x$_i$}}))}}

\hspace*{4em}$\in$ $\sigma$.cont.unplugged
\end{ex} 
\preveg{} says that if any of the unplugged interpretations in a sign
contains a quantifier indexed by $i$, that is, it is a parametric
quantifier whose domain type is
\smallrecord{\smalltfield{x$_i$}{\textit{Ind}}}, and the core of this
unplugged interpretation is a function from contexts to a type, that
is, it corresponds to a (declarative) sentence interpretation, where
the context is required to provide a value for `x$_i$', that is, the
domain type is a subtype of
\smallrecord{\smalltfield{x$_i$}{\textit{Ind}}},\footnote{Note that
  this requirement will prevent the kind of vacuous binding that
  Keller storage \citep{Keller1988} avoided.} then the quantifier is
removed from the store and given scope over the sentence
interpretation in the core binding `x$_i$'.  The dependence on `x$_i$'
in the context is removed from the core interpretation. 

Finally, we show one way in which stored quantifiers can be
``percolated'' to higher constituents by showing in \nexteg{} how to
construct an unplugged content from two constituent unplugged contents
when the basic semantic composition involves function application.  To
facilitate this we first define the `x'-incrementation of an object
(such as a set or a function) with respect to a set of records $R$.

\begin{ex} 
The `x'\textit{-incrementation of an object $O$ with respect
  to a set of records $R$}, $[O]_{\mathrm{incr}_x(R)}$, is the result of
replacing each instance of x$_i$ in $O$, for any natural number $i$,
with x$_{\mathrm{incr}_x(R)+i}$.  
\end{ex} 
Now application for unplugged contents (that is, objects of type
\textit{UInterp}) can be characterized as in \nexteg{}.

\begin{ex} 
\textbf{Application}

If $\alpha$ and $\beta$ are of type \textit{UInterp} and $\beta$.core
is in the domain of $\alpha$.core, then the \textit{application of
  $\alpha$ to $\beta$}, $\alpha @@\beta$, is
\begin{quote}

\record{\field{quants}{$\alpha$.quants$\cup$[$\beta$.quants]$_{\mathrm{incr}(\alpha\mathrm{.quants})}$}\\
        \field{core}{$\alpha$.core@[$\beta$.core]$_{\mathrm{incr}(\alpha\mathrm{.quants})}$}}
\end{quote}
\end{ex} 
\preveg{} says that the application of an unplugged content, $\alpha$,
to another unplugged content, $\beta$, involves first changing the
quantifier indices in $\beta$ so that they increment the quantifier
indices in $\alpha$ in the way that is illustrated in the examples
discussed above.  For example, if both $\alpha$ and $\beta$ contain
the quantifier index `x$_0$' and this is the maximum in $\alpha$, then
`x$_0$' will be changed to `x$_1$' in $\beta$.  Now the result of
application is an unplugged content whose `quants' are the union of
$\alpha$'s `quants' and the incremented `quants' of $\beta$.  The core
is the result of applying $\alpha$'s core to $\beta$'s core using the
same incrementation. 
  

%[Grammar to be added.]

\section{Long distance dependencies}

Let us consider how to derive the content of \textit{who Sam hugged}
as a relative clause.  We use \textbf{hug}$_V$ to represent the
foreground of the content of \textit{hug} as a transitive verb,
characterized as in \nexteg{}.
\begin{ex} 
\textbf{hug}$_V$ = $\lambda\mathfrak{s}$:\textit{Rec} . $\lambda
r_2$:\smallrecord{\smalltfield{x}{\textit{Quant}}} . $\lambda
r_1$:\smallrecord{\smalltfield{x}{\textit{Ind}}}
. \smallrecord{\smalltfield{e}{hug($r_1$.x, $r_2$.x)}} 
\end{ex} 
Our theory of syntax in terms of event types means that what have been
called ``gaps'' or ``traces'' in other theories of syntax would
somehow have to correspond to non-events in which nothing happens and
do not have any temporal extent.  This strongly suggests that they do
not exist.  Instead we will take a strategy similar to that pursued,
for example, in combinatory categorial grammar (see, for example,
\citealp{Steedman2012}) where \textit{hug} can also be interpreted as
a verb phrase whose content is ``looking for'' a \textit{wh}-phrase
content.  We will represent the foreground of this content as
\textbf{hug}$_{\mathit{VP}}$ and characterize it in terms of
\textbf{hug}$_V$ as in \nexteg{a} which is identical with \nexteg{b}.
\begin{ex} 
\begin{subex}
\item \textbf{hug}$_{\mathit{VP}}$ = $\lambda\mathfrak{s}$:\smallrecord{\smalltfield{wh$_0$}{\textit{Ind}}}
. \textbf{hug}$_V$($\mathfrak{s}$)($\lambda P$:\textit{Ppty}
. $P$(\smallrecord{\field{x}{$\mathfrak{s}$.wh$_0$}})) 

\item
  $\lambda\mathfrak{s}$:\smallrecord{\smalltfield{wh$_0$}{\textit{Ind}}}
  . $\lambda r_1$:\smallrecord{\smalltfield{x}{\textit{Ind}}}
  . \smallrecord{\smalltfield{e}{hug($r_1$.x, $\lambda P$:\textit{Ppty}
. $P$(\smallrecord{\field{x}{$\mathfrak{s}$.wh$_0$}}))}}
\end{subex}
\end{ex}

Let us now consider the content of \textit{Sam hugged}.  In order to
avoid the additional complexity of combining different kinds of
context which we will undertake in Section~\ref{sec:struc-cntxt} we
will go back to the treatment of proper names in Chapter~\ref{ch:gram}
rather than the treatment in Chapter~\ref{ch:propnames} where proper
names introduce a constraint on the context. Thus we will consider the
foreground of the parametric content of \textit{Sam}, which we will
represent as \textbf{Sam}, to be \nexteg{}
where `sam' is a particular individual named Sam.
\begin{ex} 
\textbf{Sam} = $\lambda\mathfrak{s}$:\textit{Rec} . $\lambda P$:\textit{Ppty} . $P$(\smallrecord{\field{x}{sam}}) 
\end{ex}
The foreground of the content of \textit{Sam hugged}, which we will
represent as \textbf{Sam$^\frown$hugged}, is \nexteg{a} which is
identical with \nexteg{b}.
\begin{ex} 
\begin{subex} 
 
\item \textbf{Sam$^\frown$hugged} =
  $\lambda\mathfrak{s}$:\smallrecord{\smalltfield{wh$_0$}{\textit{Ind}}}
  . \textbf{Sam}($\mathfrak{s}$)(\textbf{hug}$_{\mathit{VP}}$($\mathfrak{s}$)) 
 
\item
  $\lambda\mathfrak{s}$:\smallrecord{\smalltfield{wh$_0$}{\textit{Ind}}}
  . \smallrecord{\smalltfield{e}{hug(sam, $\lambda P$:\textit{Ppty} . $P$(\smallrecord{\field{x}{$\mathfrak{s}$.wh$_0$}}))}}
 
\end{subex} 
   
\end{ex}
The foreground of the content of \textit{who}, which we will represent
as \textbf{who}, is given in \nexteg{}.
\begin{ex} 
\textbf{who} =
$\lambda\mathfrak{s}$:\smallrecord{\smalltfield{wh}{\textit{Ind}}}
. $\lambda P$:\textit{Ppty} . $P$(\smallrecord{\field{x}{$\mathfrak{s}$.wh}}) 
\end{ex} 
In order to be able to characterize the kind of binding involved in
combining \textit{who} with \textit{Sam hugged} we will define an
operation, $\oplus$, which adds a field to a record or, if the label
of the field being added
is already present in the record, will replace that field with the new
field.  This operation is characterized in \nexteg{}.
\begin{ex} 
If $r$ is a record, $\ell$ is a label and $v$ is some TTR object, then
\\
\hspace*{1em} $r\oplus[\ell=v]$ is \\
\hspace*{2em} $r\cup\{\langle\ell,v\rangle\}$, if there is no $v'$
such that $\langle\ell,v'\rangle\in r$\\
\hspace*{2em}
$(r-\{\langle\ell,r.\ell\rangle\})\cup\{\langle\ell,v\rangle\}$, otherwise 
\end{ex} 
We are using records here to do the work of partial variable
assignments in logic and the $\oplus$-operation carries out the kind
of modification which is performed on variable assignments when
defining binding in the logical treatment.  This now gives us what we
need to characterize the foreground of the content of the relative
clause \textit{who Sam hugged}, which we will represent as
\textbf{who}$^\frown$\textbf{Sam}$^\frown$\textbf{hugged}, as in
\nexteg{a} which is identical with \nexteg{b}.
\begin{ex} 
\begin{subex} 
 
\item \textbf{who}$^\frown$\textbf{Sam}$^\frown$\textbf{hugged} = \\
  \hspace*{1em}$\lambda\mathfrak{s}$:\textit{Rec} . \\
\hspace*{2em}$\lambda
  r_1$:\smallrecord{\smalltfield{x}{\textit{Ind}}}
  . \\
\hspace*{3em}\textbf{who}($\mathfrak{s}\oplus[\text{wh}=r_1.\text{x}]$)
($\lambda
  r_2$:\smallrecord{\smalltfield{x}{\textit{Ind}}} . \textbf{Sam}$^\frown$\textbf{hugged}($\mathfrak{s}\oplus[\text{wh}_0=r_2.\text{x}]$)) 
 
\item $\lambda\mathfrak{s}$:\textit{Rec} . $\lambda
  r_1$:\smallrecord{\smalltfield{x}{\textit{Ind}}}
  . \smallrecord{\smalltfield{e}{hug(sam, $\lambda P$:\textit{Ppty} . $P$(\smallrecord{\field{x}{$r_1$.x}}))}} 
 
\end{subex} 
   
\end{ex} 
     
   
    
    



\section{Conservativity and dynamic generalized quantifiers}

In Chapter~\ref{ch:commonnouns}, example (\ref{ex:dynqconstr}), we
introduced the notion of dynamic generalized quantifier and pointed
out that one of the original motivations for them was what is known as
donkey anaphora which we will discuss in Section~\ref{sec:anaph}.
Here we will point out a connection between dynamic quantifiers and
conservativity of quantifiers.  The informal way to state
conservativity for quantifiers is as in \nexteg{a} and an example is
given in \nexteg{b}.
\begin{ex} 
\begin{subex} 
 
\item Q A B is true just in case Q A A\&B is true 
 
\item \textit{every farmer likes a donkey} is true just in case
  \textit{every farmer is a farmer and likes a donkey} (or more
  naturally, \textit{every farmer is a farmer who likes a donkey}) is true
 
\end{subex} 
   
\end{ex} 
Most, if not all, natural language quantifiers have this
property.\footnote{For discussion, see \cite{PetersWesterstahl2006},
  p. 138f.}

Now consider the definition of dynamic generalized quantifiers given
in Chapter~\ref{ch:commonnouns}, Section~\ref{sec:dgqs}.  There in
example~(\ref{ex:dynqconstr}) we gave dynamic versions of generalized
quantifiers.  We can give a general characterization of the witness
conditions for dynamic generalized quantifiers as in \nexteg{}, where
$q$ is a predicate corresponding to a quantifier and $q^*$ is the
relation between sets corresponding to $q$ in classical generalized
quantifier theory.
\begin{ex} 
$e:q(P,Q)$ iff $q^*$ holds between $\downP{P}$ and
$\downP{Q|_{\mathcal{F}(P.\mathrm{fg})}\restriction e}$
\label{ex:witness-cond-GQ}
\end{ex} 
According to the definitions given in Chapter~\ref{ch:commonnouns},
$\downP{Q|_{\mathcal{F}(P.\mathrm{fg})}\restriction e}$ is the set of
individuals which have property $Q|_{\mathcal{F}(P.\mathrm{fg})}$ in
$e$ and $Q|_{\mathcal{F}(P.\mathrm{fg})}$ is $Q$ with its
domain restricted by $P$ (example~(\ref{ex:function-restriction}), p.~\pageref{ex:function-restriction}).  We spell this out for the type
corresponding to \textit{every dog runs} in \nexteg{}.  Here, as
before, we only represent the foregrounds of the properties which are
arguments to the predicate `every' in order to make the presentation
clearer.    
\begin{ex} 
\begin{subex} 
 
\item dog$'$ = $\lambda r$:\smallrecord{\smalltfield{x}{\textit{Ind}}}
    . \smallrecord{\smalltfield{e}{dog($r$.x)}} 

run$'$ = $\lambda r$:\smallrecord{\smalltfield{x}{\textit{Ind}}}
    . \smallrecord{\smalltfield{e}{run($r$.x)}}
 
\item $T$ = every(dog$'$, run$'$)

\item \begin{tabbing}
$s:T$ \=iff
$\downP{\text{dog}'}\subseteq\downP{\text{run}'|_{\mathcal{F}(\text{dog}')}\restriction
s}  $ \hspace*{2em}(\ref{ex:witness-cond-GQ})\\
\> iff  $\downP{\text{dog}'}\subseteq\downP{\text{run}'|_{\text{\smallrecord{\smalltfield{x}{\textit{Ind}}\\\smalltfield{e}{dog(x)}}}}\restriction
s}  $ \hspace*{2em}(def. of fixed point types p.~\pageref{pg:fixedpointtype}f)\\
\> iff $\downP{\text{dog}'}\subseteq$ [$\downarrow\lambda
r$:\smallrecord{\smalltfield{x}{\textit{Ind}}}\d{$\wedge$}\smallrecord{\smalltfield{x}{\textit{Ind}}\\\smalltfield{e}{dog(x)}}
     . \smallrecord{\smalltfield{e}{run($r$.x)}}$\restriction s$ ] \hspace*{2em} (Ch.~\ref{ch:commonnouns},
(\ref{ex:function-restriction}), p.~\pageref{ex:function-restriction})\\
\> iff $\downP{\text{dog}'}\subseteq$ [$\downarrow\lambda r$:\smallrecord{\smalltfield{x}{\textit{Ind}}\\\smalltfield{e}{dog(x)}}
     . \smallrecord{\smalltfield{e}{run($r$.x)}}$\restriction s$ ]
     \hspace*{2em}\begin{minipage}{.3\textwidth}(def. of merge, Appendix~\ref{app:merge})\end{minipage}\\
\> iff $\downP{\text{dog}'}\subseteq$ [$\downarrow\lambda r$:\smallrecord{\smalltfield{x}{\textit{Ind}}\\\smalltfield{e}{dog(x)}}
     . \smallrecord{\smallrfield{e}{$s$}{run($r$.x)}}
     ]\hspace*{2em}\begin{minipage}{.3\textwidth}(def. of property restriction,
     Appendix~\ref{app:gramrulesuniv}, p.~\pageref{pg:property-restriction})\end{minipage}\\
\> iff \=$\{a\mid\exists r$:\smallrecord{\smalltfield{x}{\textit{Ind}}}
  $\wedge
  r.\text{x}=a\wedge\down{\text{\smallrecord{\smalltfield{e}{dog($r$.x)}}}}\not=\emptyset\}$\\
\>\>$\subseteq\{a\mid\exists r$:\smallrecord{\smalltfield{x}{\textit{Ind}}\\\smalltfield{e}{dog(x)}}
  $\wedge
  r.\text{x}=a\wedge\down{\text{\smallrecord{\smallrfield{e}{$s$}{run($r$.x)}}}}\not=\emptyset\}$
  \\
\> iff
$\{a\mid\down{\text{dog}(a)}\not=\emptyset\}\subseteq\{a\mid\down{\text{dog}(a)}\not=\emptyset\wedge\exists
s'[s'\underline{\varepsilon}s\wedge s':\text{run}(a)]\}$\\
\> iff $\{a\mid\exists s'[s':\text{dog}(a)]\}\subseteq\{a\mid\exists
s'[s':\text{dog}(a)]\wedge\exists
s'[s'\underline{\varepsilon}s\wedge s':\text{run}(a)]\}$ 
\end{tabbing} 
\end{subex} 
   
\end{ex} 
  

[@@ 
(p. 12 of notes)]


\section{Anaphora}
\label{sec:anaph}

[donkey anaphora]

\section{Structured contexts, binding principles and projection}
\label{sec:struc-cntxt}

\section{Compositional semantics and incremental processing}

%%% Local Variables: 
%%% mode: latex
%%% TeX-master: "ttl"
%%% End: 
