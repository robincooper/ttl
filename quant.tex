\chapter{Quantification, anaphora and underspecification}
\label{ch:quant}
\setcounter{examplectr}{0}

\section{The interpretation of unbound pronouns}
We will consider how to recreate a simple interpretation of pronouns
ranging over individuals, first treating them in a similar way to free
variables in logic and then showing how they can be bound by
quantifiers.  The central idea is to use records as \textit{pronominal
  contexts} which correspond to partial assignments to variables in
logical treatments.  Consider first a simple sentence with a deictic
pronoun as in \nexteg{}.
\begin{ex} 
he left 
\end{ex} 
In our initial pass we will ignore matters
of gender to make things simpler.  The content of \preveg{} is a type
which depends on a context (a situation) which provides a value of the
pronoun \textit{he}. Thus it will have a parametric content which is a
function from a context assigning a value to the pronoun to a
type.\footnote{We will consider later how to combine types of context
  which assign values to pronouns with other context types which we
  have introduced with parametric contents.}  We will use the variable
`$\sigma$' to represent such contexts.  Thus a parametric content for
\textit{he} could be that given in \nexteg{a}, for \textit{left}
(ignoring tense) as \nexteg{b} and their combination, using a variant
of S-combination which we will discuss as we progress, is represented
in \nexteg{c}.
\begin{ex} 
\begin{subex} 
 
\item $\lambda\sigma$:\smallrecord{\smalltfield{x$_0$}{\textit{Ind}}} . 
        $\lambda P$:\textit{Ppty} . $P$(\smallrecord{\field{x}{$\sigma$.x$_0$}})
 
\item $\lambda\sigma$:\textit{Rec} . 
        $\lambda r$:\smallrecord{\smalltfield{x}{\textit{Ind}}} . 
              \record{\tfield{e}{leave($r$.x)}}

\item $\lambda\sigma$:\smallrecord{\smalltfield{x$_0$}{\textit{Ind}}}
  . 
         \record{\tfield{e}{leave($\sigma$.x$_0$)}}
 
\end{subex} 
   
\end{ex}
We follow Montague's strategy in allowing the content of \textit{he}
to be 
\begin{quote}
$\lambda\sigma$:\smallrecord{\smalltfield{x$_i$}{\textit{Ind}}} . 
        $\lambda P$:\textit{Ppty}
        . $P$(\smallrecord{\field{x}{$\sigma$.x$_i$}})
\end{quote} 
for any natural number $i$.  Thus considering the boy and the dog, a
content for \nexteg{a} will be \nexteg{b}.
\begin{ex} 
\begin{subex} 
 
\item he hugged it 
 
\item $\lambda\sigma$:\smallrecord{\smalltfield{x$_0$}{\textit{Ind}}\\
                                   \smalltfield{x$_1$}{\textit{Ind}}}
                                 . 
\record{\tfield{e}{hug($\sigma$.x$_0$, $\sigma$.x$_1$)}} 
 
\end{subex} 
   
\end{ex} 
An advantage of using record types to characterize pronominal contexts
rather than variable assignments is that we can add further
information represented by the pronoun such as gender.   Thus a simple
treatment of gender for \preveg{} might be given by making the content
be \nexteg{}.
\begin{ex} 
$\lambda\sigma$:\smallrecord{\smalltfield{x$_0$}{\textit{Ind}}\\
                             \smalltfield{c$_0$}{male(x$_0$)}\\
                             \smalltfield{x$_1$}{\textit{Ind}}\\
                             \smalltfield{c$_1$}{neuter(x$_1$)}}
                                 . 
\record{\tfield{e}{hug($\sigma$.x$_0$, $\sigma$.x$_1$)}} 
\end{ex} 
As we will see below, there are some complications with this simple
idea when it comes to the interpretation of pronouns which are bound
by quantifiers.  Even for deictic pronouns there are problems
determining which predicates should be used in a semantic treatment of
gender.  Even for a language like English which apparently has
semantic gender (as opposed to grammatical gender like German or
French), neuter can be used for objects which do not have gender (like
tables) and for animals other than humans which do have gender and
which can be referred to with masculine and feminine pronouns.

\section{Quantifier scope and underspecification}

Given the kind of interpretation rules we have so far we can obtain a
reading for \nexteg{a} which corresponds to the kind of parametric
content using pronominal contexts
in \nexteg{b}, using the abbreviations `boy$'$' and `dog$'$' as
introduced in Chapter~\ref{ch:commonnouns}.
\begin{ex} 
\begin{subex} 
 
\item a boy hugged every dog 
 
\item $\lambda \sigma$:\textit{Rec} . \smallrecord{\smalltfield{e}{exist(boy$'$, $\lambda
      r_1$:\smallrecord{\smalltfield{x}{\textit{Ind}}} . 
\smallrecord{\smalltfield{e}{every(dog$'$, 
$\lambda r_2$: \smallrecord{\smalltfield{x}{\textit{Ind}}} . 
\smallrecord{\smalltfield{e}{hug($r_1$.x, $r_2$.x)}})}})}}
 
\end{subex} 
   
\end{ex} 
This, of course, represents the reading where there is a boy such that
he hugs every dog.  Notice that here we have vacuous abstraction over
pronominal contexts and the constraint on them is that are a record of
some kind without requiring any particular fields.    

In order to obtain the reading where \textit{every dog} has wide
scope, we follow Montague and pretty much everybody else in basing our
treatment of quantifier scope on the treatment of free pronouns,
though without the contribution of any gender information.  Let us
imagine just for a moment that such a pronoun existed in English and
is written as `x'.  Then the content for \nexteg{a} could be
\nexteg{b}.
   
  

\section{Generalized quantifiers}



\section{Anaphora}

[donkey anaphora]

%%% Local Variables: 
%%% mode: latex
%%% TeX-master: "ttl"
%%% End: 
