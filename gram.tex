\chapter{Grammar in a theory of action}
\label{ch:gram}
\setcounter{equation}{0}

% \todo{Christiansen and Chater Now-or-Never bottleneck}

\section{Introduction}

In Chapter~\ref{ch:infex} we made the simplifying assumption that
sentences come as single unanalyzed units (something like the
assumption that is made in propositional logic).  In this chapter we
will deal with the same simple examples but break the sentences down
into their constituent parts.  (This will be something like moving
from propositional logic to predicate logic without quantifiers.)
In order to do this we will need more complex signs.

\section{Constituent structure and events}

We will first consider how linguistic constituent structure is related
to our general perception of events.  We have so far talked of events
in terms of string types which we have related to finite state
automata.  Finite state automata are equivalent to regular grammars.
We will now consider an example of how we perceive events which
suggest a more complex structure in terms of strings of regular
types.  This gives us something which is equivalent to recursive
transition networks (RTNs) which are in turn equivalent to context
free grammars.\footnote{For a general introduction to automata theory
  and its relation to the Chomsky hierarchy see, for example,
  \cite{ParteeMeulenWall1990}.}  Consider an event type of bus trips,
\textit{BusTrip}.  This could be defined as in \nexteg{}.
\begin{ex} 
%\textit{BusTrip} $\equiv$
\textit{GetBus}$^{\frown}$\textit{TravelOnBus}$^{\frown}$\textit{GetOffBus} 
\label{ex:bustrip}
\end{ex} 
Each of the three event types which are concatenated in \preveg{}
could be further broken down into strings of events.  For example,
\textit{GetBus} might be defined as in \nexteg{}.
\begin{ex} 
  % \textit{GetBus} $\equiv$
 \textit{WaitAtBusstop}*$^{\frown}$\textit{BusArrive}$^{\frown}$\textit{GetOnBus} 
\end{ex} 
The elements in \preveg{} could be broken down further.  For example,
getting on the bus could be analyzed in terms of going towards a door
on the bus, waiting for the door to open, placing one foot on the step
into the bus and then the other, paying for your ticket and so on.
There seems almost no limit to how finegrained an analysis of events
we can give.  Which muscles do you have to move in order to place your
right foot inside the bus?  What events are involved in the contraction
of this muscle?  However, there seems to be a limit on the level of
detail we need to be conscious of (or even are capable of being
conscious of) in order to carry out a high level action like getting
on a bus.  We can also build upwards from the type \textit{BusTrip}.
For example, many bus trips are not direct in that we have to change
buses in order to reach our destination.  Thus a bus trip can consist
of a string of events where you get on a bus, travel on it and then
get off it again.  A return bus trip involves a bus trip from one
place to another followed (after intervening events) by a bus trip
from the second place back to the first.  Both of those bus trips
might involve several buses if the connection is not direct.   

% The notation we have used in (\ref{ex:bustrip}) and \preveg{} is used
% to mean that what occurs to the left of $\equiv$ is a convenient
% notational abbreviation for what occurs on the right.  That is,
% whenever we write the symbol on the left, that is just shorthand for
% the longer expression on the right.
Given the two definitions in
(\ref{ex:bustrip}) and \preveg{}, \textit{BusTrip} could be seen as an
abbreviation for the regular string type in \nexteg{}.
\begin{ex} 
\textit{WaitAtBusstop}*$^{\frown}$\textit{BusArrive}$^{\frown}$\textit{GetOnBus}$^{\frown}$\textit{TravelOnBus}$^{\frown}$\textit{GetOffBus} 
\end{ex} 
Thus while our notation is giving us the beginnings of a hierarchical
organization, the type that is represented by the notation is not
hierarchically organized.  We are still in the realm of a finite state
system.
Compare this with the statements in \nexteg{}.
\begin{ex} 
\begin{subex} 
 
\item $e$ : \textit{BusTrip} iff $e$ : \textit{GetBus}$^{\frown}$\textit{TravelOnBus}$^{\frown}$\textit{GetOffBus} 
 
\item $e$ : \textit{GetBus} iff $e$ : \textit{WaitAtBusstop}*$^{\frown}$\textit{BusArrive}$^{\frown}$\textit{GetOnBus} 
 
\end{subex} 
   
\end{ex} 
The statements in \preveg{} claim that there are distinct types
\textit{BusTrip} and \textit{GetBus} in addition to the regular types
used on the right-hand side of `iff'.  These types are
\textit{equivalent} to the regular types in the sense that anything of
the one type will be of the other type.  Now the type system
is
hierarchically organized and includes two additional ``higher''
types \textit{BusTrip} and \textit{GetBus}.  On the face of it one
might think that the type system with the additional higher types
would be just a more complicated way of achieving the same result and
would be less efficient than a system which just includes the regular
types.  However, there seems to be good reason to suppose that an
organism that organizes its event perception in terms of such a
hierarchical type system would have serious advantages over an
organism that lacks the hierachical organization.  These advantages
include at least the following:
\begin{description}
\item[access and compact representation] Recall from
  Chapter~\ref{ch:percint} that we want to consider the types that an
  agent has available as resources as being represented in the brain
  states of the agent.  Having higher types means
  that something corresponding to a complex type can be stored as a
  single element.  In a complex reasoning task this can give
  considerable advantage in that the task can be represented in a more
  compact fashion and it can be easier to access (search and find)
  something which is a single element rather than something which is
  represented in terms of a complex string each element of which has
  to be checked in order to be sure that you have found the right element.
\item[planning] Having a compact representation facilitates
  planning.  It is feasible to plan to take a bus trip given that we
  can conceive of it as such without having to plan for all the small
  subevents that make it up, for example, all that is involved in
  lifting your legs in the right way in order get on the bus.  The
  ability to plan actions seems based on an ability to classify events
  in a hierarchical way.
\item[reuse] A hierarchical organization of event types means that
  certain event types can be reused in other event types.  For
  example, getting on a bus (waiting for the doors to open, putting
  one foot inside and so on) can be very much like getting on a
  train.  Similarly, paying for a ticket on a bus trip involves an
  exchange of money for a ticket in much the same way for a bus, a
  tram, a train, a theatre performance and so on.  An agent which is
  not able to perceive this kind of generalization would at best use
  up a lot of memory coding the same event types over and over as
  parts of different larger event types.
\item[learning] The hierarchical organization of event types and the reuse
  capabilities it offers also facilitates learning of new event
  types.  In learning to take the tram it can be useful to reuse what
  you have learnt about buying tickets on buses and insert it ready
  made into your type for tram trips.  If it turns out that the
  procedure for buying tickets for trams is slightly different from
  for buses (for example, you can buy a ticket on the bus but you have
  to pay before you get on the tram) you nevertheless have a buying
  ticket type which you can modify.  This might involve creating more
  types corresponding to those strings which the two ticket buying
  procedures have in common to separate out the differences between
  the two procedures. 

\end{description}   
Related observations about the importance of hierarchical structure
for behaviour and its relationship to hierarchical reinforcement
learning and neurological structure have been
made for example by
\cite{Botvinick2008,BotvinickNivBarto2009,Ribas-FernandesSolwayDiukMcGuireBartoNivBotvinick2011}.

Introducing hierarchical types in this way is an important step in our
cognitive processing of events because of the computational and
learning processes indicated above even if the class of events we are
formally able to recognize is the same as what could be recognized by
non-hierarchical regular string types, that is, technically, finite
state languages.  An organism with hierarchically organized types will
have important advantages in acquiring new finite state event
patterns.  An evolutionary step from non-hierarchically organized
string types to hierarchically organized types is a significant
development and organisms with hierarchical types will have clear
evolutionary advantages over those that do not.

However, hierarchical organization brings with it, almost as a kind of
side effect, something which means that the organism could recognize
classes of events that are not finite state.  This is known as
\textit{recursion}. Hierarchical organization means that we can give
type definitions of the form in \nexteg{}.
\begin{ex} 
$a:T$ iff $a:T_1^{\frown}\ldots^{\frown}T_n$ 
\end{ex} 
If we do not explicitly rule it out there is nothing to say that one
of the $T_i$ is not $T$ itself.  Of course, things will go badly wrong
if we have a definition such as \nexteg{}.
\begin{ex} 
$a:T$ iff $a:T_1^{\frown}T^{\frown}T_2$ 
\end{ex} 
If we try to perceive or create something of this type we will not be
able to terminate and get into an endless string of objects of type
$T_1$ and never be able to move on to $T_2$. However, if we define $T$
in terms of a join type where at least one of the types in the join
does not contain $T$, things will work fine.  For example, \nexteg{}:
\begin{ex} 
$a:T$ iff $a:(T_1^{\frown}T^{\frown}T_2\vee T_1^{\frown}T_2)$
\end{ex} 
According to \preveg{} anything of type $T$ will be a string of
objects of type $T_1$ followed by a string of equal length of objects
of type $T_2$.  It is the requirement ``of equal length'' which means
that this type is not a regular type.  For example, we could have the
the regular type $T_1^{+\frown}T_2^+$ but this only expresses that we
require a non-empty string of objects of type $T_1$ followed by a
non-empty string of objects of type $T_2$ without the equal length
requirement.   What we have done here is restate a basic result from
formal language theory in terms of our types.  In formal language
theory one talks of languages of the form $a^nb^m$ (the set of strings of $n$
$a$'s followed by a string of $m$ $b$'s, for any $n$ and $m$ greater
than 0) which is
a regular or finite state language and $a^nb^n$ (the set of strings of
$n$ $a$'s followed by $n$ $b$'s, for any $n$ greater than 0) which is
context free.  While this possibility of recursion is offered as soon as we allow
the hierarchical typing of events in this way, it is not clear that it
is exploited to a great extent in non-linguistic events.  The clear
examples that seem to exist are examples like opening and closing Chinese boxes, that is,
boxes within boxes.  The type of opening and closing (reassembling) a
Chinese box could be characterized as the $a^nb^n$-type in \nexteg{}.
\begin{ex} 
$e$ : \textit{OpenClose} iff \\
$e$ : (\textit{Open}$^{\frown}$\textit{OpenClose}$^{\frown}$\textit{Close}
$\vee$ \textit{Open}$^{\frown}$\textit{Close}) 
\end{ex} 
It is significant in this kind of example that the ordering of the
events is forced on the agent by the physical reality of the boxes.
There is only one order in which you can open all the boxes and only
one order (the reverse order) in which you can close them if you are
going to assemble all the boxes within a single box.  It is unclear
that such ordering is required in non-linguistic event types when it
is not dictated by physical reality.  



     

\section{Syntax}

We now turn our attention to how this hierarchical organization is
reflected in the nature of linguistic events.\footnote{A version of
  some material from this section has appeared in \cite{Cooper2014}.}  In
Chapter~\ref{ch:infex} we used \nexteg{} as our sign type.
\begin{ex} 
\record{\tfield{s-event}{\textit{SEvent}} \\
        \tfield{cont}{\textit{Cont}}} 
\end{ex} 
This represents the pairing of a speech event with content in a
Saussurean sign.  It does not, however, require the presence of any
hierarchical information in the sign corresponding to what in
linguistic theory is normally referred to as the \textit{constituent} (or
\textit{phrase}) structure of the utterance.  To some extent it is
arbitrary where we add this information.  We could, for example, add
it under the label `s-event', perhaps by dividing `s-event.e' into two
fields `phon' and `syn' (``syntax'').  However, it will be more
convenient (in terms of keeping paths that we need to refer to often
shorter) to add a third field labelled `syn' at the top level of the
sign type as in \nexteg{}.
\begin{ex} 
 \record{\tfield{s-event}{\textit{SEvent}} \\
         \tfield{syn}{\textit{Syn}} \\
        \tfield{cont}{\textit{Cont}}} 
\end{ex}
As we will see below, \textit{Syn} will require a
`daughters'-field for a string of signs.  This means that
\textit{Sign} becomes a recursive type.  It will be a \textit{basic}
type with its witnesses defined by \nexteg{}.
\begin{ex} 
$\sigma$ : \textit{Sign} iff $\sigma$ :  \record{\tfield{s-event}{\textit{SEvent}} \\
         \tfield{syn}{\textit{Syn}} \\
        \tfield{cont}{\textit{Cont}}} 
\end{ex} 
We shall take \textit{Syn} to be the type \nexteg{}.\footnote{One
  might think that \textit{Syn} should also be defined as a recursive
  type since it can contain \textit{Sign} which in its turn can
  contain \textit{Syn}.  However, in the types we are currently
  proposing the only way for \textit{Syn} to recur is through
  \textit{Sign} and it is sufficient for \textit{Sign} to
  be defined recursively to ensure that we do not introduce record
  types that are non-well founded sets of ordered pairs.  That is, we
  want to avoid the mathematical object which is the type being a set
  which contains itself.  In contrast the set of witnesses for
  a recursive type, while it will be infinite, will be well-founded.}
\begin{ex} 
\record{\tfield{cat}{\textit{Cat}} \\
        \tfield{daughters}{\textit{Sign}$^*$}} 
\end{ex}

The type \textit{Sign}, as so far defined, can be seen as a
\textit{universal resource}.  By this we mean that it is a type which
is available for all languages. 
\textit{Cat} is the type of names of syntactic categories.  In this chapter we
will take the witnesses of \textit{Cat} to be: `s' (``sentence''), `np' (``noun phrase''),
`det' (``determiner''), `n' (``noun''), `v' (``verb'') and `vp' (``verb
phrase'').  These correspond to the categories we will use to cover
the expressions of the fragment of English we introduced in
Chapter~\ref{ch:infex}. We will use capitalized versions of these
category names to represent types of signs with the appropriate path in a
sign type as in \nexteg{}.
\begin{ex} 
\begin{subex} 

\item \textit{S} represents (\textit{Sign} \d{$\wedge$} \smallrecord{\smalltfield{syn}{\smallrecord{\smallmfield{cat}{s}{\textit{Cat}}}}})
 
\item \textit{NP} represents  (\textit{Sign} \d{$\wedge$} \smallrecord{\smalltfield{syn}{\smallrecord{\smallmfield{cat}{np}{\textit{Cat}}}}}) 
 
\item \textit{Det} represents
   (\textit{Sign} \d{$\wedge$} \smallrecord{\smalltfield{syn}{\smallrecord{\smallmfield{cat}{det}{\textit{Cat}}}}})

\item \textit{N} represents
   (\textit{Sign} \d{$\wedge$} \smallrecord{\smalltfield{syn}{\smallrecord{\smallmfield{cat}{n}{\textit{Cat}}}}}) 

\item \textit{V} represents
   (\textit{Sign} \d{$\wedge$} \smallrecord{\smalltfield{syn}{\smallrecord{\smallmfield{cat}{v}{\textit{Cat}}}}})

\item \textit{VP} represents  (\textit{Sign} \d{$\wedge$} \smallrecord{\smalltfield{syn}{\smallrecord{\smallmfield{cat}{vp}{\textit{Cat}}}}})
 
\end{subex}
\label{ex:catsigntypes}   
\end{ex}

Recall that the symbol \d{$\wedge$} represents the merge operation on
types as defined in Appendix~\ref{app:merge}.  This means that, for
example, \preveg{a} is the type in \nexteg{}.
\begin{ex} 
\record{\tfield{s-event}{\record{\tfield{e-loc}{\textit{Loc}} \\
        \tfield{sp}{\textit{Ind}} \\
        \tfield{au}{\textit{Ind}} \\
        \tfield{e}{\textit{Phon}} \\
        \tfield{c$_{\mathrm{loc}}$}{loc(e,e-loc)} \\
        \tfield{c$_{\mathrm{sp}}$}{speaker(e,sp)} \\
        \tfield{c$_{\mathrm{au}}$}{audience(e,au)}}} \\
         \tfield{syn}{\record{\mfield{cat}{s}{\textit{Cat}} \\
        \tfield{daughters}{\textit{Sign}$^*$}}} \\
        \tfield{cont}{\textit{Cont}}}  
\end{ex} 
  

We might think that the type \textit{Cat} is a language specific
resource and indeed if we were being more precise we might introduce
separate types for different languages such as
\textit{Cat}$_{\mathit{eng}}$, \textit{Cat}$_{\mathit{swe}}$ and
\textit{Cat}$_{\mathit{tag}}$ for
the type of category names of English, Swedish and Tagalog respectively.
However, there is a strong intuition that categories in different
languages are more or less related.  For example, we would not be
surprised to find that the categories available for English and
Swedish closely overlap (despite the fact that their internal
syntactic structure differs) whereas the categories of English and Tagalog
have less overlap.  (See \citealp{Gil2000} for discussion.)  For this
reason we assume that there is a universal resource \textit{Cat} and
that each language will have a subtype of \textit{Cat} which specifies
which of the categories are used in that particular language.  This is
related to the kind of view of linguistic universals as a kind of
toolbox from which languages can choose which is put forward by
\cite{Jackendoff2002}.

The ontological status of objects of type \textit{Cat} as we have
presented them is a little suspicious.  Intuitively, categories should
be subtypes of \textit{Sign}, that is, like the types such as \textit{S},
\textit{NP} and so on in (\ref{ex:catsigntypes}).  We have identified
signs belonging to these types as containing a particular object in
\textit{Cat} in their `cat'-field.  But one might try to characterize
such signs in a different way, for example,as fulfilling certain conditions
such as having  certain kinds of daughters.  However, this is not quite enough, for
example, for lexical categories, which do not have daughters.  We have
to have a way of assigning categories to words and we need to create
something in the sign-type that will indicate the arbitrary assignment
of a category to a word.  For want of a better solution we will
introduce the category names which belong to the type \textit{Cat}
as a kind of ``book-keeping'' device that will identify a sign-type as
being one whose witnesses belong to the category bearing that name.  

The `daughters'-field is required to be a string of signs, possibly
the empty string, since the type \textit{Sign}$^*$ uses the Kleene-*,
that is the type of strings of signs including the empty string,
$\varepsilon$.  (See Appendix~\ref{sec:regular}.) Lexical items, that is words and phrases which are
entered in the lexicon, will be related to signs which have the empty
string of daughters.  We will use \textit{NoDaughters} to represent
the type
\smallrecord{\smalltfield{syn}{\smallrecord{\smallmfield{daughters}{$\varepsilon$}{\textit{Sign}$^*$}}}}.

If $T_{\mathrm{phon}}$ is a phonological type (that is,
$T_{\mathrm{phon}}\sqsubseteq\textit{Phon}$) and $T_{\mathrm{sign}}$
is a sign type (that is, $T_{\mathrm{sign}}\sqsubseteq\textit{Sign}$, then we shall use
Lex($T_{\mathrm{phon}}$, $T_{\mathrm{sign}}$) to represent \nexteg{}
\begin{ex} 
(($T_{\mathrm{sign}}$ \d{$\wedge$}
\smallrecord{\smalltfield{s-event}{\smallrecord{\smalltfield{e}{$T_{\mathrm{phon}}$}}}})
      \d{$\wedge$} \textit{NoDaughters}) 
\end{ex} 
This means, for
example, that \nexteg{a} represents the type in \nexteg{b}
which, after spelling out the abbreviations, can be seen to be the
type in \nexteg{c}.
\begin{ex} 
\begin{subex}

\item  Lex(``Dudamel'', \textit{NP})
 
\item ((\textit{NP} \d{$\wedge$}
\smallrecord{\smalltfield{s-event}{\smallrecord{\smalltfield{e}{``Dudamel''}}}})
      \d{$\wedge$} \textit{NoDaughters}) 
 
\item \record{\tfield{s-event}{\record{\tfield{e-loc}{\textit{Loc}} \\
        \tfield{sp}{\textit{Ind}} \\
        \tfield{au}{\textit{Ind}} \\
        \tfield{e}{``Dudamel''} \\
        \tfield{c$_{\mathrm{loc}}$}{loc(e,e-loc)} \\
        \tfield{c$_{\mathrm{sp}}$}{speaker(e,sp)} \\
        \tfield{c$_{\mathrm{au}}$}{audience(e,au)}}} \\
         \tfield{syn}{\record{\mfield{cat}{np}{\textit{Cat}} \\
        \mfield{daughters}{$\varepsilon$}{\textit{Sign}$^*$}}} \\
        \tfield{cont}{\textit{Cont}}} 
 
\end{subex} 
\label{ex:DudamelLex}   
\end{ex} 
We can think of `Lex' as the function in \nexteg{}\footnote{We are
  using the notational convention for function application as used,
  for example, by
  \cite{Montague1973} that if $f$ is a function $f(a,b)$ is $f(b)(a)$.}
\begin{ex} 
$\lambda T_1$:\textit{Type}\\
\hspace*{.25em}$\lambda T_2$:\textit{Type} . \\
\hspace*{.5em}\mbox{(($T_1$ \d{$\wedge$}
\smallrecord{\smalltfield{s-event}{\smallrecord{\smalltfield{e}{$T_2$}}}})
      \d{$\wedge$} \textit{NoDaughters})}
\end{ex} 
This function, which is used to create sign types for lexical items in a
language, associating types with a syntactic category, can be seen as
a universal resource.  We can think of it as representing a (somewhat
uninteresting, but nevertheless true) linguistic universal:  ``There
can be speech events of given types which have 
no daughters (lexical items)''.  
  
   

The lexical resources needed to cover our example
fragment is given in \nexteg{}.
\begin{ex} 
Lex(``Dudamel'', \textit{NP}) \\
Lex(``Beethoven'', \textit{NP}) \\
Lex(``a'', \textit{Det}) \\
Lex(``composer'', \textit{N}) \\
Lex(``conductor'', \textit{N}) \\
Lex(``is'', \textit{V}) \\
Lex(``ok'', \textit{S}) \\
Lex(``aha'',\textit{S})
\label{ex:gramlexres} 
\end{ex}
The choice of \textit{S} for ``ok'' and ``aha'' might be seen as an
arbitrary choice for the sake of this particular restricted fragment
of English.  We will not pursue details of the syntax of these
particles here.

The types in \preveg{} belong to the specific resources required for
English. This is not to say that these resources cannot be shared with
other languages.  Proper names like \textit{Dudamel} and
\textit{Beethoven} have a special status in that they can be reused in
any language, though often in modified form, at least in terms of the
phonological type with which they are associated without this being
perceived as quotation, code-switching or simply showing off that you
know another language. 

Resources like \preveg{} can be exploited by action rules.  If
Lex($T_w$, $C$)
is one of the lexical resources available to an agent $A$
and $A$ judges an event $e$ to be of type $T_w$, then $A$ is
licensed to update their gameboard with the type Lex($T_w$, $C$).
Intuitively, this means that if the agent hears an utterance of the
word ``composer'', then they can conclude that they have heard a sign
which has the category noun.  This is the beginning of
\textit{parsing}, which we will regard as the same kind of update
involved in event perception as discussed in the previous chapters.
The action rule corresponding to lexical resources
like \preveg{}, \textsc{LexRes}, is given in \nexteg{}, where we use `$T$
resource$_A$' to mean that the type $T$ is a resource available to
$A$.  Here the relevant resource is a type returned by the function `Lex'.)  We will return below to how
this relates to gameboard update.  
\begin{ex}
  \begin{prooftree}
    \hypo{\text{Lex}(T,C) \text{ resource}_A}
    \hypo{u:_A T}
    \infer[enth]2{:_A(\text{Lex}(T,C)\text{\d{$\wedge$}\smallrecord{\smalltfield{s-event}{\smallrecord{\smallmfield{e}{$u$}{$T$}}}}})}
  \end{prooftree}
  
% If Lex($T$, $C$) is a resource available to agent $A$, then for any
% $u$, $u:_A T$ licenses $:_A$ Lex($T$, $C$)
% \d{$\wedge$}\smallrecord{\smalltfield{s-event}{\smallrecord{\smallmfield{e}{$u$}{$T_1$}}}}

\end{ex} 
\preveg{} says that an agent with lexical resource Lex($T$, $C$) who
judges a speech event, $u$, to be of type $T$ is licensed to judge that
there is a sign of type Lex($T$, $C$) whose `s-event.e'-field contains
$u$. 

Strings of utterances of words can be classified as utterances of
phrases.  That is, speech events are hierarchically organized into
types of speech events in the way that we discussed at the beginning
of this chapter.  Agents have resources which allow them to reclassify
a string of signs of certain types (``the daughters'') into a single
sign of another type (``the mother'').  So for example a string of
type \textit{Det}$^{\frown}$\textit{N} can lead us to the conclusion
that we have observed a sign of type \textit{NP} whose daughters are
of the type \textit{Det}$^{\frown}$\textit{N}.  The resource that
allows us to do this is a rule which we will
model as the function in \nexteg{a} which we will represent as \nexteg{b}.
\begin{ex}
\begin{subex} 
\item $\lambda u$ : \textit{Det}$^{\frown}$\textit{N} . \\
\hspace*{1em}\textit{NP} \d{$\wedge$}
\smallrecord{\smalltfield{syn}{\smallrecord{\smallmfield{daughters}{$u$}{\textit{Det}$^{\frown}$\textit{N}}}}}

\item RuleDaughters(\textit{NP}, \textit{Det}$^{\frown}$\textit{N})
\end{subex}
\label{ex:NPDetNDaughters}
\end{ex}



`RuleDaughters' is to be the function in \nexteg{}.
\begin{ex} 
$\lambda T_1$:\textit{Type} \\
\hspace*{1em} $\lambda T_2$:\textit{Type}\ . \\
\hspace*{2em} $\lambda u\! :\! T_1$\ . $T_2$ \d{$\wedge$} \smallrecord{\smalltfield{syn}{\smallrecord{\smallmfield{daughters}{$u$}{$T_1$}}}} 
\end{ex} 
Thus `RuleDaughters', if provided with a subtype of \textit{Sign}$^+$
and a subtype of \textit{Sign} as arguments, will return a function which 
maps a string of signs of the first type to the second type with the
restriction that the daughters field is filled by the string of
signs. `RuleDaughters' is one of a number of sign type construction
operations which we will introduce as universal resources which have
the property of returning a which combine signs.  The action rule
associated with `RuleDaughters', \textsc{CombineDaughters}, is characterized in
\nexteg{}.

%@@ Give names to such principles

\begin{ex}
  \begin{prooftree}
    \hypo{f=\text{RuleDaughters}(T_{\text{mother}},T_{\text{daughters}})}
    \hypo{f\text{ resource}_A}
    \hypo{u:_A T_{\text{daughters}}}
    \infer[enth]3{:_A f(u)}
  \end{prooftree}
  
% If $f:(T_1\rightarrow Type)$ is a sign combination function available to agent $A$, then for
% any $u$, $u :_A T_1$ licenses $:_A f(u)$ 
\end{ex} 
This means, for example, that if you categorize a string of signs as
being of type \textit{Det}$^{\frown}$\textit{N} then you are licensed to conclude
that there is a sign of type \textit{NP} with the additional
restriction that its daughters are $u$.   

`RuleDaughters' takes care of the `daughters'-field but it says
nothing about the `s-event.e'-field, that is the phonological type
associated with the new sign.  This should be required to be the
concatenation of all the `s-event.e'-fields in the daughters. If
$u:T^+$ where $T$ is a record type containing the path $\pi$, we will
use concat$_i$($u[i].\pi$),
the concatenation of all the values $u[i].\pi$ for each element in the
string $u$ in the order in which they occur in the string.

\begin{shaded}
This
definition is repeated in
Appendix~\ref{sec:regular}.\label{pg:concat-i}.

If $s$ is a string of length $n$ of records such that for each $i$,
$0\leq i<n$, $s[i].\pi$ is a defined path, 
$\displaystyle\mathop{\mathrm{concat}}_{0\leq i<n}(s[i].\pi)$ denotes
$s[0].\pi\ldots s[n-1].\pi$.  We use
concat$_i$($s[i].\pi$) to represent
$\displaystyle\mathop{\mathrm{concat}}_{0\leq
  i<\mathrm{length}(s)}(s[i].\pi)$.
\end{shaded}

We can now
formulate the function ConcatPhon as in \nexteg{}
\begin{ex} 
$\lambda
u$:\smallrecord{\smalltfield{s-event}{\smallrecord{\smalltfield{e}{\textit{Phon}}}}}$^+$\
. \\
\hspace*{1em}\record{\tfield{s-event}{\record{\mfield{e}{concat$_i$($u[i]$.s-event.e)}{\textit{Phon}}}}} 
\end{ex} 
ConcatPhon will map any string of speech events to the type of a
single speech event whose phonology (that is the value of `s-event.e')
is the concatenation of the phonologies of the individual speech
events in the string.

We want to combine the function \preveg{} with a function like that in
(\ref{ex:NPDetNDaughters}). We do this by merging the domain types of
the two functions and also merging the types that they return.  This
is shown in \nexteg{a} which in deference to standard linguistic
notation for phrase structure rules could be represented as
\nexteg{b}.\footnote{Note that `$\longrightarrow$' used in the phrase
  structure rule in \nexteg{b} is not the same arrow as `$\rightarrow$'
  which is used in our notation for function types.  We trust that the
  different contexts in which they occur will help to distinguish them.}
\begin{ex} 
\begin{subex}
\item  $\lambda u$ : \textit{Det}$^{\frown}$\textit{N} \d{$\wedge$} \smallrecord{\smalltfield{s-event}{\smallrecord{\smalltfield{e}{\textit{Phon}}}}}$^+$ . \\
\hspace*{1em}\textit{NP} \d{$\wedge$}
\smallrecord{\smalltfield{syn}{\smallrecord{\smallmfield{daughters}{$u$}{\textit{Det}$^{\frown}$\textit{N}}}}}
\\
\hspace*{2.5em}\d{$\wedge$}
\smallrecord{\smalltfield{s-event}{\smallrecord{\smallmfield{e}{concat$_i$($u[i]$.s-event.e)}{\textit{Phon}}}}}

\item \textit{NP} $\longrightarrow$ \textit{Det} \textit{N}
\end{subex}
\label{eg:NPDetN}
\end{ex}

For any dependent types (that is, functions which return a type) of
the form \nexteg{a} and \nexteg{b}, \nexteg{c} (the merge of the two
functions) represents the function \nexteg{d}.
\begin{ex} 
\begin{subex} 
 
\item $\lambda r\!:\!T_1\ .\ T_2\dep{r}$ 
 
\item $\lambda r\!:\!T_3\ .\ T_4\dep{r}$
  
\item $\lambda r\!:\!T_1\ .\ T_2\dep{r}$ \d{\d{$\wedge$}} $\lambda
  r\!:\!T_3\ .\ T_4\dep{r}$
  
\item $\lambda r\!:\!T_1$\d{$\wedge$}$T_3\ .\
T_2\dep{r}$\d{$\wedge$}$T_4\dep{r}$
 
\end{subex} 
   
\end{ex} 

If $C,C_1,\ldots,C_n$ are category sign types
as in (\ref{ex:catsigntypes}) then \nexteg{a} represents \nexteg{b}
\begin{ex} 
\begin{subex} 
 
\item $C \longrightarrow C_1 \ldots C_n$ 
 
\item RuleDaughters($C$,
${C_1}^{\frown}\ldots^{\frown}C_n$) \d{\d{$\wedge$}} ConcatPhon 
 
\end{subex} 
   
\end{ex} 
Thus the function in (\ref{eg:NPDetN} can be
represented in a third way as in \nexteg{}.
\begin{ex} 
RuleDaughters(\textit{NP}, \textit{Det}$^{\frown}$\textit{N})
\d{\d{$\wedge$}} ConcatPhon 
\end{ex} 
The ability to factorize rules into 
components in this way enables us to build a theory of resources
that will allow us to study them in isolation and also facilitates
the development of theories of learning.  It gives us a clue to how agents can build new
rules by combining existing components in novel ways.  It has implications for universality
as well. For example, while the rule
\textit{NP} $\longrightarrow$ \textit{Det} \textit{N} is not universal
(though it may be shared by a large number of languages), ConcatPhon
is a universally available rule component, allbeit a trivial universal
which says that you can have concatenations of speech events to make a
larger speech event.    
     

The rules associated with our small grammar are given by \nexteg{}
\begin{ex} 
\textit{S} $\longrightarrow$ \textit{NP} \textit{VP} \\
\textit{NP} $\longrightarrow$ \textit{Det} \textit{N} \\
\textit{VP} $\longrightarrow$ \textit{V} \textit{NP} 
\end{ex} 
  
It may seem that we have done an awful lot of work to arrive at simple
phrase structure rules.  Some readers might wonder why it is worth all
this trouble to ground the rules in a theory of events and action when
what we come up with in the end is something that can be expressed in
a standard notation which is one of the first things that a student of
syntax learns.  One reason has to do with our desire to explore the
relationship between the perception and processing of non-linguistics
events and speech events as discussed at the beginning of this
chapter.  Another reason has to do with placing natural constraints on
syntax.  By grounding syntactic structure in types of events we
provide a motivation for the kind of discussion in
\cite{Cooper1982}.  An abstract
syntax which proposes constituent structure which does not correspond
to speech events is not grounded in the same way and thus presents a
different kind of theory.

 
\section{Semantics}
\label{sec:gram-semantics}
We have so far specified our sign types in terms of phonology and
syntax.  Now we need to specify the content in the `cont'-field.  We
shall start by accounting for the contents of the lexical items
specified in (\ref{ex:gramlexres}).  We consider first the common
nouns \textit{composer} and \textit{conductor}.  For each of these we
introduce a predicate of arity $\langle$\textit{Ind}$\rangle$.  Our universal resources will include a function,
`SemCommonNoun' which will construct a common noun content from such a
predicate, $p$.  This is defined as in \nexteg{}.
\begin{ex} 
SemCommonNoun($p$) = $\lambda
r$:\smallrecord{\smalltfield{x}{\textit{Ind}}} . \record{\tfield{e}{$p$($r$.x)}} 
\end{ex} 
The function in \preveg{} is of type
(\smallrecord{\smalltfield{x}{\textit{Ind}}}$\rightarrow$\textit{RecType}).
That is, it is a function which maps any record containing a field
labelled `x' with an individual as value to a record type.  We will
abbreviate this type as \textit{Ppty} (for ``property'') and we will
call functions of this type \textit{properties}.  In our compositional
semantics, properties will play a similar role as functions from
individuals to truth values  in Montague
semantics, (that is, functions of type $\langle e,t\rangle$ in
Montague's system).  In place of individuals, we use records with an `x'-field
containing an individual.  The motivation for this will become
apparent later. Basically, we want to be
able to restrict the function by adding additional fields to the
domain type of the function.

In place of Montague's truth-values (that is, objects of
Montague's type $t$) we use record types.  Record types play the role
of ``propositions'' in our system.  Types, thought of as types of
situations, can be considered as truth-bearing objects.  They are true
just in case there is something of the type and false otherwise, that
is, if there is
nothing of the type.  The fact that we use the ``proposition-like''
objects as the results that our properties return is an essential
ingredient in our intensional treatment of properties.  In this way it
follows in the tradition of property theory
\citep{ChierchiaTurner1988,FoxLappin2005}  and Thomason's intensional
approach to propositional attitudes \citep{Thomason1980}.

We can now combine the `Lex'-function which builds sign types
excluding content information with our new way of constructing common
noun content.  We define a function Lex$_{\mathrm{CommonNoun}}$ which
takes a phonological type and a predicate and returns a sign type.
This is defined in \nexteg{}.
\begin{ex} 
Lex$_{\mathrm{CommonNoun}}$($T_{\mathrm{phon}}$, $p$) = \\
\hspace*{1em}Lex($T_{\mathrm{phon}}$, \textit{N}) \d{$\wedge$} \smallrecord{\smallmfield{cont}{SemCommonNoun($p$)}{\textit{Ppty}}} 
\end{ex} 
Note that the type of the content required here is \textit{Ppty}.  In
Chapter~\ref{ch:infex} we defined the content type \textit{Cont} to be
identical with \textit{RecType}.  Now we have to revise the definition
of \textit{Cont} to be (\textit{RecType} $\vee$ \textit{Ppty}).  We
will add further disjuncts to allow for more possibilities as we
progress.

In order to cover the two common nouns \textit{conductor} and
\textit{composer} we can include the sign types in \nexteg{} among our
resources.
\begin{ex} 
\begin{subex} 
 
\item Lex$_{\mathrm{CommonNoun}}$(``composer'', composer) 
 
\item Lex$_{\mathrm{CommonNoun}}$(``conductor'', conductor) 
 
\end{subex} 
   
\end{ex} 

Following Montague's (\citeyear{Montague1973}) original strategy we
shall treat the contents of noun-phrases such as \textit{Dudamel} or
\textit{a conductor} as being functions from properties to
truth-bearing elements, that is, in our terms, record types.  That is,
noun-phrase contents will be of type
(\textit{Ppty}$\rightarrow$\textit{RecType}) which we will abbreviate
as \textit{Quant} (for ``quantifier''). This means that we should now
redefine the type of contents, \textit{Cont}, as \textit{RecType}
$\vee$ \textit{Ppty} $\vee$ \textit{Quant}.

Some readers may find this a rather old-fashioned approach to the
treatment of proper names which in proposals after Montague are
analyzed as representing individuals rather than quantifiers.  There
is nothing in principle which rules out these more modern treatments
in TTR but for now we will keep to the classical Montague approach.
We will in any case change Montague's proposal in later chapters of
this book.
   

\textit{Dudamel} and
\textit{Beethoven} will receive proper name contents.  The recipe for
constructing a proper name content based on a particular individual
$a$ is given by SemPropName($a$) as defined in \nexteg{}.
\begin{ex} 
SemPropName($a$) = \\
\hspace*{1em}$\lambda P$:\textit{Ppty} . $P$(\smallrecord{\field{x}{$a$}}) 
\end{ex} 
We define Lex$_{\mathrm{PropName}}$ which takes a phonological type (a
name) and an individual (the referent of the name) and returns a sign
type as in \nexteg{}.
\begin{ex} 
Lex$_{\mathrm{PropName}}$($T_{\mathrm{Phon}}$, $a$) = \\
\hspace*{1em}Lex($T_{\mathrm{Phon}}$, \textit{NP}) \d{$\wedge$} \smallrecord{\smallmfield{cont}{SemPropName($a$)}{\textit{Quant}}} 
\end{ex} 
Resources to cover the proper names in our grammar could be as in
\nexteg{} where $d,b$ : \textit{Ind} (two individuals, Dudamel and
Beethoven).
\begin{ex} 
\begin{subex} 
 
\item Lex$_{\mathrm{PropName}}$(``Dudamel'', $d$) 
 
\item Lex$_{\mathrm{PropName}}$(``Beethoven'', $b$) 
 
\end{subex} 
   
\end{ex} 
Note that there is nothing to prevent us from constructing sign types
with the same phonological type but different contents.  Thus proper
names are not required to be ``logically proper'' in the sense that
there is one and only one individual which can be referred to by an
utterance belonging to the phonological type.  Names can be ambiguous.
For example, there are many composers named Bach and Strauss.  We have
the means to construct sign types for all of them on an as needed
basis.  This sign-based approach, even though it holds closely to
Montague's original treatment of proper names, has the advantage that
it does not require a proper name to be limited to a single individual
for its content.  See \cite{Cooper2017a} for some further discussion.

Now that we have both properties and quantifiers let us check at this
point that we are on the right track for combining them in something
like the kind of way that we will need for compositional semantics.
Suppose we want to combine a proper name content for \textit{Dudamel}
\nexteg{a} with the property of being a conductor \nexteg{b}.  The
obvious way to do this is by applying the function in \nexteg{a} to
the argument \nexteg{b} as represented in \nexteg{c}.  According to
the definition of functional application in
Appendix~\ref{app:funtypes}, \nexteg{c} is identical to \nexteg{d}
which in turn is identical to \nexteg{e}.  In turn the dot notation
for record path values shows
\nexteg{e} to be identical to \nexteg{f}.
\begin{ex} 
\begin{subex} 
 
\item $\lambda P$:\textit{Ppty} . $P$(\smallrecord{\field{x}{$d$}}) 
 
\item $\lambda
r$:\smallrecord{\smalltfield{x}{\textit{Ind}}}
. \record{\tfield{e}{conductor($r$.x)}} 

\item $\lambda P$:\textit{Ppty} . $P$(\smallrecord{\field{x}{$d$}}) \\
\hspace*{1em}($\lambda
r$:\smallrecord{\smalltfield{x}{\textit{Ind}}}
. \record{\tfield{e}{conductor($r$.x)}})

\item $\lambda
r$:\smallrecord{\smalltfield{x}{\textit{Ind}}}
. \record{\tfield{e}{conductor($r$.x)}}(\smallrecord{\field{x}{$d$}})

\item \record{\tfield{e}{conductor(\smallrecord{\field{x}{$d$}}.x)}}

\item \record{\tfield{e}{conductor($d$)}}
 
\end{subex} 
\label{ex:diac}  
\end{ex} 
      

This means that if we were dealing with a language like Russian where
\textit{Dudamel is a conductor} corresponds to a proper name followed
by a common noun we would have a good way of combining the two
contents by applying the content of the proper name to the content of
the common noun.\footnote{An alternative would be to treat the content
  of a proper name as a record rather than a quantifier and apply the
  property to the record as in \preveg{d}.  This would correspond to
  the treatment of proper names as individual denoting as discussed,
  for example, by \cite{Partee1986}.} However, things are not quite so
straightforward in English.  Here we use an indefinite article to form
the noun phrase \textit{a conductor}.  We shall treat the content of
indefinite articles as a function that maps properties to quantifiers
involving the existential relation between properties.  That is, it
will be a function of type (\textit{Ppty}$\rightarrow$\textit{Quant}),
a type which should be added to our definition of \textit{Cont} which
now becomes \textit{RecType}
$\vee$ \textit{Ppty} $\vee$ \textit{Quant} $\vee$
(\textit{Ppty}$\rightarrow$\textit{Quant}).  As part of our
universal resources we introduce a function `SemIndefArt' which is
defined as the function in \nexteg{}.
\begin{ex} 
$\lambda Q$:\textit{Ppty} . \\
\hspace*{1em} $\lambda P$:\textit{Ppty}
. \record{\mfield{restr}{$Q$}{\textit{Ppty}} \\
          \mfield{scope}{$P$}{\textit{Ppty}} \\
          \tfield{e}{exist(restr, scope)}}
\label{ex:semindefart}
\end{ex} 
We can also define a universal resource, Lex$_{\mathrm{IndefArt}}$, which associates a
phonological type (corresponding to an indefinite article in the
language) with this content, as defined in \nexteg{}.
\begin{ex} 
Lex$_{\mathrm{IndefArt}}$($T_{\mathrm{Phon}}$) = \\
\hspace*{1em} Lex($T_{\mathrm{Phon}}$, \textit{Det}) \d{$\wedge$} \smallrecord{\smallmfield{cont}{SemIndefArt}{(\textit{Ppty}$\rightarrow$\textit{Quant})}} 
\end{ex} 
The local resource for the English indefinite article would thus be
\nexteg{}.
\begin{ex} 
Lex$_{\mathrm{IndefArt}}$(``a'') 
\end{ex} 

The compositional semantics of a noun-phrase consisting of a
determiner followed by a noun will be the content of the determiner
applied to the content of the noun.  This is a case of \textit{content
  forward application}.  We define a function `ContForwardApp', which is
part of the universal resources, as in \nexteg{}.
\begin{ex} 
$\lambda T_1$:\textit{Type} $\lambda T_2$:\textit{Type} . \\
\hspace*{1em}$\lambda
u$:\smallrecord{\smalltfield{cont}{$(T_2\rightarrow
    T_1)$}}$^{\frown}$\smallrecord{\smalltfield{cont}{$T_2$}} . \\
\hspace*{2em}\smallrecord{\smallmfield{cont}{$u$[0].cont($u$[1].cont)}{$T_1$}}
\label{ex:ContForwardApp}
\end{ex} 
The intuition behind this function is that if you observe a string of
two utterances, the first of which has a content of type
$(T_2\rightarrow T_1)$ and the second of which has a content of type
$T_2$ then you are licensed to conclude that there is an utterance
whose content is the result of applying the content of the first
element in the string to the content of the second element of the
string. (For the notation $s[n]$ representing the $n$th element of a
string $s$ see Appendix~\ref{app:strings}.)  We can use
`ContForwardApp' to add constraints on content to a phrase structure
rule as in the example in \nexteg{}.
\begin{ex} 
\textit{NP} $\longrightarrow$ \textit{Det} \textit{N} \d{\d{$\wedge$}} ContForwardApp(\textit{Ppty},\textit{Quant}) 
\end{ex} 
Recall from (\ref{eg:NPDetN}a) that  \textit{NP} $\longrightarrow$
\textit{Det} \textit{N} is the function
\nexteg{a}. ContForwardApp(\textit{Ppty},\textit{Quant}) is the
function \nexteg{b}.  Merging these two functions yields \nexteg{c}.
\begin{ex} 
\begin{subex} 
 
\item $\lambda u$ : \textit{Det}$^{\frown}$\textit{N} \d{$\wedge$} \smallrecord{\smalltfield{s-event}{\smallrecord{\smalltfield{e}{\textit{Phon}}}}}$^+$ . \\
\hspace*{1em}\textit{NP} \d{$\wedge$}
\smallrecord{\smalltfield{syn}{\smallrecord{\smallmfield{daughters}{$u$}{\textit{Det}$^{\frown}$\textit{N}}}}}
\\
\hspace*{2.5em}\d{$\wedge$}
\smallrecord{\smalltfield{s-event}{\smallrecord{\smallmfield{e}{concat$_i$($u[i]$.s-event.e)}{\textit{Phon}}}}} 
 
\item $\lambda
u$:\smallrecord{\smalltfield{cont}{$(\mathit{Ppty}\rightarrow
    \mathit{Quant})$}}$^{\frown}$\smallrecord{\smalltfield{cont}{\textit{Ppty}}} . \\
\hspace*{2em}\smallrecord{\smallmfield{cont}{$u$[0].cont($u$[1].cont)}{\textit{Quant}}} 

\item $\lambda u$ : \textit{Det}$^{\frown}$\textit{N} \d{$\wedge$}
  \smallrecord{\smalltfield{s-event}{\smallrecord{\smalltfield{e}{\textit{Phon}}}}}$^+$ \\
\hspace*{4.9em}  \d{$\wedge$} \smallrecord{\smalltfield{cont}{$(\mathit{Ppty}\rightarrow
    \mathit{Quant})$}}$^{\frown}$\smallrecord{\smalltfield{cont}{\textit{Ppty}}}
. \\
\hspace*{1em}\textit{NP} \d{$\wedge$}
\smallrecord{\smalltfield{syn}{\smallrecord{\smallmfield{daughters}{$u$}{\textit{Det}$^{\frown}$\textit{N}}}}}
\\
\hspace*{2.5em}\d{$\wedge$}
\smallrecord{\smalltfield{s-event}{\smallrecord{\smallmfield{e}{concat$_i$($u[i]$.s-event.e)}{\textit{Phon}}}}}
\\
\hspace*{2.5em}\d{$\wedge$}
\smallrecord{\smallmfield{cont}{$u$[0].cont($u$[1].cont)}{\textit{Quant}}}
 
\end{subex} 
   
\end{ex}
A convenient abbreviatory notation for this interpreted phrase
structure rule is given in \nexteg{a} or more simply, since we can
read the content types off the types \textit{Det} and \textit{N}, \nexteg{b}.
\begin{ex}
  \begin{subex}
  \item \textit{NP} $\longrightarrow$ \textit{Det N} $\mid$
    \textit{Det}$'$(\textit{N}$'$:\textit{Ppty}):\textit{Quant}
    
  \item \textit{NP} $\longrightarrow$ \textit{Det N} $\mid$
    \textit{Det}$'$(\textit{N}$'$)
  \end{subex}
  

\end{ex} 
Here \textit{Det}$'$ and \textit{N}$'$ represent the contents of the
determiner and noun.
  
We can represent the type \nexteg{a} using an informal diagrammatic
tree notation which is common in linguistics as in
\nexteg{b}.\footnote{A similar use of tree notation, though relating
  to typed feature structures rather than types, is used in HPSG
  \citep[see, for example,][Chapter~2]{GinzburgSag2000}.}
\begin{ex} 
\begin{subex} 
 
\item \hspace*{1em}\textit{NP} \d{$\wedge$}
\smallrecord{\smalltfield{s-event}{\smallrecord{\smallmfield{e}{syn.daughters[0].s-event.e$^{\frown}$syn.daughters[0].s-event.e}{\textit{Phon}}}}\\
\smalltfield{syn}{\smallrecord{\smalltfield{daughters}{\textit{Det}$^{\frown}$\textit{N}}}}
\\
\smallmfield{cont}{syn.daughters[0].cont(syn.daughters[1].cont)}{\textit{Quant}}}


 
\item \Tree [.\textit{NP}\\$\alpha(\beta)$ 
\textit{Det}\\$\alpha$ \textit{N}\\$\beta$ ]
 
\end{subex} 
   
\end{ex}
Here what is written under the category type (e.g. $\alpha$, $\beta$) represents the value in
the `cont'-field. 
     
The content of an utterance of \textit{a conductor}
will be \nexteg{a} applied to \nexteg{b}, that is \nexteg{c}.
\begin{ex} 
\begin{subex} 
 
\item  $\lambda Q$:\textit{Ppty} . \\
\hspace*{1em} $\lambda P$:\textit{Ppty}
. \record{\mfield{restr}{$Q$}{\textit{Ppty}} \\
          \mfield{scope}{$P$}{\textit{Ppty}} \\
          \tfield{e}{exist(restr, scope)}}
 
\item $\lambda
r$:\smallrecord{\smalltfield{x}{\textit{Ind}}}
. \record{\tfield{e}{conductor($r$.x)}} 

\item $\lambda P$:\textit{Ppty}
. \record{\mfield{restr}{$\lambda
r$:\smallrecord{\smalltfield{x}{\textit{Ind}}}
. \record{\tfield{e}{conductor($r$.x)}}}{\textit{Ppty}} \\
          \mfield{scope}{$P$}{\textit{Ppty}} \\
          \tfield{e}{exist(restr, scope)}}
 
\end{subex} 
\label{ex:a-conductor}   
\end{ex} 

We will now look in more detail at the nature of the generalized
quantifier in \preveg{c}.  `exist' is a predicate with arity
$\langle$\textit{Ppty},\textit{Ppty}$\rangle$, that is, it corresponds
to a relation between two properties.  The classical account of
generalized quantifiers \citep[][and much other
literature]{BarwiseCooper1981,PetersWesterstahl2006} treats such
quantifier relations as relations between sets.  Here we will follow
\cite{Cooper2011,Cooper2013} in relating our treatment directly to the
classical relation between sets, although, as argued in
\cite{Cooper2012a} based on earlier work by \cite{KeenanStavi1986}, there are ultimately good reasons for exploiting
the intensionality of properties.   If $P$ is a property the relevant
set is the set of individuals which have the property, which we will
represent as $\downP{P}$.  This is defined as in \nexteg{} where
we use the notation $\down{T}$ to represent $\{a\mid a:T\}$.\label{pg:type-extension}

\begin{ex} 
$\downP{P}$ = $\{a\mid\exists r[r:$\smallrecord{\smallmfield{x}{$a$}{\textit{Ind}}}
\text {and} $\down{P(r)}\not=\emptyset]\}$ 
\label{ex:propext}
\end{ex} 
Following the terminology of \cite{Cooper2011,Cooper2013} we will call
$\downP{P}$ the property extension, or \textit{P-extension}, of
property $P$. Intuitively the property extension of $P$ is the set of
objects which have the property in some situation.  Let us compute
this for a particular example of a property, the property of being a
dog given in \nexteg{}. 
\begin{ex} 
$\lambda r$:\smallrecord{\smalltfield{x}{\textit{Ind}}} . \record{\tfield{e}{dog($r$.x)}} 
\end{ex}
The property extension of \preveg{} is given in \nexteg{}.
\begin{ex} 
$\{a\mid\exists
  r[r$:\smallrecord{\smallmfield{x}{$a$}{\textit{Ind}}} and \\
\hspace*{2em}$\down{\lambda
  r:\text{\smallrecord{\smalltfield{x}{\textit{Ind}}}}
  . \text{\record{\tfield{e}{dog($r$.x)}}}(r)} \not=\emptyset]\}$ 
\end{ex} 
By $\beta$-reduction (that is, the definition of function application) \preveg{} is the same set as \nexteg{}.
\begin{ex} 
$\{a\mid\exists
  r[r$:\smallrecord{\smallmfield{x}{$a$}{\textit{Ind}}} and 
$\down{\text{\record{\tfield{e}{dog($r$.x)}}}} \not=\emptyset]\}$  
\end{ex} 
Since $r$ is required to be of the type
\smallrecord{\smallmfield{x}{$a$}{\textit{Ind}}} we know that $r$.x
must be $a$.  Therefore \preveg{} is identical to \nexteg{}.
\begin{ex} 
$\{a\mid\exists
  r[r$:\smallrecord{\smallmfield{x}{$a$}{\textit{Ind}}} and 
$\down{\text{\record{\tfield{e}{dog($a$)}}}} \not=\emptyset]\}$ 
\end{ex} 
By the definition of record types, a record
\smallrecord{\field{e}{$s$}} is of type
\smallrecord{\smalltfield{e}{dog($a$)}} just in case $s:\text{dog}(a)$.
Therefore this type is non-empty just in case there is such an $s$.
For this reason \preveg{} is the same set as \nexteg{}.
\begin{ex} 
$\{a\mid\exists
  r[r$:\smallrecord{\smallmfield{x}{$a$}{\textit{Ind}}} and 
$\exists s [s:\text{dog}(a)]]\}$ 
\end{ex}
Since $r$ is no longer bound in the second conjunct of \preveg{},
\nexteg{} also defines the same set.
\begin{ex} 
$\{a\mid\exists
  r[r$:\smallrecord{\smallmfield{x}{$a$}{\textit{Ind}}}] and 
$\exists s [s:\text{dog}(a)]\}$ 
\end{ex} 
Given the nature of records, there will be an $r$ of the required type
just in case $a$:\textit{Ind}.  Therefore we can characterize the same set
as in \nexteg{}.
\begin{ex} 
$\{a\mid a$:\textit{Ind} and 
$\exists s [s:\text{dog}(a)]\}$ 
\end{ex} 
Finally, since the existence of a situation of type dog($a$) requires
the $a$ is an individual given that the arity of `dog' is
$\langle\textit{Ind}\rangle$ we can eliminate the first conjunct
altogether so the minimal characterization of this set is \nexteg{}.
\begin{ex} 
$\{a\mid  
\exists s [s:\text{dog}(a)]\}$  
\end{ex} 
       
   
   
  

If $P$ and $Q$ are properties we want exist($P$, $Q$)
to be a type of situations which will be non-empty (that is, ``true'')
just in case the P-extensions of $P$ and $Q$ have a non-empty overlap,
that is there is some individual which has both property $P$ and
property $Q$.  In symbols we can express this as \nexteg{}.
\begin{ex} 
$\down{\mathrm{exist}(P,Q)}\not=\emptyset$ iff
$\downP{P}\cap\downP{Q}\not=\emptyset$ 
\label{ex:constraintexist}
\end{ex} 
This places a requirement on objects which are assigned to the type `exist($P$, $Q$)'
without actually tying down what kind of object they have to be.  That
is, it leaves it open as to which objects get assigned to the type, as
long as they respect this requirement.  It places a constraint on $F$
in the models discussed on p.~\pageref{pg:models}.  We can, however,
go a step further and make precise exactly which objects these should
be.  The intution is that a situation $e$ should be of type exist($P$,
$Q$) just in case it is a witness (or ``proof'') of the fact that the
``exist''-relation holds between $P$ and $Q$ (that is, that the
P-extensions of $P$ and $Q$ have a non-empty overlap). We will say
that a situation is such a witness just in case the P-extensions of
the properties restricted to the situation in question stand in the
required relation, that is, intuitively that the set of objects in the
situation which have $P$ overlaps with the set of objects in the
situation which have $Q$.  We will get at this notion by restricting
properties to a particular situation (what we have called a
resource situation in previous literature such as
\citealp{BarwisePerry1983,Cooper1996a}). We will represent the
restriction of property $P$ to situation $s$ as $P\!\restriction\!
s$. We take our previous example of the property of being a dog,
repeated in \nexteg{a}.  Its restriction to the situation $s$ is given
in \nexteg{b}.
\begin{ex} 
\begin{subex} 
 
\item $\lambda r$:\smallrecord{\smalltfield{x}{\textit{Ind}}} . \record{\tfield{e}{dog($r$.x)}} 
 
\item $\lambda r$:\smallrecord{\smalltfield{x}{\textit{Ind}}} . \record{\rfield{e}{$s$}{dog($r$.x)}} 
 
\end{subex} 
\label{ex:restricteddog}   
\end{ex} 
In \preveg{b} the restricted field\label{pg:restfld}
\smallrecord{\smallrfield{e}{$s$}{dog($r$.x)}} requires that the
object in `e'-field is not only of type `dog($r$.x)' but also that it
is either $s$ itself or a component of $s$, that is, for some
path $\pi$ in $s$ it is the object $s.\pi$.\label{pg:component}
\begin{shaded}
The notion \textit{component} is defined in \nexteg{}, repeated in 
Appendix~\ref{app:rec}.
\begin{ex} 
An object, $a$, is a \textit{component} of a record, $r$, in symbols,
$a\varepsilon r$, just in case there is some path, $\pi$, in $r$ such
that $r.\pi=a$. 
\end{ex}

In terms of our informal proof theoretic notation this could be
expressed as \nexteg{}.
\begin{ex} 
\begin{subex} 
 
\item
  \begin{prooftree}
    \hypo{\Gamma\vdash r\text{ record}}
    \hypo{\pi\in\text{paths(r)}}
    \infer2{\Gamma\vdash r.\pi\ \varepsilon\ r}
  \end{prooftree}
 
\item
  \begin{prooftree}
    \hypo{\Gamma\vdash r.\pi\ \varepsilon\ r}
    \infer1{\Gamma\vdash r \text{ record}}
  \end{prooftree}
  \hspace*{2em}
  \begin{prooftree}
    \hypo{\Gamma\vdash r.\pi\ \varepsilon\ r}
    \infer1{\pi\in\text{paths}(r)}
  \end{prooftree}
  
    
 
\end{subex} 
   
\end{ex}

Based on this notion we can define a notion of \textit{present in a
  record} as in \nexteg{}, repeated in 
Appendix~\ref{app:rec}.\label{pg:present}
\begin{ex} 
An object, $a$, is \textit{present} in a record, $r$, in
symbols, $a\underline{\varepsilon} r$, just in case either $a=r$ or
$a\varepsilon r$. 
\end{ex} 
We can express this in our informal proof theoretic notation as in
\nexteg{}.
\begin{ex} 
\begin{subex} 
 
\item
  \begin{prooftree}
    \hypo{\Gamma\vdash r\text{ record}}
    \infer1{\Gamma\vdash r\ \underline{\varepsilon}\ r}
  \end{prooftree}
  \hspace*{2em}
  \begin{prooftree}
    \hypo{\Gamma\vdash a\ \varepsilon\ r}
    \infer1{\Gamma\vdash a\ \underline{\varepsilon}\ r}
  \end{prooftree}
  
 
\item
  \begin{prooftree}
    \hypo{\Gamma\vdash a\ \underline{\varepsilon}\ r}
    \hypo{[\Delta\vdash r'\text{ record}]}
    \ellipsis{}{\Delta\vdash\varphi}
    \hypo{[\Delta\vdash a'\ \varepsilon\ r']}
    \ellipsis{}{\Delta\vdash\varphi}
    \infer3{\Gamma\vdash\varphi}
  \end{prooftree}
  
 
\end{subex} 
   
\end{ex} 
  
We define a general notion of restricting a type by a record as in
\nexteg{}, repeated in Appendix~\ref{app:specrec}.\label{pg:typerestriction}
\begin{ex}
  A type system $\mathbb{T}$ \textit{has restricted types according to a
  set of labels} $\mathcal{L}$ if it is the case that
\begin{enumerate} 
 
\item   if  $T$ is a type according to $\mathbb{T}$ and
  $r$ is a record according to $\mathcal{L}$ and $\mathbb{T}$, then the
  restriction of $T$ by $r$, $T\!\restriction\!r$, is a
type according to $\mathbb{T}$.
 
\item $a:_{\mathbb{T}}T\!\restriction\!r$ iff $a\underline{\varepsilon} r$  and $a:_{\mathbb{T}}T$. 
\label{ex:restrictedtypes} 
\end{enumerate} 
   
\end{ex}

In terms of our informal prooftheoretic notation this can be expressed
as \nexteg{}.
\begin{ex} 
\begin{subex} 
 
\item
  \begin{prooftree}
    \hypo{\Gamma,\mathcal{L}\vdash T\in\textbf{Type}}
    \hypo{\Gamma,\mathcal{L}\vdash r \text{ record}}
    \infer2{\Gamma,\mathcal{L}\vdash
      T\!\restriction\!r\in\textbf{Type}}
  \end{prooftree}
  
 
\item
  \begin{prooftree}
    \hypo{\Gamma,\mathcal{L}\vdash r \text{ record}}
    \hypo{a\ \underline{\varepsilon}\ r}
    \hypo{\Gamma,\mathcal{L}\vdash a:T}
    \infer3{\Gamma,\mathcal{L}\vdash a:T\!\restriction\!r}
  \end{prooftree}
  
  \begin{prooftree}
    \hypo{\Gamma,\mathcal{L}\vdash a:T\!\restriction\!r}
    \infer1{\Gamma,\mathcal{L}\vdash r \text{ record}}
  \end{prooftree}
  \hspace*{2em}
  \begin{prooftree}
    \hypo{\Gamma,\mathcal{L}\vdash a:T\!\restriction\!r}
    \infer1{a\ \underline{\varepsilon}\ r}
  \end{prooftree}
  \hspace*{2em}
  \begin{prooftree}
    \hypo{\Gamma,\mathcal{L}\vdash a:T\!\restriction\!r}
    \infer1{\Gamma,\mathcal{L}\vdash a:T}
  \end{prooftree}
  
 
\end{subex} 
   
\end{ex} 
  
We use a similar notation for restricted fields in record types as we
do for manifest fields.  That is, we represent \nexteg{a} as
\nexteg{b}.
\begin{ex} 
\begin{subex} 
 
\item \record{\tfield{$\ell$}{$T\!\restriction\!r$}} 
 
\item \record{\rfield{$\ell$}{$r$}{$T$}} 
 
\end{subex} 
   
\end{ex}

If $T$ is a record type other than \textit{Rec} and \textit{ERec} then
\textit{the recursive restriction of $T$ by $r$},
$T\!\restriction\restriction\!r$, is the result of replacing each
field in $T$ of the form $[\ell:T']$, where $T'$ is a type, with the
field $[\ell:T'\!\restriction\restriction\!r]$ and each field of the
form $\langle\ell,\langle f,\Pi\rangle\rangle$ (where $f$ is a
function and $\Pi$ a sequence of paths) with $\langle\ell,\langle
f',\Pi\rangle\rangle$ where $f'$ is like $f$ except that for any
arguments $a_1,\ldots,a_n$ appropriate to $f$,
$f'(a_1)\ldots(a_n)=f(a_1)\ldots(a_n)\!\restriction\restriction r$.

Otherwise if $T$ is a type, then
$T\!\restriction\restriction\!r=T\!\restriction\!r$.

If $\mathcal{T}$ is a dependent type (for example, a property)
$\lambda r_1\!:\!T_1\ldots\lambda r_n\!:\!T_n\ .\
T\dep{r_1,\ldots,r_n}$, then $\mathcal{T}\!\restriction\!s$ is $\lambda r_1\!:\!T_1\ldots\lambda r_n\!:\!T_n\ .\
T\dep{r_1,\ldots,r_n}\!\restriction\restriction\!s$.

\end{shaded}

% A definition of restriction for properties in general is given in
% Appendix~\ref{app:gramrulesuniv}.  We will not be concerned with the
% general definition here.  

Now we can compute the property extension of
\preveg{b} in a similar fashion to the calculation for the
non-restricted property.   
The property extension of \preveg{b} is given in \nexteg{}.\label{pg:set-reduction-gq}
\begin{ex} 
$\{a\mid\exists
  r[r$:\smallrecord{\smallmfield{x}{$a$}{\textit{Ind}}} and \\
\hspace*{2em}$\down{\lambda
  r:\text{\smallrecord{\smalltfield{x}{\textit{Ind}}}}
  . \text{\record{\rfield{e}{$s$}{dog($r$.x)}}}(r)} \not=\emptyset]\}$ 
\end{ex} 
By $\beta$-reduction \preveg{} is the same set as \nexteg{}.
\begin{ex} 
$\{a\mid\exists
  r[r$:\smallrecord{\smallmfield{x}{$a$}{\textit{Ind}}} and 
$\down{\text{\record{\rfield{e}{$s$}{dog($r$.x)}}}} \not=\emptyset]\}$  
\end{ex} 
Since $r$ is required to be of the type
\smallrecord{\smallmfield{x}{$a$}{\textit{Ind}}} we know that $r$.x
must be $a$.  Therefore \preveg{} is identical to \nexteg{}.
\begin{ex} 
$\{a\mid\exists
  r[r$:\smallrecord{\smallmfield{x}{$a$}{\textit{Ind}}} and 
$\down{\text{\record{\rfield{e}{$s$}{dog($a$)}}}} \not=\emptyset]\}$ 
\end{ex} 
By the definition of record types, a record
\smallrecord{\field{e}{$s'$}} is of type
\smallrecord{\smallrfield{e}{$s$}{dog($a$)}} just in case $s'\underline{\varepsilon}s$ and
$s':\text{dog}(a)$.
Therefore this type is non-empty just in case there is some $s'$ such
that $s'\underline{\varepsilon}s$ and $s':\text{dog}(a)$.
For this reason \preveg{} is the same set as \nexteg{}.
\begin{ex} 
$\{a\mid\exists
  r[r$:\smallrecord{\smallmfield{x}{$a$}{\textit{Ind}}} and 
$\exists s'[s'\underline{\varepsilon}s \text{ and } s':\text{dog}(a)]\}$ 
\end{ex}
Since $r$ is no longer bound in the second conjunct of \preveg{},
\nexteg{} also defines the same set.
\begin{ex} 
$\{a\mid\exists
  r[r$:\smallrecord{\smallmfield{x}{$a$}{\textit{Ind}}}] and 
$\exists s'[s'\underline{\varepsilon}s \text{ and } s':\text{dog}(a)\}$ 
\end{ex} 
Given the nature of records, there will be an $r$ of the required type
just in case $a$:\textit{Ind}.  Therefore we can characterize the same set
as in \nexteg{}.
\begin{ex} 
$\{a\mid a$:\textit{Ind} and 
$\exists s'[s'\underline{\varepsilon}s \text{ and } s':\text{dog}(a)\}$ 
\end{ex} 
Finally, since the existence of a situation of type dog($a$) requires
the $a$ is an individual given that the arity of `dog' is
$\langle\textit{Ind}\rangle$ we can eliminate the first conjunct
altogether so the minimal characterization of this set is \nexteg{}.
\begin{ex} 
$\{a\mid  
\exists s'[s'\underline{\varepsilon}s \text{ and } s':\text{dog}(a)\}$  
\end{ex} 

Now we can use the notion of property restriction to characterize the
witness condition for ptypes constructed with `exist', as in
\nexteg{}.
\begin{ex} 
$e$ : exist($P$,$Q$) iff $\downP{P}\cap\downP{Q\!\restriction\! e}\not=\emptyset$ 
\label{ex:existwitcond}
\end{ex} 
This will have the consequence that any record of the type \nexteg{a}
will be of type \nexteg{b}.
\begin{ex} 
\begin{subex} 
 
\item \record{\tfield{x}{\textit{Ind}}\\
              \tfield{c}{dog(x)}\\
              \tfield{e}{run(x)}}
 
\item exist($\lambda r$:\smallrecord{\smalltfield{x}{\textit{Ind}}}
    . \smallrecord{\smalltfield{e}{dog($r$.x)}},
            $\lambda r$:\smallrecord{\smalltfield{x}{\textit{Ind}}} . \smallrecord{\smalltfield{e}{run($r$.x)}})
 
\end{subex} 
\label{ex:adogruns}
   
\end{ex} 
In other words, \preveg{a} is a subtype of \preveg{b}.  We abbreviate
the two properties in \preveg{b} as `dog$'$' and `run$'$'
respectively.  Consider an
arbitrary record, $r$, of type \preveg{a} as given in \nexteg{}.
\begin{ex} 
$r$ = \record{\field{x}{$a$}\\
        \field{c}{$s_1$}\\
        \field{e}{$s_2$}\\
        \ldots}

\medskip

where $s_1$ : dog($a$) and $s_2$ : run($a$)
\end{ex} 
Given $r$ we know that $a$ must be a member of both
$\downP{\text{dog}'}$ and $\downP{\text{run}'\restriction r}$ and that
therefore $r$ must be of type (\ref{ex:adogruns}b).  Therefore
(\ref{ex:adogruns}a)$\sqsubseteq$(\ref{ex:adogruns}b).  The argument does
not go the other way,
however: (\ref{ex:adogruns}b)$\not\sqsubseteq$(\ref{ex:adogruns}a).  Consider
the situation $r$ in \nexteg{}.
\begin{ex} 
$r$ = \record{\field{e}{$s_1$}}

\medskip

where there is some situation $s\not=r$ such that $s$ : dog($a$) and $s_1$ : run($a$) 
\end{ex} 
In \preveg{}, $r$:(\ref{ex:adogruns}b) but $r$ is not of type
(\ref{ex:adogruns}a), since $r$ does not ``contain the information'' that
$a$ is a dog.  

% The two relevant property extensions to be considered will both be the
% set $\{a\}$ and thus the conditions for \preveg{} being of type
% (\ref{ex:adogruns}b) will be fulfilled since the two property
% extensions have a non-empty overlap.   


% Another
% intuition is that a witness for the type would be the pair $\langle
% P,Q\rangle$ just in case the
% ``exist''-relation holds between $P$ and $Q$.  This is reminiscent of
% the way in which $\Sigma$-types are treated in constructive type
% theory (as discussed on p.~\pageref{pg:sigmatypes}).  There is a way
% of combining these two intuitions.  We model situations as records and
% we can model a pair as a record with two fields containing the
% respective members of the pair.  Thus we can think of the pair as a
% situation which has the two properties in appropriate roles.  Our
% definition is given in \nexteg{}.\footnote{Note that this runs
%   dangerously close to Russell's paradox.  In
%   Appendix~\ref{app:comptypes} we say that ptypes are labelled
%   sets.  The ptype we represent as exist($P$,$Q$) would be the
%   labelled set
%   $\{\langle\mathrm{pred},\mathrm{exist}\rangle,\langle\mathrm{arg}_1,P\rangle,\langle\mathrm{arg}_2,Q\rangle\}$.
%   Records are also labelled sets.  The only thing that stops this
%   labelled set from being a record and therefore according to
%   \nexteg{} allowing for the possibility that exist($P$,$Q$) :
%   exist($P$,$Q$) is that predicates are not technically assigned to a
%   type according to our definitions.  They are considered as type
%   constructors but not themselves objects of a type.  There are a
%   number of other ways to avoid the problem if this way were to be
%   deemed unacceptable for some reason.  We just flag here, that
%   however one constructs ptypes and the objects that are of ptypes, we
% should probably avoid allowing ptypes to be of themselves.  Doing so,
% would not only run the risk of paradox, but would also not make much
% intuitive sense.}
% \begin{ex} 
% $e$ : exist($P$, $Q$) iff $e$ :
% \smallrecord{\smallmfield{arg$_1$}{$P$}{\textit{Ppty}} \\
%              \smallmfield{arg$_2$}{$Q$}{\textit{Ppty}}} and $\downP{P}\cap\downP{Q}\not=\emptyset$ 
% \end{ex} 
Let us consider what we get when we apply the content we have for \textit{a
conductor}, \nexteg{a} (repeated from (\ref{ex:a-conductor}c)), to the property of composing, \nexteg{b}.  The
result which would correspond to \textit{a conductor composes} if we
were to introduce \textit{composes} as an intransitive verb in our
resources, is given in
\nexteg{c}.
\begin{ex} 
\begin{subex} 
 
\item  $\lambda P$:\textit{Ppty}
. \record{\mfield{restr}{$\lambda
r$:\smallrecord{\smalltfield{x}{\textit{Ind}}}
. \record{\tfield{e}{conductor($r$.x)}}}{\textit{Ppty}} \\
          \mfield{scope}{$P$}{\textit{Ppty}} \\
          \tfield{e}{exist(restr, scope)}}
 
\item $\lambda
r$:\smallrecord{\smalltfield{x}{\textit{Ind}}}
. \record{\tfield{e}{compose($r$.x)}}

\item \record{\mfield{restr}{$\lambda
r$:\smallrecord{\smalltfield{x}{\textit{Ind}}}
. \record{\tfield{e}{conductor($r$.x)}}}{\textit{Ppty}} \\
          \mfield{scope}{$\lambda
r$:\smallrecord{\smalltfield{x}{\textit{Ind}}}
. \record{\tfield{e}{compose($r$.x)}}}{\textit{Ppty}} \\
          \tfield{e}{exist(restr, scope)}} 
 
\end{subex} 
   
\end{ex} 
What would it mean for there to be something of type \preveg{c}? In
other words, what would be required to make the sentence \textit{a
  conductor composes} true? There would have to be a record, $r^*$, which
contains the three fields in the record in \nexteg{} and which meets
the condition indicated.
\begin{ex} 
$r^*$ = \record{\field{restr}{$\lambda
                      r$:\smallrecord{\smalltfield{x}{\textit{Ind}}}
                        . \record{\tfield{e}{conductor($r$.x)}}} \\
          \field{scope}{$\lambda
r$:\smallrecord{\smalltfield{x}{\textit{Ind}}}
. \record{\tfield{e}{compose($r$.x)}}} \\
          \field{e}{$s$}}

where $\downP{r^*.\mathrm{restr}\!\!\restriction\!\!s}$ and $\downP{r^*.\mathrm{scope}\!\!\restriction\!\!s}$
                        have a non-empty overlap.    
\end{ex} 
% Given the availability of appropriate labels and predicates, the type
% theory will guarantee that a record of the form in \preveg{} will
% exist, but it will not necessarily guarantee that the condition is
% met.  This will depend on what is assigned to the basic types and
% ptype, that is, it will depend on the model.

This gives us a version of the classical treatment of indefinite
articles as involving the existential quantifier, expressed in terms
of a generalized quantifier which compares sets.  There is, of course,
a real and important question whether this is an appropriate content
for the sentence \textit{a conductor composes} which tends to get a
generic reading something like ``conductors, in general, compose''.
We will return to this issue in Chapter~\ref{ch:quant} where we will
deal with the indefinite article in more detail.  For now, we will
ignore the sentence \textit{a conductor composes} since we are not
considering syntactic resources for it anyway.  

We are concerned
with finding a way to interpret the verb phrase \textit{is a
  conductor}.  Can we find a content for \textit{is} which could be
combined with the content for \textit{a conductor} given in
(\ref{ex:a-conductor}c) to produce an appropriate interpretation for
the verb-phrase?  Montague's (\citeyear{Montague1973}) strategy for
assigning a content to \textit{is} is reproduced in our terms in
\nexteg{}.
\begin{ex} 
$\lambda\mathcal{Q}$:\textit{Quant} . \\
\hspace*{1em} $\lambda r_1$:\smallrecord{\tfield{x}{\textit{Ind}}}
. \\
\hspace*{2em} $\mathcal{Q}$($\lambda
r_2$:\smallrecord{\tfield{x}{\textit{Ind}}}
. \record{\mfield{x}{$r_2$.x, $r_1$.x}{\textit{Ind}}\\
          \tfield{e}{be(x)}})
\label{ex:cont-is} 
\end{ex}



Here we use a manifest field based on a multiple singleton type (a
singleton type formed from a singleton type, see
Chapter~\ref{ch:infex}, p.~\pageref{pg:multiple-singleton-types}) to require
the identity of $r_1$.x and $r_2$.x.  In the `e'-field of the type
with the manifest field we use the predicate 'be' which we will take
to be polymorphic with the set of arities  as given in \nexteg{a}.  The witness
condition associated with types constructed with `be' is given in
\nexteg{b}.
\begin{ex} 
\begin{subex} 
 
\item arity(be) = \{$\langle T\rangle\mid T$ is a type\} 
 
\item $e:\text{be}(a)$ iff $a\varepsilon e$ 
 
\end{subex} 
   
\end{ex} 
The intuition behind \preveg{b} could be expressed as ``To be is to be
a component of a situation'', that is, more technically, $a$ ``is''
just in case there is a path, $\pi$, 
in some record, $r$, such that $r.\pi=a$.\footnote{This might be
  compared with Quine's (\citeyear{Quine1948}) dictum:  ``To be is to
  be the value of a variable''.}   
  
% take `=' to be a predicate (used in infix notation, that is
% $a=b$ to mean $=(a,b)$) with arity
% $\langle$\textit{Ind},\textit{Ind}$\rangle$.\footnote{Actually, we
%   would want `=' to be polymorphic and assign it the set of arities
%   \{$T\in$ \textbf{Type} $\mid$ $\langle T,T\rangle$\}}  The
% conditions for being of this type could be expressed as \nexteg{}.
% \begin{ex} 
% $e$ : $a=b$ iff $e$ :
% \smallrecord{\smallmfield{arg$_1$}{$a$}{\textit{Ind}} \\
%              \smallmfield{arg$_2$}{$b$}{\textit{Ind}}} and $a$ is
%            identical with $b$ 
% \end{ex}
We will call (\ref{ex:cont-is}) `SemBe'.  It will be included among the
universal resources, together with the `Lex$_{\mathrm{be}}$' as defined in
\nexteg{}.
\begin{ex} 
If $T_{\mathrm{Phon}}$ is a phonological type, then
Lex$_{\mathrm{be}}$($T_{\mathrm{Phon}}$) is Lex($T_{\mathrm{Phon}}$,
\textit{V}) \d{$\wedge$} \smallrecord{\smallmfield{cont}{SemBe}{(\textit{Quant}$\rightarrow$\textit{Ppty})}} 
\end{ex} 
Among the lexical resources for English we have Lex$_{\mathrm{be}}$(``is'').
   
Now let us see what we get when we combine  (\ref{ex:cont-is}) with the content of
\textit{a conductor}.  This involves applying (\ref{ex:cont-is}),
repeated as \nexteg{a}, to (\ref{ex:a-conductor}c), repeated as
\nexteg{b}.  The result of this application is \nexteg{c}.
\begin{ex} 
\begin{subex} 
 
\item $\lambda\mathcal{Q}$:\textit{Quant} . \\
\hspace*{1em} $\lambda r_1$:\smallrecord{\tfield{x}{\textit{Ind}}}
. \\
\hspace*{2em} $\mathcal{Q}$($\lambda
r_2$:\smallrecord{\tfield{x}{\textit{Ind}}}
. \record{\mfield{x}{$r_2$.x, $r_1$.x}{\textit{Ind}} \\
          \tfield{e}{be(x)}}) 
 
\item $\lambda P$:\textit{Ppty}
. \record{\mfield{restr}{$\lambda
r$:\smallrecord{\smalltfield{x}{\textit{Ind}}}
. \record{\tfield{e}{conductor($r$.x)}}}{\textit{Ppty}} \\
          \mfield{scope}{$P$}{\textit{Ppty}} \\
          \tfield{e}{exist(restr, scope)}}

\item $\lambda
r_1$:\smallrecord{\tfield{x}{\textit{Ind}}}
. \\
\hspace*{1em}\record{\mfield{restr}{$\lambda
r$:\smallrecord{\smalltfield{x}{\textit{Ind}}}
. \record{\tfield{e}{conductor($r$.x)}}}{\textit{Ppty}} \\
          \mfield{scope}{$\lambda
r_2$:\smallrecord{\tfield{x}{\textit{Ind}}}
. \record{\mfield{x}{$r_2$.x, $r_1$.x}{\textit{Ind}} \\
          \tfield{e}{be(x)}}}{\textit{Ppty}} \\
          \tfield{e}{exist(restr, scope)}} 
 
\end{subex} 
   
\end{ex} 

In order to obtain a content for \textit{Dudamel is a conductor} we
apply the content of \textit{Dudamel}, (\ref{ex:diac}a), repeated as
\nexteg{a}, to  \preveg{c}, repeated as \nexteg{b}, with result
\nexteg{c}.
\begin{ex} 
\begin{subex} 
 
\item $\lambda P$:\textit{Ppty} . $P$(\smallrecord{\field{x}{$d$}})
 
\item $\lambda
r_1$:\smallrecord{\tfield{x}{\textit{Ind}}}
. \\
\hspace*{1em}\record{\mfield{restr}{$\lambda
r$:\smallrecord{\smalltfield{x}{\textit{Ind}}}
. \record{\tfield{e}{conductor($r$.x)}}}{\textit{Ppty}} \\
          \mfield{scope}{$\lambda
r_2$:\smallrecord{\tfield{x}{\textit{Ind}}}
. \record{\mfield{x}{$r_2$.x, $r_1$.x}{\textit{Ind}} \\
          \tfield{e}{be(x)}}}{\textit{Ppty}} \\
          \tfield{e}{exist(restr, scope)}} 

\item  \record{\mfield{restr}{$\lambda
r$:\smallrecord{\smalltfield{x}{\textit{Ind}}}
. \record{\tfield{e}{conductor($r$.x)}}}{\textit{Ppty}} \\
          \mfield{scope}{$\lambda
r_2$:\smallrecord{\tfield{x}{\textit{Ind}}}
. \record{\mfield{x}{$r_2$.x, $d$}{\textit{Ind}} \\
          \tfield{e}{be(x)}}}{\textit{Ppty}} \\
          \tfield{e}{exist(restr, scope)}}
 
\end{subex} 
\label{ex:diac1}   
\end{ex} 

The type \preveg{c} is distinct from the type (\ref{ex:diac}f),
repeated as \nexteg{}, which we obtained by applying the content of
\textit{Dudamel} directly to the content of \textit{conductor}.
\begin{ex} 
\record{\tfield{e}{conductor($d$)}}
\label{ex:diac-simple} 
\end{ex} 
There is, however, an equivalence that holds between (\ref{ex:diac1}c)
and \preveg{}.  The equivalence is not that they share the same set of
witnesses.  We can characterize the set of witnesses of
(\ref{ex:diac1}c) and \nexteg{a} and the witnesses of \preveg{} as
\nexteg{b}.
\begin{ex} 
\begin{subex} 
 
\item  \{\record{\field{restr}{$P$} \\
          \field{scope}{$Q$} \\
          \field{e}{$s$}} $\mid$ \begin{tabular}{l}
$P$ = $\lambda
                      r$:\smallrecord{\smalltfield{x}{\textit{Ind}}}
                        . \record{\tfield{e}{conductor($r$.x)}} \\
                        and 
$Q$ = $\lambda
r$:\smallrecord{\smalltfield{x}{\textit{Ind}}}
. \record{\mfield{x}{$r$.x, $d$}{\textit{Ind}} \\
          \tfield{e}{be(x)}} \\ and  
$\downP{P\!\!\restriction\!\!s}\cap\downP{Q\!\!\restriction\!\!s}\not=\emptyset$\end{tabular}\}
 
\item \{\record{\field{e}{$s$}} $\mid$ $s$ : conductor($d$)\} 
 
\end{subex} 
   
\end{ex} 
The sets in \preveg{} do not have any members in common.  The
equivalence is a weaker ``truth-conditional'' equivalence.
(\ref{ex:diac1}c) has a witness (``is true'') if and only if
(\ref{ex:diac-simple}) has a witness.  This is because the
P-extensions of the property of being a conductor and the property of
being identical with Dudamel can have a non-empty overlap if and only
if Dudamel is a conductor.  We might try to characterize the
difference between the property associated with \textit{conductor} and
the property associated with \textit{is a conductor} as ``the property
of being an $x$ such that conductor($x$)'' and ``the property of being
an $x$ such that there is a $y$ such that conductor($y$) and $y=x$''.
The two are truth-conditionally equivalent and for this reason in
Montague's system they turn out to be the same property.  For us,
since we are taking a more intensional approach than Montague, they
are distinct properties but they are nevertheless truth-conditionally
equivalent.

Since we have two distinct properties, the question is raised whether
the property that is associated with the verb-phrase should be the
same as the property associated with the common noun or whether it
should be the property proposed here involving existential
quantification. One way to do this is to create a type corresponding
to the tree in \nexteg{}.
\begin{ex} 

\Tree [.\textit{VP}\\$\gamma$ 
         [.\textit{V}\\$\alpha$ ``is'' ]
         [.\textit{NP}\\$\beta(\gamma)$ 
               [.\textit{Det}\\$\beta$ ``a'' ]
               \textit{N}\\$\gamma$ ] ] 



\end{ex} 
This is not compositional in the standard sense because the content of
the verb phrase is not defined as some operation applied to the
contents of the verb and the noun phrase, but rather it makes the
content of the verb phrase be the content of the noun.  Furthermore,
it requires the verb and determiner utterances be of the specific
types ``is'' and ``a'' respectively.  This gives \preveg{} the flavour
of representing a construction type as discussed in a variety of
approaches to Construction Grammar 
\citep[see, for example,][]{BoasSag2012}.  We can allow the type corresponding to
\preveg{} by introducing the update function \nexteg{}.
\begin{ex} 
$\lambda
u$:\textit{V}\d{$\wedge$}\smallrecord{\smalltfield{s-event}{\smallrecord{\smalltfield{e}{``is''}}}}$^{\frown}$\textit{NP}\d{$\wedge$}\smallrecord{\smalltfield{syn}{\smallrecord{\smalltfield{daughters}{\textit{Det}\d{$\wedge$}\smallrecord{\smalltfield{s-event}{\smallrecord{\smalltfield{e}{``a''}}}} \\
                                                                    \hspace*{5em}$^{\frown}$
\textit{N}\d{$\wedge$}\smallrecord{\smalltfield{cont}{\textit{Ppty}}}}
}}}. \\
\hspace*{1em} \textit{VP}\d{$\wedge$}\smallrecord{\smallmfield{cont}{$u$[2].syn.daughters[2].cont}{\textit{Ppty}}} 
\end{ex} 
We can call this function CnstrIsA (``is-a construction'') and merge
it with \textit{VP} $\longrightarrow$ \textit{V} \textit{NP}.   Thus
one of the resources available for English is \nexteg{}.
\begin{ex} 
\textit{VP} $\longrightarrow$ \textit{V} \textit{NP} \d{\d{$\wedge$}} CnstrIsA 
\label{ex:ruleIsA}
\end{ex} 
This suggests that a phrase structure and construction based
approach can be combined within a single framework.  Since we are
working with a toy fragment where the only verb is \textit{is} and the
only determiner is \textit{a}, we can make do with \preveg{} as the
only resource for assigning content to verb-phrases.  In a more
general grammar we would, of course, require in addition a rule that
applies the content of the verb to the content of the object
noun-phrase as in \nexteg{}.
\begin{ex} 
\textit{VP} $\longrightarrow$ \textit{V} \textit{NP} \d{\d{$\wedge$}}
ContForwardApp(\textit{Quant}, \textit{Ppty})  
\label{ex:VPForwardApp}
\end{ex} 
Allowing both resources (\ref{ex:ruleIsA}) and \preveg{} simultaneously raises the issue of what the
relationship should be between them.  Should the more specific rule
(\ref{ex:ruleIsA}) take precedence and guarantee that the only content
associated with the verb phrase \textit{is a conductor} is the
property which is the content of \textit{conductor}?  Or should the
verb phrase be ambiguous between this interpretation and the property
obtained by applying the content of \textit{is} to the content of
\textit{a conductor}?  

A more pressing issue, perhaps, is what to do about the sentence in
\nexteg{}.
\begin{ex} 
\#A conductor is Dudamel
\label{ex:acid} 
\end{ex} 
We have used the marking `\#' in \preveg{} to indicate that an
utterance of this sentence would under most, if not all, circumstances
be considered to be odd, though it is difficult to rule it out as
ungrammatical, particularly if we are to use something corresponding
to context-free phrase structure rules as we are.  The oddness of
\preveg{} may have something to do with the tendency to interpret noun
phrases with indefinite articles in subject position as generic as in
\nexteg{}.
\begin{ex} 
A conductor is a high-ranking individual in the musical hierarchy 
\end{ex} 
(\ref{ex:acid}) can be improved without becoming generic.  Examples
are given in \nexteg{}.
\begin{ex} 
\begin{subex} 
 
\item A conductor to reckon with is Dudamel 
 
\item A conductor to consider is Dudamel

\item A conductor who impresses me as a leader in his generation is
  Dudamel

\item A conductor I would like to see more often in Gothenburg is Dudamel 
 
\end{subex} 
   
\end{ex} 
This raises a lot of issues which we do not currently have tools to
deal with.  There is, however, something we can say, if we choose to
allow the is-a construction interpretation of \textit{is a conductor}.
The content of the sentence \textit{Dudamel is a conductor} on the
construction analysis becomes (\ref{ex:diac-simple}), repeated as
\nexteg{a}, rather than (\ref{ex:diac1}c), repeated as \nexteg{b}.
\begin{ex} 
\begin{subex} 
 
\item \record{\tfield{e}{conductor($d$)}} 
 
\item \record{\mfield{restr}{$\lambda
r$:\smallrecord{\smalltfield{x}{\textit{Ind}}}
. \record{\tfield{e}{conductor($r$.x)}}}{\textit{Ppty}} \\
          \mfield{scope}{$\lambda
r_2$:\smallrecord{\tfield{x}{\textit{Ind}}}
. \record{\mfield{x}{$r_2$.x, $d$}{\textit{Ind}} \\
          \tfield{e}{be(x)}}}{\textit{Ppty}} \\
          \tfield{e}{exist(restr, scope)}} 

\end{subex} 
\label{ex:diacTwice}    
\end{ex} 
If we include the resource (\ref{ex:VPForwardApp}) then the
content of \textit{is Dudamel} is \nexteg{a} applied to \nexteg{b},
that is, \nexteg{c}.
\begin{ex} 
\begin{subex} 
 
\item $\lambda\mathcal{Q}$:\textit{Quant} . \\
\hspace*{1em} $\lambda r_1$:\smallrecord{\tfield{x}{\textit{Ind}}}
. \\
\hspace*{2em} $\mathcal{Q}$($\lambda
r_2$:\smallrecord{\tfield{x}{\textit{Ind}}}
. \record{\mfield{x}{$r_2$.x, $r_1$.x}{\textit{Ind}} \\
          \tfield{e}{be(x)}})  
 
\item $\lambda P$:\textit{Ppty} . $P$(\smallrecord{\field{x}{$d$}})

\item  $\lambda r_1$:\smallrecord{\tfield{x}{\textit{Ind}}}
. \record{\mfield{x}{$d$, $r_1$.x}{\textit{Ind}} \\
          \tfield{e}{be(x)}})
 
\end{subex} 
   
\end{ex}
The content of \textit{A conductor is Dudamel} is \nexteg{a} applied
to \nexteg{b} (identical with \preveg{c}), which is \nexteg{c}.
\begin{ex} 
\begin{subex} 
 
\item $\lambda P$:\textit{Ppty}
. \record{\mfield{restr}{$\lambda
r$:\smallrecord{\smalltfield{x}{\textit{Ind}}}
. \record{\tfield{e}{conductor($r$.x)}}}{\textit{Ppty}} \\
          \mfield{scope}{$P$}{\textit{Ppty}} \\
          \tfield{e}{exist(restr, scope)}} 
 
\item $\lambda r_1$:\smallrecord{\tfield{x}{\textit{Ind}}}
. \record{\mfield{x}{$d$, $r_1$.x}{\textit{Ind}}})

\item \record{\mfield{restr}{$\lambda
r$:\smallrecord{\smalltfield{x}{\textit{Ind}}}
. \record{\tfield{e}{conductor($r$.x)}}}{\textit{Ppty}} \\
          \mfield{scope}{$\lambda r_1$:\smallrecord{\tfield{x}{\textit{Ind}}}
. \record{\mfield{x}{$d$, $r_1$.x}{\textit{Ind}} \\
          \tfield{e}{be(x)}})}{\textit{Ppty}} \\
          \tfield{e}{exist(restr, scope)}} 
 
\end{subex} 
   
\end{ex} 
\preveg{c} is almost exactly the same type as (\ref{ex:diacTwice}b).
The difference between them is that $d$ is the first restrictor of
\textit{Ind} in \preveg{c} whereas in (\ref{ex:diacTwice}b) it is
the second restrictor.   In \preveg{c} we have, for some conductor,
$c$, \textit{Ind}$_{d,c}$ whereas in (\ref{ex:diacTwice}b) we have
\textit{Ind}$_{c,d}$.  Thus while an analysis that only uses the content of
\textit{is} that is based on Montague's original interpretation does
predict different contents for \textit{Dudamel is a conductor} and
\textit{a conductor is Dudamel}, the difference between the types
hardly seems enough to explain the difference in reaction we have to
the two sentences.  Given the construction analysis for
\textit{Dudamel is a conductor} we get a markedly different type
(\ref{ex:diacTwice}a) which
does not involve existential quantification (even though it is truth
conditionally equivalent to both the types with existential
quantification). The only way that  (\ref{ex:diacTwice}a) can be
expressed according to the resources that we have developed in this
chapter is by the sentence \textit{Dudamel is a conductor}, using the
non-compositional construction `CnstrIsA'.  Thus if
(\ref{ex:diacTwice}a) is the target content and we do not wish to
express a content involving existential quantification, \textit{a
  conductor is Dudamel} is not an option.  

We thus have the beginnings of an explanation of the difference in
acceptability between the two sentences.  It is not the whole story
since we have not explained why the quantificational readings appear
odd in these cases.  Note that the distinction we are making between a
non-quantified reading and a reading involving an existential
quantification is not available on Montague's \citeyear{Montague1973}
original approach since the fact that the two contents are
truth-conditionally equivalent means for Montague that they are
identical.  The same holds for the kind of analysis discussed in
\cite{Partee1986} where even though the content may not be built up
using existential quantification the final result is still the same
content that would be expressed by using existential quantification
because of the truth-conditional equivalence.  One might try to
introduce the distinction we are making by relating utterances to an expression
in an artificial logical language in addition to the content.  This
would correspond to the notion of logical form as discussed for
example by \cite{HeimKratzer1998} and much current work in linguistic
semantics.  The idea might be that there are two distinct logical
forms such as \nexteg{} which correspond to identical contents in
Montague's terms.
\begin{ex} 
\begin{subex} 
 
\item conductor(dudamel) 
 
\item $\exists x$ [conductor($x$) $\wedge$ $x$=dudamel] 
 
\end{subex} 
   
\end{ex} 
Here the challenge would be to give an explanatory account of why one
expression in an artificial language, \preveg{a}, should be preferred
over another, \preveg{b}, when they both express the same content.  An
alternative is to follow \cite{Lewis1972} (further developed by
\citealp{Cresswell1985}). The idea here is that we keep a record not
only of the final content but the way in which that content is
constructed -- that is we keep a record of the content of each of the syntactic
constituents of the English sentence and the way these contents are
combined.  This idea, which goes back to the notion of intensional
isomorphism introduced by \cite{Carnap1956}, provides enough structure
to make the distinction required here.  However, there are other
problems with the proposal which we will take up in
Chapter~\ref{ch:intensional} when we discuss intensionality. 

%@@Do we really want to say that ``ok'' and ``aha'' don't have content?

Since acknowledgements like \textit{aha} and
\textit{ok} do not have a specified content (\textit{cf.} the function
`sign$_{\mathit{uc}}$' we used for these words in
Chapter~\ref{ch:infex}), we do not need a function that specifies
their content but can make do with the function `Lex' which associates
them with a sign type in which the content is unspecified.  Their
content in a given dialogue is given by the update functions and
action rules.

\section{Incremental processing and building a chart type}
\label{sec:chart}

% In Chapter~\ref{ch:infex} we made the simplifying assumption that a
% chart was a sign.  Now we have a grammar we need to complicate this
% picture.

If we are to take the idea that language is action seriously, we must
take account of the fact that speech events happen in real time.
People understand what is being said during the time that the
utterance is produced.  If somebody stops talking mid-sentence we are
often able to complete the sentence for them or respond to what we
have heard so far.  There are many good accounts in the literature of
why linguistic processing in \textit{incremental} in this way.  A
couple of good overviews of the arguments are
\cite{GinzburgPoesio2016} and
\cite{KempsonCannGregoromichelakiChatzikyriakidis2016}.

It may seem that recreating phrase structure rules and their
compositional semantic interpretation in terms of a theory of action
is not the best route to building a theory of the incremental nature
of language processing.  However, drawing this conclusion would be
over-hasty.  In the remainder of this chapter we will sketch a view of
incrementality based on the computational technique of chart parsing
which was prevalent in computational linguistics in the 1980's and
1990's.  Despite the fact that it has fallen into disuse in current
data-driven computational linguistics it still has significance for a
theory of language as action.

% We will present here a version of chart parsing as it is
% used in computational linguistics.
For a textbook introduction
to chart parsing see \cite{JurafskyMartin2009}, Chap.~13.  The idea of
a chart is that it should store all the hypotheses that we make during
the processing of an utterance and allow us to compute new hypotheses
to be added to the chart
on the basis of what is already present in the chart.  We will say
that a chart is a record and we will use our resources to compute a
chart type on the basis of utterance events.  We will first go through
an example of the incremental construction of a chart type for an
agent processing an utterance of the sentence \textit{Dudamel is a
  conductor}.  Then we will consider what kind of update functions and
action rules are
needed in order to achieve this.  We will, as usual, make the
simplifying assumption that what we have at bottom is a string of word
utterances as we are not dealing with the details of phonology.  Thus
we are giving a simplified view of incremental processing at the word
level.

Suppose that we have so far heard an utterance of the word
\textit{Dudamel}.  At this point we will say that the type of the
chart is \nexteg{}.
\begin{ex} 
\record{\tfield{e$_1$}{``Dudamel''} \\
        \tfield{e}{\smallrecord{\smalltfield{e$_1$}{start($\Uparrow^2$e$_1$)}}$^{\frown}$\smallrecord{\smalltfield{e$_1$}{end($\Uparrow^2$e$_1$)}}}} 
\label{ex:Dudamelchart}
\end{ex} 
The main event of the chart type (represented by the `e'-field) breaks
the phonological event of type ``Dudamel'' down into a string of two
events, the start and the end of the ``Dudamel''-event.\footnote{These
  starting and ending events correspond to what are standardly called
  \textit{vertices} in the chart parsing literature.}  

Why are the arguments to `start' and `end' in the string type prefixed
by `$\Uparrow^2$'?  Recall from the discussion on
p.~\pageref{pg:stringsasrecords} that a string of type \nexteg{a} will
be a record of type \nexteg{b}.  
\begin{ex} 
\begin{subex} 
 
\item  \smallrecord{\smalltfield{e$_1$}{$T_1$}}$^{\frown}$\smallrecord{\smalltfield{e$_1$}{$T_2$}}
 
\item
  \record{\tfield{t$_0$}{\smallrecord{\smalltfield{e$_1$}{$T_1$}}}\\
          \tfield{t$_1$}{\smallrecord{\smalltfield{e$_1$}{$T_2$}}}}
 
\end{subex} 
 

   
\end{ex} 
Thus a record of type (\ref{ex:Dudamelchart}) will be of the type
\nexteg{}.
\begin{ex} 
\record{\tfield{e$_1$}{``Dudamel''} \\
        \tfield{e}{\smallrecord{\smalltfield{t$_0$}{\smallrecord{\smalltfield{e$_1$}{start($\Uparrow^2$e$_1$)}}}\\
                                \smalltfield{t$_1$}{\smallrecord{\smalltfield{e$_1$}{end($\Uparrow^2$e$_1$)}}}}}} 
\end{ex} 
Thus the arguments to the `start' and `end' predicates are to be found
two levels up.  

Thus (\ref{ex:Dudamelchart})
records that we have observed an event of the phonological type
``Dudamel'' and an event consisting of the start of that event
followed by the end of that event.  Given that we have the resource
the resource Lex$_{\mathrm{PropName}}$(``Dudamel'', $d$) available,
% (see Appendix~\ref{app:gramrules})
 we can update \preveg{} to
\nexteg{}.
\begin{ex} 
\record{\tfield{e$_1$}{``Dudamel''} \\
        \tfield{e$_2$}{Lex$_{\mathrm{PropName}}$(``Dudamel'', $d$)
          \d{$\wedge$} \smallrecord{\smalltfield{s-event}{\smallrecord{\smallmfield{e}{e$_1$}{\textit{Phon}}}
         }
        }} \\
        \tfield{e}{\smallrecord{\smalltfield{e$_1$}{start($\Uparrow^2$e$_1$)} \\
                                \smalltfield{e$_2$}{start($\Uparrow^2$e$_2$)}}$^{\frown}$\smallrecord{\smalltfield{e$_1$}{end($\Uparrow^2$e$_1$)}
                              \\
         \smalltfield{e$_2$}{end($\Uparrow^2$e$_2$)}}}} 
\end{ex} 
That is, we add the information to the chart that there is an event
(labelled `e$_2$') of
the type which is the sign type corresponding to ``Dudamel'' and that
the event which is the speech event referred to in that sign type is
the utterance event, labelled by `e$_1$'.  Furthermore the duration of
the event labelled `e$_2$' is the same as that labelled `e$_1$'.  One
could discuss where there are two events which are contemporaneous or whether there is a single utterance event
which is of both types.  The fact that we have presented two fields
labelled `e$_1$' and `e$_2$' does not of itself prevent the two fields
containing the same event.  However, the fact that we have analyzed
the sign as containing the speech event as a part (corresponding to
the basic intuition that signs are pairings of utterances and
contents) decides the issue for us.  A sign is a record (a labelled
set) which models a situation and we are not allowing sets to be members of themselves.  Thus
records cannot be a part of themselves.\footnote{`e$_1$' and `e$_2$'
  correspond to what are known as \textit{passive edges} in the chart
  parsing literature.  They represent information about potential
  constituents that have been found.}   

The type Lex$_{\mathrm{PropName}}$(``Dudamel'', $d$) is a subtype of
\textit{NP}.  Thus the event labelled `e$_2$' could be the first item
in a string that would be appropriate for the function which we have
abbreviated as \nexteg{a} % (see Appendix~\ref{app:gramrules})
which has the type \nexteg{b}.
\begin{ex} 
\begin{subex} 
 
\item \textit{S} $\longrightarrow$ \textit{NP} \textit{VP} $\mid$
\textit{NP}$'$(\textit{VP}$'$) 
 
\item (\textit{NP}$^{\frown}$\textit{VP} $\rightarrow$ \textit{Type}) 
 
\end{subex} 
   
\end{ex} 
Thus in a way that is similar to the prediction by the dog in
Chapter~\ref{ch:percint} that it should run after the stick which is
held up and the
kind of event that this will contribute to is a game of fetch. Similarly, on
observing a noun-phrase event we can predict that it might be followed
by a verb phrase event thus creating a sentence event.  We will add a
hypothesis event to our chart which takes place at the end of the
noun-phrase event as in \nexteg{}.\footnote{In terms of the traditional chart parsing
  terminology this corresponds to an \textit{active edge} involving a
  \textit{dotted rule}.  The fact that the addition of this type to
  the chart type is triggered by finding something of an appropriate
  type to be the leftmost element in a string the would be an
  appropriate argument to the rule corresponds to what is called a
  \textit{left-corner} parsing strategy.} 
\begin{ex} 
\record{\tfield{e$_1$}{``Dudamel''} \\
        \tfield{e$_2$}{Lex$_{\mathrm{PropName}}$(``Dudamel'', $d$)
          \d{$\wedge$} \smallrecord{\smalltfield{s-event}{\smallrecord{\smallmfield{e}{$\Uparrow^2$e$_1$}{\textit{Phon}}}
         }
        }} \\
       \tfield{e$_3$}{\smallrecord{\smallmfield{rule}{\textit{S} $\longrightarrow$ \textit{NP} \textit{VP} $\mid$
\textit{NP}$'$(\textit{VP}$'$)}{(\textit{NP}$^{\frown}$\textit{VP}
$\rightarrow$ \textit{Type})} \\
                                   \smallmfield{fnd}{$\Uparrow$e$_2$}{\textit{Sign}}
                                   \\
                                   \smallmfield{req}{\textit{VP}}{\textit{Type}}
                                     \\
                                   \smalltfield{e}{required(req,rule)}}} \\  
        \tfield{e}{\smallrecord{\smalltfield{e$_1$}{start($\Uparrow^2$e$_1$)} \\
                                \smalltfield{e$_2$}{start($\Uparrow^2$e$_2$)}}$^{\frown}$\smallrecord{\smalltfield{e$_1$}{end($\Uparrow^2$e$_1$)}
                              \\
         \smalltfield{e$_2$}{end($\Uparrow^2$e$_2$)} \\
         \smalltfield{e$_3$}{start($\Uparrow^3$e$_3$)$^{\frown}$end($\Uparrow^3$e$_3$)}}}}  
 \end{ex}


In the e$_3$-field the  `rule'-field is for a syntactic rule, that is,
a function from a string of signs of a given type to a type.  The
`fnd'-field is for a sign or string of signs so far found which match an initial
segment of a string of the type required by the rule.  The `req'-field
is the type of the remaining string required to satisfy the rule as
expressed in the `e'-field.  This hypothesis event both starts and
ends at the end of the event of the noun-phrase event
e$_2$.\footnote{With respect to the word string event labelled by `e',
  it is a \textit{punctual} event.} 

We can now progress to the next word in the input string as shown in
\nexteg{}.
\begin{ex} 
 \record{\tfield{e$_1$}{``Dudamel''} \\
        \tfield{e$_2$}{Lex$_{\mathrm{PropName}}$(``Dudamel'', $d$)
          \d{$\wedge$} \smallrecord{\smalltfield{s-event}{\smallrecord{\smallmfield{e}{$\Uparrow^2$e$_1$}{\textit{Phon}}}
         }
        }} \\
       \tfield{e$_3$}{\smallrecord{\smallmfield{rule}{\textit{S} $\longrightarrow$ \textit{NP} \textit{VP} $\mid$
\textit{NP}$'$(\textit{VP}$'$)}{(\textit{NP}$^{\frown}$\textit{VP}
$\rightarrow$ \textit{Type})} \\
                                   \smallmfield{fnd}{$\Uparrow$e$_2$}{\textit{Sign}}
                                   \\
                                   \smallmfield{req}{\textit{VP}}{\textit{Type}}
                                     \\
                                   \smalltfield{e}{required(req,rule)}}}
                               \\
\tfield{e$_4$}{``is''} \\  
        \tfield{e}{\smallrecord{\smalltfield{e$_1$}{start($\Uparrow^2$e$_1$)} \\
                                \smalltfield{e$_2$}{start($\Uparrow^2$e$_2$)}}$^{\frown}$\smallrecord{\smalltfield{e$_1$}{end($\Uparrow^2$e$_1$)}
                              \\
         \smalltfield{e$_2$}{end($\Uparrow^2$e$_2$)} \\
         \smalltfield{e$_3$}{start($\Uparrow^3$e$_3$)$^{\frown}$end($\Uparrow^3$e$_3$)} \\
         \smalltfield{e$_4$}{start($\Uparrow^2$e$_4$)}}$^{\frown}$\smallrecord{\smalltfield{e$_4$}{end($\Uparrow^2$e$_4$)}}}}   
\end{ex} 
Note that the start of the ``is''-event is aligned with the end of
``Dudamel''-event.  This allows for the fact that there is no break
between the words and that the exact pronunciation of the final /l/ in
``Dudamel'' is influenced by the pronuniciation of the initial /i/ in
``is'' through coarticulation.\footnote{It also means that the number
  of elements in the string labelled `e' is the same as the number of
  vertices in a standard chart.}

We can now go through similar procedures as we did for
\textit{Dudamel} adding both a lexical event based on our lexical
resources and a hypothesis event based on the only rule for
strings beginning with a $V$ that we have in our resources. The result
of these two steps is given in \nexteg{}.
\begin{figure}
\begin{ex} 
 \record{\tfield{e$_1$}{``Dudamel''} \\
        \tfield{e$_2$}{Lex$_{\mathrm{PropName}}$(``Dudamel'', $d$)
          \d{$\wedge$} \smallrecord{\smalltfield{s-event}{\smallrecord{\smallmfield{e}{$\Uparrow^2$e$_1$}{\textit{Phon}}}
         }
        }} \\
       \tfield{e$_3$}{\smallrecord{\smallmfield{rule}{\textit{S} $\longrightarrow$ \textit{NP} \textit{VP} $\mid$
\textit{NP}$'$(\textit{VP}$'$)}{(\textit{NP}$^{\frown}$\textit{VP}
$\rightarrow$ \textit{Type})} \\
                                   \smallmfield{fnd}{$\Uparrow$e$_2$}{\textit{Sign}}
                                   \\
                                   \smallmfield{req}{\textit{VP}}{\textit{Type}}
                                     \\
                                   \smalltfield{e}{required(req,rule)}}}
                               \\
\tfield{e$_4$}{``is''} \\
\tfield{e$_5$}{Lex$_{\mathrm{be}}$(``is'')\d{$\wedge$}\smallrecord{\smalltfield{s-event}{\smallrecord{\smallmfield{e}{$\Uparrow^2$e$_4$}{\textit{Phon}}}
         }
        }} \\
\tfield{e$_6$}{\smallrecord{\smallmfield{rule}{\textit{VP} $\longrightarrow$ \lb{\textit{V}}{``is''}
\lb{\textit{NP}}{\lb{\textit{Det}}{``a''} \textit{N}} $\mid$
\textit{N}$'$}{\\ 
\hspace*{9.5em}(\textit{V}\d{$\wedge$}\smallrecord{\smalltfield{s-event}{\smallrecord{\smalltfield{e}{``is''}}}}$^{\frown}$\\
\hspace*{10em}\textit{NP}\d{$\wedge$}\smallrecord{\smalltfield{syn}{\smallrecord{\smalltfield{daughters}{\textit{Det}\d{$\wedge$}\smallrecord{\smalltfield{s-event}{\smallrecord{\smalltfield{e}{``a''}}}} \\
                                                                    \hspace*{5em}$^{\frown}$
\textit{N}\d{$\wedge$}\smallrecord{\smalltfield{cont}{\textit{Ppty}}}}
}}} \\
\hspace*{12em}$\rightarrow$ \textit{Type})} \\
                                   \smallmfield{fnd}{$\Uparrow$e$_5$}{\textit{Sign}}
                                   \\
                                   \smallmfield{req}{\textit{NP}}{\textit{Type}}
                                     \\
                                   \smalltfield{e}{required(req,rule)}}}
                               \\  
        \tfield{e}{\smallrecord{\smalltfield{e$_1$}{start($\Uparrow^2$e$_1$)} \\
                                \smalltfield{e$_2$}{start($\Uparrow^2$e$_2$)}}$^{\frown}$\smallrecord{\smalltfield{e$_1$}{end($\Uparrow^2$e$_1$)}
                              \\
         \smalltfield{e$_2$}{end($\Uparrow^2$e$_2$)} \\
         \smalltfield{e$_3$}{start($\Uparrow^3$e$_3$)$^{\frown}$end($\Uparrow^3$e$_3$)} \\
         \smalltfield{e$_4$}{start($\Uparrow^2$e$_4$)} \\
         \smalltfield{e$_5$}{start($\Uparrow^2$e$_5$)}}$^{\frown}$\smallrecord{\smalltfield{e$_4$}{end($\Uparrow^2$e$_4$)}\\
                                                    \smalltfield{e$_5$}{end($\Uparrow^2$e$_5$)}\\
                                                  \smalltfield{e$_6$}{start($\Uparrow^3$e$_6$)$^{\frown}$end($\Uparrow^3$e$_6$)}}}}  
                                          \end{ex}
                                        \end{figure}
                                        
Now we can add \textit{a} and \textit{conductor} in a similar way with
the result shown in \nexteg{}.
\begin{figure}
\begin{ex} 
 \record{\tfield{e$_1$}{``Dudamel''} \\
        \tfield{e$_2$}{Lex$_{\mathrm{PropName}}$(``Dudamel'', $d$)
          \d{$\wedge$} \smallrecord{\smalltfield{s-event}{\smallrecord{\smallmfield{e}{$\Uparrow^2$e$_1$}{\textit{Phon}}}
         }
        }} \\
       \tfield{e$_3$}{\smallrecord{\smallmfield{rule}{\textit{S} $\longrightarrow$ \textit{NP} \textit{VP} $\mid$
\textit{NP}$'$(\textit{VP}$'$)}{(\textit{NP}$^{\frown}$\textit{VP}
$\rightarrow$ \textit{Type})} \\
                                   \smallmfield{fnd}{$\Uparrow$e$_2$}{\textit{Sign}}
                                   \\
                                   \smallmfield{req}{\textit{VP}}{\textit{Type}}
                                     \\
                                   \smalltfield{e}{required(req,rule)}}}
                               \\
\tfield{e$_4$}{``is''} \\
\tfield{e$_5$}{Lex$_{\mathrm{be}}$(``is'')\d{$\wedge$}\smallrecord{\smalltfield{s-event}{\smallrecord{\smallmfield{e}{$\Uparrow^2$e$_4$}{\textit{Phon}}}
         }
        }} \\
\tfield{e$_6$}{\smallrecord{\smallmfield{rule}{\textit{VP} $\longrightarrow$ \lb{\textit{V}}{``is''}
\lb{\textit{NP}}{\lb{\textit{Det}}{``a''} \textit{N}} $\mid$
\textit{N}$'$}{\\ 
\hspace*{9.5em}(\textit{V}\d{$\wedge$}\smallrecord{\smalltfield{s-event}{\smallrecord{\smalltfield{e}{``is''}}}}$^{\frown}$\\
\hspace*{10em}\textit{NP}\d{$\wedge$}\smallrecord{\smalltfield{syn}{\smallrecord{\smalltfield{daughters}{\textit{Det}\d{$\wedge$}\smallrecord{\smalltfield{s-event}{\smallrecord{\smalltfield{e}{``a''}}}} \\
                                                                    \hspace*{5em}$^{\frown}$
\textit{N}\d{$\wedge$}\smallrecord{\smalltfield{cont}{\textit{Ppty}}}}
}}} \\
\hspace*{12em}$\rightarrow$ \textit{Type})} \\
                                   \smallmfield{fnd}{$\Uparrow$e$_5$}{\textit{Sign}}
                                   \\
                                   \smallmfield{req}{\textit{NP}}{\textit{Type}}
                                     \\
                                   \smalltfield{e}{required(req,rule)}}}
                               \\
\tfield{e$_7$}{``a''} \\
\tfield{e$_8$}{Lex$_{\mathrm{IndefArt}}$(``a'') }\d{$\wedge$}\smallrecord{\smalltfield{s-event}{\smallrecord{\smallmfield{e}{$\Uparrow^2$e$_7$}{\textit{Phon}}}
         }
        } \\ 
\tfield{e$_9$}{\smallrecord{\smallmfield{rule}{\textit{NP} $\longrightarrow$ \textit{Det} \textit{N} $\mid$
\textit{Det}$'$(\textit{N}$'$)}{(\textit{Det}$^{\frown}$\textit{N}
$\rightarrow$ \textit{Type})} \\
                                   \smallmfield{fnd}{$\Uparrow$e$_8$}{\textit{Sign}}
                                   \\
                                   \smallmfield{req}{\textit{N}}{\textit{Type}}
                                     \\
                                   \smalltfield{e}{required(req,rule)}}}
                               \\
\tfield{e$_{10}$}{``conductor''}\\ 
\tfield{e$_{11}$}{Lex$_{\mathrm{CommonNoun}}$(``conductor'', conductor)\d{$\wedge$}\smallrecord{\smalltfield{s-event}{\smallrecord{\smallmfield{e}{$\Uparrow^2$e$_{10}$}{\textit{Phon}}}
         }
        }}\\
        \tfield{e}{\smallrecord{\smalltfield{e$_1$}{start($\Uparrow^2$e$_1$)} \\
                                \smalltfield{e$_2$}{start($\Uparrow^2$e$_2$)}}$^{\frown}$\smallrecord{\smalltfield{e$_1$}{end($\Uparrow^2$e$_1$)}
                              \\
         \smalltfield{e$_2$}{end($\Uparrow^2$e$_2$)} \\
         \smalltfield{e$_3$}{start($\Uparrow^3$e$_3$)$^{\frown}$end($\Uparrow^3$e$_3$)} \\
         \smalltfield{e$_4$}{start($\Uparrow^2$e$_4$)} \\
         \smalltfield{e$_5$}{start($\Uparrow^2$e$_5$)}}$^{\frown}$\smallrecord{\smalltfield{e$_4$}{end($\Uparrow^2$e$_4$)}\\
                                                    \smalltfield{e$_5$}{end($\Uparrow^2$e$_5$)}\\
                                                  \smalltfield{e$_6$}{start($\Uparrow^3$e$_6$)$^{\frown}$end($\Uparrow^3$e$_6$)}\\
                                                  \smalltfield{e$_7$}{start($\Uparrow^2$e$_7$)}\\
                                                  \smalltfield{e$_8$}{start($\Uparrow^2$e$_8$)}}$^{\frown}$\\
& & \hspace{2em}\smallrecord{\smalltfield{e$_7$}{end($\Uparrow^2$e$_7$)}\\
                             \smalltfield{e$_8$}{end($\Uparrow^2$e$_8$)}\\
                             \smalltfield{e$_9$}{start($\Uparrow^3$e$_9$)$^{\frown}$end($\Uparrow^3$e$_9$)}\\
                             \smalltfield{e$_{10}$}{start($\Uparrow^2$e$_{10}$)}\\
                             \smalltfield{e$_{11}$}{start($\Uparrow^2$e$_{11}$)}}$^{\frown}$\smallrecord{\smalltfield{e$_{10}$}{end($\Uparrow^2$e$_{10}$)}\\
                                                                                           \smalltfield{e$_{11}$}{end($\Uparrow^2$e$_{11}$)}}}}  
                                                                                   \end{ex}
                                                                                 \end{figure}
                                                                                 
Note that there is no possibility of adding a hypothesis event based
on the utterance of \textit{conductor} given the resources we have
since our small grammar does not include a phrase structure rule for
strings whose first element is of type \textit{N}.  However, now for
the first time we have found something which fulfills one of our
hypotheses.  The hypothesis event labelled `e$_9$' has the type
\textit{N} in its `req'-field.  The event labelled `e$_{11}$' is
required to be of a subtype of \textit{N} and thus fulfils the
requirement of `e$_9$'.  Furthermore, the start of e$_{11}$ is aligned
with the end (and also the start) of `e$_9$'.  This means that we can
update the chart-type by adding a new field for an event of the type
returned by applying `e$_9$.rule' (a function) to the string
e$_9$.fnd$^{\frown}$e$_{11}$.  The start of this new \textit{NP}-event
will be aligned with the start of e$_9$.fnd (that is, e$_8$).  The end
of the new event is aligned with the end of e$_{11}$.  The resulting
chart-type is given in \nexteg{}.
\begin{figure}
\begin{ex} 
 \record{\tfield{e$_1$}{``Dudamel''} \\
        \tfield{e$_2$}{Lex$_{\mathrm{PropName}}$(``Dudamel'', $d$)
          \d{$\wedge$} \smallrecord{\smalltfield{s-event}{\smallrecord{\smallmfield{e}{$\Uparrow^2$e$_1$}{\textit{Phon}}}
         }
        }} \\
       \tfield{e$_3$}{\smallrecord{\smallmfield{rule}{\textit{S} $\longrightarrow$ \textit{NP} \textit{VP} $\mid$
\textit{NP}$'$(\textit{VP}$'$)}{(\textit{NP}$^{\frown}$\textit{VP}
$\rightarrow$ \textit{Type})} \\
                                   \smallmfield{fnd}{$\Uparrow$e$_2$}{\textit{Sign}}
                                   \\
                                   \smallmfield{req}{\textit{VP}}{\textit{Type}}
                                     \\
                                   \smalltfield{e}{required(req,rule)}}}
                               \\
\tfield{e$_4$}{``is''} \\
\tfield{e$_5$}{Lex$_{\mathrm{be}}$(``is'')\d{$\wedge$}\smallrecord{\smalltfield{s-event}{\smallrecord{\smallmfield{e}{$\Uparrow^2$e$_4$}{\textit{Phon}}}
         }
        }} \\
\tfield{e$_6$}{\smallrecord{\smallmfield{rule}{\textit{VP} $\longrightarrow$ \lb{\textit{V}}{``is''}
\lb{\textit{NP}}{\lb{\textit{Det}}{``a''} \textit{N}} $\mid$
\textit{N}$'$}{\\ 
\hspace*{9.5em}(\textit{V}\d{$\wedge$}\smallrecord{\smalltfield{s-event}{\smallrecord{\smalltfield{e}{``is''}}}}$^{\frown}$\\
\hspace*{10em}\textit{NP}\d{$\wedge$}\smallrecord{\smalltfield{syn}{\smallrecord{\smalltfield{daughters}{\textit{Det}\d{$\wedge$}\smallrecord{\smalltfield{s-event}{\smallrecord{\smalltfield{e}{``a''}}}} \\
                                                                    \hspace*{5em}$^{\frown}$
\textit{N}\d{$\wedge$}\smallrecord{\smalltfield{cont}{\textit{Ppty}}}}
}}} \\
\hspace*{12em}$\rightarrow$ \textit{Type})} \\
                                   \smallmfield{fnd}{$\Uparrow$e$_5$}{\textit{Sign}}
                                   \\
                                   \smallmfield{req}{\textit{NP}}{\textit{Type}}
                                     \\
                                   \smalltfield{e}{required(req,rule)}}}
                               \\
\tfield{e$_7$}{``a''} \\
\tfield{e$_8$}{Lex$_{\mathrm{IndefArt}}$(``a'') }\d{$\wedge$}\smallrecord{\smalltfield{s-event}{\smallrecord{\smallmfield{e}{$\Uparrow^2$e$_7$}{\textit{Phon}}}
         }
        } \\ 
\tfield{e$_9$}{\smallrecord{\smallmfield{rule}{\textit{NP} $\longrightarrow$ \textit{Det} \textit{N} $\mid$
\textit{Det}$'$(\textit{N}$'$)}{(\textit{Det}$^{\frown}$\textit{N}
$\rightarrow$ \textit{Type})} \\
                                   \smallmfield{fnd}{$\Uparrow$e$_8$}{\textit{Sign}}
                                   \\
                                   \smallmfield{req}{\textit{N}}{\textit{Type}}
                                     \\
                                   \smalltfield{e}{required(req,rule)}}}
                               \\
\tfield{e$_{10}$}{``conductor''}\\ 
\tfield{e$_{11}$}{Lex$_{\mathrm{CommonNoun}}$(``conductor'', conductor)\d{$\wedge$}\smallrecord{\smalltfield{s-event}{\smallrecord{\smallmfield{e}{$\Uparrow^2$e$_{10}$}{\textit{Phon}}}
         }
        }}\\
\tfield{e$_{12}$}{e$_9$.rule(e$_9$.fnd$^{\frown}$e$_{11}$)}\\
        \tfield{e}{\smallrecord{\smalltfield{e$_1$}{start($\Uparrow^2$e$_1$)} \\
                                \smalltfield{e$_2$}{start($\Uparrow^2$e$_2$)}}$^{\frown}$\smallrecord{\smalltfield{e$_1$}{end($\Uparrow^2$e$_1$)}
                              \\
         \smalltfield{e$_2$}{end($\Uparrow^2$e$_2$)} \\
         \smalltfield{e$_3$}{start($\Uparrow^3$e$_3$)$^{\frown}$end($\Uparrow^3$e$_3$)} \\
         \smalltfield{e$_4$}{start($\Uparrow^2$e$_4$)} \\
         \smalltfield{e$_5$}{start($\Uparrow^2$e$_5$)}}$^{\frown}$\smallrecord{\smalltfield{e$_4$}{end($\Uparrow^2$e$_4$)}\\
                                                    \smalltfield{e$_5$}{end($\Uparrow^2$e$_5$)}\\
                                                  \smalltfield{e$_6$}{start($\Uparrow^3$e$_6$)$^{\frown}$end($\Uparrow^3$e$_6$)}\\
                                                  \smalltfield{e$_7$}{start($\Uparrow^2$e$_7$)}\\
                                                  \smalltfield{e$_8$}{start($\Uparrow^2$e$_8$)}\\
                                                  \smalltfield{e$_{12}$}{start($\Uparrow^2$e$_{12}$)}}$^{\frown}$\\
& & \hspace{2em}\smallrecord{\smalltfield{e$_7$}{end($\Uparrow^2$e$_7$)}\\
                             \smalltfield{e$_8$}{end($\Uparrow^2$e$_8$)}\\
                             \smalltfield{e$_9$}{start($\Uparrow^3$e$_9$)$^{\frown}$end($\Uparrow^3$e$_9$)}\\
                             \smalltfield{e$_{10}$}{start($\Uparrow^2$e$_{10}$)}\\
                             \smalltfield{e$_{11}$}{start($\Uparrow^2$e$_{11}$)}}$^{\frown}$\smallrecord{\smalltfield{e$_{10}$}{end($\Uparrow^2$e$_{10}$)}\\
                                                                                           \smalltfield{e$_{11}$}{end($\Uparrow^2$e$_{11}$)}\\
                                                                                           \smalltfield{e$_{12}$}{end($\Uparrow^2$e$_{12}$)}}}}  
                                                                                   \end{ex}
                                                                                 \end{figure}
                                                                                 

The event labelled `e$_{12}$' will be of type \textit{NP} and thus
satisfy the requirement of e$_6$.  By carrying out the same procedure
as before we will obtain a new event (labelled `e$_{13}$') of type
\textit{VP} which will satisfy the requirement of `e$_3$' which will
allow us to add a new event (labelled `e$_{14}$') of type \textit{S}
whose start is at the beginning of the string labelled `e' and whose
end is at the end of that string.  The final chart type is given in
\nexteg{}.
\begin{figure}
\begin{ex} 
 \record{\tfield{e$_1$}{``Dudamel''} \\
        \tfield{e$_2$}{Lex$_{\mathrm{PropName}}$(``Dudamel'', $d$)
          \d{$\wedge$} \smallrecord{\smalltfield{s-event}{\smallrecord{\smallmfield{e}{$\Uparrow^2$e$_1$}{\textit{Phon}}}
         }
        }} \\
       \tfield{e$_3$}{\smallrecord{\smallmfield{rule}{\textit{S} $\longrightarrow$ \textit{NP} \textit{VP} $\mid$
\textit{NP}$'$(\textit{VP}$'$)}{(\textit{NP}$^{\frown}$\textit{VP}
$\rightarrow$ \textit{Type})} \\
                                   \smallmfield{fnd}{$\Uparrow$e$_2$}{\textit{Sign}}
                                   \\
                                   \smallmfield{req}{\textit{VP}}{\textit{Type}}
                                     \\
                                   \smalltfield{e}{required(req,rule)}}}
                               \\
\tfield{e$_4$}{``is''} \\
\tfield{e$_5$}{Lex$_{\mathrm{be}}$(``is'')\d{$\wedge$}\smallrecord{\smalltfield{s-event}{\smallrecord{\smallmfield{e}{$\Uparrow^2$e$_4$}{\textit{Phon}}}
         }
        }} \\
\tfield{e$_6$}{\smallrecord{\smallmfield{rule}{\textit{VP} $\longrightarrow$ \lb{\textit{V}}{``is''}
\lb{\textit{NP}}{\lb{\textit{Det}}{``a''} \textit{N}} $\mid$
\textit{N}$'$}{\\ 
\hspace*{9.5em}(\textit{V}\d{$\wedge$}\smallrecord{\smalltfield{s-event}{\smallrecord{\smalltfield{e}{``is''}}}}$^{\frown}$\\
\hspace*{10em}\textit{NP}\d{$\wedge$}\smallrecord{\smalltfield{syn}{\smallrecord{\smalltfield{daughters}{\textit{Det}\d{$\wedge$}\smallrecord{\smalltfield{s-event}{\smallrecord{\smalltfield{e}{``a''}}}} \\
                                                                    \hspace*{5em}$^{\frown}$
\textit{N}\d{$\wedge$}\smallrecord{\smalltfield{cont}{\textit{Ppty}}}}
}}} \\
\hspace*{12em}$\rightarrow$ \textit{Type})} \\
                                   \smallmfield{fnd}{$\Uparrow$e$_5$}{\textit{Sign}}
                                   \\
                                   \smallmfield{req}{\textit{NP}}{\textit{Type}}
                                     \\
                                   \smalltfield{e}{required(req,rule)}}}
                               \\
\tfield{e$_7$}{``a''} \\
\tfield{e$_8$}{Lex$_{\mathrm{IndefArt}}$(``a'') }\d{$\wedge$}\smallrecord{\smalltfield{s-event}{\smallrecord{\smallmfield{e}{$\Uparrow^2$e$_7$}{\textit{Phon}}}
         }
        } \\ 
\tfield{e$_9$}{\smallrecord{\smallmfield{rule}{\textit{NP} $\longrightarrow$ \textit{Det} \textit{N} $\mid$
\textit{Det}$'$(\textit{N}$'$)}{(\textit{Det}$^{\frown}$\textit{N}
$\rightarrow$ \textit{Type})} \\
                                   \smallmfield{fnd}{$\Uparrow$e$_8$}{\textit{Sign}}
                                   \\
                                   \smallmfield{req}{\textit{N}}{\textit{Type}}
                                     \\
                                   \smalltfield{e}{required(req,rule)}}}
                               \\
\tfield{e$_{10}$}{``conductor''}\\ 
\tfield{e$_{11}$}{Lex$_{\mathrm{CommonNoun}}$(``conductor'', conductor)\d{$\wedge$}\smallrecord{\smalltfield{s-event}{\smallrecord{\smallmfield{e}{$\Uparrow^2$e$_{10}$}{\textit{Phon}}}
         }
        }}\\
\tfield{e$_{12}$}{e$_9$.rule(e$_9$.fnd$^{\frown}$e$_{11}$)}\\
\tfield{e$_{13}$}{e$_6$.rule(e$_6$.fnd$^{\frown}$e$_{12}$)}\\
\tfield{e$_{14}$}{e$_3$.rule(e$_3$.fnd$^{\frown}$e$_{13}$)}\\
        \tfield{e}{\smallrecord{\smalltfield{e$_1$}{start($\Uparrow^2$e$_1$)} \\
                                \smalltfield{e$_2$}{start($\Uparrow^2$e$_2$)}\\
                                \smalltfield{e$_{14}$}{start($\Uparrow^2$e$_{14}$)}}$^{\frown}$\smallrecord{\smalltfield{e$_1$}{end($\Uparrow^2$e$_1$)}
                              \\
         \smalltfield{e$_2$}{end($\Uparrow^2$e$_2$)} \\
         \smalltfield{e$_3$}{start($\Uparrow^3$e$_3$)$^{\frown}$end($\Uparrow^3$e$_3$)} \\
         \smalltfield{e$_4$}{start($\Uparrow^2$e$_4$)} \\
         \smalltfield{e$_5$}{start($\Uparrow^2$e$_5$)}\\
         \smalltfield{e$_{13}$}{start($\Uparrow^2$e$_{13}$)}}$^{\frown}$\smallrecord{\smalltfield{e$_4$}{end($\Uparrow^2$e$_4$)}\\
                                                    \smalltfield{e$_5$}{end($\Uparrow^2$e$_5$)}\\
                                                  \smalltfield{e$_6$}{start($\Uparrow^3$e$_6$)$^{\frown}$end($\Uparrow^3$e$_6$)}\\
                                                  \smalltfield{e$_7$}{start($\Uparrow^2$e$_7$)}\\
                                                  \smalltfield{e$_8$}{start($\Uparrow^2$e$_8$)}\\
                                                  \smalltfield{e$_{12}$}{start($\Uparrow^2$e$_{12}$)}}$^{\frown}$\\
& & \hspace{2em}\smallrecord{\smalltfield{e$_7$}{end($\Uparrow^2$e$_7$)}\\
                             \smalltfield{e$_8$}{end($\Uparrow^2$e$_8$)}\\
                             \smalltfield{e$_9$}{start($\Uparrow^3$e$_9$)$^{\frown}$end($\Uparrow^3$e$_9$)}\\
                             \smalltfield{e$_{10}$}{start($\Uparrow^2$e$_{10}$)}\\
                             \smalltfield{e$_{11}$}{start($\Uparrow^2$e$_{11}$)}}$^{\frown}$\smallrecord{\smalltfield{e$_{10}$}{end($\Uparrow^2$e$_{10}$)}\\
                                                                                           \smalltfield{e$_{11}$}{end($\Uparrow^2$e$_{11}$)}\\
                                                                                           \smalltfield{e$_{12}$}{end($\Uparrow^2$e$_{12}$)}\\
                                                                                           \smalltfield{e$_{13}$}{end($\Uparrow^2$e$_{13}$)}\\
                                                                                           \smalltfield{e$_{14}$}{end($\Uparrow^2$e$_{14}$)}}}}  
                                                                                   \end{ex}
                                                                                 \end{figure}
                                                                                \clearpage 

We now need to turn our attention to the action rules that will
achieve this building of the chart type.  We will introduce a field
`current-utterance' into the field `shared' on the gameboard.  This
field will be used for the incremental construction of a chart during
the course of an utterance.  % We will not at this point include a
% `move'-field here but reserve that for the field `latest-utterance',
% though one could, of course, consider an alternative with incremental hypotheses about
% moves that have or about to be made formed on the basis of the
% utterance so far.  Here, however, we will restrict ourselves to the
% mechanisms involved in the construction of the chart.  We will add a
% field to the gameboard `shared.current-utterance' which will be used
% to store the chart during the course of processing an utterance.
The
new type \textit{InfoState} (which is a modification of the
\textit{InfoState} as defined in Chapter~\ref{ch:infex},
example~(\ref{ex:InfoStatePrelim}), p.~\pageref{ex:InfoStatePrelim}) is given in \nexteg{}.
\begin{ex} 
% \smallrecord{\smalltfield{private}{\smallrecord{\smalltfield{agenda}{[\textit{RecType}]}}} \\
%         \smalltfield{shared}{\smallrecord{\smalltfield{latest-utterance}{\smallrecord{\smalltfield{move}{\textit{Move}} \\
%                                                                   \smalltfield{chart}{\textit{RecType}}
%                                                                 \\
%                                                                   \smalltfield{e}{m-interp(chart,move)}}$\vee$\textit{ERec}}\\
%                                \smalltfield{current-utterance}{\smallrecord{\smalltfield{chart}{\textit{RecType}}}} \\                               
%                                \smalltfield{commitments}{\textit{RecType}}}}} 
\record{
    \tfield{private}{
      \record{
        \tfield{agenda}{$\mathrm{list}(\textit{RecType})$}}} \\
    \tfield{shared}{
      \record{
        \tfield{latest-utterance}{\textit{Sign}$^*$}\\
        \tfield{current-utterance}{\textit{RecType}}\\
        \tfield{commitments}{\textit{RecType}}}}}
  \label{ex:InfoState-chgram}
\end{ex}
The initial type \textit{InitInfoState} is now \nexteg{}.
\begin{ex} 
% \smallrecord{\smalltfield{private}{\smallrecord{\smallmfield{agenda}{[]}{[\textit{RecType}]}}} \\
%         \smalltfield{shared}{\smallrecord{\smalltfield{latest-utterance}{\textit{ERec}}\\
% \smalltfield{current-utterance}{\smallrecord{\smallmfield{chart}{\textit{Rec}}{\textit{RecType}}}} \\
% \smallmfield{commitments}{\textit{Rec}}{\textit{RecType}}}}}

  \record{
    \tfield{private}{
      \record{
        \mfield{agenda}{[ ]}{$\mathrm{list}(\textit{RecType})$}}} \\
    \tfield{shared}{
      \record{
        \mfield{latest-utterance}{$\varepsilon$}{\textit{Sign}$^*$}\\
        \mfield{current-utterance}{\textit{ERec}}\textit{RecType}\\
        \mfield{commitments}{\textit{Rec}}{\textit{RecType}}}}}
                         
                       \end{ex}

We first characterize an update function, f$_{\textsc{IntegLex}}$, and
action rule, \textsc{IntegLex}, for integrating lexical events into
the chart.  In order to do this we will need to introduce two new
technical notions.  First we will need to find the maximum $i$ such
`e$_i$' is among the labels in the current chart type, so that we can
make sure that what we add to the chart type will have the label
`e$_{i+1}$'.  If $X$ is a labelled set, we say that $\mathrm{max}_\ell(X)$ is
the number defined in \nexteg{}.
\begin{ex} 
$\mathrm{max}(\{i\mid\ell_i\in\mathrm{labels}(X)\})$ 
\end{ex} 
For convenience we abbreviate \nexteg{a} as \nexteg{b}, where
`$\mathrm{incr}$' stands for ``increment''.
\begin{ex} 
\begin{subex} 
 
\item $\mathrm{max}_\ell(X)+1$ 
 
\item $\mathrm{incr}_\ell(X)$ 
 
\end{subex} 
   
\end{ex} 

Secondly, we introduce an operation of \textit{concatenative merge},
`\d{$\wedge$}$_{\text{concat}}$', which is exactly like ordinary merge
except in the case of two string types where $T_1$
\d{$\wedge$}$_{\text{concat}}$ $T_2$ is $T_1^\frown T_2$.

We can now define `f$_{\textsc{IntegLex}}$' as \nexteg{a} and
`\textsc{IntegLex}' as \nexteg{b}.
\begin{ex} 
\begin{subex} 
 
\item  $\lambda T_{\text{chart}}$:\textit{RecType}\\
  \hspace*{1em}$\lambda T_{\text{phon}}$:\textit{RecType} . \\
  \hspace*{2em} $T_{\text{chart}}$ \d{$\wedge$}$_{\text{concat}}$
  \smallrecord{
    \smalltfield{e$_{\text{incr}_{\text{e}}(T_{\text{chart}})}$}{$T_{\text{phon}}$}\\
    \smalltfield{e}{\smallrecord{
        \smalltfield{e$_{\text{incr}_{\text{e}}(T_{\text{chart}})}$}{start($\Uparrow^2$e$_{\text{incr}_{\text{e}}(T_{\text{chart}})}$)}}$^\frown$\smallrecord{
        \smalltfield{e$_{\text{incr}_{\text{e}}(T_{\text{chart}})}$}{end($\Uparrow^2$e$_{\text{incr}_{\text{e}}(T_{\text{chart}})}$)}}}}
 
\item 
  \begin{prooftree}
    \hypo{\text{Lex}(T_{\text{phon}}, T_{\text{sign}})
      \text{ resource}_A}
    \hypo{s_{i,A}:_A T_{\text{infstate}}}
    \hypo{u^*:_A T_{\text{phon}}}
    \infer[no rule]3{T_{\text{infstate}}\sqsubseteq\text{\smallrecord{
          \smalltfield{shared}{\smallrecord{
              \smalltfield{current-utterance}{$T_{\text{chart}}$}}}}}}
    \infer[enth]1{s_{i+1,A}:_A T_{\text{infstate}}\text{
        \fbox{\d{$\wedge$}} }
      \text{\smallrecord{
          \smalltfield{shared}{\smallrecord{
              \smalltfield{current-utterance}{f$_{\textsc{IntegLex}}$($T_{\text{chart}}$,
                $T_{\text{phon}}$)}}}}}}
    \end{prooftree}
 
\end{subex} 
   
\end{ex} 
  

% We introduce update functions defined by
% \textbf{IntegrateLexicalEvent}($T_{\mathrm{phon}}$,
% $T_{\mathrm{chart}}$) where $T_{\mathrm{phon}}$ is the type the agent
% assigns to the
% phonological event perceived and $T_{\mathrm{chart}}$ is the type the
% agent assigns to the current chart.  This is governed by the clauses
% in \nexteg{}.
% \begin{ex} 
% \begin{subex} 
 
% \item If $T_{\mathrm{phon}}$ is a lexical phonological resource and
%   $T_{\mathrm{chart}}$ is \textit{Rec}, then
%   \textbf{IntegrateLexicalEvent}($T_{\mathrm{phon}}$,
%   $T_{\mathrm{chart}}$) is

% $\lambda
% r$:\smallrecord{\smalltfield{shared}{\smallrecord{\smalltfield{current-move}{\smallrecord{\smalltfield{chart}{$T_{\mathrm{chart}}$}}}}}}
% \\
% \hspace*{1em} $\lambda u$:$T_{\mathrm{phon}}$ . \\
% \hspace*{2em}
% \smallrecord{\smalltfield{shared}{\smallrecord{\smalltfield{current-move}{\smallrecord{\smalltfield{chart}{\smallrecord{\smalltfield{e$_1$}{$T_{\mathrm{phon}}$}\\
%                                                                                                                         \smalltfield{e}{\smallrecord{\smalltfield{e$_1$}{start(e$_1$)}}$^{\frown}$\smallrecord{\smalltfield{e$_1$}{end(e$_1$)}}}}}}}}}}


 
% \item If $T_{\mathrm{phon}}$ is a lexical phonological resource,
%   $T_{\mathrm{chart}}$ is a record type such that `e$_n$' is the maximal
%   distinguished label `e$_i$' in $T_{\mathrm{chart}}$ and
%   $T_{\mathrm{chart}}$ is
%   $T_1$\d{$\wedge$}\smallrecord{\smalltfield{e}{${T_2}^{\frown}T_3$}}
%   where $T_1$ is a record type, $T_2$ is a string type and
%   $T_3\sqsubseteq$ \smallrecord{\smalltfield{e$_n$}{end(e$_n$)}}, \\
% then
%   \textbf{IntegrateLexicalEvent}($T_{\mathrm{phon}}$,
%   $T_{\mathrm{chart}}$) is

%   $\lambda
% r$:\smallrecord{\smalltfield{shared}{\smallrecord{\smalltfield{current-move}{\smallrecord{\smalltfield{chart}{$T_{\mathrm{chart}}$}}}}}}
% \\
% \hspace*{1em} $\lambda u$:$T_{\mathrm{phon}}$ . \\
% \hspace*{2em}
% \smallrecord{\smalltfield{shared}{\smallrecord{\smalltfield{current-move}{\smallrecord{\smalltfield{chart}{\smallrecord{\smalltfield{e$_{n+1}$}{$T_{\mathrm{phon}}$}\\
%                                                                                                                         \smalltfield{e}{${T_2}^{\frown}$($T_3$
%                                                                                                                           \d{$\wedge$}
% \smallrecord{\smalltfield{e$_{n+1}$}{start(e$_{n+1}$)}})$^{\frown}$\\ \hspace*{1em}\smallrecord{\smalltfield{e$_{n+1}$}{end(e$_{n+1}$)}}}}}}}}}}
 
% \end{subex} 
   
% \end{ex} 
% The licensing condition associated with chart update functions is the
% same as for other update functions (see Appendix~\ref{app:updatefuns}).

We now need action rules that will add signs to the chart which are
derived from the lexical resources for signs associated with
phonological types.  % For the lexical resources associated with this
% chapter in Appendix~\ref{app:lexeng} we will define the notion of a
% resource lexical sign type based on a phonological type as in
% \nexteg{}.
% \begin{ex} 
% $T_{\mathrm{lex}}$ is a \textit{resource lexical sign type based on
%   phonological type $T_{\mathrm{phon}}$ according to a
%   collection of resources $R$} just in case $T_{\mathrm{lex}}$ is in
% $R$ and is identical with either
% Lex$_{\mathrm{PropName}}$($T_{\mathrm{phon}}$, $a$), for some
% $a$:\textit{Ind},  Lex$_{\mathrm{CommonNoun}}$($T_{\mathrm{phon}}$,
% $p$), for some predicate $p$,
% Lex$_{\mathrm{IndefArt}}$($T_{\mathrm{phon}}$) or Lex$_{\mathrm{be}}$($T_{\mathrm{phon}}$)
% \end{ex} 
% We introduce update functions for integrating such lexical resources
% into the chart.  These update functions are defined by
% \textbf{IntegrateLexicalResources}($T_{\mathrm{phon}}$,
% $T_{\mathrm{chart}}$) where $T_{\mathrm{phon}}$ is the type the agent
% assigns to the phonological event perceived and $T_{\mathrm{chart}}$
% is the type the agent assigns to the current chart.  This is governed
% by the clause in \nexteg{}.
% \begin{ex} 
% If 
% \begin{enumerate}
% \item $T_{\mathrm{phon}}$ is a resource phonological type
% \item $T_{\mathrm{event}}$ is either
% ${T_{\mathrm{start}}}^{\frown}T_{\mathrm{end}}$ or
% ${T_{\mathrm{evpref}}}^{\frown}{T_{\mathrm{start}}}^{\frown}T_{\mathrm{end}}$ 
% \hspace*{1em}(where $T_{\mathrm{start}}\sqsubseteq$
% \smallrecord{\smalltfield{e$_k$}{start(e$_k$)}},
% $T_{\mathrm{end}}\sqsubseteq$
% \smallrecord{\smalltfield{e$_k$}{end(e$_k$)}} and
% $T_{\mathrm{evpref}}$, ``event prefix'', is a type)
% \label{ex:integlexresTevent} 
% \item $T_{\mathrm{chart}}\sqsubseteq$
%   \smallrecord{\smalltfield{e$_k$}{$T_{\mathrm{phon}}$}\\
%                \smalltfield{e}{$T_{\mathrm{event}}$}} whose maximal
%              `e$_i$' label is `e$_n$' 
% \label{ex:integlexresTchart}
% \item $T_{\mathrm{sign}}$ is a resource lexical sign type based on
% $T_{\mathrm{phon}}$ such that for no $j$
% \begin{enumerate}
% \item $T_{\mathrm{chart}}\sqsubseteq$
% \smallrecord{\smalltfield{e$_j$}{$T_{\mathrm{sign}}$}},
% \item   $T_{\mathrm{start}}\sqsubseteq$
%   \smallrecord{\smalltfield{e$_j$}{start(e$_j$)}} and 
% \item   $T_{\mathrm{end}}\sqsubseteq$
%   \smallrecord{\smalltfield{e$_j$}{end(e$_j$)}} 
% \end{enumerate}
% \label{ex:integlexresTsign}
% \end{enumerate}
% then \textbf{IntegrateLexicalResources}($T_{\mathrm{phon}}$,
% $T_{\mathrm{chart}}$) is 

% \medskip

% $\lambda
% r:$\smallrecord{\smalltfield{shared}{\smallrecord{\smalltfield{current-move}{\smallrecord{\smalltfield{chart}{$T_{\mathrm{chart}}$}}}}}}
% . \\
% \hspace*{1em}\smallrecord{\smalltfield{shared}{\smallrecord{\smalltfield{current-move}{\smallrecord{\smalltfield{chart}{\smallrecord{\smalltfield{e$_{n+1}$}{$T_{\mathrm{sign}}$}\\
%                                                                                                                                      \smalltfield{e}{$T_{\mathrm{newevent}}$}}}}}}}}

% \medskip                                                                                                                       

% where $T_{\mathrm{newevent}}$ is 
% \begin{enumerate}
% \item []\textbf{either} 
% ($T_{\mathrm{start}}$\d{$\wedge$}\smallrecord{\smalltfield{e$_{n+1}$}{start(e$_{n+1}$)}})$^{\frown}$($T_{\mathrm{end}}$\d{$\wedge$}\smallrecord{\smalltfield{e$_{n+1}$}{end(e$_{n+1}$)}})

% \item []\textbf{or} 
% ${T_{\mathrm{evpref}}}^{\frown}$($T_{\mathrm{start}}$\d{$\wedge$}\smallrecord{\smalltfield{e$_{n+1}$}{start(e$_{n+1}$)}})$^{\frown}$($T_{\mathrm{end}}$\d{$\wedge$}\smallrecord{\smalltfield{e$_{n+1}$}{end(e$_{n+1}$)}})
% \end{enumerate}
% depending on whether $T_{\mathrm{event}}$ is ${T_{\mathrm{start}}}^{\frown}T_{\mathrm{end}}$ or
% ${T_{\mathrm{evpref}}}^{\frown}{T_{\mathrm{start}}}^{\frown}T_{\mathrm{end}}$ 
% \label{ex:integlexres}
% \end{ex} 
% There are several complexities in \preveg{} which need some
% explanation.  Firstly, notice that the update functions generated by
% \textbf{IntegrateLexicalResources} are of the form \nexteg{}.
% \begin{ex} 
% $\lambda r$:$T_1$ . $T_2$ 
% \end{ex} 
% That is, they are \textit{tacit} update functions which do not require
% a second event argument.  They map directly from an information state
% of a certain type to a type for the new information state.  An update
% using this update function is thus not driven by an agent-external
% event, merely by the state that the agent is currently in. An
% important issue in the design of tacit update functions is to develop
% mechanisms to prevent them from applying indefinitely many times
% adding the same information repeatedly and getting the agent carrying
% out the updates into a infinite loop.  We will discuss how this has
% been avoided here below.

% Condition~\ref{ex:integlexresTevent} in (\ref{ex:integlexres}) allows
% the `e'-field in the current chart to contain either a concatenation
% of just two events or to be a string of events ending in two events.
% The two final events of the event string are required to include the
% starting and ending respectively of some particular event labelled by
% `e$_k$' (for some natural number $k$).  Note that other things can
% also be going on in these two final events as indicated by the use of
% subtyping to characterize $T_{\mathrm{start}}$ and $T_{\mathrm{end}}$
% here.

% Condition~\ref{ex:integlexresTchart} in (\ref{ex:integlexres})
% requires that the event labelled `e$_k$' in the current information
% state is of the phonological type which we are going to use to construct
% the sign which we are going to add to the chart.  We might have required
% `e$_k$' to be the maximal `e$_i$', that is, we might have required
% that the field labelled `e$_k$' was the last to have been added.  This
% would have been one way of avoiding an infinite loop, since once we
% have added the new field with the sign type, `e$_k$' would no longer
% be maximal and \textbf{IntegrateLexicalResources} would not become
% applicable again until a further lexical event was entered into the
% chart.  This would in fact have worked given the restricted collection
% of resources we are considering in this chapter, since they only allow
% for one sign type to be associated with any phonological type
% corresponding to a word.  In general, this will not be the case since
% we want to allow for ambiguous words like \textit{bank} and
% \textit{can} to be associated with different sign types and we want to
% allow for all of the alternative sign types to be added to the same
% chart.  For this reason we want \textbf{IntegrateLexicalResources} to
% apply even if `e$_k$' is not maximal.

% As condition~\ref{ex:integlexresTchart} does not prevent looping we
% introduce a mechanism that will prevent it in
% condition~\ref{ex:integlexresTsign} in (\ref{ex:integlexres})
% instead.  This introduces a sign type based on the relevant
% phonological type which is going to be used for the update.  But it
% requires that the sign type has not already be introduced on the chart
% with the start and end of the sign at the end of the event string we
% are considering.  (We do not wish to prevent it having be associated
% with a previous part of the event, of course, since the same word can
% occur more than once in an utterance.)
In \nexteg{} we specify an action rule \textsc{IntegPropName} for
integrating proper names into the chart.
\begin{ex}
  \hspace*{-2em}
  \begin{prooftree}
    \hypo{\text{Lex}_{\text{PropName}}(T_{\text{Phon}},a)\text{
        resource}_A}
    \infer[no rule]1{s_{i,A}:_A T_{\text{infstate}}}
    \infer[no rule]1{s_{i,A}.\text{shared}.\text{current-utterance}:_A
      T_{\text{chart}}}
    \hypo{T_{\text{chart}}\sqsubseteq\text{\smallrecord{
          \smalltfield{e$_j$}{$T_{\text{phon}}$}\\
          \smalltfield{e}{\smallrecord{
              \smalltfield{t$_k$}{\smallrecord{
                  \smalltfield{e$_j$}{start($\Uparrow^2$e$_j$)}}}\\
              \smalltfield{t$_l$}{\smallrecord{
                  \smalltfield{e$_j$}{end($\Uparrow^2$e$_j$)}}}}}}}}
    \infer[no rule]1{\hspace*{13em}}
    \infer[enth]2[\begin{minipage}{.3\textwidth}\footnotesize$T_{\text{chart}}\not\sqsubseteq$\\ \smallrecord{
      \smalltfield{e$_j$}{$T_{\text{phon}}$}\\
      \smalltfield{e$_n$}{Lex$_{\text{PropName}}$($T_{\text{phon}}$,
        $a$) \d{$\wedge$}\\ \hspace*{1em}\smallrecord{
          \smalltfield{s-event}{\smallrecord{
              \smallmfield{e}{$\Uparrow^2$e$_j$}{\textit{Phon}}}}}}}\\
    for any $n$\end{minipage}]{s_{i+1,A}:_A T_{\text{infstate}}
      \text{ \fbox{\d{$\wedge$}} } \text{\smallrecord{
          \smalltfield{shared}{\smallrecord{
              \smalltfield{current-utterance}{$T_{\text{chart}}$\d{$\wedge$}\\
                \smallrecord{
                  \smalltfield{e$_j$}{$T_{\text{phon}}$}\\
                  \smalltfield{e$_{\text{incr}_e(T_{\text{chart}})}$}{Lex$_{\text{PropName}}$($T_{\text{phon}}$,
                    $a$) \d{$\wedge$}\\ \hspace*{3em}\smallrecord{
                      \smalltfield{s-event}{\smallrecord{
                          \smallmfield{e}{$\Uparrow^2$e$_j$}{\textit{Phon}}}}}}\\
                  \smalltfield{e}{\smallrecord{
                      \smalltfield{t$_k$}{\smallrecord{
                          \smalltfield{e$_{\text{incr}_e(T_{\text{chart}})}$}{start($\Uparrow^2$e$_{\text{incr}_e(T_{\text{chart}})}$)}}}\\
                      \smalltfield{t$_l$}{\smallrecord{
                          \smalltfield{e$_{\text{incr}_e(T_{\text{chart}})}$}{end($\Uparrow^2$e$_{\text{incr}_e(T_{\text{chart}})}$)}}}}}}}}}}}}
    \end{prooftree}
\end{ex} 



For common nouns we have a similar rule, \textsc{IntegCommonNoun}, as in \nexteg{}.
\begin{ex}
  \hspace*{-2em}
  \begin{prooftree}
    \hypo{\text{Lex}_{\text{CommonNoun}}(T_{\text{Phon}},p)\text{
        resource}_A}
    \infer[no rule]1{s_{i,A}:_A T_{\text{infstate}}}
    \infer[no rule]1{s_{i,A}.\text{shared}.\text{current-utterance}:_A
      T_{\text{chart}}}
    \hypo{T_{\text{chart}}\sqsubseteq\text{\smallrecord{
          \smalltfield{e$_j$}{$T_{\text{phon}}$}\\
          \smalltfield{e}{\smallrecord{
              \smalltfield{t$_k$}{\smallrecord{
                  \smalltfield{e$_j$}{start($\Uparrow^2$e$_j$)}}}\\
              \smalltfield{t$_l$}{\smallrecord{
                  \smalltfield{e$_j$}{end($\Uparrow^2$e$_j$)}}}}}}}}
    \infer[no rule]1{\hspace*{13em}}
    \infer[enth]2[\begin{minipage}{.3\textwidth}\footnotesize$T_{\text{chart}}\not\sqsubseteq$\\ \smallrecord{
      \smalltfield{e$_j$}{$T_{\text{phon}}$}\\
      \smalltfield{e$_n$}{Lex$_{\text{CommonNoun}}$($T_{\text{phon}}$,
        $p$) \d{$\wedge$}\\ \hspace*{1em}\smallrecord{
          \smalltfield{s-event}{\smallrecord{
              \smallmfield{e}{$\Uparrow^2$e$_j$}{\textit{Phon}}}}}}}\\
    for any $n$\end{minipage}]{s_{i+1,A}:_A T_{\text{infstate}}
      \text{ \fbox{\d{$\wedge$}} } \text{\smallrecord{
          \smalltfield{shared}{\smallrecord{
              \smalltfield{current-utterance}{$T_{\text{chart}}$\d{$\wedge$}\\
                \smallrecord{
                  \smalltfield{e$_j$}{$T_{\text{phon}}$}\\
                  \smalltfield{e$_{\text{incr}_e(T_{\text{chart}})}$}{Lex$_{\text{CommonNoun}}$($T_{\text{phon}}$,
                    $p$) \d{$\wedge$}\\ \hspace*{3em}\smallrecord{
                      \smalltfield{s-event}{\smallrecord{
                          \smallmfield{e}{$\Uparrow^2$e$_j$}{\textit{Phon}}}}}}\\
                  \smalltfield{e}{\smallrecord{
                      \smalltfield{t$_k$}{\smallrecord{
                          \smalltfield{e$_{\text{incr}_e(T_{\text{chart}})}$}{start($\Uparrow^2$e$_{\text{incr}_e(T_{\text{chart}})}$)}}}\\
                      \smalltfield{t$_l$}{\smallrecord{
                          \smalltfield{e$_{\text{incr}_e(T_{\text{chart}})}$}{end($\Uparrow^2$e$_{\text{incr}_e(T_{\text{chart}})}$)}}}}}}}}}}}}
    \end{prooftree}
  \end{ex}
Finally for indefinite articles we have \textsc{IntegIndefArt} given
in \nexteg{}.
\begin{ex}
  \hspace*{-2em}
  \begin{prooftree}
    \hypo{\text{Lex}_{\text{IndefArt}}(T_{\text{Phon}})\text{
        resource}_A}
    \infer[no rule]1{s_{i,A}:_A T_{\text{infstate}}}
    \infer[no rule]1{s_{i,A}.\text{shared}.\text{current-utterance}:_A
      T_{\text{chart}}}
    \hypo{T_{\text{chart}}\sqsubseteq\text{\smallrecord{
          \smalltfield{e$_j$}{$T_{\text{phon}}$}\\
          \smalltfield{e}{\smallrecord{
              \smalltfield{t$_k$}{\smallrecord{
                  \smalltfield{e$_j$}{start($\Uparrow^2$e$_j$)}}}\\
              \smalltfield{t$_l$}{\smallrecord{
                  \smalltfield{e$_j$}{end($\Uparrow^2$e$_j$)}}}}}}}}
    \infer[no rule]1{\hspace*{13em}}
    \infer[enth]2[\begin{minipage}{.3\textwidth}\footnotesize$T_{\text{chart}}\not\sqsubseteq$\\ \smallrecord{
      \smalltfield{e$_j$}{$T_{\text{phon}}$}\\
      \smalltfield{e$_n$}{Lex$_{\text{IndefArt}}$($T_{\text{phon}}$) \d{$\wedge$}\\ \hspace*{1em}\smallrecord{
          \smalltfield{s-event}{\smallrecord{
              \smallmfield{e}{$\Uparrow^2$e$_j$}{\textit{Phon}}}}}}}\\
    for any $n$\end{minipage}]{s_{i+1,A}:_A T_{\text{infstate}}
      \text{ \fbox{\d{$\wedge$}} } \text{\smallrecord{
          \smalltfield{shared}{\smallrecord{
              \smalltfield{current-utterance}{$T_{\text{chart}}$\d{$\wedge$}\\
                \smallrecord{
                  \smalltfield{e$_j$}{$T_{\text{phon}}$}\\
                  \smalltfield{e$_{\text{incr}_e(T_{\text{chart}})}$}{Lex$_{\text{IndefArt}}$($T_{\text{phon}}$) \d{$\wedge$}\\ \hspace*{3em}\smallrecord{
                      \smalltfield{s-event}{\smallrecord{
                          \smallmfield{e}{$\Uparrow^2$e$_j$}{\textit{Phon}}}}}}\\
                  \smalltfield{e}{\smallrecord{
                      \smalltfield{t$_k$}{\smallrecord{
                          \smalltfield{e$_{\text{incr}_e(T_{\text{chart}})}$}{start($\Uparrow^2$e$_{\text{incr}_e(T_{\text{chart}})}$)}}}\\
                      \smalltfield{t$_l$}{\smallrecord{
                          \smalltfield{e$_{\text{incr}_e(T_{\text{chart}})}$}{end($\Uparrow^2$e$_{\text{incr}_e(T_{\text{chart}})}$)}}}}}}}}}}}}
    \end{prooftree}
\end{ex}
These three rules are identical except for the different categories
involved.  They are tacit action rules in that they do not require any
external event such as an utterance in order for their premises (or
preconditions) to be met.  This means that we have to be careful to
avoid them being used repeatedly adding the same information over and
over to the chart.  The side conditions one these rules are meant to
check that the information that they will add is not already present
on the chart.

Recall that we use `t$_i$' where $i$ is a natural number as the
distinguished labels in records that represent strings.  Thus the type
\nexteg{} is the type of strings whose $k+1$th and $l+1$th positions
are of type $T_1$ and $T_2$ respectively.
\begin{ex} 
  \record{
    \tfield{t$_k$}{$T_1$}\\
    \tfield{t$_l$}{$T_2$}}
    
\end{ex}
Note that because we are modelling strings as records these types can
be merged as required by these rules.  Thus the merge of \preveg{}
with \nexteg{a} will be \nexteg{b}.
\begin{ex} 
\begin{subex} 
 
\item \record{
    \tfield{t$_k$}{$T_1'$}\\
    \tfield{t$_l$}{$T_2'$}} 
 
\item \record{
    \tfield{t$_k$}{$T_1$\d{$\wedge$}$T_1'$}\\
    \tfield{t$_l$}{$T_2$\d{$\wedge$}$T_2'$}} 
 
\end{subex} 
   
\end{ex} 
  


After integrating lexical sign types into the chart, the next step is to
integrate rules from our resources that apply to strings which could
begin with a lexical sign of this type.  We will use an action rule
called \textsc{IntegRule} which integrates a syntactic rule of the
kind we have discussed into the chart.  It is given in \nexteg{}.
\begin{sidewaysfigure}
\begin{ex} 
  %\hspace*{-6em}
\begin{prooftree}
    \hypo{s_{i,A}:_A T_{\text{infstate}}}
    \hypo{s_{i,A}.\text{shared}.\text{current-utterance}:_A
      T_{\text{chart}}}
    \infer[no rule]2{T_{\text{chart}}\sqsubseteq\text{\smallrecord{
          \tfield{e$_k$}{$T_{\text{sign}}$}\\
          \tfield{e}{${T_{\text{evpref}}}^\frown T_{\text{end}}$}}}
    \hspace*{2em}\text{incr}_e(s_{i,A}.\text{shared}.\text{current-utterance})=n}
    \infer[no rule]1{T_{\text{cat}}\in\{\textit{NP},\textit{VP},\ldots\}
    \hspace*{2em}T_{\text{sign}}\sqsubseteq T_{\text{cat}}
    \hspace*{2em}T_{\text{end}}\sqsubseteq\text{\smallrecord{
          \tfield{e$_k$}{end(e$_k$)}}}}
    \infer[no rule]1{f_{\text{rule}}\text{ resource$_A$}
    \hspace*{2em}f_{\text{rule}}:T_{\text{rule}}}
    \infer[no rule]1{T_{\text{rule}}=({T_{\text{leftcorner}}}^\frown T_{\text{rest}}\rightarrow\textit{Type})
    \hspace*{2em}T_{\text{sign}}\sqsubseteq T_{\text{leftcorner}}}
    \infer[enth]1[\begin{tabular}{l}$T_{\text{chart}}\not\sqsubseteq$\\\smallrecord{
      \smallmfield{e$_l$}{$f_{\text{rule}}$}{$T_{\text{rule}}$}\\
      \smalltfield{e}{${T_{\text{evpref}}}^\frown$\\\hspace*{.5em}\smallrecord{
          \smalltfield{e$_1$}{start($\Uparrow^3$e$_l$)$^\frown$end($\Uparrow^3$e$_l$)}}}}\\
        for any $l$\end{tabular}]
        {\begin{array}{l} s_{i+1,A}:_A
          T_{\text{infstate}}\text{\fbox{\d{$\wedge$}}}\\\hspace*{1em}\text{\smallrecord{
              \smalltfield{shared}{\smallrecord{
\smalltfield{current-utterance}{$T_{\text{chart}}$\d{$\wedge$}\smallrecord{
                      \smalltfield{e$_n$}{\smallrecord{
\smallmfield{rule}{$f_{\text{rule}}$}{$T_{\text{rule}}$}\\
\smallmfield{fnd}{$s_{i,A}$.shared.current-utterance.e$_k$}{$T_{\text{sign}}$}\\
                       \smallmfield{req}{$T_{\text{rest}}$}{\textit{Type}}}}\\
\smalltfield{e}{${T_{\text{evpref}}}^\frown(T_{\text{end}}$\d{$\wedge$}\smallrecord{  \smalltfield{e$_n$}{start($\Uparrow^3$e$_n$)$^\frown$end($\Uparrow^3$e$_n$)}}$)$}}}}}}}\end{array}}
\end{prooftree}

      
    
\end{ex}
\end{sidewaysfigure}

  
% \textbf{IntegrateRule}($f_{\mathrm{rule}}$, $T_{\mathrm{chart}}$) to
% generate such update rules.  This is governed by the clause in
% \nexteg{}.
% \begin{ex} 
% If
% \begin{enumerate}
% \item $T_{\mathrm{chart}}$ $\sqsubseteq$ 
% \smallrecord{\smalltfield{e$_k$}{$T_{\mathrm{sign}}$}\\
%              \smalltfield{e}{${T_{\mathrm{evpref}}}^{\frown}T_{\mathrm{end}}$}}

% where: 
% \begin{enumerate}

% \item [] $T_{\mathrm{sign}}$ $\sqsubseteq$ $T_{\mathrm{cat}}$
% ($T_{\mathrm{cat}}$ is one of \textit{NP}, \textit{VP}, \ldots)

% \item [] $T_{\mathrm{end}}$ $\sqsubseteq$ 
%  \smallrecord{\smalltfield{e$_k$}{end(e$_k$)}}
% \end{enumerate}


% \item `e$_i$' max in $T_{\mathrm{chart}}$ is `e$_n$'

% \item $f_{\mathrm{rule}} :
%   (({T_1}^{\frown}{T_2}^{\frown}\ldots^{\frown}T_m)\rightarrow\textit{Type})$
%   where $T_{\mathrm{sign}}\sqsubseteq T_1$

% \item there is no $l$ such that 
% \begin{enumerate}
% \item []$T_{\mathrm{chart}}$ $\sqsubseteq$ 
% \smallrecord{\smalltfield{e$_l$}{\smallrecord{\smallmfield{rule}{$f_{\mathrm{rule}}$}{((${T_1}^{\frown}{T_2}^{\frown}\ldots^{\frown}T_m)\rightarrow\textit{Type}$)}}}
%   \\
% \smalltfield{e}{${T_{\mathrm{evpref}}}^{\frown}$\smallrecord{\smalltfield{e$_l$}{start($\Uparrow^3$e$_l$)$^{\frown}$end($\Uparrow^3$e$_l$)}}}}
% \end{enumerate}
% \end{enumerate}

% then \textbf{IntegrateRule}($f_{\mathrm{rule}}$, $T_{\mathrm{chart}}$)
% is

% $\lambda r$:$T_{\mathrm{chart}}$ . \\
% \hspace*{1em}
% \smallrecord{\smallmfield{e$_k$}{$r$.e$_k$}{$T_{\mathrm{sign}}$}\\
%              \smalltfield{e$_{n+1}$}{\smallrecord{\smallmfield{rule}{$f_{\mathrm{rule}}$}{$(({T_1}^{\frown}{T_2}^{\frown}\ldots^{\frown}T_m)\rightarrow\textit{Type})$}\\
%                                                  \smallmfield{fnd}{$\Uparrow$e$_k$}{\textit{Sign}}\\
%                                                  \smallmfield{req}{${T_2}^{\frown}\ldots^{\frown}T_m$}{\textit{Type}}\\
%                                                  \smalltfield{e}{required(req,rule)}}}\\
%              \smalltfield{e}{${T_{\mathrm{evpref}}}^{\frown}$($T_{\mathrm{end}}$\d{$\wedge$}\smallrecord{\smalltfield{e$_{n+1}$}{start($\Uparrow^3$e$_{n+1}$)$^{\frown}$end($\Uparrow^3$e$_{n+1}$)}})}}

% \end{ex}

% Condition~1 in \preveg{} identifies a category field in the chart
% whose event is the latest event in the event string which has been
% processed.  Condition~2 identifies the label of the latest addition to
% the chart so that it can be incremented for what is now going to be
% added.  Condition~3 identifies a rule whose ``left corner'' ($T_1$) is
% a supertype of the type in the category field and Condition~4 requires
% that this rule has not already been added to the chart and related to
% the current final event in the event string --- this in order to
% prevent an infinite loop.  The result of
% \mbox{\textbf{IntegrateRule}($f_{\mathrm{rule}}$, $T_{\mathrm{chart}}$)} is
% then a function which adds a new field to the chart which contains a
% record of the rule, what has so far been found matching the ``left
% corner'' of the rule (that is, the category field that has been
% identified by Condition~1), and what is still required in order for the
% rule to be fully satisfied (that is, that is the type of strings
% required by the rule minus the ``left corner'').  Finally, the new
% event is added as both starting and ending at the current end of the
% event string (that is, it does not extend the length of the event
% string, but the new event starts and ends simultaneously with the end
% of the event matching the ``left corner'' of the rule.
The action rule in \preveg{} adds a rule to the chart if there is
already something on the chart of the right type to match its ``left
corner'', that is the first type in the string type which is the
domain type of the function which models the rule.  It marks the rule
as having its left corner found and requiring the remaining string
type (without the left corner) as what is still required.\footnote{The fact that
the new event has a non-empty requirement for future events means that
it corresponds to what is known in the chart parsing literature as an
active edge and the new event encodes a dotted rule.}

The final action rule that we need in order to build charts
involves combining an event with a non-empty requirement with an event
of a type matching the requirement whose start coincides with the end
of the first event.  We will call this action rule \textsc{Combine}.
There are two variants of this action rule: one for the case where what is required is a
string of category signs of greater than length 1 and one for the case where what is required
is a string of a single category sign.  In the first case we need to create a new
event with a requirement which is the remainder of the requirement
after removing the left corner of the original requirement and a found
string which concatenates the found event at the end of the original
found event string.  In the second case we need to add an event of the type
which results from applying the rule to the concatenation of the found
event to
original found event string. As we only have binary rules in our small
grammar the first case will not be necessary here as we will only introduce
a rule onto the chart when we have found an event matching its first
element and the requirement result from this addition will thus be a
string consisting of a
single event of a given category type.  We will thus only introduce
the action rule for the second case here.  We do this in \nexteg{}.% We will use
% \textbf{Combine}($T_{\mathrm{chart}}$, $\ell_1$, $\ell_2$) to generate
% such update rules.  This is governed by the clause in \nexteg{}.
\begin{sidewaysfigure}
%\hspace*{-5em}
\begin{ex}
  \hspace*{-6em}
\begin{prooftree}
    \hypo{s_{i,A}:_A T_{\text{infstate}}}
    \hypo{s_{i,A}.\text{shared}.\text{current-utterance}:_A
      T_{\text{chart}}}
    \infer[no rule]2{T_{\text{chart}}\sqsubseteq\text{
        \smallrecord{
          \smalltfield{e$_f$}{$T_{\mathrm{sign}_1}$} \\
          \smalltfield{e$_k$}{\smallrecord{
              \smallmfield{rule}{$f_{\mathrm{rule}}$}{($T\rightarrow\textit{Type}$)}\\
              \smallmfield{fnd}{$\Uparrow$e$_f$}{\textit{Sign}}\\
              \smallmfield{req}{$T_{\mathrm{sign}_2}$}{\textit{Type}}}} \\
          \smalltfield{e$_l$}{$T_{\mathrm{sign}_3}$} \\
          \smalltfield{e}{${T_1}^{\frown}$\smallrecord{
              \smalltfield{e$_f$}{start($\Uparrow^3$e$_f$)}}$^{\frown}{T_2}^{\frown}$ \\
            \hspace*{2.5em}\smallrecord{
              \smalltfield{e$_f$}{end($\Uparrow^3$e$_f$)}\\
              \smalltfield{e$_k$}{start($\Uparrow^3${e$_k$})$^{\frown}$end($\Uparrow^3$e$_k$)}\\
              \smalltfield{e$_l$}{start($\Uparrow^3$e$_l$)}}$^{\frown}{T_3}^{\frown}$ \\
            \hspace*{9em}\smallrecord{\smalltfield{e$_l$}{end($\Uparrow^3$e$_l$)}}$^{\frown}{T_4}$}}}}
    \infer[no rule]1{1\leq i\leq 3\hspace*{2em}T_{\text{sign}_i}\sqsubseteq
      T_i\hspace*{2em}T_i\in\{\textit{NP},\textit{VP},\ldots\}}
    \infer[no rule]1{\text{max}_e(s_{i,A}.\text{shared}.\text{current-utterance})=n}
    \infer[enth]1[\begin{tabular}{l}$T_{\text{chat}}\not\sqsubseteq$\\
                         \smallrecord{
                         \smalltfield{e$_f$}{$T_{\mathrm{sign}_1}$} \\
                         \smalltfield{e$_l$}{$T_{\mathrm{sign}_3}$} \\
                         \smalltfield{e$_i$}{$f_{\mathrm{rule}}$({e$_f$}$^{\frown}$e$_l$)} \\
                         \smalltfield{e}{{\textit{Rec}$^*$}$^{\frown}$\smallrecord{\smalltfield{e$_i$}{start($\Uparrow$e$_i$)}
                 \\
                         \smalltfield{e$_f$}{start($\Uparrow$e$_f$)}}$^{\frown}$\textit{Rec}$^*$}}\\
                         for any $i$\end{tabular}]{\hspace*{-.5em}\begin{array}{l}s_{i+1,A}:_A
                         T_{\text{infstate}}
                         \text{\fbox{\d{$\wedge$}}}\\
                         \hspace*{.5em}\text{\smallrecord{
                             \smalltfield{shared}{\smallrecord{
                                 \smalltfield{current-utterance}{$T_{\text{chart}}$\d{$\wedge$}\smallrecord{
                                     \smalltfield{e$_f$}{$T_{\mathrm{sign}_1}$} \\
                                     \smalltfield{e$_k$}{\smallrecord{
                                         \smallmfield{rule}{$f_{\mathrm{rule}}$}{($T\rightarrow\textit{Type}$)}\\
                                         \smallmfield{fnd}{$\Uparrow$e$_f$}{\textit{Sign}}\\
                                         \smallmfield{req}{$T_{\mathrm{sign}_2}$}{\textit{Type}}}} \\
                                     \smalltfield{e$_l$}{$T_{\mathrm{sign}_3}$} \\
                                     \smalltfield{e$_{n+1}$}{$r$.e$_k$.rule({$r$.e$_f$}$^{\frown}r$.e$_l$)}\\
                                     \smalltfield{e}{${T_1}^{\frown}$\smallrecord{
                                         \smalltfield{e$_f$}{start($\Uparrow^3$e$_f$)}\\
                                         \smalltfield{e$_{n+1}$}{start($\Uparrow^3$e$_{n+1}$)}}$^{\frown}{T_2}^{\frown}$\smallrecord{
                                         \smalltfield{e$_k$}{start($\Uparrow^3$e$_k$)$^{\frown}$end($\Uparrow^3$e$_k$)}\\
                                         \smalltfield{e$_l$}{start($\Uparrow^3$e$_l$)}}\\\hspace*{3em}$^{\frown}{T_3}^{\frown}$\smallrecord{
                                         \smalltfield{e$_l$}{end($\Uparrow^3$e$_l$)}\\
                                         \smalltfield{e$_{n+1}$}{end($\Uparrow^3$e$_{n+1}$)}}$^{\frown}T_4$}}}}}}}\end{array}}
                     \end{prooftree}
                     
                         
                                                            
                                                            
                                                            
                   \end{ex}
\end{sidewaysfigure}

 
  
% \begin{ex} 
% If
% \begin{enumerate} 
 
% \item $T_{\mathrm{chart}}$ $\sqsubseteq$ 
% \smallrecord{\smalltfield{e$_f$}{$T_{\mathrm{sign}_1}$} \\
%              \smalltfield{e$_k$}{\smallrecord{\smallmfield{rule}{$f_{\mathrm{rule}}$}{($T\rightarrow\textit{Type}$)}
%                  \\
%                                               \smallmfield{fnd}{$\Uparrow$e$_f$}{\textit{Sign}}
%                                               \\
%                                               \smallmfield{req}{$T_{\mathrm{sign}_2}$}{\textit{Type}}}} \\
%              \smalltfield{e$_l$}{$T_{\mathrm{sign}_3}$} \\
%              \smalltfield{e}{${T_1}^{\frown}$\smallrecord{\smalltfield{e$_f$}{start($\Uparrow^3$e$_f$)}}$^{\frown}{T_2}^{\frown}$ \\
%                                              \hspace*{2.5em}\smallrecord{\smalltfield{e$_f$}{end($\Uparrow^3$e$_f$)}\\
%                                                                       \smalltfield{e$_k$}{start($\Uparrow^3${e$_k$})$^{\frown}$end($\Uparrow^3$e$_k$)}
%                                \\
%                                                                       \smalltfield{e$_l$}{start($\Uparrow^3$e$_l$)}}$^{\frown}{T_3}^{\frown}$ \\
%                              \hspace*{2.5em}\smallrecord{\smalltfield{e$_l$}{end($\Uparrow^3$e$_l$)}}$^{\frown}{T_4}$}}

% where

% \begin{enumerate} 
 
% \item [] $T_{\mathrm{sign}_1}$, $T_{\mathrm{sign}_2}$ and
%   $T_{\mathrm{sign}_3}$ are subtypes of one of \textit{NP},
%   \textit{VP}, \ldots, that is, they are types of category signs.

% \item [] $T_{\mathrm{sign}_3}$ $\sqsubseteq$ $T_{\mathrm{sign}_2}$
 
% \item [] $T$ is a type of strings of category signs
 
 
% \end{enumerate} 
   
 
% \item `e$_i$' max in $T_{\mathrm{chart}}$ is `e$_n$'

% \item There is no $i$ such that

% \begin{enumerate}
% \item []$T_{\mathrm{chart}}\ \sqsubseteq$ 
% \smallrecord{\smalltfield{e$_f$}{$T_{\mathrm{sign}_1}$} \\
%              \smalltfield{e$_l$}{$T_{\mathrm{sign}_3}$} \\
%              \smalltfield{e$_i$}{$f_{\mathrm{rule}}$({e$_f$}$^{\frown}$e$_l$)} \\
%              \smalltfield{e}{{\textit{Rec}$^*$}$^{\frown}$\smallrecord{\smalltfield{e$_i$}{start($\Uparrow$e$_i$)}
%                  \\
%                                                                     \smalltfield{e$_f$}{start($\Uparrow$e$_f$)}}$^{\frown}$\textit{Rec}$^*$}}
% \end{enumerate}
 
% \end{enumerate}

% then \textbf{Compose}($T_{\mathrm{chart}}$, e$_k$, e$_l$) is

% $\lambda r$:$T_{\mathrm{chart}}$ . \\
% \hspace*{1em}\smallrecord{\smalltfield{e$_f$}{$T_{\mathrm{sign}_1}$} \\
%                           \smalltfield{e$_k$}{\smallrecord{\smallmfield{rule}{$f_{\mathrm{rule}}$}{($T\rightarrow\textit{Type}$)}
%                  \\
%                                                            \smallmfield{fnd}{$\Uparrow$e$_f$}{\textit{Sign}}
%                                               \\
%                                                            \smallmfield{req}{$T_{\mathrm{sign}_2}$}{\textit{Type}}}} \\
%              \smalltfield{e$_l$}{$T_{\mathrm{sign}_3}$} \\
%                           \smalltfield{e$_{n+1}$}{$r$.e$_k$.rule({$r$.e$_f$}$^{\frown}r$.e$_l$)}
%     \\
%                           \smalltfield{e}{${T_1}^{\frown}$\smallrecord{\smalltfield{e$_f$}{start($\Uparrow^3$e$_f$)}\\
%                                                                       \smalltfield{e$_{n+1}$}{start($\Uparrow^3$e$_{n+1}$)}}$^{\frown}{T_2}^{\frown}$\smallrecord{\smalltfield{e$_k$}{start($\Uparrow^3$e$_k$)$^{\frown}$end($\Uparrow^3$e$_k$)}\\
%                                                                                                                                                               \smalltfield{e$_l$}{start($\Uparrow^3$e$_l$)}}\\\hspace*{3em}$^{\frown}{T_3}^{\frown}$\smallrecord{\smalltfield{e$_l$}{end($\Uparrow^3$e$_l$)}\\
%                                                                \smalltfield{e$_{n+1}$}{end($\Uparrow^3$e$_{n+1}$)}}$^{\frown}T_4$}} 
   
% \end{ex} 
% Condition~1 in \preveg{} requires that the chart to be updated has a
% rule event labelled `e$_k$' where the found event is labelled `e$_f$'
% and that there is an event `e$_l$', starting at the end of `e$_f$',
% simultaneously with `e$_k$'.  The type specified for `e$_l$' must be a
% subtype of the type identified as the required type in `e$_k$', that
% is $T_{\mathrm{sign}_2}$.

% Condition~2 identifies the maximum event index in the chart as $n$.

% Condition~3 requires that the result of applying the rule to the event
% string {e$_f$}$^{\frown}$e$_l$ has not already been added to the
% chart.  (This will prevent the creation of an infinite loop.)

% The resulting update function \textbf{Compose}($T_{\mathrm{chart}}$,
% e$_k$, e$_l$) is a function which adds a new event field labelled
% `e$_{n+1}$' for an event of the type returned by applying the rule to
% the event string consisting of the found event, `e$_f$', followed by the
% required event, `e$_l$'.  Event `e$_{n+1}$' starts at the beginning of
% `e$_f$' and ends at the end of `e$_l$'.    

\clearpage

\section{Summary of resources introduced}

This summary does not include the resources for chart processing
introduced in Section~\ref{sec:chart}.

\subsection{Universal grammar resources} 

\subsubsection{Types}

\begin{description}

  \item[\textnormal{\textit{Loc}}] --- a basic type

  $l$ : \textit{Loc} iff $l$ is a region in three dimensional space

  
\item[\textnormal{\textit{Phon}}] --- a basic type

  $e$ : \textit{Phon} iff $e$ is a phonological event
  
\item[\textnormal{\textit{SEvent}}] --- \record{\tfield{e-loc}{\textit{Loc}} \\
        \tfield{sp}{\textit{Ind}} \\
        \tfield{au}{\textit{Ind}} \\
        \tfield{e}{\textit{Phon}} \\
        \tfield{c$_{\mathrm{loc}}$}{loc(e,e-loc)} \\
        \tfield{c$_{\mathrm{sp}}$}{speaker(e,sp)} \\
        \tfield{c$_{\mathrm{au}}$}{audience(e,au)}} (as in
      Chapter~\ref{ch:infex})

      \item[\textnormal{\textit{Ppty}}] ---
        (\smallrecord{\smalltfield{x}{\textit{Ind}}}$\rightarrow$\textit{RecType})
        
      \item[\textnormal{\textit{Quant}}] --- (\textit{Ppty}$\rightarrow$\textit{RecType})

    \item[\textnormal{\textit{Cont}}] --- \textit{RecType}$\vee$\textit{Ppty}$\vee$\textit{Quant}$\vee$(\textit{Ppty}$\rightarrow$\textit{Quant})

      
    \item[\textnormal{\textit{Cat}}] --- a basic type

      s, np, det, n, v, vp : \textit{Cat}

    
    \item[\textnormal{\textit{Syn}}] ---  \record{\tfield{cat}{\textit{Cat}} \\
        \tfield{daughters}{\textit{Sign}$^*$}} 
 


  
    \item[\textnormal{\textit{Sign}}] ---  a basic type

      $\sigma$ : \textit{Sign} iff $\sigma$ :
      \record{\tfield{s-event}{\textit{SEvent}} \\
         \tfield{syn}{\textit{Syn}} \\
        \tfield{cont}{\textit{Cont}}} 

  
\item[\textnormal{\textit{SignType}}] --- a basic type

  $T:\textit{SignType}$ iff $T\sqsubseteq\textit{Sign}$ (as in
  Chapter~\ref{ch:infex})

  
\item[\textnormal{\textit{S}}] --- (\textit{Sign} \d{$\wedge$}
  \smallrecord{\smalltfield{syn}{\smallrecord{\smallmfield{cat}{s}{\textit{Cat}}}}})
  
\item[\textnormal{\textit{NP}}] --- (\textit{Sign} \d{$\wedge$}
  \smallrecord{\smalltfield{syn}{\smallrecord{\smallmfield{cat}{np}{\textit{Cat}}}}})

  
\item[\textnormal{\textit{Det}}] --- (\textit{Sign} \d{$\wedge$}
  \smallrecord{\smalltfield{syn}{\smallrecord{\smallmfield{cat}{det}{\textit{Cat}}}}})
  
\item[\textnormal{\textit{N}}] --- (\textit{Sign} \d{$\wedge$}
  \smallrecord{\smalltfield{syn}{\smallrecord{\smallmfield{cat}{n}{\textit{Cat}}}}})
  
\item[\textnormal{\textit{V}}] --- (\textit{Sign} \d{$\wedge$}
  \smallrecord{\smalltfield{syn}{\smallrecord{\smallmfield{cat}{v}{\textit{Cat}}}}})
  
\item[\textnormal{\textit{VP}}] --- (\textit{Sign} \d{$\wedge$}
  \smallrecord{\smalltfield{syn}{\smallrecord{\smallmfield{cat}{vp}{\textit{Cat}}}}})

  
\item[\textnormal{\textit{NoDaughters}}] ---
  \smallrecord{\smalltfield{syn}{\smallrecord{\smallmfield{daughters}{$\varepsilon$}{\textit{Sign}$^*$}}}}

  


\end{description}

\subsubsection{Predicates}

\begin{description}

  \item[with arity \textnormal{$\langle$\textit{Phon},
    \textit{Loc}$\rangle$}] \mbox{}
  
    \begin{description}

    \item[\textnormal{loc}] --- $e$ : loc($u$, $l$) iff $u$ is located
      at $l$ in $e$

    \end{description}
    
  \item[with arity \textnormal{$\langle$\textit{Phon},
      \textit{Ind}$\rangle$}] \mbox{}

    \begin{description}

    \item[\textnormal{speaker}] --- $e$ : speaker($u$, $a$) iff $u$
        is the speaker of $u$ in $e$

    \item[\textnormal{audience}] --- $e$ : audience($u$, $a$) iff
        $u$ is the audience of $u$ in $e$

    \end{description}

  
\item[with arity
  \textnormal{$\langle\textit{Ppty},\textit{Ppty}\rangle$} New!] \mbox{}

  \begin{description}
    
  \item[\textnormal{exist}] --- $e$ : exist($P$,$Q$) iff
    $\downP{P}\cap\downP{Q\!\restriction\! e}\not=\emptyset$

  \end{description}
\item[with arity \textnormal{\{$\langle T\rangle\mid T$ is a type\}} New!]
  \mbox{}

  \begin{description}
    
  \item[\textnormal{be}] --- $e:\text{be}(a)$ iff $a\varepsilon e$

  \end{description}
  

\end{description}


\subsubsection{Lexicon}
\begin{description}
\item[\textnormal{Lex}] \mbox{}

  If $T_{\mathrm{phon}}$ is a phonological type (that is,
$T_{\mathrm{phon}}\sqsubseteq\textit{Phon}$) and $T_{\mathrm{sign}}$
is a sign type (that is, $T_{\mathrm{sign}}\sqsubseteq\textit{Sign}$), then we shall use
Lex($T_{\mathrm{phon}}$, $T_{\mathrm{sign}}$) to represent
\begin{quote}
(($T_{\mathrm{sign}}$ \d{$\wedge$}
\smallrecord{\smalltfield{s-event}{\smallrecord{\smalltfield{e}{$T_{\mathrm{phon}}$}}}})
\d{$\wedge$} \textit{NoDaughters})
\end{quote}

\item[\textnormal{SemCommonNoun($p$)}] \mbox{}

  If $p$ is a predicate, then SemCommonNoun($p$) is
  \begin{quote}
    $\lambda
r$:\smallrecord{\smalltfield{x}{\textit{Ind}}}
. \record{\tfield{e}{$p$($r$.x)}}
\end{quote}

\item[\textnormal{Lex$_{\mathrm{CommonNoun}}$($T_{\mathrm{phon}}$,
    $p$)}] \mbox{}

  If $T_{\mathrm{phon}}$ is a phonological type and $p$ is a
  predicate, then Lex$_{\mathrm{CommonNoun}}$($T_{\mathrm{phon}}$,
  $p$) is
  \begin{quote}
    Lex($T_{\mathrm{phon}}$, \textit{N}) \d{$\wedge$}
    \smallrecord{\smallmfield{cont}{SemCommonNoun($p$)}{\textit{Ppty}}}
  \end{quote}
  
\item[\textnormal{SemPropName($a$)}] \mbox{}

  If $a$:\textit{Ind}, then SemPropName($a$) is
  \begin{quote}
    $\lambda P$:\textit{Ppty} . $P$(\smallrecord{\field{x}{$a$}})
  \end{quote}
  
\item[\textnormal{Lex$_{\mathrm{PropName}}$($T_{\mathrm{Phon}}$,
    $a$)}] \mbox{}

  If $T_{\mathrm{Phon}}$ is a phonological type and $a$:\textit{Ind},
  then Lex$_{\mathrm{PropName}}$($T_{\mathrm{Phon}}$,
  $a$) is
  \begin{quote}
    Lex($T_{\mathrm{Phon}}$, \textit{NP}) \d{$\wedge$}
    \smallrecord{\smallmfield{cont}{SemPropName($a$)}{\textit{Quant}}}
  \end{quote}

  
\item[\textnormal{SemIndefArt}] \mbox{}

  $\lambda Q$:\textit{Ppty} . \\
\hspace*{1em} $\lambda P$:\textit{Ppty}
. \record{\mfield{restr}{$Q$}{\textit{Ppty}} \\
          \mfield{scope}{$P$}{\textit{Ppty}} \\
          \tfield{e}{exist(restr, scope)}}

        
      \item[\textnormal{Lex$_{\mathrm{IndefArt}}$($T_{\mathrm{Phon}}$)}]
        \mbox{}

        If $T_{\mathrm{Phon}}$ is a phonological type, then
        Lex$_{\mathrm{IndefArt}}$($T_{\mathrm{Phon}}$) is
        \begin{quote}
          Lex($T_{\mathrm{Phon}}$, \textit{Det}) \d{$\wedge$}
          \smallrecord{\smallmfield{cont}{SemIndefArt}{(\textit{Ppty}$\rightarrow$\textit{Quant})}}
        \end{quote}
        
      \item[\textnormal{SemBe}] \mbox{}

        $\lambda\mathcal{Q}$:\textit{Quant} . \\
\hspace*{1em} $\lambda r_1$:\smallrecord{\tfield{x}{\textit{Ind}}}
. \\
\hspace*{2em} $\mathcal{Q}$($\lambda
r_2$:\smallrecord{\tfield{x}{\textit{Ind}}}
. \record{\mfield{x}{$r_2$.x, $r_1$.x}{\textit{Ind}}\\
  \tfield{e}{be(x)}})

\item[\textnormal{Lex$_{\mathrm{be}}$($T_{\mathrm{Phon}}$)}] \mbox{}

  If $T_{\mathrm{Phon}}$ is a phonological type, then
  Lex$_{\mathrm{be}}$($T_{\mathrm{Phon}}$) is
\begin{quote}
  Lex($T_{\mathrm{Phon}}$,
\textit{V}) \d{$\wedge$}
\smallrecord{\smallmfield{cont}{SemBe}{(\textit{Quant}$\rightarrow$\textit{Ppty})}}
\end{quote}
  
        

  
  
\end{description}


\subsubsection{Constituent structure}
\begin{description}
\item[\textnormal{RuleDaughters($T_{\text{daughters}}$,
$T_{\text{mother}}$)}] \mbox{}

If $T_{\text{mother}}$ is a sign type and $T_{\text{daughters}}$ is a
type of strings of signs then
\begin{quote}
RuleDaughters($T_{\text{daughters}}$,
$T_{\text{mother}}$)
\end{quote}
is
\begin{quote}
  $\lambda u\! :\! T_{\text{daughters}}$\ . $T_{\text{mother}}$ \d{$\wedge$} \smallrecord{\smalltfield{syn}{\smallrecord{\smallmfield{daughters}{$u$}{$T_{\text{daughters}}$}}}}
\end{quote}

\item[\textnormal{ConcatPhon}] \mbox{}

  $\lambda
u$:\smallrecord{\smalltfield{s-event}{\smallrecord{\smalltfield{e}{\textit{Phon}}}}}$^+$\
. \\
\hspace*{1em}\record{\tfield{s-event}{\record{\mfield{e}{concat$_i$($u[i]$.s-event.e)}{\textit{Phon}}}}}

\item[\textnormal{$T_{\text{mother}}\longrightarrow T_{\text{daughter}_1},\ldots
    T_{\text{daughter}_n}$}] \mbox{}

  If $T_{\text{mother}}$ is a sign type and
  $T_{\text{daughter}_1},\ldots T_{\text{daughter}_n}$ are sign types,
  then
  \begin{quote}
    $T_{\text{mother}}\longrightarrow T_{\text{daughter}_1}\ldots
    T_{\text{daughter}_n}$
  \end{quote}
  represents
  \begin{quote}
RuleDaughters($T_{\text{mother}}$,
${T_{\text{daughter}_1}}^\frown\ldots^\frown T_{\text{daughter}_n}$)\d{\d{$\wedge$}}ConcatPhon 
\end{quote}

\item[\textnormal{ContForwardApp($T_{\text{arg}}$, $T_{\text{res}}$)}] \mbox{}

  If $T_{\text{arg}}$ and $T_{\text{res}}$ are types, then
  ContForwardApp($T_{\text{arg}}$, $T_{\text{res}}$) is
\begin{quote}
  $\lambda
u$:\smallrecord{\smalltfield{cont}{$(T_{\text{arg}}\rightarrow
    T_{\text{res}})$}}$^{\frown}$\smallrecord{\smalltfield{cont}{$T_{\text{arg}}$}} . \\
\hspace*{2em}\smallrecord{\smallmfield{cont}{$u$[0].cont($u$[1].cont)}{$T_{\text{res}}$}}

\end{quote}

\item[\textnormal{$T_{\text{mother}}\longrightarrow
    T_{\text{daughter}_1}\ T_{\text{daughter}_2}\ \mid\
    T_{\text{daughter}_1}'(T_{\text{daughter}_2}':T_{\text{arg}}):T_{\text{res}}$}]
  \mbox{}

  If $T_{\text{mother}}$, $T_{\text{daughter}_1}$ and
  $T_{\text{daughter}_2}$ are sign types and $T_{\text{arg}}$ and
  $T_{\text{res}}$ are content types, then
  \begin{quote}
    $T_{\text{mother}}\longrightarrow
    T_{\text{daughter}_1}\ T_{\text{daughter}_2}\ \mid\
    T_{\text{daughter}_1}'(T_{\text{daughter}_2}':T_{\text{arg}}):T_{\text{res}}$
  \end{quote}
  is
  \begin{quote}
  $T_{\text{mother}}\longrightarrow
    T_{\text{daughter}_1}\ T_{\text{daughter}_2}$ \d{\d{$\wedge$}}
    ContForwardApp($T_{\text{arg}}$, $T_{\text{res}}$)
  \end{quote}

  


  
  

\end{description}


\subsubsection{Action rules}

\begin{description}

  




\item[\textnormal{\textsc{LexRes}}] \mbox{}

  \begin{prooftree}
    \hypo{\text{Lex}(T,C) \text{ resource}_A}
    \hypo{u:_A T}
    \infer[enth]2{:_A(\text{Lex}(T,C)\text{\d{$\wedge$}\smallrecord{\smalltfield{s-event}{\smallrecord{\smallmfield{e}{$u$}{$T$}}}}})}
  \end{prooftree}


  \end{description}



\subsection{Universal speech act resources} (as in Chapter~\ref{ch:infex})

% \subsubsection{Types}

% \begin{description}
  
% \item[\textnormal{\textit{Assertion}}] --- \record{
%   \tfield{s-event}{\textit{SEvent}}\\
%   \tfield{cont}{\textit{RecType}}\\
%   \tfield{illoc}{assert(s-event, cont)}
% }
% \item[\textnormal{\textit{Query}}] --- \record{
%   \tfield{s-event}{\textit{SEvent}}\\
%   \tfield{cont}{\textit{Question}}\\
%   \tfield{illoc}{query(s-event, cont)}
% }

% \item[\textnormal{\textit{Command}}] --- \record{
%   \tfield{s-event}{\textit{SEvent}}\\
%   \tfield{cont}{\textit{RecType}}\\
%   \tfield{illoc}{command(s-event, cont)}
% }

% \item[\textnormal{\textit{Acknowledgement}}] --- \record{
%   \tfield{s-event}{\textit{SEvent}}\\
%   \tfield{cont}{\textit{RecType}}\\
%   \tfield{illoc}{acknowledge(s-event, cont)}
% }

% \item[\textnormal{\textit{AssertionType}}] --- a basic type

%   $T:\textit{AssertionType}$ iff $T\sqsubseteq \textit{Assertion}$

  
% \item[\textnormal{\textit{QueryType}}] --- a basic type

%   $T:\textit{QueryType}$ iff $T\sqsubseteq \textit{Query}$

  
% \item[\textnormal{\textit{CommandType}}] --- a basic type

%   $T:\textit{CommandType}$ iff $T\sqsubseteq \textit{Command}$


 
% \item[\textnormal{\textit{AcknowledgementType}}] --- a basic type

%   $T:\textit{AcknowledgementType}$ iff $T\sqsubseteq \textit{Acknowledgement}$

% \end{description}

\subsection{Universal dialogue resources} (as in Chapter~\ref{ch:infex})

% \subsubsection{Types}

% \begin{description}
% \item[\textnormal{\textit{InfoState}}] --- \record{
%     \tfield{private}{
%       \record{
%         \tfield{agenda}{$\mathrm{list}(\textit{RecType})$}}} \\
%     \tfield{shared}{
%       \record{
%         \tfield{latest-utterance}{\textit{Sign}$^*$}\\
%         \tfield{commitments}{\textit{RecType}}}}}
  
% \item[\textnormal{\textit{InitInfoState}}] --- \record{
%     \tfield{private}{
%       \record{
%         \mfield{agenda}{[ ]}{$\mathrm{list}(\textit{RecType})$}}} \\
%     \tfield{shared}{
%       \record{
%         \mfield{latest-utterance}{$\varepsilon$}{\textit{Sign}$^*$}\\
%         \mfield{commitments}{\textit{Rec}}{\textit{RecType}}}}}  



% \end{description}

% \subsubsection{Update functions and action rules}

% \begin{description}
% \item[\textnormal{f$_{\textsc{PlanAckAss}}$}] $\lambda r$:\textit{InfoState} . \\
%   \hspace*{4em}$\lambda u$:\textit{Assertion} . \\
%   \hspace*{6em}
%   \smallrecord{
%     \smalltfield{private}{\smallrecord{
%         \smalltfield{agenda}{\smallrecord{
%             \smalltfield{fst}{\smallrecord{
%                 \smalltfield{s-event}{\textit{SEvent} \d{$\wedge$} \smallrecord{
%                     \smallmfield{sp}{$u$.s-event.au}{\textit{Ind}}\\
%                     \smallmfield{au}{$u$.s-event.sp}{\textit{Ind}}}}\\
%                 \smallmfield{cont}{$u$.cont}{\textit{Cont}}\\
%                 \smalltfield{illoc}{acknowledge(s-event, cont)}}}\\
%             \smallmfield{rst}{$r$.private.agenda}{$\mathrm{list}(\textit{RecType})$}}}}}\\
%     \smalltfield{shared}{\smallrecord{
%         \smallmfield{latest-utterance}{$u$}{\textit{Assertion}}}}}
  
% \item[\textnormal{\textsc{PlanAckAss}}]
%   \begin{prooftree}
%     \hypo{s_{i,A}:_A T_{\mathrm{curr}}}
%     \hypo{T_{\mathrm{curr}}\sqsubseteq\mathrm{domtype}(\text{f}_{\textsc{PlanAckAss}})}
%     \hypo{u^*:_A T_{\mathrm{utt}}}
%     \hypo{T_{\mathrm{utt}}\sqsubseteq\textit{Assertion}}
%     \infer[enth]4{s_{i+1,A}:_A
%       T_{\mathrm{curr}}\text{\fbox{\d{$\wedge$}}}(\text{f}_{\textsc{PlanAccAss}}(s_{i,A})(u^*)\text{\d{$\wedge$}}\text{\smallrecord{\smalltfield{shared}{\smallrecord{
%               \smalltfield{latest-utterance}{$T_{\mathrm{utt}}$}}}}})}
%   \end{prooftree}

  
% \item[\textnormal{f$_{\textsc{IntegAck}}$}]
%   $\lambda r$:\textit{InfoState} . \\
%   \hspace*{2em}$\lambda u$:\textit{Acknowledgement} . \\
%   \hspace*{4em}\smallrecord{
%     \smalltfield{shared}{\smallrecord{
%         \smallmfield{commitments}{\smallrecord{
%             \smalltfield{prev}{$r$.shared.commitments}}\d{$\wedge$}$u$.cont}{\textit{RecType}}\\
%         \smallmfield{latest-utterance}{$u$}{\textit{Acknowledgement}}}}}

  
% \item[\textnormal{\textsc{IntegAck}}]
%    \begin{prooftree}
%     \hypo{s_{i,a}:_A T_{\mathrm{curr}}}
%     \hypo{T_{\mathrm{curr}}\sqsubseteq\mathrm{domtype}(\text{f}_{\mathrm{IntegAck}})}
%     \hypo{u^*:_A T_{\mathrm{utt}}}
%     \hypo{T_{\mathrm{utt}}\sqsubseteq\textit{Acknowledgement}}
%     \infer[enth]4{s_{i+1,A}:_A
%       T_{\mathrm{curr}}\text{\fbox{\d{$\wedge$}}}(\text{f}_{\textsc{IntegAck}}(s_{i,A})(u^*)\text{
%         \d{$\wedge$} \smallrecord{\smalltfield{shared}{\smallrecord{
%              \smalltfield{latest-utterance}{$T_{\mathrm{utt}}$}}}}})} 
%  \end{prooftree}

 
% \item[\textnormal{\textsc{ExecTopAgenda}}]
%   \begin{prooftree}
%     \hypo{s_{i,A}:_A\textit{InfoState}\text{\d{$\wedge$}
%         \smallrecord{\smalltfield{private}{\smallrecord{
%               \smalltfield{agenda}{\smallrecord{
%                   \smalltfield{fst}{\textit{RecType}}\\
%                   \smalltfield{rst}{$\mathrm{list}(\textit{RecType})$}}}}}}}}
%     \infer[enth]1{:_A
%       s_{i,A}.\text{private}.\text{agenda}.\text{fst}!}
%   \end{prooftree}

  
% \item[\textnormal{\textsc{DowndateAgenda}}]
%   \mbox{}

%   \hspace*{-2.5em}
%   \begin{prooftree}
%     \hypo{s_{i,A}:_A T_{\mathrm{curr}}}
%     \hypo{T_{\mathrm{curr}}\sqsubseteq\text{
%         \smallrecord{
%           \smalltfield{private}{\smallrecord{
%               \smalltfield{agenda}{\smallrecord{
%                   \smalltfield{fst}{\textit{RecType}}\\
%                   \smalltfield{rst}{$\mathrm{list}(\textit{RecType})$}}}}}}}}
%     \hypo{u^*:_A s_{i,A}.\text{private}.\text{agenda}.\text{fst}}
%     \infer[enth]3{s_{i+1,A}:_A
%       T_{\mathrm{curr}}\text{\fbox{\d{$\wedge$}}
%         \smallrecord{
%           \smalltfield{private}{\smallrecord{
%               \smallmfield{agenda}{$s_{i,A}$.private.agenda.rst}{$\mathrm{list}(\textit{RecType})$}}}}}}
%   \end{prooftree}

% \end{description}

\subsection{English resources} 

\subsubsection{Basic types and predicates} (as in Chapter~\ref{ch:infex})

\begin{description}

  
\item[Basic phonological  types for words] \mbox{}

  \{``Dudamel'', ``is'', ``a'', ``conductor'', ``Beethoven'',
  ``composer'', ``Uchida'', ``pianist'', ``aha'', ``ok''\}

\item [Witnesses for basic types] \mbox{}

  \begin{description}
\item[\textnormal{\textit{Ind}}] --- dudamel, beethoven, uchida :
  \textit{Ind}
  \end{description}

  
\item[Predicates] \mbox{}
  
  \begin{description}
  
  \item[with arity \textnormal{$\langle\textit{Ind}\rangle$}] \{conductor, composer, pianist\}
  \end{description}
  

\end{description}

\subsubsection{Grammar}

\begin{description}



\item[Lexical sign types] \mbox{}

\begin{tabbing}
    \{\=Lex$_{\mathrm{PropName}}$(``Dudamel'', $d$), \\
\> Lex$_{\mathrm{PropName}}$(``Beethoven'', $b$), \\
\> Lex$_{\mathrm{IndefArt}}$(``a''), \\
\> Lex$_{\text{CommonNoun}}$(``composer'', composer), \\
\> Lex$_{\text{CommonNoun}}$(``conductor'', conductor), \\
\> Lex$_{\mathrm{be}}$(``is''), \\
\> Lex(``ok'', \textit{S}) (as in Chapter~\ref{ch:infex}),  \\
\> Lex(``aha'', \textit{S}) (as in Chapter~\ref{ch:infex}) \}
\end{tabbing}

\item[Constituent structure rule components] \mbox{}
  \begin{description}

    \item[\textnormal{CnstrIsA}] \mbox{}

  $\lambda
u$:\textit{V}\d{$\wedge$}\smallrecord{\smalltfield{s-event}{\smallrecord{\smalltfield{e}{``is''}}}}$^{\frown}$\textit{NP}\d{$\wedge$}\smallrecord{\smalltfield{syn}{\smallrecord{\smalltfield{daughters}{\textit{Det}\d{$\wedge$}\smallrecord{\smalltfield{s-event}{\smallrecord{\smalltfield{e}{``a''}}}} \\
                                                                    \hspace*{5em}$^{\frown}$
\textit{N}\d{$\wedge$}\smallrecord{\smalltfield{cont}{\textit{Ppty}}}}
}}}. \\
\hspace*{1em} \textit{VP}\d{$\wedge$}\smallrecord{\smallmfield{cont}{$u$[2].syn.daughters[2].cont}{\textit{Ppty}}} 

  \end{description}
  


\item[Constituent structure rules] \mbox{}
  
  \begin{tabbing}
    \{\=\textit{S} $\longrightarrow$ \textit{NP VP} $\mid$ \textit{NP}$'$(\textit{VP}$'$:\textit{Ppty}):\textit{RecType},\\
    \>\textit{NP} $\longrightarrow$ \textit{Det N} $\mid$
    \textit{Det}$'$(\textit{N}$'$:\textit{Ppty}):\textit{Quant},\\
    \>\textit{VP} $\longrightarrow$ \textit{V} \textit{NP}
    \d{\d{$\wedge$}} CnstrIsA, \\
    \>\textit{VP} $\longrightarrow$ \textit{V} \textit{NP} $\mid$
    \textit{V}$'$(\textit{NP}$'$:\textit{Quant}):\textit{Ppty}\}

  \end{tabbing}

\end{description}
  



  
\section{Summary}

% In this chapter we have explored how the type theoretical apparatus
% developed in Chapters~\ref{ch:percint} and \ref{ch:infex} can be
% applied to the notion of grammar, viewing grammatical phenomena in
% terms of event perception and information state update.

This chapter has focussed on a view of grammar within a theory of
action.  We started by relating the kind of constituent structure
found in syntax to the hierarchical structure of event types which
seems necessary for even non-linguistic agents to interact effectively
with their environment and to learn new types of actions efficiently
by exploiting components of actions they already know how to perform.
The kind of syntax we proposed builds closely on the insights of
Head-Driven Phrase Structure Grammar (HPSG) and exploits the
similarities between the feature structures used there and records.
One major difference between the two approaches is that we have both
records and record types where HPSG has just feature structures which
sometimes leads to confusion as to whether they should be thought of intuitively
as objects or types within the theory.

Another important difference
is that the type theoretical approach includes functions and function
types which cannot be characterized in terms of feature structures.
This was important, for example, when we modelled phrase structure
rules in terms of functions and associated action rules.  It was, of
course, of central importance when we turned our attention to
semantics, where much of what we did in this chapter followed the
classical formal semantic analyses proposed by Montague.

If we are going to give a theoretical account of language in terms of
a theory of action, we need to take seriously the fact that language
is processed incrementally, that is, not in terms of complete
sentences but in terms of small phonological events corresponding to
words or parts of words.  Our suggestion for this was to adopt the
notion of parse chart which was developed in computational linguistics
in the 1980's, developed from formal methods used for the parsing of
programming languages in computer science.  We sketched how chart
parsing could be conceived of in terms of our types and action rules
and showed how this could be integrated into our view of updates
carried out on information states in terms of gameboards.  Thus the
incremental processing of an utterance becomes part of the general
theory of information state update needed to account for dialogical interaction.

While we have included both syntax and semantics in this framework and
taken a fairly detailed look at how incremental parsing can be
incorporated in this approach, the actual grammatical phenomena that
we have looked at are linguistically trivial. In Part~\ref{part:sem}
we will look at a variety of linguistic phenomena and argue that the
approach we have developed provides theoretically interesting insights
both in terms of old problems and into new problems that arise when we
view language from a dialogical perspective.



%%% Local Variables: 
%%% mode: latex
%%% TeX-master: "ttl"
%%% End: 
