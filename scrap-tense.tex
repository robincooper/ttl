Let us take a look at how these ideas can be exploited in a
compositional semantics.  In order to do a compositional semantics for
modal verbs we need to distinguish between tensed and non-tensed
verbs.  Our strategy for the structure of sentences with modal verbs
is represented by the informal tree in \nexteg{}.
\begin{ex} 
\Tree [.\textit{S} [.\textit{NP} Mary ] [.\textit{VP}\\\mbox{[+tns]}
[.\textit{V} should ]
\qroof{eat her brocolli}.\textit{VP}\\\mbox{[-tns]} ] ]
\end{ex} 
That is,  we treat the modal \textit{should} as combining with a
non-tensed verb phrase to form a tensed verb-phrase. 

For simplicity of discussion let us consider the intransitive verb
\textit{eat} rather than the complex verb phrase \textit{eat her
  brocolli}.  The version of the operation `SemIntransVerb' defined in
Chapter~\ref{ch:commonnouns} (given in
Appendix~\ref{app:lexuniversal}) yields \nexteg{} when applied to the
predicate `eat' with arity $\langle$\textit{Ind}$\rangle$ and no
restrictions introduced on the domain of the content or background
conditions on the context.  That is, \nexteg{} is SemIntransVerb(eat,
\textit{Ind}, \textit{Ind}, \textit{Rec}).
\begin{ex} 
\record{\field{bg}{\textit{Rec}}\\
        \field{fg}{$\lambda c$:\textit{Rec} . 
                       \record{\field{bg}{\textit{Ind}}\\
                               \field{fg}{$\lambda
                                 r$:\smallrecord{\smalltfield{x}{\textit{Ind}}}
                                 . \record{\tfield{e}{eat($r$.x)}}}}}} 
\label{ex:SemIntransVerbeat}
\end{ex} 
This parametric content for an utterance of \textit{eat} requires that
a sentence such as \textit{Mary eats} has a content which is the event
type \nexteg{} (assuming $m$ is Mary).
\begin{ex} 
\record{\tfield{e}{eat($m$)}} 
\label{ex:typeMaryeat}
\end{ex} 
This type does not require any relationship between the eating event
and the utterance event.  We therefore conclude that this corresponds
best to a tenseless  expression.  It says nothing about when an event
of this type needs to occur.  Note that this is something that is
natural in a system based on types whereas in a semantics based on the
kind of tense operators we find in tense logic it is not so
straightforward to represent that content of a non-tensed utterance.
How could we modify the type in \preveg{} to represent the
relationship of the eating event to some particular speech event, $s$?
In a simple-minded tense system there are basically three
possibilities. The eating event is either required to be simultaneous
with $s$, prior to $s$ or after $s$.  We model this by creating event
types for events which have two components, the speech event, $s$ and
the eating event.  The types are given in \nexteg{a--c} corresponding
to \textit{Mary eats}, \textit{Mary ate} and \textit{Mary will eat}, respectively.
\begin{ex} 
\begin{subex} 
 
\item \record{\mfield{s-event}{$s$}{\textit{SEvent}}\\
              \tfield{e}{eat($m$)}} 
 
\item
  \record{\tfield{e}{eat($m$)}}$^{\frown}$\record{\mfield{s-event}{$s$}{\textit{SEvent}}}

\item \record{\mfield{s-event}{$s$}{\textit{SEvent}}}$^{\frown}$\record{\tfield{e}{eat($m$)}}
 
\end{subex} 
   
\end{ex} 
Of course, we would expect an actual tense and aspect system for a
natural language to involve more complex types than this, for example,
allowing partial overlap between the eating event and the speech
event.  Our aim here is not to develop a realistic account of tense
but rather to show how we can distinguish between tensed and tenseless
contents in the kind of system we are proposing. The types in
\preveg{} can be derived from (\ref{ex:typeMaryeat}) by tense
operators which take  a speech event and a type as arguments and return
a new type.  These operators are defined in \nexteg{}.
\begin{ex} 
If $s$ : \textit{SEvent} and $T$ is a type, then
\begin{enumerate} 
 
\item pres($s$)($T$) = $T$ \d{$\wedge$} \record{\mfield{s-event}{$s$}{\textit{SEvent}}} 
 
\item past($s$)($T$) =
  $T^{\frown}$\record{\mfield{s-event}{$s$}{\textit{SEvent}}}

\item fut($s$)($T$) = \record{\mfield{s-event}{$s$}{\textit{SEvent}}}$^{\frown}T$ 
 
\end{enumerate} 
   
\end{ex} 
In a more complete treatment of tense we might want to generalize
these operators so that they can relate types to other kinds of events
in addition to speech events in order to be able to deal with embedded
tenses and phenomena like the historic present (as in \textit{So I was
  in the pub and this man \textbf{comes} up to me \ldots}).  

In addition to non-tensed contents for verbs, as illustrated by the
result of applying `SemIntransVerb' given in
(\ref{ex:SemIntransVerbeat}), we will also have tensed contents for
verbs.  Thus in addition to `SemIntransVerb' we will also have
`SemIntransVerb$_\alpha$' where $\alpha$ is one of `pres', `past' and
`fut'.  These functions will return a function from speech events to
parametric contents as given by the example in \nexteg{}.
\begin{ex} 
SemIntransVerb$_\alpha$(eat,
\textit{Ind}, \textit{Ind}, \textit{Rec}) = 

$\lambda s$:\textit{SEvent} . 
\record{\field{bg}{\textit{Rec}}\\
        \field{fg}{$\lambda c$:\textit{Rec} . 
                       \record{\field{bg}{\textit{Ind}}\\
                               \field{fg}{$\lambda
                                 r$:\smallrecord{\smalltfield{x}{\textit{Ind}}}
                                 . $\alpha(s)$(\record{\tfield{e}{eat($r$.x)}})}}}} 
\end{ex} 
This indicates that the contents of tensed expressions depend on
speech event in a way that non-tensed expressions do not.

We now turn our attention to how information about tense plays a role
in sign types.  Recall that in Chapter~\ref{ch:gram} we defined
\textit{Sign} as a recursive type whose witness condition is as in
\nexteg{}.  (See also Appendix~\ref{app:gramrulesuniv}.)
\begin{ex} 
$\sigma$ : \textit{Sign} iff $\sigma$ :\record{\tfield{s-event}{\textit{SEvent}} \\
         \tfield{syn}{\textit{Syn}} \\
        \tfield{cnt}{\textit{Cnt}}}  
\end{ex} 
Here the type \textit{Syn} (for ``syntax'') was defined as in
\nexteg{}.
\begin{ex} 
\record{\tfield{cat}{\textit{Cat}} \\
        \tfield{daughters}{\textit{Sign}$^*$}}  
\end{ex} 
Now we are going to add a further field to this type to indicate
whether a sign is tensed or non-tensed.  The new definition of
\textit{Syn} is given in \nexteg{}.
\begin{ex} 
\record{\tfield{cat}{\textit{Cat}} \\
        \tfield{tns}{\textit{Bool}} \\
        \tfield{daughters}{\textit{Sign}$^*$}}  
\end{ex} 
The definitions of the category sign types in Chapter~\ref{ch:gram}
(see Appendix~\ref{app:gramrulesuniv}) for \textit{S}, \textit{V} and
\textit{VP} can remain the same, since these categories are
underspecified for tense; they can be either tensed or non-tensed.  We
will use \nexteg{a} to represent the type \nexteg{b} and \nexteg{c} to
represent the type \nexteg{d} and we will do similarly for \textit{VP}
and \textit{S}.
\begin{ex} 
\begin{subex} 
 
\item $\displaystyle\V_{[+\mathrm{tns}]}$ 
 
\item \textit{Sign} \d{$\wedge$} 
\record{\tfield{syn}{\record{\mfield{cat}{v}{\textit{Cat}}\\
                             \mfield{tns}{1}{\textit{Bool}}}}}
         
\item $\displaystyle\V_{[-\mathrm{tns}]}$ 
 
\item \textit{Sign} \d{$\wedge$} 
\record{\tfield{syn}{\record{\mfield{cat}{v}{\textit{Cat}}\\
                             \mfield{tns}{0}{\textit{Bool}}}}} 
\end{subex} 
   
\end{ex} 
We will assume that the categories \textit{NP}, \textit{Det} and
\textit{N} are universally untensed\footnote{This is something of an
  open question.  See \cite{Tonhauser2007} for discussion.} and
therefore take \textit{NP} to be the type \nexteg{} and similarly
for \textit{Det} and \textit{N}.
\begin{ex} 
\textit{Sign} \d{$\wedge$} 
\record{\tfield{syn}{\record{\mfield{cat}{np}{\textit{Cat}}\\
                             \mfield{tns}{0}{\textit{Bool}}}}}  
\end{ex} 

We now define tensed versions of the universal resource for lexical
sign type construction, Lex$_{\mathrm{IntransVerb}}$ as defined in
Chapter~\ref{ch:commonnouns} (also in
Appendix~\ref{app:lexuniversal}). Letting $\alpha$ stand for `past',
`pres' or `fut' we characterize Lex$_{\mathrm{IntransVerb}_\alpha}$ as
in \nexteg{}.
\begin{ex} 
Lex$_{\mathrm{IntransVerb}_\alpha}(T_{\mathrm{phon}},p,T_{\mathrm{arg}},T_{\mathrm{restr}},T_{\mathrm{bg}})$,
where $T_{\mathrm{phon}}$ is a phonological type, $p$ is a predicate
  with arity $\langle T_{\mathrm{arg}}\rangle$,
  $T_{\mathrm{restr}}\sqsubseteq T_{\mathrm{arg}}$ and
  $T_{\mathrm{bg}}$ is a record type \\
is defined as \\
Lex($T_{\mathrm{phon}}$, \textit{VP}) \d{$\wedge$} 
\smallrecord{
\smalltfield{s-event}{\textit{SEvent}}\\
\smalltfield{syn}{\smallrecord{\smallmfield{tns}{1}{\textit{Bool}}}}\\
\smallmfield{cnt}{SemIntransVerb$_\alpha$($p$, $T_{\mathrm{arg}}$,
    $T_{\mathrm{restr}}$, $T_{\mathrm{bg}}$)(s-event)}{\textit{PPpty}}} 
\end{ex} 
We will use `Lex$_{\mathrm{IntransVerb}}$' (without the $\alpha$) to
construct sign types for non-finite verbs characterized by \nexteg{}.
\begin{ex} 
Lex$_{\mathrm{IntransVerb}}(T_{\mathrm{phon}},p,T_{\mathrm{arg}},T_{\mathrm{restr}},T_{\mathrm{bg}})$,
where $T_{\mathrm{phon}}$ is a phonological type, $p$ is a predicate
  with arity $\langle T_{\mathrm{arg}}\rangle$,
  $T_{\mathrm{restr}}\sqsubseteq T_{\mathrm{arg}}$ and
  $T_{\mathrm{bg}}$ is a record type \\
is defined as \\
Lex($T_{\mathrm{phon}}$, \textit{VP}) \d{$\wedge$} 
\smallrecord{
\smalltfield{syn}{\smallrecord{\smallmfield{tns}{0}{\textit{Bool}}}}\\
\smallmfield{cnt}{SemIntransVerb$_\alpha$($p$, $T_{\mathrm{arg}}$,
    $T_{\mathrm{restr}}$, $T_{\mathrm{bg}}$)}{\textit{PPpty}}} 
\end{ex}       

We now turn our attention to the modal verbs.  The parametric content
of a modal verb (such as \textit{should}) is a function which requires a background with a modal
base (a type) and a topos.  Given such a background this function
returns a function from properties (such as \textit{eat}) to
properties (such as \textit{should eat}).  We define
`SemModalVerb$_{\mathrm{nec}}$' and `SemModalVerb$_{\mathrm{poss}}$'
as \nexteg{a and b} respectively.
\begin{ex} 
\begin{subex} 
 
\item \record{\field{bg}{\record{\tfield{base}{\textit{Type}}\\
                                 \tfield{topos}{\textit{Topos}}}}\\
              \field{fg}{$\lambda c$:\smallrecord{\smalltfield{base}{\textit{Type}}\\
                                 \smalltfield{topos}{\textit{Topos}}}
                               . \\
& & \record{\field{bg}{\textit{Ppty}}\\
        \field{fg}{$\lambda P$:\textit{Ppty} . \\ 
& & \record{\field{bg}{\textit{Ind}}\\
        \field{fg}{$\lambda
          r$:\smallrecord{\smalltfield{x}{\textit{Ind}}} . \\
& & \record{\tfield{e}{nec($P$.fg($r$), $c$.base, $c$.topos)}}}}}}}}
 
\item \record{\field{bg}{\record{\tfield{base}{\textit{Type}}\\
                                 \tfield{topos}{\textit{Topos}}}}\\
              \field{fg}{$\lambda c$:\smallrecord{\smalltfield{base}{\textit{Type}}\\
                                 \smalltfield{topos}{\textit{Topos}}}
                               . \\
& & \record{\field{bg}{\textit{Ppty}}\\
        \field{fg}{$\lambda P$:\textit{Ppty} . \\
& & \record{\field{bg}{\textit{Ind}}\\
        \field{fg}{$\lambda
          r$:\smallrecord{\smalltfield{x}{\textit{Ind}}} . \\
& & \record{\tfield{e}{poss($P$.fg($r$), $c$.base, $c$.topos)}}}}}}}} 
 
\end{subex} 
   
\end{ex} 
  

The type, \textit{Modal}, of modal parametric contents, that is, a
type of objects like those in \preveg{}, is given in \nexteg{}.
\begin{ex} 
\record{\mfield{bg}{\smallrecord{\smalltfield{base}{\textit{Type}}\\
                                 \smalltfield{topos}{\textit{Topos}}}}{\textit{Type}}\\
       \tfield{fg}{(bg$\rightarrow$(\textit{Ppty}$\rightarrow$\textit{Ppty}))}} 
\end{ex} 

We will introduce a syntactic category for modal verbs, `vm'.  Thus we
will now characterize the type, \textit{Cat} as in \nexteg{}.
\begin{ex} 
s, np, det, n, v, vp, vm : \textit{Cat} 
\end{ex} 
We will use the symbol $\displaystyle\V_{[+M]}$ to represent the type
\nexteg{}.
\begin{ex} 
\textit{Sign} \d{$\wedge$} \record{\tfield{syn}{\record{\mfield{cat}{vm}{\textit{Cat}}}}} 
\end{ex} 
We can now characterize a universal resource, Lex$_{\mathrm{ModalV}}$,
for creating lexical sign types for modal verbs.  This is done in
\nexteg{}.
\begin{ex} 
If $T_{\mathrm{phon}}$ is a phonological type and $p$ is either `nec'
or `poss', then Lex$_{\mathrm{ModalV}}$($T_{\mathrm{phon}}$, $p$)\\
is defined as\\
Lex($T_{\mathrm{phon}}$, $\displaystyle\V_{[+M]}$) \d{$\wedge$}\\
\hspace*{2em}\smallrecord{\smalltfield{s-event}{\textit{SEvent}}\\
        \smalltfield{syn}{\smallrecord{\smallmfield{tns}{1}{\textit{Bool}}}}\\
        \smallmfield{cnt}{
\smallrecord{\field{bg}{\smallrecord{\smalltfield{base}{\textit{Type}}\\
                                 \smalltfield{topos}{\textit{Topos}}}}\\
             \field{fg}{$\lambda c$:\smallrecord{\smalltfield{base}{\textit{Type}}\\
                                 \smalltfield{topos}{\textit{Topos}}}
                               . 
pres(s-event)(SemModalVerb$_p$.fg($c$))}}}{\textit{Modal}}}
\end{ex} 
         
The lexical resources for English can now tell us that \textit{should}
is a modal verb of necessity, as in \nexteg{}.
\begin{ex} 
Lex$_{\mathrm{ModalV}}$(``should'', nec) 
\end{ex} 
Finally, English resources will need to include the tense and modal
sensitive phrase structure rules in \nexteg{} (using the abbreviatory
conventions of Appendix~\ref{app:interpps}).
\begin{ex} 
\begin{subex} 
 
\item $\displaystyle\Sent_{[+\mathrm{tns}]}$ $\longrightarrow$ \textit{NP}
  $\displaystyle\VP_{[+\mathrm{tns}]}$ $\mid$ \textit{NP}$'$@\textit{VP}$'$ 
 
\item $\displaystyle\VP_{[+\mathrm{tns}]}$ $\longrightarrow$
  $\displaystyle\V_{[+M]}$ $\displaystyle\VP_{[-\mathrm{tns}]}$ $\mid$ \textit{V}$'$@\textit{VP}$'$ 
 
\end{subex} 
   
\end{ex} 

%%% Local Variables:
%%% mode: latex
%%% TeX-master: "ttl"
%%% End:
