% by
% refining our update functions with the type which has been assigned to
% the current information state.  If we have an update function
% \nexteg{a} then an update-refinement of this function is another
% function \nexteg{b} where both the domain type, $T_i'$, and the result type,$T_{i+1}'$ are
% subtypes of the original domain type, $T_i$ and result type, $T_{i+1}$, that is, they
% place more restrictions on what the current information state and the
% new information state can be.
% \begin{ex} 
% \begin{subex} 
 
% \item $\lambda r:T_i(\lambda u:T_{\mathit{utt}}(T_{i+1}))$ 
 
% \item $\lambda r:T_i'(\lambda u:T_{\mathit{utt}}(T_{i+1}'))$ 
 
% \end{subex} 
   
% \end{ex} 
  


and a hypothesized type of the current
information state $T_{\mathit{curr}}$, (\ref{eg:updateFun}) will
\textit{fire} only if $T_{\mathit{curr}} \sqsubseteq T_i$ and we can
refine (\ref{eg:updateFun}) to 

$\lambda r:T_{\mathit{curr}}$($\lambda
e:T_e(r)$($T_{i+1}(r,e)$\d{$\wedge$}$(T_{\mathit{curr}}[r]-T_{i+1})$))

That is, we require the new information state type to be a simplified
conjunction\footnote{The meet (or conjunction) $T$ of two types $T_1\wedge
  T_2$ is such that $a:T$ iff $a:T_1$ and $a:T_2$.  Suppose that $T_1$
  and $T_2$ are record types.  Then $T_1$\d{$\wedge$}$T_2$ is a record
  type which is equivalent to $T_1\wedge T_2$, that is
  $a:T_1$\d{$\wedge$}$T_2$ iff $a:T_1\wedge T_2$.  As an example
  consider
  \smallrecord{\smalltfield{f}{$T_1$}}$\wedge$\smallrecord{\smalltfield{g}{$T_2$}}
  which is equivalent to \smallrecord{\smalltfield{f}{$T_1$} \\
                                      \smalltfield{g}{$T_2$}} which is
  defined to be \smallrecord{\smalltfield{f}{$T_1$}}\d{$\wedge$}\smallrecord{\smalltfield{g}{$T_2$}}.}
of what is required by the unrefined rule with the current information
state type grounded in $r$ (i.e. with each path-name $\pi$ referring within
$T_{\mathit{curr}}$ replaced with $r.\pi$) but minus any paths that
occur in $T_{i+1}$.  Thus the refinement of
(\ref{eg:updateFunIntegOwn}) with (\ref{eg:initialInfoDiac}) is

$\lambda
r$:\record{\tfield{private}{\record{\mfield{agenda}{$\left[\mbox{\smallrecord{\smallmfield{actor}{User}{\textit{Ind}} \\
                                                              \smallmfield{audience}{System}{\{\textit{Ind}\}} \\
                                                              \smallmfield{i-force}{assert}{\textit{IForce}} \\
                                                              \smallmfield{content}{\smallrecord{\smalltfield{c}{conductor(Dudamel)}}}{\textit{RecType}} \\
                                                              \smalltfield{c}{move(actor, audience, i-force, content)}}}\right]$}{[\textit{RecType}]}}} \\
        \tfield{shared}{\record{\tfield{latest-utterance}{\record{\mfield{moves}{$\emptyset$}{\{\textit{Move}\}} \\
                                                                  \mfield{chart}{$\emptyset$}{\textit{Chart}}}}\\
                               \mfield{commitments}{[]}{\textit{RecType}}}}}
\\
\hspace*{2em}$\lambda e$:\record{\tfield{moves}{\{\textit{Move}\}} \\
                                \tfield{m}{$\in$moves$\wedge$fst($r$.private.agenda)$\wedge$\record{\mfield{actor}{User}{\textit{Ind}}}} \\
                                \tfield{chart}{\textit{Chart}}} \\
\hspace*{4em}(\record{\tfield{private}{\record{\mfield{agenda}{rst($r$.is.private.agenda)}{[\textit{Move}]}}}
  \\
                      \tfield{shared}{\record{\tfield{latest-utterance}{\record{\mfield{moves}{$e$.moves}{\{\textit{Move}\}} \\
                                                                                \mfield{chart}{$e$.chart}{\textit{Chart}}}}}}}
               \d{$\wedge$} \\
 \hspace*{8em}             \record{\tfield{shared}{\record{\mfield{commitments}{[]}{\textit{RecType}}}}})

which can be simplified to

$\lambda
r$:\record{\tfield{private}{\record{\mfield{agenda}{$\left[\mbox{\smallrecord{\smallmfield{actor}{User}{\textit{Ind}} \\
                                                              \smallmfield{audience}{System}{\{\textit{Ind}\}} \\
                                                              \smallmfield{i-force}{assert}{\textit{IForce}} \\
                                                              \smallmfield{content}{\smallrecord{\smalltfield{c}{conductor(Dudamel)}}}{\textit{RecType}} \\
                                                              \smalltfield{c}{move(actor, audience, i-force, content)}}}\right]$}{[\textit{RecType}]}}} \\
        \tfield{shared}{\record{\tfield{latest-utterance}{\record{\mfield{moves}{$\emptyset$}{\{\textit{Move}\}} \\
                                                                  \mfield{chart}{$\emptyset$}{\textit{Chart}}}}\\
                               \mfield{commitments}{[]}{\textit{RecType}}}}}
\\
\hspace*{2em}$\lambda e$:\record{\tfield{moves}{\{\textit{Move}\}} \\
                                \tfield{m}{$\in$moves$\wedge$fst($r$.private.agenda)$\wedge$\record{\mfield{actor}{User}{\textit{Ind}}}} \\
                                \tfield{chart}{\textit{Chart}}} \\
\hspace*{4em}(\record{\tfield{private}{\record{\mfield{agenda}{rst($r$.is.private.agenda)}{[\textit{RecType}]}}}
  \\
                      \tfield{shared}{\record{\tfield{latest-utterance}{\record{\mfield{moves}{$e$.moves}{\{\textit{Move}\}} \\
                                                                                \mfield{chart}{$e$.chart}{\textit{Chart}}}} \\
                                              \mfield{commitments}{[]}{\textit{RecType}}}}})

Given a supposition that the current information state is of type
(\ref{eg:initialInfoDiac}) and that there is an event of type 

\record{\mfield{moves}{\{$m$\}}{\{\textit{RecType}\}} \\
        \mfield{m}{$m$}{\smallrecord{\smallmfield{actor}{User}{\textit{Ind}} \\
                       \smallmfield{audience}{System}{\{\textit{Ind}\}} \\
                       \smallmfield{i-force}{assert}{\textit{IForce}} \\
                       \smallmfield{content}{\smallrecord{\smalltfield{c}{conductor(Dudamel)}}}{\textit{RecType}} \\
                       \smalltfield{c}{move(actor, audience, i-force,
                         content)}}} \\
        \mfield{chart}{$K$}{\textit{Chart}}}

then we can come to the conclusion that the new information state is
of type

\record{\tfield{private}{\record{\mfield{agenda}{[]}{[\textit{RecType}]}}}
  \\
                      \tfield{shared}{\record{\tfield{latest-utterance}{\record{\mfield{moves}{\{$m$\}}{\{\textit{Move}\}} \\
                                                                                \mfield{chart}{$K$}{\textit{Chart}}}} \\
                                              \mfield{commitments}{[]}{\textit{RecType}}}}}



Note that this
function is specific to a particular agent `User' since it makes
reference to the user in characterizing the actor of the perceived
event.  In order to give an agent independent characterization of the
update function we need to add another argument to it for the agent's
``self'':

$\lambda a$:\record{\tfield{self}{\textit{Ind}}} \\
\hspace*{2em}$\lambda
r$:\record{\tfield{private}{\record{\tfield{agenda}{$_{\mathit{ne}}$[\textit{RecType}]}}}}
\\
\hspace*{4em}$\lambda e$:\record{\tfield{moves}{\{\textit{Move}\}} \\
                                \tfield{m}{$\in$moves$\wedge$fst($r$.private.agenda)$\wedge$\record{\mfield{actor}{$a$.self}{\textit{Ind}}}} \\
                                \tfield{chart}{\textit{Chart}}} \\
\hspace*{6em}(\record{\tfield{private}{\record{\mfield{agenda}{rst($r$.is.private.agenda)}{[\textit{RecType}]}}}
  \\
                      \tfield{shared}{\record{\tfield{latest-utterance}{\record{\mfield{moves}{$e$.moves}{\{\textit{Move}\}} \\
                                                                                \mfield{chart}{$e$.chart}{\textit{Chart}}}}}}})

The stock of update functions that an agent has available result from
applying functions like this to a record representing the agent's view
of itself.


\\
                                              \mfield{commitments}{$r$.shared.commitments\d{$\wedge$}$e$.m.content}{\textit{RecType}}


 \\
           \tfield{shared}{\record{\tfield{commitments}{\textit{RecType}}

This raises, of course,  the question of how the user discovers what
the content of the acknowledgement is, given that the system only says
\textit{ok}.  This issue is not addressed by the update function.
Rather we see this as a problem of assigning an appropriate move type
to the system's utterance and thus we will return to this question
below when we discuss semantics and we will treat it as a case of
anaphora resolution which is to be solved by reference to content
introduced in the latest-utterance field of the information state.

% $\lambda a$:\record{\tfield{self}{\textit{Ind}}} \\
% \hspace*{2em}$\lambda
% r$:\record{\tfield{shared}{\record{\tfield{commitments}{\textit{RecType}}}}}
% \\
% \hspace*{4em}$\lambda e$:\record{\tfield{moves}{\{\textit{Move}\}} \\
%                                 \tfield{m}{$\in$moves$\wedge$\record{\tfield{actor}{\textit{Ind}--\textit{Ind}$_{a.\mathrm{self}}$} \\
%                                                                     \mfield{i-force}{acknowledge}{\textit{IForce}} \\
%                                                                     \tfield{content}{\textit{RecType}}}} \\
%                                 \tfield{chart}{\textit{Chart}}} \\
% \hspace*{6em}(\record{\tfield{shared}{\record{\smalltfield{latest-utterance}{\record{\mfield{moves}{$e$.moves}{\{\textit{Move}\}} \\
%                                                                                      \mfield{chart}{$e$.chart}{\textit{Chart}}}} \\
%                                               \mfield{commitments}{$r$.shared.commitments\d{$\wedge$}$e$.m.content}{\textit{RecType}}}}})
