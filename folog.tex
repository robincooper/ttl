\chapter{First order logic}
\label{app:folog}

\section{Interpreting first order logic in TTR}

We assume a standard definition of the syntax of first order logic with a set of
constants $C$, a countably infinite set of variables $V$ and sets of
$n$-ary predicate symbols $\mathcal{P}^n$.  In order to embed first order
logic we shall use an intensional system of complex types {\bf TYPE$_\mathit{IC}$} = $\langle${\bf Type}$^n$, {\bf BType},
$\langle$\textbf{PType}$^n$, {\bf Pred}, \textbf{ArgIndices}, {\it
  Arity\/}$\rangle$, $\langle
A,F^n\rangle$$\rangle_{n\in\mathit{Nat}}$ with dependent record types
based on $\langle L, \mathbf{RType}^n\rangle_{n\in\mathit{Nat}}$
where:
\begin{enumerate} 
 
\item \textit{Ind} $\in$ \textbf{BType} (the type of individuals)
 
\item we introduce a function $\kappa$ from $C$ to $A(\mathit{Ind})$
  (a function from constants of first order logical to objects of type
  \textit{Ind})

\item we introduce a one-one function $\gamma$ from $V$ into $L$ (a
  one-one function from variables of first order logic into the set of
  labels used in record types)

\item there is a countably infinite proper subset of $L$, $L_c$, whose members are
  represented by c$_i$ where $i$ is a natural number.  (This is the
  set of labels used for ``constraints'' corresponding to formulae of
  first order logic.)  $L_c$ and the range of $\gamma$ are disjoint
  (i.e. the labels used for constraints are distinct from those which
  correspond to the variables of first order logic).

\item we introduce a function $\rho$ from $\bigcup_n\mathcal{P}^n$ to
  \textbf{Pred} such that if $P\in\mathcal{P}^n$ then
  $\mathit{Arity}(\rho(P))\in\{\mathit{Ind}\}^n$ (a function from
  predicate symbols of first order logic to predicates of corresponding arity
  in the type theory)
 
\end{enumerate} 

We define a function \mng{.}$^{.,.,.}$ which maps an expression, $\alpha$ of first order
logic to a function from a countably infinite sequence, $\sigma$, of members of $A(\mathit{Ind})$ to
type theoretic objects.  (Below we shall refer to $\sigma$ simply as a
sequence of individuals.)  The mapping is relative to a natural number,
$n$, a path, $\pi$, and a function, $g$, from the set of first order
logic variables, $V$, to natural numbers. We define \fotrans{$\alpha$}
as follows:
\begin{enumerate} 
 
\item if $\alpha\in C$ (a constant), then \fotrans{$\alpha$} is that
  function $f$ such that for any sequence of individuals, $\sigma$, $f(\sigma)=\kappa(\alpha)$.
 
\item if $\alpha\in V$ (a variable),
  then \fotrans{$\alpha$} is that function $f$ such that for any
  sequence of individuals, $\sigma$, $f(\sigma)=\sigma_{g(\alpha)}$.

\item if $\alpha\in\mathcal{P}^k$, for some $k$, (an $k$-place
  predicate symbol), then \fotrans{$\alpha$} is that function $f$ such
    that for any sequence of individuals, $\sigma$, $f(\sigma)=\rho(P)$.

\item If $P$ is an $k$-place predicate symbol of first order logic and
  $t_1,\ldots,t_k$ are terms (constants or variables) of first order
  logic and $\alpha=P(t_1,\ldots,t_k)$, then \fotrans{$\alpha$} is
  that function $f$ such that for any sequence of individuals,
  $\sigma$,
  $f(\sigma)=$\record{\tfield{c$_n$}{\fotrans{$P$}($\sigma$)(\fotrans{$t_1$}($\sigma$),\ldots,\fotrans{$t_k$}($\sigma$))}}.

\item if $\phi$ and $\psi$ are formulae of first order logic and
  $\alpha=\phi\wedge\psi$, then \fotrans{$\alpha$} is that function
  $f$ such that for any sequence of individuals, $\sigma$,
  $f(\sigma)=$\record{\tfield{c$_n$}{\mng{$\phi$}}$^{n+1,\pi,g}$($\sigma$) \d{$\wedge$}
  \mng{$\psi$}$^{n+2,\pi,g}$($\sigma$)}.

\item if $\phi$ and $\psi$ are formulae of first order logic and
  $\alpha=\phi\vee\psi$, then \fotrans{$\alpha$} is that function
  $f$ such that for any sequence of individuals, $\sigma$,
  $f(\sigma)=$\record{\tfield{c$_n$}{\mng{$\phi$}$^{n+1,\pi,g}$($\sigma$) $\vee$
  \mng{$\psi$}$^{n+2,\pi,g}$($\sigma$)}}.

\item \label{focl:implication} if $\phi$ and $\psi$ are formulae of first order logic and
  $\alpha=\phi\rightarrow\psi$, then \fotrans{$\alpha$} is that function
  $f$ such that for any sequence of individuals, $\sigma$,
  $f(\sigma)=$\record{\tfield{c$_n$}{\mng{$\phi$}$^{n+1,\pi,g}$($\sigma$) $\rightarrow$
  \mng{$\psi$}$^{n+2,\pi,g}$($\sigma$)}}.

\item \label{focl:negation} if $\phi$ is a formula of first order logic and
  $\alpha=\neg\phi$, then \fotrans{$\alpha$} is that function
  $f$ such that for any sequence of individuals, $\sigma$, $f(\sigma)=$\record{\tfield{c$_n$}{\mng{$\phi$}$^{n+1,\pi,g}$($\sigma$) $\rightarrow$
  $\bot$}}.

\item if $\phi$ is a formula and $x$ is a variable of first order
  logic and $\alpha=\exists x\phi$, then \fotrans{$\alpha$} is that function
  $f$ such that for any sequence of individuals, $\sigma$,
  $f(\sigma)=$
\record{\tfield{$\gamma(x)$}{\textit{Ind}} \\
        \tfield{c$_n$}{$\langle\lambda v_n:$
          \textit{Ind}(\mng{$\phi$}$^{n+1,\pi.c_n,g(n/x)}$($\sigma(v_n/n)$)),
          $\langle\pi.\gamma(x)\rangle\rangle$}}.

\item if $\phi$ is a formula and $x$ is a variable of first order
  logic and $\alpha=\forall x\phi$, then \fotrans{$\alpha$} is that function
  $f$ such that for any sequence of individuals, $\sigma$,
  $f(\sigma)=$
\smallrecord{\smalltfield{c$_n$}{($r_n$:\smallrecord{\smalltfield{$\gamma(x)$}{\textit{Ind}}})
    $\rightarrow$ $\lambda r_{n+1}:$\smallrecord{\smalltfield{$\gamma(x)$}{\textit{Ind}}}(\mng{$\phi$}$^{n+2,\emptyset,g(n+1/x)}$($\sigma(r_{n+1}.\gamma(x)/n+1)$))($r_n$)}}.


  
 
\end{enumerate}

The \textit{TTR interpretation}, \mng{$\alpha$}$^g$, \textit{of a formula}, $\alpha$, \textit{of first
order logic with respect to an assignment}, $g$, \textit{of natural numbers to
variables} is \mng{$\alpha$}$^{0,\emptyset,g}$.

If $\alpha$ is a sentence (a closed formula) of first order logic,
then for any sequences $\sigma$ and $\sigma'$ and any assignments to
variables $g$ and $g'$,
\mng{$\alpha$}$^g$($\sigma$)=\mng{$\alpha$}$^g$($\sigma'$) and
\mng{$\alpha$}$^g$($\sigma$)=\mng{$\alpha$}$^{g'}$($\sigma$). Therefore
we say that the \textit{TTR interpretation}, \mng{$\alpha$} \textit{of a sentence},
$\alpha$, \textit{of first order logic} is \mng{$\alpha$}$^g$($\sigma$) for
some arbitrarily chosen $g$ and $\sigma$.

\section{Examples of first order logic interpretations}

For the sake of readability let us assume that any first order
logic variable $x$ is mapped by $\gamma$ to a TTR label represented by
`x'. Similarly any constant $c$ is mapped by $\kappa$ to a TTR object
represented by `c' and any predicate symbol $P$ is mapped by $\rho$ to
a predicate `P'.  Let us further assume that `j' and `m' are constants,
`run' and `person' are one-place predicate symbols and `hug' is a
two-place predicate symbol.  Then
\mng{run(j)} is 
\begin{display}
\record{\tfield{c$_0$}{run(j)}}
\end{display}
\mng{run(j)$\wedge$run(m)} is
\begin{display}
\record{\tfield{c$_0$}{\record{\tfield{c$_1$}{run(j)} \\
                               \tfield{c$_2$}{run(m)}}}}
\end{display}
which can be flattened and relabelled to
\begin{display}
\record{\tfield{c$_0$}{run(j)} \\
        \tfield{c$_1$}{run(m)}}
\end{display}
\mng{(run(j)$\wedge$person(j))$\wedge$run(m)} is
\begin{display}
\record{\tfield{c$_0$}{\record{\tfield{c$_1$}{\record{\tfield{c$_2$}{run(j)} \\
                                                      \tfield{c$_3$}{person(j)}}\\
                               \tfield{c$_2$}{run(m)}}}}}
\end{display}
which can be flattened and relabelled to 
\begin{display}
\record{\tfield{c$_0$}{run(j)} \\
        \tfield{c$_1$}{person(j)} \\
        \tfield{c$_2$}{run(m)}}
\end{display}
\mng{run(j)$\vee$run(m)} is
\begin{display}
\record{\tfield{c$_0$}{\record{\tfield{c$_1$}{run(j)}}$\vee$\record{\tfield{c$_2$}{run(m)}}}}
\end{display}
\mng{run(j)$\rightarrow$run(m)} is
\begin{display}
\record{\tfield{c$_0$}{\record{\tfield{c$_1$}{run(j)}}$\rightarrow$\record{\tfield{c$_2$}{run(m)}}}}
\end{display}
\mng{$\exists x \mathrm{person}(x)\wedge\mathrm{run}(x)$} is
\begin{display}
\record{\tfield{x}{\textit{Ind}} \\
        \tfield{c$_0$}{$\langle\lambda v_0:$\textit{Ind}
                                 (\record{\tfield{c$_1$}{\record{\tfield{c$_2$}{person($v_0$)}
                                       \\
                                                                 \tfield{c$_3$}{run($v_0$)}}}}),
                                                           $\langle$x$\rangle\rangle$}}
\end{display}
which by using the abbreviatory notation for dependencies within
record types, flattening and relabelling can be reduced to
\begin{display}
\record{\tfield{x}{\textit{Ind}} \\
        \tfield{c$_0$}{person(x)} \\
        \tfield{c$_1$}{run(x)}}
\end{display}
\mng{$\forall x \mathrm{person}(x)\rightarrow\mathrm{run}(x)$} is
\begin{display}
\record{\tfield{c$_0$}{($r_0$:\smallrecord{\tfield{x}{\textit{Ind}}})$\rightarrow$
                              $\lambda
                              r_1$:\smallrecord{\tfield{x}{\textit{Ind}}}(\smallrecord{\smalltfield{c$_2$}{\smallrecord{\smalltfield{c$_3$}{person($r_1$.x)}}$\rightarrow$\smallrecord{\smalltfield{c$_4$}{run($r_1$.x)}}}})($r_0$)}}
\end{display}
that is,
\begin{display}
\record{\tfield{c$_0$}{($r_0$:\smallrecord{\tfield{x}{\textit{Ind}}})$\rightarrow$
                              \smallrecord{\smalltfield{c$_2$}{\smallrecord{\smalltfield{c$_3$}{person($r_0$.x)}}$\rightarrow$\smallrecord{\smalltfield{c$_4$}{run($r_0$.x)}}}}}}
\end{display}

\section{Examples of inference in first order logic}

\paragraph{Implication.}  In classical first order logic implication
is represented by the material condition, that is, we have:
\begin{display}
$\phi\rightarrow\psi \leftrightarrow \neg\phi\vee\psi$
\end{display}
Implication is interpreted in terms of a function type by
clause~\ref{focl:implication} of our definition.  We obtain the
material conditional if we assume that for each pair of types, $T_1,T_2$,
$\{f\mid f:T_1\rightarrow T_2\}$ includes a function for each function
graph (i.e. set of ordered pairs representing a function) from members
of $T_1$ to members of $T_2$.  This means that there will be a
function of type $T_1\rightarrow T_2$ just in case either $T_1$ has no
members (the empty function) or $T_2$ is non-empty (in which case
there will be some function which maps each member of $T_1$ to some
member of $T_2$).  There will fail to be a function of type
$T_1\rightarrow T_2$ just in case $T_1$ is non-empty and $T_2$ is
empty.\footnote{In order to obtain an intuitionistic interpretation of
  implication we need to make a different assumption about the
  membership of function types, namely that there can be function
  graphs for which there is no corresponding function.  There may be
  certain sets of ordered pairs which are function graphs for which we
  do not have a method of computing the second member of the ordered
  pairs from the first and thus there is not a function in the
  intuitionistic sense.}

\paragraph{Negation.}  Classical negation gives us the following
validities:
\begin{display}
$\phi\leftrightarrow\neg\neg\phi$

$\phi\vee\neg\phi$

\end{display}
In clause~\ref{focl:negation} we have interpreted negation using the
standard strategy from intuitionistic logic, though cast in terms of a
function type.  However, the result will be classical negation if we
make the same assumption about the function space as we did in order
to obtain material implication.  A type $T\rightarrow\bot$ will be
non-empty (witnessed by a function corresponding to the empty function
graph) just in case $T$ is empty.  Thus $T$ will be empty just in case
$(T\rightarrow\bot)\rightarrow\bot$ is empty.  Note that $T$ and
$(T\rightarrow\bot)\rightarrow\bot$ are not the same type and will not
have the same members.  From the perspective of type theory
equivalence in classical logic only depends on whether types are empty
or not.  Given this interpretation of negation we also get the law of
the excluded middle since there is no third choice between a type
being empty or not.\footnote{As negation is interpreted in terms of
  function types, making an assumption about functions that would give
  us an intuitionistic interpretation of implication would also give
  us an intuitionistic interpretation of negation.}

\paragraph{Existential quantification.}  In first order logic we have
\begin{display}
$\exists x \exists y R(x,y) \leftrightarrow \exists y
\exists x R(x,y)$
\end{display}
\mng{$\exists x \exists y R(x,y)$} is
\begin{display}
\record{\tfield{x}{\textit{Ind}} \\
        \tfield{c$_0$}{$\langle\lambda v_0:$\textit{Ind}
                                 (\record{\tfield{y}{\textit{Ind}} \\
\tfield{c$_1$}{$\langle\lambda v_1:$\textit{Ind}(\record{\tfield{c$_2$}{$R(v_0,v_1)$}}),
                                                           $\langle$c$_0$.y$\rangle\rangle$}}),
                                                           $\langle$x$\rangle\rangle$}}
\end{display}
and \mng{$\exists y \exists x R(x,y)$} is
\begin{display}
\record{\tfield{y}{\textit{Ind}} \\
        \tfield{c$_0$}{$\langle\lambda v_0:$\textit{Ind}
                                 (\record{\tfield{x}{\textit{Ind}} \\
\tfield{c$_1$}{$\langle\lambda v_1:$\textit{Ind}(\record{\tfield{c$_2$}{$R(v_1,v_0)$}}),
                                                           $\langle$c$_0$.x$\rangle\rangle$}}),
                                                           $\langle$y$\rangle\rangle$}}
\end{display}
Both of these interpretations, using the abbreviatory notation for
dependencies, flattening and relabelling, simplify to\footnote{Recall
  that record types are sets of fields, thus the x and y-fields are
  not ordered with respect to each other.} 
\begin{display}
\record{\tfield{x}{\textit{Ind}}\\
        \tfield{y}{\textit{Ind}}\\
        \tfield{c$_0$}{$R$(x,y)}}
\end{display}

\paragraph{Universal quantification.}   In first order logic we have
\begin{display}
$\forall x \forall y R(x,y) \leftrightarrow \forall y
\forall x R(x,y)$
\end{display}
\mng{$\forall x \forall y R(x,y)$} is
\begin{display}
\record{\tfield{c$_0$}{($r_0$:\smallrecord{\tfield{x}{\textit{Ind}}})$\rightarrow$
                              $\lambda
                              r_1$:\smallrecord{\tfield{x}{\textit{Ind}}}(\smallrecord{\smalltfield{c$_2$}{($r_2$:\smallrecord{\tfield{y}{\textit{Ind}}})$\rightarrow
                                \lambda r_3$:\smallrecord{\tfield{y}{\textit{Ind}}}(\smallrecord{\smalltfield{c$_3$}{$R$($r_1$.x,$r_3$.y)}})($r_0$)}})($r_2$)}}
\end{display}
that is,
\begin{display}
\record{\tfield{c$_0$}{($r_0$:\smallrecord{\tfield{x}{\textit{Ind}}})$\rightarrow$
                              \smallrecord{\smalltfield{c$_2$}{($r_2$:\smallrecord{\tfield{y}{\textit{Ind}}})$\rightarrow$
                                \smallrecord{\smalltfield{c$_3$}{$R$($r_0$.x,$r_2$.y)}}}}}}
\end{display}
 \mng{$\forall y \forall x R(x,y)$} is
\begin{display}
\record{\tfield{c$_0$}{($r_0$:\smallrecord{\tfield{y}{\textit{Ind}}})$\rightarrow$
                              \smallrecord{\smalltfield{c$_2$}{($r_2$:\smallrecord{\tfield{x}{\textit{Ind}}})$\rightarrow$
                                \smallrecord{\smalltfield{c$_3$}{$R$($r_2$.x,$r_0$.y)}}}}}}
\end{display} 
Both of these interpretations are equivalent to 
\begin{display}
\record{\tfield{c$_0$}{($r_0$:\smallrecord{\tfield{x}{\textit{Ind}}\\
                                           \tfield{y}{\textit{Ind}}})$\rightarrow$
                              \smallrecord{\smalltfield{c$_2$}{$R$($r_0$.x,$r_0$.y)}}}}
\end{display}
where there is no order between x and y. 

