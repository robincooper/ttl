\subsection{Common nouns and frames}

\subsection{Definite descriptions}

\subsection{Indefinite descriptions}

\subsection{Generalized quantifiers}

\subsection{Conjunctions and disjunctions}

\subsection{Plurality}

\newpage

\section{Verbal constructions}

\subsection{Intransitive verbs}

\subsection{Transitive verbs}

\subsection{Intensional verbs}

\subsection{Verbs expressing propositional attitudes}

\subsection{Verbs expressing attitudes to properties}

\subsection{Conjunction and disjunction}

\subsection{Negation}

\subsection{Tense and aspect}

\section{Adverbial constructions}

\subsection{Verb phrase adverbs}

\subsection{Intensional adverbs}

\subsection{Modal adverbs}

\subsection{Prepositional phrases}

\newpage

\section{Long distance dependencies}

\subsection{Relative clauses}

\subsection{\textit{wh}-questions}

\subsection{Quantifier scope ambiguities}

\section{Sentential phenomena}

\subsection{Conjunction and disjunction}

\subsection{Polar Questions}

\subsection{Imperatives}

\section{Anaphora}

\subsection{Intrasentential anaphora}

\subsection{Intersentential anaphora}

\subsection{Anaphora in dialogue}

\section*{Scrap}

In this section we will recreate part of Montague's PTQ fragment using the
kind of type system we used for the interpretation of intensional
logic.  We will concentrate our attention on the semantics of
verbs.\footnote{For more general discussion of the application of TTR to
  semantics see, for example, \cite{Cooper2005a,Fernandez2006,Ginzburgforthcoming}.}  In order to
do this we will introduce a basic type \textit{Str} (for strings of
English words which we will use as an imprecise representation of
phonology) and a basic type \textit{Cat} to which will belong the
syntactic categories that Montague uses.  Montague's categories
are defined as follows:
\begin{quote}
e:\textit{Cat} \\
t:\textit{Cat} \\
if $A$:\textit{Cat} and $B$:\textit{Cat} then $A/B$:\textit{Cat} and
$A//B$:\textit{Cat}

Like Montague, we use the following abbreviations for categories:\\
IV, ``intransitive verb phrases'' is to be t/e \\
T, ``term (or noun) phrases'' is to be t/IV \\
TV, ``transitive verb phrases'' is to be IV/T \\
% IAV, ``IV-modifying adverbs'' is to be IV/IV \\
CN, ``common noun phrases'' is to be t//e
\end{quote}
We will present our grammar in terms of signs (in a similar sense to
HPSG, see, for example, \cite{Sag:Wasow:ea:03}).  Our signs will be
records of a type
\begin{display}
\record{\tfield{s-event}{\textit{SEvent}}
                               \\
        \tfield{synsem}{\record{\tfield{cat}{\textit{Cat}}\\
                                \tfield{cnt}{\textit{Seq}$_{\mathit{Ind}}\rightarrow
                                  T$}}}}
\end{display}
where \textit{SEvent} is the type 
\begin{quote}
\record{\tfield{phon}{\textit{Str}}\\
                                 \tfield{s-time}{\record{\tfield{start}{\textit{Time}}
                                     \\
                                                         \tfield{end}{\textit{Time}}}}\\
                                 \tfield{utt$_\mathrm{at}$}{$\langle
                                   \lambda v_1:$\textit{Str}$(\lambda
                                   v_2:$\textit{Time}($\lambda
                                   v_3:$\textit{Time}(uttered\_at($v_1$,$v_2$,$v_3$)))), \\
                                   & &
                                   \hspace*{2em}$\langle$s-event.phon,
                                   s-event.s-time.start,
                                   s-event.s-time.end$\rangle\rangle$}}
\end{quote}
and \textit{Seq}$_{\mathit{Ind}}$ is an abbreviation for the type
\textit{Nat}$\rightarrow$\textit{Ind}, that is the type of sequences
of individuals, and $T$ is \smallrecord{\smalltfield{x}{\textit{Ind}}},
\textit{RecType} or a function type constructed from these. 
A sign 
has two main components, one corresponding to the physical nature of
the speech event (`s-event') and the other to its interpretation
(syntax and semantics, `synsem', using the label which is
well-established in HPSG).   In the s-event component the
phon-field represents the phonology of an expression, here represented
as a string of words although in a complete treatment of spoken
language we would need phonological and phonetic attributes.  The
s-time (``speech time'') field represents the starting and ending time
for the utterance.  We assume the existence of a predicate
`uttered\_at' with arity
$\langle\mathit{Str},\mathit{Time},\mathit{Time}\rangle$.  An object of type
`uttered\_at($a$,$t_1$,$t_2$)' could be an event where $a$ is uttered
beginning at $t_1$ and ending at $t_2$ or a corresponding hypothesis
produced by a speech recognizer with time-stamps, depending on the application of the
theory.  In a more complete treatment we would need additional
information about the physical nature of the speech event, such as the
identity of the speaker and where it took place.  

In the synsem component
the cat-field introduces Montague's category for the phrase.  This
could, and probably should, be replaced by categories more familiar to
linguists as has been done in many applications of Montague's
semantics.  But nothing is at stake in this matter for the purposes of
this paper and there are a number of alternatives in the linguistic
literature to choose from.
The cnt-field represents the content or interpretation of the
utterance.  It could be represented in terms of intensional logic with a
similar interpretation to that we presented above.  However, it is
more perspicuous to represent the content directly in terms of TTR.
Since the content types become rather long we will introduce
abbreviations to make them readable:
\begin{quote}
\textit{Prop}, ``property'' is to be
\smallrecord{\smalltfield{x}{\textit{Ind}}}$\rightarrow$\textit{RecType}
  \\
\textit{Quant}, ``quantifier'' is to be
\textit{Prop}$\rightarrow$\textit{RecType} \\
if $T$ is a type then $^\sigma T$ is to be
\textit{Seq}$_{\mathit{Ind}}\rightarrow T$ 
\end{quote}
We only use a small finite number of function types for content
types and thus we are able to define a type \textit{CntType} which is
\begin{quote}
\ignore{\smallrecord{\smalltfield{x}{\textit{Ind}}} $\vee$\\(}$^\sigma$\textit{RecType}
$\vee$\\($^\sigma$\textit{Prop} $\vee$\\($^\sigma$\textit{Quant}
%$\vee$\\($^\sigma$(\textit{Prop}$\rightarrow$\textit{Prop})
$\vee$\\($^\sigma$(\textit{Quant}$\rightarrow$\textit{Prop})
$\vee$\\($^\sigma$(\textit{RecType}$\rightarrow$\textit{Prop})
$\vee$\\($^\sigma$(\textit{Prop}$\rightarrow$\textit{Quant}) 
%$\vee$\\$^\sigma$(\textit{Quant}$\rightarrow$(\textit{Prop}$\rightarrow$\textit{Prop}))
)))))\ignore{))}
\end{quote}
This means that we can define \textit{Sign}, the type of signs, to be
\begin{display}
\record{\tfield{s-event}{\record{\tfield{phon}{\textit{Str}}\\
                                 \tfield{s-time}{\record{\tfield{start}{\textit{Time}}
                                     \\
                                                         \tfield{end}{\textit{Time}}}}\\
                                 \tfield{utt$_\mathrm{at}$}{$\langle
                                   \lambda v_1:$\textit{Str}$(\lambda
                                   v_2:$\textit{Time}($\lambda
                                   v_3:$\textit{Time}(uttered\_at($v_1$,$v_2$,$v_3$)))), \\
                                   & &
                                   \hspace*{2em}$\langle$s-event.phon,
                                   s-event.s-time.start, s-event.s-time.end$\rangle\rangle$}}}
                               \\
        \tfield{synsem}{\record{\tfield{cat}{\textit{Cat}}\\
                                \tfield{cnt}{\textit{CntType}}}}}

\end{display}

We will present first the lexicon and then rules for
combining phrases.  We shall use the following notations:
\begin{quote}
If $s$ is a string, then c$_s$ is a distinguished label associated
with $s$, such that if $s\not=s'$ then c$_s$$\not=$c$_s'$.

If $s$ is a string then $s'_{\mathit{Arity}}$ is a predicate of arity
$\mathit{Arity}$.  When $\mathit{Arity}$ is clear from the context we
suppress it.

If $s$ is a string then $s'_T$ is an object of type $T$.  When it is
clear from the context we often suppress $T$.

% necessarily$'_{\langle\mathit{RecType}\rangle}$ is to be the predicate
% `nec'.

\end{quote}


\subsection{Lexicon}
\ignore{Let $\sigma$ be a function from TTR-types, $T$, such that $\sigma_T$
(the result of applying $\sigma$ to $T$) is a countably infinite
sequence of objects of type $T$.  We will call $\sigma$ a
\textit{family of typed sequences}.}  We will use sequences of
individuals as an
apparatus to treat pronouns and binding in the interpretations given
in content fields.  In the lexicon all contents except for pronoun
contents are constant functions defined on these sequences.  We will present a series of lexical functions which map strings and
occasionally additional arguments record types for signs.  

We will use a number of auxiliary types in the definition of our
lexical classes:
\begin{enumerate}

\renewcommand{\labelenumi}{LTy\arabic{enumi}.} 
 
\item the gender type, \textit{Gen}, such that $a$:\textit{Gen}
  iff $a$ is m, f or n (masculine, feminine or neuter). 
 
\item the number type, \textit{Num}, such that $a$:\textit{Num} iff
  $a$ is sg or pl (singular or plural).

\item the person type, \textit{Pers}, such that $a$:\textit{Pers} iff
  $a$ is 1st, 2nd or 3rd (first, second or third person)

\item we use \textit{Agr} to represent the type
\begin{display}
\record{\tfield{gen}{\textit{Gen}} \\
        \tfield{num}{\textit{Num}} \\
        \tfield{pers}{\textit{Pers}}}
\end{display}

\item we use \textit{MascSg3rd} to represent the type
\begin{display}
\record{\tfield{synsem}{\record{\tfield{agr}{\record{\mfield{gen}{m}{\textit{Gen}} \\
                                             \mfield{num}{sg}{\textit{Num}} \\
                                             \mfield{pers}{3rd}{\textit{Pers}}}}}}}
\end{display}

\item we use \textit{FemSg3rd} to represent the type
\begin{display}
\record{\tfield{synsem}{\record{\tfield{agr}{\record{\mfield{gen}{f}{\textit{Gen}} \\
                                             \mfield{num}{sg}{\textit{Num}} \\
                                             \mfield{pers}{3rd}{\textit{Pers}}}}}}}
\end{display}

\item we use \textit{NeutSg3rd} to represent the type
\begin{display}
\record{\tfield{synsem}{\record{\tfield{agr}{\record{\mfield{gen}{n}{\textit{Gen}} \\
                                             \mfield{num}{sg}{\textit{Num}} \\
                                             \mfield{pers}{3rd}{\textit{Pers}}}}}}}
\end{display}

\item we use \textit{Sg3rd} to represent the type
\begin{display}
\record{\tfield{synsem}{\record{\tfield{agr}{\record{\tfield{gen}{\textit{Gen}} \\
                                             \mfield{num}{sg}{\textit{Num}} \\
                                             \mfield{pers}{3rd}{\textit{Pers}}}}}}}
\end{display}

 
\end{enumerate} 
  

\subsubsection{Intransitive verbs}\label{pg:lexicon-iv}
We define a function lex$_\mathrm{IV}$ for intransitive verbs like
\textit{run} which predicate directly of individuals such that
lex$_\mathrm{IV}$($s$) is
\begin{quote}
\hspace*{-2em}\textit{Sign} \d{$\wedge$} \\
\hspace*{-2em}\record{\tfield{s-event}{\record{\mfield{phon}{$s$}{\textit{Str}}}} \\
        \tfield{synsem}{\record{\mfield{cat}{IV}{\textit{Cat}} \\
                                \mfield{cnt}{$\lambda
                                  \sigma$:\textit{Seq}$_{\mathit{Ind}}$
                                  \\ \hspace*{3em}($\lambda
                                  r$:\smallrecord{\smalltfield{x}{\textit{Ind}}}
                                  (\smallrecord{\smalltfield{e-time}{\textit{Time}}
                                    \\
                                           \smalltfield{c$_s$}{$\langle\lambda
                                             v$:\textit{Time}($s'$($r$.x,$v$)), $\langle$e-time$\rangle\rangle$}}))}{$^\sigma$\textit{Prop}}}}}

\end{quote}
We define a function lex$_{\mathrm{IV}\textrm{-}\mathrm{fr}}$ for intransitive verbs like
\textit{rise} which correspond to predicates of frames such that
lex$_{\mathrm{IV}\textrm{-}\mathrm{fr}}$($s$) is
\begin{quote}
\hspace*{-2em}\textit{Sign} \d{$\wedge$} \\
\hspace*{-2em}\record{\tfield{s-event}{\record{\mfield{phon}{$s$}{\textit{Str}}}} \\
        \tfield{synsem}{\record{\mfield{cat}{IV}{\textit{Cat}} \\
                                \mfield{cnt}{$\lambda
                                  \sigma$:\textit{Seq}$_{\mathit{Ind}}$
                                  \\ \hspace*{3em}($\lambda
                                  r$:\smallrecord{\smalltfield{x}{\textit{Ind}}}
                                  (\smallrecord{\smalltfield{e-time}{\textit{Time}}
                                    \\
                                           \smalltfield{c$_s$}{$\langle\lambda
                                             v$:\textit{Time}($s'$($r$,$v$)), $\langle$e-time$\rangle\rangle$}}))}{$^\sigma$\textit{Prop}}}}}

\end{quote}

We define the set of lexical types of category IV,
B$_{\mathrm{IV}}$, to be
\begin{quote}
\begin{tabbing}
\{\=lex$_{\mathrm{IV}}$(``run''), \\
  \>lex$_{\mathrm{IV}}$(``walk''), \\
  \>lex$_{\mathrm{IV}}$(``talk''), \\
  \>lex$_{\mathrm{IV}\textrm{-}\mathrm{fr}}$(``rise''), \\
  \>lex$_{\mathrm{IV}\textrm{-}\mathrm{fr}}$(``change'')\}
\end{tabbing}
\end{quote}

\subsubsection{Basic terms (proper names and pronouns)}
We define a function lex$_{\mathrm{T}\textrm{-}\mathrm{PN}}$ for
proper names like
\textit{John} such that
lex$_{\mathrm{T}\textrm{-}\mathrm{PN}}$($s$) is
\begin{quote}
\textit{Sign} \d{$\wedge$} \\
\record{\tfield{s-event}{\record{\mfield{phon}{$s$}{\textit{Str}}}} \\
        \tfield{synsem}{\record{\mfield{cat}{T}{\textit{Cat}} \\
                                \mfield{cnt}{$\lambda
                                  \sigma$:\textit{Seq}$_{\mathit{Ind}}$
                                  \\ \hspace*{3em}($\lambda
                                  v$:\textit{Prop}($v$(\smallrecord{\field{x}{$s'_{\mathit{Ind}}$}})))}{$^\sigma$\textit{Quant}}}}}
\end{quote}
We define a function lex$_{\mathrm{T}\textrm{-}\mathrm{Pron}}$ for
pronouns like
\textit{he} such that
lex$_{\mathrm{T}\textrm{-}\mathrm{Pron}}$($s,n$) is
\begin{quote}
\textit{Sign} \d{$\wedge$} \\
\record{\tfield{s-event}{\record{\mfield{phon}{$s$}{\textit{Str}}}} \\
        \tfield{synsem}{\record{\mfield{cat}{T}{\textit{Cat}} \\
                                \mfield{cnt}{$\lambda
                                  \sigma$:\textit{Seq}$_{\mathit{Ind}}$
                                  \\ \hspace*{3em}($\lambda
                                  v$:\textit{Prop}($v$(\smallrecord{\field{x}{$\sigma_n$}})))}{\textit{$^\sigma$Quant}}}}}
\end{quote}

We define the set of lexical types of category T,
B$_{\mathrm{T}}$, to be
\begin{quote}
\begin{tabbing}
\{\=lex$_{\mathrm{T}\textrm{-}\mathrm{PN}}$(``John'')\d{$\wedge$}MascSg3rd, \\
  \>lex$_{\mathrm{T}\textrm{-}\mathrm{PN}}$(``Mary'')\d{$\wedge$}FemSg3rd, \\
  \>lex$_{\mathrm{T}\textrm{-}\mathrm{PN}}$(``Bill'')\d{$\wedge$}MascSg3rd, \\
  \>lex$_{\mathrm{T}\textrm{-}\mathrm{PN}}$(``ninety'')\d{$\wedge$}NeutSg3rd\} \\
$\cup$
\{lex$_{\mathrm{T}\textrm{-}\mathrm{Pron}}$(``he'', $3n+2$)\d{$\wedge$}MascSg3rd $\mid$ $n \in
\mathit{Nat}$\} \\
$\cup$
\{lex$_{\mathrm{T}\textrm{-}\mathrm{Pron}}$(``she'', $3n+3$)\d{$\wedge$}FemSg3rd $\mid$ $n \in
\mathit{Nat}$\} \\
$\cup$
\{lex$_{\mathrm{T}\textrm{-}\mathrm{Pron}}$(``it'', $3n+4$)\d{$\wedge$}NeutSg3rd\} $\mid$ $n \in
\mathit{Nat}$\}
\end{tabbing}
\end{quote}

\subsubsection{Transitive verbs}
We define a function lex$_\mathrm{TV}$ for extensional transitive verbs like
\textit{find} and \textit{be} such that
lex$_\mathrm{TV}$($s$) is
\begin{quote}
\hspace*{-3em}\textit{Sign} \d{$\wedge$} \\
\hspace*{-3em}\record{\tfield{s-event}{\record{\mfield{phon}{$s$}{\textit{Str}}}} \\
        \tfield{synsem}{\smallrecord{\smallmfield{cat}{TV}{\textit{Cat}} \\
                                \smallmfield{cnt}{$\lambda
                                  \sigma$:\textit{Seq}$_{\mathit{Ind}}$
                                  \\ \hspace*{3em}(
$\lambda v_1$:\textit{Quant} \\
\hspace*{4em}($\lambda v_2$:\smallrecord{\smalltfield{x}{\textit{Ind}}} \\
\hspace*{5em}($v_1$($\lambda v_3$:\smallrecord{\smalltfield{x}{\textit{Ind}}} \\
 \hspace*{7em}                                 (\smallrecord{\smalltfield{e-time}{\textit{Time}}
                                    \\
                                           \smalltfield{c$_s$}{$\langle\lambda
                                             v$:\textit{Time}($s'$($v_2$.x,$v_3$.x,$v$)),
                                             $\langle$e-time$\rangle\rangle$}})))))
}{$^\sigma$(\textit{Quant}$\rightarrow$\textit{Prop})}}}}

\end{quote}

We define a function lex$_{\mathrm{TV}\textrm{-}\mathrm{int}}$ for intensional transitive verbs like
\textit{conceive}\footnote{which at least on Montague's analysis is
  intensional and does not have a decompositional interpretation -- we
  do not take up the issue here of whether Montague's analysis was correct} such that
lex$_{\mathrm{TV}\textrm{-}\mathrm{int}}$($s$) is
\begin{quote}
\hspace*{-4em}\textit{Sign} \d{$\wedge$} \\
\hspace*{-4em}\record{\tfield{s-event}{\record{\mfield{phon}{$s$}{\textit{Str}}}} \\
        \tfield{synsem}{\smallrecord{\smallmfield{cat}{TV}{\textit{Cat}} \\
                                \smallmfield{cnt}{$\lambda
                                  \sigma$:\textit{Seq}$_{\mathit{Ind}}$
                                  \\ \hspace*{3em}(
$\lambda v_1$:\textit{Quant} \\
\hspace*{4em}($\lambda v_2$:\smallrecord{\smalltfield{x}{\textit{Ind}}}
\\
\hspace*{5em}(\smallrecord{\smalltfield{e-time}{\textit{Time}} \\
\smalltfield{c$_s$}{$\langle\lambda v$:\textit{Time}\\
\hspace*{3em}($s'$($v_2$.x,$v_1$($\lambda
  v_3$:\smallrecord{\smalltfield{x}{\textit{Ind}}}(\record{})), $v$)),
  $\langle$e-time$\rangle\rangle$}})))}
{$^\sigma$(\textit{Quant}$\rightarrow$\textit{Prop})}}}}

\end{quote} 

We define a function lex$_{\mathrm{TV}\textrm{-}\mathrm{int}\textrm{-}\mathrm{decomp}}$ for intensional transitive verbs like
\textit{seek} such that
lex$_{\mathrm{TV}\textrm{-}\mathrm{int}\textrm{-}\mathrm{decomp}}$($s_1,s_2$) is
\begin{quote}
\hspace*{-12em}\textit{Sign} \d{$\wedge$} \\
\hspace*{-12em}\smallrecord{\smalltfield{s-event}{\record{\mfield{phon}{$s_1$}{\textit{Str}}}} \\
        \smalltfield{synsem}{\smallrecord{\smallmfield{cat}{TV}{\textit{Cat}} \\
                                \smallmfield{cnt}{$\lambda
                                  \sigma$:\textit{Seq}$_{\mathit{Ind}}$
                                  \\ \hspace*{3em}(
$\lambda v_1$:\textit{Quant} \\
\hspace*{4em}($\lambda v_2$:\smallrecord{\smalltfield{x}{\textit{Ind}}}
\\
\hspace*{5em}(\smallrecord{\smalltfield{e-time}{\textit{Time}} \\
\smalltfield{c$_{s_1}$}{$\langle\lambda v$:\textit{Time}\\
\hspace*{4em}(${s_1}'$($v_2$.x,$v_1$($\lambda
  v_3$:\smallrecord{\smalltfield{x}{\textit{Ind}}} \\
\hspace*{10em}(\smallrecord{
\smalltfield{e-time}{\textit{Time}} \\
\smalltfield{c$_{s_2}$}{$\langle\lambda
  v:$\textit{Time}(${s_2}'$($v_2$.x,$v_3$.x,$v$)),
  $\langle$e-time$\rangle\rangle$}}), \\ 
\hspace*{6em}$v$)),
  $\langle$e-time$\rangle\rangle$}}))))}
{$^\sigma$(\textit{Quant}$\rightarrow$\textit{Prop})}}}}

\end{quote} 

We define the set of lexical types of category TV,
B$_{\mathrm{TV}}$, to be
\begin{quote}
\begin{tabbing}
\{\=lex$_{\mathrm{TV}}$(``find''), \\
  \>lex$_{\mathrm{TV}}$(``lose''), \\
  \>lex$_{\mathrm{TV}}$(``eat''), \\
  \>lex$_{\mathrm{TV}}$(``love''), \\
  \>lex$_{\mathrm{TV}}$(``date''), \\
  \>lex$_{\mathrm{TV}}$(``be''), \\
  \>lex$_{\mathrm{TV}\textrm{-}\mathrm{int}\textrm{-}\mathrm{decomp}}$(``seek'',
  ``find''), \\
  \>lex$_{\mathrm{TV}\textrm{-}\mathrm{int}}$(``conceive'')\}
\end{tabbing}
\end{quote}

\ignore{
\subsubsection{IV adverbs}
We define a function lex$_{\mathrm{IAV}}$ for IV (verb phrase) adverbs like
\textit{slowly} such that
lex$_{\mathrm{IAV}}$($s$) is
\begin{quote}
\textit{Sign} \d{$\wedge$} \\
\record{\tfield{s-event}{\record{\mfield{phon}{$s$}{\textit{Str}}}} \\
        \tfield{synsem}{\smallrecord{\smallmfield{cat}{IAV}{\textit{Cat}} \\
                                \smallmfield{cnt}{$\lambda
                                  \sigma$:\textit{Seq}$_{\mathit{Ind}}$
                                  \\ \hspace*{3em}(
$\lambda v_1$:\textit{Prop} \\
\hspace*{4em}($\lambda v_2$:\smallrecord{\smalltfield{x}{\textit{Ind}}} \\
\hspace*{5em}(\smallrecord{\smalltfield{c$_s$}{$s'$($v_2$.x,$v_1$)}})))}{$^\sigma$(\textit{Prop}$\rightarrow$\textit{Prop})}}}}

\end{quote}

We define the set of lexical types of category IAV,
B$_{\mathrm{IAV}}$, to be\footnote{Montague incorrectly
    classified \textit{allegedly} as a verb-phrase adverb rather than
    a sentence adverb and we do not correct this here.}
\begin{quote}
\begin{tabbing}
\{\=lex$_{\mathrm{IAV}}$(``rapidly''), \\
  \>lex$_{\mathrm{IAV}}$(``slowly''), \\
  \>lex$_{\mathrm{IAV}}$(``voluntarily''), \\
  \>lex$_{\mathrm{IAV}}$(``allegedly'')\}
\end{tabbing}
\end{quote}
}
\subsubsection{Common nouns}
We define a function lex$_{\mathrm{CN}}$ for common nouns like
\textit{fish} \label{pg:lexicon-cn} such that
lex$_{\mathrm{CN}}$($s$) is
\begin{quote}
\textit{Sign} \d{$\wedge$} \\
\record{\tfield{s-event}{\record{\mfield{phon}{$s$}{\textit{Str}}}} \\
        \tfield{synsem}{\smallrecord{\smallmfield{cat}{CN}{\textit{Cat}} \\
                                \smallmfield{cnt}{$\lambda
                                  \sigma$:\textit{Seq}$_{\mathit{Ind}}$
                                  \\ \hspace*{3em}($\lambda v$:\smallrecord{\smalltfield{x}{\textit{Ind}}} \\
\hspace*{4em}(\smallrecord{\smalltfield{c$_s$}{$s'$($v$.x)}}))}{$^\sigma$\textit{Prop}}}}}

\end{quote}

We define a function lex$_{\mathrm{CN}\textrm{-}\mathrm{fr}}$ for common nouns like
\textit{temperature} such that
lex$_{\mathrm{CN}\textrm{-}\mathrm{fr}}$($s$) is
\begin{quote}
\textit{Sign} \d{$\wedge$} \\
\record{\tfield{s-event}{\record{\mfield{phon}{$s$}{\textit{Str}}}} \\
        \tfield{synsem}{\smallrecord{\smallmfield{cat}{CN}{\textit{Cat}} \\
                                \smallmfield{cnt}{$\lambda
                                  \sigma$:\textit{Seq}$_{\mathit{Ind}}$
                                  \\ \hspace*{3em}($\lambda v$:\smallrecord{\smalltfield{x}{\textit{Ind}}} \\
\hspace*{4em}(\smallrecord{\smalltfield{c$_s$}{$s'$($v$)}}))}{$^\sigma$\textit{Prop}}}}}

\end{quote}

We define the set of lexical types of category CN,
B$_{\mathrm{CN}}$, to be
\begin{quote}
\begin{tabbing}
\{\=lex$_{\mathrm{CN}}$(``man'')\d{$\wedge$}\textit{MascSg3rd}, \\
  \>lex$_{\mathrm{CN}}$(``woman'')\d{$\wedge$}\textit{FemSg3rd}, \\
  \>lex$_{\mathrm{CN}}$(``park'')\d{$\wedge$}\textit{NeutSg3rd}, \\
  \>lex$_{\mathrm{CN}}$(``fish'')\d{$\wedge$}\textit{Sg3rd}, \\
  \>lex$_{\mathrm{CN}}$(``pen'')\d{$\wedge$}\textit{NeutSg3rd}, \\
  \>lex$_{\mathrm{CN}}$(``unicorn'')\d{$\wedge$}\textit{Sg3rd}, \\
  \>lex$_{\mathrm{CN}\textrm{-}\mathrm{fr}}$(``price'')\d{$\wedge$}\textit{NeutSg3rd}, \\
  \>lex$_{\mathrm{CN}\textrm{-}\mathrm{fr}}$(``temperature'')\d{$\wedge$}\textit{NeutSg3rd}\}
\end{tabbing}
\end{quote}

\ignore{
\subsubsection{Sentence adverbs}
We define a function lex$_{\mathrm{t/t}}$ for sentence adverbs like
\textit{necessarily} such that
lex$_{\mathrm{t/t}}$($s$) is
\begin{quote}
\textit{Sign} \d{$\wedge$} \\
\record{\tfield{s-event}{\record{\mfield{phon}{$s$}{\textit{Str}}}} \\
        \tfield{synsem}{\smallrecord{\smallmfield{cat}{t/t}{\textit{Cat}} \\
                                \smallmfield{cnt}{$\lambda
                                  \sigma$:\textit{Seq}$_{\mathit{Ind}}$
                                  \\ \hspace*{3em}($\lambda v$:\textit{RecType} \\
\hspace*{4em}(\smallrecord{\smalltfield{c$_s$}{$s'$($v$)}}))}{$^\sigma$(\textit{RecType}$\rightarrow$\textit{RecType})}}}}

\end{quote}


We define the set of lexical types of category t/t,
B$_{\mathrm{t/t}}$, to be
\begin{quote}
\begin{tabbing}
\{\=lex$_{\mathrm{t/t}}$(``necessarily'')\}
\end{tabbing}
\end{quote}

\subsubsection{Prepositions}
We define a function lex$_{\mathrm{IAV/T}}$ for prepositions like
\textit{in} such that
lex$_{\mathrm{IAV/T}}$($s$) is
\begin{quote}
\textit{Sign} \d{$\wedge$} \\
\record{\tfield{s-event}{\record{\mfield{phon}{$s$}{\textit{Str}}}} \\
        \tfield{synsem}{\smallrecord{\smallmfield{cat}{IAV/T}{\textit{Cat}} \\
                                \smallmfield{cnt}{$\lambda
                                  \sigma$:\textit{Seq}$_{\mathit{Ind}}$
                                  \\ \hspace*{3em}($\lambda v_1$:\textit{Quant} \\
\hspace*{4em}($\lambda v_2$:\textit{Prop} \\
\hspace*{5em}($\lambda v_3$:\smallrecord{\smalltfield{x}{\textit{Ind}}} \\
\hspace*{6em}($v_1$($\lambda v_4$:\smallrecord{\smalltfield{x}{\textit{Ind}}}\\
\hspace*{7em}(\smallrecord{\smalltfield{c$_s$}{$s'$($v_2(v_3),v_4.\mathrm{x}$)}}))))))}{$^\sigma$(\textit{Quant}$\rightarrow$(\textit{Prop}$\rightarrow$\textit{Prop}))}}}}

\end{quote}


We define the set of lexical types of category IAV/T,
B$_{\mathrm{IAV/T}}$, to be
\begin{quote}
\begin{tabbing}
\{\=lex$_{\mathrm{IAV/T}}$(``in''), \\
  \>lex$_{\mathrm{IAV/T}\textrm{-}\mathrm{Int}}$(``about'')\}
\end{tabbing}
\end{quote}
}

\ignore{
\subsubsection{Verbs with \emph{that}-complements}

We define a function lex$_{\mathrm{IV/t}}$ for verbs which take
\textit{that}-complements like
\textit{believe}  such that
lex$_{\mathrm{IV/t}}$($s$) is
\begin{quote}
\textit{Sign} \d{$\wedge$} \\
\record{\tfield{s-event}{\record{\mfield{phon}{$s$}{\textit{Str}}}} \\
        \tfield{synsem}{\smallrecord{\smallmfield{cat}{IV/t}{\textit{Cat}} \\
                                \smallmfield{cnt}{$\lambda
                                  \sigma$:\textit{Seq}$_{\mathit{Ind}}$
                                  \\ \hspace*{3em}($\lambda v_1$:\textit{RecType} \\
\hspace*{4em}($\lambda v_2$:\smallrecord{\smalltfield{x}{\textit{Ind}}} \\
\hspace*{5em}(\smallrecord{\smalltfield{e-time}{\textit{Time}} \\
                           \smalltfield{c$_s$}{$\langle\lambda v$:\textit{Time}($s'$($v_2$.x,$v_1$,$v$)),
                                             $\langle$e-time$\rangle\rangle$}})))
}{$^\sigma$(\textit{RecType}$\rightarrow$\textit{Prop})}}}}

\end{quote}

We define the set of lexical types of category IV/t,
B$_{\mathrm{IV/t}}$, to be
\begin{quote}
\begin{tabbing}
\{\=lex$_{\mathrm{IV/t}}$(``believe''), \\
  \>lex$_{\mathrm{IAV/T}\textrm{-}\mathrm{Int}}$(``assert'')\}
\end{tabbing}
\end{quote}

\subsubsection{Verbs with infinitival complements}

We define a function lex$_{\mathrm{IV//IV}}$ for verbs which take
infinitival complements like
\textit{try}  such that
lex$_{\mathrm{IV//IV}}$($s$) is
\begin{quote}
\textit{Sign} \d{$\wedge$} \\
\record{\tfield{s-event}{\record{\mfield{phon}{$s$}{\textit{Str}}}} \\
        \tfield{synsem}{\smallrecord{\smallmfield{cat}{IV//IV}{\textit{Cat}} \\
                                \smallmfield{cnt}{$\lambda
                                  \sigma$:\textit{Seq}$_{\mathit{Ind}}$
                                  \\ \hspace*{3em}($\lambda v_1$:\textit{Prop} \\
\hspace*{4em}($\lambda v_2$:\smallrecord{\smalltfield{x}{\textit{Ind}}} \\
\hspace*{5em}(\smallrecord{\smalltfield{e-time}{\textit{Time}} \\
                           \smalltfield{c$_s$}{$\langle\lambda v$:\textit{Time}($s'$($v_2$.x,$v_1$,$v$)),
                                             $\langle$e-time$\rangle\rangle$}})))
}{$^\sigma$(\textit{Prop}$\rightarrow$\textit{Prop})}}}}

\end{quote}

We define the set of lexical types of category IV//IV,
B$_{\mathrm{IV//IV}}$, to be
\begin{quote}
\begin{tabbing}
\{\=lex$_{\mathrm{IV//IV}}$(``try''), \\
  \>lex$_{\mathrm{IV//IV}}$(``wish'')\}
\end{tabbing}
\end{quote}
}

\subsubsection{Determiners}

We define a function lex$_{\mathrm{T/CN}\textrm{-}\mathrm{ex}}$ for
the indefinite article
\label{pg:lexicon-indefart}
\textit{a}  such that
lex$_{\mathrm{T/CN}\textrm{-}\mathrm{ex}}$($s$) is
\begin{quote}
\textit{Sign} \d{$\wedge$} \\
\record{\tfield{s-event}{\record{\mfield{phon}{$s$}{\textit{Str}}}} \\
        \tfield{synsem}{\smallrecord{\smallmfield{cat}{T/CN}{\textit{Cat}} \\
                                \smallmfield{cnt}{$\lambda
                                  \sigma$:\textit{Seq}$_{\mathit{Ind}}$
                                  \\ \hspace*{3em}($\lambda v_1$:\textit{Prop} \\
\hspace*{4em}($\lambda v_2$:\textit{Prop} \\
\hspace*{5em}(\record{\tfield{par}{\smallrecord{\smalltfield{x}{\textit{Ind}}}}
  \\
                      \tfield{restr}{$\langle\lambda v$:\smallrecord{\smalltfield{x}{\textit{Ind}}}($v_1(v)$),
                        $\langle\mathrm{par}\rangle\rangle$} \\
                      \tfield{scope}{$\langle\lambda v$:\smallrecord{\smalltfield{x}{\textit{Ind}}}($v_2(v)$), $\langle\mathrm{par}\rangle\rangle$}})))
}{$^\sigma$(\textit{Prop}$\rightarrow$\textit{Quant})}}}}

\end{quote}


We define a function lex$_{\mathrm{T/CN}\textrm{-}\mathrm{uni}}$ for
the universal determiner
\textit{every}  such that
lex$_{\mathrm{T/CN}\textrm{-}\mathrm{uni}}$($s$) is
\begin{quote}
\hspace*{-4em}\textit{Sign} \d{$\wedge$} \\
\hspace*{-4em}\record{\tfield{s-event}{\record{\mfield{phon}{$s$}{\textit{Str}}}} \\
        \tfield{synsem}{\smallrecord{\smallmfield{cat}{T/CN}{\textit{Cat}} \\
                                \smallmfield{cnt}{$\lambda
                                  \sigma$:\textit{Seq}$_{\mathit{Ind}}$
                                  \\ \hspace*{3em}($\lambda v_1$:\textit{Prop} \\
\hspace*{4em}($\lambda v_2$:\textit{Prop} \\
\hspace*{5em}(\smallrecord{\smalltfield{f}{($r$:\smallrecord{\smalltfield{par}{\smallrecord{\smalltfield{x}{\textit{Ind}}}}
  \\
                                              \smalltfield{restr}{$\langle\lambda v$:\smallrecord{\smalltfield{x}{\textit{Ind}}}($v_1(v)$),
                                              $\langle\mathrm{par}\rangle\rangle$}})$\rightarrow
                                          v_2(r.\mathrm{par})$}})))}{$^\sigma$(\textit{Prop}$\rightarrow$\textit{Quant})}}}}

\end{quote}

We define a function lex$_{\mathrm{T/CN}\textrm{-}\mathrm{def}}$ for
the definite article
\textit{the}  such that
lex$_{\mathrm{T/CN}\textrm{-}\mathrm{def}}$($s$) is
\begin{quote}
\hspace*{-12em}\textit{Sign} \d{$\wedge$} \\
\hspace*{-12em}\record{\tfield{s-event}{\record{\mfield{phon}{$s$}{\textit{Str}}}} \\
        \tfield{synsem}{\smallrecord{\smallmfield{cat}{T/CN}{\textit{Cat}} \\
                                \smallmfield{cnt}{$\lambda
                                  \sigma$:\textit{Seq}$_{\mathit{Ind}}$
                                  \\ \hspace*{3em}($\lambda v_1$:\textit{Prop} \\
\hspace*{4em}($\lambda v_2$:\textit{Prop} \\
\hspace*{5em}(\smallrecord{\smalltfield{par}{\smallrecord{\smalltfield{x}{\textit{Ind}}}}
  \\
                      \smalltfield{restr}{$\langle\lambda v$:\smallrecord{\smalltfield{x}{\textit{Ind}}}($v_1(v)$\d{$\wedge$}\smallrecord{\smalltfield{f}{($r$:\smallrecord{\smalltfield{par}{\smallrecord{\smalltfield{x}{\textit{Ind}}}}
  \\             \smalltfield{restr}{$\langle\lambda v$:\smallrecord{\smalltfield{x}{\textit{Ind}}}($v_1(v)$),
                                              $\langle\mathrm{par}\rangle\rangle$}})
                                           \\ \hspace*{2em}$\rightarrow$
                                          \smallrecord{\tfield{scope}{eq($v$,$r.\mathrm{par}$)}}}},
                        \\ \hspace*{3em}$\langle\mathrm{par}\rangle\rangle$)} \\
                      \smalltfield{scope}{$\langle\lambda v$:\smallrecord{\smalltfield{x}{\textit{Ind}}}($v_2(v)$), $\langle\mathrm{par}\rangle\rangle$}})))}{$^\sigma$(\textit{Prop}$\rightarrow$\textit{Quant})}}}}

\end{quote}

We define the set of lexical types of category T/CN,
B$_{\mathrm{T/CN}}$, to be
\begin{quote}
\begin{tabbing}
\{\=lex$_{\mathrm{T/CN}\textrm{-}\mathrm{ex}}$(``a''), \\
  \>lex$_{\mathrm{T/CN}\textrm{-}\mathrm{uni}}$(``every''), \\
  \>lex$_{\mathrm{T/CN}\textrm{-}\mathrm{def}}$(``the'')\}
\end{tabbing}
\end{quote}



\subsection{Constraints on lexical predicates}
We restrict our attention to a collection of stratified models,
$\mathfrak{M}$, in which for any $\mathscr{M}\in\mathfrak{M}$ and
order $n$, the following
hold true.

\subsubsection{Frame constraints}
\label{pg:frameconstraints}
\begin{enumerate} 
 
\item If ``price''$'$($a$) is a non-empty type according to
  $\mathbf{TYPE}_{\mathit{IMC}_{\mathscr{M}_n}}$ then
\begin{quote}
$a:_{\mathbf{TYPE}_{\mathit{IMC}_{\mathscr{M}_n}}}$\record{\tfield{x}{\textit{Ind}} \\
            \tfield{e-time}{\textit{Time}} \\
            \tfield{commodity}{\textit{Ind}} \\
            \tfield{c$_{\mathrm{money}}$}{$\langle\lambda
              v$:\textit{Ind}(money($v$)), $\langle$x$\rangle\rangle$}
            \\
            \tfield{c$_{\mathrm{price\_of\_at}}$}{$\langle\lambda
              v_1$:\textit{Ind} \\
& & \hspace*{1em}($\lambda v_2$:\textit{Ind} \\
& & \hspace*{2em}($\lambda v_3$:\textit{Time} \\
& & \hspace*{3em}(price\_of\_at($v_1,v_2,v_3$)))), \\
& & \hspace*{1ex}$\langle$x,
commodity, e-time$\rangle\rangle$}}

\end{quote}

We call this type \textit{PriceFrame}
 
\item If ``temperature''$'$($a$) is a non-empty type according to
  $\mathbf{TYPE}_{\mathit{IMC}_{\mathscr{M}_n}}$ then
\begin{quote}
$a:_{\mathbf{TYPE}_{\mathit{IMC}_{\mathscr{M}_n}}}$\record{\tfield{x}{\textit{Ind}} \\
        \tfield{e-time}{\textit{Time}} \\
        \tfield{e-location}{\textit{Loc}} \\
        \tfield{c$_{\mathrm{temp\_at\_in}}$}{$\langle\lambda
              v_1$:\textit{Ind} \\
& & \hspace*{1em}($\lambda v_2$:\textit{Ind} \\
& & \hspace*{2em}($\lambda v_3$:\textit{Time} \\
& & \hspace*{3em}(temp\_at\_in($v_1,v_2,v_3$)))), \\
& & \hspace*{1ex}$\langle$x, e-location, e-time$\rangle\rangle$}} 
\end{quote}

We call this type \textit{AmbTempFrame}, ``ambient temperature frame''. 
\end{enumerate} 

\subsubsection{General predicate constraints}
\label{pg:predconstraints}
\begin{enumerate} 
 
\item $a:_{\mathbf{TYPE}_{\mathit{IMC}_{\mathscr{M}_n}}}\textrm{``}\mathrm{be}\textrm{"}'(b,c,t)$ iff $a:_{\mathbf{TYPE}_{\mathit{IMC}_{\mathscr{M}_n}}}\mathrm{eq}(b,c)$ 
 
\item if $r:_{\mathbf{TYPE}_{\mathit{IMC}_{\mathscr{M}_n}}}$\textit{AmbTempFrame}, then
  $a:_{\mathbf{TYPE}_{\mathit{IMC}_{\mathscr{M}_n}}}\textrm{``}\mathrm{rise}\textrm{"}'(r)$
  iff $a=\langle r,r'\rangle$ where
\begin{enumerate} 
 
\item $r':_{\mathbf{TYPE}_{\mathit{IMC}_{\mathscr{M}_n}}}$\textit{AmbTempFrame}
 
\item $r$.x$<$$r'$.x

\item $r$.e-time$<$$r'$.e-time

\item $r$.e-location=$r'$.e-location
 
\end{enumerate} 


\item if $r:_{\mathbf{TYPE}_{\mathit{IMC}_{\mathscr{M}_n}}}$\textit{PriceFrame}, then
  $a:_{\mathbf{TYPE}_{\mathit{IMC}_{\mathscr{M}_n}}}\textrm{``}\mathrm{rise}\textrm{"}'(r)$
  iff $a=\langle r,r'\rangle$ where
\begin{enumerate} 
 
\item $r':_{\mathbf{TYPE}_{\mathit{IMC}_{\mathscr{M}_n}}}$\textit{PriceFrame}
 
\item $r$.x$<$$r'$.x

\item $r$.e-time$<$$r'$.e-time

\item $r$.commodity=$r'$.commodity
 
\end{enumerate} 
 
\end{enumerate}

\subsection{Syntactic rules with compositional semantics}

Our syntactic rules define types of phrases (phrasal types) on the basis of lexical
items (objects of lexical types) and phrases (objects of phrasal
types) which are used to construct these phrases.  The
rules we use will be either unary (employing a function of the form $\lambda
r$:$T_1(T_2)$ where $T_2$ can depend on $r$)
or binary (employing a function of the form $\lambda r_1$:$T_1$($\lambda r_2$:$T_2(T_3))$ where
$T_3$ can depend on $r_1$ and $r_2$)
although there is no reason why rules should not take
greater numbers of arguments.  The general idea is that if $T$ is a
phrasal type and $r:T$ then $r.$s-event.phon is a phrase of the
language of category $r$.synsem.cat with interpretation $r$.synsem.cnt.

We use a number of auxiliary types in the definition of our rules.
These are:
\begin{enumerate}

\renewcommand{\labelenumi}{Ty\arabic{enumi}.}

\item
  \record{\tfield{synsem}{\record{\mfield{finite}{1}{\textit{Bool}}}}}

This is the type used for finite verbs and verb-phrases which we will
designate by \textit{Finite}.

\item \record{\tfield{quant}{$^\sigma$\textit{Quant}} \\
              \tfield{index}{\textit{Nat}}}

This is the type of binding operators used in the quantifier store
(``qstore'') used in the treatment of quantifier scope ambiguities.
We will use \textit{BO} to represent this type.
 
\item
  \record{\tfield{local}{\{\textit{Nat}\}} \\
          \tfield{non-local}{\{\textit{Nat}\}} \\
         % \tfield{slash}{\textit{Bool}} \\
          \tfield{qstore}{\{\textit{BO}\}}}

This is a type which performs record-keeping functions for anaphora.
It keeps track of local and non-local indices on pronouns\ignore{, slash
categories corresponding to gaps which are to be bound by
\textit{wh}-phrases} and a quantifier store (``qstore'') used to give
wide scope interpretations to quantifiers.  We will use \textit{Anaph}
to denote this type.
 
\item  \record{\mfield{local}{$\emptyset$}{\{\textit{Nat}\}} \\
          \mfield{non-local}{$\emptyset$}{\{\textit{Nat}\}} \\
          %\mfield{slash}{0}{\textit{Bool}} \\
          \mfield{qstore}{$\emptyset$}{\{\textit{BO}\}}}

This type corresponds to a situation where no pronouns\ignore{, gaps} or
wide-scope interpretations of quantifiers have been introduced.  We
refer to it as \textit{AnaphEmpty}.

\ignore{\item \record{\mfield{slash}{0}{\textit{Bool}}}

We will refer to this type by \textit{SlashEmpty}.

\item \record{\mfield{slash}{1}{\textit{Bool}}}

We will refer to this type by \textit{SlashFull}.}

\item \record{\mfield{qstore}{$\emptyset$}{\{\textit{BO}\}}}

We will refer to this type as \textit{QStoreEmpty}.
 
\end{enumerate} 
  

We use a number of auxiliary operations in the definition of our
rules.  These are:
\begin{enumerate}

\renewcommand{\labelenumi}{Op\arabic{enumi}.} 
 
\item if $a$:\textit{Str} and $b$:\textit{Str} then $a^\frown
  b$:\textit{Str} where $a^\frown b$ is the concatenation of $a$ and
  $b$. Note that concatenation is an associative operation, that is
  $a^\frown(b^\frown c) = (a^\frown b)^\frown c$, and we will usually
  write $a^\frown b^\frown c$.  This is our standard operation for the
  combination of phonological strings.



\item if $r_1$:\textit{SEvent} and $r_2$:\textit{SEvent} then
  \flconcat{$r_1$}{$r_2$} is to be the type:
\begin{quote}
\textit{SEvent}\d{$\wedge$}\smallrecord{\smallmfield{phon}{$r_1$.phon$^\frown
    r_2$.phon}{\textit{Str}} \\
                                   \smalltfield{s-time}{\smallrecord{\smallmfield{start}{$r_1$.s-time.start}{\textit{Time}}
                                       \\
                                                           \smallmfield{end}{$r_2$.s-time.end}{\textit{Time}}}}}
\end{quote}

\item if $r_1$:\textit{SEvent}, $w$ is a word and $r_2$:\textit{SEvent} then
  \flconcat{$r_1$}{\flconcat{$w$}{$r_2$}} is to be the type:
\begin{quote}
\textit{SEvent}\d{$\wedge$}\smallrecord{\smallmfield{phon}{$r_1$.phon$^\frown w^\frown
    r_2$.phon}{\textit{Str}} \\
                                   \smalltfield{s-time}{\smallrecord{\smallmfield{start}{$r_1$.s-time.start}{\textit{Time}}
                                       \\
                                                           \smallmfield{end}{$r_2$.s-time.end}{\textit{Time}}}}}
\end{quote}

\item if $a$ : $^\sigma(T_1\rightarrow T_2)$ and $b$ : $^\sigma T_1$ then
  $a@b$ is
  $\lambda\sigma$:\textit{Seq$_{\mathit{Ind}}$}($a(\sigma)(b(\sigma))$),
  which is of type $^\sigma T_2$. The operation represented by `@'
  (related to the S-combinator of combinatory logic) is our standard
  operation for ``functional application'' in the construction of
  contents for complex phrases.

\item if $\sigma$:\textit{Seq}$_{\mathit{Ind}}$, $a$:\textit{Ind} and $n$:\textit{Nat},
  then $\sigma[a/n]$ is that function like $\sigma$ except for the
  possible difference that $\sigma[a/n](n)=a$.

\item If $n$:\textit{Nat} and $a$:\textit{Gen}, then the gender offset
  of $n$ with respect to $a$, genoff($n,a$), is
\begin{enumerate} 
 
\item $3n+2$, if $a$ is m 
 
\item $3n+3$, if $a$ is f

\item $3n+4$, if $a$ is n
 
\end{enumerate} 
  

\item if $p$ : $^\sigma$\textit{Prop} $\vee$
   $^\sigma$(\textit{Quant}$\rightarrow$\textit{Prop})\ignore{$\vee$ $^\sigma$(\textit{Prop}$\rightarrow$\textit{Prop})}, then
\begin{enumerate} 
 
\item if $p$ : $^\sigma$\textit{Prop}, then Pres($p$) is 
\begin{quote}
$\lambda t$:\textit{Time} \\
\hspace*{1em}($\lambda\sigma$:\textit{Seq}$_{\mathit{Ind}}$ \\
\hspace*{2em}($\lambda r$:\smallrecord{\smalltfield{x}{\textit{Ind}}}
\\
\hspace*{3em}($p(\sigma)(r)$\d{$\wedge$}\smallrecord{\smalltfield{e-time}{\textit{Time}}
  \\
                                                     \smalltfield{tns}{$\langle\lambda
                                                       v$:\textit{Time}(eq($v$,$t$)), $\langle$e-time$\rangle\rangle$}}))) 
 \end{quote}
\item if $p$ : $^\sigma$(\textit{Quant}$\rightarrow$\textit{Prop}), then Pres($p$) is 
\begin{quote}
$\lambda t$:\textit{Time} \\
\hspace*{1em}($\lambda\sigma$:\textit{Seq}$_{\mathit{Ind}}$ \\
\hspace*{2em}($\lambda q$:\textit{Quant} \\
\hspace*{3em}($\lambda r$:\smallrecord{\smalltfield{x}{\textit{Ind}}}
\\
\hspace*{4em}($p(\sigma)(q)(r)$\d{$\wedge$}\smallrecord{\smalltfield{e-time}{\textit{Time}}
  \\
                                                     \smalltfield{tns}{$\langle\lambda
                                                       v$:\textit{Time}(eq($v$,$t$)), $\langle$e-time$\rangle\rangle$}})))) 
 \end{quote}
\ignore{\item if $p$ : $^\sigma$(\textit{Prop}$\rightarrow$\textit{Prop}), then Pres($p$) is 
\begin{quote}
$\lambda t$:\textit{Time} \\
\hspace*{1em}($\lambda\sigma$:\textit{Seq}$_{\mathit{Ind}}$ \\
\hspace*{2em}($\lambda a$:\textit{Prop} \\
\hspace*{3em}($\lambda r$:\smallrecord{\smalltfield{x}{\textit{Ind}}}
\\
\hspace*{4em}($p(\sigma)(a)(r)$\d{$\wedge$}\smallrecord{\smalltfield{e-time}{\textit{Time}}
  \\
                                                     \smalltfield{tns}{$\langle\lambda
                                                       v$:\textit{Time}(eq($v$,$t$)), $\langle$e-time$\rangle\rangle$}})))) 
 \end{quote}
}
\end{enumerate} 
  

\item If $r$:\textit{Anaph} then lnl($r$) (``move local to
  non-local'') is like $r$
  except that lnl($r$).local = $\emptyset$ and
  lnl($r$).non-local = $r$.non-local $\cup$ $r$.local. 

\item If $r_1$:\textit{Anaph} and $r_2$:\textit{Anaph}, then
  $r_1\sqcup r_2$ is
\begin{display}
\record{\field{local}{$r_1$.local$\cup r_2$.local} \\
          \field{non-local}{$r_1$.non-local$\cup r_2$.non-local} \\
          %\field{slash}{max($r_1$.slash, $r_2$.slash)} \\
          \field{qstore}{$r_1$.qstore$\cup r_2$.qstore}}
\end{display}

 
\item if $a$ is the indefinite article string ``a'' and $b$ is a string
  beginning with a vowel, then
  morph$_{\mathrm{det}}$($a$,$b$)=``an''. Otherwise for any strings
  $a$ and $b$, morph$_{\mathrm{det}}$($a$,$b$)=$a$.

\item if $w$ is a word then addS($w$) is 
\begin{enumerate} 
 
\item $w'$+``s'' if $w$ ends in ``y'' and $w'$ is the result of
  replacing the final ``y'' with ``ie''
 
\item $w$+``es'' if $w$ ends in ``sh''

\item $w$+``s'' otherwise
 
\end{enumerate} 

\item if $w$ is a word then Acc($w$) (the accusative form of $w$) is
\begin{enumerate} 
 
\item ``him'', if $w$ is ``he'' 
 
\item ``her'', if $w$ is ``she'' 

\item  $w$ otherwise

\end{enumerate} 
  
  
 
\end{enumerate} 
  
We now define our rules corresponding to a small subset of the rules in Montague's PTQ.

\begin{enumerate}

\renewcommand{\labelenumi}{R\arabic{enumi}.}
 
\item (corresponding to S1 and T1 in PTQ) \label{rule:lexphrases}

\begin{enumerate} 
 
\item if for some category $A$, $l\in B_A$ (that is, $l$ is a lexical
  type of category $A$) and
  $l\not=$lex$_{\mathrm{T}\textrm{-}\mathrm{Pron}}$($s,n$) for any
  string $s$ or natural number $n$, then 
\begin{display}
$l$\d{$\wedge$}\smallrecord{\smalltfield{synsem}{\smallrecord{\smallmfield{finite}{0}{\textit{Bool}} \\
                                                              \smalltfield{anaph}{\textit{AnaphEmpty}}}}}
\end{display}
is a member of $P_A$ (that is, it is a phrasal
  type of category $A$). 
 
\item  $l=$lex$_{\mathrm{T}\textrm{-}\mathrm{Pron}}$($s,n$) for some
  string $s$ and natural number $n$, then 
\begin{display}
$l$\d{$\wedge$}\smallrecord{\smalltfield{synsem}{\smallrecord{\smallmfield{finite}{0}{\textit{Bool}} \\
                                                              \smalltfield{anaph}{\record{\mfield{local}{\{$n$\}}{\{\textit{Nat}\}} \\
                                                                                          \mfield{non-local}{$\emptyset$}{\{\textit{Nat}\}}
                                                                       \\
                                                                                          %\mfield{slash}{0}{\textit{Bool}}
                                                                     %  \\
                                                                                          \mfield{qstore}{$\emptyset$}{\{\textit{BO}\}}}}}}}
\end{display}
is a member of $P_T$ (that is, it is a phrasal
  type of category $T$).

\item if $l\in B_{\mathit{IV}}$, and $r:l$, then 
\begin{quote}
\hspace*{-10em}\textit{Sign}\d{$\wedge$}\\
\hspace*{-10em}\record{\tfield{s-event}{\record{\mfield{phon}{addS($r$.s-event.phon)}{\textit{Str}} \\
                                  \tfield{s-time}{\record{\tfield{start}{\textit{Time}}}}}}
  \\
        \tfield{synsem}{\record{\mfield{cat}{$r$.synsem.cat}{\textit{Cat}}
            \\
                                \tfield{agr}{\record{\tfield{gen}{\textit{Gen}}
                                    \\
                                                     \mfield{num}{sg}{\textit{Num}}
                                                     \\
                                                     \mfield{pers}{3rd}{\textit{Pers}}}}
                                                 \\
                                \mfield{finite}{1}{\textit{Bool}} \\
                                \tfield{anaph}{\textit{AnaphEmpty}} \\
                                \tfield{cnt}{$\langle\lambda
                                  v$:\textit{Time}(($^\sigma$\textit{Prop})$_{\mathrm{Pres}(r.\mathrm{synsem}.\mathrm{cnt})(v)}$),
                                $\langle$s-event.s-time.start$\rangle\rangle$}}}} 

\end{quote}
$\in P_{\mathit{IV}}$

\item if $l\in B_{\mathit{TV}}$, and $r:l$, then 
\begin{quote}
\hspace*{-10em}\textit{Sign}\d{$\wedge$}\\
\hspace*{-10em}\smallrecord{\smalltfield{s-event}{\smallrecord{\smallmfield{phon}{addS($r$.s-event.phon)}{\textit{Str}} \\
                                  \smalltfield{s-time}{\smallrecord{\smalltfield{start}{\textit{Time}}}}}}
  \\
        \smalltfield{synsem}{\smallrecord{\smallmfield{cat}{$r$.synsem.cat}{\textit{Cat}}
            \\
                                         \smalltfield{agr}{\smallrecord{\smalltfield{gen}{\textit{Gen}}
                                    \\
                                                                        \smallmfield{num}{sg}{\textit{Num}}
                                                     \\
                                                                        \smallmfield{pers}{3rd}{\textit{Pers}}}}
                                                 \\
                                         \smallmfield{finite}{1}{\textit{Bool}} \\
                                         \smalltfield{anaph}{\textit{AnaphEmpty}} \\
                                 \smalltfield{cnt}{$\langle\lambda
                                  v$:\textit{Time}(($^\sigma$(\textit{Quant}$\rightarrow$\textit{Prop}))$_{\mathrm{Pres}(r.\mathrm{synsem}.\mathrm{cnt})(v)}$),
                                $\langle$s-event.s-time.start$\rangle\rangle$}}}}  

\end{quote}
$\in P_{\mathit{TV}}$

\ignore{\item if $l\in B_{\mathit{IV//IV}}$, and $r:l$, then 
\begin{quote}
\hspace*{-10em}\textit{Sign}\d{$\wedge$}\\
\hspace*{-10em}\smallrecord{\smalltfield{s-event}{\smallrecord{\smallmfield{phon}{addS($r$.s-event.phon)}{\textit{Str}} \\
                                  \smalltfield{s-time}{\smallrecord{\smalltfield{start}{\textit{Time}}}}}}
  \\
        \smalltfield{synsem}{\smallrecord{\smallmfield{cat}{$r$.synsem.cat}{\textit{Cat}}
            \\
                                         \smalltfield{agr}{\smallrecord{\smalltfield{gen}{\textit{Gen}}
                                    \\
                                                                        \smallmfield{num}{sg}{\textit{Num}}
                                                     \\
                                                                        \smallmfield{pers}{3rd}{\textit{Pers}}}}
                                                 \\
                                         \smallmfield{finite}{1}{\textit{Bool}} \\
                                         \smalltfield{anaph}{\textit{AnaphEmpty}} \\
                                 \smalltfield{cnt}{$\langle\lambda
                                  v$:\textit{Time}(($^\sigma$(\textit{Prop}$\rightarrow$\textit{Prop}))$_{\mathrm{Pres}(r.\mathrm{synsem}.\mathrm{cnt})(v)}$),
                                $\langle$s-event.s-time.start$\rangle\rangle$}}}}  

\end{quote}
$\in P_{\mathit{IV//IV}}$
} 

\item if $l\in$ \textit{B}$_{\mathit{T}}$, $n$:\textit{Nat} and $r:l$,
  then \\
$\lambda r$:\smallrecord{\smalltfield{s-event}{\textit{SEvent}} \\
                         \smalltfield{synsem}{\smallrecord{\smallmfield{cat}{T}{\textit{Cat}}
                           \\
                                                           \smalltfield{agr}{\textit{Agr}}
                                                           \\
                                                           \smalltfield{cnt}{$^\sigma$\textit{Quant}}}}}
                                                     \\
\hspace*{1em}(\smallrecord{\smallmfield{s-event}{$r$.s-event}{\textit{SEvent}}
  \\
                           \smalltfield{synsem}{\smallrecord{\smallmfield{cat}{$r$.synsem.cat}{\textit{Cat}}
                               \\
                                                             \smallmfield{agr}{$r$.synsem.agr}{\textit{Agr}}
                                                             \\
                                                             \smallmfield{finite}{0}{\textit{Bool}}
                                                             \\
                                                             \smalltfield{anaph}{\smallrecord{\smallmfield{local}{$\emptyset$}{\{\textit{Nat}\}} \\
                                                                                              \smallmfield{non-local}{$\emptyset$}{\{\textit{Nat}\}}
                                                                       \\
                                                                                              %\smallmfield{slash}{0}{\textit{Bool}}
                                                                   %     \\
                                                                                              \smallmfield{qstore}{\{\smallrecord{\smallmfield{quant}{$r$.synsem.cnt}{$^\sigma$\textit{Quant}}\\
                                                                                                                                  \smallmfield{index}{genoff($n$,
                                                                                                                                  $r$.synsem.agr.gen)}{\textit{Nat}}}\}}{\{\textit{BO}\}}}}
                                                                                                                          \\
                                                           \smallmfield{cnt}{$\lambda
                                  \sigma$:\textit{Seq}$_{\mathit{Ind}}$
                                  \\ \hspace*{3em}($\lambda
                                  v$:\textit{Prop}($v$(\smallrecord{\field{x}{$\sigma_{\mathrm{genoff}n,r.\mathrm{synsem.agr.gen}}$}})))}{\textit{$^\sigma$Quant}}}}})($l$)
                          \\
$\in P_T$
                                                                                                                          
                      
\end{enumerate} 
  
 
\item  \label{rule:detcn}(corresponding to S2 and T2 in PTQ) \\
If $T_1\in P_{\mathrm{T/CN}}$ and $T_2\in P_{\mathrm{CN}}$,
  $a:T_1$ and \\$b:T_2\wedge$\smallrecord{
                    \smalltfield{synsem}{\smallrecord{
                                                       \smalltfield{anaph}{\ignore{\textit{SlashEmpty}\d{$\wedge$}}\textit{QStoreEmpty}}}}},
                                                \\ then \\ \hspace*{-4em}$\lambda
  r_1$:\smallrecord{\smalltfield{s-event}{\smallrecord{\smalltfield{phon}{\textit{Str}}}}
    \\
                    \smalltfield{synsem}{\smallrecord{\smallmfield{cat}{T/CN}{\textit{Cat}}
                          \\
                                                      \smalltfield{cnt}{$^\sigma$(\textit{Prop}$\rightarrow$\textit{Quant})}}}}
                                              \\
\hspace*{-3em}$\lambda r_2$:\smallrecord{\smalltfield{s-event}{\smallrecord{\smalltfield{phon}{\textit{Str}}}}
    \\
                    \smalltfield{synsem}{\smallrecord{\smallmfield{cat}{CN}{\textit{Cat}}
                          \\
                                                      \smalltfield{agr}{\textit{Agr}}
                                                      \\
                                                      \smalltfield{anaph}{\textit{Anaph}\ignore{\d{$\wedge$}\textit{SlashEmpty}}\d{$\wedge$}\textit{QStoreEmpty}} \\
                                                      \smalltfield{cnt}{$^\sigma$\textit{Prop}}}}}
                                              \\
\hspace*{-2em}(\smallrecord{\smalltfield{s-event}{\smallrecord{\smallmfield{phon}{morph$_{\mathrm{det}}$($r_1$.s-event.phon,$r_2$.s-event.phon)$^\frown
        r_2$.s-event.phon}{\textit{Str}} \\ 
                                                              \smalltfield{s-time}{\smallrecord{\smallmfield{start}{$r_1$.s-event.s-time.start}{\textit{Time}}
                                                                \\
                                                                                                \smallmfield{end}{$r_2$.s-event.s-time.end}{\textit{Time}}}}}\d{$\wedge$}\textit{SEvent}}
                                                                     \\
                           \smalltfield{synsem}{\smallrecord{\smallmfield{cat}{T}{\textit{Cat}}
                               \\
                                                             \smallmfield{agr}{$r_2$.agr}{\textit{Agr}}
                                                             \\
                                                             \smallmfield{finite}{0}{\textit{Bool}} \\
                                                             \smallmfield{anaph}{lnl($r_2$.anaph)}{\textit{Anaph}} \\
                                                             \smallmfield{cnt}{$r_1$.synsem.cnt@$r_2$.synsem.cnt}{$^\sigma$\textit{Quant}}}}})($a$)($b$)
                                                       $\in$
                                                       $P_{\mathrm{T}}$.

%\item \textit{omitted}
\ignore{\begin{enumerate} 
 
\item If $T_1\in P_{\mathrm{CN}}$ and $T_2\in P_{\mathrm{t}}$, $a:T_1$
  and
  $b:T_2\wedge$\smallrecord{\smalltfield{synsem}{\smallrecord{\smalltfield{anaph}{\textit{SlashFull}}}}},
  \\
then \\
\hspace*{-9em}$\lambda
  r_1$:\smallrecord{\smalltfield{s-event}{\smallrecord{\smalltfield{phon}{\textit{Str}}}}
    \\
                    \smalltfield{synsem}{\smallrecord{\smallmfield{cat}{CN}{\textit{Cat}}
                          \\
                                                      \smalltfield{agr}{\textit{Agr}} \\
                                                      \smalltfield{cnt}{$^\sigma$\textit{Prop}}}}}
                                              \\
\hspace*{-8em}$\lambda r_2$:\smallrecord{\smalltfield{s-event}{\smallrecord{\smalltfield{phon}{\textit{Str}}}}
    \\
                    \smalltfield{synsem}{\smallrecord{\smallmfield{cat}{t}{\textit{Cat}}
                          \\
                                                      \smalltfield{anaph}{\textit{Anaph}\d{$\wedge$}\textit{SlashFull}} \\
                                                      \smalltfield{cnt}{$^\sigma$\textit{RecType}}}}}
                                              \\
\hspace*{-7em}(\smallrecord{\smalltfield{s-event}{\flconcat{$r_1$.s-event}{\flconcat{``that''}{$r_2$.s-event}}} \\
                           \smalltfield{synsem}{\smallrecord{\smallmfield{cat}{CN}{\textit{Cat}}
                               \\
                                                             \smallmfield{agr}{$r_1$.agr}{\textit{Agr}}
                                                             \\
                                                             \smallmfield{finite}{0}{\textit{Bool}} \\
                                                             \smalltfield{anaph}{\smallrecord{\smallmfield{local}{$r_1$.synsem.anaph.local}{\{\textit{Nat}\}}
                                                                 \\
                                                                                              \smallmfield{non-local}{$r_1$.synsem.anaph.non-local$\cup$lnl($r_2$.synsem.anaph).non-local}{\{\textit{Nat}\}}
                                                                                            \\
                                                                                              \smallmfield{slash}{0}{\textit{Bool}}\\
                                                                                              \smallmfield{qstore}{$r_1$.synsem.anaph.qstore$\cup
                                                                                                r_2$.synsem.anaph.qstore}{\{\textit{BO}\}}}}
                                                                     \\
                                                             \smallmfield{cnt}{$\lambda\sigma$:\textit{Seq$_{\mathit{Ind}}$}\\
\hspace*{4em}($\lambda r$:\smallrecord{\smalltfield{x}{\textit{Ind}}}\\
\hspace*{5em}(\smallrecord{\smalltfield{c}{\smallrecord{\smalltfield{pred}{$r_1$.cnt($\sigma$)($r$)} \\
                            \smalltfield{mod}{$r_2$.cnt($\sigma$[$r$.x/1])($r$)}}}}))}{$^\sigma$\textit{Prop}}}}})($a$)($b$)
                                                       $\in$ $P_{\mathrm{CN}}$.
\item If $T_1\in P_{\mathrm{CN}}$ and $T_2\in P_{\mathrm{IV}}$, $a:T_1$
  and \\
  $b:T_2\wedge$\smallrecord{\smalltfield{synsem}{\smallrecord{\smalltfield{anaph}{\textit{SlashEmpty}}}}},
then \\
\hspace*{-9em}$\lambda
  r_1$:\smallrecord{\smalltfield{s-event}{\smallrecord{\smalltfield{phon}{\textit{Str}}}}
    \\
                    \smalltfield{synsem}{\smallrecord{\smallmfield{cat}{CN}{\textit{Cat}}
                          \\
                                                      \smalltfield{agr}{\textit{Agr}} \\
                                                      \smalltfield{cnt}{$^\sigma$\textit{Prop}}}}}
                                              \\
\hspace*{-8em}$\lambda r_2$:\smallrecord{\smalltfield{s-event}{\smallrecord{\smalltfield{phon}{\textit{Str}}}}
    \\
                    \smalltfield{synsem}{\smallrecord{\smallmfield{cat}{IV}{\textit{Cat}}
                          \\
                                                      \smalltfield{anaph}{\textit{Anaph}\d{$\wedge$}\textit{SlashEmpty}} \\
                                                      \smalltfield{cnt}{$^\sigma$\textit{RecType}}}}}
                                              \\
\hspace*{-7em}(\smallrecord{\smalltfield{s-event}{\flconcat{$r_1$.s-event}{\flconcat{``that''}{$r_2$.s-event}}} \\
                           \smalltfield{synsem}{\smallrecord{\smallmfield{cat}{CN}{\textit{Cat}}
                               \\
                                                             \smallmfield{agr}{$r_1$.agr}{\textit{Agr}}
                                                             \\
                                                             \smallmfield{finite}{0}{\textit{Bool}} \\
                                                             \smalltfield{anaph}{\smallrecord{\smallmfield{local}{$r_1$.synsem.anaph.local}{\{\textit{Nat}\}}
                                                                 \\
                                                                                              \smallmfield{non-local}{$r_1$.synsem.anaph.non-local$\cup$lnl($r_2$.synsem.anaph).non-local}{\{\textit{Nat}\}}
                                                                                            \\
                                                                                              \smallmfield{slash}{0}{\textit{Bool}}\\
                                                                                              \smallmfield{qstore}{$r_1$.synsem.anaph.qstore$\cup
                                                                                                r_2$.synsem.anaph.qstore}{\{\textit{BO}\}}}}
                                                                     \\
                                                             \smallmfield{cnt}{$\lambda\sigma$:\textit{Seq$_{\mathit{Ind}}$}\\
\hspace*{4em}($\lambda r$:\smallrecord{\smalltfield{x}{\textit{Ind}}}\\
\hspace*{5em}(\smallrecord{\smalltfield{c}{\smallrecord{\smalltfield{pred}{$r_1$.synsem.cnt($\sigma$)($r$)} \\
                            \smalltfield{mod}{$r_2$.synsem.cnt($\sigma$)($r$)}}}}))}{$^\sigma$\textit{Prop}}}}})($a$)($b$)
                                                       $\in$ $P_{\mathrm{CN}}$. 
 
\end{enumerate} 
}

\item \label{rule:tiv}(corresponding to S4 and T4 in PTQ) \\ 
If $T_1\in P_{\mathrm{T}}$ and $T_2\in P_{\mathrm{IV}}$, $a:T_1$
  and $b:T_2$\d{$\wedge$}\textit{Sg3rd}\d{$\wedge$}\textit{Finite}, then \\
$\lambda
r_1$:\smallrecord{\smalltfield{s-event}{\smallrecord{\smalltfield{phon}{\textit{Str}}}}
  \\
                  \smalltfield{synsem}{\smallrecord{\smallmfield{cat}{T}{\textit{Cat}}
                      \\
                                                    \smalltfield{anaph}{\textit{Anaph}} \\
                                                    \smalltfield{cnt}{$^\sigma$\textit{Quant}}}}}
                                              \\
\hspace*{1em}$\lambda
r_2$:\smallrecord{\smalltfield{s-event}{\smallrecord{\smalltfield{phon}{\textit{Str}}}}
  \\
                  \smalltfield{synsem}{\smallrecord{\smallmfield{cat}{IV}{\textit{Cat}}
                      \\
                                                    \smalltfield{anaph}{\textit{Anaph}}
                                                    \\
                                                    \smalltfield{cnt}{$^\sigma$\textit{Prop}}}}}
                                              \\
\hspace*{2em}(\smallrecord{\smalltfield{s-event}{\flconcat{$r_1$.s-event}{$r_2$.s-event}} \\

              \smalltfield{synsem}{\smallrecord{\smallmfield{cat}{t}{\textit{Cat}}
                  \\
                                                \smallmfield{finite}{1}{\textit{Bool}} \\
                                                \smallmfield{anaph}{$r_1$.synsem.anaph$\sqcup
                                                  r_2$.synsem.anaph}{\textit{Anaph}}
                                                \\
                                                \smallmfield{cnt}{$r_1$.synsem.cnt@$r_2$.synsem.cnt}{$^\sigma$\textit{RecType}}}}})($a$)($b$)
                                          \\
$\in P_{\mathrm{t}}$

\item (corresponding to S5 and T5 in PTQ) \\
\ignore{\begin{enumerate}
\item} If $T_1\in P_{\mathrm{TV}}$ and $T_2\in P_{\mathrm{T}}$, $a:T_1$
  and $b:T_2$, then \\
$\lambda
r_1$:\smallrecord{\smalltfield{s-event}{\textit{SEvent}}
  \\
                  \smalltfield{synsem}{\smallrecord{\smallmfield{cat}{TV}{\textit{Cat}}
                      \\
                                                    \smalltfield{agr}{\textit{Agr}}
                                                    \\
                                                    \smalltfield{finite}{\textit{Bool}} \\
                                                    \smalltfield{anaph}{\textit{Anaph}} \\
                                                    \smalltfield{cnt}{$^\sigma$(\textit{Quant}$\rightarrow$\textit{Prop})}}}}
                                              \\
\hspace*{1em}$\lambda
r_2$:\smallrecord{\smalltfield{s-event}{\textit{SEvent}}
  \\
                  \smalltfield{synsem}{\smallrecord{\smallmfield{cat}{T}{\textit{Cat}}
                      \\
                                                    \smalltfield{anaph}{\textit{Anaph}}
                                                    \\
                                                    \smalltfield{cnt}{$^\sigma$\textit{Quant}}}}}
                                              \\
\hspace*{2em}(\smallrecord{\smalltfield{s-event}{\smallrecord{\smallmfield{phon}{$r_1$.s-event.phon$^\frown$Acc($r_2$.s-event.phon)}{\textit{Str}}
    \\
                                                              \smalltfield{s-time}{\smallrecord{\smallmfield{start}{$r_1$.s-event.s-time.start}{\textit{Time}}
                                                                \\
                                                                                                \smallmfield{end}{$r_2$.s-event.s-time.end}{\textit{Time}}}}}\d{$\wedge$}\textit{SEvent}} \\

              \smalltfield{synsem}{\smallrecord{\smallmfield{cat}{IV}{\textit{Cat}}
                  \\
                                                \smallmfield{agr}{$r_1$.synsem.agr}{\textit{Agr}} \\
                                                \smallmfield{finite}{$r_1$.synsem.finite}{\textit{Bool}} \\
                                                \smallmfield{anaph}{$r_1$.synsem.anaph$\sqcup
                                                  r_2$.synsem.anaph}{\textit{Anaph}}
                                                \\
                                                \smallmfield{cnt}{$r_1$.synsem.cnt@$r_2$.synsem.cnt}{$^\sigma$\textit{Prop}}}}})($a$)($b$)
                                          \\
$\in P_{\mathrm{IV}}$

\ignore{\item If $T_1\in P_{\mathrm{TV}}$ and $a:T_1$\d{$\wedge$}\textit{SlashEmpty}, then \\
$\lambda
r_1$:\smallrecord{\smalltfield{s-event}{\textit{SEvent}}
  \\
                  \smalltfield{synsem}{\smallrecord{\smallmfield{cat}{TV}{\textit{Cat}}
                      \\
                                                    \smalltfield{agr}{\textit{Agr}}
                                                    \\
                                                    \smalltfield{finite}{\textit{Bool}} \\
                                                    \smalltfield{anaph}{\textit{Anaph}} \\
                                                    \smalltfield{cnt}{$^\sigma$(\textit{Quant}$\rightarrow$\textit{Prop})}}}}
                                              \\

\hspace*{1em}(\smallrecord{\smallmfield{s-event}{$r_1$.s-event}{\textit{SEvent}} \\
                           \smalltfield{synsem}{\smallrecord{\smallmfield{cat}{IV}{\textit{Cat}}
                  \\
                                                \smallmfield{agr}{$r_1$.synsem.agr}{\textit{Agr}} \\
                                                \smallmfield{finite}{$r_1$.synsem.finite}{\textit{Bool}} \\
                                                \smalltfield{anaph}{\smallrecord{\smallmfield{local}{$r_1$.synsem.anaph.local}{\{\textit{Nat}\}} \\
                                                                                 \smallmfield{non-local}{$r_1$.synsem.anaph.non-local}{\{\textit{Nat}\}}
                                                                       \\
                                                                                 \smallmfield{slash}{1}{\textit{Bool}}
                                                                        \\
                                                                                 \smallmfield{qstore}{$r_1$.synsem.anaph.qstore}{\{\textit{BO}\}}}}
                                                \\
                                                \smallmfield{cnt}{$r_1$.synsem.cnt@$\lambda
                                  \sigma$:\textit{Seq}$_{\mathit{Ind}}$
                                  ($\lambda
                                  v$:\textit{Prop}($v$(\smallrecord{\field{x}{$\sigma_1$}})))}{$^\sigma$\textit{Prop}}}}})($a$)
                                          \\
$\in P_{\mathrm{IV}}$

\end{enumerate}
}

\ignore{\item \textit{omitted}}
\ignore{
\begin{enumerate} 
 
\item If $T_1\in P_{\mathrm{IAV/T}}$ and $T_2\in P_{\mathrm{T}}$, $a:T_1$
  and $b:T_2$, then \\
$\lambda
r_1$:\smallrecord{\smalltfield{s-event}{\smallrecord{\smalltfield{phon}{\textit{Str}}}}
  \\
                  \smalltfield{synsem}{\smallrecord{\smallmfield{cat}{IAV/T}{\textit{Cat}}
                      \\
                                                    \smalltfield{agr}{\textit{Agr}}
                                                    \\
                                                    \smalltfield{finite}{\textit{Bool}} \\
                                                    \smalltfield{anaph}{\textit{Anaph}} \\
                                                    \smalltfield{cnt}{$^\sigma$(\textit{Quant}$\rightarrow$(\textit{Prop}$\rightarrow$\textit{Prop}))}}}}
                                              \\
\hspace*{1em}$\lambda
r_2$:\smallrecord{\smalltfield{s-event}{\smallrecord{\smalltfield{phon}{\textit{Str}}}}
  \\
                  \smalltfield{synsem}{\smallrecord{\smallmfield{cat}{T}{\textit{Cat}}
                      \\
                                                    \smalltfield{anaph}{\textit{Anaph}}
                                                    \\
                                                    \smalltfield{cnt}{$^\sigma$\textit{Quant}}}}}
                                              \\
\hspace*{2em}(\smallrecord{\smalltfield{s-event}{\smallrecord{\smallmfield{phon}{$r_1$.s-event.phon$^\frown$Acc($r_2$.s-event.phon)}{\textit{Str}}}} \\

              \smalltfield{synsem}{\smallrecord{\smallmfield{cat}{IAV}{\textit{Cat}}
                  \\
                                                \smallmfield{agr}{$r_1$.synsem.agr}{\textit{Agr}} \\
                                                \smallmfield{finite}{$r_1$.synsem.finite}{\textit{Bool}} \\
                                                \smallmfield{anaph}{$r_1$.synsem.anaph$\sqcup
                                                  r_2$.synsem.anaph}{\textit{Anaph}}
                                                \\
                                                \smallmfield{cnt}{$r_1$.synsem.cnt@$r_2$.synsem.cnt}{$^\sigma$(\textit{Prop}$\rightarrow$\textit{Prop})}}}})($a$)($b$)
                                          \\
$\in P_{\mathrm{IAV}}$ 
 
\item If $T_1\in P_{\mathrm{IAV/T}}$ and  $a:T_1$\d{$\wedge$}\textit{SlashEmpty}, then \\
$\lambda
r_1$:\smallrecord{\smalltfield{s-event}{\smallrecord{\smalltfield{phon}{\textit{Str}}}}
  \\
                  \smalltfield{synsem}{\smallrecord{\smallmfield{cat}{IAV/T}{\textit{Cat}}
                      \\
                                                    \smalltfield{agr}{\textit{Agr}}
                                                    \\
                                                    \smalltfield{finite}{\textit{Bool}} \\
                                                    \smalltfield{anaph}{\textit{Anaph}} \\
                                                    \smalltfield{cnt}{$^\sigma$(\textit{Quant}$\rightarrow$(\textit{Prop}$\rightarrow$\textit{Prop}))}}}}
                                              \\
\hspace*{1em}(\smallrecord{\smalltfield{s-event}{\smallrecord{\smallmfield{phon}{$r_1$.s-event.phon$^\frown$Acc($r_2$.s-event.phon)}{\textit{Str}}}} \\

              \smalltfield{synsem}{\smallrecord{\smallmfield{cat}{IAV}{\textit{Cat}}
                  \\
                                                \smallmfield{agr}{$r_1$.synsem.agr}{\textit{Agr}} \\
                                                \smallmfield{finite}{$r_1$.synsem.finite}{\textit{Bool}} \\
                                                \smalltfield{anaph}{\smallrecord{\smallmfield{local}{$r_1$.synsem.anaph.local}{\{\textit{Nat}\}} \\
                                                                                 \smallmfield{non-local}{$r_1$.synsem.anaph.non-local}{\{\textit{Nat}\}}
                                                                       \\
                                                                                 \smallmfield{slash}{1}{\textit{Bool}}
                                                                        \\
                                                                                 \smallmfield{qstore}{$r_1$.synsem.anaph.qstore}{\{\textit{BO}\}}}}
                                                \\
                                                \smallmfield{cnt}{$r_1$.synsem.cnt@$\lambda
                                  \sigma$:\textit{Seq}$_{\mathit{Ind}}$
                                  ($\lambda
                                  v$:\textit{Prop}($v$(\smallrecord{\field{x}{$\sigma_1$}})))}{$^\sigma$(\textit{Prop}$\rightarrow$\textit{Prop})}}}})($a$)
                                          \\
$\in P_{\mathrm{IAV}}$ 
 
\end{enumerate} 
}  

\ignore{\item \textit{omitted}} \ignore{If $T_1\in P_{\mathrm{IV/t}}$ and $T_2\in P_{\mathrm{T}}$, $a:T_1$
  and $b:T_2$, then \\
$\lambda
r_1$:\smallrecord{\smalltfield{s-event}{\smallrecord{\smalltfield{phon}{\textit{Str}}}}
  \\
                  \smalltfield{synsem}{\smallrecord{\smallmfield{cat}{IV/t}{\textit{Cat}}
                      \\
                                                    \smalltfield{agr}{\textit{Agr}}
                                                    \\
                                                    \smalltfield{finite}{\textit{Bool}} \\
                                                    \smalltfield{anaph}{\textit{Anaph}} \\
                                                    \smalltfield{cnt}{$^\sigma$(\textit{RecType}$\rightarrow$\textit{Prop})}}}}
                                              \\
\hspace*{1em}$\lambda
r_2$:\smallrecord{\smalltfield{s-event}{\smallrecord{\smalltfield{phon}{\textit{Str}}}}
  \\
                  \smalltfield{synsem}{\smallrecord{\smallmfield{cat}{T}{\textit{Cat}}
                      \\
                                                    \smalltfield{anaph}{\textit{Anaph}}
                                                    \\
                                                    \smalltfield{cnt}{$^\sigma$\textit{RecType}}}}}
                                              \\
\hspace*{2em}(\smallrecord{\smalltfield{s-event}{\smallrecord{\smallmfield{phon}{$r_1$.s-event.phon$^\frown$``that''$^\frown
        r_2$.s-event.phon}{\textit{Str}}}} \\

              \smalltfield{synsem}{\smallrecord{\smallmfield{cat}{IV}{\textit{Cat}}
                  \\
                                                \smallmfield{agr}{$r_1$.synsem.agr}{\textit{Agr}} \\
                                                \smallmfield{finite}{$r_1$.synsem.finite}{\textit{Bool}} \\
                                                \smallmfield{anaph}{$r_1$.synsem.anaph$\sqcup
                                                  r_2$.synsem.anaph}{\textit{Anaph}}
                                                \\
                                                \smallmfield{cnt}{$r_1$.synsem.cnt@$r_2$.synsem.cnt}{$^\sigma$\textit{Prop}}}}})($a$)($b$)
                                          \\
$\in P_{\mathrm{IV}}$
}

\ignore{\item \textit{omitted}}\ignore{If $T_1\in P_{\mathrm{IV//IV}}$ and $T_2\in P_{\mathrm{IV}}$,
  $a:T_1$, $b:T_2$ and $a$.synsem.anaph.slash + $b$.synsem.anaph.slash : \textit{Bool}, then \\
$\lambda
r_1$:\smallrecord{\smalltfield{s-event}{\smallrecord{\smalltfield{phon}{\textit{Str}}}}
  \\
                  \smalltfield{synsem}{\smallrecord{\smallmfield{cat}{IV//IV}{\textit{Cat}}
                      \\
                                                    \smalltfield{agr}{\textit{Agr}}
                                                    \\
                                                    \smalltfield{finite}{\textit{Bool}} \\
                                                    \smalltfield{anaph}{\textit{Anaph}} \\
                                                    \smalltfield{cnt}{$^\sigma$(\textit{Prop}$\rightarrow$\textit{Prop})}}}}
                                              \\
\hspace*{1em}$\lambda
r_2$:\smallrecord{\smalltfield{s-event}{\smallrecord{\smalltfield{phon}{\textit{Str}}}}
  \\
                  \smalltfield{synsem}{\smallrecord{\smallmfield{cat}{IV}{\textit{Cat}}
                      \\
                                                    \smalltfield{anaph}{\textit{Anaph}}
                                                    \\
                                                    \smalltfield{cnt}{$^\sigma$\textit{Prop}}}}}
                                              \\
\hspace*{2em}(\smallrecord{\smalltfield{s-event}{\smallrecord{\smallmfield{phon}{$r_1$.s-event.phon$^\frown$``to''$^\frown
        r_2$.s-event.phon}{\textit{Str}}}} \\

              \smalltfield{synsem}{\smallrecord{\smallmfield{cat}{IV}{\textit{Cat}}
                  \\
                                                \smallmfield{agr}{$r_1$.synsem.agr}{\textit{Agr}} \\
                                                \smallmfield{finite}{$r_1$.synsem.finite}{\textit{Bool}} \\
                                                \smallmfield{anaph}{$r_1$.synsem.anaph$\sqcup
                                                  r_2$.synsem.anaph}{\textit{Anaph}}
                                                \\
                                                \smallmfield{cnt}{$r_1$.synsem.cnt@$r_2$.synsem.cnt}{$^\sigma$\textit{Prop}}}}})($a$)($b$)
                                          \\
$\in P_{\mathrm{IV}}$ 
}

\ignore{\item \textit{omitted}}\ignore{If $T_1\in P_{\mathrm{t/t}}$ and $T_2\in P_{\mathrm{t}}$, $a:T_1$
  and $b:T_2$ (and $a$.synsem.anaph.slash + $b$.synsem.anaph.slash : \textit{Bool}), then \\
$\lambda
r_1$:\smallrecord{\smalltfield{s-event}{\smallrecord{\smalltfield{phon}{\textit{Str}}}}
  \\
                  \smalltfield{synsem}{\smallrecord{\smallmfield{cat}{t/t}{\textit{Cat}}
                      \\
                                                    \smalltfield{agr}{\textit{Agr}}
                                                    \\
                                                    \smalltfield{finite}{\textit{Bool}} \\
                                                    \smalltfield{anaph}{\textit{Anaph}} \\
                                                    \smalltfield{cnt}{$^\sigma$(\textit{RecType}$\rightarrow$\textit{RecType})}}}}
                                              \\
\hspace*{1em}$\lambda
r_2$:\smallrecord{\smalltfield{s-event}{\smallrecord{\smalltfield{phon}{\textit{Str}}}}
  \\
                  \smalltfield{synsem}{\smallrecord{\smallmfield{cat}{t}{\textit{Cat}}
                      \\
                                                    \smalltfield{anaph}{\textit{Anaph}}
                                                    \\
                                                    \smalltfield{cnt}{$^\sigma$\textit{RecType}}}}}
                                              \\
\hspace*{2em}(\smallrecord{\smalltfield{s-event}{\smallrecord{\smallmfield{phon}{$r_1$.s-event.phon$^\frown
        r_2$.s-event.phon}{\textit{Str}}}} \\

              \smalltfield{synsem}{\smallrecord{\smallmfield{cat}{t}{\textit{Cat}}
                  \\
                                                \smallmfield{agr}{$r_1$.synsem.agr}{\textit{Agr}} \\
                                                \smallmfield{finite}{$r_1$.synsem.finite}{\textit{Bool}} \\
                                                \smallmfield{anaph}{$r_1$.synsem.anaph$\sqcup
                                                  r_2$.synsem.anaph}{\textit{Anaph}}
                                                \\
                                                \smallmfield{cnt}{$r_1$.synsem.cnt@$r_2$.synsem.cnt}{$^\sigma$\textit{RecType}}}}})($a$)($b$)
                                          \\
$\in P_{\mathrm{t}}$
}

\ignore{\item \textit{omitted}} \ignore{If $T_1\in P_{\mathrm{IV/IV}}$ and $T_2\in P_{\mathrm{IV}}$, $a:T_1$,
 $b:T_2$ and $a$.synsem.anaph.slash + $b$.synsem.anaph.slash : \textit{Bool}, then \\
$\lambda
r_1$:\smallrecord{\smalltfield{s-event}{\smallrecord{\smalltfield{phon}{\textit{Str}}}}
  \\
                  \smalltfield{synsem}{\smallrecord{\smallmfield{cat}{IV/IV}{\textit{Cat}}
                      \\
                                                    \smalltfield{agr}{\textit{Agr}}
                                                    \\
                                                    \smalltfield{finite}{\textit{Bool}} \\
                                                    \smalltfield{anaph}{\textit{Anaph}} \\
                                                    \smalltfield{cnt}{$^\sigma$(\textit{Prop}$\rightarrow$\textit{Prop})}}}}
                                              \\
\hspace*{1em}$\lambda
r_2$:\smallrecord{\smalltfield{s-event}{\smallrecord{\smalltfield{phon}{\textit{Str}}}}
  \\
                  \smalltfield{synsem}{\smallrecord{\smallmfield{cat}{IV}{\textit{Cat}}
                      \\
                                                    \smalltfield{anaph}{\textit{Anaph}}
                                                    \\
                                                    \smalltfield{cnt}{$^\sigma$\textit{Prop}}}}}
                                              \\
\hspace*{2em}(\smallrecord{\smalltfield{s-event}{\smallrecord{\smallmfield{phon}{$r_2$.s-event.phon$^\frown
        r_1$.s-event.phon}{\textit{Str}}}} \\

              \smalltfield{synsem}{\smallrecord{\smallmfield{cat}{IV}{\textit{Cat}}
                  \\
                                                \smallmfield{agr}{$r_1$.synsem.agr}{\textit{Agr}} \\
                                                \smallmfield{finite}{$r_1$.synsem.finite}{\textit{Bool}} \\
                                                \smallmfield{anaph}{$r_1$.synsem.anaph$\sqcup
                                                  r_2$.synsem.anaph}{\textit{Anaph}}
                                                \\
                                                \smallmfield{cnt}{$r_1$.synsem.cnt@$r_2$.synsem.cnt}{$^\sigma$\textit{Prop}}}}})($a$)($b$)
                                          \\
$\in P_{\mathrm{IV}}$ 
}

\ignore{
\item 
\begin{enumerate} 
 
\item  If $T_1,T_2\in P_{\mathrm{t}}$, $a:T_1$,
  $b:T_2$ and $a$.synsem.anaph.slash = $b$.synsem.anaph.slash, then \\
$\lambda
r_1$:\smallrecord{\smalltfield{s-event}{\smallrecord{\smalltfield{phon}{\textit{Str}}}}
  \\
                  \smalltfield{synsem}{\smallrecord{\smallmfield{cat}{t}{\textit{Cat}}
                      \\
                                                    \smalltfield{finite}{\textit{Bool}} \\
                                                    \smalltfield{anaph}{\textit{Anaph}} \\
                                                    \smalltfield{cnt}{$^\sigma$\textit{RecType}}}}}
                                              \\
\hspace*{1em}$\lambda
r_2$:\smallrecord{\smalltfield{s-event}{\smallrecord{\smalltfield{phon}{\textit{Str}}}}
  \\
                  \smalltfield{synsem}{\smallrecord{\smallmfield{cat}{t}{\textit{Cat}}
                      \\
                                                    \smalltfield{anaph}{\textit{Anaph}}
                                                    \\
                                                    \smalltfield{cnt}{$^\sigma$\textit{RecType}}}}}
                                              \\
\hspace*{2em}(\smallrecord{\smalltfield{s-event}{\smallrecord{\smallmfield{phon}{$r_1$.s-event.phon$^\frown$``and''$^\frown
        r_2$.s-event.phon}{\textit{Str}}}} \\

              \smalltfield{synsem}{\smallrecord{\smallmfield{cat}{t}{\textit{Cat}}
                  \\
                                                \smallmfield{finite}{$r_1$.synsem.finite}{\textit{Bool}} \\
                                                \smallmfield{anaph}{$r_1$.synsem.anaph$\sqcup
                                                  r_2$.synsem.anaph}{\textit{Anaph}}
                                                \\
                                                \smallmfield{cnt}{$\lambda
                                  \sigma$:\textit{Seq}$_{\mathit{Ind}}$(\smallrecord{\smalltfield{c$_1$}{$r_1$.synsem.cnt($\sigma$)}
                                      \\
                                                                                   \smalltfield{c$_2$}{$r_2$.synsem.cnt($\sigma$)}})}{$^\sigma$\textit{RecType}}}}})($a$)($b$)
                                          \\
$\in P_{\mathrm{t}}$ 

 
\item If $T_1,T_2\in P_{\mathrm{t}}$, $a:T_1$, $b:T_2$ and $a$.synsem.anaph.slash = $b$.synsem.anaph.slash, then \\
$\lambda
r_1$:\smallrecord{\smalltfield{s-event}{\smallrecord{\smalltfield{phon}{\textit{Str}}}}
  \\
                  \smalltfield{synsem}{\smallrecord{\smallmfield{cat}{t}{\textit{Cat}}
                      \\
                                                    \smalltfield{finite}{\textit{Bool}} \\
                                                    \smalltfield{anaph}{\textit{Anaph}} \\
                                                    \smalltfield{cnt}{$^\sigma$\textit{RecType}}}}}
                                              \\
\hspace*{1em}$\lambda
r_2$:\smallrecord{\smalltfield{s-event}{\smallrecord{\smalltfield{phon}{\textit{Str}}}}
  \\
                  \smalltfield{synsem}{\smallrecord{\smallmfield{cat}{t}{\textit{Cat}}
                      \\
                                                    \smalltfield{anaph}{\textit{Anaph}}
                                                    \\
                                                    \smalltfield{cnt}{$^\sigma$\textit{RecType}}}}}
                                              \\
\hspace*{2em}(\smallrecord{\smalltfield{s-event}{\smallrecord{\smallmfield{phon}{$r_1$.s-event.phon$^\frown$``or''$^\frown
        r_2$.s-event.phon}{\textit{Str}}}} \\

              \smalltfield{synsem}{\smallrecord{\smallmfield{cat}{t}{\textit{Cat}}
                  \\
                                                \smallmfield{finite}{$r_1$.synsem.finite}{\textit{Bool}} \\
                                                \smallmfield{anaph}{$r_1$.synsem.anaph$\sqcup
                                                  r_2$.synsem.anaph}{\textit{Anaph}}
                                                \\
                                                \smallmfield{cnt}{$\lambda
                                  \sigma$:\textit{Seq}$_{\mathit{Ind}}$(\smallrecord{\smalltfield{c}{$r_1$.synsem.cnt($\sigma$)$\vee
                                   r_2$.synsem.cnt($\sigma$)}})}{$^\sigma$\textit{RecType}}}}})($a$)($b$)
                                          \\
$\in P_{\mathrm{t}}$  
 
\end{enumerate} 

\item 
\begin{enumerate} 
 
\item  If $T_1,T_2\in P_{\mathrm{IV}}$, $a:T_1$,
  $b:T_2$ and $a$.synsem.anaph.slash = $b$.synsem.anaph.slash, then \\
$\lambda
r_1$:\smallrecord{\smalltfield{s-event}{\smallrecord{\smalltfield{phon}{\textit{Str}}}}
  \\
                  \smalltfield{synsem}{\smallrecord{\smallmfield{cat}{IV}{\textit{Cat}}
                      \\
                                                    \smalltfield{finite}{\textit{Bool}} \\
                                                    \smalltfield{anaph}{\textit{Anaph}} \\
                                                    \smalltfield{cnt}{$^\sigma$\textit{Prop}}}}}
                                              \\
\hspace*{1em}$\lambda
r_2$:\smallrecord{\smalltfield{s-event}{\smallrecord{\smalltfield{phon}{\textit{Str}}}}
  \\
                  \smalltfield{synsem}{\smallrecord{\smallmfield{cat}{IV}{\textit{Cat}}
                      \\
                                                    \smalltfield{anaph}{\textit{Anaph}}
                                                    \\
                                                    \smalltfield{cnt}{$^\sigma$\textit{Prop}}}}}
                                              \\
\hspace*{2em}(\smallrecord{\smalltfield{s-event}{\smallrecord{\smallmfield{phon}{$r_1$.s-event.phon$^\frown$``and''$^\frown
        r_2$.s-event.phon}{\textit{Str}}}} \\

              \smalltfield{synsem}{\smallrecord{\smallmfield{cat}{IV}{\textit{Cat}}
                  \\
                                                \smallmfield{finite}{$r_1$.synsem.finite}{\textit{Bool}} \\
                                                \smallmfield{anaph}{$r_1$.synsem.anaph$\sqcup
                                                  r_2$.synsem.anaph}{\textit{Anaph}}
                                                \\
                                                \smallmfield{cnt}{$\lambda
                                  \sigma$:\textit{Seq}$_{\mathit{Ind}}$($\lambda
                                  r$:\smallrecord{\smalltfield{x}{\textit{Ind}}}(\smallrecord{\smalltfield{c$_1$}{$r_1$.synsem.cnt($\sigma$)($r$)}
                                      \\
                                                                                   \smalltfield{c$_2$}{$r_2$.synsem.cnt($\sigma$)($r$)}}))}{$^\sigma$\textit{Prop}}}}}))($a$)($b$)
                                          \\
$\in P_{\mathrm{IV}}$ 

 
\item If $T_1,T_2\in P_{\mathrm{IV}}$, $a:T_1$, $b:T_2$ and $a$.synsem.anaph.slash = $b$.synsem.anaph.slash, then \\
\hspace*{-8em}$\lambda
r_1$:\smallrecord{\smalltfield{s-event}{\smallrecord{\smalltfield{phon}{\textit{Str}}}}
  \\
                  \smalltfield{synsem}{\smallrecord{\smallmfield{cat}{IV}{\textit{Cat}}
                      \\
                                                    \smalltfield{finite}{\textit{Bool}} \\
                                                    \smalltfield{anaph}{\textit{Anaph}} \\
                                                    \smalltfield{cnt}{$^\sigma$\textit{Prop}}}}}
                                              \\
\hspace*{-7em}$\lambda
r_2$:\smallrecord{\smalltfield{s-event}{\smallrecord{\smalltfield{phon}{\textit{Str}}}}
  \\
                  \smalltfield{synsem}{\smallrecord{\smallmfield{cat}{IV}{\textit{Cat}}
                      \\
                                                    \smalltfield{anaph}{\textit{Anaph}}
                                                    \\
                                                    \smalltfield{cnt}{$^\sigma$\textit{Prop}}}}}
                                              \\
\hspace*{-6em}(\smallrecord{\smalltfield{s-event}{\smallrecord{\smallmfield{phon}{$r_1$.s-event.phon$^\frown$``or''$^\frown
        r_2$.s-event.phon}{\textit{Str}}}} \\

              \smalltfield{synsem}{\smallrecord{\smallmfield{cat}{IV}{\textit{Cat}}
                  \\
                                                \smallmfield{finite}{$r_1$.synsem.finite}{\textit{Bool}} \\
                                                \smallmfield{anaph}{$r_1$.synsem.anaph$\sqcup
                                                  r_2$.synsem.anaph}{\textit{Anaph}}
                                                \\
                                                \smallmfield{cnt}{$\lambda
                                  \sigma$:\textit{Seq}$_{\mathit{Ind}}$($\lambda
                                  r$:\smallrecord{\smalltfield{x}{\textit{Ind}}}(\smallrecord{\smalltfield{c}{$r_1$.synsem.cnt($\sigma$)($r$)$\vee
                                   r_2$.synsem.cnt($\sigma$)($r$)}}))}{$^\sigma$\textit{Prop}}}}})($a$)($b$)
                                          \\
$\in P_{\mathrm{IV}}$  
 
\end{enumerate}

\item If $T_1,T_2\in P_{\mathrm{T}}$,
  $a:T_1$\d{$\wedge$}\textit{SlashEmpty} and $b:T_2$\d{$\wedge$}\textit{SlashEmpty}, then \\
\hspace*{-5em}$\lambda
r_1$:\smallrecord{\smalltfield{s-event}{\smallrecord{\smalltfield{phon}{\textit{Str}}}}
  \\
                  \smalltfield{synsem}{\smallrecord{\smallmfield{cat}{T}{\textit{Cat}}
                      \\
                                                    \smalltfield{finite}{\textit{Bool}} \\
                                                    \smalltfield{anaph}{\textit{Anaph}} \\
                                                    \smalltfield{cnt}{$^\sigma$\textit{Quant}}}}}
                                              \\
\hspace*{-4em}$\lambda
r_2$:\smallrecord{\smalltfield{s-event}{\smallrecord{\smalltfield{phon}{\textit{Str}}}}
  \\
                  \smalltfield{synsem}{\smallrecord{\smallmfield{cat}{T}{\textit{Cat}}
                      \\
                                                    \smalltfield{anaph}{\textit{Anaph}}
                                                    \\
                                                    \smalltfield{cnt}{$^\sigma$\textit{Quant}}}}}
                                              \\
\hspace*{-3em}(\smallrecord{\smalltfield{s-event}{\smallrecord{\smallmfield{phon}{$r_1$.s-event.phon$^\frown$``or''$^\frown
        r_2$.s-event.phon}{\textit{Str}}}} \\

              \smalltfield{synsem}{\smallrecord{\smallmfield{cat}{T}{\textit{Cat}}
                  \\
                                                \smallmfield{finite}{$r_1$.synsem.finite}{\textit{Bool}} \\
                                                \smallmfield{anaph}{$r_1$.synsem.anaph$\sqcup
                                                  r_2$.synsem.anaph}{\textit{Anaph}}
                                                \\
                                                \smallmfield{cnt}{$\lambda
                                  \sigma$:\textit{Seq}$_{\mathit{Ind}}$($\lambda
                                  p$:\textit{Prop}(\smallrecord{\smalltfield{c}{$r_1$.synsem.cnt($\sigma$)($p$)$\vee
                                   r_2$.synsem.cnt($\sigma$)($p$)}}))}{$^\sigma$\textit{Quant}}}}})($a$)($b$)
                                          \\
$\in P_{\mathrm{T}}$  
}

\item (corresponding to S14 and T14 in PTQ) \\
If $T\in P_{\mathrm{t}}$,
  $a:T$ and \smallrecord{\field{quant}{$q$} \\
                         \field{index}{$i$}}$\in
                       a$.synsem.anaph.qstore, then \\
$\lambda
r$:\smallrecord{\smalltfield{s-event}{\textit{SEvent}}
  \\
                  \smalltfield{synsem}{\smallrecord{\smallmfield{cat}{t}{\textit{Cat}}
                      \\
                                                    \smalltfield{finite}{\textit{Bool}} \\
                                                    \smalltfield{anaph}{\smallrecord{\smalltfield{local}{\{\textit{Nat}\}} \\
                                                                                     \smalltfield{non-local}{\{\textit{Nat}\}}
                                                                       \\
                                                                                     \smalltfield{qstore}{\{\textit{BO}\}}}}
                                                                    \\
                                                    \smalltfield{cnt}{$^\sigma$\textit{RecType}}}}}
                                              \\
\hspace*{2em}(\smallrecord{\smallmfield{s-event}{$r$.s-event}{\textit{SEvent}}
  \\
                  \smalltfield{synsem}{\smallrecord{\smallmfield{cat}{t}{\textit{Cat}}
                      \\
                                                    \smallmfield{finite}{$r$.synsem.finite}{\textit{Bool}} \\
                                                    \smalltfield{anaph}{\smallrecord{\smallmfield{local}{$r$.synsem.anaph.local}{\{\textit{Nat}\}} \\
                                                                                     \smallmfield{non-local}{$r$.synsem.anaph.non-local}{\{\textit{Nat}\}}
                                                                       \\
                                                                                     \smallmfield{qstore}{$r$.synsem.anaph.qstore$-$\{\smallrecord{\field{quant}{$q$} \\
                         \field{index}{$i$}}\}}{\{\textit{BO}\}}}}
                                                                    \\
                                                    \smallmfield{cnt}{$\lambda
                                  \sigma$:\textit{Seq}$_{\mathit{Ind}}$(q($\lambda
                                r$:\smallrecord{\smalltfield{x}{\textit{Ind}}}($r$.synsem.cnt($\sigma[r.\mathrm{x}/i]$))))}{$^\sigma$\textit{RecType}}}}})($a$)\\
$\in P_{\mathrm{t}}$
 
\end{enumerate} 

\subsubsection{Examples of phrasal types}
\label{pg:phrasaltypes}
In order to illustrate how to derive phrasal types using this
apparatus we will show a detailed derivation of a phrasal type
corresponding to the phonological string \textit{a woman runs}.  We
will compare this with a similar phrasal type derived for \textit{a
  price rises}.

Corresponding to the indefinite article \textit{a}, we have the
following type provided by the lexicon 
(p. \pageref{pg:lexicon-indefart}) and case (a) of rule
R\ref{rule:lexphrases} (p. \pageref{rule:lexphrases})
\begin{quote}
\smallrecord{\smalltfield{s-event}{\smallrecord{\smallmfield{phon}{``a''}{\textit{Str}}\\
                                   \smalltfield{s-time}{\smallrecord{\smalltfield{start}{\textit{Time}}
                                     \\
                                                                     \smalltfield{end}{\textit{Time}}}}\\
                                   \smalltfield{utt$_\mathrm{at}$}{$\langle
                                   \lambda v_1:$\textit{Str}$(\lambda
                                   v_2:$\textit{Time}($\lambda
                                   v_3:$\textit{Time}(uttered\_at($v_1$,$v_2$,$v_3$)))), \\
                                   \hspace*{4em}$\langle$s-event.phon,
                                   s-event.s-time.start, s-event.s-time.end$\rangle\rangle$}}}
                               \\
              \smalltfield{synsem}{\smallrecord{\smallmfield{cat}{T/CN}{\textit{Cat}}\\
                                                \smallmfield{finite}{0}{\textit{Bool}} \\
                                                \smalltfield{anaph}{\textit{AnaphEmpty}} \\
                                                \smallmfield{cnt}{$\lambda
                                  \sigma$:\textit{Seq}$_{\mathit{Ind}}$
                                  \\ \hspace*{3em}($\lambda v_1$:\textit{Prop} \\
\hspace*{4em}($\lambda v_2$:\textit{Prop} \\
\hspace*{5em}(\record{\tfield{par}{\smallrecord{\smalltfield{x}{\textit{Ind}}}}
  \\
                      \tfield{restr}{$\langle\lambda v$:\smallrecord{\smalltfield{x}{\textit{Ind}}}($v_1(v)$),
                        $\langle\mathrm{par}\rangle\rangle$} \\
                      \tfield{scope}{$\langle\lambda v$:\smallrecord{\smalltfield{x}{\textit{Ind}}}($v_2(v)$), $\langle\mathrm{par}\rangle\rangle$}})))
}{$^\sigma$(\textit{Prop}$\rightarrow$\textit{Quant})}}}}

\end{quote}

Corresponding to the common noun \textit{woman} we have the following
type provided by the lexicon (p. \pageref{pg:lexicon-cn}) and case (a) of rule
R\ref{rule:lexphrases}
\begin{quote}
\smallrecord{\smalltfield{s-event}{\smallrecord{\smallmfield{phon}{``woman''}{\textit{Str}}\\
                                                \smalltfield{s-time}{\smallrecord{\smalltfield{start}{\textit{Time}}
                                     \\
                                                                                  \smalltfield{end}{\textit{Time}}}}\\
                                                                                  \smalltfield{utt$_\mathrm{at}$}{$\langle
                                                                                               \lambda
                                                                                               v_1:$\textit{Str}$(\lambda
                                                                                               v_2:$\textit{Time}($\lambda
                                                                                               v_3:$\textit{Time}(uttered\_at($v_1$,$v_2$,$v_3$)))),
                                                                       \\
                                                                                               \hspace*{4em}$\langle$s-event.phon,
                                                                                               s-event.s-time.start,
                                                                                               s-event.s-time.end$\rangle\rangle$}}}
                               \\
            \smalltfield{synsem}{\smallrecord{\smallmfield{cat}{CN}{\textit{Cat}}\\
                                              \smalltfield{agr}{\textit{FemSg3rd}}
                                 \\
                                             \smallmfield{finite}{0}{\textit{Bool}} \\
                                                              \smalltfield{anaph}{\textit{AnaphEmpty}} \\
                                              \smallmfield{cnt}{$\lambda
                                                                  \sigma$:\textit{Seq}$_{\mathit{Ind}}$
                                                                  \\ \hspace*{3em}($\lambda v$:\smallrecord{\smalltfield{x}{\textit{Ind}}} \\
                                                                   \hspace*{4em}(\smallrecord{\smalltfield{c$_{\textrm{``}\mathrm{woman}\textrm{"}}$}{$\textrm{``woman"}'$($v$.x)}}))}{$^\sigma$\textit{Prop}}}}}

\end{quote}

Suppose we have two objects \textbf{\textit{a}} and
\textbf{\textit{woman}} which are respectively of these types. Then by
rule R\ref{rule:detcn} on p. \pageref{rule:detcn} we can conclude that
the following is a phrasal type:
\begin{quote}
\hspace*{-2em}\smallrecord{\smalltfield{s-event}{\smallrecord{\smallmfield{phon}{``a
      woman''}{\textit{Str}} \\ 
                                                              \smalltfield{s-time}{\smallrecord{\smallmfield{start}{\textbf{\textit{a}}.s-event.s-time.start}{\textit{Time}}
                                                                \\
                                                                                                \smallmfield{end}{\textbf{\textit{woman}}.s-event.s-time.end}{\textit{Time}}}}
                                                                                          \\
\smalltfield{utt$_\mathrm{at}$}{$\langle
                                                                                               \lambda
                                                                                               v_1:$\textit{Str}$(\lambda
                                                                                               v_2:$\textit{Time}($\lambda
                                                                                               v_3:$\textit{Time}(uttered\_at($v_1$,$v_2$,$v_3$)))),
                                                                       \\
                                                                                               \hspace*{4em}$\langle$s-event.phon,
                                                                                               s-event.s-time.start,
                                                                                               s-event.s-time.end$\rangle\rangle$}}}
                                                                     \\
                           \smalltfield{synsem}{\smallrecord{\smallmfield{cat}{T}{\textit{Cat}}
                               \\
                                                             \smalltfield{agr}{\textit{FemSg3rd}}
                                                             \\
                                                             \smallmfield{finite}{0}{\textit{Bool}} \\
                                                             \smalltfield{anaph}{\textit{AnaphEmpty}} \\
                                                             \smallmfield{cnt}{$\lambda
                                                                  \sigma$:\textit{Seq}$_{\mathit{Ind}}$ \\
\hspace*{4em}($\lambda v_2$:\textit{Prop} \\
\hspace*{5em}(\record{\tfield{par}{\smallrecord{\smalltfield{x}{\textit{Ind}}}}
  \\
                      \tfield{restr}{$\langle\lambda v$:\smallrecord{\smalltfield{x}{\textit{Ind}}}(\smallrecord{\smalltfield{c$_{\textrm{``}\mathrm{woman}\textrm{"}}$}{$\textrm{``woman"}'$($v$.x)}}),
                        $\langle\mathrm{par}\rangle\rangle$} \\
                      \tfield{scope}{$\langle\lambda v$:\smallrecord{\smalltfield{x}{\textit{Ind}}}($v_2(v)$), $\langle\mathrm{par}\rangle\rangle$}}))}{$^\sigma$\textit{Quant}}}}}
\end{quote}

The idea is that an agent can infer the existence of a such a type on
the basis of observing two speech events with phonology ``a'' and
``woman'' respectively which are sufficiently adjacent in time.  If
the utterance of ``a'' starts at $t_1$ and the utterance of ``woman''
ends at $t_2$ then we can conclude that there is a speech event (an
utterance of ``a woman'') starting at $t_1$ and ending at $t_2$ and
that this speech event represents a linguistic object of this type.
Thus parsing can be construed as the perception of events as being of
certain linguistic types and cognitive processing which leads us from
types of simple objects to types of complex objects.

Corresponding to the verb \textit{runs} we have the following
type provided by the lexicon (p. \pageref{pg:lexicon-iv}) and case (c) of rule
R\ref{rule:lexphrases}
\begin{quote}
\smallrecord{\smalltfield{s-event}{\smallrecord{\smallmfield{phon}{``runs''}{\textit{Str}}\\
                                                \smalltfield{s-time}{\smallrecord{\smalltfield{start}{\textit{Time}}
                                     \\
                                                                                  \smalltfield{end}{\textit{Time}}}}\\
                                                                                  \smalltfield{utt$_\mathrm{at}$}{$\langle
                                                                                               \lambda
                                                                                               v_1:$\textit{Str}$(\lambda
                                                                                               v_2:$\textit{Time}($\lambda
                                                                                               v_3:$\textit{Time}(uttered\_at($v_1$,$v_2$,$v_3$)))),
                                                                       \\
                                                                                               \hspace*{4em}$\langle$s-event.phon,
                                                                                               s-event.s-time.start,
                                                                                               s-event.s-time.end$\rangle\rangle$}}}
                               \\
            \smalltfield{synsem}{\smallrecord{\smallmfield{cat}{IV}{\textit{Cat}}\\
                                              \smalltfield{agr}{\textit{Sg3rd}}
                                 \\
                                             \smallmfield{finite}{1}{\textit{Bool}} \\
                                                              \smalltfield{anaph}{\textit{AnaphEmpty}} \\
                                              \smallmfield{cnt}{$\langle\lambda
                                                t$:\textit{Time}\textbf{\{}
                                                \\ \hspace*{4em}$\lambda\sigma$:\textit{Seq}$_{\textit{Ind}}$ \\
\hspace*{5em}($\lambda
r$:\smallrecord{\smalltfield{x}{\textit{Ind}}} \\ 
\hspace*{6em} (\smallrecord{\smalltfield{e-time}{\textit{Time}}
                                                                     \\
                            \smalltfield{tns}{$\langle\lambda
                              v$:\textit{Time} 
(eq($v$,$t$)),$\langle$e-time$\rangle\rangle$} \\
                            \smalltfield{c$_{\textrm{``}\mathrm{run}\textrm{"}}$}{$\langle\lambda
  v$:\textit{Time}(``run"$'$($r$.x,$v$)),
  $\langle$e-time$\rangle\rangle$}})) \\
\hspace*{4em}\textbf{\}}, \\
\hspace*{6em}$\langle$s-event.s-time.start$\rangle\rangle$)}{$^\sigma$\textit{Prop}}}}}

\end{quote}


The treatment of the English simple present tense here, which requires
that the event-time is identical to the start of the speech time is,
like Montague's original analysis, woefully inadequate.  We are not
concerned here with getting an accurate rendition of English tenses.
The event time should at least be treated as an interval with a start
and end time in the same
manner as the speech time.  This would, for example, enable us to
capture the sports-reporter use of the simple present by requiring that
the event interval be included in the speech interval.

Suppose we have two objects \textbf{\textit{a woman}} and
\textbf{\textit{runs}} which are respectively of the type we derived
corresponding to ``a woman'' and this type. Then by
rule R\ref{rule:tiv} on p. \pageref{rule:tiv} we can conclude that
the following is a phrasal type:
\begin{quote}
\hspace*{-4em}\smallrecord{\smalltfield{s-event}{\smallrecord{\smallmfield{phon}{``a
      woman runs''}{\textit{Str}} \\ 
                                                              \smalltfield{s-time}{\smallrecord{\smallmfield{start}{\textbf{\textit{a
                                                                      woman}}.s-event.s-time.start}{\textit{Time}}
                                                                \\
                                                                                                \smallmfield{end}{\textbf{\textit{runs}}.s-event.s-time.end}{\textit{Time}}}}
                                                                                          \\
\smalltfield{utt$_\mathrm{at}$}{$\langle
                                                                                               \lambda
                                                                                               v_1:$\textit{Str}$(\lambda
                                                                                               v_2:$\textit{Time}($\lambda
                                                                                               v_3:$\textit{Time}(uttered\_at($v_1$,$v_2$,$v_3$)))),
                                                                       \\
                                                                                               \hspace*{4em}$\langle$s-event.phon,
                                                                                               s-event.s-time.start,
                                                                                               s-event.s-time.end$\rangle\rangle$}}}
                                                                     \\
                           \smalltfield{synsem}{\smallrecord{\smallmfield{cat}{t}{\textit{Cat}}
                               \\
                                                             \smallmfield{finite}{1}{\textit{Bool}} \\
                                                             \smalltfield{anaph}{\textit{AnaphEmpty}} \\
                                                             \smallmfield{cnt}{$\langle\lambda
                                                t$:\textit{Time}\textbf{\{}
                                                \\ \hspace*{4em}$\lambda\sigma$:\textit{Seq}$_{\textit{Ind}}$ \\
\hspace*{5em} (\record{\tfield{par}{\smallrecord{\smalltfield{x}{\textit{Ind}}}}
  \\
        \tfield{restr}{$\langle\lambda v$:\smallrecord{\smalltfield{x}{\textit{Ind}}}(\smallrecord{\smalltfield{c$_{\textrm{``}\mathrm{woman}\textrm{"}}$}{$\textrm{``woman"}'$($v$.x)}}),
                        $\langle\mathrm{par}\rangle\rangle$} \\
                      \tfield{scope}{$\langle\lambda v_1$:\smallrecord{\smalltfield{x}{\textit{Ind}}}(\smallrecord{\smalltfield{e-time}{\textit{Time}}
                                                                     \\
                            \smalltfield{tns}{$\langle\lambda
                              v$:\textit{Time} 
(eq($v$,$t$)),$\langle$e-time$\rangle\rangle$} \\
                            \smalltfield{c$_{\textrm{``}\mathrm{run}\textrm{"}}$}{$\langle\lambda
  v$:\textit{Time}(``run"$'$($v_1$.x,$v$)),
  $\langle$e-time$\rangle\rangle$}}), \\ & &
\hspace*{8em}$\langle$par$\rangle\rangle$}}) \\
\hspace*{5em}\textbf{\}}, \\ \hspace*{8em}$\langle$s-event.s-time.start$\rangle\rangle$}{$^\sigma$\textit{RecType}}}}}

\end{quote}


A similar derivation leads to the following phrasal type corresponding
to ``a price rises''.
\begin{quote}
\hspace*{-9em}\smallrecord{\smalltfield{s-event}{\smallrecord{\smallmfield{phon}{``a
     price rises''}{\textit{Str}} \\ 
                                                              \smalltfield{s-time}{\smallrecord{\smallmfield{start}{\textbf{\textit{a
                                                                      price}}.s-event.s-time.start}{\textit{Time}}
                                                                \\
                                                                                                \smallmfield{end}{\textbf{\textit{rises}}.s-event.s-time.end}{\textit{Time}}}}
                                                                                          \\
\smalltfield{utt$_\mathrm{at}$}{$\langle
                                                                                               \lambda
                                                                                               v_1:$\textit{Str}$(\lambda
                                                                                               v_2:$\textit{Time}($\lambda
                                                                                               v_3:$\textit{Time}(uttered\_at($v_1$,$v_2$,$v_3$)))),
                                                                       \\
                                                                                               \hspace*{4em}$\langle$s-event.phon,
                                                                                               s-event.s-time.start,
                                                                                               s-event.s-time.end$\rangle\rangle$}}}
                                                                     \\
                           \smalltfield{synsem}{\smallrecord{\smallmfield{cat}{t}{\textit{Cat}}
                               \\
                                                             \smallmfield{finite}{1}{\textit{Bool}} \\
                                                             \smalltfield{anaph}{\textit{AnaphEmpty}}
                                                             \\
\smallmfield{cnt}{$\langle\lambda
                                                t$:\textit{Time}\textbf{\{}
                                                \\ \hspace*{4em}$\lambda\sigma$:\textit{Seq}$_{\textit{Ind}}$ \\
\hspace*{5em} (\record{\tfield{par}{\smallrecord{\smalltfield{x}{\textit{Ind}}}}
  \\
        \tfield{restr}{$\langle\lambda v$:\smallrecord{\smalltfield{x}{\textit{Ind}}}(\smallrecord{\smalltfield{c$_{\textrm{``}\mathrm{price}\textrm{"}}$}{$\textrm{``price"}'$($v$)}}),
                        $\langle\mathrm{par}\rangle\rangle$} \\
                      \tfield{scope}{$\langle\lambda v_1$:\smallrecord{\smalltfield{x}{\textit{Ind}}}(\smallrecord{\smalltfield{e-time}{\textit{Time}}
                                                                     \\
                            \smalltfield{tns}{$\langle\lambda
                              v$:\textit{Time} 
(eq($v$,$t$)),$\langle$e-time$\rangle\rangle$} \\
                            \smalltfield{c$_{\textrm{``}\mathrm{rise}\textrm{"}}$}{$\langle\lambda
  v$:\textit{Time}(``rise"$'$($v_1$,$v$)),
  $\langle$e-time$\rangle\rangle$}}), \\ & &
\hspace*{8em}$\langle$par$\rangle\rangle$}}) \\
\hspace*{5em}\textbf{\}}, \\ \hspace*{8em}$\langle$s-event.s-time.start$\rangle\rangle$}{$^\sigma$\textit{RecType}}
                                                             }}}
\end{quote}


In this type the predicates corresponding to ``price'' and ``rises''
have a frame as an argument where those corresponding to ``woman'' and
``runs'' have an individual thus preventing the offending inference
concerning price discussed on p. \pageref{pg:temppuzzle}.

\subsubsection{Type-theoretic inferences related to frames}  
Suppose that \textbf{\textit{apr}} is an object of the above type
corresponding to ``a price rises''.  Then \textbf{\textit{apr}}.cnt is
a vacuous function from sequences of individuals to a record type --
vacuous because the utterance does not contain any free occurrences of
pronouns.  If $\sigma$ is an arbitrary sequence of individuals then
\textbf{\textit{apr}}.cnt($\sigma$) is the record type
\begin{quote}
\record{\tfield{par}{\smallrecord{\smalltfield{x}{\textit{Ind}}}}
  \\
        \tfield{restr}{$\langle\lambda v$:\smallrecord{\smalltfield{x}{\textit{Ind}}}(\smallrecord{\smalltfield{c$_{\textrm{``}\mathrm{price}\textrm{"}}$}{$\textrm{``price"}'$($v$)}}),
                        $\langle\mathrm{par}\rangle\rangle$} \\
                      \tfield{scope}{$\langle\lambda v_1$:\smallrecord{\smalltfield{x}{\textit{Ind}}}(\smallrecord{\smalltfield{e-time}{\textit{Time}}
                                                                     \\
                            \smalltfield{tns}{$\langle\lambda
                              v$:\textit{Time} 
(eq($v$,$t$)),$\langle$e-time$\rangle\rangle$} \\
                            \smalltfield{c$_{\textrm{``}\mathrm{rise}\textrm{"}}$}{$\langle\lambda
  v$:\textit{Time}(``rise"$'$($v_1$,$v$)),
  $\langle$e-time$\rangle\rangle$}}), \\ & &
\hspace*{8em}$\langle$par$\rangle\rangle$}}
\end{quote} 
where $t$ is the start-time of the utterance.  This type represents
the ``proposition expressed by the utterance \textbf{\textit{apr}}''.
What it would mean for the utterance to be true is that this type is
non-empty.  An object of this type is a record containing a frame, in the sense we
introduce on p. \pageref{pg:frame}ff, in its par field. This follows
because of the constraint in the restr field which requires that there
be a proof that the predicate ``price''$'$ hold of the object in the
par field.  By the frame constraint on ``price''$'$ introduced on
p. \pageref{pg:frameconstraints} we know that this frame not only
contains an object in its x field but also contains an event-time and
a commodity.  The object in the x field is constrained to be money and
also to be the price of the commodity at the event-time.  The scope
field of \textbf{\textit{apr}}.cnt($\sigma$) requires that this frame
fall under the ``rise''$'$ predicate at the start time of the
utterance.  By the general predicate constraints introduced on
p. \pageref{pg:predconstraints} we know that a proof of this would be
a pair of frames consisting of the original frame and a frame with the
same commodity but a later time and a higher price.

While the lexical analysis represented by this is rudimentary to say
the least (one feels, for example, that the event time for rising
should be an interval and that rising perhaps involves something more
sophisticated than just the comparison of price at just two time
points), the inclusion of frames does show a potential for including
the kind of account of lexical meaning that many people thought was
missing from classical work on model-theoretic semantics for natural
language.  


\section{Scrap?}

We will define lexical functions which tell us how to construct a type
for a lexical item on the basis of a phonological type and either an
object or a type corresponding to an observation of the world.  The
idea is that an agent which is constructing a grammar for use in a
particular communicative situation will construct lexical types on the
basis of a coordinated pair of observations: an observation of a
speech event and an observation of an object or event with which the
speech event is associated.  This is related to the idea from
situation semantics that meaning
is a relation between an utterance situation and a described situation \cite{BarwisePerry1983}.
The use of types here relates to the idea
of type judgements as being involved in perception as discussed in
section~\ref{sec:basic}.  

We shall use the following notation:
\begin{quote}
If $W$ is a phonological type, then c$_W$ is a distinguished label associated
with $W$, such that if $W_1\not=W_2$ then c$_{W_1}$$\not=$c$_{W_2}$.

% If $W$ is a phonological type then $W'_{\mathit{Arity}}$ is a predicate of arity
% $\mathit{Arity}$.  When $\mathit{Arity}$ is clear from the context we
% suppress it.

% If $W$ is a phonological type then $W'_T$ is an object of type $T$.  When it is
% clear from the context we often suppress $T$.

% necessarily$'_{\langle\mathit{RecType}\rangle}$ is to be the predicate
% `nec'.

\end{quote}

We shall also make use of singleton types.  \textbf{TYPE}$_C$ = $\langle${\bf Type}, {\bf BType},
$\langle$\textbf{PType}, {\bf Pred}, \textbf{ArgIndices}, {\it
  Arity\/}$\rangle$, $\langle A,F\rangle$$\rangle$ \textit{has singleton types} if
\begin{enumerate} 
 
\item for any $T \in \mathbf{Type}$ and $b:_{\mathbf{TYPE_C}}T$, $T_b \in \mathbf{Type}$ 
 
\item for any $T \in \mathbf{Type}$, 
$a:_{\mathbf{TYPE_C}}T_b$ iff
  $a:_{\mathbf{TYPE_C}}T$ and $a=b$
   
 
\end{enumerate}
\noindent In the case of a singleton type $T_x$ we allow a variant
notation in records (corresponding to the manifest fields of Coquand
et al., 2004) using
\begin{display}
\record{\mfield{$\ell$}{$x$}{$T$}}
\end{display}
\noindent for
\begin{display}
\record{\tfield{$\ell$}{$T_x$}}
\end{display}  

When we have a field
\begin{quote}
\record{\tfield{$\ell$}{$\langle\lambda v_1:T_1\ldots\lambda
    v_n:T_n(T_x), \langle\pi_1\ldots\pi_n\rangle\rangle$}}
\end{quote}
we allow for convenience notations such as
\begin{quote}
\record{\mfield{$\ell$}{$\langle\lambda v_1:T_1\ldots\lambda
    v_n:T_n\textbf{\{}x\textbf{\}}, \langle\pi_1\ldots\pi_n\rangle\rangle$}{$T$}}

\record{\mfield{$\ell$}{$x$}{$\langle\lambda v_1:T_1\ldots\lambda
    v_n:T_n(T), \langle\pi_1\ldots\pi_n\rangle\rangle$}}
\end{quote}
or
\begin{quote}
\hspace*{-1em}\smallrecord{\smallmfield{$\ell$}{$\langle\lambda v_1:T_1\ldots\lambda
    v_n:T_n\textbf{\{}x\textbf{\}}, \langle\pi_1\ldots\pi_n\rangle\rangle$}{$\langle\lambda v_1:T_1\ldots\lambda
    v_n:T_n(T), \langle\pi_1\ldots\pi_n\rangle\rangle$}}
\end{quote}
depending on how $T_x$ depends on $\pi_1\ldots\pi_n$.  We use
\textbf{\{} and \textbf{\}} to delimit $x$ since $x$ itself may be a
function thus leading to ambiguity in the notation if we do not
distinguish which $\lambda$'s represent dependency and which belong to
the resulting object.  Note that this ambiguity only arises in the
notation we are adopting for convenience.

\subsubsection{Proper names}

The most straightforward view of proper names is that they are based
on pairings of proper noun utterances and individuals.  While the full
story about proper names may have to be more complex, this will
suffice for our present purposes.

We define a function lex$_{\mathrm{n}_\mathrm{Prop}}$  which maps
 phonological types corresponding to proper names like
\textit{Sam} and individuals to record types, such that if $W$ is
a phonological type such as ``Sam'' or ``John'' and $a$:\textit{Ind},
lex$_{\mathrm{n}_\mathrm{Prop}}$($W,a$) is
\begin{quote}
\textit{Sign} \d{$\wedge$} \\
\record{\tfield{s-event}{\record{\tfield{phon}{\textit{W}}}} \\
        \tfield{synsem}{\record{\mfield{cat}{n$_{\mathrm{Prop}}$}{\textit{Cat}} \\
                                \mfield{cnt}{$\lambda
                                  v$:\textit{Ppty}($v$(\smallrecord{\field{x}{$a$}}))}{\textit{Quant}}}}}
\end{quote}

\noindent The idea of this function is that an agent could have it as a resource
to construct a lexical item for a local language on observing a
pairing of a particular type of utterance (e.g. utterances of
\textit{Sam}) and a particular individual. If the language we are building is small enough there will
be only one individual associated with a given phonological type such
as ``sam'' but it is easy to imagine situations where there will be a
need to have different individuals associated with the same name even
within a local language, for example, if you need to talk about two
people named \textit{Sam} who write a book together.  While this
creates potential for misunderstanding there is nothing technically
mysterious about having two lexical types which happen to share the
same phonology.  This is in contrast to the classical formal semantics
view of proper names as related to logical constants where it seems
unexpected that proper nouns should be able to refer to different
individuals on different uses. 

An example of a set of basic proper names which could be generated
with these resources given two individuals $a$ and $b$ (that is,
$a,b$:\textit{Ind}) would be
\begin{quote}
\begin{tabbing}
\{\=lex$_{\mathrm{n}_\mathrm{Prop}}$(``Sam'',$a$), \\
\>lex$_{\mathrm{n}_\mathrm{Prop}}$(``John'',$b$)\}
\end{tabbing}

\end{quote}



\subsubsection{Intransitive verbs}\label{pg:lexicon-iv}

For intransitive verbs we will take the paired observations to involve
a phonological type corresponding to an intransitive verb on the one
hand and a predicate on the other.  Philosophically, it may appear
harder to explain what it means to observe a predicate compared to
observing an individual, even though if you dig deep enough even
individuals are problematical.  However, it seems that any reasonable
theory of perception should account for the fact that we perceive the
world in terms of various kinds of objects standing in relations to
each other.  Our predicates correspond to these relations and we would
want to say that our cognitive apparatus is such that relations are
reified in a way that they need to be in order to become associated
with types of utterances.  For a verb like \textit{run} we will say
that the predicate is one that holds between individuals and time
intervals.  We will argue in section~\ref{sec:partee-puzzle} that for
other verbs we need frames instead of individuals.

We define a function lex$_{\mathrm{V_{\mathrm{i}}}}$ which maps phonological types corresponding to intransitive verbs like
\textit{run} and predicates with arity
$\langle$\textit{Ind},\textit{TimeInt}$\rangle$, such that if $W$ is
a phonological type like ``run'' or ``walk'' and $p$ is a predicate with arity $\langle$\textit{Ind},\textit{TimeInt}$\rangle$,
lex$_{\mathrm{V_{\mathrm{i}}}}$($W,p$) is
\begin{quote}
\hspace*{-2em}\textit{Sign} \d{$\wedge$} \\
\hspace*{-2em}\smallrecord{\smalltfield{s-event}{\smallrecord{\smalltfield{phon}{\textit{$W$}}}} \\
        \smalltfield{synsem}{\smallrecord{\smallmfield{cat}{v$_{\mathrm{i}}$}{\textit{Cat}} \\
                                \smallmfield{cnt}{$\lambda
                                  r$:\smallrecord{\smalltfield{x}{\textit{Ind}}}
                                  (\smallrecord{\smalltfield{e-time}{\textit{TimeInt}}
                                    \\
                                           \smalltfield{c$_W$}{$\langle\lambda
                                             v$:\textit{TimeInt}($p$($r$.x,$v$), $\langle$e-time$\rangle\rangle$}}))}{\textit{Ppty}}}}}

\end{quote}

Similar remarks hold for this function as for the one we used for
proper names.  For different local languages different predicates may
be associated with utterances of \textit{run} and even within the same
local language, confusing though it may be, we may need to associate
different predicates with different occurrences of \textit{run}.  In
this way verbs are like proper names and one can think of verbs as
proper names of predicates.

However, this is not quite enough if we want to handle different forms
of verbs such as infinitives, and present and past tenses.  For purposes
of simplification as our concern is not with the details of
morphological types we will assume that all finite verb occurrences are third person
singular and will not represent these features.  In order to
achieve this we need to define lex$_{\mathrm{V_{\mathrm{i}}}}$ not in
terms of a single phonological type but a paradigm of phonological
types corresponding to different configurations of morphological
features.  For present purposes we will think of there just being one
morphological feature of tense which can take the values: inf
(``infinitive''), pres (``present tense''), past (``past tense'').  We
will think of paradigms as functions which map records of type
\smallrecord{\smalltfield{tns}{\textit{Tns}}} to phonological types.
  Here the type \textit{Tns} has elements inf, pres and past.  Let
  \textbf{\textsf{run}} be the paradigm for \textit{run}. The function
  is defined by
\begin{quote}
\textbf{\textsf{run}}(\smallrecord{\field{tns}{inf}})= ``run'' \\
\textbf{\textsf{run}}(\smallrecord{\field{tns}{pres}})= ``runs'' \\
\textbf{\textsf{run}}(\smallrecord{\field{tns}{past}})= ``ran''
\end{quote}
and for \textit{walk} we have
\begin{quote}
\textbf{\textsf{walk}}(\smallrecord{\field{tns}{inf}})= ``walk'' \\
\textbf{\textsf{walk}}(\smallrecord{\field{tns}{pres}})= ``walks'' \\
\textbf{\textsf{walk}}(\smallrecord{\field{tns}{past}})= ``walked''
\end{quote}

In order to obtain the interpretations of the tensed forms of the verb
we will need the following functions for present and past tense.
\begin{description}

\item[Pres] which is to be $\lambda
  t$:\textit{TimeInt}(\smallrecord{\smalltfield{e-time}{\textit{TimeInt}}
    \\
                                \smalltfield{tns}{$\langle\lambda
                                  v$:\textit{TimeInt}($v=t$), $\langle$e-time$\rangle\rangle$}})

\item[Past] which is to be $\lambda
  t$:\textit{TimeInt}(\smallrecord{\smalltfield{e-time}{\textit{TimeInt}}
    \\
                                \smalltfield{tns}{$\langle\lambda
                                  v$:\textit{TimeInt}($v$.end$<t$.start),$\langle$e-time$\rangle\rangle$}})

\end{description}
\noindent The present tense function expresses that the event time is
identical with the interval to which it is being compared. This is normally
the speech time as in the grammar defined here, though it could also
be a different time interval, for example in the interpretation of
historic presents.  The past tense function expresses that the end of
the event time interval has to be prior to the start of the interval
(e.g. the speech time) with which it is being compared.

We need also to make the distinction between finite and non-finite
verb utterances and we will do this by introducing a field labelled
`fin' which will take values in the type \textit{Bool} (``boolean'')
whose members are 0 and 1.

Now we redefine lex$_{\mathrm{V_{\mathrm{i}}}}$ to be a function which
takes a paradigm $\mathcal{W}$ such as \textbf{\textsf{run}} or \textbf{\textsf{walk}}, a
predicate $p$ with arity
$\langle$\textit{Ind},\textit{TimeInt}$\rangle$ and morphological
record $m$ of type \smallrecord{\smalltfield{tns}{\textit{Tns}}} such
that 

\begin{enumerate} 
 
\item if $m$ is \smallrecord{\field{tns}{inf}},
  lex$_{\mathrm{V_{\mathrm{i}}}}(\mathcal{W},p,m)$ is 

\hspace*{-2em}\textit{Sign} \d{$\wedge$} \\
\hspace*{-2em}\smallrecord{\smalltfield{s-event}{\smallrecord{\smalltfield{phon}{$\mathcal{W}(m)$}}} \\
        \smalltfield{synsem}{\smallrecord{\smallmfield{cat}{v$_{\mathrm{i}}$}{\textit{Cat}}
            \\
\smallmfield{fin}{0}{\textit{Bool}} \\
                                \smallmfield{cnt}{$\lambda
                                  r$:\smallrecord{\smalltfield{x}{\textit{Ind}}}
                                  (\smallrecord{\smalltfield{e-time}{\textit{TimeInt}}
                                    \\
                                           \smalltfield{c$_{\mathcal{W}(m)}$}{$\langle\lambda
                                             v$:\textit{TimeInt}($p$($r$.x,$v$)), $\langle$e-time$\rangle\rangle$}})}{\textit{Ppty}}}}}

 
\item  if $m$ is \smallrecord{\field{tns}{pres}},
  lex$_{\mathrm{V_{\mathrm{i}}}}(\mathcal{W},p,m)$ is 

\hspace*{-2em}\textit{Sign} \d{$\wedge$} \\
\hspace*{-2em}\smallrecord{\smalltfield{s-event}{\smallrecord{\smalltfield{phon}{$\mathcal{W}(m)$}
    \\
\smalltfield{s-time}{\textit{TimeInt}}}} \\
        \smalltfield{synsem}{\smallrecord{\smallmfield{cat}{v$_{\mathrm{i}}$}{\textit{Cat}}
            \\
\smallmfield{fin}{1}{\textit{Bool}} \\
                                \smallmfield{cnt}{$\langle\lambda v_1$:\textit{Time}\textbf{\{}$\lambda
                                  r$:\smallrecord{\smalltfield{x}{\textit{Ind}}} \\
                                  \hspace*{4em}(\smallrecord{\smalltfield{e-time}{\textit{TimeInt}}
                                    \\
                                           \smalltfield{c$_{\mathcal{W}(m)}$}{$\langle\lambda
                                             v_2$:\textit{TimeInt}($p$($r$.x,$v_2$)),
                                             $\langle$e-time$\rangle\rangle$}}\d{$\wedge$}\textbf{Pres}($v_1$))\textbf{\}}, \\
                                       \hspace*{20em}$\langle$s-event.s-time$\rangle\rangle$}{\textit{Ppty}}}}}

\item if $m$ is \smallrecord{\field{tns}{past}},
  lex$_{\mathrm{V_{\mathrm{i}}}}(\mathcal{W},p,m)$ is 

\hspace*{-2em}\textit{Sign} \d{$\wedge$} \\
\hspace*{-2em}\smallrecord{\smalltfield{s-event}{\smallrecord{\smalltfield{phon}{$\mathcal{W}(m)$}
    \\
\smalltfield{s-time}{\textit{TimeInt}}}} \\
        \smalltfield{synsem}{\smallrecord{\smallmfield{cat}{v$_{\mathrm{i}}$}{\textit{Cat}}
            \\
\smallmfield{fin}{1}{\textit{Bool}} \\
                                \smallmfield{cnt}{$\langle\lambda v_1$:\textit{Time}\textbf{\{}$\lambda
                                  r$:\smallrecord{\smalltfield{x}{\textit{Ind}}} \\
                                  \hspace*{4em}(\smallrecord{\smalltfield{e-time}{\textit{TimeInt}}
                                    \\
                                           \smalltfield{c$_{\mathcal{W}(m)}$}{$\langle\lambda
                                             v_2$:\textit{TimeInt}($p$($r$.x,$v_2$)),
                                             $\langle$e-time$\rangle\rangle$}}\d{$\wedge$}\textbf{Past}($v_1$))\textbf{\}}, \\
                                       \hspace*{20em}$\langle$s-event.s-time$\rangle\rangle$}{\textit{Ppty}}}}}

 
\end{enumerate}

An example of a set of intransitive verbs which could be generated
with these resources given appropriate predicates `run' and `walk' is
\begin{quote}
\begin{tabbing}
$\bigcup_{\alpha\in\{\textrm{inf,pres,past}\}}$\{\=lex$_{\mathrm{V_{\mathrm{i}}}}$(\textbf{\textsf{run}},run,\smallrecord{\field{tns}{$\alpha$}}),
\\
\>
lex$_{\mathrm{V_{\mathrm{i}}}}$(\textbf{\textsf{walk}},walk,\smallrecord{\field{tns}{$\alpha$}})\}
\end{tabbing}
\end{quote} 
  

 

\subsection*{Syntactic and semantic composition}

We will think of composition rules as functions which take a string of
utterances of various types and return a type for the whole string.
That is, the basic form of our composition rules will be:
\begin{quote}
$\lambda s:T_1(T_2)$
\end{quote}
where $T_1$ is a type of strings of signs and $T_2$ is a type of
signs.  More specifically we can say that \textit{unary} rules are
functions of the form
\begin{quote}
$\lambda s:T_1(T_2)$, where $T_1,T_2\sqsubseteq$\textit{Sign}
\end{quote}
and \textit{binary} rules are of the form
\begin{quote}
$\lambda s:T_1^{\frown}T_2(T_3)$, where $T_1,T_2,T_3\sqsubseteq$\textit{Sign} 
\end{quote}
\noindent `$\sqsubseteq$' here denotes the subtype relation defined in
section~\ref{sec:modal}.  (We are suppressing the subscript used
there.)  We can, of course, generalize these notions to $n$-ary rules
but unary and binary will be sufficient for our present purposes.

Note that to say that there is a string of signs $s_1^{\frown}s_2$ does not necessarily mean that
the signs are temporally ordered in the sense that
$s_1$.s-event.s-time.end $<$ $s_2$.s-event.s-time.start.  There could
be an advantage in this for the treatment of discontinuous
constituents or free word order.  But we can also define a special
``temporal concatenation'' type for concatenation of signs:

\begin{quote}
A system of complex types \textbf{TYPE}$_C$ = $\langle${\bf Type}, {\bf BType},
$\langle$\textbf{PType}, {\bf Pred}, \textbf{ArgIndices}, {\it
  Arity\/}$\rangle$, $\langle A,F\rangle$$\rangle$ \textit{has
  temporal concatenation types for the type Sign} if 
\begin{enumerate} 
 
\item for any $T_1$, $T_2$ $\sqsubseteq$ \textit{Sign},  ${T_1}^{\frown_{\mathrm{temp}}}\!T_2$ $\in$ {\bf Type} 
 
\item $s : {T_1}^{\frown_{\mathrm{temp}}}\!T_2$ iff
  $s=s_1^\frown\!s_2$, $s_1:T_1$, $s_2:T_2$ and $s_1$.s-event.s-time.end
  $<$ $s_2$.s-event.s-time.start. 
 
\end{enumerate}

\end{quote}
\noindent 

We will factor our rules into component functions which we will then
combine in order to make a complete rule.  The components we will use
here are:
\begin{description}

\item[\textsf{unary\_sign}] which we define to be
\begin{quote}
$\lambda s$:\textit{Sign}(\textit{Sign})
\end{quote}
\noindent This takes any sign and returns the type \textit{Sign}

\item[\textsf{binary\_sign}] which we define to be
\begin{quote}
$\lambda s$:\textit{Sign}$^{\frown_{\mathrm{temp}}}$\textit{Sign}(\textit{Sign})
\end{quote}
\noindent This takes any temporal concatenation of two signs and returns the type
\textit{Sign}

\item[\textsf{phon\_id}] which we define to be
\begin{quote}
$\lambda
s$:\smallrecord{\smalltfield{s-event}{\smallrecord{\smalltfield{phon}{\textit{Phon}}}}}(\smallrecord{\smalltfield{s-event}{\smallrecord{\smallmfield{phon}{$s$.s-event.phon}{\textit{Phon}}}}})
\end{quote}
\noindent This takes any record $s$ of type
\smallrecord{\smalltfield{s-event}{\smallrecord{\smalltfield{phon}{\textit{Phon}}}}}
and returns a type which is the same except that the phonology field
is now required to be filled by the value of that field in $s$.

\item[\textsf{phon\_concat}] which we define to be
\begin{quote}
$\lambda
s$:\smallrecord{\smalltfield{s-event}{\smallrecord{\smalltfield{phon}{\textit{Phon}}}}}$^{\frown}$\smallrecord{\smalltfield{s-event}{\smallrecord{\smalltfield{phon}{\textit{Phon}}}}}
\\
\hspace*{2em}(\smallrecord{\smalltfield{s-event}{\smallrecord{\smallmfield{phon}{$s[1]$.s-event.phon$^{\frown}s[2]$.s-event.phon}{\textit{Phon}}}}})
\end{quote}
\noindent This takes a string of two records with phonology fields and
returns the type of a single record with a phonology field whose value
is required to be the concatenation of the values of the phonology
fields in the first and second elements of the string.

\item[\textsf{unary\_cat}] which we define to be
\begin{quote}
$\lambda c_1$:\textit{Cat}($\lambda c_2$:\textit{Cat}($\lambda
s$:\smallrecord{\smallmfield{cat}{$c_1$}{\textit{Cat}}}(\smallrecord{\smallmfield{cat}{$c_2$}{\textit{Cat}}})))
\end{quote}
\noindent This takes two categories and returns a function which maps
a record with a category field with value the first category to a type
of records with a category field which is required to be filled by the
second category.

\item[\textsf{binary\_cat}] which we define to be
\begin{quote}
$\lambda c_1$:\textit{Cat}($\lambda c_2$:\textit{Cat}($\lambda c_3$:\textit{Cat}($\lambda
s$:\smallrecord{\smallmfield{cat}{$c_1$}{\textit{Cat}}}$^\frown$\smallrecord{\smallmfield{cat}{$c_2$}{\textit{Cat}}}(\smallrecord{\smallmfield{cat}{$c_3$}{\textit{Cat}}})))
\end{quote}
\noindent This takes three categories and returns a function which maps
a string of two records with a category field with values identical to the respective categories to a type
of records with a category field which is required to be filled by the
third category.



\item[\textsf{cnt\_id}] which we define to be
\begin{quote}
$\lambda
s$:\smallrecord{\smalltfield{synsem}{\smallrecord{\smalltfield{cnt}{\textit{Cnt}}}}}(\smallrecord{\smalltfield{synsem}{\smallrecord{\smallmfield{cnt}{$s$.synsem.cnt}{\textit{Cnt}}}}})
\end{quote}
\noindent This takes any record $s$ of type
\smallrecord{\smalltfield{synsem}{\smallrecord{\smalltfield{cnt}{\textit{Cnt}}}}}
and returns a type which is the same except that the content field
is now required to be filled by the value of that field in $s$.

\item[\textsf{cnt\_forw\_app}] which we define to be
\begin{quote}
$\lambda T_1$:\textit{Type}($\lambda T_2$:\textit{Type}($\lambda
s$:\smallrecord{\smalltfield{synsem}{\smallrecord{\smalltfield{cnt}{$T_1\rightarrow
      T_2$}}}}$^\frown$\smallrecord{\smalltfield{synsem}{\smallrecord{\smalltfield{cnt}{$T_1$}}}}
\\
\hspace*{2em}(\smallrecord{\smalltfield{synsem}{\smallrecord{\smallmfield{cnt}{$s[1]$.synsem.cnt($s[2]$.synsem.cnt)}{$T_2$}}}})
\end{quote}
\noindent This takes any binary string of records $s$ such that the
content of the first record is a function which takes arguments of a
type to which the content of the second record belongs
and returns a type whose content field
is now required to be filled by the result of applying the content of
the first record to the content of the second record.

\item[\textsf{fin\_id}] which we define to be
\begin{quote}
$\lambda s$:\smallrecord{\smalltfield{fin}{\textit{Bool}}}(\smallrecord{\smallmfield{fin}{$s$.fin}{\textit{Bool}}})
\end{quote}
\noindent This requires that the value of a `fin'-field will be copied
into the new type (corresponding to feature percolation in a
non-branching tree in a more traditional feature-based grammar).


\item[\textsf{fin\_hd}] which we define to be
\begin{quote}
$\lambda
s$:\textit{Sign}$^\frown$\smallrecord{\smallmfield{fin}{1}{\textit{Bool}}}(\smallrecord{\smallmfield{fin}{$s$.fin}{\textit{Bool}}})
\end{quote}
\noindent This requires that the second sign in a string of two has a positive
specification for finiteness and copies it into the new type.

\end{description}

We will use the notion of merge defined in section~\ref{sec:merge} in
the characterization of how these component functions are to be
combined in order to form rules.  Since the combination of these
functions is so closely connected to the merge operation we will use
a related symbol `\d{\d{$\wedge$}}' with two dots rather than one.  In
the following definition we will use $T_i$ to represent types which
are not string types and $v$ to represent an arbitrary variable.
\begin{enumerate} 
 
\item $\lambda v$:$T_1(T_2)$ \d{\d{$\wedge$}} $\lambda v$:$T_3(T_4)$
  is to be $\lambda v$:$T_1$\d{$\wedge$}$T_3$($T_2$\d{$\wedge$}$T_4$)
 
\item $\lambda v$:$T_1^{\frown}{T_2}(T_3)$ \d{\d{$\wedge$}} $\lambda
  v$:$T_4^\frown T_5(T_6)$
  is to be $\lambda v$:$(T_1$\d{$\wedge$}$T_4)^\frown (T_2$\d{$\wedge$}$T_5$) ($T_3$\d{$\wedge$}$T_6$) 

\item $\lambda v$:$T_1^{\frown_\mathit{temp}}{T_2}(T_3)$ \d{\d{$\wedge$}} $\lambda
  v$:$T_4^\frown T_5(T_6)$
  is to be $\lambda v$:$(T_1$\d{$\wedge$}$T_4)^{\frown_\mathit{temp}} (T_2$\d{$\wedge$}$T_5$) ($T_3$\d{$\wedge$}$T_6$)  
\end{enumerate}
Since \d{\d{$\wedge$}}, like \d{$\wedge$}, is associative we will
write $f$\d{\d{$\wedge$}}$g$\d{\d{$\wedge$}}$h$ instead of
($f$\d{\d{$\wedge$}}$g$)\d{\d{$\wedge$}}$h$ or
$f$\d{\d{$\wedge$}}($g$\d{\d{$\wedge$}}$h$).

Now we can use the rule components we have defined to express the
three rules we need for this small fragment.
    

\begin{description}

\item[\textsf{S} $\rightarrow$ \textsf{NP VP}] \mbox{} \\
\textsf{binary\_sign} \d{\d{$\wedge$}} \textsf{phon\_concat} \d{\d{$\wedge$}}
\textsf{binary\_cat}(np)(vp)(s) \d{\d{$\wedge$}} \textsf{fin\_hd}\\
\hspace*{5em}\d{\d{$\wedge$}}
\textsf{cnt\_forw\_app}(\textit{Ppty})(\textit{RecType})

\item[\textsf{NP} $\rightarrow$ \textsf{N}] \mbox{} \\
\textsf{unary\_sign} \d{\d{$\wedge$}} \textsf{phon\_id} \d{\d{$\wedge$}}
\textsf{unary\_cat}(n$_{\mathrm{Prop}}$)(np) \d{\d{$\wedge$}} \textsf{cnt\_id}

\item[\textsf{VP} $\rightarrow$ \textsf{V}$_i$] \mbox{} \\
\textsf{unary\_sign} \d{\d{$\wedge$}} \textsf{phon\_id} \d{\d{$\wedge$}}
\textsf{unary\_cat}(v$_i$)(vp) \d{\d{$\wedge$}} \textsf{fin\_id} \d{\d{$\wedge$}} \textsf{cnt\_id}


\end{description}

This gives us a concise way to express rather complex functions
corresponding to simple rules.  The point of this is, however,  not
merely to
give us yet another formalism for expressing natural language phrase
structure and its interpretation but to show how such rules can be
broken down into abstract components which an agent learning the
language could combine in order to create rules which it has not
previously had available in its resources.  Thus an agent (such as a
child in the one-word stage) which does not have a rule \textsf{S}
$\rightarrow$ \textsf{NP VP} but who observes strings of linguistic
events where NP's are followed by VP's may reason its way to a rule
that combine NP-events followed by VP-events into a single event.
While this concerns linguistic events it is closely related to the way
we take strings of non-linguistic events to form single events, for example, a
going-to-bed-event for a child might normally consist of a string of
events
having-hot-milk$^\frown$putting-on-pyjamas$^\frown$getting-into-bed$^\frown$listening-to-a-story.
Our general ability to perceive events, that is, assign types to events and to combine these
types into larger event types seems to be a large part of the basis
for our linguistic ability.  We will return to this in our discussion
of Fernando's string theory of events in section~\ref{sec:events}.
