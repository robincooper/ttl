\chapter{Type-based underspecification}
\label{ch:underspec}
\setcounter{equation}{0}

\section{Introduction}

In Section~\ref{sec:qscope-underspec}, we will explore a way
of treating underspecification in terms of types of content rather
than the standard view of underspecification in terms of
underspecified representations corresponding to sets of contents.

In Section~\ref{sec:anaph}, we will extend this treatment of
underspecification to include anaphoric readings.

\section{Quantifier scope and underspecification}
\label{sec:qscope-underspec}

Given the kind of interpretation rules we have so far we can obtain a
reading for \nexteg{a} which corresponds to the 
content 
in \nexteg{b}, using the abbreviations `boy$'$' as
introduced in Chapter~\ref{ch:commonnouns}, example
(\ref{ex:dogprime}), \textbf{hug} for the parametric content of
\textit{hug} and \textbf{every}$^\frown$\textbf{dog} for the
parametric content of \textit{every dog}.
\begin{ex} 
\begin{subex} 
 
\item a boy hugged every dog 
 
\item % $\lambda \mathfrak{s}$:\textit{Rec} . \smallrecord{\smalltfield{e}{exist(boy$'$, $\lambda
%       r_1$:\smallrecord{\smalltfield{x}{\textit{Ind}}} . 
% \smallrecord{\smalltfield{e}{every(dog$'$, 
% $\lambda r_2$: \smallrecord{\smalltfield{x}{\textit{Ind}}} . 
% \smallrecord{\smalltfield{e}{hug($r_1$.x, $r_2$.x)}})}})}}
  \needspace{9\baselineskip}
  $\ulcorner\lambda c$:\smallrecord{
    \footnotesize{\textit{Cntxt}}\\
    \smalltfield{$\mathfrak{c}$}{\smallrecord{
        \smalltfield{f}{\smallrecord{
            \smalltfield{f}{\textit{PropCntxt}}\\
            \smalltfield{a}{\textit{PropCntxt}}}}\\
        \smalltfield{a}{\smallrecord{
            \smalltfield{f}{\textit{PropCntxt}}\\
            \smalltfield{a}{\smallrecord{
                \smalltfield{f}{\textit{PropCntxt}}\\
                \smalltfield{a}{\textit{PropCntxt}}}}}}}}} . \\
  \hspace*{2em}\record{
    \mfield{restr}{boy$'$}{\textit{Ppty}}\\
    \mfield{scope}{\textbf{hug}($c$)(\textbf{every}$^\frown$\textbf{dog}($c$))$\mid_{\mathcal{F}(\text{boy}')}$}{\textit{Ppty}}\\
    \tfield{e}{exist(restr, scope)}}$\urcorner$
 
\end{subex} 
   
\end{ex} 
This, of course, represents the reading where there is a boy such that
he hugs every dog.  % Notice that here we have vacuous abstraction over
% pronominal contexts and the constraint on them is that they are a record of
% some kind without requiring any particular fields.
In order to obtain the reading where \textit{every dog} has wide
scope, we follow Montague and pretty much everybody else in basing our
treatment of quantifier scope on the treatment of free pronouns,
though without the contribution of any gender information.  Let us
imagine just for a moment that such a pronoun existed in English and
is written as \textit{it$^*$} with the kind of pronoun interpretations
given in Chapter~\ref{ch:propnames}, Section~\ref{sec:unbound}.  Then the content for \nexteg{a} could be
\nexteg{b}.



\begin{ex} 
\begin{subex} 
 
\item a boy hugged it$^*$
 
\item % $\lambda \mathfrak{s}$:\smallrecord{\smalltfield{x$_0$}{\textit{Ind}}} . 
% \smallrecord{\smalltfield{e}{exist(boy$'$, $\lambda
%     r$:\smallrecord{\smalltfield{x}{\textit{Ind}}}
%     . \smallrecord{\smalltfield{e}{hug($r$.x,
%     $\mathfrak{s}$.x$_0$)}})}}
  $\lambda c$:\smallrecord{
    \footnotesize{\textit{Cntxt}}\\
    \smalltfield{$\mathfrak{s}$}{\smallrecord{
        \smalltfield{x$_0$}{\textit{Ind}}}}\\
    \smalltfield{$\mathfrak{c}$}{\smallrecord{
        \smalltfield{f}{\smallrecord{
            \smalltfield{f}{\textit{PropCntxt}}\\
            \smalltfield{a}{\textit{PropCntxt}}}}\\
        \smalltfield{a}{\smallrecord{
            \smalltfield{f}{\textit{PropCntxt}}\\
            \smalltfield{a}{\textit{PropCntxt}}}}}}} . \\
  \hspace*{1em}\record{
    \mfield{restr}{boy$'$}{\textit{Ppty}}\\
    \mfield{scope}{\textbf{hug}($c$)($\lambda P$:\textit{Ppty}
      . $P\{c.\mathfrak{s}.\text{x}_0\}$)$\mid_{\mathcal{F}(\text{boy}')}$}{\textit{Ppty}}\\
    \tfield{e}{exist(restr, scope)}}
 
\end{subex} 
 \label{ex:it-star}  
\end{ex}
We will refer to \preveg{b} as
`\textbf{a}$^\frown$\textbf{boy}$^\frown$\textbf{hugged}$^\frown$\textbf{it$^*$}'.
Let
us further imagine, contrary to fact, that English represented the fact
that a noun phrase has wide scope over a sentence by placing it at the
beginning of a sentence as in \nexteg{a} and giving it an
interpretation where the interpretation of \textit{it$^*$} gets bound
as in \nexteg{b}.
\begin{ex} 
\begin{subex} 
 
\item every dog, a boy hugged it$^*$ 
 
\item % $\lambda\mathfrak{s}$:\textit{Rec} . 
% \smallrecord{\smalltfield{e}{every(dog$'$,
% $\lambda r_1$:\smallrecord{\smalltfield{x$_0$}{\textit{Ind}}} . 
% \smallrecord{\smalltfield{e}{exist(boy$'$, $\lambda
%     r_2$:\smallrecord{\smalltfield{x}{\textit{Ind}}}
%     . \smallrecord{\smalltfield{e}{hug($r_2$.x, $r_1$.x$_0$)}})}})}}
  $\lambda c$:\smallrecord{
    \footnotesize{\textit{Cntxt}}\\
    \smalltfield{$\mathfrak{c}$}{\smallrecord{
        \smalltfield{f}{\smallrecord{
            \smalltfield{f}{\textit{PropCntxt}}\\
            \smalltfield{a}{\textit{PropCntxt}}}}\\
        \smalltfield{a}{\smallrecord{
            \smalltfield{f}{\smallrecord{
                \smalltfield{f}{\textit{PropCntxt}}\\
                \smalltfield{a}{\textit{PropCntxt}}}}\\
            \smalltfield{a}{\smallrecord{
                \smalltfield{f}{\textit{PropCntxt}}\\
                \smalltfield{a}{\textit{PropCntxt}}}}}}}}} . \\
  \hspace*{1em}\record{
    \mfield{restr}{dog$'$}{\textit{Ppty}}\\
    \mfield{scope}{$\lambda r$:\smallrecord{
        \smalltfield{x}{\textit{Ind}}}
      . \textbf{a}$^\frown$\textbf{boy}$^\frown$\textbf{hugged}$^\frown$\textbf{it$^*$}
      ($c[\mathfrak{s}.\text{x}_0=r.\text{x}]$)$\mid_{\mathcal{F}(\text{dog}')}$}{\textit{Ppty}}\\
    \tfield{e}{every(restr, scope)}}

 
\end{subex} 
\label{ex:edabhi}   
\end{ex} 
The imaginary English expression \preveg{a} corresponds quite closely
to the kind of representation for wide scope readings that are used in
various theories of logical form.  A major difference is that in
logical form there is an index corresponding to the label `x$_0$' that
we use in the interpretation which shows that \textit{every dog} binds
\textit{it$^*$}.  This might be represented something like in
\nexteg{}.
\begin{ex} 
every dog$_{x_0}$, a boy hugged it$^*_{x_0}$ 
\end{ex} 
The imaginary sentence (\ref{ex:edabhi}a) also corresponds closely to
Montague's (\citeyear{Montague1973}) treatment of scope phenomena.
Montague would also index the pronoun and use a quantification rule
with the same index which would replace the pronoun with the
noun-phrase being quantified in.

Neither of these options are open to us since our syntax is defined in
terms of types of utterance situations and signs which relate
utterance situations to contents.  Our realistic strategy does not
allow for the use of additional imaginary utterance structures.  For this reason
we will adapt the kind of storage technique used in \cite{Cooper1983}.
In the original version of storage we moved from assigning a
single content to a syntactic structure to assigning a set of contents
in order to allow for the ambiguous interpretation of a single
syntactic structure.  In our sign-based approach using types the
corresponding move is not such a major change and the result yields a
theory involving underspecified content rather than a set of contents.

To see this consider the type \textit{Sign} introduced in
Chapter~\ref{ch:gram}.  Any object of type \textit{Sign} will be of
the type in \nexteg{}.
\begin{ex} 
\record{\tfield{s-event}{\textit{SEvent}} \\
         \tfield{syn}{\textit{Syn}} \\
        \tfield{cont}{\textit{Cont}}}  
\end{ex} 
The type in \preveg{} is completely underspecified.  Any sign will be
of this type.  We could specify it with respect to content by making
the `cont'-field be a manifest field as in \nexteg{}, where $\varphi$
is a particular content.
\begin{ex} 
\record{\tfield{s-event}{\textit{SEvent}} \\
         \tfield{syn}{\textit{Syn}} \\
        \mfield{cont}{$\varphi$}{\textit{Cont}}}  
 
\end{ex} 
Now recall that the manifest field
\smallrecord{\smallmfield{cont}{$\varphi$}{\textit{Cont}}} is just a
convenient way of writing
\smallrecord{\smalltfield{cont}{\textit{Cont}$_\varphi$}} where
\textit{Cont}$_\varphi$ is a singleton type whose only witness is
$\varphi$ if $\varphi:\textit{Cont}$; otherwise it has no witnesses.  It
is in this sense that the content has been specified to be $\varphi$.
Suppose now that we do not have enough information to fully specify
the content, that is, tie it down to be one particular content, but we
know that it has to be one of either $\varphi$ or $\psi$.  This could be represented by using a join type of two
singleton types, \textit{Cont}$_\varphi$$\vee$\textit{Cont}$_\psi$, as in
\nexteg{}.
\begin{ex} 
\record{\tfield{s-event}{\textit{SEvent}} \\
         \tfield{syn}{\textit{Syn}} \\
        \tfield{cont}{\textit{Cont}$_\varphi$$\vee$\textit{Cont}$_\psi$}}  
 
\label{ex:underspec-join}
   
\end{ex} 
\preveg{} is the type of signs whose contents are either $\varphi$ or $\psi$.
This is, then, a single type, which corresponds to an ``underspecified
content''.  Of course, the set of witnesses of the join type,
$\{\varphi,\psi\}$, could correspond to the set of contents that could be generated
by a storage algorithm.  Another way to achieve this is to use the
operator, $\mathfrak{T}$, introduced in Chapter~\ref{ch:quant},
example (\ref{ex:mathfrakT}) on
p.~\pageref{ex:mathfrakT} which takes a set and returns a type whose
witnesses are exactly the members of the set.  We could thus use
\nexteg{} instead of \preveg{} with the same effect.
\begin{ex} 
\record{\tfield{s-event}{\textit{SEvent}} \\
         \tfield{syn}{\textit{Syn}} \\
        \tfield{cont}{$\mathfrak{T}(\{\varphi,\psi\})$}} 
\end{ex} 

Our strategy is to devise a way of
computing such types on the basis of compositional interpretation
without having to enumerate the members of the set of contents in the
way that we did in (\ref{ex:underspec-join}) and \preveg{}.
 
% Notice that saying that the content is of type
% \textit{Cont}$_c$$\vee$\textit{Cont}$_{c'}$, that is, that it is
% either $c$ or $c'$ is distinct from saying that the content is the
% disjunction of $c$ and $c'$, that is, if $c$ and $c'$ are types, the
% type $c\vee c'$.  The witnesses for $c\vee c'$ will be situations $s$
% such that either $s:c$ or $s:c'$ whereas the witness for
% \textit{Cont}$_c$$\vee$\textit{Cont}$_{c'}$ will be the contents $c$
% and $c'$.  This will make an important different in compositional
% interpretation.  For example, if a sentence $S$ has a content of type     
% \textit{Cont}$_c$$\vee$\textit{Cont}$_{c'}$ we would want to say that
% the sentence \textit{Sam believes $S$} has a content of type
% \textit{Cont}$_{believe(sam,c)}$$\vee$\textit{Cont}$_{believe(sam,c')}$
% rather than that it has the content `believe(sam,$c\vee c'$)'.

% In order to make the relationship between sets of contents and join
% types smoother we will use generalized join types
% (Appendix~\ref{app:jointypes}).  Suppose we have computed a set of
% possible contents, $\{c_1,\ldots,c_n\}$, for a phrase, then the type
% of the content for that phrase is
% $\bigvee\{\textit{Cont}_{c_1},\ldots,\textit{Cont}_{c_n}\}$.  If
% $\mathbb{C}$ represents the set of contents we have computed, we can
% use the more convenient notation
% $\displaystyle{\bigvee_{c\in\mathbb{C}}}\textit{Cont}_c$ to mean
% $\bigvee\{\textit{Cont}_c\mid c\in\mathbb{C}\}$.

We will exploit our treatment of context to include a store of
parametric quantifiers that are to be given a wide scope
interpretation.  Thus our characterization of the type \textit{Cntxt}
will be extended to \nexteg{}.
\begin{ex} 
  \record{
    \tfield{$\mathfrak{q}$}{\textit{QStore}}\\
    \tfield{$\mathfrak{s}$}{\textit{Assgnmnt}}\\
    \tfield{$\mathfrak{w}$}{\textit{Assgnmnt}}\\
    \tfield{$\mathfrak{g}$}{\textit{Assgnmnt}}\\
    \tfield{$\mathfrak{c}$}{\textit{PropCntxt}}}
\end{ex}

\textit{QStore} is the type of assignments all of whose values are
parametric quantifiers, that is, as characterized in \nexteg{}.
\begin{ex} 
\textit{QStore} is a basic type.  $r:\textit{QStore}$ iff $r:Assgnmnt$
and for any $\text{x}_i\in\mathrm{labels}(r)$, $r.\text{x}_i:\textit{PQuant}$ 
\end{ex} 
An example of a witness for \textit{QStore} would thus be \nexteg{}
where we have stored the basic parametric content of \textit{every
  dog} under the label `x$_0$'.
\begin{ex} 
  \record{
    \field{x$_0$}{$\ulcorner\lambda c$:\smallrecord{
        \footnotesize{\textit{Cntxt}}\\
        \smalltfield{$\mathfrak{c}$}{\smallrecord{
            \smalltfield{f}{\textit{PropCntxt}}\\
            \smalltfield{a}{\textit{PropCntxt}}}}} . $\lambda
      P$:\textit{Ppty} . \smallrecord{
      \smallmfield{restr}{dog$'$}{\textit{Ppty}}\\
      \smallmfield{scope}{$P$}{\textit{Ppty}}\\
      \smalltfield{e}{every(restr, scope)}}$\urcorner$}}
        
\end{ex} 
The record in \preveg{} is, among other types, of the type in
\nexteg{}, which will be the kind of type we will be using in our
content types which will be underspecified for quantifier scope.
\begin{ex} 
  \record{
    \footnotesize{\textit{QStore}}\\
    \mfield{x$_0$}{$\ulcorner\lambda c$:\smallrecord{
        \footnotesize{\textit{Cntxt}}\\
        \smalltfield{$\mathfrak{c}$}{\smallrecord{
            \smalltfield{f}{\textit{PropCntxt}}\\
            \smalltfield{a}{\textit{PropCntxt}}}}} . $\lambda
      P$:\textit{Ppty} . \smallrecord{
      \smallmfield{restr}{dog$'$}{\textit{Ppty}}\\
      \smallmfield{scope}{$P$}{\textit{Ppty}}\\
      \smalltfield{e}{every(restr, scope)}}$\urcorner$}{\textit{PQuant}}}
        
\end{ex}
Suppose that the basic parametric content associated with a
noun-phrase is $\mathcal{Q}$, that is, the type of the content as we
have been expressing it so far (using a manifest field) will be
\textit{Cont}$_{\mathcal{Q}}$.  Now we want to generalize this type so
that not only $\mathcal{Q}$ will be a witness for the type but also a
parametric content where $\mathcal{Q}$ is required in the qstore of
the context.  This parametric quantifier will be \nexteg{}.
\begin{ex} 
  $\ulcorner\lambda c$:\smallrecord{
    \footnotesize{\textit{Cntxt}}\\
    \smalltfield{$\mathfrak{q}$}{\smallrecord{
        \smallmfield{x$_0$}{$\mathcal{Q}$}{\textit{PQuant}}}}\\
    \smalltfield{$\mathfrak{s}$}{\smallrecord{
        \smalltfield{x$_0$}{\textit{Ind}}}}} . $\lambda
  P$:\textit{Ppty} . $P\{c.\mathfrak{s}.\text{x}_0\}\urcorner$
\label{ex:stored-Q}        
\end{ex} 
Note that \preveg{} is exactly like the pronoun content introduced in
Chapter~\ref{ch:propnames}, example~(\ref{ex:he-leave-abbrev}), except that we have a parametric quantifier required in the
qstore labelled with the same label `x$_0$' as is used in the pronoun
content.  Thus this corresponds to the content of \textit{it}$^*$ in
(\ref{ex:it-star}) and (\ref{ex:edabhi}).  Note that \preveg{} is of
type \textit{PQuant}.  It is distinguished from a parametric
quantifier like \nexteg{}, however, in that it requires the qstore in
the context to be non-empty.
\begin{ex} 
$\ulcorner\lambda c$:\smallrecord{
        \footnotesize{\textit{Cntxt}}\\
        \smalltfield{$\mathfrak{c}$}{\smallrecord{
            \smalltfield{f}{\textit{PropCntxt}}\\
            \smalltfield{a}{\textit{PropCntxt}}}}} . $\lambda
      P$:\textit{Ppty} . \smallrecord{
      \smallmfield{restr}{dog$'$}{\textit{Ppty}}\\
      \smallmfield{scope}{$P$}{\textit{Ppty}}\\
      \smalltfield{e}{every(restr, scope)}}$\urcorner$ 
\end{ex} 
Using terminology that goes back to
\cite{Bos1996}, we will say that \preveg{} is \textit{plugged} (that
is, in our terms, does not require anything to be in the qstore)
whereas (\ref{ex:stored-Q}) is \textit{unplugged} (that is, in our
terms, \textit{does} require something to be in the qstore).  The
intuition is that when a quantifier has been placed in the qstore it
has been unplugged from the main interpretation and needs to be
plugged back in at some point in order to get a fully specified
interpretation.  We formally characterize the notion \textit{plugged}
in \nexteg{}.
\begin{ex} 
A parametric content, $\alpha$, is \textit{unplugged} iff $c:\alpha$.bg implies $c.\mathfrak{q}\not=\emptyset$
(that is, $c.\mathfrak{q}$ is not the empty record).  Otherwise $\alpha$
is \textit{plugged}. 
\end{ex} 
  

We can derive (\ref{ex:stored-Q}) from $\mathcal{Q}$ by an operation
`$\mathrm{store}$' characterized in \nexteg{}.
\begin{ex}
If $\mathcal{Q}:\textit{PQuant}$ and $\mathcal{Q}$ is plugged, then
$\mathrm{store}(\mathcal{Q})$ is
\begin{quote}
  $\ulcorner\lambda c$:\smallrecord{
    \footnotesize{\textit{Cntxt}}\\
    \smalltfield{$\mathfrak{q}$}{\smallrecord{
        \smallmfield{x$_0$}{$\mathcal{Q}$}{\textit{PQuant}}}}\\
    \smalltfield{$\mathfrak{s}$}{\smallrecord{
        \smalltfield{x$_0$}{\textit{Ind}}}}} . $\lambda
  P$:\textit{Ppty} . $P\{c.\mathfrak{s}.\text{x}_0\}\urcorner$
\end{quote}

\end{ex} 
  

% The `unplugged'-field contains a set of unplugged
% interpretations consisting of a set of parametric quantifiers
% (possibly empty) and a core interpretation consisting of a parametric
% content.  Thus we define a type \textit{UInterp} of unplugged
% interpretations as in \nexteg{}.
% \begin{ex} 
% \textit{UInterp} = \record{\tfield{quants}{\{\textit{PQuant}\}}\\
%                            \tfield{core}{\textit{PCont}}}
% \end{ex} 
% The `cont'-field in a sign will now be required to be of the type in
% \nexteg{}.
% \begin{ex} 
% \record{\tfield{unplugged}{\{\textit{UInterp}\}}\\
%         \tfield{plugged}{$\displaystyle{\bigvee_{c\in\mathrm{unplugged},\
%               c.\mathrm{quants}=\emptyset}\textit{Cont}_{c.\mathrm{core}}}$}}
% \end{ex} 
% That is, the `unplugged'-field in the content will contain a set of
% unplugged interpretations (consisting of a quantifier store, possibly empty, and a core
% interpretation) and the `plugged'-field will contain one of the cores
% of those unplugged interpretations whose quantifier store is
% empty.

% Let us consider what this will look like for the example \textit{every
%   boy wants a dog} using \textit{every}$'$, \textit{boy}$'$ etc. as
% abbreviations for non-parametric contents for these words.  We show
% the `cont'-field of the sign type associated with constituents of this
% sentence.
% \begin{ex} 
% \begin{subex} 
 
% \item a dog 
 
% \item
%   \smallrecord{\smallmfield{unplugged}{\{\smallrecord{\field{quants}{$\emptyset$}\\
%                                             \field{core}{\smallrecord{\field{bg}{\textit{Rec}}\\
%                                                                       \field{fg}{$\lambda\mathfrak{s}$:\textit{Rec}
%                                                                         . \textit{a}$'$(\textit{dog}$'$)}}}}, \\
%    \hspace*{5em}                            \smallrecord{\field{quants}{\{\smallrecord{\field{bg}{\smallrecord{\smalltfield{x$_0$}{\textit{Ind}}}} \\
%                                                                           \field{fg}{$\lambda\mathfrak{s}$:\smallrecord{\smalltfield{x$_0$}{\textit{Ind}}}
%                                                                             . \textit{a}$'$(\textit{dog}$'$)}}\}} \\
%                                             \field{core}{\smallrecord{\field{bg}{\smallrecord{\smalltfield{x$_0$}{\textit{Ind}}}}
%                                                 \\
%                                                                       \field{fg}{$\lambda\mathfrak{s}$:\smallrecord{\smalltfield{x$_0$}{\textit{Ind}}}
%                                                                         . $\lambda
%                                                                         P$:\textit{Ppty}
%                                                                         . $P(\mathfrak{s}.\textrm{x}_0)$}}}}\}} {\{\textit{UInterp}\}}
%                                                             \\
%           \smalltfield{plugged}{\textit{Cont}$_{\text{\smallrecord{\field{bg}{\textit{Rec}}\\
%                                                                       \field{fg}{$\lambda\mathfrak{s}$:\textit{Rec}
%                                                                         . \textit{a}$'$(\textit{dog}$'$)}}}}$}}
 
% \end{subex} 
   
% \end{ex} 
% \begin{ex} 
% \begin{subex} 
 
% \item want a dog 
 
% \item
%   \smallrecord{\smallmfield{unplugged}{\{\smallrecord{\field{quants}{$\emptyset$}\\
%                                             \field{core}{\smallrecord{\field{bg}{\textit{Rec}}\\
%                                                                       \field{fg}{$\lambda\mathfrak{s}$:\textit{Rec}
%                                                                         . \textit{want}$'$(\textit{a}$'$(\textit{dog}$'$))}}}}, \\
%    \hspace*{5em}                            \smallrecord{\field{quants}{\{\smallrecord{\field{bg}{\smallrecord{\smalltfield{x$_0$}{\textit{Ind}}}} \\
%                                                                           \field{fg}{$\lambda\mathfrak{s}$:\smallrecord{\smalltfield{x$_0$}{\textit{Ind}}}
%                                                                             . \textit{a}$'$(\textit{dog}$'$)}}\}} \\
%                                             \field{core}{\smallrecord{\field{bg}{\smallrecord{\smalltfield{x$_0$}{\textit{Ind}}}}
%                                                 \\
%                                                                       \field{fg}{$\lambda\mathfrak{s}$:\smallrecord{\smalltfield{x$_0$}{\textit{Ind}}}
%                                                                         . \textit{want}$'$($\lambda
%                                                                         P$:\textit{Ppty}
%                                                                         . $P(\mathfrak{s}.\textrm{x}_0)$)}}}}\}} {\{\textit{UInterp}\}}
%                                                             \\
%           \smalltfield{plugged}{\textit{Cont}$_{\text{\smallrecord{\field{bg}{\textit{Rec}}\\
%                                                                       \field{fg}{$\lambda\mathfrak{s}$:\textit{Rec}
%                                                                         . \textit{want}(\textit{a}$'$(\textit{dog}$'$))}}}}$}}
 
% \end{subex} 
   
% \end{ex} 

% \begin{ex} 
% \begin{subex} 
 
% \item every boy 
 
% \item
%   \smallrecord{\smallmfield{unplugged}{\{\smallrecord{\field{quants}{$\emptyset$}\\
%                                             \field{core}{\smallrecord{\field{bg}{\textit{Rec}}\\
%                                                                       \field{fg}{$\lambda\mathfrak{s}$:\textit{Rec}
%                                                                         . \textit{every}$'$(\textit{boy}$'$)}}}}, \\
%    \hspace*{5em}                            \smallrecord{\field{quants}{\{\smallrecord{\field{bg}{\smallrecord{\smalltfield{x$_0$}{\textit{Ind}}}} \\
%                                                                           \field{fg}{$\lambda\mathfrak{s}$:\smallrecord{\smalltfield{x$_0$}{\textit{Ind}}}
%                                                                             . \textit{every}$'$(\textit{boy}$'$)}}\}} \\
%                                             \field{core}{\smallrecord{\field{bg}{\smallrecord{\smalltfield{x$_0$}{\textit{Ind}}}}
%                                                 \\
%                                                                       \field{fg}{$\lambda\mathfrak{s}$:\smallrecord{\smalltfield{x$_0$}{\textit{Ind}}}
%                                                                         . $\lambda
%                                                                         P$:\textit{Ppty}
%                                                                         . $P(\mathfrak{s}.\textrm{x}_0)$}}}}\}} {\{\textit{UInterp}\}}
%                                                             \\
%           \smalltfield{plugged}{\textit{Cont}$_{\text{\smallrecord{\field{bg}{\textit{Rec}}\\
%                                                                       \field{fg}{$\lambda\mathfrak{s}$:\textit{Rec}
%                                                                         . \textit{every}$'$(\textit{boy}$'$)}}}}$}}
 
% \end{subex} 
   
% \end{ex} 
    
% \begin{ex} 
% \begin{subex} 
 
% \item every boy wants a dog 
 
% \item
% % \newsavebox{\test}
% % \sbox\test{\begin{tabular}{l}a,\\b\end{tabular}}
% % \usebox{\test}

% % \newsavebox{\unplugged}
% % \sbox\unplugged{$\left\{ \text{\begin{tabular}{l}\usebox{\unpluggeda},\\
% %                                     \usebox{\unpluggedb},\\
% %                                     \usebox{\unpluggedc},\\
% %                                     \usebox{\unpluggedd},\\
% %                                     \usebox{\unpluggede},\\
% %                                     \usebox{\unpluggedf},\\
% %                                     \usebox{\unpluggedg}
% %                   \end{tabular}}\right\}$}

% \newsavebox{\plugged}
% \sbox\plugged{$\displaystyle{\bigvee_{c\in\mathbb{C}_{\mathrm{plugged}}}\textit{Cont}_c}$}
                              

% \smallrecord{\smallmfield{unplugged}{$\mathbb{C}_{\mathrm{unplugged}}$}{\{\textit{UInterp}\}}\\
%              \smalltfield{plugged}{\usebox{\plugged}}}

 
% \end{subex} 
   
% \end{ex} 
% where $\mathbb{C}_{\mathrm{unplugged}}$ is the set whose members are
% listed in \nexteg{} and $\mathbb{C}_{\mathrm{plugged}}$ is the set
% whose members are listed in (\ref{ex:plugged}).
% \newsavebox{\unpluggeda}
% \sbox\unpluggeda{\smallrecord{\field{quants}{$\emptyset$}\\
%                                                        \field{core}{\smallrecord{\field{bg}{\textit{Rec}}\\
%                                                                                  \field{fg}{$\lambda\mathfrak{s}$:\textit{Rec}
%                                                                          . \textit{every}$'$(\textit{boy}$'$)(\textit{want}$'$(\textit{a}$'$(\textit{dog}$'$)))}}}}}

% \newsavebox{\unpluggedb}
% \sbox\unpluggedb{\smallrecord{\field{quants}{\{\smallrecord{\field{bg}{\smallrecord{\smalltfield{x$_0$}{\textit{Ind}}}} \\
%                                                                            \field{fg}{$\lambda\mathfrak{s}$:\smallrecord{\smalltfield{x$_0$}{\textit{Ind}}}
%                                                                             . \textit{every}$'$(\textit{boy}$'$)}}\}} \\
%                                             \field{core}{\smallrecord{\field{bg}{\smallrecord{\smalltfield{x$_0$}{\textit{Ind}}}}
%                                                 \\
%                                                                       \field{fg}{\begin{tabular}{l}$\lambda\mathfrak{s}$:\smallrecord{\smalltfield{x$_0$}{\textit{Ind}}}
%                                                                         . \\
%                                                                                    \hspace*{1em}$\lambda
%                                                                         P$:\textit{Ppty}
%                                                                         . \\
%                                                                                    \hspace*{2em}$P(\mathfrak{s}.\textrm{x}_0)$(\textit{want}$'$(\textit{a}$'$(\textit{dog}$'$)))\end{tabular}}}}}}

% \newsavebox{\unpluggedc}
% \sbox\unpluggedc{\smallrecord{\field{quants}{\{\begin{tabular}{l}\smallrecord{\field{bg}{\smallrecord{\smalltfield{x$_0$}{\textit{Ind}}}} \\
%                                                                           \field{fg}{$\lambda\mathfrak{s}$:\smallrecord{\smalltfield{x$_0$}{\textit{Ind}}}
%                                                                             . \textit{every}$'$(\textit{boy}$'$)}}, \\
%                                                                         \smallrecord{\field{bg}{\smallrecord{\smalltfield{x$_1$}{\textit{Ind}}}} \\
%                                                                           \field{fg}{$\lambda\mathfrak{s}$:\smallrecord{\smalltfield{x$_1$}{\textit{Ind}}}
%                                                                             . \textit{a}$'$(\textit{dog}$'$)}}\end{tabular}\}} \\
%                                             \field{core}{\smallrecord{\field{bg}{\smallrecord{\smalltfield{x$_0$}{\textit{Ind}}\\
%                                                                                               \smalltfield{x$_1$}{\textit{Ind}}}}
%                                                 \\
%                                                                       \field{fg}{\begin{tabular}{l}$\lambda\mathfrak{s}$:\smallrecord{\smalltfield{x$_0$}{\textit{Ind}} \\
%                                                                                                                 \smalltfield{x$_1$}{\textit{Ind}}}
%                                                                         . \\
%                                                                                                        \hspace*{1em}$\lambda
%                                                                         P$:\textit{Ppty}
%                                                                         . \\
%                                                                                    \hspace*{2em}$P(\mathfrak{s}.\textrm{x}_0)$(\textit{want}$'$($\lambda
%                                                                                    P$:\textit{Ppty}
%                                                                                    .
%                                                                                    $P(\mathfrak{s}.\textrm{x}_1)$))\end{tabular}}}}}}

% \newsavebox{\unpluggedd}
% \sbox\unpluggedd{\smallrecord{\field{quants}{\{\smallrecord{\field{bg}{\smallrecord{\smalltfield{x$_1$}{\textit{Ind}}}} \\
%                                                                           \field{fg}{$\lambda\mathfrak{s}$:\smallrecord{\smalltfield{x$_1$}{\textit{Ind}}}
%                                                                             . \textit{a}$'$(\textit{dog}$'$)}}\}} \\
%                                             \field{core}{\smallrecord{\field{bg}{\smallrecord{\smalltfield{x$_1$}{\textit{Ind}}}}
%                                                 \\
%                                                                       \field{fg}{\begin{tabular}{l}$\lambda\mathfrak{s}$:\smallrecord{\smalltfield{x$_1$}{\textit{Ind}}}
%                                                                         . \\
%                                                                                    \hspace*{1em}\textit{every}$'$(\textit{boy}$'$) \\ 
% \hspace*{2em}($\lambda
%    r$:\smallrecord{\smalltfield{x$_0$}{\textit{Ind}}} . \\
%    \hspace*{3em}$\lambda P$:\textit{Ppty} .  $P$($r$.x$_0$)\\
%                                                                                    \hspace*{4em}(\textit{want}$'$\\ \hspace*{5em}($\lambda
%                                                                         P$:\textit{Ppty}
%                                                                         .
                                                                                
% $P(\mathfrak{s}.\textrm{x}_1)$)))\end{tabular}}}}}}

% \newsavebox{\unpluggede}
% \sbox\unpluggede{\smallrecord{\field{quants}{\{\smallrecord{\field{bg}{\smallrecord{\smalltfield{x$_0$}{\textit{Ind}}}} \\
%                                                                           \field{fg}{$\lambda\mathfrak{s}$:\smallrecord{\smalltfield{x$_0$}{\textit{Ind}}}
%                                                                             . \textit{every}$'$(\textit{boy}$'$)}}\}} \\
%                                             \field{core}{\smallrecord{\field{bg}{\smallrecord{\smalltfield{x$_0$}{\textit{Ind}}}}
%                                                 \\
%                                                                       \field{fg}{\begin{tabular}{l}$\lambda\mathfrak{s}$:\smallrecord{\smalltfield{x$_0$}{\textit{Ind}}}
%                                                                         . \\
%                                                                                    \hspace*{1em}\textit{a}$'$(\textit{dog}$'$)\\
%                                                                                    \hspace*{2em}($\lambda
%                                                                         r$:\smallrecord{\smalltfield{x$_1$}{\textit{Ind}}}
%                                                                           . \\
%                                                                                    \hspace*{3em}$\lambda
%                                                                           P$:\textit{Ppty}
%                                                                           . \\
%                                                                                    \hspace*{4em}$P$($r$.x$_1$)\\
%                                                                                    \hspace*{5em}(\textit{want}$'$($\lambda
%                                                                         P$:\textit{Ppty}
%                                                                         . $P(\mathfrak{s}.\textrm{x}_0)$))})\end{tabular}}}}}

% \newsavebox{\unpluggedf}
% \sbox\unpluggedf{\smallrecord{\field{quants}{\{\}}\\
%                               \field{core}{\smallrecord{\field{bg}{\textit{Rec}}\\
%                                                         \field{fg}{\begin{tabular}{l}
%                                                                     $\lambda\mathfrak{s}$:\textit{Rec}
%                                                                      \\
%                                                                     \hspace*{1em}\textit{a}$'$(\textit{dog}$'$)
%                                                                       \\
%                                                                     \hspace*{2em}($\lambda
%                                                                      r_1$:\smallrecord{\smalltfield{x$_1$}{\textit{Ind}}}
%                                                                      . \\
%                                                                     \hspace*{3em}\textit{every}$'$(\textit{boy}$'$)\\
%                                                                     \hspace*{4em}($\lambda
%                                                                      r_2$:\smallrecord{\smalltfield{x$_0$}{\textit{Ind}}}
%                                                                      . \\
%                                                                     \hspace*{5em}$\lambda
%                                                                      P$:\textit{Ppty}
%                                                                      . $P$($r_1$.x$_0$)\\
%                                                                     \hspace*{6em}\textit{wants}$'$($\lambda
%                                                                      P$:\textit{Ppty}
%                                                                      . $P$($r_1$.x$_1$))))\end{tabular}}}}}}

% \newsavebox{\unpluggedg}
% \sbox\unpluggedg{\smallrecord{\field{quants}{\{\}}\\
%                               \field{core}{\smallrecord{\field{bg}{\textit{Rec}}\\
%                                                         \field{fg}{\begin{tabular}{l}
%                                                                     $\lambda\mathfrak{s}$:\textit{Rec}
%                                                                      \\
%                                                                     \hspace*{1em}\textit{every}$'$(\textit{boy}$'$)
%                                                                       \\
%                                                                     \hspace*{2em}($\lambda
%                                                                      r_1$:\smallrecord{\smalltfield{x$_0$}{\textit{Ind}}}
%                                                                      . \\
%                                                                     \hspace*{3em}\textit{a}$'$(\textit{dog}$'$)\\
%                                                                     \hspace*{4em}($\lambda
%                                                                      r_2$:\smallrecord{\smalltfield{x$_1$}{\textit{Ind}}}
%                                                                      . \\
%                                                                     \hspace*{5em}$\lambda
%                                                                      P$:\textit{Ppty}
%                                                                      . $P$($r_1$.x$_0$)\\
%                                                                     \hspace*{6em}\textit{wants}$'$($\lambda
%                                                                      P$:\textit{Ppty}
%                                                                      . $P$($r_1$.x$_1$))))\end{tabular}}}}}}



% \begin{ex} 
% \begin{subex} 
 
% \item \usebox{\unpluggeda} 
 
% \item \usebox{\unpluggedb}

% \item \usebox{\unpluggedc}

% \item \usebox{\unpluggedd}

% \item \usebox{\unpluggede}

% \item \usebox{\unpluggedf}

% \item \usebox{\unpluggedg} 
 
% \end{subex} 
   
% \end{ex} 

% \begin{ex} 
% \begin{subex} 
 
% \item \smallrecord{\field{bg}{\textit{Rec}}\\
%                    \field{fg}{$\lambda\mathfrak{s}$:\textit{Rec} . \textit{every}$'$(\textit{boy}$'$)(\textit{want}$'$(\textit{a}$'$(\textit{dog}$'$)))}}
                               
% \item  \smallrecord{\field{bg}{\textit{Rec}}\\
%                     \field{fg}{\begin{tabular}{l}
%                                     $\lambda\mathfrak{s}$:\textit{Rec}
%                                                                      \\
%                                       \hspace*{1em}\textit{a}$'$(\textit{dog}$'$)
%                                                                       \\
%                                       \hspace*{2em}($\lambda
%                                                            r_1$:\smallrecord{\smalltfield{x$_1$}{\textit{Ind}}}
%                                                                      . \\
%                                       \hspace*{3em}\textit{every}$'$(\textit{boy}$'$)\\
%                                       \hspace*{4em}($\lambda
%                                                           r_2$:\smallrecord{\smalltfield{x$_0$}{\textit{Ind}}}
%                                                                      . \\
%                                       \hspace*{5em}$\lambda
%                                                         P$:\textit{Ppty}
%                                                                      . $P$($r_1$.x$_0$)\\
%                                        \hspace*{6em}\textit{wants}$'$($\lambda
%                                                                      P$:\textit{Ppty}
%                                                                      . $P$($r_1$.x$_1$))))\end{tabular}}}
 
% \item  \smallrecord{\field{bg}{\textit{Rec}}\\
%                                                         \field{fg}{\begin{tabular}{l}
%                                                                     $\lambda\mathfrak{s}$:\textit{Rec}
%                                                                      \\
%                                                                     \hspace*{1em}\textit{every}$'$(\textit{boy}$'$)
%                                                                       \\
%                                                                     \hspace*{2em}($\lambda
%                                                                      r_1$:\smallrecord{\smalltfield{x$_0$}{\textit{Ind}}}
%                                                                      . \\
%                                                                     \hspace*{3em}\textit{a}$'$(\textit{dog}$'$)\\
%                                                                     \hspace*{4em}($\lambda
%                                                                      r_2$:\smallrecord{\smalltfield{x$_1$}{\textit{Ind}}}
%                                                                      . \\
%                                                                     \hspace*{5em}$\lambda
%                                                                      P$:\textit{Ppty}
%                                                                      . $P$($r_1$.x$_0$)\\
%                                                                     \hspace*{6em}\textit{wants}$'$($\lambda
%                                                                      P$:\textit{Ppty}
%                                                                      . $P$($r_1$.x$_1$))))\end{tabular}}}
 
% \end{subex} 
% \label{ex:plugged}   
% \end{ex} 
  
% We now take a look at what is needed in order achieve the analysis we
% have discussed above.  We will express storage and retrieval as
% constraints on linguistically allowable sign types.  As a preliminary
% we will define a bookkeeping function, max$_x$, which enables us to determine
% the maximum $i$ such that `x$_i$' is used as a label in a set of
% records, $R$.  This will enable us to store quantifiers using labels
% `x$_0$', `x$_1$' and so on in order as in the examples discussed
% above. `max$_x$' is defined as in \nexteg{} where $R$ is an arbitrary
% set of records.
% \begin{ex} 
% max$_x$($R$) = 
% \begin{quote}
% max($\{i\mid\exists r\in R\ \exists v\ \langle \mathrm{x}_i,v\rangle\in r.\mathrm{bg}\}$), if
% defined

% otherwise: -1 
% \end{quote}
% \end{ex}
% That is, max$_x$($R$) is the maximum $i$ such that `x$_i$' is a label
% in the `bg'-field of one of the records in $R$.  If there is no such label max$_x$($R$)
% returns -1. Thus in adding to a quantifier store, $R$, (as in the
% `quants'-fields in the examples above) we will always add a new
% quantifier indexed by max$_x$($R$)+1.  For brevity we will denote this by
% incr$_x$($R$), that is, the result of incrementing the maximal x in $R$. 

% It will also be useful to define an operation that adds a field
% \smallrecord{\field{$\ell$}{$v$}} to a record $r$ if there is no field
% with the label $\ell$ or replaces the $\ell$-field in $r$ with
% \smallrecord{\field{$\ell$}{$v$}}.  We will represent the new record
% thus derived as $r\oplus$\smallrecord{\field{$\ell$}{$v$}}.  This
% is defined in \nexteg{}.
% \begin{ex} 
% $r\oplus$\smallrecord{\field{$\ell$}{$v$}} is \\
% \hspace*{1em}$r\cup\{\langle\ell,v\rangle\}$, if there is no $v'$ such
% that $\langle\ell,v'\rangle\in\mathfrak{s}$\\
% \hspace*{1em}$(r-\{\langle\ell,r.l\rangle\})\cup\{\langle\ell,v\rangle\}$, otherwise 
% \end{ex} 
% Similarly we define $T\ominus$\smallrecord{\smalltfield{$\ell$}{$v$}}
% for record types, $T$, in
% \nexteg{}.
% \begin{ex} 
% $T\ominus$\smallrecord{\smalltfield{$\ell$}{$v$}} is \\ 
% \hspace*{1em}$T-\{\langle\ell,v\rangle\}$ 
% \end{ex} 
% Note that this version of subtraction for record types will only
% result in a well-formed record type if there are not fields in $T$
% which depend on $\ell$.  In order to rectify this we need to make
% $T\ominus$\smallrecord{\smalltfield{$\ell$}{$v$}} remove in addition
% all fields which depend on $\ell$.  A field, $f$, is dependent on $\ell$
% just in case it is of the form
% $\langle\ell',\langle\mathcal{F},\pi\rangle\rangle$ where $\pi$ is a
% sequence of paths containing $\ell$. Therefore we can refine the
% definition in \preveg{} as that in \nexteg{}.
% \begin{ex} 
% $T\ominus$\smallrecord{\smalltfield{$\ell$}{$v$}} is \\ 
% \hspace*{1em}$T-(\{\langle\ell,v\rangle\}\cup\{f\in T\mid f\
% \text{dependent on}\ \ell\})$ 
% \end{ex} 
% This is technically not enough, since there may still be fields left
% which are dependent on the dependent fields we have removed.
% Therefore we need to make the definition recursive as in \nexteg{}.
% \begin{ex} 
% $T\ominus$\smallrecord{\smalltfield{$\ell$}{$v$}} is \\ 
% \hspace*{1em}$(\ldots(T-\{\langle\ell,v\rangle\}\ominus
% f_0)\ldots)\ominus f_n$ where $\{f_0,\ldots,f_n\}$ is $\{f\in T\mid f\
% \text{dependent on}\ \ell\}$ 
% \end{ex}   

% [Is this enough to cover paths which contain $\ell$?]

% We will also for convenience use the notation $T\ominus\ell$ to mean
% $T\ominus[\ell:v]$ if $\langle\ell,v\rangle\in T$ or $T$ if there is
% no field labelled $\ell$ in $T$.

% We can now define storage as in
% \nexteg{}, where $\sigma$ is an arbitrary sign.
% \begin{ex} 
% \textbf{Storage}

% If $\alpha\in\sigma$.cont.unplugged, $\alpha$.core : \textit{PQuant}
% and \\ $\alpha$.core.fg $\not=$
% $\lambda\mathfrak{s}$:\smallrecord{\smalltfield{x$_i$}{\textit{Ind}}}
%   . $\lambda P$:\textit{Ppty} . $P$($\mathfrak{s}$.x$_i$) for any $i$, \\then

% \hspace*{1em}\record{\field{quants}{$\alpha$.quants $\cup$ \{
%                               \record{\field{bg}{\smallrecord{\smalltfield{x$_{\mathrm{incr}_x(\alpha.\mathrm{quants})}$}{\textit{Ind}}}}\\
%         \field{fg}{$\lambda\mathfrak{s}$:\smallrecord{\smalltfield{x$_{\mathrm{incr}_x(\alpha.\mathrm{quants})}$}{\textit{Ind}}}
%           . $\alpha$.core.fg($\mathfrak{s}$)}}\}}\\
% \field{core}{\record{\field{bg}{\smallrecord{\smalltfield{x$_{\mathrm{incr}_x(\alpha.\mathrm{quants})}$}{\textit{Ind}}}}\\
%                      \field{fg}{$\lambda\mathfrak{s}$:\smallrecord{\smalltfield{x$_{\mathrm{incr}_x(\alpha.\mathrm{quants})}$}{\textit{Ind}}}
%                        . $\lambda P$:\textit{Ppty}
%                        . $P$($\mathfrak{s}$.x$_{\mathrm{incr}_x(\alpha.\mathrm{quants})}$)}}}}

% \hspace*{4em}$\in$ $\sigma$.cont.unplugged
                                 
% \end{ex} 
% If $\sigma$ is a sign, $\sigma$.cont.unplugged is a set of unplugged interpretations.
% \preveg{} says that if any
% of those is a noun-phrase interpretation, that is, its core is of type
% \textit{PQuant}, and is not the result of applying storage, that is,
% the foreground of its core is not an interpretation depending on a
% context, $\mathfrak{s}$, as introduced by storage, then
% $\sigma$.cont.unplugged also contains an unplugged interpretation
% where the core has been stored. 

We will now turn our attention to retrieval which removes a quantifier
from the qstore and quantifies over the virtual pronoun created by
storage.


\begin{shaded}
In order to do this we need to generalize our
characterization of `$\ominus$' in Chapter~\ref{ch:quant}, example (\ref{ex:ominus}) to include
\textit{QStore} as a special case exactly similar to
\textit{Assgnmnt}.  This is done in \nexteg{}.
\begin{ex}

  \begin{enumerate} 

  
 
\item If $T$ is \textit{Assgnmnt}$\wedge T'$, $T'$ is a record type and
  $\ell\in\mathrm{labels}(T')$, then
  \begin{enumerate} 
    
  \item if $\mathrm{labels}(T')=\{\ell\}$, $T\ominus\ell=\textit{Assgnmnt}$ 
    
  \item otherwise, $T\ominus\ell= \textit{Assgnmnt}\wedge (T'\ominus\ell/T)$
    
  \end{enumerate}
  
\item If $T$ is \textit{PropCntxt}$\wedge T'$ and
  $\pi\in\mathrm{tpaths}(T')$ then
  \begin{enumerate}
    
  \item if $\pi$ is $\ell$ and $\mathrm{labels}(T')=\{\ell\}$, then $T\ominus\pi=\textit{PropCntxt}$
    
  \item otherwise, $T\ominus\pi= \textit{PropCntxt}\wedge(T'\ominus \pi/T')$
  \end{enumerate}

  
\item If $T$ is \textit{QStore}$\wedge T'$ and
  $\pi\in\mathrm{tpaths}(T')$ then
  \begin{enumerate}
    
  \item if $\pi$ is $\ell$ and $\mathrm{labels}(T')=\{\ell\}$, then $T\ominus\pi=\textit{PropCntxt}$
    
  \item otherwise, $T\ominus\pi= \textit{QStore}\wedge(T'\ominus \pi/T')$
  \end{enumerate}
\end{enumerate}

\label{ex:ominus-with-qstore}   
\end{ex}

\end{shaded}

We now characterize retrieval in a version corresponding to
quantification with scope over sentences.  In a more complete
treatment we would at least add quantification with scope over verb phrases
and common nouns corresponding to Montague's (\citeyear{Montague1973})
treatment.  In \nexteg{}, we characterize an operation
`$\mathrm{retrieve}$' which maps a parametric record type with a
quantifier in store to one where the quantifier is removed from the
store and given scope over the non-parametric content.

\begin{ex} 
% \textbf{Retrieval (S)} 

% If $\alpha\in\sigma$.cont.unplugged, for some $i$,
% $\lambda\mathfrak{s}$:\smallrecord{\smalltfield{x$_i$}{\textit{Ind}}}
% . $Q$ $\in$ $\alpha$.quants and $\alpha$.core : ($T$ $\rightarrow$
% \textit{Type}), where $T$ $\sqsubseteq$
% \smallrecord{\smalltfield{x$_i$}{\textit{Ind}}}, then \\

% \hspace*{1em}\record{\field{quants}{$\alpha$.quants $-$
%     $\{\lambda\mathfrak{s}$:\smallrecord{\smalltfield{x$_i$}{\textit{Ind}}}
%     . $Q\}$} \\
%         \field{core}{$\lambda\mathfrak{s}$:$T-$\{\smallrecord{\smalltfield{x$_i$}{\textit{Ind}}}\}
%           . $Q$($\lambda
%           r$:\smallrecord{\smalltfield{x$_i$}{\textit{Ind}}}
%           . $\alpha$.core($\mathfrak{s}\oplus$\smallrecord{\field{x$_i$}{$r$.x$_i$}}))}}

%           \hspace*{4em}$\in$ $\sigma$.cont.unplugged

  If $\alpha:\textit{PRecType}$, $\mathcal{Q}:\textit{PQuant}$ 
  $\alpha.\text{bg}\sqsubseteq$ \smallrecord{
    \tfield{$\mathfrak{q}$}{\smallrecord{
          \smallmfield{x$_i$}{$\mathcal{Q}$}{\textit{Ind}}}}}, $\mathcal{Q}'$ is
$[\mathcal{Q}]_{\mathfrak{c}\leadsto\mathfrak{c}.\text{f}}$ and
$\alpha'$ is
$[\mathrm{incr}(\alpha,\mathcal{Q}')]_{\mathfrak{c}\leadsto\mathfrak{c}.a}$, then
  $\mathrm{retrieve}(\text{x}_i,\alpha)$ is
  \begin{quote}
    $\lambda
    c$:$(\mathcal{Q}'.\text{bg}$\d{$\wedge$}$\alpha'.\text{bg}\ominus\mathfrak{q}.\text{x}_i,\mathfrak{s}.\text{x}_i)$
    . \\ \hspace*{2em}$\mathcal{Q'}(c)(\mathfrak{P}(\ulcorner\lambda
    r$:\smallrecord{
      \smalltfield{x}{\textit{Ind}}\\
      \smalltfield{$\mathfrak{s}$}{\smallrecord{
\footnotesize{\textit{Assgnmnt}}\\          \smallmfield{x$_i$}{$\Uparrow$x}{\textit{Ind}}}}}\d{$\wedge$}$\alpha'.\text{bg}^{\mathfrak{s}.\text{x}_i}$
    . $\alpha'(c[r][\mathfrak{q}.\text{x}_i=\mathcal{Q}])\urcorner))$
  \end{quote}
  
\end{ex}

% \preveg{} says that if any of the unplugged interpretations in a sign
% contains a quantifier indexed by $i$, that is, it is a parametric
% quantifier whose domain type is
% \smallrecord{\smalltfield{x$_i$}{\textit{Ind}}}, and the core of this
% unplugged interpretation is a function from contexts to a type, that
% is, it corresponds to a (declarative) sentence interpretation, where
% the context is required to provide a value for `x$_i$', that is, the
% domain type is a subtype of
% \smallrecord{\smalltfield{x$_i$}{\textit{Ind}}},\footnote{Note that
%   this requirement will prevent the kind of vacuous binding that
%   Keller storage \citep{Keller1988} avoided.} then the quantifier is
% removed from the store and given scope over the sentence
% interpretation in the core binding `x$_i$'.  The dependence on `x$_i$'
% in the context is removed from the core interpretation.

`$\mathrm{retrieve}$' applies to a label, `x$_i$' and a parametric
record type, $\alpha$, which contains a parametric quantifier,
$\mathcal{Q}$, in its qstore labelled by `x$_i$'.  It returns a
parametric record type where the context type is an appropriate
combination of the context types (labelled `bg') associated with
$\mathcal{Q}$ and $\alpha$ including path adjustment and
incrementation but with fields labelled `$\mathfrak{q}.\text{x}_i$'
and `$\mathfrak{s}.\text{x}_i$' removed together with any fields
depending on them.  The field labelled `$\mathfrak{s}$.x$_i$' and
fields dependent on it are added to the
domain type of the property to which $\mathcal{Q}$ is applied  and a
`$\mathfrak{q}$.x$_i$'-field is added to the context argument for
$\alpha'$ to make the context an appropriate argument,
though this field will not appear in the result.  
Purification ($\mathfrak{P}$) is applied to the property so that the
additional fields from the domain type are ``moved down'' into $\alpha$ giving the effect
of what in Discourse Representation Theory would be called local
accommodation of any presuppositions \citep{VanderSandt1992}.  This
means that any presuppositions associated with the quantifier will
apply at the level at which it is quantified in.

Now we have two ways, storage and retrieval, in which we can derive parametric contents from
other parametric contents.  How can we, then, characterize the type of
parametric contents associated with some particular phrase?  We will introduce a type \textit{ContType}
characterized in \nexteg{}.
\begin{ex} 
\begin{subex} 
 
\item \textit{ContType}, ``the type of types of contents'', is a basic type 
 
\item $T:\textit{ContType}$ iff $T\sqsubseteq\textit{Cont}$ 
 
\end{subex} 
   
\end{ex}
Suppose
that $T:\textit{ContType}$.  Then we define a new type
$\mathfrak{S}(T)$ whose witnesses are the closure of the set of
witnesses of $T$ under `$\mathrm{store}$' and `$\mathrm{retrieve}$'.
We give a precise characterization of this in \nexteg{}.
\begin{ex} 
\begin{subex} 
 
\item If $T:\textit{ContType}$, then $\mathfrak{S}(T)$ is a type 
 
\item The witnesses of $\mathfrak{S}(T)$ are characterized by
  \begin{enumerate} 
 
  \item if $\varphi:T$ then $\varphi:\mathfrak{S}(T)$ 
 
  \item if $\varphi:\mathfrak{S}(T)$ and $\varphi$ is in the range of `$\mathrm{store}$', then
    $\mathrm{store}(\varphi):\mathfrak{S}(T)$

  \item if $\varphi:\mathfrak{S}(T)$ and `x$_i$' and $\varphi$ are appropriate
    arguments to `$\mathrm{retrieve}$', then
    $\mathrm{retrieve}(\text{x}_i,\varphi):\mathfrak{S}(T)$
    
  \item nothing is a witness for $\mathfrak{S}(T)$ except as required above.
 
  \end{enumerate} 
  
 
\end{subex} 
\label{ex:storage-type}   
\end{ex}

If $\varphi$ is a parametric content, that is,
$\varphi:\textit{Cont}$, we use the notation $\varphi^{\mathfrak{S}}$
to represent $\mathfrak{S}(\textit{Cont}_{\varphi})$, that is, the type
whose witnesses are the closure of $\{\varphi\}$ under
`$\mathrm{store}$' and `$\mathrm{retrieve}$'.

How should such types of parametric contents be combined in
compositional semantics? First, the value in the `cont'-field in a
sign will now not be a parametric content as previously but a type of
parametric contents, that is, it will be of type \textit{ContType}.
Thus we redefine the type \textit{Sign} as in \nexteg{}.
\begin{ex} 
\begin{subex} 
 
\item \textit{Sign}, ``the type of signs'', is a basic type 
 
\item $\sigma$ : \textit{Sign} iff $\sigma$ :
      \record{\tfield{s-event}{\textit{SEvent}} \\
         \tfield{syn}{\textit{Syn}} \\
        \tfield{cont}{\textit{ContType}}}  
 
\end{subex} 
   
\end{ex}
  
  
Suppose that $T_1$ and
$T_2$ are of type \textit{ContType} and that $\mathcal{O}$ is a combination
operation such as @, @@, @$_{\text{wh}_{i,j}}$, then we say $T_1\mathcal{O}^{\mathfrak{S}}T_2$
is also a type with the witness condition in \nexteg{}.
\begin{ex} 
If $\alpha:T_1$, $\beta:T_2$ and $\alpha\mathcal{O}\beta$ is defined,
then $\alpha\mathcal{O}\beta:T_1\mathcal{O}^{\mathfrak{S}}T_2$.
Nothing else is a witness for $T_1\mathcal{O}^{\mathfrak{S}}T_2$. 
\end{ex} 
What we need then for the type of parametric contents for the combined
constituents is
$\mathfrak{S}(T_1\mathcal{O}^{\mathfrak{S}}T_2)$, that
is, the type of the closure of the set of all combinations under
`$\mathrm{store}$' and `$\mathrm{retrieve}$'.  Note that \preveg{} has
the consequence that if `$\alpha\mathcal{O}\beta$' is undefined for all
witnesses, $\alpha$ and $\beta$ of $T_1$ and $T_2$ respectively we
will still obtain a result for $T_1\mathcal{O}^{\mathfrak{S}}T_2$, albeit a
type which has no witnesses.  This will mean that we do not have to be
as careful in keeping track of typing when interpreting syntactic
constructions, though with the consequence that some phrases will not
have any content.

In \nexteg{} we introduce versions of `ContForwardApp' operations which
apply at the content type level.
\begin{ex} 
If $\mathcal{O}$ is one of @, @@, @$_{\text{wh}_{i,j}}$ (for some
natural numbers $i$ and $j$) or @$_{\&}$, then
ContForwardApp$_{\mathfrak{S},\mathcal{O}}$ is
\begin{quote}
  $\lambda u$:\smallrecord{
    \smalltfield{cont}{\textit{ContType}}}$^\frown$\smallrecord{
    \smalltfield{cont}{\textit{ContType}}} . \smallrecord{
    \smalltfield{cont}{$\mathfrak{S}(u[0].\text{cont}\mathcal{O}^{\mathfrak{S}}u[1].\text{cont})$}}
\end{quote}

\end{ex} 
Note that we now no longer have to keep track of the arguments to
ContForwardApp that we had in some of the variants since the
combination of the types will always return a result, though if the
types of contents do not match for the particular combination
operation the combined type will have no witnesses.  This means that
we can simplify our notation for constituent structure rules as in
\nexteg{}.
\begin{ex} 
If $T_{\text{mother}}$, $T_{\text{daughter}_1}$ and
  $T_{\text{daughter}_2}$ are sign types and $\mathcal{O}$ is a
  combination operation, then
  \begin{quote}
    $T_{\text{mother}}\longrightarrow
    T_{\text{daughter}_1}\ T_{\text{daughter}_2}\ \mid\
    T_{\text{daughter}_1}'(_{\mathcal{O}}T_{\text{daughter}_2}')$
  \end{quote}
  is
  \begin{quote}
  $T_{\text{mother}}\longrightarrow
    T_{\text{daughter}_1}\ T_{\text{daughter}_2}$ \d{\d{$\wedge$}}
    ContForwardApp$_{\mathfrak{S},\mathcal{O}}$
  \end{quote}
  
\end{ex} 
  
  
% @@

% Finally, we show one way in which stored quantifiers can be
% ``percolated'' to higher constituents by showing in \nexteg{} how to
% construct an unplugged content from two constituent unplugged contents
% when the basic semantic composition involves function application.  To
% facilitate this we first define the `x'-incrementation of an object
% (such as a set or a function) with respect to a set of records $R$.

% \begin{ex} 
% The `x'\textit{-incrementation of an object $O$ with respect
%   to a set of records $R$}, $[O]_{\mathrm{incr}_x(R)}$, is the result of
% replacing each instance of x$_i$ in $O$, for any natural number $i$,
% with x$_{\mathrm{incr}_x(R)+i}$.  
% \end{ex} 
% Now application for unplugged contents (that is, objects of type
% \textit{UInterp}) can be characterized as in \nexteg{}.

% \begin{ex} 
% \textbf{Application}

% If $\alpha$ and $\beta$ are of type \textit{UInterp} and $\beta$.core
% is in the domain of $\alpha$.core, then the \textit{application of
%   $\alpha$ to $\beta$}, $\alpha @@\beta$, is
% \begin{quote}

% \record{\field{quants}{$\alpha$.quants$\cup$[$\beta$.quants]$_{\mathrm{incr}(\alpha\mathrm{.quants})}$}\\
%         \field{core}{$\alpha$.core@[$\beta$.core]$_{\mathrm{incr}(\alpha\mathrm{.quants})}$}}
% \end{quote}
% \end{ex} 
% \preveg{} says that the application of an unplugged content, $\alpha$,
% to another unplugged content, $\beta$, involves first changing the
% quantifier indices in $\beta$ so that they increment the quantifier
% indices in $\alpha$ in the way that is illustrated in the examples
% discussed above.  For example, if both $\alpha$ and $\beta$ contain
% the quantifier index `x$_0$' and this is the maximum in $\alpha$, then
% `x$_0$' will be changed to `x$_1$' in $\beta$.  Now the result of
% application is an unplugged content whose `quants' are the union of
% $\alpha$'s `quants' and the incremented `quants' of $\beta$.  The core
% is the result of applying $\alpha$'s core to $\beta$'s core using the
% same incrementation. 
  
\section{Anaphora}
\label{sec:anaph}

We will treat anaphora by adding to the storage mechanism we have just
introduced.  In informal terms the idea is that if the content type of
an utterance has something corresponding to \nexteg{a} as a witness
then it will also have something corresponding to \nexteg{b} as a
witness where $x_1$ has been anaphorically related to $x_0$.
\begin{ex} 
\begin{subex} 
 
\item $x_0$ thinks that $x_1$ has succeeded 
 
\item $x_0$ thinks that $x_0$ has succeeded 
 
\end{subex} 
   
\end{ex} 
Thus in general the content type yielded by the grammatical resources
will be underspecified as to whether there is anaphora or not but will
nevertheless delimit what the anaphoric possibilities are.  Anaphora
will be accounted for at the point of combination.  This is
illustrated schematically in \nexteg{}.
\begin{ex} 
\begin{subex} 
 
\item $x_o$ + thinks that $x_1$ has succeeded = $x_0$ thinks that
  $x_1$ has succeeded 
 
\item If `$x_o$ + thinks that $x_1$ has succeeded' is an
  interpretation, then `$x_o$ + (thinks that $x_1$ has
  succeeded)[$x_1\leadsto x_0$]' is an interpretation
 
\end{subex} 
   
\end{ex} 
We will, of course, not be implementing this in terms of replacing
variables as in \preveg{} but rather in adjusting the contexts of
interpretation associated with pronoun utterances.  For example, we
will define a variant of the combination operation `@', `@$_{i,j}$' which
anaphorically relates a pronoun associated with the context path
`$\mathfrak{s}.j$' to one associated with the context path
`$\mathfrak{s}.i$'.  This is given in \nexteg{}.

\begin{ex} 
If $\mathcal{O}$ is a combination operator, then so is
$\mathcal{O}_{i,j}$, where $i$ and $j$ are natural numbers.

If
$\alpha$ and $\beta$ are parametric contents such that
$\alpha\mathcal{O}\beta$ is defined, $\alpha.\text{bg}\sqsubseteq$
\smallrecord{
  \smalltfield{$\mathfrak{s}$}{\smallrecord{
      \smalltfield{x$_i$}{\textit{Ind}}}}}
and $\mathrm{incr}(\beta.\text{bg},\alpha.\text{bg})\sqsubseteq$
\smallrecord{
  \smalltfield{$\mathfrak{s}$}{\smallrecord{
      \smalltfield{x$_j$}{\textit{Ind}}}}}
but $\mathrm{incr}(\beta.\text{bg},\alpha.\text{bg})\not\sqsubseteq$
\smallrecord{
  \smalltfield{$\mathfrak{q}$}{\smallrecord{
      \smalltfield{x$_j$}{\textit{PQuant}}}}},
then $\alpha\mathcal{O}_{i,j}\beta$ is
\begin{quote}
  $[\alpha\mathcal{O}\beta]_{\mathfrak{s}.\text{x}_j\leadsto\mathfrak{s}.\text{x}_i}$
\end{quote}

    
    


% If $\alpha$ : \smallrecord{\smalltfield{bg}{\textit{CntxtType}}\\
%                            \smalltfield{fg}{(bg$\rightarrow$($T_1\rightarrow
%                              T_2$))}} 
% and $\beta$ : \smallrecord{\smalltfield{bg}{\textit{CntxtType}}\\
%   \smalltfield{fg}{(bg$\rightarrow T_1$)}}
% \fbox{and $\alpha.\text{bg}\sqsubseteq$ \smallrecord{
%     \smalltfield{$\mathfrak{s}$}{\smallrecord{
%         \smalltfield{x$_i$}{\textit{Ind}}}}} and
%   $\mathrm{incr}_{\mathfrak{s}.\text{x},\mathfrak{q}.\text{x}}(\beta.\text{bg},\alpha.\text{bg})\sqsubseteq$
%   \smallrecord{
%     \smalltfield{$\mathfrak{s}$}{\smallrecord{
%         \smalltfield{x$_j$}{\textit{Ind}}}}}} \fbox{but $\mathrm{incr}_{\mathfrak{s}.\text{x},\mathfrak{q}.\text{x}}(\beta.\text{bg},\alpha.\text{bg})\not\sqsubseteq$
%   \smallrecord{
%     \smalltfield{$\mathfrak{q}$}{\smallrecord{
%         \smalltfield{x$_j$}{\textit{PQuant}}}}}},
%                          then the \textit{combination of $\alpha$ and
%     $\beta$  based on functional application \fbox{and anaphoric
%       relation of $j$ to $i$}}, $\alpha\text{@}_{\fbox{i,j}}\beta$, is
%   \begin{quote}
%     $\ulcorner\lambda c$:$[\alpha.\text{bg}]_{\mathfrak{c}\leadsto\mathfrak{c}.\text{f}}$
%       \d{$\wedge$}$[\mathrm{incr}_{\mathfrak{s}.\text{x},\mathfrak{q}.\text{x}}([\beta.\text{bg}]_{\mathfrak{c}\leadsto\mathfrak{c}.\text{a}},\alpha.\text{bg})]_{\boxed{\scriptstyle\mathfrak{s}.\text{x}_j\leadsto\mathfrak{s}.\text{x}_i}}$
%       . \\ \hspace*{2em}$[\alpha]_{\mathfrak{c}\leadsto\mathfrak{c}.\text{f}}(c)([\mathrm{incr}_{\mathfrak{s}.\text{x},\mathfrak{q}.x}([\beta.\text{fg}]_{\mathfrak{c}\leadsto\mathfrak{c}.\text{a}},\alpha.\text{bg})]_{\boxed{\scriptstyle\mathfrak{s}.\text{x}_j\leadsto\mathfrak{s}.\text{x}_i}}(c))\urcorner$
% \end{quote}

\label{ex:combine-align}

\end{ex} 
% We can define similar modifications, $\mathcal{O}_{i,j}$ for the other combination
% operators, $\mathcal{O}$.  Note that these additional combination
% operations as currently formulated will only allow one anaphoric
% relation to be introduced with each combination.  A more complete
% treatment will probably need a generalized formulation in which there
% are several pairs, $\langle i,j\rangle$, which are related
% simultaneously.

We can now add contents with anaphora to
our characterization of $\mathfrak{S}(T)$ given in
(\ref{ex:storage-type}) as in \nexteg{} where we again use boxing to
indicate the new material.
\begin{ex} 
\begin{subex} 
 
\item If $T:\textit{ContType}$, then $\mathfrak{S}(T)$ is a type 
 
\item The witnesses of $\mathfrak{S}(T)$ are characterized by
  \begin{enumerate} 
 
  \item if $\varphi:T$ then $\varphi:\mathfrak{S}(T)$

    
  \item \fbox{\begin{minipage}[t]{.85\linewidth}if $\alpha\mathcal{O}\beta:\mathfrak{S}(T)$, (for
      some combination operation, $\mathcal{O}$) and
      $\alpha\mathcal{O}_{i,j}\beta$ is defined (for some natural
      numbers, $i$ and $j$), then $\alpha\mathcal{O}_{i,j}\beta:\mathfrak{S}(T)$\end{minipage}}
 
  \item if $\varphi:\mathfrak{S}(T)$ and $\varphi$ is in the range of `$\mathrm{storage}$', then
    $\mathrm{storage}(\varphi):\mathfrak{S}(T)$

  \item if $\varphi:\mathfrak{S}(T)$ and `x$_i$' and $\varphi$ are appropriate
    arguments to `$\mathrm{retrieve}$', then
    $\mathrm{retrieve}(\text{x}_i,\varphi):\mathfrak{S}(T)$
    
  \item nothing is a witness for $\mathfrak{S}(T)$ except as required above.
 
  \end{enumerate} 
  
 
\end{subex} 
\label{ex:storage-anaph-type}   
\end{ex}
% \todo{This doesn't allow for more than one anaphoric relation
%   introduced with some combination.  Problem with some interpretation
%   just happening to be identical with $\alpha\mathcal{O}\beta$
%   although obtained differently.}

We will now take some examples of key anaphoric phenomena and discuss
how we could use these tools to account for them.



\paragraph{\textit{No girl thinks she failed}}
\label{sec:direct-binding}

Given the strategy we suggested in Section~\ref{sec:unbound} for
interpreting unbound pronouns  the foreground of a content for
\textit{she failed} would be parallel to example
(\ref{ex:he-leave-abbrev}c) as in \nexteg{}, where we in addition
express this as the content type obtained by $\mathfrak{S}$.
\begin{ex}
  

$\ulcorner\lambda c$:\smallrecord{\footnotesize{\textit{Cntxt}}\\
                \smalltfield{$\mathfrak{s}$}{\smallrecord{
                    \smalltfield{x$_0$}{\textit{Ind}}}}\\
                \smalltfield{$\mathfrak{c}$}{\smallrecord{
                    \smalltfield{f}{\textit{PropCntxt}}\\
                    \smalltfield{a}{\textit{PropCntxt}}}}
              }
  . 
         \record{\tfield{e}{fail($c.\mathfrak{s}$.x$_0$)}}$\urcorner^{\mathfrak{S}}$
% $\lambda\mathfrak{s}$:\smallrecord{\smalltfield{x$_0$}{\textit{Ind}}} . \record{\tfield{e}{fail($\mathfrak{s}$.x$_0$)}}

\end{ex} 
Call this \textbf{she$^\frown$failed}.  Then the 
content type for \textit{thinks she failed} is \nexteg{}.
\begin{ex}
  $\ulcorner\lambda c$:\smallrecord{\footnotesize{\textit{Cntxt}}\\
                \smalltfield{$\mathfrak{s}$}{\smallrecord{
                    \smalltfield{x$_0$}{\textit{Ind}}}}\\
                \smalltfield{$\mathfrak{c}$}{\smallrecord{
                    \smalltfield{f}{\textit{PropCntxt}}\\
                    \smalltfield{a}{\smallrecord{
                        \smalltfield{f}{\textit{PropCntxt}}\\
                        \smalltfield{a}{\textit{PropCntxt}}}}}}
              }
              . $\ulcorner\lambda r$:\smallrecord{
                \smalltfield{x}{\textit{Ind}}} . 
              \record{\tfield{e}{think($r$.x, \smallrecord{
                    \smalltfield{e}{fail($c.\mathfrak{s}$.x$_0$)}})}}$\urcorner\urcorner^{\mathfrak{S}}$
% $\lambda\mathfrak{s}$:\smallrecord{\smalltfield{x$_0$}{\textit{Ind}}}
%   . $\lambda r$:\smallrecord{\smalltfield{x}{\textit{Ind}}}
%     . \record{\tfield{e}{think($r$.x,
%         \textbf{she$^\frown$failed}($\mathfrak{s}$))}}
\label{ex:thinks-she-failed-x0} 
\end{ex} 
Call this \textbf{thinks$^\frown$she$^\frown$failed}. Here
\textit{she} is still a free occurrence of a pronoun dependent on the
context for resolution.  % An alternative interpretation is \nexteg{},
% where the pronoun has become bound as the subject of the property.
% \begin{ex} 
% $\lambda\mathfrak{s}$:\textit{Rec} . $\lambda
% r$:\smallrecord{\smalltfield{x}{\textit{Ind}}}
% . \record{\tfield{e}{think($r$.x, \textbf{she$^\frown$failed}($\mathfrak{s}\oplus$\smallrecord{\field{x$_0$}{$r$.x}}))}} 
% \end{ex} 
% We will call this a logophoric interpretation of the verb phrase,
% \textbf{thinks$^\frown$she$^\frown$failed$_{\mathrm{lg}}$}. These two
% readings for the verb-phrase yields two alternatives for the complete
% sentence, one where the pronoun remains unbound as in \nexteg{a} and
% one where it is bound as in \nexteg{b}.
% \begin{ex} 
% \begin{subex} 
 
% \item
%   $\lambda\mathfrak{s}$:\smallrecord{\smalltfield{x$_0$}{\textit{Ind}}}
%     . \record{\tfield{e}{no(girl$'$, \textbf{thinks$^\frown$she$^\frown$failed$_{x_0}$}($\mathfrak{s}$))}} 
 
% \item $\lambda\mathfrak{s}$:\textit{Rec}
%   . \record{\tfield{e}{no(girl$'$, \textbf{thinks$^\frown$she$^\frown$failed$_{\mathrm{lg}}$}($\mathfrak{s}$))}} 
 
% \end{subex} 
   
% \end{ex} 

% Let us now consider an option where we store the interpretation of
% \textit{no girl}.  For this we ill use the abbreviatory notation
% \nexteg{b} for \nexteg{a}
% \begin{ex} 
% \begin{subex} 
 
% \item \record{\field{bg}{$T_1$}\\
%               \field{fg}{$\lambda\mathfrak{v}$:$T_1$ . $T_2(\mathfrak{v})$}} 
 
% \item $\ulcorner\lambda\mathfrak{v}$:$T_1$ . $T_2(\mathfrak{v})\urcorner$
 
% \end{subex} 
   
% \end{ex} 
% [????We should introduce this notation from the beginning, starting
% p. 121 ``Parametric contents as we have presented them so far are
% problematic...''] If $\frec{f}$ is a record as
% characterized in \preveg{} then we use the notation $\frec{f}(a)$ to
% represent $\frec{f}.\text{fg}(a)$, that is, $f(a)$.

The content type associated with \textit{no girl} will be \nexteg{}.
\begin{ex} 
  $\ulcorner\lambda c$:\smallrecord{
    \footnotesize{\textit{Cntxt}}\\
    \smalltfield{$\mathfrak{c}$}{\smallrecord{
        \smalltfield{f}{\textit{PropCntxt}}\\
        \smalltfield{a}{\textit{PropCntxt}}}}} . $\lambda
  P$:\textit{Ppty} . \smallrecord{
    \smallmfield{restr}{girl$'$}{\textit{Ppty}}\\
    \smallmfield{scope}{$P$}{\textit{Ppty}}\\
    \smalltfield{e}{no(restr, scope)}}$\urcorner^{\mathfrak{S}}$
  \label{ex:no-girl-cont-type}
\end{ex}
Let us represent the \textit{generator} of this type, that is,
\nexteg{}, by `no$'$(girl$'$)'.
\begin{ex} 
  $\ulcorner\lambda c$:\smallrecord{
    \footnotesize{\textit{Cntxt}}\\
    \smalltfield{$\mathfrak{c}$}{\smallrecord{
        \smalltfield{f}{\textit{PropCntxt}}\\
        \smalltfield{a}{\textit{PropCntxt}}}}} . $\lambda
  P$:\textit{Ppty} . \smallrecord{
    \smallmfield{restr}{girl$'$}{\textit{Ppty}}\\
    \smallmfield{scope}{$P$}{\textit{Ppty}}\\
    \smalltfield{e}{no(restr, scope)}}$\urcorner$ 
\end{ex}
This means that one witness for the type (\ref{ex:no-girl-cont-type})
is \nexteg{} where `no$'$(girl$'$)' has been stored.
\begin{ex} 
$\ulcorner\lambda c$:\smallrecord{
    \footnotesize{\textit{Cntxt}}\\
    \smalltfield{$\mathfrak{q}$}{\smallrecord{
        \smallmfield{x$_0$}{no$'$(girl$'$)}{\textit{PQuant}}}}\\
    \smalltfield{$\mathfrak{s}$}{\smallrecord{
        \smalltfield{x$_0$}{\textit{Ind}}}}} . $\lambda
  P$:\textit{Ppty} . $P\{c.\mathfrak{s}.\text{x}_0\}\urcorner$ 
\end{ex} 
Note that the context type for this parametric content has the path
`$\mathfrak{s}$.x$_0$' which means that it is available for anaphoric
version of combination operations, thus enabling \textit{she} in
\textit{thinks she failed} to be related to
`$\mathfrak{s}$.x$_0$'. This can be achieved by the combination of
parametric contents expressed in \nexteg{a} which is identical with
\nexteg{b}.  This uses `@$_{0,1}$' as characterized in
(\ref{ex:combine-align}).
\begin{ex} 
\begin{subex} 
 
\item $\ulcorner\lambda c$:\smallrecord{
    \footnotesize{\textit{Cntxt}}\\
    \smalltfield{$\mathfrak{q}$}{\smallrecord{
        \smallmfield{x$_0$}{no$'$(girl$'$)}{\textit{PQuant}}}}\\
    \smalltfield{$\mathfrak{s}$}{\smallrecord{
        \smalltfield{x$_0$}{\textit{Ind}}}}} . $\lambda
  P$:\textit{Ppty} . $P\{c.\mathfrak{s}.\text{x}_0\}\urcorner$
  
  @$_{0,1}$

  $\ulcorner\lambda c$:\smallrecord{\footnotesize{\textit{Cntxt}}\\
    \smalltfield{$\mathfrak{s}$}{\smallrecord{
        \smalltfield{x$_0$}{\textit{Ind}}}}\\
    \smalltfield{$\mathfrak{c}$}{\smallrecord{
        \smalltfield{f}{\textit{PropCntxt}}\\
        \smalltfield{a}{\smallrecord{
            \smalltfield{f}{\textit{PropCntxt}}\\
            \smalltfield{a}{\textit{PropCntxt}}}}}}
  }
  . $\ulcorner\lambda r$:\smallrecord{
    \smalltfield{x}{\textit{Ind}}} . 
  \record{\tfield{e}{think($r$.x, \smallrecord{
        \smalltfield{e}{fail($c.\mathfrak{s}$.x$_0$)}})}}$\urcorner\urcorner$
  
\item $\ulcorner\lambda c$:\smallrecord{
    \footnotesize{\textit{Cntxt}}\\
    \smalltfield{$\mathfrak{q}$}{\smallrecord{
        \smallmfield{x$_0$}{no$'$(girl$'$)}{\textit{PQuant}}}}\\
    \smalltfield{$\mathfrak{s}$}{\smallrecord{
        \smalltfield{x$_0$}{\textit{Ind}}}}\\
    \smalltfield{$\mathfrak{c}$}{\smallrecord{
        \smalltfield{f}{\textit{PropCntxt}}\\
        \smalltfield{a}{\smallrecord{
            \smalltfield{f}{\textit{PropCntxt}}\\
            \smalltfield{a}{\smallrecord{
                \smalltfield{f}{\textit{PropCntxt}}\\
                \smalltfield{a}{\textit{PropCntxt}}}}}}}}} . \record{
    \tfield{e}{think($c.\mathfrak{s}.\text{x}_0$, \smallrecord{
        \smalltfield{e}{fail($c.\mathfrak{s}.\text{x}_0$)}})}}$\urcorner$
           
 
\end{subex} 
   
\end{ex} 
  
Application of `$\mathrm{retrieve}$' to the content \preveg{} will obtain a content where
the scope of the quantifier is the property of ``being a girl who
thinks she (the girl) failed''.   The whole content is given in
\nexteg{} where \nexteg{a--c} are identical.
\begin{ex} 
\begin{subex} 
 
\item $\ulcorner\lambda c$:\smallrecord{
    \footnotesize{\textit{Cntxt}}\\
    \smalltfield{$\mathfrak{c}$}{\smallrecord{
        \smalltfield{f}{\textit{PropCntxt}}\\
        \smalltfield{a}{\smallrecord{
            \smalltfield{f}{\textit{PropCntxt}}\\
            \smalltfield{a}{\smallrecord{
                \smalltfield{f}{\textit{PropCntxt}}\\
                \smalltfield{a}{\smallrecord{
                    \smalltfield{f}{\textit{PropCntxt}}\\
                    \smalltfield{a}{\textit{PropCntxt}}}}}}}}}}} . \\
  \hspace*{1em}($\lambda P$:\textit{Ppty} . \record{
    \mfield{restr}{girl$'$}{\textit{Ppty}}\\
    \mfield{scope}{$P|_{\mathcal{F}(\text{girl}')}$}{\textit{Ppty}}\\
    \tfield{e}{no(restr, scope)}} \\
  \hspace*{2em}($\mathfrak{P}$($\ulcorner\lambda r$:\smallrecord{
    \smalltfield{x}{\textit{Ind}}\\
    \smalltfield{$\mathfrak{s}$}{\smallrecord{
        \footnotesize{\textit{Assgnmnt}}\\
        \smallmfield{x$_0$}{$\Uparrow$x}{\textit{Ind}}}}} . \record{
    \tfield{e}{think($r.\mathfrak{s}.\text{x}_0$, \smallrecord{
        \smalltfield{e}{fail($r.\mathfrak{s}.\text{x}_0$)}})}}$\urcorner$)))$\urcorner$
 
\item  $\ulcorner\lambda c$:\smallrecord{
    \footnotesize{\textit{Cntxt}}\\
    \smalltfield{$\mathfrak{c}$}{\smallrecord{
        \smalltfield{f}{\textit{PropCntxt}}\\
        \smalltfield{a}{\smallrecord{
            \smalltfield{f}{\textit{PropCntxt}}\\
            \smalltfield{a}{\smallrecord{
                \smalltfield{f}{\textit{PropCntxt}}\\
                \smalltfield{a}{\smallrecord{
                    \smalltfield{f}{\textit{PropCntxt}}\\
                    \smalltfield{a}{\textit{PropCntxt}}}}}}}}}}} . \\
  \hspace*{1em}($\lambda P$:\textit{Ppty} . \record{
    \mfield{restr}{girl$'$}{\textit{Ppty}}\\
    \mfield{scope}{$P|_{\mathcal{F}(\text{girl}')}$}{\textit{Ppty}}\\
    \tfield{e}{no(restr, scope)}} \\
  \hspace*{2em}($\ulcorner\lambda r$:\smallrecord{
    \smalltfield{x}{\textit{Ind}}} . \smallrecord{
    \smalltfield{$\mathfrak{c}$}{\smallrecord{
        \smalltfield{$\mathfrak{s}$}{\smallrecord{
            \footnotesize{\textit{Assgnmnt}}\\
            \smallmfield{x$_0$}{$r$.x}{\textit{Ind}}}}}}\\
    \smalltfield{e}{think($\mathfrak{c}.\mathfrak{s}.\text{x}_0$, \smallrecord{
        \smalltfield{e}{fail($\mathfrak{c}.\mathfrak{s}.\text{x}_0$)}})}}$\urcorner$))$\urcorner$
  \hfill (purification)

  
\item $\ulcorner\lambda c$:\smallrecord{
    \footnotesize{\textit{Cntxt}}\\
    \smalltfield{$\mathfrak{c}$}{\smallrecord{
        \smalltfield{f}{\textit{PropCntxt}}\\
        \smalltfield{a}{\smallrecord{
            \smalltfield{f}{\textit{PropCntxt}}\\
            \smalltfield{a}{\smallrecord{
                \smalltfield{f}{\textit{PropCntxt}}\\
                \smalltfield{a}{\smallrecord{
                    \smalltfield{f}{\textit{PropCntxt}}\\
                    \smalltfield{a}{\textit{PropCntxt}}}}}}}}}}} . \\
  \hspace*{1em}\record{
    \mfield{restr}{girl$'$}{\textit{Ppty}}\\
    \mfield{scope}{$\ulcorner\lambda r$:\smallrecord{
        \smalltfield{x}{\textit{Ind}}\\
        \smalltfield{e}{girl(x)}} . \smallrecord{
        \smalltfield{$\mathfrak{c}$}{\smallrecord{
            \smalltfield{$\mathfrak{s}$}{\smallrecord{
                \footnotesize{\textit{Assgnmnt}}\\
                \smallmfield{x$_0$}{$r$.x}{\textit{Ind}}}}}}\\
        \smalltfield{e}{think($\mathfrak{c}.\mathfrak{s}.\text{x}_0$, \smallrecord{
            \smalltfield{e}{fail($\mathfrak{c}.\mathfrak{s}.\text{x}_0$)}})}}$\urcorner$}{\textit{Ppty}}\\
    \tfield{e}{no(restr, scope)}}$\urcorner$\\
  \hfill ($\beta$-reduction, property restriction)
 
\end{subex} 
   
\end{ex} 


% Interpreting \textit{no girl} with storage yields \nexteg{}.
% \begin{ex} 
% \record{\field{quants}{\{$\ulcorner\lambda\mathfrak{s}$:\smallrecord{\smalltfield{x$_1$}{\textit{Ind}}}
%       . $\lambda P$:\textit{Ppty} . \record{\tfield{e}{no(girl$'$,
%           $P$(\smallrecord{\field{x}{$\mathfrak{s}$.x$_1$}}))}}$\urcorner$\}}\\
% \field{core}{$\ulcorner\lambda\mathfrak{s}$:\smallrecord{\smalltfield{x$_1$}{\textit{Ind}}}
%   . $\lambda P$:\textit{Ppty} . $P$(\smallrecord{\smalltfield{x}{$\mathfrak{s}$.x$_1$}})$\urcorner$}} 
% \end{ex} 
% Then \textit{no girl thinks she failed} yields \nexteg{}.
% \begin{ex} 
%  \record{\field{quants}{\{$\ulcorner\lambda\mathfrak{s}$:\smallrecord{\smalltfield{x$_1$}{\textit{Ind}}}
%       . $\lambda P$:\textit{Ppty} . \record{\tfield{e}{no(girl$'$,
%           $P$(\smallrecord{\field{x}{$\mathfrak{s}$.x$_1$}})$|_{\mathcal{F}(\text{girl}')}$)}}$\urcorner$\}}\\
% \field{core}{$\ulcorner\lambda\mathfrak{s}$:\smallrecord{\smalltfield{x$_1$}{\textit{Ind}}}
%       . \record{\tfield{e}{\textbf{thinks$^\frown$she$^\frown$failed$_{x_0}$}($\mathfrak{s}\oplus$\smallrecord{\field{x$_0$}{$\mathfrak{s}$.x$_1$}})}}$\urcorner$}}
% \end{ex} 
% \preveg{} uses the definition of
% \textbf{thinks$^\frown$she$^\frown$failed$_{x_0}$} given in
% (\ref{ex:thinks-she-failed-x0}).  Let us call the value of the
% `core'-field in \preveg{}
% \textbf{thinks$^\frown$she$^\frown$failed$_{x_1}$}.  Then the result
% of applying retrieval to \preveg{} is \nexteg{}.
% \begin{ex} 
% \record{\field{quants}{\{\}}\\
%         \field{core}{$\ulcorner\lambda\mathfrak{s}$:\textit{Rec}
%           . \smallrecord{\smalltfield{e}{no(girl$'$, $\lambda
%               r$:\smallrecord{\smalltfield{x$_1$}{\textit{Ind}}\\
%                               \smalltfield{e}{girl(x$_1$)}} .  \textbf{thinks$^\frown$she$^\frown$failed$_{x_1}$}($\mathfrak{s}\oplus$\smallrecord{\field{x$_1$}{$r$.x$_1$}})})}$\urcorner$}}
% \end{ex} 
     
\paragraph{\textit{A man walked. He whistled.} }
\label{sec:discourse-anaph}
Given our strategy for defining the content of quantified sentences in
terms of generalized quantifiers a witness for the content type of \textit{a man walked}
will be \nexteg{}.
\begin{ex} 
  $\ulcorner\lambda c$:\smallrecord{
    \footnotesize{\textit{Cntxt}}\\
    \smalltfield{$\mathfrak{c}$}{\smallrecord{
        \smalltfield{f}{\smallrecord{
            \smalltfield{f}{\textit{PropCntxt}}\\
            \smalltfield{a}{\textit{PropCntxt}}}}\\
        \smalltfield{a}{\textit{PropCntxt}}}}} . \record{
    \mfield{restr}{man$'$}{\textit{Ppty}}\\
    \mfield{scope}{walk$'|_{\mathcal{F}(\text{restr})}$}{\textit{Ppty}}\\
    \tfield{e}{exist(restr, scope)}}$\urcorner$ 
\end{ex}
We know from our treatment of the witness conditions associated with
`exist' that \preveg{} is equivalent to \nexteg{}.
\begin{ex} 
  $\ulcorner\lambda c$:\smallrecord{
    \footnotesize{\textit{Cntxt}}\\
    \smalltfield{$\mathfrak{c}$}{\smallrecord{
        \smalltfield{f}{\smallrecord{
            \smalltfield{f}{\textit{PropCntxt}}\\
            \smalltfield{a}{\textit{PropCntxt}}}}\\
        \smalltfield{a}{\textit{PropCntxt}}}}} . \\
  \hspace*{4em}\record{
    \mfield{restr}{man$'$}{\textit{Ppty}}\\
    \mfield{scope}{walk$'|_{\mathcal{F}(\text{restr})}$}{\textit{Ppty}}\\
    \tfield{e}{\record{
        \tfield{x}{$\mathfrak{T}$($\Uparrow$restr)}\\
        \tfield{e}{$\mathfrak{P}$($\Uparrow$scope)\{x\}}}}}$\urcorner$
  \label{ex:a-man-walked-derived}
\end{ex} 
Using our previous treatment for free pronouns, \textit{he whistled}
will have \nexteg{} as a witness of its content type.
\begin{ex}
  $\ulcorner\lambda c$:\smallrecord{
    \footnotesize{\textit{Cntxt}}\\
    \smalltfield{$\mathfrak{s}$}{\smallrecord{
        %\footnotesize{\textit{Assgnmnt}}\\
        \smalltfield{x$_0$}{\textit{Ind}}}}\\
    \smalltfield{$\mathfrak{c}$}{\smallrecord{
        \smalltfield{f}{\textit{PropCntxt}}\\
        \smalltfield{a}{\textit{PropCntxt}}}}}
  . 
  \record{\tfield{e}{whistle($c.\mathfrak{s}$.x$_0$)}}$\urcorner$
  % $\lambda\mathfrak{s}$:\smallrecord{\smalltfield{x$_0$}{\textit{Ind}}} . \record{\tfield{e}{whistle($\mathfrak{s}$.x$_0$)}}

  \label{ex:he-whistled-parametric}
\end{ex} 
We will represent \preveg{} as \textbf{he$^{\frown}$whistled}.

The utterance of \textit{He whistled} is to be interpreted in the
context of the previous utterance of \textit{a man walked}.
We will achieve this by merging the quasi-fixed point type (see
Chapter~\ref{ch:commonnouns}, p.~\pageref{ex:quasifixedpointtype}) of
the foreground of the parametric content of the previous utterance with the context type
of the current utterance.  The quasi-fixed point type for the
foreground of 
(\ref{ex:a-man-walked-derived}) is \nexteg{}.
\begin{ex} 
  \record{
    \tfield{$\mathfrak{c}^*$}{\smallrecord{
    \footnotesize{\textit{Cntxt}}\\
    \smalltfield{$\mathfrak{c}$}{\smallrecord{
        \smalltfield{f}{\smallrecord{
            \smalltfield{f}{\textit{PropCntxt}}\\
            \smalltfield{a}{\textit{PropCntxt}}}}\\
        \smalltfield{a}{\textit{PropCntxt}}}}}}\\
    \mfield{restr}{man$'$}{\textit{Ppty}}\\
    \mfield{scope}{walk$'|_{\mathcal{F}(\text{restr})}$}{\textit{Ppty}}\\
    \tfield{e}{\record{
        \tfield{x}{$\mathfrak{T}$($\Uparrow$restr)}\\
        \tfield{e}{$\mathfrak{P}$($\Uparrow$scope)\{x\}}}}} 
\end{ex} 
We will merge this under the label `$\mathfrak{p}$' (``previous'')
into the context type in (\ref{ex:he-whistled-parametric}) yielding \nexteg{}.
\begin{ex} 
  $\ulcorner\lambda c$:\smallrecord{
    \footnotesize{\textit{Cntxt}}\\
    \smalltfield{$\mathfrak{p}$}{\smallrecord{
        \smalltfield{$\mathfrak{c}^*$}{\smallrecord{
    \footnotesize{\textit{Cntxt}}\\
    \smalltfield{$\mathfrak{c}$}{\smallrecord{
        \smalltfield{f}{\smallrecord{
            \smalltfield{f}{\textit{PropCntxt}}\\
            \smalltfield{a}{\textit{PropCntxt}}}}\\
        \smalltfield{a}{\textit{PropCntxt}}}}}}\\
        \smallmfield{restr}{man$'$}{\textit{Ppty}}\\
        \smallmfield{scope}{walk$'|_{\mathcal{F}(\text{restr})}$}{\textit{Ppty}}\\
        \smalltfield{e}{\smallrecord{
            \smalltfield{x}{$\mathfrak{T}$($\Uparrow$restr)}\\
            \smalltfield{e}{$\mathfrak{P}$($\Uparrow$scope)\{x\}}}}}}\\
    \smalltfield{$\mathfrak{s}$}{\smallrecord{
        %\footnotesize{\textit{Assgnmnt}}\\
        \smalltfield{x$_0$}{\textit{Ind}}}}}
  . \record{\tfield{e}{whistle($c.\mathfrak{s}$.x$_0$)}}$\urcorner$
\end{ex} 
\preveg{} makes the content of the previous utterance be part of the
context for content of the current utterance.  It does not, however,
express the anaphoric relation between \textit{he} and \textit{a man}.
In order to do this we need to require that $\mathfrak{s}.\text{x}_0$
in the context is identical with $\mathfrak{p}.\text{e}.\text{s}$.
This can be done by introducing a manifest field under the
`$\mathfrak{s}$'-label as in \nexteg{}.
\begin{ex} 
  $\ulcorner\lambda c$:\smallrecord{
    \footnotesize{\textit{Cntxt}}\\
      \smalltfield{$\mathfrak{p}$}{\smallrecord{
          \smalltfield{$\mathfrak{c}^*$}{\smallrecord{
    \footnotesize{\textit{Cntxt}}\\
    \smalltfield{$\mathfrak{c}$}{\smallrecord{
        \smalltfield{f}{\smallrecord{
            \smalltfield{f}{\textit{PropCntxt}}\\
            \smalltfield{a}{\textit{PropCntxt}}}}\\
        \smalltfield{a}{\textit{PropCntxt}}}}}}\\
          \smallmfield{restr}{man$'$}{\textit{Ppty}}\\
          \smallmfield{scope}{walk$'|_{\mathcal{F}(\text{restr})}$}{\textit{Ppty}}\\
          \smalltfield{e}{\smallrecord{
              \smalltfield{x}{$\mathfrak{T}$($\Uparrow$restr)}\\
              \smalltfield{e}{$\mathfrak{P}$($\Uparrow$scope)\{x\}}}}}}\\
      \smalltfield{$\mathfrak{s}$}{\smallrecord{
          %\footnotesize{\textit{Assgnmnt}}\\
          \smallmfield{x$_0$}{$\Uparrow\!\!\mathfrak{p}$.e.x}{\textit{Ind}}}}}
  . \record{\tfield{e}{whistle($c.\mathfrak{s}$.x$_0$)}}$\urcorner$
\end{ex}
On the basis of \preveg{}, we can create a new function with the same
effect which will have the same domain and return the same results for
each element in the domain but in which any dependency on
`$c.\mathfrak{s}.\text{x}_0$' is replaced by a dependency on
`$c.\mathfrak{p}.\text{e}.\text{x}$' as in \nexteg{}.
\begin{ex} 
  $\ulcorner\lambda c$:\smallrecord{
    \footnotesize{\textit{Cntxt}}\\
      \smalltfield{$\mathfrak{p}$}{\smallrecord{
          \smalltfield{$\mathfrak{c}^*$}{\smallrecord{
    \footnotesize{\textit{Cntxt}}\\
    \smalltfield{$\mathfrak{c}$}{\smallrecord{
        \smalltfield{f}{\smallrecord{
            \smalltfield{f}{\textit{PropCntxt}}\\
            \smalltfield{a}{\textit{PropCntxt}}}}\\
        \smalltfield{a}{\textit{PropCntxt}}}}}}\\
          \smallmfield{restr}{man$'$}{\textit{Ppty}}\\
          \smallmfield{scope}{walk$'|_{\mathcal{F}(\text{restr})}$}{\textit{Ppty}}\\
          \smalltfield{e}{\smallrecord{
              \smalltfield{x}{$\mathfrak{T}$($\Uparrow$restr)}\\
              \smalltfield{e}{$\mathfrak{P}$($\Uparrow$scope)\{x\}}}}}}\\
      \smalltfield{$\mathfrak{s}$}{\smallrecord{
          %\footnotesize{\textit{Assgnmnt}}\\
          \smallmfield{x$_0$}{$\Uparrow\!\!\mathfrak{p}$.e.x}{\textit{Ind}}}}}
  . \record{\tfield{e}{whistle($c.\mathfrak{p}$.e.x)}}$\urcorner$ 
\end{ex} 
Since nothing now depends on the path `$\mathfrak{s}.\text{x}_0$' in
the context and $\mathfrak{T}(\text{man}')$', that is, the type
restriction on the path `$\mathfrak{p}$.e.x', is a subtype of
`\textit{Ind}', the type from which the singleton type on the path
`$\mathfrak{s}.\text{x}_0$' is derived, we can remove the path `$\mathfrak{s}.\text{x}_0$'
without changing the extension of any records that are witnesses for the
context type.  Thus we obtain \nexteg{}.
\begin{ex} 
  $\ulcorner\lambda c$:\smallrecord{
    \footnotesize{\textit{Cntxt}}\\
      \smalltfield{$\mathfrak{p}$}{\smallrecord{
          \smalltfield{$\mathfrak{c}^*$}{\smallrecord{
    \footnotesize{\textit{Cntxt}}\\
    \smalltfield{$\mathfrak{c}$}{\smallrecord{
        \smalltfield{f}{\smallrecord{
            \smalltfield{f}{\textit{PropCntxt}}\\
            \smalltfield{a}{\textit{PropCntxt}}}}\\
        \smalltfield{a}{\textit{PropCntxt}}}}}}\\
          \smallmfield{restr}{man$'$}{\textit{Ppty}}\\
          \smallmfield{scope}{walk$'|_{\mathcal{F}(\text{restr})}$}{\textit{Ppty}}\\
          \smalltfield{e}{\smallrecord{
              \smalltfield{x}{$\mathfrak{T}$($\Uparrow$restr)}\\
              \smalltfield{e}{$\mathfrak{P}$($\Uparrow$scope)\{x\}}}}}}
      }
  . \record{\tfield{e}{whistle($c.\mathfrak{p}$.e.x)}}$\urcorner$ 
\end{ex}

\begin{shaded}
\label{pg:path-alignment-types}(Repeated in
Appendix~\ref{app:path-align-rectype}) Suppose that $T$ is a record type and that $\pi_1$ and $\pi_2$ are
paths in $T$.  Then we use $T_{\pi_1=\pi_2}$ to represent the type
exactly like $T$ except that $T_{\pi_1=\pi_2}.\pi_1=
(T.\pi_1)_{\pi_2}$, that is, whatever type, $T'$, is at the end of the path
$\pi_1$, is replaced by the singleton type $T'_{\pi_2}$, or if
$T.\pi_1$ is
\begin{quote}
$\langle\lambda v_1\!:\!T_1\ldots \lambda v_n\!:\!T_n\
.\ T'\dep{v_1,\ldots,v_n}, \Pi\rangle$
\end{quote}
then it is replaced by
\begin{quote}
$\langle\lambda v_1\!:\!T_1\ldots \lambda v_n\!:\!T_n\ 
.\ (T'\dep{v_1,\ldots,v_n})_{\pi_2}, \Pi\rangle$
\end{quote}
We use $T_{\pi_{11}=\pi_{21},\ldots,\pi_{1n}=\pi_{2n}}$ to
represent $(\ldots(T_{\pi_{11}=\pi_{21}})\ldots)_{\pi_{1n}=\pi_{2n}}$.

Suppose that $\varphi_1$ and $\varphi_2$ and parametric contents,
$\pi_{11}\ldots\pi_{1n}\in\mathrm{paths}(\varphi_1.\text{bg})$ and \\
$\pi_{21}\ldots\pi_{2n}\in\mathrm{paths}(\mathcal{F}_{\text{quasi}^*}(\varphi_2.\text{fg})$,
then the \textit{content $\varphi_1$ given $\varphi_2$ with
  alignment of $\pi_{11}$ and $\pi_{21}$,\ldots,$\pi_{1n}$ and $\pi_{2n}$},
$\varphi_1\ |_{\pi_{11},\pi_{21};\ldots;\pi_{1n},\pi_{2n}}\ \varphi_2$, is
\begin{quote}
  \record{
    \field{bg}{($\varphi_1$.bg \d{$\wedge$} \smallrecord{
        \smalltfield{$\mathfrak{p}$}{$\mathcal{F}_{\text{quasi}^*}(\varphi_2.\text{fg})$}})$_{\pi_{11}=\mathfrak{p}.\pi_{21},\ldots,\pi_{1n}=\mathfrak{p}.\pi_{2n}}$}\\
    \field{fg}{$\lambda c$:bg . $\varphi_1(c)$}}
\end{quote}

This gives us a way of combining two parametric contents.  Now we need
a way of combining the kinds of parametric content types that we are
using for underspecified interpretation.  If $T_1$ and $T_2$ are types
of parametric contents, then there is a combined type,
$\mathfrak{C}(T_1,T_2)$, whose witnesses include witnesses for $T_1$
given a witness for $T_2$ with some possible alignment between the
two.  The witnesses of $\mathfrak{C}(T_1,T_2)$ are characterized
recursively by:
\begin{enumerate} 
 
\item if $\varphi:T_1$, then $\varphi:\mathfrak{C}(T_1,T_2)$ 
 
\item \begin{tabbing}
    if \=$\varphi_1:\mathfrak{C}(T_1,T_2)$,\\
  \>$\pi_{11},\ldots,\pi_{1n}\in\mathrm{paths}(\varphi_1.\text{bg})$,\\ \>$\varphi_2:T_2$ and\\
\>$\pi_{21},\ldots,\pi_{2n}\in\mathrm{paths}(\mathcal{F}_{\text{quasi}^*}(\varphi_2.\text{fg}))$,
  \end{tabbing}
  then
  \begin{quote}
    $\varphi_1|_{\pi_{11},\pi_{21};\ldots;\pi_{1n},\pi_{2n}}\varphi_2:\mathfrak{C}(T_1,T_2)$
  \end{quote}
  
 
\end{enumerate} 
What we need then for the content type of the utterance is
$\mathfrak{S}(\mathfrak{C}(T_1,T_2))$.  We can express this by means
of the action rule in \nexteg{}.
\end{shaded}
\begin{sidewaysfigure}
\begin{ex}
  
  \begin{prooftree}
    \hypo{s_{i,A}:_A\text{\smallrecord{
          \smalltfield{shared}{\smallrecord{
              \smalltfield{latest-utterance}{\smallrecord{
                  \smalltfield{cont}{\textit{ContType}}}}}}}}}
    \hypo{u^*:_A\text{\smallrecord{
          \smalltfield{cont}{\textit{ContType}}}}}
    \infer[enth]2{s_{i+1,A}:_A\text{\smallrecord{
          \smalltfield{shared}{\smallrecord{
              \smalltfield{latest-utterance}{\smallrecord{
                  \smallmfield{cont}{$\mathfrak{S}(\mathfrak{C}(u^*.\text{cont},s_{i,A}.\text{shared}.\text{latest-utterance}.\text{cont}))$}{\textit{ContType}}}}}}}}}
  \end{prooftree}
\label{ex:interp-cntxt-previous-utterance}  
\end{ex} 
\end{sidewaysfigure}  
  



  




% In
% order to obtain the content of the whole discourse, we will embed the
% content of the first sentence below the label `prev' (meaning
% ``previous'') and embed \textbf{he$^{\frown}$whistled} under `e'.
% This is the same technique we used to encode order (and salience) and
% to avoid unwanted label clash in our representation of attitudinal
% states in Chapter~\ref{ch:intensional}.  At the same time we need to
% ensure that the resulting content (on the anaphoric reading of the
% pronoun) does not depend on the context and that the `x$_0$' in
% \textbf{he$^{\frown}$whistled} gets bound to the path `e.x' in the
% content of the first sentence.  The result we obtain is \nexteg{a}
% which, spelling out the function application in the `e'-field, is
% identical with \nexteg{b}.
% \begin{ex}
% \begin{subex} 
 
% \item $\lambda\mathfrak{s}$:\textit{Rec} . 
% \record{\tfield{prev}{\record{\tfield{e}{\record{\tfield{x}{$\mathfrak{T}$(man$'$)}\\
%                                                  \tfield{e}{$\mathfrak{P}$(walk$'_{\mathcal{F}(\text{man}')}$)\{x\}}}}}}\\
%          \tfield{e}{\textbf{he$^{\frown}$whistled}($\mathfrak{s}\oplus$\smallrecord{\field{x$_0$}{$\Uparrow$prev.e.x}})}}   
 
% \item $\lambda\mathfrak{s}$:\textit{Rec} . 
% \record{\tfield{prev}{\record{\tfield{e}{\record{\tfield{x}{$\mathfrak{T}$(man$'$)}\\
%                                                  \tfield{e}{$\mathfrak{P}$(walk$'_{\mathcal{F}(\text{man}')}$)\{x\}}}}}}\\
%          \tfield{e}{\record{\tfield{e}{whistle($\Uparrow$prev.e.x)}}}} 
 
% \end{subex} 
   
 
% \end{ex} 

% Now we must consider how we would construct general rules that would
% combine the parametric content of the discourse so far with the
% parametric content of a new declarative sentence added to the
% discourse.  We will first consider the simple case where no anaphora
% occurs (for example, an interpretation of \textit{A man walked.  He
%   whistled} here \textit{he} is refers deictically and is not
% anaphorically related to \textit{a man}).  We will use
% $\mathcal{T}_{\text{curr}}$ to refer to the current parametric
% content of the discourse so far.  (We are making the simplifying
% assumption that the discourse consists of a string of declarative
% sentence utterances.) We will use $\mathcal{T}_{\text{new}}$ to refer
% to the paramtric content of the new (declarative) sentence with which
% we are updating the content of the discourse.  We will use
% $\mathcal{T}_{\text{curr}}+\mathcal{T}_{\text{new}}$ to represent the
% result of updating $\mathcal{T}_{\text{curr}}$ with
% $\mathcal{T}_{\text{new}}$.  The non-anaphoric update is then defined
% as in \nexteg{}.
% \begin{ex} 
% If $\mathcal{T}_{\text{curr}}$ : $(T_1\rightarrow\textit{Type})$ and
% $\mathcal{T}_{\text{new}}$ : $(T_2\rightarrow\textit{Type})$, then
% $\mathcal{T}_{\text{curr}}+\mathcal{T}_{\text{new}}$ is
% \begin{quote}
% $\lambda\mathfrak{s}$:$(T_1$\d{$\wedge$}$[T_2]_{\text{incr}_x(T_1)})$
% . \record{\tfield{prev}{$\mathcal{T}_{\text{curr}}(\mathfrak{s})$}\\
%           \tfield{e}{$[\mathcal{T}_{\text{new}}]_{\text{incr}_x(T_1)}(\mathfrak{s})$}}
% \end{quote}
% \end{ex} 
% This method of combination is similar to the S-combinator in
% combinatory logic, in that it applies both parametric contents to a
% context, $\mathfrak{s}$.  It abstracts over $\mathfrak{s}$ and makes
% sure that it is of an appropriate type to be an argument to both
% parametric contents.  Rather than applying the first result of
% application to $\mathfrak{s}$ to the second it
% creates a record type involving both the contents.  In addition it
% increments the `x'-labels in the second content so that there will no
% be unintended label clash between the two.

% In the case of making a discourse anaphoric connection two additional things
% have to happen.  The pronoun content within $\mathcal{T}_{\text{new}}$
% has to be connected to some path in $\mathcal{T}_{\text{curr}}$ and
% the updated parametric content
% $\mathcal{T}_{\text{curr}}+\mathcal{T}_{\text{new}}$ must be made not
% to depend on the context to determine the pronoun content.  We first
% give a definition which will allow one discourse anaphoric connection
% and we will then generalize this to a set of anaphoric connections.
% The definition given in \preveg{} will be a specific case of this
% general definition, where the set of anaphoric connections is empty.
% The case of a single anaphoric connection is given in \nexteg{}.
% \begin{ex} 
% If $\mathcal{T}_{\text{curr}}$ : $(T_1\rightarrow\textit{Type})$,
% $\mathcal{T}_{\text{new}}$ : $(T_2\rightarrow\textit{Type})$, for any
% $s:T_1$, $\pi\in\text{paths}(\mathcal{T}_{\text{curr}}(s))$ and
% $[\ell:v]\in[T_2]_{\text{incr}_x(T_1)}$, then
% $\mathcal{T}_{\text{curr}}+_{\pi,\ell}\mathcal{T}_{\text{new}}$ is
% \begin{quote}
% $\lambda\mathfrak{s}$:$(T_1$\d{$\wedge$}$[T_2]_{\text{incr}_x(T_1)}\ominus[\ell,v])$
% . \record{\tfield{prev}{$\mathcal{T}_{\text{curr}}(\mathfrak{s})$}\\
%           \tfield{e}{$[\mathcal{T}_{\text{new}}]_{\text{incr}_x(T_1)}(\mathfrak{s}\oplus[\ell=\text{prev}.\pi])$}}
% \end{quote} 
% \end{ex} 
% Here the dependence of $\mathcal{T}_{\text{new}}$ on $\ell$ is
% discharged locally by requiring that the $\ell$-field contains the
% same as the $\pi$-field from $\mathcal{T}_{\text{curr}}$.  The
% dependence on $\ell$ is thus removed from the domain of the function
% representing the parametric content for the whole discourse.

% Now let us consider how we can upgrade \preveg{} to allow for more
% than one pronoun resolution at a time as in an example like \textit{A
%   dog chased a cat.  She didn't catch him.} In order to facilitate
% this we introduce two new operators, $\ominus_{\text{set}}$ and
% $\oplus_{\text{set}}$, which perform $\ominus$ and $\oplus$
% respectively for each member of a set in their second arguments.  Thus
% $\ominus_{\text{set}}$ will subtract a set of fields from a record
% type and $\oplus_{\text{set}}$ will add a set of fields to a record.
% We present the upgraded version of \preveg{} in \nexteg{} where if
% $\pi$ is an ordered pair we use $\pi_1$ and $\pi_2$ to represent the
% first and second members of $\pi$ respectively and if $f$ is a field
% then we use label($f$) to represent the label in the field, that is
% the first member of the ordered pair which is the field.
% \begin{ex} 
% If $\mathcal{T}_{\text{curr}}$ : $(T_1\rightarrow\textit{Type})$,
% $\mathcal{T}_{\text{new}}$ : $(T_2\rightarrow\textit{Type})$ and\\
% \hspace*{2em}$\vec{\Pi}\subseteq[T_2]_{\text{incr}_x(T_1)}\times\{\pi\mid$ for any
% $s:T_1$, $\pi\in\text{paths}(\mathcal{T}_{\text{curr}}(s))\}$ such
% that $\vec{\Pi}$ is the graph of a one-one function,\\ then
% $\mathcal{T}_{\text{curr}}+_{\vec{\Pi}}\mathcal{T}_{\text{new}}$ is
% \begin{quote}
% $\lambda\mathfrak{s}$:$(T_1$\d{$\wedge$}$[T_2]_{\text{incr}_x(T_1)})\ominus_{\text{set}}\vec{\Pi}_1$
% .\\ \hspace*{2em} \record{\tfield{prev}{$\mathcal{T}_{\text{curr}}(\mathfrak{s})$}\\
%           \tfield{e}{$[\mathcal{T}_{\text{new}}]_{\text{incr}_x(T_1)}(\mathfrak{s}\oplus_{\text{set}}\{[\text{label}(\vec{\pi}_1)=\text{prev}.\vec{\pi}_2]\mid\vec{\pi}\in\vec{\Pi}\})$}}
% \end{quote} 
% \end{ex} 
% If $\vec{\Pi}$ is
% $\{\langle\langle\ell_1,v_1\rangle,\pi_1\rangle,\ldots,\langle\langle\ell_n,v_n\rangle,\pi_n\rangle\}$
% then for convenience we represent
% $\mathcal{T}_{\text{curr}}+_{\vec{\Pi}}\mathcal{T}_{\text{new}}$ as $\mathcal{T}_{\text{curr}}+_{\ell_1\leadsto\pi_1,\ldots,\ell_n\leadsto\pi_n}\mathcal{T}_{\text{new}}$.

This does not express any of the linguistic constraints concerning
what anaphors can be related to what antecedents.  % This will be
% addressed in Section~\ref{sec:struc-cntxt}.
  
  
% This does not give us a field which \textit{he} could pick up on to
% obtain the anaphoric reference.  However, we can show that the type
% in \nexteg{a} and \nexteg{b} are truth-conditionally equivalent, that is,  \nexteg{a}
% has a witness just in case \nexteg{b} has a witness.
% \begin{ex} 
% \begin{subex} 
 
% \item exist(man$'$, walk$'$) 
 
% \item \record{\tfield{x}{\textit{Ind}}\\
%               \tfield{c}{man(x)}\\
%               \tfield{e}{walk(x)}}
 
% \end{subex} 
% \label{ex:amw-drt}  
% \end{ex} 
% The argument for this goes as follows.  By an argument parallel to
% that in (\ref{ex:witnessconds-edr}c) we can show that \nexteg{a} and
% \nexteg{b} are equivalent.
% \begin{ex} 
% \begin{subex} 
 
% \item $s$ : exist(man$'$, walk$'$) 
 
% \item $\{a\mid\exists s'[s':\text{man}(a)]\}\cap\{a\mid\exists
%   s'[s':\text{man}(a)]\wedge\exists s'[s'\underline{\varepsilon}s\wedge s':\text{walk}(a)]\}\not=\emptyset$ 
 
% \end{subex} 
% \label{ex:amw-wit}   
% \end{ex} 
% This entails \nexteg{a} which, because of the arity of `man' and
% `walk', is equivalent to \nexteg{b}.
% \begin{ex} 
% \begin{subex} 
 
% \item $\exists a[\exists s'[s':\text{man}(a)]\wedge\exists s'[s':\text{walk}(a)]]$
 
% \item $\exists a[a:\textit{Ind}\wedge\exists s'[s':\text{man}(a)]\wedge\exists s'[s':\text{walk}(a)]]$ 
 
% \end{subex} 
   
% \end{ex} 
% If \preveg{b} is true, then it will be possible to construct a witness
% for (\ref{ex:amw-drt}b).  On the other hand, if $s$ is of type
% (\ref{ex:amw-drt}b), then (\ref{ex:amw-wit}b) will be true so there
% will be a witness for (\ref{ex:amw-drt}a), for example, $s$ itself.
% Thus if it has been asserted that there is a situation of type
% (\ref{ex:amw-drt}a), it is safe to assume that there is a situation of
% type (\ref{ex:amw-drt}b) (assuming that the assertion was true).
% (\ref{ex:amw-drt}b) gives us a field which can be picked up by anaphora.  

  

\paragraph{\textit{no dog which chases a cat catches it}}
\label{sec:donkey-anaph}
This example is an instance of what is known in the literature as
\textit{donkey anaphora}.  For a brief overview with references to a
large linguistic literature see \cite{KingLewis2018}.  For good
overviews up to the mid nineties from a linguistic perspective see
\cite{Chierchia1995}, Chapter~2 and \cite{Kanazawa1994}.
Our treatment of donkey anaphora will treat 
\textit{it} in \nexteg{a}  more like the kind of discourse
anaphora discussed in the previous example rather than
direct binding of a pronoun by a quantifier.  In this way it follows
the classic linguistic treatment of donkey anaphora in DRT, first
formulated in \cite{Kamp1981}.  Some evidence for this can be taken
from \nexteg{b}, where it is difficult to relate the singular pronoun
\textit{it} to \textit{every cat}, and \nexteg{c} where the plural
pronoun \textit{them} can be related \textit{every cat}.  This follows
the pattern of discourse anaphora illustrated in \nexteg{d} and
\nexteg{e}.
\begin{ex} 
\begin{subex} 
 
\item no dog which chases a cat catches it 
 
\item no dog which chases every cat catches it

\item no dog which chases every cat catches them

\item Every cat miaowed.  It wanted milk.

\item Every cat miaowed.  They wanted milk. 
 
\end{subex} 
   
\end{ex}

The key to the treatment of donkey anaphora is a process of local
accommodation of context in a parametric property.  Consider a
parametric content for the verb-phrase \textit{catches it} given in
\nexteg{}.
\begin{ex} 
  $\ulcorner\lambda c$:\smallrecord{
    \footnotesize{\textit{Cntxt}}\\
    \smalltfield{$\mathfrak{s}$}{\smallrecord{
        %\footnotesize{\textit{Assgnmnt}}\\
        \smalltfield{x$_0$}{\textit{Ind}}}}\\
    \smalltfield{$\mathfrak{c}$}{\smallrecord{
        \smalltfield{f}{\textit{PropCntxt}}\\
        \smalltfield{a}{\textit{PropCntxt}}}}
} .
  $\ulcorner\lambda r$:\smallrecord{
    \smalltfield{x}{\textit{Ind}}} .
  \record{
    \tfield{e}{catch$^{\dagger}$($r$.x, $c.\mathfrak{s}.\text{x}_0$)}}$\urcorner\urcorner$
\end{ex} 
Local accommodation involves ``moving'' the type of the context into
the domain type of the property under the label `$\mathfrak{c}$' as in
\nexteg{}, adjusting any paths in $c$ addressed in the resulting
function to paths in $r$ beginning with `$\mathfrak{c}$'.
\begin{ex} 
 $\ulcorner\lambda c$:\textit{Cntxt} .
  $\ulcorner\lambda r$:\smallrecord{
    \smalltfield{x}{\textit{Ind}}\\
    \smalltfield{$\mathfrak{c}$}{\smallrecord{
    \footnotesize{\textit{Cntxt}}\\
    \smalltfield{$\mathfrak{s}$}{\smallrecord{
        %\footnotesize{\textit{Assgnmnt}}\\
        \smalltfield{x$_0$}{\textit{Ind}}}}\\
    \smalltfield{$\mathfrak{c}$}{\smallrecord{
        \smalltfield{f}{\textit{PropCntxt}}\\
        \smalltfield{a}{\textit{PropCntxt}}}}
}}} .
  \record{
    \tfield{e}{catch$^{\dagger}$($r$.x, $r.\mathfrak{c}.\mathfrak{s}.\text{x}_0$)}}$\urcorner\urcorner$
\label{ex:catch-it-localized}
\end{ex}
In general we define a localization operation, $\mathcal{L}$, on
parametric properties characterized in \nexteg{}.
\begin{ex} 
  If $\mathcal{P}$ is a parametric property of the form
  \begin{quote}
    $\ulcorner\lambda c\!:\!T_1\ . \ulcorner\lambda r\!:\!T_2\ .\
    \varphi\urcorner\urcorner$
  \end{quote}
  then the \textit{localization of $\mathcal{P}$},
  $\mathcal{L}(\mathcal{P})$, is
  \begin{quote}
    $\ulcorner\lambda c\!:\!\textit{Cntxt}\ .\ \ulcorner\lambda
    r\!:\!T_2\text{\d{$\wedge$}\smallrecord{
        \smalltfield{$\mathfrak{c}$}{$T_1$}}}\ .\ 
    \varphi_{c.\pi\leadsto r.\mathfrak{c}.\pi}\urcorner\urcorner$
  \end{quote}
\label{ex:localization}  
\end{ex} 
  
If we use the localized content (\ref{ex:catch-it-localized}) as the
content of the verb phrase, then, after combination with \textit{no
  dog which chases a cat}, the scope of the quantifier will become
\nexteg{}.
\begin{ex} 
  $\ulcorner\lambda r$:\smallrecord{
    \smalltfield{x}{\textit{Ind}}\\
    \smalltfield{$\mathfrak{c}$}{\smallrecord{
        \footnotesize{\textit{Cntxt}}\\
        \smalltfield{$\mathfrak{s}$}{\smallrecord{
            %\footnotesize{\textit{Assgnmnt}}\\
            \smalltfield{x$_0$}{\textit{Ind}}}}\\
        \smalltfield{$\mathfrak{c}$}{\smallrecord{
            \smalltfield{f}{\textit{PropCntxt}}\\
            \smalltfield{a}{\textit{PropCntxt}}}}
}}\\
    \smalltfield{e$_1$}{dog(x)}\\
    \smalltfield{e$_2$}{\smallrecord{
        \smalltfield{x}{$\mathfrak{T}$(cat$'$)}\\
        \smalltfield{e}{chase$^\dagger$($\Uparrow$x, x)}}}}
   .
  \record{
    \tfield{e}{catch$^{\dagger}$($r$.x, $r.\mathfrak{c}.\mathfrak{s}.\text{x}_0$)}}$\urcorner$
\end{ex}



We can paraphrase this as ``the property of being a dog which chases a
cat and there is something which it catches''.  In order to obtain the
anaphora we need to align the following two paths in the domain type
of this function: `$\mathfrak{c}.\mathfrak{s}.\text{x}_0$' and
`e$_2$.x'.  This we do in \nexteg{} by creating a manifest field on
the former path.

\begin{ex} 
  $\ulcorner\lambda r$:\smallrecord{
    \smalltfield{x}{\textit{Ind}}\\
    \smalltfield{$\mathfrak{c}$}{\smallrecord{
        \footnotesize{\textit{Cntxt}}\\
        \smalltfield{$\mathfrak{s}$}{\smallrecord{
            %\footnotesize{\textit{Assgnmnt}}\\
            \smallmfield{x$_0$}{$\Uparrow^2$e$_2$.x}{\textit{Ind}}}}\\
        \smalltfield{$\mathfrak{c}$}{\smallrecord{
            \smalltfield{f}{\textit{PropCntxt}}\\
            \smalltfield{a}{\textit{PropCntxt}}}}
}}\\
    \smalltfield{e$_1$}{dog(x)}\\
    \smalltfield{e$_2$}{\smallrecord{
        \smalltfield{x}{$\mathfrak{T}$(cat$'$)}\\
        \smalltfield{e}{chase$^\dagger$($r$.x, x)}}}}
   .
  \record{
    \tfield{e}{catch$^{\dagger}$($r$.x, $r.\mathfrak{c}.\mathfrak{s}.\text{x}_0$)}}$\urcorner$
\end{ex}
We can paraphrase this as ``the property of being a dog which chases a
cat and catches that cat''.

In order to include such properties with aligned paths as the scope of
quantifiers in our interpretations, we will first generalize the
characterization of alignment of paths given on
p.~\pageref{pg:path-alignment-types}f to functions in \nexteg{}
(repeated in Appendix~\ref{app:path-align-rectype}).
\begin{ex} 
  If $\varphi=\ulcorner\lambda r\!:\!T\ .\ \psi\urcorner$ and
  $\pi_1,\pi_2\in\mathrm{paths}(T)$, then
  \begin{quote}
    $\varphi_{\pi_1=\pi_2}=\ulcorner\lambda r\!:\!T_{\pi_1=\pi_2}\ .\
    \psi\urcorner$
  \end{quote}
  \label{ex:path-alignment-fcts}
\end{ex} 
We then add two further clauses to the characterization of witnesses
for $\mathfrak{S}(T)$ for content types, $T$, in \nexteg{} with the
new additions boxed.
\begin{ex} 
\begin{subex} 
 
\item If $T:\textit{ContType}$, then $\mathfrak{S}(T)$ is a type 
 
\item The witnesses of $\mathfrak{S}(T)$ are characterized by
  \begin{enumerate} 
 
  \item if $\varphi:T$ then $\varphi:\mathfrak{S}(T)$
    
  \item \fbox{if $\varphi:\mathfrak{S}(T)$ and $\varphi:\textit{PPpty}$,
      then $\mathcal{L}(\varphi):\mathfrak{S}(T)$}
    
  \item \fbox{\begin{minipage}[t]{.9\linewidth}if $\varphi:\mathfrak{S}(T)$,
    $\varphi\sqsubseteq\text{\smallrecord{
        \smallmfield{scope}{$\psi$}{\textit{Ppty}}}}$ and
    $\pi_1,\pi_2\in\mathrm{paths}(\psi.\text{bg})$, then $\varphi[\text{scope}=\psi_{\pi_1=\pi_2}:\textit{Ppty}]:\mathfrak{S}(T)$\end{minipage}}

    
  \item if $\alpha\mathcal{O}\beta:\mathfrak{S}(T)$, (for
      some combination operation, $\mathcal{O}$) and
      $\alpha\mathcal{O}_{i,j}\beta$ is defined (for some natural
      numbers, $i$ and $j$), then $\alpha\mathcal{O}_{i,j}\beta:\mathfrak{S}(T)$
 
  \item if $\varphi:\mathfrak{S}(T)$ and $\varphi$ is in the range of `$\mathrm{storage}$', then
    $\mathrm{storage}(\varphi):\mathfrak{S}(T)$

  \item if $\varphi:\mathfrak{S}(T)$ and `x$_i$' and $\varphi$ are appropriate
    arguments to `$\mathrm{retrieve}$', then
    $\mathrm{retrieve}(\text{x}_i,\varphi):\mathfrak{S}(T)$
    
  \item nothing is a witness for $\mathfrak{S}(T)$ except as required above.
 
  \end{enumerate} 
  
 
\end{subex} 
\label{ex:storage-donkey-type}   
\end{ex}



Overall we can get a parametric content for \textit{no dog which
  chases a cat catches it} which looks like \nexteg{} where
\savebox{\boxone}{\textbf{dog}$^\frown$\textbf{which}$^\frown$\textbf{chases}$^\frown$\textbf{a}$^\frown$\textbf{cat}} \usebox{\boxone}
and \savebox{\boxtwo}{\textbf{catches}$^\frown$\textbf{it}}\usebox{\boxtwo} represent parametric contents
which have not undergone operations introduced by $\mathfrak{S}$.
\begin{ex} 
  $\lambda c$:\textit{Cntxt} . \record{
    \mfield{restr}{\usebox{\boxone}($c$)}{\textit{Ppty}}\\
    \mfield{scope}{$(\mathcal{L}(\text{\usebox{\boxtwo}})(c)|_{\mathcal{F}(\text{\usebox{\boxone}})})_{\mathfrak{c}.\mathfrak{s}.\text{x}_0=\text{e}_2.\text{x}}$}{\textit{Ppty}}\\
    \smalltfield{e}{no(restr, scope)}}
\end{ex} 
Given the semantics we have specified for the determiner \textit{no},
\preveg{} is \nexteg{}.
\begin{ex} 
  $\lambda c$:\textit{Cntxt} . \record{
    \mfield{restr}{\usebox{\boxone}($c$)}{\textit{Ppty}}\\
    \mfield{scope}{$(\mathcal{L}(\text{\usebox{\boxtwo}})(c)|_{\mathcal{F}(\text{\usebox{\boxone}})})_{\mathfrak{c}.\mathfrak{s}.\text{x}_0=\text{e}_2.\text{x}}$}{\textit{Ppty}}\\
    \smalltfield{e}{\record{
        \tfield{X}{every$^w$(restr)}\\
        \tfield{f}{$((x:\mathfrak{T}(X))\rightarrow\neg\mathfrak{P}(\text{scope})\{x\})$}
      }}}
\end{ex} 
We can paraphrase this content as ``for every dog which chases a cat
it's not the case that it's a dog which chases a cat$_i$ and catches
it$_i$'' where the subscript on \textit{cat} and \textit{it} indicates
that \textit{it} is anaphorically related to \textit{a cat}.


% The basic content related to \textit{no} is given in \nexteg{}.
% \begin{ex} 
% $\lambda\mathfrak{s}$:\textit{Rec} $\lambda Q$:\textit{Ppty} $\lambda
% P$:\textit{Ppty} . \record{\tfield{e}{no($Q$,$P|_{\mathcal{F}(Q)}$)}} 
% \end{ex} 
% As usual we abbreviate the content of \textit{dog} as `dog$'$'.  The
% relative pronoun \textit{which} has the content specified for
% \textit{who} in Section~\ref{sec:long-distance}.  (We are simplifying
% by not accounting for the gender distinction between the two.)  We
% repeat this in \nexteg{} and will refer to it here as \textbf{which}.
% \begin{ex} 
% $\lambda\mathfrak{s}$:\smallrecord{\smalltfield{wh}{\textit{Ind}}}
%   . $\lambda P$:\textit{Ppty} . $P$\{$\mathfrak{s}$.wh\} 
% \end{ex} 
% The basic content of an utterance of \textit{chase} is given in
% \nexteg{}.
% \begin{ex} 
% $\lambda\mathfrak{s}$:\textit{Rec} $\lambda
% r_2$:\smallrecord{\smalltfield{x}{\textit{Quant}}} $\lambda
% r_1$:\smallrecord{\smalltfield{x}{\textit{Ind}}}
%   . \record{\tfield{e}{chase($r_1$.x, $r_2$.x)}} 
% \end{ex} 
% The indefinite article $a$ will have the content in \nexteg{}.
% \begin{ex} 
% $\lambda\mathfrak{s}$:\textit{Rec} $\lambda Q$:\textit{Ppty} $\lambda
% P$:\textit{Ppty} . \record{\tfield{e}{exist($Q$, $P|_{\mathcal{F}(Q)}$)}} 
% \end{ex} 
% Thus the phrase \textit{a cat} will have the content in \nexteg{}.
% \begin{ex} 
% $\lambda\mathfrak{s}$:\textit{Rec} $\lambda P$:\textit{Ppty}
% . \record{\tfield{e}{exist(cat$'$, $P|_{\mathcal{F}(\text{cat}')}$)}} 
% \end{ex} 
% We will refer to \preveg{} as \textbf{a$^\frown$cat}.  Given this the
% content of \textit{chase a cat} will be \nexteg{}.
% \begin{ex} 
% $\lambda\mathfrak{s}$:\textit{Rec} $\lambda
% r_1$:\smallrecord{\smalltfield{x}{\textit{Ind}}}
% . \record{\tfield{e}{chase($r_1$.x,
%     \textbf{a$^\frown$cat}($\mathfrak{s}$))}}
% \label{ex:chaseacat-noexport} 
% \end{ex} 
% Given that \textit{chase} is an extensional verb it will obey a
% constraint like (\ref{ex:mp-find}) on p.~\pageref{ex:mp-find}, as
% given in \nexteg{}.
% \begin{ex} 
% $e$ : chase($a$, $Q$) iff $e$ : $Q$($\lambda
% r$:\smallrecord{\smalltfield{x}{\textit{Ind}}} . \smallrecord{\smalltfield{e}{chase$^{\dagger}$($a$,
% $r$.x)}}) 
% \end{ex} 
% This means that we can construe the content to be a function which
% returns an equivalent type to that returned in
% (\ref{ex:chaseacat-noexport}).  This new content is given in
% \nexteg{}.
% \begin{ex} 
% $\lambda s$:\textit{Rec} $\lambda
% r_1$:\smallrecord{\smalltfield{x}{\textit{Ind}}}
%   . \record{\tfield{e}{\textbf{a$^\frown$cat}($\mathfrak{s}$)($\lambda
%       r$:\smallrecord{\smalltfield{x}{\textit{Ind}}}
%       . \smallrecord{\smalltfield{e}{chase$^\dagger$($r_1$.x, $r$.x)}})}} 
% \end{ex} 
% \preveg{} is equivalent to \nexteg{}.
% \begin{ex} 
% $\lambda\mathfrak{s}$:\textit{Rec} $\lambda
% r_1$:\smallrecord{\smalltfield{x}{\textit{Ind}}}
% . \record{\tfield{e}{exist(cat$'$, $\lambda
%     r$:\smallrecord{\smalltfield{x}{\textit{Ind}}\\
%                     \smalltfield{e}{cat(x)}}
%                   . \smallrecord{\smalltfield{e}{chase$^\dagger$($r_1$.x, $r$.x)}})}} 
% \end{ex} 
% In turn, \preveg{} is equivalent to \nexteg{}.
% \begin{ex} 
% $\lambda\mathfrak{s}$:\textit{Rec} $\lambda
% r_1$:\smallrecord{\smalltfield{x}{\textit{Ind}}}
%   . \record{\tfield{e}{\record{\tfield{x}{$\mathfrak{T}$(cat$'$)}\\
%                                \tfield{e}{\record{\tfield{$\mathfrak{c}$}{\smallrecord{\smallmfield{x}{$\Uparrow^2$x}{\textit{Ind}}\\
%                                                    \smalltfield{e}{cat(x)}}}\\
%                    \tfield{e}{\smallrecord{\smalltfield{e}{chase$^\dagger$($r_1$.x,
%                          $\Uparrow\!\mathfrak{c}$.x)}}}}}}}}
% \label{ex:chase-a-cat} 
% \end{ex} 
% We represent \preveg{} as \textbf{chase$^\frown$a$^\frown$cat}.  From
% this we can form a ``sentence with a gap'' interpretation,
% \textbf{chase$^\frown$a$^\frown$cat$_S$}, in the manner described in
% Section~\ref{sec:long-distance}.  This is given in \nexteg{}.
% \begin{ex} 
% $\lambda\mathfrak{s}$:\smallrecord{\smalltfield{wh$_0$}{\textit{Ind}}} . $\lambda
% P$:\textit{Ppty} . $P$\{$\mathfrak{s}$.wh$_0$\}(\textbf{chase$^\frown$a$^\frown$cat}($\mathfrak{s}$)) 
% \end{ex} 
% Unpacking \preveg{} we obtain \nexteg{}.
% \begin{ex} 
% $\lambda\mathfrak{s}$:\smallrecord{\smalltfield{wh$_0$}{\textit{Ind}}}
% . \record{\tfield{e}{\record{\tfield{x}{$\mathfrak{T}$(cat$'$)}\\
%                                \tfield{e}{\record{\tfield{$\mathfrak{c}$}{\smallrecord{\smallmfield{x}{$\Uparrow^2$x}{\textit{Ind}}\\
%                                                    \smalltfield{e}{cat(x)}}}\\
%                    \tfield{e}{\smallrecord{\smalltfield{e}{chase$^\dagger$($\mathfrak{s}$.wh$_0$, $\Uparrow\!\mathfrak{c}$.x)}}}}}}}} 
% \end{ex} 
% Again following the analysis of long-distance dependencies developed
% in Section~\ref{sec:long-distance}, the content of \textit{which
%   chased a cat} is given in \nexteg{}.
% \begin{ex} 
% $\lambda\mathfrak{s}$:\textit{Rec} . \\
% \hspace*{1em} $\lambda
% r_1$:\smallrecord{\smalltfield{x}{\textit{Ind}}}
% . \textbf{which}($\mathfrak{s}\oplus[\text{wh}=r_1.\text{x}$)($\lambda
% r_2$:\smallrecord{\smalltfield{x}{\textit{Ind}}} . \textbf{chase$^\frown$a$^\frown$cat$_S$}($\mathfrak{s}\oplus[\text{wh}_0=r_2.\text{x}]$) 
% \end{ex} 
% Unpacking \preveg{} gives us (\ref{ex:chase-a-cat}).  That is, the content of
% \textit{which chased a cat} is identical with the content of
% \textit{chased a cat}.  For clarity we will represent this content
% also as \textbf{which$^\frown$chase$^\frown$a$^\frown$cat}.
% Again following the proposal in Section~\ref{sec:long-distance} the
% content for \textit{dog which chases a cat} will be
% \nexteg{}. 
% \begin{ex} 
% $\lambda\mathfrak{s}$:\textit{Rec} $\lambda
% r$:\smallrecord{\smalltfield{x}{\textit{Ind}}}
% . \record{\tfield{e$_1$}{dog$'$\{$r$.x\}}\\
%           \tfield{e$_2$}{\textbf{which$^\frown$chase$^\frown$a$^\frown$cat}($\mathfrak{s}$)\{$r$.x\}}} 
% \end{ex} 
% We call this
% \textbf{dog$^\frown$which$^\frown$chase$^\frown$a$^\frown$cat}. Now
% \textit{no dog which chases a cat} will correspond to \nexteg{a} which
% is equivalent to \nexteg{b}.
% \begin{ex} 
% \begin{subex} 
 
% \item $\lambda\mathfrak{s}$:\textit{Rec} $\lambda P$:\textit{Ppty}
%   . \record{\tfield{e}{no(\textbf{dog$^\frown$which$^\frown$chase$^\frown$a$^\frown$cat}($\mathfrak{s}$),
%       $P|_{\mathcal{F}(\textbf{dog}^\frown\textbf{which}^\frown\textbf{chase}^\frown
%           \textbf{a}^\frown\textbf{cat}(\mathfrak{s}))}$)}} 
 
% \item $\lambda\mathfrak{s}$:\textit{Rec} $\lambda P$:\textit{Ppty}
%   . \record{\tfield{e}{\record{\tfield{X}{every$^w$(\textbf{dog$^\frown$which$^\frown$chase$^\frown$a$^\frown$cat}($\mathfrak{s}$))}\\
%                            \tfield{f}{(($x:\mathfrak{T}(\text{X}))\rightarrow\neg\mathfrak{P}(P|_{\mathcal{F}(\textbf{dog$^\frown$which$^\frown$chase$^\frown$a$^\frown$cat}(\mathfrak{s}))})\{\text{x}\}$)}}}} 
 
% \end{subex} 
% \label{ex:no-dog-which-chases-a-cat}   
% \end{ex} 
% Call this
% \textbf{no$^\frown$dog$^\frown$which$^\frown$chase$^\frown$a$^\frown$cat}.
% This is to combine with the content of \textit{catches it}, given in
% \nexteg{}. 
% \begin{ex} 
% $\lambda\mathfrak{s}$:\smallrecord{\smalltfield{x$_0$}{\textit{Ind}}}
% $\lambda r$:\smallrecord{\smalltfield{x}{\textit{Ind}}}
% . \record{\tfield{e}{catch$^\dagger$($r$.x, $\mathfrak{s}$.x$_0$)}} 
% \end{ex}
% We will represent this as \textbf{catch$^\frown$it}. 
% Using the S-combinator strategy to
% (\ref{ex:no-dog-which-chases-a-cat}b) yields \nexteg{a}, equivalently
% \nexteg{b}, which represents a reading of the sentence where
% \textit{it} is not captured and refers to a particular object provided
% by the context.
% \begin{ex}
% \begin{subex} 
 
% \item $\lambda\mathfrak{s}$:\smallrecord{\smalltfield{x$_0$}{\textit{Ind}}} 
%   . \smallrecord{\smalltfield{e}{\record{\tfield{X}{every$^w$(\textbf{dog$^\frown$which$^\frown$chase$^\frown$a$^\frown$cat}($\mathfrak{s}$))}\\
%                            \tfield{f}{(($x:\mathfrak{T}(\text{X}))\rightarrow$\\
% &&$\neg\mathfrak{P}(\lambda r$:\smallrecord{\smalltfield{x}{\textit{Ind}}}
% . \record{\tfield{e}{catch$^\dagger$($r$.x, $\mathfrak{s}$.x$_0$)}}$|_{\mathcal{F}(\textbf{dog$^\frown$which$^\frown$chase$^\frown$a$^\frown$cat}(\mathfrak{s}))})\{\text{x}\}$)}}}} 
 
% \item $\lambda\mathfrak{s}$:\smallrecord{\smalltfield{x$_0$}{\textit{Ind}}} 
%   . \smallrecord{\smalltfield{e}{\record{\tfield{X}{every$^w$(\textbf{dog$^\frown$which$^\frown$chase$^\frown$a$^\frown$cat}($\mathfrak{s}$))}\\
%                            \tfield{f}{(($x:\mathfrak{T}(\text{X}))\rightarrow$\\
% &&\hspace*{2em}$\neg$\smallrecord{\smalltfield{$\mathfrak{c}$}{\smallrecord{\smallmfield{x}{$x$}{\textit{Ind}}\\
%                                                                \smalltfield{e$_1$}{dog$'$\{x\}}\\
%                                                                \smalltfield{e$_2$}{\smallrecord{\smalltfield{e}{\smallrecord{\smalltfield{x}{$\mathfrak{T}$(cat$'$)}\\
%                                                                                                 \smalltfield{e}{\smallrecord{\smalltfield{$\mathfrak{c}$}{\smallrecord{\smallmfield{x}{$\Uparrow^2$x}{\textit{Ind}}\\
%                                                                                                                                                                        \smalltfield{e}{cat(x)}}}\\
%                                                                                                      \smalltfield{e}{\smallrecord{\smalltfield{e}{chase$^\dagger$($\Uparrow^4$x,
%                                                                                                            $\Uparrow\!\mathfrak{c}$.x)}}}}}}}}}}}\\
%                \smalltfield{e}{catch$^\dagger$($x$, $\mathfrak{s}$.x$_0$)}}
% )}}}} 
 
% \end{subex} 
   
 
% \end{ex} 
% In terms of the notation in \preveg{b}, it is simple enough to see
% what needs to be done in order to change it to a case of donkey
% anaphora.  The second argument to `catch$^\dagger$',
% `$\mathfrak{s}$.x$_0$', has to be changed so that it presents a path
% to the cat, that is, `$\mathfrak{c}$.e$_2$.e.x'.  Also the dependence
% of this parametric content on `x$_0$' has to be removed.  Thus the
% domain type for the $\lambda$-abstraction, currently
% `\smallrecord{\smalltfield{x$_0$}{\textit{Ind}}}', should be replaced
% by `\textit{Rec}'.  The result would be \nexteg{}.
% \begin{ex} 
%  $\lambda\mathfrak{s}$:\textit{Rec} 
%   . \smallrecord{\smalltfield{e}{\record{\tfield{X}{every$^w$(\textbf{dog$^\frown$which$^\frown$chase$^\frown$a$^\frown$cat}($\mathfrak{s}$))}\\
%                            \tfield{f}{(($x:\mathfrak{T}(\text{X}))\rightarrow$\\
% &&\hspace*{2em}$\neg$\smallrecord{\smalltfield{$\mathfrak{c}$}{\smallrecord{\smallmfield{x}{$x$}{\textit{Ind}}\\
%                                                                \smalltfield{e$_1$}{dog$'$\{x\}}\\
%                                                                \smalltfield{e$_2$}{\smallrecord{\smalltfield{e}{\smallrecord{\smalltfield{x}{$\mathfrak{T}$(cat$'$)}\\
%                                                                                                 \smalltfield{e}{\smallrecord{\smalltfield{$\mathfrak{c}$}{\smallrecord{\smallmfield{x}{$\Uparrow^2$x}{\textit{Ind}}\\
%                                                                                                                                                                        \smalltfield{e}{cat(x)}}}\\
%                                                                                                      \smalltfield{e}{\smallrecord{\smalltfield{e}{chase$^\dagger$($\Uparrow^4$x,
%                                                                                                            $\Uparrow\!\mathfrak{c}$.x)}}}}}}}}}}}\\
%                \smalltfield{e}{catch$^\dagger$($x$, $\mathfrak{c}$.e$_2$.e.x)}}
% )}}}}
% \label{ex:no-dog-which-chases-a-cat-catches-it-captured} 
% \end{ex} 
% We will achieve this by introducing a notion of restriction on a
% parametric property in which we will allow anaphora to take place.
% Let us first consider a case without anaphora.  Suppose that
% $\mathcal{P}$ is a parametric property of type
% $(T_1\rightarrow(T_2\rightarrow \textit{Type}))$ and that $T$ is a
% type then we define the restriction of $\mathcal{P}$ by $T$,
% $\mathcal{P}|^p_T$, to be \nexteg{}.
% \begin{ex} 
% $\lambda\mathfrak{s}$:$T_1$ $\lambda r$:$T_2$\d{$\wedge$}$T$ . $\mathcal{P}(\mathfrak{s})(r)$ 
% \end{ex} 
% Now suppose that $[\ell,v]$ is a field in $T_1$ and $\pi$ is a path in
% $T_2$ and we want to make an anaphoric association between $\ell$ and
% $\pi$.  We define such a restriction,
% $\mathcal{P}|^p_{T,\ell\leadsto\pi}$ as \nexteg{}.
% \begin{ex} 
% $\lambda\mathfrak{s}$:$T_1\ominus[\ell:v]$ $\lambda
% r$:$T_2$\d{$\wedge$}$T$ . $\mathcal{P}(\mathfrak{s}\oplus[\ell=r.\pi])(r)$ 
% \end{ex}
% To enable the capture of several pronouns, we let $\vec\Pi\subseteq
% T_1\times\text{paths}(T)$ such that $\vec\Pi$ is the graph of a one-one
% function (whose domain is included in the fields of $T_1$ and whose
% range is included in the paths of $T$).  We then define
% $\mathcal{P}|^p_{T,\vec{\Pi}}$ to be \nexteg{}.
% \begin{ex} 
% $\lambda\mathfrak{s}$:$T_1\ominus_{\text{set}}\vec{\Pi}_1$ $\lambda
% r$:$T_2$\d{$\wedge$}$T$ . $\mathcal{P}(\mathfrak{s}\oplus_{\text{set}}\{[\text{label}(\vec{\pi}_1)=r.\vec{\pi}_2]\mid\vec{\pi}\in\vec{\Pi}\})$ 
% \end{ex} 
% If $\vec{\Pi}$ is
% $\{\langle\langle\ell_1,v_1\rangle,\pi_1\rangle,\ldots,\langle\langle\ell_n,v_n\rangle,\pi_n\rangle\}$
% then for convenience we represent $\mathcal{P}|^p_{T,\vec{\Pi}}$ as
% $\mathcal{P}|^p_{T,\ell_1\leadsto\pi_1,\ldots,\ell_n\leadsto\pi_n}$. Using
% this notation,  the content of \textit{no dog which chases a cat catches
%   it} where the pronoun is captured is \nexteg{}.
% \begin{ex} 
% $\lambda\mathfrak{s}$:\textit{Rec}
%   . \smallrecord{\smalltfield{e}{\record{\tfield{X}{every$^w$(\textbf{dog$^\frown$which$^\frown$chase$^\frown$a$^\frown$cat}($\mathfrak{s}$))}\\
%                            \tfield{f}{(($x:\mathfrak{T}(\text{X}))\rightarrow$\\
% &&$\neg\mathfrak{P}$(\textbf{catch$^\frown$it}$|^p_{\mathcal{F}(\textbf{dog$^\frown$which$^\frown$chase$^\frown$a$^\frown$cat}(\mathfrak{s})),\text{x}_0\leadsto\text{e}_2.\text{e}.\text{x}}(\mathfrak{s})$)\{x\})}}}} 
% \end{ex} 
% Unpacking \preveg{} yields
% (\ref{ex:no-dog-which-chases-a-cat-catches-it-captured}).

This treatment of donkey anaphora is essentially similar to that of
\cite{Chierchia1995} in that it associates an existential reading the
pronoun it, that is, it is not the case that there is a cat which the
dog chases that it also catches.  The dog does not catch any of the
cats it chases.  If we use \textit{every} instead of \textit{no} we
get a reading which is paraphrased as ``every dog which chases a cat
is a dog which chases a cat and catches it''.  This is what is known
in the literature as a \textit{weak reading} or a
\textit{$\exists$-reading}. It says that every dog which chases a cat
catches \textit{some} cat that it chases (but not necessarily all).
This reading may intuitively not be appropriate for \textit{every dog
  that chases a cat catches it} which for many speakers would suggest
that the dogs catch all the cats they chase.  However, the weak
reading is important for the examples in \nexteg{}.

% Notice that \preveg{}
% forces what is known in the literature as a \textit{strong} reading
% for the donkey anaphora.  That is, none of the dogs which chase a cat
% catch \textit{any} of the cats which they chase.  That monotone
% decreasing quantifiers force such strong readings was noted, for
% example, by [Chierchia???? Pelletier????].  Monotone increasing
% quantifiers, however, allow \textit{weak} readings.  Examples of
% donkey sentences mentioned in the literature that have naturally weak
% readings are given in \nexteg{}.
\begin{ex} 
\begin{subex} 
 
\item Every person who had a dime put it in the parking
  meter. \citep{PelletierSchubert1989} 
 
\item Every man who has a daughter thinks she is the most beautiful
  girl in the world \citep{Cooper1979} 
 
\end{subex} 
   
\end{ex} 
\preveg{a} does not seem to suggest that anybody who had several dimes
put them all in the meter and \preveg{b} does not seem to
commit a man who has two daughters to believe the contradictory
proposition that they are both the one and only most beautiful girl in
the world.
%
% @@
%
% Let us consider the content for \textit{every person who
%   had a dime put it in the parking meter} given in \nexteg{}.
% \begin{ex} 
%  $\lambda\mathfrak{s}$:\textit{Rec} 
%   . \smallrecord{\smalltfield{e}{\record{\tfield{X}{every$^w$(\textbf{person$^\frown$who$^\frown$have$^\frown$a$^\frown$dime}($\mathfrak{s}$))}\\
%                            \tfield{f}{(($x:\mathfrak{T}(\text{X}))\rightarrow$\\
% &&\hspace*{2em}\smallrecord{\smalltfield{$\mathfrak{c}$}{\smallrecord{\smallmfield{x}{$x$}{\textit{Ind}}\\
%                                                                \smalltfield{e$_1$}{person$'$\{x\}}\\
%                                                                \smalltfield{e$_2$}{\smallrecord{\smalltfield{e}{\smallrecord{\smalltfield{x}{$\mathfrak{T}$(dime$'$)}\\
%                                                                               %                  \smalltfield{e}{\smallrecord{\smalltfield{$\mathfrak{c}$}{\smallrecord{\smallmfield{x}{$\Uparrow^2$x}{\textit{Ind}}\\
%                                                                               %                                                                                         \smalltfield{e}{dime(x)}}}\\
%                                                                               %                       \smalltfield{e}{\smallrecord{\smalltfield{e}{have$^\dagger$($\Uparrow^4$x,
%                                                                                                            $\Uparrow\!\mathfrak{c}$.x)}}}}}}}}}}}\\
%                \smalltfield{e}{put\_in\_the\_meter$^\dagger$($x$, $\mathfrak{c}$.e$_2$.e.x)}}
% )}}}} 
% \end{ex} 
% \preveg{} is exactly similar to
% (\ref{ex:no-dog-which-chases-a-cat-catches-it-captured}) modulo the
% properties and predicates involved except that there is no negation in
% \preveg{}.  \preveg{} requires then that every person who had a dime
% is a person who had a dime and put it in the meter, that is, a weak
% reading which requires only that each relevant person put some dime in
% the meter, not all of their dimes. 
What then do we say about the
original donkey sentences like \textit{every farmer who owns a donkey
  likes it} which were analyzed by Geach and the classical analyses in
DRT and type theory as having a strong reading in which every farmer
who owns a donkey likes any donkey that he owns?  One option is to say
that we only need the weak reading for such sentences as it is
consistent with the stronger reading.  Many speakers feel that it
unclear what the sentence means if some man owns more than one donkey.
% \begin{ex} 
%  $\lambda\mathfrak{s}$:\textit{Rec} 
%   . \smallrecord{\smalltfield{e}{\record{\tfield{X}{every$^w$(\textbf{farmer$^\frown$who$^\frown$owns$^\frown$a$^\frown$donkey}($\mathfrak{s}$))}\\
%                            \tfield{f}{(($x:\mathfrak{T}(\text{X}))\rightarrow$\\
% &&\hspace*{2em}\smallrecord{\smalltfield{$\mathfrak{c}$}{\smallrecord{\smallmfield{x}{$x$}{\textit{Ind}}\\
%                                                                \smalltfield{e$_1$}{farmer$'$\{x\}}\\
%                                                                \smalltfield{e$_2$}{\smallrecord{\smalltfield{e}{\smallrecord{\smalltfield{x}{$\mathfrak{T}$(donkey$'$)}\\
%                                                                                                 \smalltfield{e}{\smallrecord{\smalltfield{$\mathfrak{c}$}{\smallrecord{\smallmfield{x}{$\Uparrow^2$x}{\textit{Ind}}\\
%                                                                                                                                                                        \smalltfield{e}{donkey(x)}}}\\
%                                                                                                      \smalltfield{e}{\smallrecord{\smalltfield{e}{have$^\dagger$($\Uparrow^4$x,
%                                                                                                            $\Uparrow\!\mathfrak{c}$.x)}}}}}}}}}}}\\
%                \smalltfield{e}{like$^\dagger$($x$, $\mathfrak{c}$.e$_2$.e.x)}}
% )}}}} 
% \label{ex:donkey-weak} 
% \end{ex} 
That is,
\textit{every farmer who owns a donkey likes it} requires that for
every donkey owning farmer there is at least one donkey the farmer
owns such that she likes it.  This allows for the farmers to like all
their donkeys but does not require it. (See \citealp{Kanazawa1994} for
a discussion of this.)  In the case of \textit{every dog which chases
  a cat catches it} I have the following intuition:
\begin{quote}
  if there is a dog under consideration which is involved in two
  distinct cat-chasing events (with a single cat) and only succeeds in catching the cat in
  one of the two events, then this seems to make the sentence false;

  if there is a dog under consideration which is involved in a single
  event of chasing two cats and only succeeds in catching one of the
  cats in that event, then it seems that the sentence could still be
  true.
\end{quote}
This suggests an analysis which requires that every relevant
cat-chasing event must involve the catching of at least one of the
cats chased in that event.  Firm judgements concerning such intuitions
are notoriously hard to come by.  \cite{Chierchia1995} argues that the
strong $\forall$-reading is necessary because of examples like
\nexteg{}.
\begin{ex} 
Every man who owned a slave owned his offspring 
\end{ex} 
\preveg{} is a modification of an example in \cite{Heim1990} which is
used to make a different argument.  It seems like a single instance of
somebody owning a slave but not the slave's offspring would be
sufficient to falsify \preveg{}.  Though again, in the case of a slave
who has several offspring one of which is owned by somebody else, it
seems to me that it is unclear whether that is sufficient to falsify
the sentence.

% One might also
% argue that the strong reading is necessary for interpreting sentences
% such as \nexteg{}.
% \begin{ex} 
% It is not true that every farmer who owns a donkey likes it 
% \end{ex} 
% If there is a reading of \preveg{} on which it is true even if every
% farmer likes at least one of her donkeys, but at least one farmer 
% does not like all of them, then we need to analyze this a strong reading. [????Check
% for discussion of this.] 

\label{pg:donkey-purification-universal}
One option is to recreate the original Geach reading by
using the variant of the purification operation, `$\mathfrak{P}^\forall$', in
(\ref{ex:purification-universal}), p.~\pageref{ex:purification-universal} which introduces a function on local contexts.
% \begin{ex} 
% If $P$ : \textit{Ppty}, then
% \begin{quote}
% if $P$.bg$^x$ = $P$.bg, then
% \begin{quote}
% $\mathfrak{P}(P)=P$
% \end{quote}
% otherwise:
% \begin{quote}
% $\mathfrak{P}(P)$ is $\ulcorner\lambda r_1$:$P$.bg$^{\text{x}}$
% . ($(r_2\!:\!P.\text{bg}\!\parallel\!\!\text{\smallrecord{\field{x}{$r$.x}}})\rightarrow$
% \record{
%           \tfield{e}{$P(r_2)$}})$\urcorner$
% \end{quote}
% \end{quote}
% %\label{ex:purification-classic}
% \end{ex}
Using `$\mathfrak{P}^\forall$' instead of `$\mathfrak{P}$' will have the
consequence that the content of \textit{every farmer who owns a donkey
  likes it} will be \nexteg{}.
\savebox{\boxone}{\textbf{farmer}$^\frown$\textbf{who}$^\frown$\textbf{owns}$^\frown$\textbf{a}$^\frown$\textbf{donkey}}
\savebox{\boxtwo}{\textbf{likes}$^\frown$\textbf{it}}
\begin{ex} 
  $\lambda c$:\textit{Cntxt} . \record{
    \mfield{restr}{\usebox{\boxone}($c$)}{\textit{Ppty}}\\
    \mfield{scope}{$(\mathcal{L}(\text{\usebox{\boxtwo}})(c)|_{\mathcal{F}(\text{\usebox{\boxone}})})_{\mathfrak{c}.\mathfrak{s}.\text{x}_0=\text{e}_2.\text{x}}$}{\textit{Ppty}}\\
    \smalltfield{e}{every(restr, scope)}}
\end{ex} 

Given the general witness condition associated with `every', \preveg{} is
equivalent to \nexteg{}.
\begin{ex} 
  $\lambda c$:\textit{Cntxt} . \record{
    \mfield{restr}{\usebox{\boxone}($c$)}{\textit{Ppty}}\\
    \mfield{scope}{$(\mathcal{L}(\text{\usebox{\boxtwo}})(c)|_{\mathcal{F}(\text{\usebox{\boxone}})})_{\mathfrak{c}.\mathfrak{s}.\text{x}_0=\text{e}_2.\text{x}}$}{\textit{Ppty}}\\
    \smalltfield{e}{\record{
        \tfield{X}{every$^w$(restr)}\\
        \tfield{f}{$((x:\mathfrak{T}(X))\rightarrow\mathfrak{P}(\text{scope})\{x\})$}
      }}}
\end{ex}
If we use a variant of this witness condition which uses
`$\mathfrak{P}^\forall$' instead of `$\mathfrak{P}$' we have
\nexteg{}.
\begin{ex} 
  $\lambda c$:\textit{Cntxt} . \record{
    \mfield{restr}{\usebox{\boxone}($c$)}{\textit{Ppty}}\\
    \mfield{scope}{$(\mathcal{L}(\text{\usebox{\boxtwo}})(c)|_{\mathcal{F}(\text{\usebox{\boxone}})})_{\mathfrak{c}.\mathfrak{s}.\text{x}_0=\text{e}_2.\text{x}}$}{\textit{Ppty}}\\
    \smalltfield{e}{\record{
        \tfield{X}{every$^w$(restr)}\\
        \tfield{f}{$((x:\mathfrak{T}(X))\rightarrow\mathfrak{P}^\forall(\text{scope})\{x\})$}
      }}}
  \label{ex:efwoadli-classic}
\end{ex}
Let us check that \preveg{} does in fact give us the strong
$\forall$-reading.  We will do this by showing that the value
associated with the label `scope' will be paraphrasable as ``the
property of being an individual such that if it's a farmer who owns a
donkey, she likes that donkey'', that is, likes every donkey she
owns. We can show this by unpacking the expression representing the
scope in \preveg{}.  `\textbf{likes}$^\frown$\textbf{it}', according to
our treatment of free pronouns and extensional verbs, will be
\nexteg{}.
\begin{ex} 
  $\ulcorner\lambda c$:\smallrecord{
    \footnotesize{Cntxt}\\
    \smalltfield{$\mathfrak{s}$}{\smallrecord{
        %\footnotesize{Assgnmnt}\\
        \smalltfield{x$_0$}{\textit{Ind}}}}} .
  $\ulcorner\lambda r$:\smallrecord{
    \smalltfield{x}{\textit{Ind}}} . \record{
    \tfield{e}{like$^\dagger$($r$.x, $c.\mathfrak{s}$.x$_0$)}}$\urcorner\urcorner$
\end{ex}
Using the definition of the localization operation, $\mathcal{L}$, in
(\ref{ex:localization}), we can see that
`$\mathcal{L}(\textbf{likes}^\frown\textbf{it})$' is \nexteg{}.
\begin{ex} 
$\ulcorner\lambda c$:\textit{Cntxt} . $\ulcorner\lambda
r$:\smallrecord{
  \smalltfield{x}{\textit{Ind}}\\
  \smalltfield{$\mathfrak{c}$}{\smallrecord{
      \footnotesize{\textit{Cntxt}}\\
      \smalltfield{$\mathfrak{s}$}{\smallrecord{
          \smalltfield{x$_0$}{\textit{Ind}}}}}}} . \record{
    \tfield{e}{like$^\dagger$($r$.x, $r.\mathfrak{c}.\mathfrak{s}$.x$_0$)}}$\urcorner\urcorner$
\end{ex} 
Applying \preveg{} to any context, $c$, will obtain \nexteg{}.
\begin{ex} 
 $\ulcorner\lambda
r$:\smallrecord{
  \smalltfield{x}{\textit{Ind}}\\
  \smalltfield{$\mathfrak{c}$}{\smallrecord{
      \footnotesize{\textit{Cntxt}}\\
      \smalltfield{$\mathfrak{s}$}{\smallrecord{
          \smalltfield{x$_0$}{\textit{Ind}}}}}}} . \record{
    \tfield{e}{like$^\dagger$($r$.x, $r.\mathfrak{c}.\mathfrak{s}$.x$_0$)}}$\urcorner$
\end{ex} 
Restricting \preveg{} by
`$\mathcal{F}(\textbf{farmer}^\frown\textbf{who}^\frown\textbf{owns}^\frown\textbf{a}^\frown\textbf{donkey})$',
yields \nexteg{}.
\begin{ex} 
$\ulcorner\lambda r$:\smallrecord{
    \smalltfield{x}{\textit{Ind}}\\
    \smalltfield{$\mathfrak{c}$}{\smallrecord{
        \footnotesize{\textit{Cntxt}}\\
        \smalltfield{$\mathfrak{s}$}{\smallrecord{
            %\footnotesize{\textit{Assgnmnt}}\\
            \smalltfield{x$_0$}{\textit{Ind}}}}\\
        \smalltfield{$\mathfrak{c}$}{\smallrecord{
            \smalltfield{f}{\textit{PropCntxt}}\\
            \smalltfield{a}{\textit{PropCntxt}}}}
}}\\
    \smalltfield{e$_1$}{farmer(x)}\\
    \smalltfield{e$_2$}{\smallrecord{
        \smalltfield{x}{$\mathfrak{T}$(donkey$'$)}\\
        \smalltfield{e}{own$^\dagger$($\Uparrow$x, x)}}}}
   .
  \record{
    \tfield{e}{like$^{\dagger}$($r$.x, $r.\mathfrak{c}.\mathfrak{s}.\text{x}_0$)}}$\urcorner$
\end{ex}
Aligning the paths `$\mathfrak{c}.\mathfrak{s}.\text{x}_0$' and
`e$_2$.x' in \preveg{} yields \nexteg{}.
\begin{ex} 
$\ulcorner\lambda r$:\smallrecord{
    \smalltfield{x}{\textit{Ind}}\\
    \smalltfield{$\mathfrak{c}$}{\smallrecord{
        \footnotesize{\textit{Cntxt}}\\
        \smalltfield{$\mathfrak{s}$}{\smallrecord{
            %\footnotesize{\textit{Assgnmnt}}\\
            \smallmfield{x$_0$}{$\Uparrow^2$e$_2$.x}{\textit{Ind}}}}\\
        \smalltfield{$\mathfrak{c}$}{\smallrecord{
            \smalltfield{f}{\textit{PropCntxt}}\\
            \smalltfield{a}{\textit{PropCntxt}}}}
}}\\
    \smalltfield{e$_1$}{farmer(x)}\\
    \smalltfield{e$_2$}{\smallrecord{
        \smalltfield{x}{$\mathfrak{T}$(donkey$'$)}\\
        \smalltfield{e}{own$^\dagger$($\Uparrow$x, x)}}}}
   .
  \record{
    \tfield{e}{like$^{\dagger}$($r$.x, $r.\mathfrak{c}.\mathfrak{s}.\text{x}_0$)}}$\urcorner$ 
\end{ex} 
Finally, in the `e.f'-field in (\ref{ex:efwoadli-classic})
`$\mathfrak{P}^\forall$' is applied to the scope, that is, \preveg{}.
Thus `$\mathfrak{P}^\forall(\text{scope})$' represents \nexteg{}.
\begin{ex} 
 $\ulcorner\lambda r$:\smallrecord{
    \smalltfield{x}{\textit{Ind}}} . (($r'$:\smallrecord{
    \smallmfield{x}{$r$.x}{\textit{Ind}}\\
    \smalltfield{$\mathfrak{c}$}{\smallrecord{
        \footnotesize{\textit{Cntxt}}\\
        \smalltfield{$\mathfrak{s}$}{\smallrecord{
            %\footnotesize{\textit{Assgnmnt}}\\
            \smallmfield{x$_0$}{$\Uparrow^2$e$_2$.x}{\textit{Ind}}}}\\
        \smalltfield{$\mathfrak{c}$}{\smallrecord{
            \smalltfield{f}{\textit{PropCntxt}}\\
            \smalltfield{a}{\textit{PropCntxt}}}}
}}\\
    \smalltfield{e$_1$}{farmer(x)}\\
    \smalltfield{e$_2$}{\smallrecord{
        \smalltfield{x}{$\mathfrak{T}$(donkey$'$)}\\
        \smalltfield{e}{own$^\dagger$($\Uparrow$x, x)}}}})$\rightarrow$
  \record{
    \tfield{e}{like$^{\dagger}$($r'$.x, $r'.\mathfrak{c}.\mathfrak{s}.\text{x}_0$)}}$)\urcorner$
\end{ex} 
\preveg{} can be paraphrased as ``the property of being an individual,
$x$, if $x$ is a farmer and owns a donkey, $y$, then $x$ likes $y$''
thus requiring that every farmer likes all of the donkeys she owns.
The whole sentence says that every farmer who owns a donkey has this property.  


% \begin{ex} 
% $\lambda\mathfrak{s}$:\textit{Rec} 
%   . \smallrecord{\smalltfield{e}{\record{\tfield{X}{every$^w$(\textbf{farmer$^\frown$who$^\frown$owns$^\frown$a$^\frown$donkey}($\mathfrak{s}$))}\\
%                            \tfield{f}{(($x:\mathfrak{T}(\text{X}))\rightarrow$\\
% &&\hspace*{2em}(($r$:\smallrecord{\smallmfield{x}{$x$}{\textit{Ind}}\\
%                                                                \smalltfield{e$_1$}{farmer$'$\{x\}}\\
%                                                                \smalltfield{e$_2$}{\smallrecord{\smalltfield{e}{\smallrecord{\smalltfield{x}{$\mathfrak{T}$(donkey$'$)}\\
%                                                                                                 \smalltfield{e}{\smallrecord{\smalltfield{$\mathfrak{c}$}{\smallrecord{\smallmfield{x}{$\Uparrow^2$x}{\textit{Ind}}\\
%                                                                                                                                                                        \smalltfield{e}{donkey(x)}}}\\
%                                                                                                      \smalltfield{e}{\smallrecord{\smalltfield{e}{have$^\dagger$($\Uparrow^4$x,
%                                                                                                            $\Uparrow\!\mathfrak{c}$.x)}}}}}}}}}})
%                                                                                        $\rightarrow$\\
%                &&\hspace*{5em}\smallrecord{\smalltfield{e}{like$^\dagger$($x$, $r$.e$_2$.e.x)}}
% ))}}}}  
% \end{ex} 
% \preveg{} requires that the function labelled `f' maps any farmer who
% owns a donkey to a function which maps any situation in which that
% farmer owns a donkey to a situation in which the farmer likes the
% donkey.  Since for any donkey the farmer owns there will be a
% situation in which the farmer owns that donkey, this has the effect of
% universal quantification over the farmer's donkeys.  The difference
% between strong and weak readings is, however, cashed out in terms of
% whether we map farmers to functions from ``donkey owning'' situations
% to ``liking'' situations or to situations where there is ``donkey
% owning'' and ``liking''.

 

The domain of the function introduced is farmers who own a donkey in
our analysis of both the weak and the strong readings.  This is in
contrast to the classical treatment of the strong reading
\citep{Geach1962,KampReyle1993} which can be construed as 
quantification over pairs of farmers and donkeys.  This works in the
case of universal quantification. The sentence \textit{every farmer
  who owns a donkey likes it} can be construed as ``every
farmer-donkey pair such that the farmer owns the donkey is such that
the farmer likes the donkey''.  However, for other generalized
quantifiers, such as \textit{most} this represents an incorrect
paraphrase.  Thus \textit{most farmers who own a donkey like it}
cannot be construed as ``most farmer-donkey pairs such that the farmer
owns the donkey are such that the farmer likes the donkey''.
\cite{Chierchia1995} gives a good account of why this is so.  Suppose
we have five farmers four of which have exactly one donkey and do not
like it.  The fifth farmer has fifty donkeys and likes them all.
Clearly in this case most farmer-donkey pairs are such that the farmer
likes the donkey but it is not the case that most farmers who own a
donkey like it.  This problem is known in the literature as the
\textit{proportion problem}.  The treatment that we have proposed here
does not suffer from it since the quantification is over farmers who
own a donkey rather than farmer-donkey pairs. 
    
% In order to get this to work out in a compositional way we will use
% the kind of quantifier anatomy  that we introduced in
% Chapters~\ref{ch:gram} and~\ref{ch:commonnouns} except that we will
% use parametric properties instead of non-parametric properties.  Let
% us introduce some abbreviatory notation [should introduce from the
% beginning!].  Suppose that $q$ is a quantifier relation,
% $\mathcal{P}$ and $\mathcal{Q}$ are parametric properties and
% $\mathfrak{s}$ is a context of the kind we have been using in this
% chapter, then we use $\mathfrak{Q}(q,\mathcal{P},\mathcal{Q},\mathfrak{s})$ to represent
% \nexteg{}.
% \begin{ex} 
% \record{\mfield{restr}{$\mathcal{P}$}{\textit{PPpty}}\\
%         \mfield{scope}{$\mathcal{Q}$}{\textit{PPpty}}\\
%         \tfield{e}{$q$(restr($\mathfrak{s}$), scope($\mathfrak{s}$))}} 
% \end{ex} 
% Instead of (\ref{ex:no-dog-which-chases-a-cat}) we will have
% \nexteg{a}, which is identical with \nexteg{b}.
% \begin{ex} 
% \begin{subex} 
 
% \item $\lambda\mathfrak{s}$:\textit{Rec}
%   $\lambda\mathcal{P}$:\textit{PPpty}
%   . $\mathfrak{Q}$(no, \textbf{dog$^\frown$which$^\frown$chase$^\frown$a$^\frown$cat},
%   $\mathcal{P}$, $\mathfrak{s}$)

 
% \item $\lambda\mathfrak{s}$:\textit{Rec}
%   $\lambda\mathcal{P}$:\textit{PPpty}
%   . \record{\mfield{restr}{\textbf{dog$^\frown$which$^\frown$chase$^\frown$a$^\frown$cat}}{\textit{PPpty}}\\
%             \mfield{scope}{$\mathcal{P}$}{\textit{PPpty}}\\
%             \tfield{e}{no(restr($\mathfrak{s}$), scope($\mathfrak{s}$))}} 
% % \record{\mfield{restr}{\textbf{dog$^\frown$which$^\frown$chase$^\frown$a$^\frown$cat}}{\textit{PPpty}}\\
% %             \mfield{scope}{$\mathcal{P}$}{\textit{PPpty}}\\
% %             \tfield{e}{\record{\tfield{X}{every$^w$($\Uparrow$restr($\mathfrak{s}$))}\\
% %                                \tfield{f}{(($x:\mathfrak{T}(\text{X}))\rightarrow\neg\mathfrak{P}$($\Uparrow$scope($\mathfrak{s}$)))\{x\}}}}}
                         
 
% \end{subex} 
   
% \end{ex} 
% We will now use
% \textbf{no$^\frown$dog$^\frown$which$^\frown$chase$^\frown$a$^\frown$cat}
% to refer to \preveg{} rather than
% (\ref{ex:no-dog-which-chases-a-cat}).  Note that we have not
% restricted the scope (that is the second argument of the quantifier
% relation) by the fixed point type of the restriction (that is the
% first argument of the quantifier relation).  Now that we have put the
% parametric properties used to form the arguments to the quantifier
% relation in their own fields (`restr' and `scope') we can have access
% to them at the point at which we combine the noun-phrase
% interpretation with that of another constituent and can at that point
% make modifications to the second argument.  Since we have the
% arguments to the quantifier relation accessible via the labels `restr'
% and `scope' 


     
\paragraph{\textit{Sam likes him/himself}}
The treatment of anaphora that we have presented so far overgenerates
in that it does not take account of the restrictions on anaphoric
possibilities which exist in natural languages.  For example, the
sentence \nexteg{} does not allow a reading in which
\textit{him} is anaphorically related to \textit{Sam}.
\begin{ex} 
Sam likes him 
\end{ex} 
Note that this is not the same as saying that \textit{Sam} and
\textit{him} are not allowed to refer to the same individual.  There
are cases where this is possible, though often bizarre.  Suppose that
Sam finds a wikipedia page describing a person with interests and
achievements very similar to his own but he does not realize that in
fact somebody has written a page about Sam himself.  Sam can decide
that he likes the person described in the entry and this can be
described by \preveg{}.  % An similar example well-known in the
% literature from \cite{Lakoff1970} is \nexteg{}.
% \begin{ex} 
% I dreamt that I was Brigitte Bardot and that I kissed me 
% \end{ex}
The basic technique that we employ for making this distinction is
illustrated by the types in \nexteg{}.
\begin{ex} 
  \begin{subex} 
 
\item \record{
    \tfield{x}{\textit{Ind}}\\
    \tfield{c}{named(x, ``Sam'')}\\
    \tfield{y}{\textit{Ind}}\\
    \tfield{e}{like(x, y)}} 
 
\item \record{
    \tfield{x}{\textit{Ind}}\\
    \tfield{c}{named(x, ``Sam'')}\\
    \mfield{y}{x}{\textit{Ind}}\\
    \tfield{e}{like(x, y)}}
  
\item \record{
    \tfield{x}{\textit{Ind}}\\
    \tfield{c}{named(x, ``Sam'')}\\
    \tfield{e}{like(x, x)}} 
 
\end{subex} 
  
  
\end{ex} 
In \preveg{a} we have two fields for individuals labelled by `x' and
`y' respectively.  There is nothing to prevent these fields from being
filled by the same individual in some of the witnesses for the type.
In \preveg{b}, however, the fields will be filled by the same
individual in any witness for the type.  A further possibility is to
have just one field for an individual as in \preveg{c}. Let us see how
these options play out in actual parametric contents for \textit{Sam
  likes him} resulting currently in a content where \textit{him} is
anaphorically related to \textit{Sam}.  A parametric content for
\textit{likes him} is \nexteg{}, parallel to contents for similar verb
phrases we have seen previously.
\begin{ex} 
  $\ulcorner\lambda c$:\smallrecord{
    \footnotesize{\textit{Cntxt}}\\
    \smalltfield{$\mathfrak{s}$}{\smallrecord{
        \smalltfield{x$_0$}{\textit{Ind}}}}\\
    \smalltfield{$\mathfrak{c}$}{\smallrecord{
        \smalltfield{f}{\textit{PropCntxt}}\\
        \smalltfield{a}{\textit{PropCntxt}}}}} . $\ulcorner\lambda
  r$:\smallrecord{
    \smalltfield{x}{\textit{Ind}}} . \record{
    \tfield{e}{like$^\dagger$($r$.x,
      $c.\mathfrak{s}.\text{x}_0$)}}$\urcorner\urcorner$
  \label{ex:likes-him}
\end{ex} 
Using the lexical rule for proper nouns we obtain the parametric
content \nexteg{} for \textit{Sam}.
\begin{ex} 
  $\ulcorner\lambda c$:\smallrecord{
    \footnotesize{\textit{Cntxt}}\\
    \smalltfield{$\mathfrak{c}$}{\smallrecord{
        \smalltfield{x}{\textit{Ind}}\\
        \smalltfield{e}{named(x, ``Sam'')}}}} . $\lambda
  P$:\textit{Ppty} . $P\{c.\mathfrak{c}.\text{x}\}\urcorner$
\end{ex} 
Let us represent \preveg{} as \textbf{Sam}.  Using
`$\mathrm{storage}$' we obtain an additioinal parametric content given
in \nexteg{}.
\begin{ex} 
  $\ulcorner\lambda c$:\smallrecord{
    \footnotesize{\textit{Cntxt}}\\
    \smalltfield{$\mathfrak{q}$}{\smallrecord{
        \smallmfield{x$_0$}{\textbf{Sam}}{\textit{PQuant}}}}\\
    \smalltfield{$\mathfrak{s}$}{\smallrecord{
        \smalltfield{x$_0$}{\textit{Ind}}}}} . $\lambda
  P$:\textit{Ppty} . $P\{c.\mathfrak{s}.\text{x}_0\}\urcorner$
\end{ex} 
We can now obtain a parametric content for \textit{Sam likes him} by
using \preveg{} and (\ref{ex:likes-him}) combined with `@$_{0,1}$'.
This is given in \nexteg{}.
\begin{ex} 
  $\ulcorner\lambda c$:\smallrecord{
    \footnotesize{\textit{Cntxt}}\\
    \smalltfield{$\mathfrak{q}$}{\smallrecord{
        \smallmfield{x$_0$}{\textbf{Sam}}{\textit{PQuant}}}}\\
    \smalltfield{$\mathfrak{s}$}{\smallrecord{
        \smalltfield{x$_0$}{\textit{Ind}}}}\\
    \smalltfield{$\mathfrak{c}$}{\smallrecord{
        \smalltfield{f}{\textit{PropCntxt}}\\
        \smalltfield{a}{\textit{PropCntxt}}}}} . \record{
    \tfield{e}{like$^\dagger$($c.\mathfrak{s}.\text{x}_0,c.\mathfrak{s}.\text{x}_0$)}}$\urcorner$
\end{ex} 
We can now apply `$\mathrm{retrieve}$' to `x$_0$' and \preveg{} to
obtain \nexteg{}.
\begin{ex} 
  $\ulcorner\lambda c$:\smallrecord{
    \smalltfield{$\mathfrak{c}$}{\smallrecord{
        \smalltfield{f}{\smallrecord{
            \smalltfield{x}{\textit{Ind}}\\
            \smalltfield{e}{named(x, ``Sam'')}}}\\
        \smalltfield{a}{\smallrecord{
            \smalltfield{f}{\textit{PropCntxt}}\\
            \smalltfield{a}{\textit{PropCntxt}}}}}}} . \record{
    \tfield{e}{like$^\dagger$($c.\mathfrak{c}$.f.x, $c.\mathfrak{c}$.f.x)}}$\urcorner$
\end{ex} 
\preveg{} is not an appropriate content for \textit{Sam likes him}
although it would be appropriate for \textit{Sam likes himself}.
Simplifying a good deal,  pronouns which are not reflexive (like
\textit{himself}) cannot be anaphorically related to an antecedent
within the same clause.  This is Principle B of Chomsky's binding
theory \citep{Chomsky1981}.  We shall treat this by adding a field
labelled `$\mathfrak{l}$' in the context requiring an assignment which keeps track of pronouns
which are local. We adjust the definition of \textit{Cntxt} to be
\nexteg{}.
\begin{ex} 
\record{
    \tfield{$\mathfrak{q}$}{\textit{QStore}}\\
    \tfield{$\mathfrak{s}$}{\textit{Assgnmnt}}\\
    \tfield{$\mathfrak{l}$}{\textit{Assgnmnt}}\\
    \tfield{$\mathfrak{w}$}{\textit{Assgnmnt}}\\
    \tfield{$\mathfrak{g}$}{\textit{Assgnmnt}}\\
    \tfield{$\mathfrak{c}$}{\textit{PropCntxt}}} 
\end{ex}
The generating parametric content for \textit{he} is \nexteg{} where we
mark `x$_0$' in the $\mathfrak{l}$-field.
\begin{ex} 
  $\ulcorner\lambda c$:\smallrecord{
    \footnotesize{\textit{Cntxt}}\\
    \smalltfield{$\mathfrak{s}$}{\smallrecord{
        \smalltfield{x$_0$}{\textit{Ind}}}}\\
    \smalltfield{$\mathfrak{l}$}{\smallrecord{
        \smalltfield{x$_0$}{\textit{Ind}}}}} . $\lambda
  P$:\textit{Ppty} . $P\{c.\mathfrak{s}.\text{x}_0\}\urcorner$
\end{ex} 
We adjust (\ref{ex:combine-align}) to include reference to paths
`$\mathfrak{l}$.x$_i$' by adding the boxed material as in \nexteg{}.
\begin{ex} 
If $\alpha$ : \smallrecord{\smalltfield{bg}{\textit{CntxtType}}\\
                           \smalltfield{fg}{(bg$\rightarrow$($T_1\rightarrow
                             T_2$))}} 
and $\beta$ : \smallrecord{\smalltfield{bg}{\textit{CntxtType}}\\
  \smalltfield{fg}{(bg$\rightarrow T_1$)}}
and $\alpha.\text{bg}\sqsubseteq$ \smallrecord{
    \smalltfield{$\mathfrak{s}$}{\smallrecord{
        \smalltfield{x$_i$}{\textit{Ind}}}}} and
  $\mathrm{incr}(\beta.\text{bg},\alpha.\text{bg})\sqsubseteq$
  \smallrecord{
    \smalltfield{$\mathfrak{s}$}{\smallrecord{
        \smalltfield{x$_j$}{\textit{Ind}}}}} but $\mathrm{incr}(\beta.\text{bg},\alpha.\text{bg})\not\sqsubseteq$
  \smallrecord{
    \smalltfield{$\mathfrak{q}$}{\smallrecord{
        \smalltfield{x$_j$}{\textit{PQuant}}}}} \fbox{and $\mathrm{incr}(\beta.\text{bg},\alpha.\text{bg})\not\sqsubseteq$
  \smallrecord{
    \smalltfield{$\mathfrak{l}$}{\smallrecord{
        \smalltfield{x$_j$}{\textit{PQuant}}}}}},
                         then the \textit{combination of $\alpha$ and
    $\beta$  based on functional application and anaphoric
      relation of $j$ to $i$}, $\alpha\text{@}_{i,j}\beta$, is
  \begin{quote}
    $\ulcorner\lambda c$:$[\alpha.\text{bg}]_{\mathfrak{c}\leadsto\mathfrak{c}.\text{f}}$
      \d{$\wedge$}$[\mathrm{incr}([\beta.\text{bg}]_{\mathfrak{c}\leadsto\mathfrak{c}.\text{a}},\alpha.\text{bg})]_{\mathfrak{s}.\text{x}_j\leadsto\mathfrak{s}.\text{x}_i}$
      . \\ \hspace*{2em}$[\alpha]_{\mathfrak{c}\leadsto\mathfrak{c}.\text{f}}(c)([\mathrm{incr}([\beta.\text{fg}]_{\mathfrak{c}\leadsto\mathfrak{c}.\text{a}},\alpha.\text{bg})]_{\mathfrak{s}.\text{x}_j\leadsto\mathfrak{s}.\text{x}_i}(c))\urcorner$
\end{quote}

\label{ex:combine-align-local}

\end{ex} 
The information about locality represented by the
`$\mathfrak{l}$'-field will now percolate up as a constraint on the
context as we combine consituents.  However, once we reach a sentence
we no longer want the pronoun to count as local since pronouns can be
anaphorically related to antecedents outside the clause in which they
occur as in \textit{Sam$_i$ thinks that she$_i$ is lucky}. We define
an operation $B$ on parametric contents (think of Principle ``B'' or
``Boundary'') which uses asymmetric merge to remove the locality
constraint.  $B$ is defined in \nexteg{}.
\begin{ex} 
  If $\alpha$ is a parametric content,
  \begin{quote}
    $\ulcorner\lambda c\!:\!T\ .\ \varphi\dep{c}\urcorner$
  \end{quote}
  then $B(\alpha)$ is
  \begin{quote}
    $\ulcorner\lambda c$:$T$\fbox{\d{$\wedge$}}\smallrecord{
      \smalltfield{$\mathfrak{l}$}{\textit{Assgnmnt}}}
    . $\varphi\dep{c}\urcorner$\footnote{This assumes that paremtric
      contents are defined in such as way that $\varphi$ will not
      depend on $c.\mathfrak{l}.\pi$ for any $\pi$.  Otherwise
      $B(\alpha)$ would need to be
      \begin{quote}
        $\ulcorner\lambda c$:$T$\fbox{\d{$\wedge$}}\smallrecord{
          \smalltfield{$\mathfrak{l}$}{\textit{Assgnmnt}}}
        . $[\varphi\dep{c}]_{c.\mathfrak{l}.\pi\leadsto c.\mathfrak{s}.\pi}\urcorner$
      \end{quote}}
      
    
  \end{quote}
  
\end{ex}
Suppose that $T_1$ and $T_2$ are of type \textit{ContType} and that
$\mathcal{O}$ is a combination operation such as @ etc., then we say
that $T_1\mathcal{O}^{\mathfrak{S},B}T_2$ is also a type with the
witness condition in \nexteg{}.
\begin{ex} 
If $\alpha:T_1$, $\beta:T_2$ and $\alpha\mathcal{O}\beta$ is defined,
then $B(\alpha\mathcal{O}\beta):T_1\mathcal{O}^{\mathfrak{S},B}T_2$.
Nothing else is a witness for $T_1\mathcal{O}^{\mathfrak{S},B}T_2$. 
\end{ex} 
In \nexteg{} we introduce versions of `ContForwardApp' operations
which involve $B$.
\begin{ex} 
  If $\mathcal{O}$ is a combination operator, then
  ContForwardApp$_{\mathfrak{S},\mathcal{O},B}$ is
  \begin{quote}
    $\lambda u$:\smallrecord{
      \smalltfield{cont}{\textit{ContType}}}$^\frown$\smallrecord{
      \smalltfield{cont}{\textit{ContType}}} . \smallrecord{
      \smalltfield{cont}{$\mathfrak{S}(u[0].\text{cont}\mathcal{O}^{\mathfrak{S},B}u[1].\text{cont})$}}
  \end{quote}
  \end{ex} 
We then introduce a notation for constituent structure rules involving
locality boundaries as in \nexteg{}.
\begin{ex} 
If $T_{\text{mother}}$, $T_{\text{daughter}_1}$ and
  $T_{\text{daughter}_2}$ are sign types and $\mathcal{O}$ is a
  combination operation, then
  \begin{quote}
    $T_{\text{mother}}\longrightarrow
    T_{\text{daughter}_1}\ T_{\text{daughter}_2}\ \mid\
    B(T_{\text{daughter}_1}'(_{\mathcal{O}}T_{\text{daughter}_2}'))$
  \end{quote}
  is
  \begin{quote}
  $T_{\text{mother}}\longrightarrow
    T_{\text{daughter}_1}\ T_{\text{daughter}_2}$ \d{\d{$\wedge$}}
    ContForwardApp$_{\mathfrak{S},\mathcal{O},B}$
  \end{quote}
  
\end{ex} 
Thus for example we can formulate the constituent rule that says
that a sentence can consist of a noun phrase followed by a verb phrase
as \nexteg{}.
\begin{ex} 
  \textit{S} $\longrightarrow$ \textit{NP VP} $\mid$ $B$(\textit{NP}$'$($_{\text{@}}$\textit{VP}$'$))
\end{ex}
  
Reflexive pronouns, like \textit{himself}, obey an almost
complementary principle to non-reflexive pronouns -- they must be
anaphorically related to a local antecedent and cannot be related to a
non-local antecedent.  In addition, they must be related to some
antecedent.  They cannot simply refer to something in the context as
ordinary pronouns can.  In Chomsky's binding theory such pronouns are
called ``anaphors'' and include both reflexive pronouns and
reciprocals such as \textit{each other}.  In order to handle such
requirements on local anaphora we will add another component to the
context type under the label $\mathfrak{r}$ (``reflexive''). Thus the
type \textit{Cntxt} will  now be defined as in \nexteg{}.
\begin{ex} 
\record{
    \tfield{$\mathfrak{q}$}{\textit{QStore}}\\
    \tfield{$\mathfrak{s}$}{\textit{Assgnmnt}}\\
    \tfield{$\mathfrak{l}$}{\textit{Assgnmnt}}\\
    \tfield{$\mathfrak{r}$}{\textit{Assgnmnt}}\\
    \tfield{$\mathfrak{w}$}{\textit{Assgnmnt}}\\
    \tfield{$\mathfrak{g}$}{\textit{Assgnmnt}}\\
    \tfield{$\mathfrak{c}$}{\textit{PropCntxt}}} 
\end{ex}
The generating parametric content for \textit{himself} is \nexteg{}
where we mark `x$_0$' in the $\mathfrak{r}$-field.
\begin{ex} 
  $\ulcorner\lambda c$:\smallrecord{
    \footnotesize{\textit{Cntxt}}\\
    \smalltfield{$\mathfrak{s}$}{\smallrecord{
        \smalltfield{x$_0$}{\textit{Ind}}}}\\
    \smalltfield{$\mathfrak{r}$}{\smallrecord{
        \smalltfield{x$_0$}{\textit{Ind}}}}} . $\lambda
  P$:\textit{Ppty} . $P\{c.\mathfrak{s}.\text{x}_0\}\urcorner$
\end{ex}
The reflexive marking in the context will percolate up to contexts
associated with higher phrases.  We provide a mechanism for removing
the marking in properties and simultaneously binding the reflexive
pronoun.  This is given in \nexteg{}.



\begin{ex} 
  If $\mathcal{P}$ is a parametric property of the form
  \begin{quote}
    $\ulcorner\lambda c\!:\!T_1\ .\ \ulcorner\lambda
    r\!:\!T_2\dep{T_1}\ .\ \varphi\dep{T_1,T_2}\urcorner\urcorner$
  \end{quote}
  where for some natural number $i$, $T_1\sqsubseteq$ \smallrecord{
    \smalltfield{$\mathfrak{r}$}{\smallrecord{
        \smalltfield{x$_i$}{\textit{Ind}}}}}, then \textit{the reflexivization
  of} $\mathcal{P}$, $\mathfrak{R}(\mathcal{P})$, is
  \begin{quote}
    $\ulcorner\lambda c$:($T_1$\fbox{\d{$\wedge$}}\smallrecord{
      \smalltfield{$\mathfrak{r}$}{\textit{Assgnmnt}}})$\ominus\mathfrak{s}.\text{x}_i$
    . $\ulcorner\lambda r$:$T_2$
    . $[\varphi]_{c.\mathfrak{s}.\text{x}_i\leadsto
      r.\text{x}}\urcorner\urcorner$
  \end{quote}
  
\end{ex}


This operation removes the reflexive marking in the
$\mathfrak{r}$-field of the context using asymmetric merge and also
removes the corresponding path `$\mathfrak{s}.\text{x}_i$' from the
context type so there is no dependence on an individual labelled
`x$_i$' in the context.  Any dependence on `$\mathfrak{s}$.x$_i$' in
the body of the property represented by $\varphi$ is replaced by a
dependence on `x' in the domain of the property --- note that the
domain type of the property, $T_2$, is guaranteed to be a record type
with `x' among its labels by the requirement that $\mathcal{P}$ is a
parametric property.

The operation, $\mathfrak{R}$, gives us a way of binding reflexive
pronouns with verb phrases by the subject of the sentence.  It does
not, however, allow us to have a reflexive bound within a verb-phrase
as in examples like \textit{The guru revealed Kim$_i$ to
  himself$_i$}.  Nor does it correctly require that both occurrences
of \textit{himself} have to be anaphorically related to \textit{the
  guru} in \textit{The guru revealed himself to himself}.\footnote{In
  other languages such as German and Scandinavian languages, these
  examples involve a different reflexive construction.  (See, for
  example, \citealp{Hellan1986}.)}  We shall not deal with such cases
here.

We will, however, introduce a mechanism for requiring that the
reflexive pronoun must be anaphorically related to something.  It
cannot be left as free and dependent on the context like a regular
pronoun.  Our strategy for doing this involves mapping a parametric
content type to a subtype which excludes witnesses which have a
reflexive marking involving the label $\mathfrak{r}$ in the context
type.  For any parametric content type, $T$, we characterize a
subtype, $\mathfrak{A}(T)$, which excludes parametric contents with
free reflexives as indicated by the $\mathfrak{r}$-field
($\mathfrak{A}$ for ``Principle A'').  This is characterized in \nexteg{}.
\begin{ex} 
If $T$ is a parametric content type, then $\mathfrak{A}(T)$ is also a
parametric content type.

$\varphi:\mathfrak{A}(T)$ iff $\varphi:T$ and $\varphi$.bg
$\not\sqsubseteq$ \smallrecord{
  \smalltfield{$\mathfrak{r}$}{\smallrecord{
      \smalltfield{x$_i$}{\textit{Ind}}}}}, for any natural number $i$.
\end{ex} 
In \nexteg{} we introduce versions of `ContForwardApp' operations
which involve $\mathfrak{A}$.
\begin{ex} 
  If $\mathcal{O}$ is a combination operator, then
  ContForwardApp$_{\mathfrak{S},\mathcal{O},\mathfrak{A}}$ is
  \begin{quote}
    $\lambda u$:\smallrecord{
      \smalltfield{cont}{\textit{ContType}}}$^\frown$\smallrecord{
      \smalltfield{cont}{\textit{ContType}}} . \smallrecord{
      \smalltfield{cont}{$\mathfrak{S}(\mathfrak{A}(u[0].\text{cont}\mathcal{O}^{\mathfrak{S}}u[1].\text{cont}))$}}
  \end{quote}
  \end{ex} 
We then introduce a notation for constituent structure rules involving
locality boundaries as in \nexteg{}.
\begin{ex} 
If $T_{\text{mother}}$, $T_{\text{daughter}_1}$ and
  $T_{\text{daughter}_2}$ are sign types and $\mathcal{O}$ is a
  combination operation, then
  \begin{quote}
    $T_{\text{mother}}\longrightarrow
    T_{\text{daughter}_1}\ T_{\text{daughter}_2}\ \mid\
    \mathfrak{A}(T_{\text{daughter}_1}'(_{\mathcal{O}}T_{\text{daughter}_2}'))$
  \end{quote}
  is
  \begin{quote}
  $T_{\text{mother}}\longrightarrow
    T_{\text{daughter}_1}\ T_{\text{daughter}_2}$ \d{\d{$\wedge$}}
    ContForwardApp$_{\mathfrak{S},\mathcal{O},\mathfrak{A}}$
  \end{quote}
  
\end{ex} 
Thus for example we can formulate the constituent rule that says
that a sentence can consist of a noun phrase followed by a verb phrase
as \nexteg{}.
\begin{ex} 
  \textit{VP} $\longrightarrow$ \textit{V NP} $\mid$ $\mathfrak{A}$(\textit{V}$'$($_{\text{@}}$\textit{NP}$'$))
\end{ex} 
Introducing the $\mathfrak{A}$-locality boundary at the VP-level as in
\preveg{} will ensure that all reflexives will be anaphorically
related within a clause.

In order to include reflexives we need to extend our characterization
of $\mathfrak{S}$ to include reflexive parametric contents as
indicated by the boxed text in \nexteg{}.

\begin{ex} 
\begin{subex} 
 
\item If $T:\textit{ContType}$, then $\mathfrak{S}(T)$ is a type 
 
\item The witnesses of $\mathfrak{S}(T)$ are characterized by
  \begin{enumerate} 
 
  \item if $\varphi:T$ then $\varphi:\mathfrak{S}(T)$
    
  \item if $\varphi:\mathfrak{S}(T)$ and $\varphi:\textit{PPpty}$,
    then $\mathcal{L}(\varphi):\mathfrak{S}(T)$
    
  \item \fbox{if $\varphi:\mathfrak{S}(T)$ and $\varphi:\textit{PPpty}$,
    then $\mathfrak{R}(\varphi):\mathfrak{S}(T)$}
    
  \item \begin{minipage}[t]{.9\linewidth}if $\varphi:\mathfrak{S}(T)$,
    $\varphi\sqsubseteq\text{\smallrecord{
        \smallmfield{scope}{$\psi$}{\textit{Ppty}}}}$ and
    $\pi_1,\pi_2\in\mathrm{paths}(\psi.\text{bg})$, then $\varphi[\text{scope}=\psi_{\pi_1=\pi_2}:\textit{Ppty}]:\mathfrak{S}(T)$\end{minipage}

    
  \item if $\alpha\mathcal{O}\beta:\mathfrak{S}(T)$, (for
      some combination operation, $\mathcal{O}$) and
      $\alpha\mathcal{O}_{i,j}\beta$ is defined (for some natural
      numbers, $i$ and $j$), then $\alpha\mathcal{O}_{i,j}\beta:\mathfrak{S}(T)$
 
  \item if $\varphi:\mathfrak{S}(T)$ and $\varphi$ is in the range of `$\mathrm{storage}$', then
    $\mathrm{storage}(\varphi):\mathfrak{S}(T)$

  \item if $\varphi:\mathfrak{S}(T)$ and `x$_i$' and $\varphi$ are appropriate
    arguments to `$\mathrm{retrieve}$', then
    $\mathrm{retrieve}(\text{x}_i,\varphi):\mathfrak{S}(T)$
    
  \item nothing is a witness for $\mathfrak{S}(T)$ except as required above.
 
  \end{enumerate} 
  
 
\end{subex} 
\label{ex:storage-reflexive}   
\end{ex}

This is only the beginning of a theory of
reflexives in English.  For example, it will not on its own prevent reflexives
occuring in subject position as in \textit{$^*$Himself saw Sam} or
\textit{$^*$Kim feels that herself is welcome}. Perhaps this is simply
a matter of case, which we have not treated here.  There is just no
nominative version of the reflexive which can be used as the subject
of a tensed sentence.  This is perhaps suggested by the acceptability
of \textit{Kim feels herself to be welcome} which seems to express the
same content.
A further example, well-known from the literature, concerns what are
known as picture noun-phrases.  While this treatment will allow for a correct
interpretation of \textit{Kim found a picture of herself} where
\textit{Kim} is the antecedent of \textit{herself} it will not
correctly account for \textit{Kim found Sam's picture of herself}
where the only possible antecedent for \textit{herself} is
\textit{Sam}.  We will leave a more complete treatment of reflexives for future exploration.



% The parametric content for \textit{him} is given in \nexteg{}.

% \begin{ex} 
% $\lambda
% c$:\smallrecord{\smalltfield{$\mathfrak{s}$}{\smallrecord{\smalltfield{x$_0$}{\textit{Ind}}}}\\
%                 \smallmfield{local}{$\mathfrak{T}$(\{$\mathfrak{s}$.x$_0$\})}{\textit{Type}}\\
% \smallmfield{refl}{$\mathfrak{T}$($\emptyset$)}{\textit{Type}}} . 
% $\lambda P$:\textit{Ppty} . $P$\{c.$\mathfrak{s}$.x$_0$\}
% \end{ex} 

% The parametric content of the reflexive pronoun \textit{himself} is
% the same as \preveg{} except that the values associated with the
% labels `local' and `refl' are switched, as shown in \nexteg{}.

% \begin{ex} 
% $\lambda
% c$:\smallrecord{\smalltfield{$\mathfrak{s}$}{\smallrecord{\smalltfield{x$_0$}{\textit{Ind}}}}\\
%                 \smallmfield{local}{$\mathfrak{T}$($\emptyset$)}{\textit{Type}}\\
% \smallmfield{refl}{$\mathfrak{T}$(\{$\mathfrak{s}$.x$_0$\})}{\textit{Type}}}
%  . 
% $\lambda P$:\textit{Ppty} . $P$\{c.$\mathfrak{s}$.x$_0$\}
% \end{ex} 
% We will refer to \preveg{} as \textbf{himself}.  The idea is that the `local'-field in the context must be passed up to
% the VP unchanged, whereas the `refl'-field must be emptied
% (discharged) at the VP-level.  Thus \textit{likes him} has the
% parametric content in \nexteg{a} whereas \textit{likes himself} has
% the parametric content in \nexteg{b}.
% \begin{ex} 
% \begin{subex} 
 
% \item $\lambda
% c$:\smallrecord{\smalltfield{$\mathfrak{s}$}{\smallrecord{\smalltfield{x$_0$}{\textit{Ind}}}}\\
%                 \smallmfield{local}{$\mathfrak{T}$(\{$\mathfrak{s}$.x$_0$\})}{\textit{Type}}\\
% \smallmfield{refl}{$\mathfrak{T}$($\emptyset$)}{\textit{Type}}} . 
% $\lambda r$:\smallrecord{\smalltfield{x}{\textit{Ind}}} . \record{\tfield{e}{like($r$.x,
% $\lambda P$:\textit{Ppty} . $P$\{$c$.$\mathfrak{s}$.x$_0$\})}} 
 
% \item $\lambda
% c$:\smallrecord{\smallmfield{local}{$\mathfrak{T}$($\emptyset$)}{\textit{Type}}\\
%   \smallmfield{refl}{$\mathfrak{T}$($\emptyset$)}{\textit{Type}}} . 
% $\lambda r$:\smallrecord{\smalltfield{x}{\textit{Ind}}} . \record{\tfield{e}{like($r$.x,
%   \textbf{himself}(\smallrecord{\field{$\mathfrak{s}$}{\smallrecord{\field{x$_0$}{$r$.x}}}\\
%                                 \field{local}{$\mathfrak{T}$($\emptyset$)}\\
%                                 \field{refl}{$\mathfrak{T}$(\{$r$.x\})}}))}}
 
 
% \end{subex} 
   
% \end{ex}
% The idea is that context parameters represented by `$\mathfrak{s}$'
% are open for binding unless object supplying that parameter are
% required to be of type `$c$.local'.  Reflexive parameters are, however,
% required to be of type `$c$.refl' and these must be bound within the VP.
% Note that since the reflexive is bound by `$r$.x' in \preveg{b} it is
% not represented in `$c$.refl' in the context for the VP.  Thus
% \textit{himself} on its own requires an individual to be supplied by
% the context to give it a referent.  However, for the content of
% \textit{likes himself} no such individual is required from the context
% since the property which is the content of the VP is that of ``being
% an $x$ such that $x$ likes $x$''.  [Better way to talk about this?].

% [Possible revision of treatment of proper names:]
% The content of an utterance of \textit{Sam} is given in \nexteg{}.
% \begin{ex} 
% $\lambda
% c$:\smallrecord{\smalltfield{$\mathfrak{c}$}{\smallrecord{\smalltfield{id$_0$}{\smallrecord{\smalltfield{x}{\textit{Ind}}\\
%                                                                                            \smalltfield{e}{named(x,
%                                                                                              ``Sam'')}}}}}}
%                                                                                  . $\lambda
%                                                                                  P$:\textit{Ppty}
%                                                                                  . $P$($c$.$\mathfrak{c}$.id$_0$) 
% \end{ex} 
% The content of \textit{Sam likes him} is given in \nexteg{a} and
% \textit{Sam likes himself} in \nexteg{b}.
% \begin{ex} 
% \begin{subex} 
 
% \item $\lambda
%   c$:\smallrecord{\smalltfield{$\mathfrak{s}$}{\smallrecord{\smalltfield{x$_0$}{\textit{Ind}}}}\\
%                   \smallmfield{local}{$\mathfrak{T}$($\emptyset$)}{\textit{Type}}\\
% \smallmfield{refl}{$\mathfrak{T}$($\emptyset$)}{\textit{Type}}\\
% \smalltfield{$\mathfrak{c}$}{\smallrecord{\smalltfield{id$_0$}{\smallrecord{\smalltfield{x}{\textit{Ind}}\\
%                                                                             \smalltfield{e}{named(x,
%                                                                               ``Sam'')}}}}}}
%                                                                   . 
% \record{\tfield{e}{like($c$.$\mathfrak{c}$.id$_0$.x, $\lambda
%     P$:\textit{Ppty} . $P$\{$c$.$\mathfrak{s}$.x$_0$\})}}
 
% \item $\lambda
%   c$:\smallrecord{
% \smallmfield{local}{$\mathfrak{T}$($\emptyset$)}{\textit{Type}}\\
% \smallmfield{refl}{$\mathfrak{T}$($\emptyset$)}{\textit{Type}}\\
% \smalltfield{$\mathfrak{c}$}{\smallrecord{\smalltfield{id$_0$}{\smallrecord{\smalltfield{x}{\textit{Ind}}\\
%                                                                             \smalltfield{e}{named(x,
%                                                                               ``Sam'')}}}}}}
%                                                                   . \\
% \hspace*{3em}                                                                   
% \record{\tfield{e}{like($c$.$\mathfrak{c}$.id$_0$.x, $\lambda
%     P$:\textit{Ppty} . 
% $P$\{$c$.$\mathfrak{c}$.id$_0$.x\})}} 
 
% \end{subex} 
   
% \end{ex} 
               
      


%[Grammar to be added.]
%\end{ex} 
  
  
% @@

% Finally, we show one way in which stored quantifiers can be
% ``percolated'' to higher constituents by showing in \nexteg{} how to
% construct an unplugged content from two constituent unplugged contents
% when the basic semantic composition involves function application.  To
% facilitate this we first define the `x'-incrementation of an object
% (such as a set or a function) with respect to a set of records $R$.

% \begin{ex} 
% The `x'\textit{-incrementation of an object $O$ with respect
%   to a set of records $R$}, $[O]_{\mathrm{incr}_x(R)}$, is the result of
% replacing each instance of x$_i$ in $O$, for any natural number $i$,
% with x$_{\mathrm{incr}_x(R)+i}$.  
% \end{ex} 
% Now application for unplugged contents (that is, objects of type
% \textit{UInterp}) can be characterized as in \nexteg{}.

% \begin{ex} 
% \textbf{Application}

% If $\alpha$ and $\beta$ are of type \textit{UInterp} and $\beta$.core
% is in the domain of $\alpha$.core, then the \textit{application of
%   $\alpha$ to $\beta$}, $\alpha @@\beta$, is
% \begin{quote}

% \record{\field{quants}{$\alpha$.quants$\cup$[$\beta$.quants]$_{\mathrm{incr}(\alpha\mathrm{.quants})}$}\\
%         \field{core}{$\alpha$.core@[$\beta$.core]$_{\mathrm{incr}(\alpha\mathrm{.quants})}$}}
% \end{quote}
% \end{ex} 
% \preveg{} says that the application of an unplugged content, $\alpha$,
% to another unplugged content, $\beta$, involves first changing the
% quantifier indices in $\beta$ so that they increment the quantifier
% indices in $\alpha$ in the way that is illustrated in the examples
% discussed above.  For example, if both $\alpha$ and $\beta$ contain
% the quantifier index `x$_0$' and this is the maximum in $\alpha$, then
% `x$_0$' will be changed to `x$_1$' in $\beta$.  Now the result of
% application is an unplugged content whose `quants' are the union of
% $\alpha$'s `quants' and the incremented `quants' of $\beta$.  The core
% is the result of applying $\alpha$'s core to $\beta$'s core using the
% same incrementation. 
  

\section{Summary of resources introduced}
\label{sec:sumresch8}



% This summary does not include the resources for chart processing
% introduced in Section~\ref{sec:chart}.

Items that are new since Chapter~\ref{ch:quant} are marked
``\textbf{New!}'' and items that have been revised since
Chapter~\ref{ch:quant} are marked ``\textbf{Revised!}''.  % We have
% included some items for completeness which were not explicitly
% introduced in the text.

\subsection{Universal grammar resources} 

\subsubsection{Types} 

\begin{description}

  \item[\textnormal{\textit{Loc}}] --- \record{\tfield{x-coord}{\textit{Real}}\\
        \tfield{y-coord}{\textit{Real}}\\
        \tfield{z-coord}{\textit{Real}}}

  
\item[\textnormal{\textit{Phon}}] --- a basic type

  $e$ : \textit{Phon} iff $e$ is a phonological event
  
\item[\textnormal{\textit{SEvent}}] --- \record{\tfield{e-loc}{\textit{Loc}} \\
        \tfield{sp}{\textit{Ind}} \\
        \tfield{au}{\textit{Ind}} \\
        \tfield{e}{\textit{Phon}} \\
        \tfield{c$_{\mathrm{loc}}$}{loc(e,e-loc)} \\
        \tfield{c$_{\mathrm{sp}}$}{speaker(e,sp)} \\
        \tfield{c$_{\mathrm{au}}$}{audience(e,au)}} (as in
      Chapter~\ref{ch:infex})

      \item[\textnormal{\textit{Assgnmnt}}] --- a basic type

      $r:\textit{Assgnmnt}$ iff
$r:\textit{Rec}$ and
$\mathrm{labels}(r)\subset\{\text{x}_0,\text{x}_1,\ldots\}$

\bigskip

If $T$ is \textit{Assgnmnt}$\wedge T'$, $T'$ is a record type and
  $\ell\in\mathrm{labels}(T')$, then
  \begin{enumerate} 
    
  \item if $\mathrm{labels}(T')=\{\ell\}$, $T\ominus\ell=\textit{Assgnmnt}$ 
    
  \item otherwise, $T\ominus\ell= \textit{Assgnmnt}\wedge (T'\ominus\ell/T)$
    
  \end{enumerate}

    
    
  %\end{enumerate}

  

\item[\textnormal{\textit{PropCntxt}}] --- a basic type

  $r:\textit{PropCntxt}$ iff
$r:\textit{Rec}$ and
$\mathrm{labels}(r)\cap\{\text{x}_0,\text{x}_1,\ldots\}=\emptyset$

\bigskip

If $T$ is \textit{PropCntxt}$\wedge T'$ and
  $\pi\in\mathrm{tpaths}(T')$ then
  \begin{enumerate}
    
  \item if $\pi$ is $\ell$ and $\mathrm{labels}(T')=\{\ell\}$, then $T\ominus\pi=\textit{PropCntxt}$
    
  \item otherwise, $T\ominus\pi= \textit{PropCntxt}\wedge(T'\ominus \pi/T')$
  \end{enumerate}

\item[\textnormal{\textit{Cntxt}} Revised!] --- \record{
    \tfield{$\mathfrak{q}$}{\textit{QStore}}\\
    \tfield{$\mathfrak{s}$}{\textit{Assgnmnt}}\\
    \tfield{$\mathfrak{l}$}{\textit{Assgnmnt}}\\
    \tfield{$\mathfrak{r}$}{\textit{Assgnmnt}}\\
    \tfield{$\mathfrak{w}$}{\textit{Assgnmnt}}\\
    \tfield{$\mathfrak{g}$}{\textit{Assgnmnt}}\\
    \tfield{$\mathfrak{c}$}{\textit{PropCntxt}}}

        \item[\textnormal{\textit{CntxtType}}] --- a basic type

    $T:\textit{CntxtType}$ iff $T\sqsubseteq\textit{Cntxt}$
      
    % \item[\textnormal{\textit{IndPpty}} New!] ---
    %   (\smallrecord{\smalltfield{x}{\textit{Ind}}}$\rightarrow$\textit{RecType})

      
    % \item[\textnormal{\textit{FramePpty}} New!] ---
    %   (\smallrecord{\smalltfield{x}{\textit{Rec}}}$\rightarrow$\textit{RecType})
      
    \item[\textnormal{\textit{xType}}] --- a basic type

      $T$ : \textit{xType} iff $T$ : \textit{RecType} and $\text{x}\in\mathrm{labels}(T)$

      \item[\textnormal{\textit{Ppty}}] ---
        \record{\tfield{bg}{\textit{xType}}\\
          \tfield{fg}(bg$\rightarrow$\textit{RecType})}

        \begin{description}
        
      \item[purification of properties, \textnormal{$\mathcal{P}(P)$}]\mbox{}

        If $P$ : \textit{Ppty}, then
\begin{quote}
if $P$.bg$^x$ = $P$.bg, then
\begin{quote}
$\mathfrak{P}(P)=P$
\end{quote}
otherwise:
\begin{quote}
$\mathfrak{P}(P)$ is $\ulcorner\lambda r$:$P$.bg$^{\text{x}}$
. \record{\tfield{$\mathfrak{c}$}{$P.\text{bg}\parallel$ \smallrecord{\field{x}{$r$.x}}}\\
          \tfield{e}{$P(\mathfrak{c})$}}$\urcorner$
\end{quote}
\end{quote}
\item[purification$^\forall$ of properties, \textnormal{$\mathcal{P}^\forall(P)$}]\mbox{}

  If $P$ : \textit{Ppty}, then
\begin{quote}
if $P$.bg$^x$ = $P$.bg, then
\begin{quote}
$\mathfrak{P^\forall}(P)=P$
\end{quote}
otherwise:
\begin{quote}
$\mathfrak{P^\forall}(P)$ is $\ulcorner\lambda r$:$P$.bg$^{\text{x}}$
. ($(r'\!:\!P.\text{bg}\!\parallel\!\!\text{\smallrecord{\field{x}{$r$.x}}})\rightarrow$
\record{
          \tfield{e}{$P(r')$}})$\urcorner$
\end{quote}
\end{quote}
\item[\textnormal{$P\{a\}$}] \mbox{}

  If $P$ is a pure property, $P\{a\}$
  represents the type $P$(\smallrecord{\field{x}{$a$}})

\item[\textnormal{$\mathfrak{T}(P)$}] \mbox{}

  If $P$ : \textit{Ppty} and $P$ is pure, then $\mathfrak{T}(P)$ : \textit{Type}.

  $a:\mathfrak{T}(P)$ iff $\mathfrak{P}(P)\{a\}$ is witnessed.
  
\item[\textnormal{exist$^{\text{w}}$($P$)}] \mbox{}

  If $P$ : \textit{Ppty}, then exist$^{\text{w}}$($P$) :
  \textit{Type}.

  $X:\text{exist}^{\text{w}}(P)$ iff
\begin{enumerate} 
 
\item $X:\mathrm{set}(\mathfrak{T}(P))$ 
 
\item $|X|=1$

  (equivalently, $p(\mathfrak{T}(X)\|\mathfrak{T}(P))=\frac{1}{|\down{\mathfrak{T}(P)}|}$)
 
\end{enumerate}

\item[\textnormal{exist$_{\text{pl}}^{\text{w}}$($P$)}] \mbox{}

  If $P$ : \textit{Ppty}, then exist$_{\text{pl}}^{\text{w}}$($P$) :
  \textit{Type}.

  $X:\text{exist}_{\text{pl}}^{\text{w}}(P)$ iff
\begin{enumerate} 
 
\item $X:\mathrm{set}(\mathfrak{T}(P))$ 
 
\item $|X|\geq 2$

  (equivalently, $p(\mathfrak{T}(X)\|\mathfrak{T}(P))\geq\frac{2}{|\down{\mathfrak{T}(P)}|}$)
 
\end{enumerate}

\item[\textnormal{no$^{\text{w}}$($P$)}] \mbox{}

  If $P$ : \textit{Ppty}, then no$^{\text{w}}$($P$) : \textit{Type}.

  $X:\text{no}^{\text{w}}(P)$ iff
\begin{enumerate} 
 
\item $X:\mathrm{set}(\mathfrak{T}(P))$ 
 
\item $|X|=0$


(equivalently, $p(\mathfrak{T}(X)\|\mathfrak{T}(P))=0$)
\end{enumerate}
  equivalently,
\begin{quote}
  $X:\text{no}^{\text{w}}(P)$ iff $X=\emptyset$
\end{quote}

\item[\textnormal{every$^{\text{w}}$($P$)}] \mbox{}

  If $P$ : \textit{Ppty}, then every$^{\text{w}}$($P$) :
  \textit{Type}.

  $X:\text{every}^{\text{w}}(P)$ iff
\begin{enumerate} 
 
\item $X:\mathrm{set}(\mathfrak{T}(P))$ 
 
\item $|X|=|\down{\mathfrak{T}(P)}|$

  (equivalently, $p(\mathfrak{T}(X)\|\mathfrak{T}(P))=1$)
 
\end{enumerate}
equivalently,
\begin{quote}
  $X:\text{every}^{\text{w}}(P)$ iff $X=\down{\mathfrak{T}(P)}$
\end{quote}


\item[\textnormal{most$^{\text{w}}$($P$)}] \mbox{}

  If $P$ : \textit{Ppty}, then most$^{\text{w}}$($P$) : \textit{Type}.

  $X:\text{most}^{\text{w}}(P)$ iff
\begin{enumerate} 
 
\item $X:\mathrm{set}(\mathfrak{T}(P))$ 
 
\item $\frac{|X|}{|\downP{P}|}\geq\theta_{\text{most}}(P)$, where
  $.5<\theta_{\text{most}}(P)<1$

  (equivalently, $p(\mathfrak{T}(X)\|\mathfrak{T}(P))\geq\theta_{\text{most}}(P)$)
 
\end{enumerate}

\item[\textnormal{many$_a^{\text{w}}$($P$)}] \mbox{}

  If $P$ : \textit{Ppty}, then many$_a^{\text{w}}$($P$) :
  \textit{Type}.

  $X:\text{many}_a^{\text{w}}(P)$ iff
\begin{enumerate} 
 
\item $X:\mathrm{set}(\mathfrak{T}(P))$ 
 
\item $|X|\geq\theta_{\text{many}_a}(P)$, where
  $\theta_{\text{many}_a}(P)$ is a natural number, $i$, such that
  $i>2$.

  (equivalently, $p(\mathfrak{T}(X)\|\mathfrak{T}(P))\geq\frac{\theta_{\text{many}_a}(P)}{[\down{\mathfrak{T}(P)}]}$)
 
\end{enumerate}

\item[\textnormal{many$_p^{\text{w}}$($P$)}] \mbox{}

  If $P$ : \textit{Ppty}, then many$_p^{\text{w}}$($P$) :
  \textit{Type}.

  $X:\text{many}_p^{\text{w}}(P)$ iff
\begin{enumerate} 
 
\item $X:\mathrm{set}(\mathfrak{T}(P))$ 
 
\item $\frac{|X|}{|\downP{P}|}\geq\theta_{\text{many}_p}(P)$, where
  $0<\theta_{\text{many}_p}(P)<1$

  (equivalently, $p(\mathfrak{T}(X)\|\mathfrak{T}(P))\geq\theta_{\text{many}_p}(P)$)
 
\end{enumerate}

\item[\textnormal{few$_a^{\text{w}}$($P$)}] \mbox{}

  If $P$ : \textit{Ppty}, then few$_a^{\text{w}}$($P$) :
  \textit{Type}.

  $X:\text{few}_a^{\text{w}}(P)$ iff
\begin{enumerate} 
 
\item $X:\mathrm{set}(\mathfrak{T}(P))$ 
 
\item $|X|\leq\theta_{\text{few}_a}(P)$, where
  $\theta_{\text{few}_a}(P)$ is a natural number, $i$, such that $i>2$

  (equivalently, $p(\mathfrak{T}(X)\|\mathfrak{T}(P))\leq\frac{\theta_{\text{few}_a}(P)}{[\down{\mathfrak{T}(P)}]}$)
 
\end{enumerate}

\item[\textnormal{few$_p^{\text{w}}$($P$)}] \mbox{}

  If $P$ : \textit{Ppty}, then few$_p^{\text{w}}$($P$) :
  \textit{Type}.

  $X:\text{few}_p^{\text{w}}(P)$ iff
\begin{enumerate} 
 
\item $X:\mathrm{set}(\mathfrak{T}(P))$ 
 
\item $\frac{|X|}{|\downP{P}|}\leq\theta_{\text{few}_p}(P)$, where
  $0<\theta_{\text{few}_p}(P)<1$

  (equivalently, $p(\mathfrak{T}(X)\|\mathfrak{T}(P))\leq\theta_{\text{few}_p}(P)$)
 
\end{enumerate}  

\item[\textnormal{a\_few$_a^{\text{w}}$($P$)}] \mbox{}

  If $P$ : \textit{Ppty}, then a\_few$_a^{\text{w}}$($P$) :
  \textit{Type}.

  $X:\text{a\_few}_a^{\text{w}}(P)$ iff
\begin{enumerate} 
 
\item $X:\mathrm{set}(\mathfrak{T}(P))$ 
 
\item $|X|\geq\theta_{\text{few}_a}(P)$, where
  $\theta_{\text{few}_a}(P)$ is a natural number, $i$, such that $i>2$

  (equivalently, $p(\mathfrak{T}(X)\|\mathfrak{T}(P))\geq\frac{\theta_{\text{few}_a}(P)}{[\down{\mathfrak{T}(P)}]}$)
 
\end{enumerate}

\item[\textnormal{a\_few$_p^{\text{w}}$($P$)}] \mbox{}

  If $P$ : \textit{Ppty}, then a\_few$_p^{\text{w}}$($P$) :
  \textit{Type}.

  $X:\text{a\_few}_p^{\text{w}}(P)$ iff
\begin{enumerate} 
 
\item $X:\mathrm{set}(\mathfrak{T}(P))$ 
 
\item $\frac{|X|}{|\downP{P}|}\geq\theta_{\text{few}_p}(P)$, where
  $0<\theta_{\text{few}_p}(P)<1$

  (equivalently, $p(\mathfrak{T}(X)\|\mathfrak{T}(P))\geq\theta_{\text{few}_p}(P)$)
 
\end{enumerate}  

  
\end{description}

\item[\textnormal{$^T\textit{Ppty}$}] --- if $T$ is a type, then
  $^T\textit{Ppty}$ is a type

  $P:{^T\textit{Ppty}}$ iff $P:\textit{Ppty}$ and
  $P.\text{bg}\sqsubseteq$ \smallrecord{
    \smalltfield{x}{$T$}}
        
      \item[\textnormal{\textit{PlPpty}}] --- a basic type

        $P$ : \textit{PlPpty} iff $P$ : \textit{Ppty} and for some type $T$,
$P$.bg $\sqsubseteq$ \smallrecord{\smalltfield{x}{$\mathrm{plurality}(T)$}} 
        
      \item[\textnormal{\textit{PPpty}}] --- \record{\tfield{bg}{\textit{CntxtType}} \\
          \tfield{fg}{(bg$\rightarrow$\textit{Ppty})}}

        \begin{description}

        
        \item[\textnormal{$\mathcal{L}(\mathcal{P})$} New!] \mbox{}

          If $\mathcal{P}$ is a parametric property of the form
  \begin{quote}
    $\ulcorner\lambda c\!:\!T_1\ . \ulcorner\lambda r\!:\!T_2\ .\
    \varphi\urcorner\urcorner$
  \end{quote}
  then the \textit{localization of $\mathcal{P}$},
  $\mathcal{L}(\mathcal{P})$, is
  \begin{quote}
    $\ulcorner\lambda c\!:\!\textit{Cntxt}\ .\ \ulcorner\lambda
    r\!:\!T_2\text{\d{$\wedge$}\smallrecord{
        \smalltfield{$\mathfrak{c}$}{$T_1$}}}\ .\ 
    \varphi_{c.\pi\leadsto r.\mathfrak{c}.\pi}\urcorner\urcorner$
  \end{quote}

  
\item[\textnormal{$\mathfrak{R}(\mathcal{P})$} New!] \mbox{}

  If $\mathcal{P}$ is a parametric property of the form
  \begin{quote}
    $\ulcorner\lambda c\!:\!T_1\ .\ \ulcorner\lambda
    r\!:\!T_2\dep{T_1}\ .\ \varphi\dep{T_1,T_2}\urcorner\urcorner$
  \end{quote}
  where for some natural number $i$, $T_1\sqsubseteq$ \smallrecord{
    \smalltfield{$\mathfrak{r}$}{\smallrecord{
        \smalltfield{x$_i$}{\textit{Ind}}}}}, then \textit{the reflexivization
  of} $\mathcal{P}$, $\mathfrak{R}(\mathcal{P})$, is
  \begin{quote}
    $\ulcorner\lambda c$:($T_1$\fbox{\d{$\wedge$}}\smallrecord{
      \smalltfield{$\mathfrak{r}$}{\textit{Assgnmnt}}})$\ominus\mathfrak{s}.\text{x}_i$
    . $\ulcorner\lambda r$:$T_2$
    . $[\varphi]_{c.\mathfrak{s}.\text{x}_i\leadsto
      r.\text{x}}\urcorner\urcorner$
  \end{quote}
  
\end{description}
        
      \item[\textnormal{$^T\textit{PPpty}$}] --- if $T$ is a
        type, then $^T\textit{PPpty}$ is a type

        $\mathcal{P}:{^T\textit{PPpty}}$ iff
  $\mathcal{P}:\textit{PPpty}$ and for any $c:\mathcal{P}.\text{bg}$, $\mathcal{P}(c):{^T\textit{Ppty}}$
        
      \item[\textnormal{\textit{Quant}}] ---
        (\textit{Ppty}$\rightarrow$\textit{RecType})
        
      \item[\textnormal{\textit{PQuant}}] --- \record{\tfield{bg}{\textit{CntxtType}} \\
          \tfield{fg}{(bg$\rightarrow$\textit{Quant})}}
        
      \item[\textnormal{\textit{QuantDet}}] ---
        (\textit{Ppty}$\rightarrow$\textit{Quant})
        
      \item[\textnormal{\textit{PQuantDet}}] ---
        \record{
          \tfield{bg}{\textit{CntxtType}}\\
          \tfield{fg}{(bg$\rightarrow$\textit{QuantDet})}}
        
      \item[\textnormal{\textit{PRecType}}] ---
        \record{
          \tfield{bg}{\textit{CntxtType}}\\
          \tfield{fg}{(bg$\rightarrow$\textit{RecType})}}
          

    \item[\textnormal{\textit{Cont}}] ---
      \textit{PRecType}$\vee$\textit{PPpty}$\vee$\textit{PQuant}$\vee$\textit{PQuantDet}

      
    \item[\textnormal{ContType} New!] ---  a basic type

      $T : \textit{ContType}$ iff $T\sqsubseteq\textit{Cont}$

      \begin{description}
        
      \item[\textnormal{$\mathfrak{A}(T)$} New!] \mbox{}

        If $T$ is a parametric content type, then $\mathfrak{A}(T)$ is also a
        parametric content type.

        $\varphi:\mathfrak{A}(T)$ iff $\varphi:T$ and $\varphi$.bg
        $\not\sqsubseteq$ \smallrecord{
          \smalltfield{$\mathfrak{r}$}{\smallrecord{
              \smalltfield{x$_i$}{\textit{Ind}}}}}, for any natural number
        $i$.

      \end{description}


      
    \item[\textnormal{\textit{QStore}} New!] --- a basic type

      $r:\textit{QStore}$ iff $r:Assgnmnt$
and for any $\text{x}_i\in\mathrm{labels}(r)$,
$r.\text{x}_i:\textit{PQuant}$

If $T$ is \textit{QStore}$\wedge T'$ and
  $\pi\in\mathrm{tpaths}(T')$ then
  \begin{enumerate}
    
  \item if $\pi$ is $\ell$ and $\mathrm{labels}(T')=\{\ell\}$, then $T\ominus\pi=\textit{PropCntxt}$
    
  \item otherwise, $T\ominus\pi= \textit{QStore}\wedge(T'\ominus \pi/T')$
  \end{enumerate}

  \begin{description}
\item[unplugged New!] --- definition

  A parametric content, $\alpha$, is \textit{unplugged} iff $c:\alpha$.bg implies $c.\mathfrak{q}\not=\emptyset$
(that is, $c.\mathfrak{q}$ is not the empty record).  Otherwise $\alpha$
is \textit{plugged}.

\item[\textnormal{$\mathrm{store}(\mathcal{Q})$} New!] \mbox{}

  If $\mathcal{Q}:\textit{PQuant}$ and $\mathcal{Q}$ is plugged, then
$\mathrm{store}(\mathcal{Q})$ is
\begin{quote}
  $\ulcorner\lambda c$:\smallrecord{
    \footnotesize{\textit{Cntxt}}\\
    \smalltfield{$\mathfrak{q}$}{\smallrecord{
        \smallmfield{x$_0$}{$\mathcal{Q}$}{\textit{PQuant}}}}\\
    \smalltfield{$\mathfrak{s}$}{\smallrecord{
        \smalltfield{x$_0$}{\textit{Ind}}}}} . $\lambda
  P$:\textit{Ppty} . $P\{c.\mathfrak{s}.\text{x}_0\}\urcorner$
\end{quote}

\item[\textnormal{$\mathrm{retrieve}(\text{x}_i,\alpha)$} New!]
  \mbox{}

    If $\alpha:\textit{PRecType}$, $\mathcal{Q}:\textit{PQuant}$ 
  $\alpha.\text{bg}\sqsubseteq$ \smallrecord{
    \tfield{$\mathfrak{q}$}{\smallrecord{
          \smallmfield{x$_i$}{$\mathcal{Q}$}{\textit{Ind}}}}}, $\mathcal{Q}'$ is
$[\mathcal{Q}]_{\mathfrak{c}\leadsto\mathfrak{c}.\text{f}}$ and\\
$\alpha'$ is
$[\mathrm{incr}(\alpha,\mathcal{Q}')]_{\mathfrak{c}\leadsto\mathfrak{c}.a}$, then
  $\mathrm{retrieve}(\text{x}_i,\alpha)$ is
  \begin{quote}
    $\lambda
    c$:$(\mathcal{Q}'.\text{bg}$\d{$\wedge$}$\alpha'.\text{bg}\ominus\mathfrak{q}.\text{x}_i,\mathfrak{s}.\text{x}_i)$
    . \\ \hspace*{2em}$\mathcal{Q'}(c)(\mathfrak{P}(\ulcorner\lambda
    r$:\smallrecord{
      \smalltfield{x}{\textit{Ind}}\\
      \smalltfield{$\mathfrak{s}$}{\smallrecord{
\footnotesize{\textit{Assgnmnt}}\\          \smallmfield{x$_i$}{$\Uparrow$x}{\textit{Ind}}}}}\d{$\wedge$}$\alpha'.\text{bg}^{\mathfrak{s}.\text{x}_i}$
    . $\alpha'(c[r][\mathfrak{q}.\text{x}_i=\mathcal{Q}])\urcorner))$
  \end{quote}



\end{description}

\item[\textnormal{$\mathfrak{S}(T)$} New!] --- if
  $T:\textit{ContType}$, then $\mathfrak{S}(T)$ is a type

  The witnesses of $\mathfrak{S}(T)$ are characterized by
  \begin{enumerate} 
 
  \item if $\varphi:T$ then $\varphi:\mathfrak{S}(T)$
    
  \item if $\varphi:\mathfrak{S}(T)$ and $\varphi:\textit{PPpty}$,
    then $\mathcal{L}(\varphi):\mathfrak{S}(T)$
    
  \item if $\varphi:\mathfrak{S}(T)$ and $\varphi:\textit{PPpty}$,
    then $\mathfrak{R}(\varphi):\mathfrak{S}(T)$

    
  \item if $\varphi:\mathfrak{S}(T)$,
    $\varphi\sqsubseteq\text{\smallrecord{
        \smallmfield{scope}{$\psi$}{\textit{Ppty}}}}$ and
    $\pi_1,\pi_2\in\mathrm{paths}(\psi.\text{bg})$, then $\varphi[\text{scope}=\psi_{\pi_1=\pi_2}:\textit{Ppty}]:\mathfrak{S}(T)$

  \item if $\alpha\mathcal{O}\beta:\mathfrak{S}(T)$, (for
      some combination operation, $\mathcal{O}$) and
      $\alpha\mathcal{O}_{i,j}\beta$ is defined (for some natural
      numbers, $i$ and $j$), then $\alpha\mathcal{O}_{i,j}\beta:\mathfrak{S}(T)$
 
  \item if $\varphi:\mathfrak{S}(T)$ and $\varphi$ is in the range of `$\mathrm{store}$', then
    $\mathrm{store}(\varphi):\mathfrak{S}(T)$

  \item if $\varphi:\mathfrak{S}(T)$ and `x$_i$' and $\varphi$ are appropriate
    arguments to `$\mathrm{retrieve}$', then
    $\mathrm{retrieve}(\text{x}_i,\varphi):\mathfrak{S}(T)$
    
  \item nothing is a witness for $\mathfrak{S}(T)$ except as required above.
 
  \end{enumerate}

  
  

      
    \item[\textnormal{\textit{Cat}}] --- a basic type

      s, np, det, n, v, vp : \textit{Cat}

    
    \item[\textnormal{\textit{Syn}}] ---  \record{\tfield{cat}{\textit{Cat}} \\
        \tfield{daughters}{\textit{Sign}$^*$}} 
 


  
    \item[\textnormal{\textit{Sign}} Revised!] ---  a basic type

      $\sigma$ : \textit{Sign} iff $\sigma$ :
      \record{\tfield{s-event}{\textit{SEvent}} \\
         \tfield{syn}{\textit{Syn}} \\
        \tfield{cont}{\textit{ContType}}} 

  
\item[\textnormal{\textit{SignType}}] --- a basic type

  $T:\textit{SignType}$ iff $T\sqsubseteq\textit{Sign}$ 

  
\item[\textnormal{\textit{S}}] --- 
  \smallrecord{
    \footnotesize{\textit{Sign}}\\
    \smalltfield{syn}{\smallrecord{\smallmfield{cat}{s}{\textit{Cat}}}}})
  
\item[\textnormal{\textit{S}/$i$}] --- if $i$ is a natural
  number, then \textit{S}/$i$ is a type

  $\alpha:\textit{S}/i$ iff $\alpha:\textit{S}$ and $\alpha.\text{cont}.\text{bg}\sqsubseteq$\smallrecord{
    \smalltfield{$\mathfrak{g}$}{\smallrecord{
        \smalltfield{x$_i$}{\textit{Ind}}}}} 
  
  
\item[\textnormal{\textit{NP}}] --- 
  \smallrecord{
    \footnotesize{\textit{Sign}}\\
    \smalltfield{syn}{\smallrecord{\smallmfield{cat}{np}{\textit{Cat}}}}}
  
\item[\textnormal{\textit{whNP}}] --- a basic type

  $\sigma$ : \textit{WhNP} iff $\sigma$ : \textit{NP}, $\sigma$.cont is $\mathcal{Q}$ and
$\mathcal{Q}.\text{bg}\sqsubseteq$\smallrecord{
  \smalltfield{$\mathfrak{w}$}{\smallrecord{
      \smalltfield{x$_i$}{\textit{Ind}}}}}, for some natural number
$i$.

\item[\textnormal{\textit{NP}$_{\text{wh}_i}$}] --- if $i$ is a
  natural number, then \textit{NP}$_{\text{wh}_i}$ is a type

  $\alpha:\textit{NP}_{\text{wh}_i}$ iff $\alpha:\textit{NP}$ and
  $\alpha.\text{cont}.\text{bg}\sqsubseteq$\smallrecord{
    \smalltfield{$\mathfrak{w}$}{\smallrecord{
        \smalltfield{x$_i$}{\textit{Ind}}}}}

  
\item[\textnormal{\textit{Det}}] --- 
  \smallrecord{
    \footnotesize{\textit{Sign}}\\
\smalltfield{syn}{\smallrecord{\smallmfield{cat}{det}{\textit{Cat}}}}}
  
\item[\textnormal{\textit{N}}] --- 
  \smallrecord{
    \footnotesize{\textit{Sign}}\\
    \smalltfield{syn}{\smallrecord{\smallmfield{cat}{n}{\textit{Cat}}}}}
  
\item[\textnormal{$^T\textit{N}$}] --- if $T$ is a type, then
  $^T\textit{N}$ is a type

  $\alpha:{^T\textit{N}}$ iff
  $\alpha:\textit{N}$ and $\alpha.\text{cont}:{^T\textit{PPpty}}$
  
\item[\textnormal{\textit{V}}] --- 
  \smallrecord{
    \footnotesize{\textit{Sign}}\\
    \smalltfield{syn}{\smallrecord{\smallmfield{cat}{v}{\textit{Cat}}}}}
  
  
\item[\textnormal{\textit{VP}}] --- 
  \smallrecord{
    \footnotesize{\textit{Sign}}\\
    \smalltfield{syn}{\smallrecord{\smallmfield{cat}{vp}{\textit{Cat}}}}}

\item[\textnormal{\textit{Rel}}] --- 
  \smallrecord{
    \footnotesize{\textit{Sign}}\\
    \smalltfield{syn}{\smallrecord{\smallmfield{cat}{rel}{\textit{Cat}}}}}

  
\item[\textnormal{$^T\textit{Rel}$}] --- if $T$ is a type, then
  $^T\textit{Rel}$ is a type

  $\alpha:{^T\textit{Rel}}$ iff
  $\alpha:\textit{Rel}$ and $\alpha.\text{cont}:{^T\textit{PPpty}}$
  
\item[\textnormal{\textit{NoDaughters}}] ---
  \smallrecord{\smalltfield{syn}{\smallrecord{\smallmfield{daughters}{$\varepsilon$}{\textit{Sign}$^*$}}}}

  
\item[\textnormal{\textit{Real}}] --- a basic type

  $n$ : \textit{Real} iff $n$ is a real number

  
\item[\textnormal{\textit{Card}}] --- a basic type

  $n$ : \textit{Card} iff $n$ is a cardinal number (natural numbers
  with the addition of $\aleph_0, \aleph_1,\ldots$)
 

  

\item[\textnormal{\textit{AmbTempFrame}}] --- \record{\tfield{x}{\textit{Real}} \\
        \tfield{loc}{\textit{Loc}} \\
        \tfield{e}{temp(loc, x)}}

      
    \item[\textnormal{\textit{TempRiseEventCntxt}}] ---
      \record{
        \tfield{fix}{\record{
            \tfield{loc}{\textit{Loc}}}}\\
        \tfield{scale}{(\textit{AmbTempFrame}
          $\rightarrow$ \textit{Real})}}
      
    \item[\textnormal{\textit{TempRiseEvent}}] ---
      
      $\lambda r$:\textit{TempRiseEventCntxt} .\\  
\hspace*{2em}\record{\tfield{e}{(\textit{AmbTempFrame}$\parallel$$r$.fix)$^2$}\\
        \tfield{c$_{\mathrm{rise}}$}{$r$.scale(e[0]) $<$
          $r$.scale(e[1])}}
      
    \item[\textnormal{\textit{PriceFrame}}] --- \record{\tfield{x}{\textit{Real}} \\
        \tfield{loc}{\textit{Loc}} \\
        \tfield{commodity}{\textit{Ind}} \\
        \tfield{e}{price(commodity, loc, x)}}
      
    \item[\textnormal{\textit{PriceRiseEventCntxt}}] --- \record{
        \tfield{fix}{\record{
            \tfield{loc}{\textit{Loc}}\\
            \tfield{commodity}{\textit{Ind}}}}\\
        \tfield{scale}{(\textit{PriceFrame}
                             $\rightarrow$ \textit{Real})}}
 
  \item[\textnormal{\textit{PriceRiseEvent}}] --- \mbox{}

   $\lambda
r$:\textit{TempRiseEventCntxt} .\\  
\hspace*{2em}\record{\tfield{e}{(\textit{PriceFrame}$\parallel$$r$.fix)$^2$}\\
        \tfield{c$_{\mathrm{rise}}$}{$r$.scale(e[0]) $<$
          $r$.scale(e[1])}}
      
    \item[\textnormal{\textit{LocFrame}}] --- \record{\tfield{x}{\textit{Ind}} \\
        \tfield{loc}{\textit{Loc}} \\
        \tfield{e}{at(x, loc)}}
      
    \item[\textnormal{\textit{LocRiseEventCntxt}}] ---
      \record{\tfield{fix}{\record{\tfield{x}{\textit{Ind}}
                                               }}\\
                         \tfield{scale}{(\textit{LocFrame}
                           $\rightarrow$ \textit{Real})}}
                       
 \item[\textnormal{\textit{LocRiseEvent}}] ---

   $\lambda
r$:\textit{LocRiseEventCntxt} .\\  
\hspace*{2em}\record{\tfield{e}{(\textit{LocFrame}$\parallel$$r$.fix)$^2$}\\
        \tfield{c$_{\mathrm{rise}}$}{$r$.scale(e[0]) $<$
          $r$.scale(e[1])}}

      
    \item[\textnormal{\textit{Topos}}] --- a basic type

      If $\tau:\textit{Topos}$, then $\tau$ : \record{\tfield{bg}{\textit{Type}}\\
        \tfield{fg}{(bg$\rightarrow$\textit{Type})}} 

                       


\end{description}

    \subsubsection{Predicates} (as in Chapter~\ref{ch:quant})

% \begin{description}

%  \item[with arity \textnormal{$\langle$\textit{Phon},
%     \textit{Loc}$\rangle$}] \mbox{}
  
%     \begin{description}

%     \item[\textnormal{loc}] --- $e$ : loc($u$, $l$) iff $u$ is located
%       at $l$ in $e$

%     \end{description}
    
%   \item[with arity \textnormal{$\langle$\textit{Phon},
%       \textit{Ind}$\rangle$}] \mbox{}

%     \begin{description}

%     \item[\textnormal{speaker}] --- $e$ : speaker($u$, $a$) iff $u$
%         is the speaker of $u$ in $e$

%     \item[\textnormal{audience}] --- $e$ : audience($u$, $a$) iff
%         $u$ is the audience of $u$ in $e$

%       \end{description}

%     \item[with arity \textnormal{$\langle\textit{Card}\rangle$}] \mbox{}

%   \begin{description}
    
%   \item[\textnormal{card}] --- $X$ : card($n$) iff for some $T$,
%     $X:\mathrm{set}(T)$ and $|X|=n$
    
%   \item[\textnormal{card\_at\_least}] --- $X$ : card\_at\_least($n$) iff for some $T$, $X:\mathrm{set}(T)$
%     and $|X|\geq n$

    
%   \item[\textnormal{card\_at\_most}] --- $X$ : card\_at\_most($n$) iff for some $T$, $X:\mathrm{set}(T)$
%     and $|X|\leq n$

%   \end{description}
      
%   \item[with arity \textnormal{$\langle$\textit{Ppty}$\rangle$}]
%     \mbox{}

%     \begin{description}
      
%     \item[\textnormal{unique}] --- $s:\textrm{unique}(P)$ iff
%       $\mid\!\downP{P\!\restriction\!s}\!\mid = 1$

%     \end{description}

%     \item[with arity \textnormal{$\langle\textit{Ppty},\textit{Ppty}\rangle$}] \mbox{}

%   \begin{description}
    
%   \item[\textnormal{exist} Revised!] \mbox{}
%     \begin{description}
      
%     \item[general witness condition] \mbox{}

%       $s:\text{exist}(P,Q)$ iff $s$ :
%       \record{\tfield{X}{exist$^w(P)$}\\
%         \tfield{f}{$((a:\mathfrak{T}(\text{X}))\rightarrow\mathfrak{P}(Q)\{a\})$}}

      
%     \item[particular witness condition] \mbox{}

%       $s:\text{exist}(P,Q)$ iff $s$ :
%   \record{\tfield{x}{$\mathfrak{T}(P)$}\\
%         \tfield{e}{$\mathfrak{P}(Q)$\{x\}}}

%     \end{description}

%     % $s$ : exist($P$,$Q$) iff
%      % $\downP{P}\cap\downP{Q\!\restriction\! s}\not=\emptyset$
%     % $s$ : exist($P$,$Q$) iff
%     % $\downP{P\!\restriction\!s}\cap\downP{Q|_{\mathcal{F}(P.\mathrm{fg})}\!\restriction\!s}\not=\emptyset$

    
%   \item[\textnormal{exist$_\text{pl}$} New!] \mbox{}

%     \begin{description}
      
%     \item[general witness condition] \mbox{}

%       $s:\text{exist}_{\text{pl}}(P,Q)$ iff $s$ :
%   \record{\tfield{X}{exist$_{\text{pl}}^w(P)$}\\
%     \tfield{f}{$((a:\mathfrak{T}(\text{X}))\rightarrow\mathfrak{P}(Q)\{a\})$}}

% \end{description}

% \item[\textnormal{no} New!] \mbox{}

%   \begin{description}
    
%   \item[general witness condition] \mbox{}

%     $s:\text{no}(P,Q)$ iff $s$ :
%     \record{\tfield{X}{$\text{no}^w(P)$}\\
%       \tfield{f}{($(a:(\mathfrak{T}(P)\wedge\mathfrak{T}(Q)))\rightarrow$
%         \record{\mfield{x}{$a$}{$\mathfrak{T}$(X)}})}}
    
%   \item[particular witness condition] \mbox{}

%     $s:\text{no}(P,Q)$ iff $s$ :
%     \record{\tfield{X}{$\text{every}^w(P)$}\\
%       \tfield{f}{$((x:\mathfrak{T}(X))\rightarrow\neg\mathfrak{P}(Q)\{x\})$}}

%   \end{description}

        
    
% \item[\textnormal{every} Revised!] \mbox{}
%   % $s$ : every($P$,$Q$) iff
%   % $\downP{P}\subseteq\downP{Q\!\restriction\!s}$

%   \begin{description}

    
%   \item[general witness condition] \mbox{}

%     $s:\text{every}(P,Q)$ iff $s$ :
%   \record{\tfield{X}{$\text{every}^w(P)$}\\
%           \tfield{f}{$((a:\mathfrak{T}(\text{X}))\rightarrow\mathfrak{P}(Q)\{a\})$}}

%       \end{description}

      
%     \item[\textnormal{most} New!] \mbox{}

%       \begin{description}
        
%       \item[general witness condition] \mbox{}

%         $s:\text{most}(P,Q)$ iff $s$ :
%   \record{\tfield{X}{$\text{most}^w(P)$}\\
%     \tfield{f}{$((a:\mathfrak{T}(\text{X}))\rightarrow\mathfrak{P}(Q)\{a\})$}}

% \end{description}

% \item[\textnormal{many$_a$} New!] \mbox{}

%   \begin{description}
    
%   \item[general witness condition] \mbox{}

%     $s:\text{many}_a(P,Q)$ iff $s$ :
%   \record{\tfield{X}{$\text{many}_a^w(P)$}\\
%     \tfield{f}{$((a:\mathfrak{T}(\text{X}))\rightarrow\mathfrak{P}(Q)\{a\})$}}

% \end{description}

% \item[\textnormal{many$_p$} New!] \mbox{}

%   \begin{description}

    
%   \item[general witness condition] \mbox{}

%     $s:\text{many}_p(P,Q)$ iff $s$ :
%   \record{\tfield{X}{$\text{many}_p^w(P)$}\\
%     \tfield{f}{$((a:\mathfrak{T}(\text{X}))\rightarrow\mathfrak{P}(Q)\{a\})$}}

% \end{description}

% \item[\textnormal{few$_a$} New!] \mbox{}

%   \begin{description}

    
%   \item[general witness condition] \mbox{}

%     $s:\text{few}_a(P,Q)$ iff $s$ :
%   \record{\tfield{X}{$\text{few}_a^w(P)$}\\
%           \tfield{f}{($(a:(\mathfrak{T}(P)\wedge\mathfrak{T}(Q)))\rightarrow$
%             \record{\mfield{x}{$a$}{$\mathfrak{T}$(X)}})}}
        
%       \item[particular witness condition] \mbox{}

%         $s:\text{few}_a(P,Q)$ iff $s$ :
%   \record{\tfield{X}{$\overline{\text{few}_a^w(P)}$}\\
%     \tfield{f}{$((x:\mathfrak{T}(X))\rightarrow\neg\mathfrak{P}(Q)\{x\})$}}
  
%       \end{description}

      
%     \item[\textnormal{few$_p$} New!] \mbox{}

%       \begin{description}
        
%       \item[general witness condition] \mbox{}

%         $s:\text{few}_p(P,Q)$ iff $s$ :
%   \record{\tfield{X}{$\text{few}_p^w(P)$}\\
%           \tfield{f}{($(a:(\mathfrak{T}(P)\wedge\mathfrak{T}(Q)))\rightarrow$
%             \record{\mfield{x}{$a$}{$\mathfrak{T}$(X)}})}}

        
%       \item[particular witness condition] \mbox{}

%         $s:\text{few}_p(P,Q)$ iff $s$ :
%   \record{\tfield{X}{$\overline{\text{few}_p^w(P)}$}\\
%           \tfield{f}{$((x:\mathfrak{T}(X))\rightarrow\neg\mathfrak{P}(Q)\{x\})$}}

%       \end{description}

      
%     \item[\textnormal{a\_few$_a$} New!] \mbox{}

%       \begin{description}


       
%       \item[general witness condition] \mbox{}

%         $s:\text{a\_few}_a(P,Q)$ iff $s$ :
%   \record{\tfield{X}{$\text{a\_few}_a^w(P)$}\\
%     \tfield{f}{$((a:\mathfrak{T}(\text{X}))\rightarrow\mathfrak{P}(Q)\{a\})$}}

% \end{description}

% \item[\textnormal{a\_few$_p$} New!] \mbox{}

%   \begin{description}
    
%   \item[general witness condition] \mbox{}

%     $s:\text{a\_few}_p(P,Q)$ iff $s$ :
%   \record{\tfield{X}{$\text{a\_few}_p^w(P)$}\\
%     \tfield{f}{$((a:\mathfrak{T}(\text{X}))\rightarrow\mathfrak{P}(Q)\{a\})$}}

% \end{description}


      
%   \end{description}

  
% \item[with arity
%   \textnormal{$\langle\textit{PlPpty},\textit{PlPpty}\rangle$}] \mbox{}
  
% \begin{description}
    
% \item[\textnormal{exactly\_$n$}] --- for $n$ a natural number,

%   $s$ : exactly\_$n$($P$, $Q$) iff $s$ : at\_least\_$n$($P$, $Q$)$\wedge$at\_most\_$n$($P$, $Q$)

%     % $s$ : exactly\_$n$($P$, $Q$) iff
%     %       $\down{\mathcal{F}((Q\!\restriction\!        s).\text{fg}\mid_{\mathcal{F}((P\restriction s).\text{fg})})
%     %     \text{ \d{$\wedge$} \smallrecord{\smalltfield{x}{card($n$)}}}}
%     %   \not=\emptyset$

      
%     \item[\textnormal{at\_least\_$n$}] --- for $n$ a natural
%       number,

%       $s$ : at\_least\_$n$($P$, $Q$) iff
%           $\down{\mathcal{F}((Q\!\restriction\!        s).\text{fg}\mid_{\mathcal{F}(P.\text{fg})})
%         \text{ \d{$\wedge$} \smallrecord{\smalltfield{x}{card\_at\_least($n$)}}}}
%       \not=\emptyset$

%       % $s$ : at\_least\_$n$($P$, $Q$) iff
%       %     $\down{\mathcal{F}((Q\!\restriction\!        s).\text{fg}\mid_{\mathcal{F}((P\restriction s).\text{fg})})
%       %   \text{ \d{$\wedge$} \smallrecord{\smalltfield{x}{card\_at\_least($n$)}}}}
%       % \not=\emptyset$

      
%     \item[\textnormal{at\_most\_$n$}] --- for $n$ a natural
%       number,

%       $s$ : at\_most\_$n$($P$, $Q$) iff
%           $r:\mathcal{F}((Q\!\restriction\!        s).\text{fg}\mid_{\mathcal{F}(P.\text{fg})})
%         \text{ implies } r:\text{ \smallrecord{\smalltfield{x}{card\_at\_most($n$)}}}$

%       % $s$ : at\_most\_$n$($P$, $Q$) iff
%       %     $r:\mathcal{F}((Q\!\restriction\!        s).\text{fg}\mid_{\mathcal{F}((P\restriction s).\text{fg})})
%       %   \text{ implies } r:\text{ \smallrecord{\smalltfield{x}{card\_at\_most($n$)}}}$

% \end{description}    

  
% \item[with arity \textnormal{\{$\langle T\rangle\mid T$ is a type\}}]
%   \mbox{}

%   \begin{description}
    
%   \item[\textnormal{be}] --- $e:\text{be}(a)$ iff $a\varepsilon e$

%   \end{description}

  
% \item[with arity
%   \textnormal{$\langle\textit{Loc},\textit{Real}\rangle$}] \mbox{}
%   \begin{description}
    
%   \item[\textnormal{temp}] --- $e:\text{temp}(l,n)$ iff $n$ is
%     the temperature at $l$ in $e$.

    
  

%   \end{description}
  
% \item[with arity
%   \textnormal{$\langle\textit{Real},\textit{Real}\rangle$}] \mbox{}

%   \begin{description}
    
%   \item[\textnormal{less-than}] --- $e$ : less-than($n$, $m$) iff $n\varepsilon e$, $m\varepsilon e$ and $n<m$

%   \end{description}

% \item[with arity
%   \textnormal{$\langle\textit{Type},\textit{Type},\textit{Topos}\rangle$}]
%   \mbox{}

%   \begin{description}

    
%   \item[\textnormal{nec}] ---

%     If $\mathbb{T}$ is a modal type system and $p\in\mathbb{T}$, then
%     \begin{quote}
%       $s:_p\mathrm{nec}(T,B,\tau)$ iff $s:_pB$, 
%       $B\sqsubseteq_{\mathbb{T}}\tau.\mathrm{bg}$ and 
%       $\tau(s)\sqsubseteq_{\mathbb{T}}T$
%     \end{quote}

    
%   \item[\textnormal{poss}] ---

%     If $\mathbb{T}$ is a modal type system and $p\in\mathbb{T}$, then
%     \begin{quote}
%       $s:_p\mathrm{poss}(T,B,\tau)$ iff $s:_pB$, 
%       $B\sqsubseteq_{\mathbb{T}}\tau.\mathrm{bg}$ and 
%       $\tau(s)\top_{\mathbb{T}}T$
%     \end{quote}

%   \end{description}

  
% \item[with arity
%   \textnormal{$\langle\textit{RecType},\textit{RecType}\rangle$}] \mbox{}

%   \begin{description}
    
%   \item[\textnormal{pov}
%     ] --- $e:\text{pov}(T_1,T_2)$ iff $T_2$
%     is a point of view on $T_1$ in $e$.

%     $e:\text{pov}(T_1,T_2)$ implies
%     $\mathrm{labels}(T_2)\subseteq\mathrm{labels(T_1)}$

%   \end{description}

  
% \item[with arity
%   \textnormal{$\langle\textit{Ind},\textit{RecType}\rangle$}] \mbox{}

%   \begin{description}
    
%   \item[\textnormal{ltm}] --- $e:\text{ltm}(a,T)$ iff $T$ is $a$'s
%     long term memory in $e$.
    
%   \item[\textnormal{rbelieve}] --- $e:\text{rbelieve}(a,T)$ iff
%     $T$ is $a$'s religious beliefs in $e$.

    
%   \item[\textnormal{des}] --- $e:\text{des}(a,T)$ iff $T$ is
%     $a$'s desires in $e$.

%   \end{description}
  
  

  
  
  


% \end{description}

\subsubsection{Properties} (as in Chapter~\ref{ch:quant})

% \begin{description}
  
% \item[\textnormal{$P_1\&P_2$} New!] \mbox{}

%   If $T$ is a type, $P_1:{^T\textit{Ppty}}$ and $P_2:{^T\textit{Ppty}}$,
% then \textit{the conjunction of $P_1$ and
%   $P_2$}, $P_1\&P_2$, is
% \begin{quote}
%   $\ulcorner\lambda r$:\smallrecord{
%     \smalltfield{x}{$T$}} .
%   \record{
%     \tfield{e$_1$}{$P_1\{r.\text{x}\}$}\\
%     \tfield{e$_2$}{$P_2\{r.\text{x}\}$}}$\urcorner$
% \end{quote}

% \end{description}

\subsubsection{Scales} (as in Chapter~\ref{ch:commonnouns})

% % % \begin{description}
  
% % % \item[\textnormal{$\zeta_{\text{temp}}$} New!] ---
% % %   $\lambda r$:\textit{AmbTempFrame} . $r$.x : (\textit{AmbTempFrame}
% % %   $\rightarrow$ \textit{Real})

  
% % % \item[\textnormal{$\zeta_{\mathrm{height}}$} New!] ---
% % %   $\lambda r$:\textit{LocFrame} . $r$.loc.z-coord : (\textit{LocFrame}
% % %   $\rightarrow$ \textit{Real})

  
% % % \item[\textnormal{$\zeta_{\text{age}}$} New!] ---  
% % % $\lambda r$:\smallrecord{\smalltfield{x}{\textit{Ind}}\\
% % %                          \smalltfield{age}{\textit{Real}}\\
% % %                          \smalltfield{c$_{\mathrm{age}}$}{age\_of(x,age)}}
% % %                        . $r$.age : (\smallrecord{\smalltfield{x}{\textit{Ind}}\\
% % %                          \smalltfield{age}{\textit{Real}}\\
% % %                          \smalltfield{c$_{\mathrm{age}}$}{age\_of(x,age)}}
% % %                        $\rightarrow$ \textit{Real})

% % % \end{description}



\subsubsection{Lexicon} 
\begin{description}
\item[\textnormal{Lex}] \mbox{}

  If $T_{\mathrm{phon}}$ is a phonological type (that is,
$T_{\mathrm{phon}}\sqsubseteq\textit{Phon}$) and $T_{\mathrm{sign}}$
is a sign type (that is, $T_{\mathrm{sign}}\sqsubseteq\textit{Sign}$), then we shall use
Lex($T_{\mathrm{phon}}$, $T_{\mathrm{sign}}$) to represent
\begin{quote}
(($T_{\mathrm{sign}}$ \d{$\wedge$}
\smallrecord{\smalltfield{s-event}{\smallrecord{\smalltfield{e}{$T_{\mathrm{phon}}$}}}})
\d{$\wedge$} \textit{NoDaughters})
\end{quote}

% \item[\textnormal{SemCommonNoun($p$)}] \mbox{}

%   If $p$ is a predicate with arity $\langle\textit{Ind}\rangle$, then SemCommonNoun($p$) is
%   \begin{quote}
%     $\lambda c$:\textit{Rec} . $\lambda
% r$:\smallrecord{\smalltfield{x}{\textit{Ind}}}
% . \record{\tfield{e}{$p$($r$.x)}}
% \end{quote}

% \item[\textnormal{Lex$_{\mathrm{CommonNoun}}$($T_{\mathrm{phon}}$,
%     $p$)}] \mbox{}

%   If $T_{\mathrm{phon}}$ is a phonological type and $p$ is a
%   predicate with arity $\langle\textit{Ind}\rangle$, then Lex$_{\mathrm{CommonNoun}}$($T_{\mathrm{phon}}$,
%   $p$) is
%   \begin{quote}
%     Lex($T_{\mathrm{phon}}$, \textit{N}) \d{$\wedge$}
%     \smallrecord{\smallmfield{cont}{SemCommonNoun($p$)}{\textit{PPpty}}}
%   \end{quote}

\item[\textnormal{SemCommonNoun($T_{\mathrm{bg}}$, $p$)}]
  \mbox{}

    If $p$ is a predicate with arity $\langle\textit{Ind}\rangle$ and
    $T_{\mathrm{bg}}$ is a type (of context), then
    SemCommonNoun($T_{\mathrm{bg}}$, $p$) is
  \begin{quote}
    $\ulcorner\lambda c$:$T_{\mathrm{bg}}$ . $\ulcorner\lambda
r$:\smallrecord{\smalltfield{x}{\textit{Ind}}}
. \record{\tfield{e}{$p$($r$.x)}}$\urcorner\urcorner$
\end{quote}

If $p$ is a predicate with arity $\langle\textit{Rec}\rangle$ and $T_{\mathrm{bg}}$ is a type (of context), then
    SemCommonNoun($T_{\mathrm{bg}}$, $p$) is
  \begin{quote}
    $\ulcorner\lambda c$:$T_{\mathrm{bg}}$ . $\ulcorner\lambda
r$:\smallrecord{\smalltfield{x}{\textit{Rec}}}
. \record{\tfield{e}{$p$($r$.x)}}$\urcorner\urcorner$
\end{quote}
  
%     \todo{Not sure this is used}If $p$ is a predicate with arity $\langle\textit{Rec},
%     \textit{Rec}\rangle$ and $T_{\mathrm{bg}}\sqsubseteq$\smallrecord{\smalltfield{$\mathfrak{c}$}{\textit{Rec}}} is a type (of context), then
%     SemCommonNoun($T_{\mathrm{bg}}$, $p$) is
%   \begin{quote}
%     $\lambda c$:$T_{\mathrm{bg}}$ . $\lambda
% r$:\textit{Rec}
% . \record{\tfield{e}{$p$($r$, $c.\mathfrak{c}$)}}
% \end{quote}

\item[\textnormal{Lex$_{\mathrm{CommonNoun}}$($T_{\mathrm{phon}}$,
    $T_{\mathrm{bg}}$, $p$)} Revised!] \mbox{}

    % If $T_{\mathrm{phon}}$ is a phonological type, $p$ is a
  % predicate with arity $\langle\textit{Ind}\rangle$ and
  % $T_{\mathrm{bg}}$ is a type (of context), then
  % Lex$_{\mathrm{CommonNoun}}$($T_{\mathrm{phon}}$, $T_{\mathrm{bg}}$,
  % $p$) is
  % \begin{quote}
  %   Lex($T_{\mathrm{phon}}$, \textit{N}) \d{$\wedge$}
  %   \smallrecord{\smallmfield{cont}{SemCommonNoun($T_{\mathrm{bg}}$,
  %       $p$)}{\textit{PPpty}}}
  % \end{quote}


  If $T_{\mathrm{phon}}$ is a phonological type, $p$ is a
  predicate with arity $\langle\textit{Ind}\rangle$ or $\langle\textit{Rec}\rangle$ and
  $T_{\mathrm{bg}}$ is a type (of context), then
  Lex$_{\mathrm{CommonNoun}}$($T_{\mathrm{phon}}$, $T_{\mathrm{bg}}$,
  $p$) is
  \begin{quote}
    Lex($T_{\mathrm{phon}}$, \textit{N}) \d{$\wedge$}
    \smallrecord{\smallmfield{cont}{SemCommonNoun($T_{\mathrm{bg}}$,
        $p$)$^{\mathfrak{S}}$}{\textit{ContType}}}
  \end{quote}
  
\item[\textnormal{SemPropName($T_{\text{phon}}$)}] \mbox{}

  If $T_{\text{phon}}$ is a phonological type, then SemPropName($T_{\text{phon}}$) is
  \begin{quote}
    $\ulcorner\lambda c$:\smallrecord{
      \footnotesize{\textit{Cntxt}}\\
      \smalltfield{$\mathfrak{c}$}{\smallrecord{\smalltfield{x}{\textit{Ind}}\\
                         \smalltfield{e}{named(x, $T_{\mathrm{phon}}$)}}}} . $\lambda
                       P$:\textit{Ppty} . $P(c.\mathfrak{c})\urcorner$
  \end{quote}
  
\item[\textnormal{Lex$_{\mathrm{PropName}}$($T_{\mathrm{phon}}$
    )} Revised!] \mbox{}

  If $T_{\mathrm{phon}}$ is a phonological type,

  then Lex$_{\mathrm{PropName}}$($T_{\mathrm{phon}}$) is
  \begin{quote}
    Lex($T_{\mathrm{phon}}$, \textit{NP}) \d{$\wedge$}
\smallrecord{\smallmfield{cnt}{SemPropName($T_{\mathrm{phon}}$)$^{\mathfrak{S}}$}{\textit{ContType}}}
\end{quote}

\item[\textnormal{SemPron} Revised!] \mbox{}

  $\ulcorner\lambda c$:\smallrecord{
    \footnotesize{\textit{Cntxt}}\\
    \smalltfield{$\mathfrak{s}$}{\smallrecord{
        \smalltfield{x$_0$}{\textit{Ind}}}}\\
    \smalltfield{$\mathfrak{l}$}{\smallrecord{
        \smalltfield{x$_0$}{\textit{Ind}}}}} . $\lambda
  P$:\textit{Ppty}
  . $P$(\smallrecord{\field{x}{$c.\mathfrak{s}$.x$_0$}})$\urcorner$

\item[\textnormal{LexPron($T_{\text{phon}}$)} Revised!] \mbox{}

If $T_{\text{phon}}$ is a phonological type, then
LexPron($T_{\text{phon}}$) is
\begin{quote}
Lex($T_{\mathrm{phon}}$, \textit{NP}) \d{$\wedge$}
\smallrecord{\smallmfield{cont}{SemPron$^{\mathfrak{S}}$}{\textit{ContType}}}
\end{quote}

\item[\textnormal{SemWhPron}] \mbox{}

  $\ulcorner\lambda c$:\smallrecord{
  \footnotesize{\textit{Cntxt}}\\
  \smalltfield{$\mathfrak{w}$}{\smallrecord{
      \smalltfield{x$_0$}{\textit{Ind}}}}}
. $\lambda P$:\textit{Ppty}
. $P$(\smallrecord{\field{x}{$c.\mathfrak{w}.\text{x}_0$}})$\urcorner$

\item[\textnormal{LexWhPron($T_{\text{phon}}$)} Revised!] \mbox{}

  If $T_{\text{phon}}$ is a phonological type, then
LexWhPron($T_{\text{phon}}$) is
\begin{quote}
Lex($T_{\mathrm{phon}}$, \textit{NP}) \d{$\wedge$}
\smallrecord{\smallmfield{cont}{SemWhPron$^{\mathfrak{S}}$}{\textit{ContType}}}
\end{quote}

\item[\textnormal{SemReflPron} New!] \mbox{}

  $\ulcorner\lambda c$:\smallrecord{
    \footnotesize{\textit{Cntxt}}\\
    \smalltfield{$\mathfrak{s}$}{\smallrecord{
        \smalltfield{x$_0$}{\textit{Ind}}}}\\
    \smalltfield{$\mathfrak{r}$}{\smallrecord{
        \smalltfield{x$_0$}{\textit{Ind}}}}} . $\lambda
  P$:\textit{Ppty} . $P\{c.\mathfrak{s}.\text{x}_0\}\urcorner$

\item[\textnormal{LexReflPron($T_{\text{phon}}$)} New!] \mbox{}

If $T_{\text{phon}}$ is a phonological type, then
LexReflPron($T_{\text{phon}}$) is
\begin{quote}
Lex($T_{\mathrm{phon}}$, \textit{NP}) \d{$\wedge$}
\smallrecord{\smallmfield{cont}{SemReflPron$^{\mathfrak{S}}$}{\textit{ContType}}}
\end{quote}


\item[\textnormal{SemNumeral($n$)}] \mbox{}

  If $n$ is a real number, then SemNumeral($n$) is
  \begin{quote}
    $\ulcorner\lambda c$:\textit{Cntxt} . $\lambda P$:\textit{Ppty}
    . $P$(\smallrecord{\field{x}{$n$}})$\urcorner$
  \end{quote}

  
\item[\textnormal{Lex$_{\mathrm{numeral}}$($T_{\mathrm{phon}}$, $n$)} Revised!] \mbox{}

    If $T_{\mathrm{phon}}$ is a phonological type and $n$ is a real
    number, then Lex$_{\mathrm{numeral}}$($T_{\mathrm{phon}}$, $n$) is
    \begin{quote}
      Lex($T_{\mathrm{phon}}$, \textit{NP}) \d{$\wedge$}
      \smallrecord{\smallmfield{cnt}{SemNumeral($n$)$^{\mathfrak{S}}$}{\textit{ContType}}}
    \end{quote}
    

  
\item[\textnormal{SemIndefArt}] \mbox{}

  % $\ulcorner\lambda c$:\textit{Rec} . \\
% \hspace*{1em}$\lambda Q$:\textit{Ppty} . \\
% \hspace*{2em} $\lambda P$:\textit{Ppty}
% . \record{\mfield{restr}{$Q$}{\textit{Ppty}} \\
%           \mfield{scope}{$P$}{\textit{Ppty}} \\
%           \tfield{e}{exist(restr, scope)}}$\urcorner$

$\lambda Q$:\textit{Ppty} . \\
  \hspace*{1em}$\ulcorner\lambda c$:\textit{Cntxt} . \\
  \hspace*{2em}$\lambda P$:\textit{Ppty} . \\
  \hspace*{3em}\record{
    \mfield{restr}{$Q$}{\textit{Ppty}}\\
    \mfield{scope}{$P|_{\mathfrak{F}(\text{restr})}$}{\textit{Ppty}}\\
    \tfield{e}{exist(restr, scope)}}$\urcorner$



 

        
      \item[\textnormal{Lex$_{\mathrm{IndefArt}}$($T_{\mathrm{Phon}}$)}
        Revised!]
        \mbox{}

        If $T_{\mathrm{Phon}}$ is a phonological type, then
        Lex$_{\mathrm{IndefArt}}$($T_{\mathrm{Phon}}$) is
        \begin{quote}
          Lex($T_{\mathrm{Phon}}$, \textit{Det}) \d{$\wedge$}
          \smallrecord{\smallmfield{cont}{SemIndefArt$^{\mathfrak{S}}$}{\textit{ContType}}}
        \end{quote}

        
      \item[\textnormal{SemUniversal}] \mbox{}

               $\lambda Q$:\textit{Ppty} . \\
  \hspace*{1em}$\ulcorner\lambda c$:\textit{Cntxt} . \\
  \hspace*{2em}$\lambda P$:\textit{Ppty} . \\
  \hspace*{3em}\record{
    \mfield{restr}{$Q$}{\textit{Ppty}}\\
    \mfield{scope}{$P|_{\mathfrak{F}(\text{restr})}$}{\textit{Ppty}}\\
    \tfield{e}{every(restr, scope)}}$\urcorner$

 

  
\item[\textnormal{Lex$_{\mathrm{Universal}}$($T_{\mathrm{Phon}}$)} Revised!]
        \mbox{}

        If $T_{\mathrm{Phon}}$ is a phonological type, then
        Lex$_{\mathrm{Universal}}$($T_{\mathrm{Phon}}$) is
        \begin{quote}
          Lex($T_{\mathrm{Phon}}$, \textit{Det}) \d{$\wedge$}
          \smallrecord{\smallmfield{cont}{SemUniversal$^{\mathfrak{S}}$}{\textit{ContType}}}
        \end{quote}

        \item[\textnormal{SemDefArt}] \mbox{}

 %  $\lambda c$:\textit{Rec} . \\
% \hspace*{1em}$\lambda Q$:\textit{Ppty} . \\
% \hspace*{2em} $\lambda P$:\textit{Ppty}
% . \record{\mfield{restr}{$Q$}{\textit{Ppty}} \\
%           \mfield{scope}{$P$}{\textit{Ppty}} \\
%           \tfield{e}{the(restr, scope)}}

                    $\lambda Q$:\textit{Ppty} . \\
                    \hspace*{1em}$\ulcorner\lambda c$:\smallrecord{
                      \footnotesize{\textit{Cntxt}}\\
    \smalltfield{$\mathfrak{c}$}{\smallrecord{
        \smalltfield{e}{unique($Q$)}}}} . \\
  \hspace*{2em}$\lambda P$:\textit{Ppty} . \\
  \hspace*{3em}\record{
    \mfield{restr}{$Q\!\restriction\!c.\mathfrak{c}$.e}{\textit{Ppty}}\\
    \mfield{scope}{$P|_{\mathfrak{F}(\text{restr})}$}{\textit{Ppty}}\\
    \tfield{e}{every(restr, scope)}}$\urcorner$


        
      \item[\textnormal{Lex$_{\mathrm{DefArt}}$($T_{\mathrm{Phon}}$)} Revised!]
        \mbox{}

        If $T_{\mathrm{Phon}}$ is a phonological type, then
        Lex$_{\mathrm{IndefArt}}$($T_{\mathrm{Phon}}$) is
        \begin{quote}
          Lex($T_{\mathrm{Phon}}$, \textit{Det}) \d{$\wedge$}
          \smallrecord{\smallmfield{cont}{SemDefArt$^{\mathfrak{S}}$}{\textit{ContType}}}
        \end{quote}

      \item[\textnormal{SemIntransVerb($T_{\mathrm{bg}}$, $p$)}]
        \mbox{}

        If $T_{\text{bg}}$ is a record type (for context) and $p$ is a
        predicate with arity $\langle\textit{Ind}\rangle$, then SemIntransVerb($T_{\mathrm{bg}}$, $p$) is
        \begin{quote}
          $\ulcorner\lambda c$:$T_{\mathrm{bg}}$ . $\ulcorner\lambda
          r$:\smallrecord{\smalltfield{x}{\textit{Ind}}}
          . \record{\tfield{e}{$p$($r$.x)}}$\urcorner\urcorner$
        \end{quote}

        If $T_{\text{bg}}\sqsubseteq$\smallrecord{\smalltfield{$\mathfrak{c}$}{\textit{Rec}}} is a record type (for context) and $p$ is a
        predicate with arity $\langle\textit{Rec}, \textit{Rec}\rangle$, then SemIntransVerb($T_{\mathrm{bg}}$, $p$) is
        \begin{quote}
          $\ulcorner\lambda c$:$T_{\mathrm{bg}}$ . $\ulcorner\lambda
          r$:\smallrecord{\smalltfield{x}{\textit{Rec}}}
          . \record{\tfield{e}{$p$($r$.x, $c.\mathfrak{c}$)}}$\urcorner\urcorner$
        \end{quote}

        
      \item[\textnormal{Lex$_{\mathrm{IntransVerb}}$($T_{\mathrm{phon}}$,
          $T_{\mathrm{bg}}$, $p$)} Revised!] \mbox{}

        % If $T_{\mathrm{phon}}$ is a phonological type,
        % $T_{\mathrm{bg}}$ a record type (for context) and $p$ is a
        % predicate with arity $\langle\textit{Ind}\rangle$, then Lex$_{\mathrm{IntransVerb}}$($T_{\mathrm{phon}}$,
        % $T_{\mathrm{bg}}$, $p$) is
        % \begin{quote}
        %   Lex($T_{\mathrm{phon}}$, \textit{V$_i$}) \d{$\wedge$}
        %   \smallrecord{\smallmfield{cnt}{SemIntransVerb($T_{\mathrm{bg}}$, $p$)}{\textit{PPpty}}}
        % \end{quote}

        If $T_{\mathrm{phon}}$ is a phonological type,
        $T_{\mathrm{bg}}\sqsubseteq$\smallrecord{\smalltfield{$\mathfrak{c}$}{\textit{Rec}}} a record type (for context) and $p$ is a
        predicate with arity $\langle\textit{Ind}\rangle$ or $\langle\textit{Rec}, \textit{Rec}\rangle$, then Lex$_{\mathrm{IntransVerb}}$($T_{\mathrm{phon}}$,
        $T_{\mathrm{bg}}$, $p$) is
        \begin{quote}
          Lex($T_{\mathrm{phon}}$, \textit{V$_i$}) \d{$\wedge$}
          \smallrecord{\smallmfield{cnt}{SemIntransVerb($T_{\mathrm{bg}}$, $p$)$^{\mathfrak{S}}$}{\textit{ContType}}}
        \end{quote}

        
      \item[\textnormal{SemTransVerb($T_{\mathrm{bg}}, p$)}] \mbox{}

        If $T_{\text{bg}}$ is a record type (for context) and $p$ is a
        predicate with arity $\langle\textit{Ind},\textit{Ind}\rangle$, then SemTransVerb($T_{\mathrm{bg}}$, $p$) is
        \begin{quote}
          $\ulcorner\lambda c$:$T_{\mathrm{bg}}$ . $\lambda
          \mathcal{Q}$:\textit{Quant} . $\ulcorner\lambda
          r_1$:\smallrecord{\smalltfield{x}{\textit{Ind}}} . $\mathcal{Q}(\ulcorner\lambda r_2$:\smallrecord{\smalltfield{x}{\textit{Ind}}}
          . \record{\tfield{e}{$p$($r_1$.x, $r_2$.x)}}$\urcorner)\urcorner\urcorner$
        \end{quote}

        If $T_{\text{bg}}$ is a record type (for context) and $p$ is a
  predicate with arity $\langle\textit{Ind},\textit{Quant}\rangle$, then SemTransVerb($T_{\mathrm{bg}}$, $p$) is
  \begin{quote}
    $\ulcorner\lambda c$:$T_{\mathrm{bg}}$ . $\lambda
    \mathcal{Q}$:\textit{Quant} . $\ulcorner\lambda
    r$:\smallrecord{\smalltfield{x}{\textit{Ind}}}
    . \record{\tfield{e}{$p$($r$.x, $\mathcal{Q}$)}}$\urcorner)\urcorner$
  \end{quote}
        
\item[\textnormal{Lex$_{\mathrm{TransVerb}}$($T_{\mathrm{phon}}$,
          $T_{\mathrm{bg}}$, $p$)} Revised!] \mbox{}

        If $T_{\mathrm{phon}}$ is a phonological type,
        $T_{\mathrm{bg}}$ a record type (for context) and $p$ is a
        predicate with arity $\langle\textit{Ind},\textit{Ind}\rangle$
        or $\langle\textit{Ind},\textit{Quant}\rangle$, then Lex$_{\mathrm{TransVerb}}$($T_{\mathrm{phon}}$,
        $T_{\mathrm{bg}}$, $p$) is
        \begin{quote}
          Lex($T_{\mathrm{phon}}$, \textit{V$_t$}) \d{$\wedge$}
          \smallrecord{\smallmfield{cnt}{SemTransVerb($T_{\mathrm{bg}}$, $p$)$^{\mathfrak{S}}$}{\textit{ContType}}}
        \end{quote}

        

        
      \item[\textnormal{SemBe}] \mbox{}

        % $\lambda c:T_{\text{bg}}$ . \\
%         \hspace*{1em}$\lambda\mathcal{Q}$:\textit{Quant} . \\
% \hspace*{2em} $\ulcorner\lambda r_1$:\smallrecord{\tfield{x}{\textit{Ind}}}
% . \\
% \hspace*{3em} $\mathcal{Q}$($\ulcorner\lambda
% r_2$:\smallrecord{\tfield{x}{\textit{Ind}}}
% . \record{\mfield{x}{$r_2$.x, $r_1$.x}{\textit{Ind}}\\
% \tfield{e}{be(x)}}$\urcorner$)$\urcorner$

        \begin{description}
          
        \item[\textnormal{SemBe$_{\text{ID}}$}] \mbox{}

          $\ulcorner\lambda c$:\smallrecord{
            \footnotesize{\textit{Cntxt}}\\
    \smalltfield{$\mathfrak{c}$}{\smallrecord{
        \smalltfield{ty}{\textit{Type}}}}} . \\
        \hspace*{1em}$\lambda\mathcal{Q}$:\textit{Quant} . \\
        \hspace*{2em} $\ulcorner\lambda r_1$:\smallrecord{
          \smalltfield{x}{$c.\mathfrak{c}$.ty}}
. \\
\hspace*{3em} $\mathcal{Q}$($\ulcorner\lambda
r_2$:\smallrecord{
  \smalltfield{x}{$c.\mathfrak{c}$.ty}}
. \record{\mfield{x}{$r_1$.x, $r_2$.x}{$c.\mathfrak{c}$.ty}\\
  \tfield{e}{be(x)}}$\urcorner$)$\urcorner\urcorner$

\item[\textnormal{SemBe$_{\text{scalar}}$}] \mbox{}

  $\ulcorner\lambda c$:\smallrecord{
    \footnotesize{\textit{Cntxt}}\\
    \smalltfield{$\mathfrak{c}$}{\smallrecord{
        \smalltfield{ty}{\textit{Type}}\\
      \smalltfield{sc}{(ty$\rightarrow$\textit{Real})}}}} . \\
        \hspace*{1em}$\lambda\mathcal{Q}$:\textit{Quant} . \\
        \hspace*{2em} $\ulcorner\lambda r_1$:\smallrecord{
          \smalltfield{x}{$c.\mathfrak{c}$.ty}}
. \\
\hspace*{3em} $\mathcal{Q}$($\ulcorner\lambda
r_2$:\smallrecord{
  \smalltfield{x}{\textit{Real}}}
. \record{\mfield{x}{$c.\mathfrak{c}$.sc($r_1$.x), $r_2$.x}{\textit{Real}}\\
  \tfield{e}{be(x)}}$\urcorner$)$\urcorner\urcorner$

\end{description}


\item[\textnormal{Lex$_{\mathrm{be}}$($T_{\mathrm{Phon}}$)} Revised!] \mbox{}

  If $T_{\mathrm{Phon}}$ is a phonological type, then
  Lex$_{\mathrm{be}_{\text{ID}}}$($T_{\mathrm{Phon}}$) is
\begin{quote}
  Lex($T_{\mathrm{Phon}}$,
\textit{V}) \d{$\wedge$}
\smallrecord{\smallmfield{cont}{{SemBe$_{\text{ID}}$}$^{\mathfrak{S}}$}{\textit{ContType}}}
\end{quote}

If $T_{\mathrm{Phon}}$ is a phonological type, then
  Lex$_{\mathrm{be}_{\text{scalar}}}$($T_{\mathrm{Phon}}$) is
\begin{quote}
  Lex($T_{\mathrm{Phon}}$,
\textit{V}) \d{$\wedge$}
\smallrecord{\smallmfield{cont}{{SemBe$_{\text{scalar}}$}$^{\mathfrak{S}}$}{\textit{ContType}}}
\end{quote}

\item[\textnormal{$\mathrm{FrameType}(p)$}] \mbox{}

  $\mathrm{FrameType}$ is a partial function on predicates, $p$, with
  arity $\langle\textit{Ind}\rangle$ which can be defined for
  particular agents and particular times, which obeys the constraint:
  \begin{quote}
    $\mathrm{FrameType}(p)\sqsubseteq$ \record{
    \tfield{x}{\textit{Ind}}\\
    \tfield{e}{$p$(x)}}
\end{quote}

\item[\textnormal{$p$\_frame}] \mbox{}

  \begin{enumerate} 
 
\item If $p$ is a predicate in the domain of $\mathrm{FrameType}$,
  then $p$\_frame is a predicate with arity $\langle\textit{Rec}\rangle$. 
 
\item $e:p\_\text{frame}(r)$ iff $r:\mathrm{FrameType}(p)$ and $e=r$ 
 
\end{enumerate}

\item[\textnormal{$p$\_pl}] \mbox{}

  \begin{enumerate} 
 
\item If $p$ is a singular predicate (i.e. there is no $p'$ such that
  $p=p'\_\text{pl}$) with arity $\langle T\rangle$, then
  $p\_\text{pl}$ is a predicate with arity $\langle\mathrm{plurality}(T)\rangle$ 
 
\item $e:p\_\text{pl}(A)$ if for all $a\in A$, $e:p(a)$ 
 
\end{enumerate} 
  

\item[\textnormal{CommonNounIndToFrame}] \mbox{}

  If $T_{\mathrm{phon}}$ is a phonological type, $p$ is a predicate with
arity $\langle\textit{Ind}\rangle$ and
$T_\mathrm{bg}$ is a record type (the ``background type'' or
``presupposition'') then \\ 
\mbox{CommonNounIndToFrame(Lex$_{\mathrm{CommonNoun}}$($T_{\mathrm{phon}}$,
  $T_\mathrm{bg}$, $p$))} =
\begin{quote}
  Lex$_{\mathrm{CommonNoun}}$($T_{\mathrm{phon}}$,
  $T_\mathrm{bg}$, $p$\_frame)
\end{quote}

\item[\textnormal{RestrictCommonNoun} Revised!] \mbox{}

  If $T_{\mathrm{phon}}$ is a phonological type, $p$ is a predicate,
$T_{\mathrm{bg}}$ and $T_{\mathrm{res}}$ are record types and $\Sigma$
is Lex$_{\mathrm{CommonNoun}}$($T_{\mathrm{phon}}$, $T_{\mathrm{bg}}$,
$p$), then RestrictCommonNoun($\Sigma$, $T_{\mathrm{res}}$) is
\begin{quote}
$\Sigma$
\fbox{\d{$\wedge$}} \record{
  \mfield{cont}{${\ulcorner\lambda c\!:\!T_{\mathrm{bg}}\ .\
    \ulcorner\text{SemCommonNoun}(T_{\mathrm{bg}},p)(c)\!\mid_{T_{\mathrm{res}}}\urcorner\urcorner}^{\mathfrak{S}}$}{\textit{ContType}}}
\end{quote}

\item[\textnormal{IntransVerbIndToFrame}] \mbox{}

  If $T_{\mathrm{phon}}$ is a phonological type, $p$ is a predicate with
arity $\langle\textit{Ind}\rangle$ and
$T_\mathrm{bg}$ is a record type (the ``background type'' or
``presupposition'') then \\ 
\mbox{IntransVerbIndToFrame(Lex$_{\mathrm{IntransVerb}}$($T_{\mathrm{phon}}$,
  $T_\mathrm{bg}$, $p$))} =
\begin{quote}
  Lex$_{\mathrm{IntransVerb}}$($T_{\mathrm{phon}}$,
  $T_\mathrm{bg}$, $p$\_frame)
\end{quote}

\item[\textnormal{PluralCommonNoun}] \mbox{}

  We assume that `pluralnoun' is a function that maps phonological
  types for singular common nouns to corresponding phonological types
  for plural common nouns.

  If $T_{\text{phon}}$ is a (singular) phonological type, $p$ is a
  singular predicate with arity $\langle T\rangle$ and $T_{\text{bg}}$
  is a record type then
  PluralCommonNoun(Lex$_{\text{CommonNoun}}$($T_{\text{phon}}$,
  $T_{\text{bg}}$, $p$)) =
  \begin{quote}
    Lex$_{\text{CommonNoun}}$(pluralnoun($T_{\text{phon}}$),
    $T_{\text{bg}}$, $p$\_pl)
  \end{quote}

  
\item[\textnormal{PluralIntransVerb}] \mbox{}

  We assume that `pluralverb' is a function that maps phonological
  types for singular verbs to corresponding phonological types
  for plural verbs.

  If $T_{\text{phon}}$ is a (singular) phonological type, $p$ is a
  singular predicate with arity $\langle T\rangle$ and $T_{\text{bg}}$
  is a record type then
  PluralIntransVerb(Lex$_{\text{IntransVerb}}$($T_{\text{phon}}$,
  $T_{\text{bg}}$, $p$)) =
  \begin{quote}
    Lex$_{\text{IntransVerb}}$(pluralverb($T_{\text{phon}}$),
    $T_{\text{bg}}$, $p$\_pl)
  \end{quote} 
        
\item[\textnormal{TransVerbToVerbPhrase} Revised!] \mbox{}

  If $T_{\text{phon}}$ is a phonological type, $T_{\text{bg}}$ a
  context type, $p$ is a predicate with arity
  $\langle\textit{Ind},\textit{Ind}\rangle$ or
  $\langle\textit{Ind},\textit{Quant}\rangle$ and $\Sigma$ is
  Lex$_{\text{TransVerb}}$($T_{\text{phon}}$, $T_{\text{bg}}$, $p$),
  then TransVerbToVerbPhrase($\Sigma$) is
  \begin{quote}
    $\Sigma$ \fbox{\d{$\wedge$}} \record{
      \mfield{cat}{vp}{\textit{Cat}}\\
      \mfield{cont}{$\Sigma.\text{cont}/\textit{Quant}$}{\textit{ContType}}}
  \end{quote}
  where for any content type, $T$, such that $\varphi:T$ implies
  $\varphi:(\textit{Quant}\rightarrow\textit{PPpty})$, $T/\textit{Quant}$
  is a type such that 
  \begin{quote}
    $\varphi:T$ iff
    $\ulcorner\lambda c$:$T$.bg\d{$\wedge$}\smallrecord{
      \smalltfield{$\mathfrak{g}$}{\smallrecord{
          \smalltfield{x$_0$}{\textit{Ind}}}}}
    . $\varphi$($c$)($\lambda P$:\textit{Ppty}
    . $P\{c.\mathfrak{g}.\text{x}_0\}$)$\urcorner:T/\textit{Quant}$
  \end{quote}
  
  
  
\end{description}



\subsubsection{Constituent structure} 
\begin{description}
  
%\item[Uninterpreted phrase structure]
  
\item[\textnormal{RuleDaughters($T_{\text{daughters}}$,
$T_{\text{mother}}$)}] \mbox{}

If $T_{\text{mother}}$ is a sign type and $T_{\text{daughters}}$ is a
type of strings of signs then
\begin{quote}
RuleDaughters($T_{\text{daughters}}$,
$T_{\text{mother}}$)
\end{quote}
is
\begin{quote}
  $\lambda u\! :\! T_{\text{daughters}}$\ . $T_{\text{mother}}$ \d{$\wedge$} \smallrecord{\smalltfield{syn}{\smallrecord{\smallmfield{daughters}{$u$}{$T_{\text{daughters}}$}}}}
\end{quote}

\item[\textnormal{ConcatPhon}] \mbox{}

  $\lambda
u$:\smallrecord{\smalltfield{s-event}{\smallrecord{\smalltfield{e}{\textit{Phon}}}}}$^+$\
. \\
\hspace*{1em}\record{\tfield{s-event}{\record{\mfield{e}{concat$_i$($u[i]$.s-event.e)}{\textit{Phon}}}}}

\item[\textnormal{$T_{\text{mother}}\longrightarrow T_{\text{daughter}_1},\ldots
    T_{\text{daughter}_n}$}] \mbox{}

  If $T_{\text{mother}}$ is a sign type and
  $T_{\text{daughter}_1},\ldots T_{\text{daughter}_n}$ are sign types,
  then
  \begin{quote}
    $T_{\text{mother}}\longrightarrow T_{\text{daughter}_1}\ldots
    T_{\text{daughter}_n}$
  \end{quote}
  represents
  \begin{quote}
RuleDaughters($T_{\text{mother}}$,
${T_{\text{daughter}_1}}^\frown\ldots^\frown T_{\text{daughter}_n}$)\d{\d{$\wedge$}}ConcatPhon 
\end{quote}

% \item[Content combination operations]

\item[\textnormal{$B(\alpha)$} New!] \mbox{}

  If $\alpha$ is a parametric content,
  \begin{quote}
    $\ulcorner\lambda c\!:\!T\ .\ \varphi\dep{c}\urcorner$
  \end{quote}
  then $B(\alpha)$ is
  \begin{quote}
    $\ulcorner\lambda c$:$T$\fbox{\d{$\wedge$}}\smallrecord{
      \smalltfield{$\mathfrak{l}$}{\textit{Assgnmnt}}}
    . $\varphi\dep{c}\urcorner$
  \end{quote}
  

\item[\textnormal{$\alpha\text{@}\beta$}] \mbox{}

  If $\alpha$ : \smallrecord{\smalltfield{bg}{\textit{CntxtType}}\\
                           \smalltfield{fg}{(bg$\rightarrow$($T_1\rightarrow
                             T_2$))}} 
and $\beta$ : \smallrecord{\smalltfield{bg}{\textit{CntxtType}}\\
                           \smalltfield{fg}{(bg$\rightarrow T_1$)}}
                         then the \textit{combination of $\alpha$ and
    $\beta$  based on functional application}, $\alpha\text{@}\beta$, is
  \begin{quote}
   $\ulcorner\lambda c$:$[\alpha.\text{bg}]_{\mathfrak{c}\leadsto\mathfrak{c}.\text{f}}$
      \d{$\wedge$}$\mathrm{incr}([\beta.\text{bg}]_{\mathfrak{c}\leadsto\mathfrak{c}.\text{a}},\alpha.\text{bg})$
      . \\
      \hspace*{2em}$[\alpha]_{\mathfrak{c}\leadsto\mathfrak{c}.\text{f}}(c)(\mathrm{incr}([\beta.\text{fg}]_{\mathfrak{c}\leadsto\mathfrak{c}.\text{a}},\alpha.\text{bg})(c))\urcorner$

      
    \end{quote}

      \item[\textnormal{$\alpha\text{@}_{\mathrm{wh}_{i,j}}\beta$}]
    \mbox{}

      If
  \begin{enumerate}
  \item $\alpha$ : \smallrecord{
      \smalltfield{bg}{\textit{CntxtType}}\\
      \smalltfield{fg}{(bg$\rightarrow$\textit{Quant})}},
    
  \item $\beta$ : \smallrecord{
      \smalltfield{bg}{\textit{CntxtType}}\\
      \smalltfield{fg}{(bg$\rightarrow$\textit{RecType})}},
    
  \item $\alpha$.bg $\sqsubseteq$ \smallrecord{
      \smalltfield{$\mathfrak{w}$}{\smallrecord{
          \smalltfield{x$_i$}{\textit{Ind}}}}} for some natural number,
    $i$, and
    
  \item $\beta$.bg $\sqsubseteq$ \smallrecord{
      \smalltfield{$\mathfrak{g}$}{\smallrecord{
          \smalltfield{x$_j$}{\textit{Ind}}}}} for some natural number,
    $j$,
  \end{enumerate}
  then \textit{the \textit{wh}$_{i,j}$-combination of $\alpha$ and
    $\beta$}, $\alpha\text{@}_{\mathrm{wh}_{i,j}}\beta$, is
  \begin{quote}
    $\ulcorner\lambda
    c$:([$\alpha$.bg$\ominus\mathrm{paths}_{\mathfrak{w}.\text{x}_i}(\alpha.\text{bg})]_{\mathfrak{c}\leadsto\mathfrak{c}.\text{f}}$\d{$\wedge$}\\
    \hspace*{5em}$\mathrm{incr}([\beta.\text{bg}\ominus\mathrm{paths}_{\mathfrak{g}.\text{x}_j}(\beta.\text{bg})]_{\mathfrak{c}\leadsto\mathfrak{c}.a},\alpha.\text{bg})$)
    . \\
    \hspace*{1em} $\mathfrak{P}(\ulcorner\lambda r_1$:$[\alpha.\text{bg}^{\mathfrak{w}.\text{x}_i}]_{\mathfrak{w}.\text{x}_i\leadsto\text{x}}$ . \\
    \hspace*{4em}$\alpha_{\mathfrak{c}\leadsto\mathfrak{c}.\text{f},\mathfrak{w}.\text{x}_i\leadsto\text{x}}(c[r_1])$
    ($\mathfrak{P}(\ulcorner\lambda r_2$:$[\beta.\text{bg}^{\mathfrak{g}.\text{x}_j}]_{\mathfrak{g}.\text{x}_j\leadsto\text{x}}$ . \\
    \hspace*{15em}$\mathrm{incr}(\beta_{\mathfrak{c}\leadsto\mathfrak{c}.\text{a},\mathfrak{g}.\text{x}_j\leadsto\text{x}},\alpha.\text{bg})(c[r_2])\urcorner)$)$\urcorner)\urcorner$
  \end{quote} 

  \item[\textnormal{$\alpha\text{@\!@}\beta$}] \mbox{}

  If $\alpha$ : ($T_1\rightarrow$ \smallrecord{\smalltfield{bg}{\textit{CntxtType}}\\
                           \smalltfield{fg}{(bg$\rightarrow T_2$)}}) 
                         and $\beta$ : \smallrecord{\smalltfield{bg}{\textit{CntxtType}}\\
                           \smalltfield{fg}{(bg$\rightarrow T_1$)}}
                         then the \textit{combination of $\alpha$ and
    $\beta$  based on functional application}, $\alpha\text{@\!@}\beta$, is
  \begin{quote}
    $\ulcorner\lambda c$:\record{
      \tfield{$\mathfrak{c}$}{\record{
          \tfield{s}{$\beta$.bg}\\
          \tfield{f}{$\alpha(\beta(s))$.bg}\\
          \mfield{a}{s.$\mathfrak{c}$}{\textit{PropCntxt}}}}\\
    \mfield{$\mathfrak{s}$}{$\mathfrak{c}$.s.$\mathfrak{s}$}{\textit{Assgnmnt}}}
      . \\*[\baselineskip]
      \hspace*{10em}$[\alpha]_{\mathfrak{c}\leadsto\mathfrak{c}.\text{f}}([\beta]_{\mathfrak{c}\leadsto\mathfrak{c}.\text{a}}(c))(c)\urcorner$
    \end{quote}

    \item[\textnormal{$\alpha\text{@}_{\&}\beta$}] \mbox{}

  If $T$ is a type, $\alpha:{^T\textit{PPpty}}$ and
  $\beta:{^T\textit{PPpty}}$ 
  then \textit{the property conjunction
    combination of $\alpha$ and $\beta$}, $\alpha\text{@}_{\&}\beta$,
  is
\begin{quote}
  $\lambda
  c$:$[\alpha.\text{bg}]_{\mathfrak{c}\leadsto\mathfrak{c}.\text{f}}$\d{$\wedge$}$\mathrm{incr}([\beta.\text{bg}]_{\mathfrak{c}\leadsto\mathfrak{c}.\text{a}},\alpha.\text{bg})$
        . $\alpha_{\mathfrak{c}\leadsto\mathfrak{c}.\text{f}}(c)\&\mathrm{incr}([\beta]_{\mathfrak{c}\leadsto\mathfrak{c}.\text{a}},\alpha.\text{bg})(c)$
      \end{quote}

     
    \item[\textnormal{$\alpha\mathcal{O}_{i,j}\beta$} New!] \mbox{}

      If $\mathcal{O}$ is a combination operator, then so is
$\mathcal{O}_{i,j}$, where $i$ and $j$ are natural numbers.

If
$\alpha$ and $\beta$ are parametric contents such that
\begin{enumerate}
\item $\alpha\mathcal{O}\beta$ is defined
  
\item $\alpha.\text{bg}\sqsubseteq$
\smallrecord{
  \smalltfield{$\mathfrak{s}$}{\smallrecord{
      \smalltfield{x$_i$}{\textit{Ind}}}}}
\item $\mathrm{incr}(\beta.\text{bg},\alpha.\text{bg})\sqsubseteq$
\smallrecord{
  \smalltfield{$\mathfrak{s}$}{\smallrecord{
      \smalltfield{x$_j$}{\textit{Ind}}}}}
\item $\mathrm{incr}(\beta.\text{bg},\alpha.\text{bg})\not\sqsubseteq$
\smallrecord{
  \smalltfield{$\mathfrak{q}$}{\smallrecord{
      \smalltfield{x$_j$}{\textit{PQuant}}}}} 
\item $\mathrm{incr}(\beta.\text{bg},\alpha.\text{bg})\not\sqsubseteq$
  \smallrecord{
    \smalltfield{$\mathfrak{l}$}{\smallrecord{
        \smalltfield{x$_j$}{\textit{PQuant}}}}}
\end{enumerate}
then $\alpha\mathcal{O}_{i,j}\beta$ is
\begin{quote}
  $[\alpha\mathcal{O}\beta]_{\mathfrak{s}.\text{x}_j\leadsto\mathfrak{s}.\text{x}_i}$
\end{quote}


      
    \item[\textnormal{$T_1\mathcal{O}^{\mathfrak{S}}T_2$} New!]
      \mbox{}

      Suppose that $T_1$ and
$T_2$ are of type \textit{ContType} and that $\mathcal{O}$ is a combination
operation such as @,\ldots as characterized above, then we say $T_1\mathcal{O}^{\mathfrak{S}}T_2$
is also a type with the witness condition:
\begin{quote} 
If $\alpha:T_1$, $\beta:T_2$ and $\alpha\mathcal{O}\beta$ is defined,
then $\alpha\mathcal{O}\beta:T_1\mathcal{O}^{\mathfrak{S}}T_2$.
Nothing else is a witness for $T_1\mathcal{O}^{\mathfrak{S}}T_2$. 
\end{quote}

\item[\textnormal{$T_1\mathcal{O}^{\mathfrak{S},B}T_2$} New!] \mbox{}

  Suppose that $T_1$ and $T_2$ are of type \textit{ContType} and that
$\mathcal{O}$ is a combination operation such as @ etc., then we say
that $T_1\mathcal{O}^{\mathfrak{S},B}T_2$ is also a type with the
witness condition:
\begin{quote} 
If $\alpha:T_1$, $\beta:T_2$ and $\alpha\mathcal{O}\beta$ is defined,
then $B(\alpha\mathcal{O}\beta):T_1\mathcal{O}^{\mathfrak{S},B}T_2$.
Nothing else is a witness for $T_1\mathcal{O}^{\mathfrak{S},B}T_2$. 
\end{quote} 


    
%    \item[Forward application] 
    
\item[\textnormal{ContForwardApp$_{\mathfrak{S},\mathcal{O}}$} New!]
  \mbox{}

  If $\mathcal{O}$ is one of @,\ldots as defined above, then
ContForwardApp$_{\mathfrak{S},\mathcal{O}}$ is
\begin{quote}
  $\lambda u$:\smallrecord{
    \smalltfield{cont}{\textit{ContType}}}$^\frown$\smallrecord{
    \smalltfield{cont}{\textit{ContType}}} . \smallrecord{
    \smalltfield{cont}{$\mathfrak{S}(u[0].\text{cont}\mathcal{O}^{\mathfrak{S}}u[1].\text{cont})$}}
\end{quote}

\paragraph{Note:}  This single characterization of ContForwardApp
replaces all the previous variants that were defined in previous
chapters.

\item[\textnormal{ContForwardApp$_{\mathfrak{S},\mathcal{O},B}$} New!]
  \mbox{}

  If $\mathcal{O}$ is a combination operator, then
  ContForwardApp$_{\mathfrak{S},\mathcal{O},B}$ is
  \begin{quote}
    $\lambda u$:\smallrecord{
      \smalltfield{cont}{\textit{ContType}}}$^\frown$\smallrecord{
      \smalltfield{cont}{\textit{ContType}}} . \smallrecord{
      \smalltfield{cont}{$\mathfrak{S}(u[0].\text{cont}\mathcal{O}^{\mathfrak{S},B}u[1].\text{cont})$}}
  \end{quote}

  
\item[\textnormal{ContForwardApp$_{\mathfrak{S},\mathcal{O},\mathfrak{A}}$}
  New!] \mbox{}

  If $\mathcal{O}$ is a combination operator, then
  ContForwardApp$_{\mathfrak{S},\mathcal{O},\mathfrak{A}}$ is
  \begin{quote}
    $\lambda u$:\smallrecord{
      \smalltfield{cont}{\textit{ContType}}}$^\frown$\smallrecord{
      \smalltfield{cont}{\textit{ContType}}} . \smallrecord{
      \smalltfield{cont}{$\mathfrak{S}(\mathfrak{A}(u[0].\text{cont}\mathcal{O}^{\mathfrak{S}}u[1].\text{cont}))$}}
  \end{quote}

% \item[\textnormal{ContForwardApp($T_{\text{arg}}$, $T_{\text{res}}$)}] \mbox{}

%   If $T_{\text{arg}}$ and $T_{\text{res}}$ are types, then
%   ContForwardApp($T_{\text{arg}}$, $T_{\text{res}}$) is
% \begin{quote}
%   $\lambda
% u$:\smallrecord{\smalltfield{cont}{\smallrecord{\smalltfield{bg}{\textit{CntxtType}}\\
%                                                \smalltfield{fg}{(bg$\rightarrow$($T_{\text{arg}}\rightarrow T_{\text{res}}$))}}}}$^{\frown}$
%    \smallrecord{\smalltfield{cont}{\smallrecord{\smalltfield{bg}{\textit{CntxtType}}\\
%                                                \smalltfield{fg}{(bg$\rightarrow T_{\text{arg}}$)}}}} . \\
% \hspace*{2em}\smallrecord{\smallmfield{cont}{$u$[0].cont@$u$[1].cont}{\smallrecord{\smalltfield{bg}{\textit{CntxtType}}\\
%                                                                                 \smalltfield{fg}{(bg$\rightarrow
%                                                                                   T_{\text{res}}$)}}}}
% \end{quote}



% % \item[\textnormal{$\alpha\text{@}\beta$}] \mbox{}

% %   If $\alpha$ : \smallrecord{\smalltfield{bg}{\textit{RecType}}\\
% %                            \smalltfield{fg}{(bg$\rightarrow$($T_1\rightarrow
% %                              T_2$))}} 
% % and $\beta$ : \smallrecord{\smalltfield{bg}{\textit{RecType}}\\
% %                            \smalltfield{fg}{(bg$\rightarrow T_1$)}}
% %                          then the \textit{combination of $\alpha$ and
% %     $\beta$  based on functional application}, $\alpha\text{@}\beta$, is
% % \begin{quote}
% %   $\ulcorner\lambda c$:$[\alpha.\text{bg}]_{\mathfrak{c}\leadsto\mathfrak{c}.\text{f}}$
% %       \d{$\wedge$}$\mathrm{incr}_{\mathfrak{s}.\text{x}}([\beta.\text{bg}]_{\mathfrak{c}\leadsto\mathfrak{c}.\text{a}},\alpha.\text{bg})$
% %       . \\
% %       \hspace*{2em}
% % $[\alpha]_{\mathfrak{c}\leadsto\mathfrak{c}.\text{f}}(c)(\mathrm{incr}_{\mathfrak{s}.\text{x}}([\beta.\text{fg}]_{\mathfrak{c}\leadsto\mathfrak{c}.\text{a}},\alpha.\text{bg})(c))\urcorner$
      
% % \end{quote}


% % \item[\textnormal{ContForwardApp($T_{\text{arg}}$, $T_{\text{res}}$)}] \mbox{}

% %   If $T_{\text{arg}}$ and $T_{\text{res}}$ are types, then
% %   ContForwardApp($T_{\text{arg}}$, $T_{\text{res}}$) is
% % \begin{quote}
% %   $\lambda
% % u$:\smallrecord{\smalltfield{cont}{$(T_{\text{arg}}\rightarrow
% %     T_{\text{res}})$}}$^{\frown}$\smallrecord{\smalltfield{cont}{$T_{\text{arg}}$}} . \\
% % \hspace*{2em}\smallrecord{\smallmfield{cont}{$u$[0].cont@$u$[1].cont}{$T_{\text{res}}$}}

% % \end{quote}

% \item[\textnormal{ContForwardApp$_{\text{@}_{\text{wh}_{i,j}}}$($T_{\text{fun}},T_{\text{arg}},T_{\text{res}}$)}] \mbox{}

%   If $T_{\text{fun}}$, $T_{\text{arg}}$ and $T_{\text{res}}$ are types
% such that if $\alpha:T_{\text{fun}}$ and $\beta:T_{\text{arg}}$, then
% $\alpha\text{@}_{\text{wh}_{i,j}}\beta$ is defined and of type
% $T_{\text{res}}$, then
% \begin{quote}
%   ContForwardApp$_{\text{@}_{\text{wh}_{i,j}}}$($T_{\text{fun}},T_{\text{arg}},T_{\text{res}}$)
% \end{quote}
% is
% \begin{quote}
%   $\lambda u$:\smallrecord{
%     \smalltfield{cont}{$T_{\text{fun}}$}}$^\frown$\smallrecord{
%     \smalltfield{cont}{$T_{\text{arg}}$}} . \smallrecord{
%     \smallmfield{cont}{$u[0]$.cont@$_{\text{wh}_{i,j}}u[1]$.cont}{$T_{\text{res}}$}}
% \end{quote}

% \item[\textnormal{ContForwardApp$_{\text{@\!@}}$($T_{\text{arg}}$,
%     $T_{\text{res}}$)}] \mbox{}

%   If $T_{\text{arg}}$ and $T_{\text{res}}$ are types, then
%   ContForwardApp$_{\text{@\!@}}$($T_{\text{arg}}$, $T_{\text{res}}$) is
% \begin{quote}
%   $\lambda
% u$:\smallrecord{\smalltfield{cont}{$(T_{\text{arg}}\rightarrow
%     T_{\text{res}})$}}$^{\frown}$\smallrecord{\smalltfield{cont}{$T_{\text{arg}}$}} . \\
% \hspace*{2em}\smallrecord{\smallmfield{cont}{$u$[0].cont@\!@$u$[1].cont}{$T_{\text{res}}$}}

% \end{quote}

%     \item[\textnormal{ContForwardApp$_{\text{@}_{\&}}$($T$)}]
%       \mbox{}

%       If $T$ is a type, then ContForwardApp$_{\text{@}_{\&}}$($T$) is
%   \begin{quote}
%     $\lambda u$:\record{
%       \tfield{cont}{${^T\textit{PPpty}}$}}$^\frown$\record{
%       \tfield{cont}{${^T\textit{PPpty}}$}} .\\
%     \hspace*{1em}\record{
%       \mfield{cont}{$u[0].\text{cont}\text{@}_{\&}u[1].\text{cont}$}{${^T\textit{PPpty}}$}}
%   \end{quote}

  
% \item[Interpreted phrase structure]

\item[\textnormal{$T_{\text{mother}}\longrightarrow
    T_{\text{daughter}_1}\ T_{\text{daughter}_2}\ \mid\
    T_{\text{daughter}_1}'(_{\mathcal{O}}T_{\text{daughter}_2}')$}
  New!] \mbox{}

  If $T_{\text{mother}}$, $T_{\text{daughter}_1}$ and
  $T_{\text{daughter}_2}$ are sign types and $\mathcal{O}$ is a
  combination operation, then
  \begin{quote}
    $T_{\text{mother}}\longrightarrow
    T_{\text{daughter}_1}\ T_{\text{daughter}_2}\ \mid\
    T_{\text{daughter}_1}'(_{\mathcal{O}}T_{\text{daughter}_2}')$
  \end{quote}
  is
  \begin{quote}
  $T_{\text{mother}}\longrightarrow
    T_{\text{daughter}_1}\ T_{\text{daughter}_2}$ \d{\d{$\wedge$}}
    ContForwardApp$_{\mathfrak{S},\mathcal{O}}$
  \end{quote}

  \paragraph{Note:} This covers all the variant notations introduced
  in previous chapters.

\item[\textnormal{$T_{\text{mother}}\longrightarrow
    T_{\text{daughter}_1}\ T_{\text{daughter}_2}\ \mid\
    B(T_{\text{daughter}_1}'(_{\mathcal{O}}T_{\text{daughter}_2}'))$}
  New!] \mbox{}

If $T_{\text{mother}}$, $T_{\text{daughter}_1}$ and
  $T_{\text{daughter}_2}$ are sign types and $\mathcal{O}$ is a
  combination operation, then
  \begin{quote}
    $T_{\text{mother}}\longrightarrow
    T_{\text{daughter}_1}\ T_{\text{daughter}_2}\ \mid\
    B(T_{\text{daughter}_1}'(_{\mathcal{O}}T_{\text{daughter}_2}'))$
  \end{quote}
  is
  \begin{quote}
  $T_{\text{mother}}\longrightarrow
    T_{\text{daughter}_1}\ T_{\text{daughter}_2}$ \d{\d{$\wedge$}}
    ContForwardApp$_{\mathfrak{S},\mathcal{O},B}$
  \end{quote}

  
\item[\textnormal{$T_{\text{mother}}\longrightarrow
    T_{\text{daughter}_1}\ T_{\text{daughter}_2}\ \mid\
    \mathfrak{A}(T_{\text{daughter}_1}'(_{\mathcal{O}}T_{\text{daughter}_2}'))$}
  New!] \mbox{}

  If $T_{\text{mother}}$, $T_{\text{daughter}_1}$ and
  $T_{\text{daughter}_2}$ are sign types and $\mathcal{O}$ is a
  combination operation, then
  \begin{quote}
    $T_{\text{mother}}\longrightarrow
    T_{\text{daughter}_1}\ T_{\text{daughter}_2}\ \mid\
    \mathfrak{A}(T_{\text{daughter}_1}'(_{\mathcal{O}}T_{\text{daughter}_2}'))$
  \end{quote}
  is
  \begin{quote}
  $T_{\text{mother}}\longrightarrow
    T_{\text{daughter}_1}\ T_{\text{daughter}_2}$ \d{\d{$\wedge$}}
    ContForwardApp$_{\mathfrak{S},\mathcal{O},\mathfrak{A}}$
  \end{quote}
  

% \item[\textnormal{$T_{\text{mother}}\longrightarrow
%     T_{\text{daughter}_1}\ T_{\text{daughter}_2}\ \mid\
%     T_{\text{daughter}_1}'(T_{\text{daughter}_2}':T_{\text{arg}}):T_{\text{res}}$}]
%   \mbox{}

%   If $T_{\text{mother}}$, $T_{\text{daughter}_1}$ and
%   $T_{\text{daughter}_2}$ are sign types and $T_{\text{arg}}$ and
%   $T_{\text{res}}$ are content types, then
%   \begin{quote}
%     $T_{\text{mother}}\longrightarrow
%     T_{\text{daughter}_1}\ T_{\text{daughter}_2}\ \mid\
%     T_{\text{daughter}_1}'(T_{\text{daughter}_2}':T_{\text{arg}}):T_{\text{res}}$
%   \end{quote}
%   is
%   \begin{quote}
%   $T_{\text{mother}}\longrightarrow
%     T_{\text{daughter}_1}\ T_{\text{daughter}_2}$ \d{\d{$\wedge$}}
%     ContForwardApp($T_{\text{arg}}$, $T_{\text{res}}$)
%   \end{quote}



% \item[\textnormal{$T_{\text{mother}}\longrightarrow
%     T_{\text{daughter}_1}\ T_{\text{daughter}_2}\ \mid\
%     T_{\text{daughter}_1}'(_{\text{@}_{\text{wh}_{i,j}}}T_{\text{daughter}_2}':T_{\text{arg}}):T_{\text{res}}$}] \mbox{}

%   If $T_{\text{mother}}$, $T_{\text{daughter}_1}$ and
%   $T_{\text{daughter}_2}$ are sign
%   types, $T_{\text{daughter}_1}\sqsubseteq$\smallrecord{
%     \smallmfield{cont}{$c_1$}{\textit{Cont}}} where
%   $c_1.\text{bg}\sqsubseteq$\smallrecord{
%     \smalltfield{$\mathfrak{w}$}{\smallrecord{
%         \smalltfield{x$_i$}{\textit{Ind}}}}}, $T_{\text{daughter}_2}\sqsubseteq$\smallrecord{
%     \smallmfield{cont}{$c_2$}{\textit{Cont}}} where
%   $c_2.\text{bg}\sqsubseteq$\smallrecord{
%     \smalltfield{$\mathfrak{g}$}{\smallrecord{
%         \smalltfield{x$_j$}{\textit{Ind}}}}} and $T_{\text{arg}}$ and
%   $T_{\text{res}}$ are content types, then
%   \begin{quote}
%     $T_{\text{mother}}\longrightarrow
%     T_{\text{daughter}_1}\ T_{\text{daughter}_2}\ \mid\
% T_{\text{daughter}_1}'(_{\text{@}_{\text{wh}_{i,j}}}T_{\text{daughter}_2}':T_{\text{arg}}):T_{\text{res}}$
%   \end{quote}
%   is
%   \begin{quote}
%   $T_{\text{mother}}\longrightarrow
%     T_{\text{daughter}_1}\ T_{\text{daughter}_2}$ \d{\d{$\wedge$}}
%     ContForwardApp$_{\text{@}_{\text{wh}_{i,j}}}$($T_{\text{arg}}$, $T_{\text{res}}$)
%   \end{quote} 



  
    
  
% \item[\textnormal{$T_{\text{mother}}\longrightarrow
%     T_{\text{daughter}_1}\ T_{\text{daughter}_2}\ \mid\
%     T_{\text{daughter}_1}'(_{\text{@\!@}}T_{\text{daughter}_2}':T_{\text{arg}}):T_{\text{res}}$}]
%   \mbox{}

%   If $T_{\text{mother}}$, $T_{\text{daughter}_1}$ and
%   $T_{\text{daughter}_2}$ are sign types and $T_{\text{arg}}$ and
%   $T_{\text{res}}$ are content types, then
%   \begin{quote}
%     $T_{\text{mother}}\longrightarrow
%     T_{\text{daughter}_1}\ T_{\text{daughter}_2}\ \mid\
%     T_{\text{daughter}_1}'(_{\text{@\!@}}T_{\text{daughter}_2}':T_{\text{arg}}):T_{\text{res}}$
%   \end{quote}
%   is
%   \begin{quote}
%   $T_{\text{mother}}\longrightarrow
%     T_{\text{daughter}_1}\ T_{\text{daughter}_2}$ \d{\d{$\wedge$}}
%     ContForwardApp$_{\text{@\!@}}$($T_{\text{arg}}$, $T_{\text{res}}$)
%   \end{quote}

  

      

  
% \item[\textnormal{$T_{\text{mother}}\longrightarrow T_{\text{daughter}_1}\
%     T_{\text{daughter}_2}\ \mid\
%     T_{\text{daughter}_1}'(_{\text{@}_{\&}}T_{\text{daughter}_2}':{^T\textit{PPpty}}):{^T\textit{PPpty}}$}] \mbox{}

%   If $T$ is a type, $T_{\text{mother}}$, $T_{\text{daughter}_1}$ and
% $T_{\text{daughter}_2}$ are sign types,
% $T_{\text{daughter}_1}\sqsubseteq$ \record{
%   \tfield{cont}{$^T\textit{PPpty}$}} and $T_{\text{daughter}_2}\sqsubseteq$ \record{
%   \tfield{cont}{$^T\textit{PPpty}$}}, then
%   \begin{quote}
%     $T_{\text{mother}}\longrightarrow T_{\text{daughter}_1}\
%     T_{\text{daughter}_2}\ \mid\
%     T_{\text{daughter}_1}'(_{\text{@}_{\&}}T_{\text{daughter}_2}':{^T\textit{PPpty}}):{^T\textit{PPpty}}$
%   \end{quote}
%   is
%   \begin{quote}
%     $T_{\text{mother}}\longrightarrow T_{\text{daughter}_1}\
%     T_{\text{daughter}_2}$ \d{\d{$\wedge$}}
%     ContForwardApp$_{\text{@}_{\&}}$($T$)
%   \end{quote}

\end{description}



\subsubsection{Action rules} (as in Chapter~\ref{ch:intensional})

% % \begin{description}

  




% % \item[\textnormal{\textsc{LexRes}}] \mbox{}

% %   \begin{prooftree}
% %     \hypo{\text{Lex}(T,C) \text{ resource}_A}
% %     \hypo{u:_A T}
% %     \infer[enth]2{:_A(\text{Lex}(T,C)\text{\d{$\wedge$}\smallrecord{\smalltfield{s-event}{\smallrecord{\smallmfield{e}{$u$}{$T$}}}}})}
% %   \end{prooftree}

  
% % \item[\textnormal{\textsc{ToposConclude}} New!] \mbox{}

% %   \begin{prooftree}
% %     \hypo{\tau:\textit{Topos}}
% %     \hypo{\tau \text{ resource}_A}
% %     \hypo{s:_A\tau.\text{bg}}
% %     \infer[enth]3{:_A\tau(s)}
% %   \end{prooftree}

  
% % \item[\textnormal{\textsc{ToposPermit}} New!] \mbox{}

% %   \begin{prooftree}
% %     \hypo{\tau:\textit{Topos}}
% %     \hypo{\tau \text{ resource}_A}
% %     \hypo{s:_A\tau.\text{bg}}
% %     \infer[enth]3{:_A\tau(s)!}
% %   \end{prooftree}

    
% % \item[\textnormal{\textsc{ToposOblige}} New!] \mbox{}

% %   \begin{prooftree}
% %       \hypo{\tau:\textit{Topos}}
% %       \hypo{\tau \text{ resource}_A}
% %       \hypo{s:_A\tau.\text{bg}}
% %       \infer[enth]3[oblig]{:_A\tau(s)!}
% %     \end{prooftree}

    
  


% %   \end{description}



\subsection{Universal speech act resources} (as in Chapter~\ref{ch:infex})

% \subsubsection{Types}

% \begin{description}
  
% \item[\textnormal{\textit{Assertion}}] --- \record{
%   \tfield{s-event}{\textit{SEvent}}\\
%   \tfield{cont}{\textit{RecType}}\\
%   \tfield{illoc}{assert(s-event, cont)}
% }
% \item[\textnormal{\textit{Query}}] --- \record{
%   \tfield{s-event}{\textit{SEvent}}\\
%   \tfield{cont}{\textit{Question}}\\
%   \tfield{illoc}{query(s-event, cont)}
% }

% \item[\textnormal{\textit{Command}}] --- \record{
%   \tfield{s-event}{\textit{SEvent}}\\
%   \tfield{cont}{\textit{RecType}}\\
%   \tfield{illoc}{command(s-event, cont)}
% }

% \item[\textnormal{\textit{Acknowledgement}}] --- \record{
%   \tfield{s-event}{\textit{SEvent}}\\
%   \tfield{cont}{\textit{RecType}}\\
%   \tfield{illoc}{acknowledge(s-event, cont)}
% }

% \item[\textnormal{\textit{AssertionType}}] --- a basic type

%   $T:\textit{AssertionType}$ iff $T\sqsubseteq \textit{Assertion}$

  
% \item[\textnormal{\textit{QueryType}}] --- a basic type

%   $T:\textit{QueryType}$ iff $T\sqsubseteq \textit{Query}$

  
% \item[\textnormal{\textit{CommandType}}] --- a basic type

%   $T:\textit{CommandType}$ iff $T\sqsubseteq \textit{Command}$


 
% \item[\textnormal{\textit{AcknowledgementType}}] --- a basic type

%   $T:\textit{AcknowledgementType}$ iff $T\sqsubseteq \textit{Acknowledgement}$

% \end{description}

\subsection{Universal discourse resources}

\begin{description}

  
\item[\textnormal{$\varphi_1\
    |_{\pi_{11},\pi_{21};\ldots;\pi_{1n},\pi_{2n}}\ \varphi_2$} New!]
  \mbox{}

  Suppose that $\varphi_1$ and $\varphi_2$ and parametric contents,
$\pi_{11}\ldots\pi_{1n}\in\mathrm{paths}(\varphi_1.\text{bg})$ and \\
$\pi_{21}\ldots\pi_{2n}\in\mathrm{paths}(\mathcal{F}_{\text{quasi}^*}(\varphi_2.\text{fg})$,
then the \textit{content $\varphi_1$ given $\varphi_2$ with
  alignment of $\pi_{11}$ and $\pi_{21}$,\ldots,$\pi_{1n}$ and $\pi_{2n}$},
$\varphi_1\ |_{\pi_{11},\pi_{21};\ldots;\pi_{1n},\pi_{2n}}\ \varphi_2$, is
\begin{quote}
  \record{
    \field{bg}{($\varphi_1$.bg \d{$\wedge$} \smallrecord{
        \smalltfield{$\mathfrak{p}$}{$\mathcal{F}_{\text{quasi}^*}(\varphi_2.\text{fg})$}})$_{\pi_{11}=\mathfrak{p}.\pi_{21},\ldots,\pi_{1n}=\mathfrak{p}.\pi_{2n}}$}\\
    \field{fg}{$\lambda c$:bg . $\varphi_1(c)$}}
\end{quote}

\item[\textnormal{$\mathfrak{C}(T_1,T_2)$} New!] \mbox{}

  If $T_1$ and $T_2$ are types
of parametric contents, then there is a combined type,
$\mathfrak{C}(T_1,T_2)$, whose witnesses include witnesses for $T_1$
given a witness for $T_2$ with some possible alignment between the
two.

The witnesses of $\mathfrak{C}(T_1,T_2)$ are characterized
recursively by:
\begin{enumerate} 
 
\item if $\varphi:T_1$, then $\varphi:\mathfrak{C}(T_1,T_2)$ 
 
\item \begin{tabbing}
    if \=$\varphi_1:\mathfrak{C}(T_1,T_2)$,\\
  \>$\pi_{11},\ldots,\pi_{1n}\in\mathrm{paths}(\varphi_1.\text{bg})$,\\ \>$\varphi_2:T_2$ and\\
\>$\pi_{21},\ldots,\pi_{2n}\in\mathrm{paths}(\mathcal{F}_{\text{quasi}^*}(\varphi_2.\text{fg}))$,
  \end{tabbing}
  then
  \begin{quote}
    $\varphi_1|_{\pi_{11},\pi_{21};\ldots;\pi_{1n},\pi_{2n}}\varphi_2:\mathfrak{C}(T_1,T_2)$
  \end{quote}
  
 
\end{enumerate}

\item[Interpretation in the context of a previous utterance New!] ---
  action rule in example~(\ref{ex:interp-cntxt-previous-utterance}).

\end{description}


\subsection{Universal dialogue resources} (as in Chapter~\ref{ch:propnames})

% \subsubsection{Types}

% \begin{description}
% \item[\textnormal{\textit{InfoState}}] --- \record{
%     \tfield{private}{
%       \record{
%         \tfield{agenda}{$\mathrm{list}(\textit{RecType})$}}} \\
%     \tfield{shared}{
%       \record{
%         \tfield{latest-utterance}{\textit{Sign}$^*$}\\
%         \tfield{commitments}{\textit{RecType}}}}}
  
% \item[\textnormal{\textit{InitInfoState}}] --- \record{
%     \tfield{private}{
%       \record{
%         \mfield{agenda}{[ ]}{$\mathrm{list}(\textit{RecType})$}}} \\
%     \tfield{shared}{
%       \record{
%         \mfield{latest-utterance}{$\varepsilon$}{\textit{Sign}$^*$}\\
%         \mfield{commitments}{\textit{Rec}}{\textit{RecType}}}}}  

  
% \item[\textnormal{\textit{GameBoard}} New!] --- a basic type

%   $T$ : \textit{GameBoard} iff $T\sqsubseteq$ \textit{InfoState}
  
% \item[\textnormal{\textit{TotalInfoState}} New!] --- \record{\tfield{ltm}{\textit{RecType}} \\
%         \tfield{gb}{(ltm$\rightarrow$\textit{GameBoard})}}


% \end{description}

% \subsubsection{Update functions and action rules}

% %(as in Chapter~\ref{ch:infex})

% \begin{description}
% \item[\textnormal{f$_{\textsc{PlanAckAss}}$}] $\lambda r$:\textit{InfoState} . \\
%   \hspace*{4em}$\lambda u$:\textit{Assertion} . \\
%   \hspace*{6em}
%   \smallrecord{
%     \smalltfield{private}{\smallrecord{
%         \smalltfield{agenda}{\smallrecord{
%             \smalltfield{fst}{\smallrecord{
%                 \smalltfield{s-event}{\textit{SEvent} \d{$\wedge$} \smallrecord{
%                     \smallmfield{sp}{$u$.s-event.au}{\textit{Ind}}\\
%                     \smallmfield{au}{$u$.s-event.sp}{\textit{Ind}}}}\\
%                 \smallmfield{cont}{$u$.cont}{\textit{Cont}}\\
%                 \smalltfield{illoc}{acknowledge(s-event, cont)}}}\\
%             \smallmfield{rst}{$r$.private.agenda}{$\mathrm{list}(\textit{RecType})$}}}}}\\
%     \smalltfield{shared}{\smallrecord{
%         \smallmfield{latest-utterance}{$u$}{\textit{Assertion}}}}}
  
% \item[\textnormal{\textsc{PlanAckAss}}]
%   \begin{prooftree}
%     \hypo{s_{i,A}:_A T_{\mathrm{curr}}}
%     \hypo{T_{\mathrm{curr}}\sqsubseteq\mathrm{domtype}(\text{f}_{\textsc{PlanAckAss}})}
%     \hypo{u^*:_A T_{\mathrm{utt}}}
%     \hypo{T_{\mathrm{utt}}\sqsubseteq\textit{Assertion}}
%     \infer[enth]4{s_{i+1,A}:_A
%       T_{\mathrm{curr}}\text{\fbox{\d{$\wedge$}}}(\text{f}_{\textsc{PlanAccAss}}(s_{i,A})(u^*)\text{\d{$\wedge$}}\text{\smallrecord{\smalltfield{shared}{\smallrecord{
%               \smalltfield{latest-utterance}{$T_{\mathrm{utt}}$}}}}})}
%   \end{prooftree}

  
% \item[\textnormal{f$_{\textsc{IntegAck}}$}]
%   $\lambda r$:\textit{InfoState} . \\
%   \hspace*{2em}$\lambda u$:\textit{Acknowledgement} . \\
%   \hspace*{4em}\smallrecord{
%     \smalltfield{shared}{\smallrecord{
%         \smallmfield{commitments}{\smallrecord{
%             \smalltfield{prev}{$r$.shared.commitments}}\d{$\wedge$}$u$.cont}{\textit{RecType}}\\
%         \smallmfield{latest-utterance}{$u$}{\textit{Acknowledgement}}}}}

  
% \item[\textnormal{\textsc{IntegAck}}]
%    \begin{prooftree}
%     \hypo{s_{i,a}:_A T_{\mathrm{curr}}}
%     \hypo{T_{\mathrm{curr}}\sqsubseteq\mathrm{domtype}(\text{f}_{\mathrm{IntegAck}})}
%     \hypo{u^*:_A T_{\mathrm{utt}}}
%     \hypo{T_{\mathrm{utt}}\sqsubseteq\textit{Acknowledgement}}
%     \infer[enth]4{s_{i+1,A}:_A
%       T_{\mathrm{curr}}\text{\fbox{\d{$\wedge$}}}(\text{f}_{\textsc{IntegAck}}(s_{i,A})(u^*)\text{
%         \d{$\wedge$} \smallrecord{\smalltfield{shared}{\smallrecord{
%              \smalltfield{latest-utterance}{$T_{\mathrm{utt}}$}}}}})} 
%  \end{prooftree}

 
% \item[\textnormal{\textsc{ExecTopAgenda}}]
%   \begin{prooftree}
%     \hypo{s_{i,A}:_A\textit{InfoState}\text{\d{$\wedge$}
%         \smallrecord{\smalltfield{private}{\smallrecord{
%               \smalltfield{agenda}{\smallrecord{
%                   \smalltfield{fst}{\textit{RecType}}\\
%                   \smalltfield{rst}{$\mathrm{list}(\textit{RecType})$}}}}}}}}
%     \infer[enth]1{:_A
%       s_{i,A}.\text{private}.\text{agenda}.\text{fst}!}
%   \end{prooftree}

  
% \item[\textnormal{\textsc{DowndateAgenda}}]
%   \mbox{}

%   \hspace*{-2.5em}
%   \begin{prooftree}
%     \hypo{s_{i,A}:_A T_{\mathrm{curr}}}
%     \hypo{T_{\mathrm{curr}}\sqsubseteq\text{
%         \smallrecord{
%           \smalltfield{private}{\smallrecord{
%               \smalltfield{agenda}{\smallrecord{
%                   \smalltfield{fst}{\textit{RecType}}\\
%                   \smalltfield{rst}{$\mathrm{list}(\textit{RecType})$}}}}}}}}
%     \hypo{u^*:_A s_{i,A}.\text{private}.\text{agenda}.\text{fst}}
%     \infer[enth]3{s_{i+1,A}:_A
%       T_{\mathrm{curr}}\text{\fbox{\d{$\wedge$}}
%         \smallrecord{
%           \smalltfield{private}{\smallrecord{
%               \smallmfield{agenda}{$s_{i,A}$.private.agenda.rst}{$\mathrm{list}(\textit{RecType})$}}}}}}
%   \end{prooftree}
  
% \item[\textnormal{\textsc{AccGB}} New!] See p.~\pageref{ex:AccGBfinal}.

  
% \item[\textnormal{\textsc{AccLTM}} New!] See p.~\pageref{ex:AccLTM}.
  
% \item[\textnormal{\textsc{AccNM}} New!] See p.~\pageref{ex:AccNM}.
  
% \item[Control regime for accommodation New!] \mbox{}

%   \begin{subex} 
 
% \item if there is a labelling, $\eta$, such that
% $s_{i,A}^{\text{tot}}.\text{gb}(s_{i,A}^{\text{tot}}).\text{ltm}.\text{shared}.\text{commitments}\sqsubseteq[u^*.\text{cont}.\text{bg}]_\eta$,
%   then use \text{AccGB} with $\eta$ 
 
% \item else if there is a labelling, $\eta$, such that
%   $s_{i,A}^{\text{tot}}.\text{ltm}\sqsubseteq[u^*.\text{cont}.\text{bg}]_\eta$
%   then use \text{AccLTM} with $\eta$

  
% \item else use \textsc{AccNM}
 
% \end{subex} 

% \end{description}

\subsection{English resources} 

 \subsubsection{Types and predicates} (as in Chapter~\ref{ch:intensional})

%  % \begin{description}

  
% % \item[Basic phonological  types for words] \mbox{}

% %   \{``Dudamel'', ``is'', ``a'', ``conductor'', ``Beethoven'',
% %   ``composer'', ``Uchida'', ``pianist'', ``aha'', ``ok'', ``leaves''
% %   , ``hugs'', ``dog'', ``nine'',
% %   ``ninety'', ``find'' \textbf{New!}, ``seek'' \textbf{New!},
% %   ``worship'' \textbf{New!}\}

% % %\item [Witnesses for basic types] \mbox{}

% % %  \begin{description}
% % % \item[\textnormal{\textit{Ind}}] --- dudamel, beethoven, uchida :
% % %   \textit{Ind} \textbf{No longer necessary for interpretation of
% % %     proper names!}
% % %   \end{description}

  
% % \item[Predicates] \mbox{}
  
% %   \begin{description}
  
% %   \item[with arity \textnormal{$\langle\textit{Ind}\rangle$}]
% %     \{conductor, composer, pianist, leave,  dog,
% %     passenger\}
    
% %   \item[with arity
% %     \textnormal{$\langle\textit{Ind},\textit{Quant}\rangle$}] \{hug
% %     \textbf{Revised!}, find \textbf{New!}, seek \textbf{New!}, worship
% %     \textbf{New!}, want$_Q$ \textbf{New!}\}

% %       $e$ : worship($a$,$\mathcal{Q}$) iff for some $T$
% % \begin{enumerate} 
 
% % \item $e$ : rbelieve($a$, $T$) 
 
% % \item $T$ $\sqsubseteq_{\leadsto}$ $\mathcal{Q}$($\lambda
% %   r$:\smallrecord{\smalltfield{x}{\textit{Ind}}}
% %   . worship$^\dagger$($a$, $r$.x))
 
% % \end{enumerate}
% % \textbf{or} for some $T'$
% % \begin{enumerate} 
 
% % \item $e$ :$_{\underline{\varepsilon}}$ rbelieve($a$, $T$)

% % \item $e$ :$_{\underline{\varepsilon}}$ pov($T'$, $T$)
 
% % \item $T$\fbox{\d{$\wedge$}}$T'$ $\sqsubseteq_{\leadsto}$ $Q$($\lambda
% %   r$:\smallrecord{\smalltfield{x}{\textit{Ind}}}
% %   . worship$^\dagger$($a$, $r$.x)) 

% % \end{enumerate}

% % \bigskip

% %     We restrict our attention to modal systems, $\mathbb{T}$, such
% %     that
% %     \begin{enumerate}
    
% %     \item if $p\in\{\text{hug},\text{find}\}$ then
% %     \begin{quote}
% %       $p(a,\mathcal{Q})\approx_{\mathbb{T}}\mathcal{Q}(\ulcorner\lambda
% %       r$:\smallrecord{\smalltfield{x}{\textit{Ind}}} . \smallrecord{\smalltfield{e}{$p^{\dagger}(a,
% %           r.\text{x})$}}$\urcorner)$
% %     \end{quote}
    
    
    
% %   \item $\text{successful}(\text{seek}(a,\mathcal{Q}))\sqsubseteq_{\mathbb{T}}
% %     \text{find}(a,\mathcal{Q})$

    
% %   \item want$_Q$($a$,$Q$) $\approx_{\mathbb{T}}$
% % want$^\dagger$($a$, $Q$($\ulcorner\lambda
% % r$:\smallrecord{\smalltfield{x}{\textit{Ind}}} . have($a$, $r$.x)$\urcorner$))
% % \end{enumerate}
  

% %     \item[with arity
% %   \textnormal{$\langle\textit{Rec},\textit{Rec}\rangle$}] --- \{rise
% %   \}
  

% %       $e$ : rise$(r,c)$ if
% %   \begin{quote}
% %     $r$ : \textit{AmbTempFrame},\\
% %     $c$ : \textit{TempRiseEventCntxt} and \\
% %     $e$ : \textit{TempRiseEvent}$(c)$ \d{$\wedge$}
% %     \smallrecord{
% %       \smalltfield{e}{\smallrecord{
% %           \smallmfield{t$_0$}{$r$}{\textit{AmbTempFrame}}}}}
% %   \end{quote}
% %   \textbf{or} if
% %   \begin{quote}
% %     $r$ : \textit{PriceFrame},\\
% %     $c$ : \textit{PriceRiseEventCntxt} and \\
% %     $e$ : \textit{PriceRiseEvent}$(c)$ \d{$\wedge$}
% %     \smallrecord{
% %       \smalltfield{e}{\smallrecord{
% %           \smallmfield{t$_0$}{$r$}{\textit{PriceFrame}}}}}
% %   \end{quote}
% %   \textbf{or} if
% %   \begin{quote}
% %     $r$ : \textit{LocFrame},\\
% %     $c$ : \textit{LocRiseEventCntxt} and \\
% %     $e$ : \textit{LocRiseEvent}$(c)$ \d{$\wedge$}
% %     \smallrecord{
% %       \smalltfield{e}{\smallrecord{
% %           \smallmfield{t$_0$}{$r$}{\textit{LocFrame}}}}}
% %   \end{quote}

  

  
% % \item[with arity \textnormal{$\langle\textit{Rec}\rangle$}] ---
% %   \{temperature\}
  
% % $e$ : temperature$(r)$ if 
% %   \begin{quote}
% %     $r$ : \textit{AmbTempFrame} and $e=r$
% %   \end{quote}

  
% % \item[with arity
% %   \textnormal{$\langle\textit{Ind},\textit{TravelFrame}\rangle$}] ---
% %   \{take\_journey\}

% %   $s$ : take\_journey($a$, $e$) iff $s=e$ and $e$.traveller = $a$

  
% % \item[with arity
% %   \textnormal{$\langle\textit{Ind},\textit{Ppty}\rangle$}] ---
% %   \{want$_P$\}

% %  We restrict our attention to modal systems, $\mathbb{T}$, such
% %  that
% %  \begin{quote}
% %    want$_P$($a$, $P$) $\approx_{\mathbb{T}}$ want$^\dagger$($a$,
% %    $P$(\smallrecord{\field{x}{$a$}}))
% %  \end{quote}
  

% % \item[with arity
% %   \textnormal{$\langle\textit{Ind},\textit{RecType}\rangle$}]
% %   --- \{believe \textbf{New!}, want$^{\dagger}$ \textbf{New!}\}

% %   $e$ : believe($a$, $T$) if
% %   \begin{quote}
% %     $e$ : ltm($a$, $T'$) \\
% %     and $T'\sqsubseteq_{\leadsto}T$
% %   \end{quote}
% %   \textbf{or} if
% %   \begin{quote}
% %     $e$ :$_{\underline{\varepsilon}}$ believe($a$, $T_1$)\\
% %     $e'$ :$_{\underline{\varepsilon}}$ pov($T_2$, $T_1$)\\
% %     and $T_1$\fbox{\d{$\wedge$}}$T_2 \sqsubseteq_{\leadsto}T$
% %   \end{quote}

% %   \bigskip

% %   $e$ : want$^{\dagger}$($a$, $T$) if for some $T'$\\
% %   \hspace*{2em} $e$ : des($a$, $T'$) \\
% %   \hspace*{2em} and $T'\sqsubseteq_{\leadsto}T$

% %   \textbf{or} if for some $T_1$ and $T_2$\\
% %   \hspace*{2em} $e$ :$_{\underline{\varepsilon}}$ want$^{\dagger}$($a$, $T_1$)\\
% %   \hspace*{2em} $e$ :$_{\underline{\varepsilon}}$ pov($T_2$, $T_1$) \\
% %   \hspace*{2em} and $T_1$\fbox{\d{$\wedge$}}$T_2
% %   \sqsubseteq_{\leadsto}T$
  
% % \end{description}

% % \item[Frame types] \mbox{}

% %   \begin{description}

    
% %   \item[\textnormal{\textit{DogFrame}}] ---
% %     \record{
% %       \tfield{x}{\textit{Ind}}\\
% %       \tfield{e}{dog(x)}\\
% %       \tfield{age}{\textit{Real}}\\
% %       \tfield{c$_{\mathrm{age}}$}{age\_of(x,age)}}
                    
% %   \item[\textnormal{\textit{TravelFrame}}] ---
% %     \record{
% %       \tfield{traveller}{\textit{Ind}}\\
% %       \tfield{source}{\textit{Loc}}\\
% %       \tfield{goal}{\textit{Loc}}}

    
% %   \item[\textnormal{\textit{PassengerFrame}}] ---
% %     \record{\tfield{x}{\textit{Ind}}\\
% %               \tfield{e}{passenger(x)}\\
% %               \tfield{journey}{\textit{TravelFrame}}\\
% %               \tfield{c$_{\mathrm{travel}}$}{take\_journey(x, journey)}}


% %           \end{description}

          
        
  
  

% % \end{description}

\subsubsection{Grammar} 

\begin{description}



\item[Lexical sign types] \mbox{}

  Let \textit{Lexicon} be the set of lexical sign types defined
  inductively as follows.  The
  following set is included in \textit{Lexicon}.

\begin{tabbing}
    \{\=Lex$_{\mathrm{PropName}}$(``Dudamel''), \\
    \> Lex$_{\mathrm{PropName}}$(``Beethoven''), \\
    \> Lex$_{\mathrm{Pron}}$(``he''), \\
    \> Lex$_{\mathrm{numeral}}$(``nine'', 9), \\
    \> Lex$_{\mathrm{numeral}}$(``ninety'', 90), \\ 
    \> Lex$_{\mathrm{IndefArt}}$(``a''), \\
    \> Lex$_{\mathrm{Universal}}$(``every''), \\
\> Lex$_{\mathrm{DefArt}}$(``the''),\\
\> Lex$_{\text{CommonNoun}}$(``composer'', \textit{Rec}, composer),\\
\> Lex$_{\text{CommonNoun}}$(``conductor'', \textit{Rec}, conductor), \\
\> Lex$_{\text{CommonNoun}}$(``dog'', \textit{Rec}, dog) (= $\Sigma_{\text{``dog''}}$), \\
\> RestrictCommonNoun(CommonNounIndToFrame($\Sigma_{\text{``dog''}}$),
\textit{DogFrame}),\\
\> Lex$_{\mathrm{CommonNoun}}$(``passenger'', \textit{Rec}, passenger) ($=\Sigma_{\text{``passenger''}}$), \\
\> RestrictCommonNoun(CommonNounIndToFrame($\Sigma_{\text{``passenger''}}$),
\textit{PassengerFrame}), \\
\> Lex$_{\text{CommonNoun}}$(``temperature'', \textit{Rec},
temperature) (= $\Sigma_{\text{``temperature''}}$), \\
\> RestrictCommonNoun($\Sigma_{\text{``temperature''}}$,
\textit{AmbTempFrame}),\\
\> Lex$_{\mathrm{IntransVerb}}$(``leave'', \textit{Rec}, leave), \\
\> Lex$_{\mathrm{IntransVerb}}$(``run'', \textit{Rec}, run), \\
\> Lex$_{\mathrm{IntransVerb}}$(``rise'', \smallrecord{\smalltfield{$\mathfrak{c}$}{\textit{TempRiseEventCntxt}}},
rise), \\
\> Lex$_{\mathrm{IntransVerb}}$(``rise'', \smallrecord{\smalltfield{$\mathfrak{c}$}{\textit{PriceRiseEventCntxt}}},
rise), \\
\> Lex$_{\mathrm{IntransVerb}}$(``rise'', \smallrecord{\smalltfield{$\mathfrak{c}$}{\textit{LocRiseEventCntxt}}},
rise), \\
\> Lex$_{\mathrm{TransVerb}}$(``hug'', \textit{Rec}, hug), \\
\> Lex$_{\mathrm{TransVerb}}$(``find'', \textit{Rec}, find), \\
\> Lex$_{\mathrm{TransVerb}}$(``seek'', \textit{Rec}, seek), \\
\> Lex$_{\mathrm{TransVerb}}$(``worship'', \textit{Rec}, worship), \\
\> Lex$_{\mathrm{be}_{\text{ID}}}$(``is''), \\
\> Lex$_{\mathrm{be}_{\text{scalar}}}$(``is''), \\
\> Lex(``ok'', \textit{S}),  \\
\> Lex(``aha'', \textit{S})  \}
\end{tabbing}

\begin{description}
  
\item[Transitive verbs as verb phrases] \mbox{}

If $\Sigma$ = Lex$_{\text{TransVerb}}$($T_{\text{Phon}}$,
$T_{\text{bg}}$, $p$), for some $T_{\text{Phon}}$, $T_{\text{bg}}$ and
$p$, and $\Sigma\in\textit{Lexicon}$, then
TransVerbToVerbPhrase($\Sigma$) $\in$ \textit{Lexicon}.
\end{description}

\item[Constituent structure rule components] \mbox{} 
  \begin{description}

    \item[\textnormal{CnstrIsA} Revised!] \mbox{}

  $\lambda
u$:\textit{V}\d{$\wedge$}\smallrecord{\smalltfield{s-event}{\smallrecord{\smalltfield{e}{``is''}}}}$^{\frown}$\textit{NP}\d{$\wedge$}\smallrecord{\smalltfield{syn}{\smallrecord{\smalltfield{daughters}{\textit{Det}\d{$\wedge$}\smallrecord{\smalltfield{s-event}{\smallrecord{\smalltfield{e}{``a''}}}} \\
                                                                    \hspace*{5em}$^{\frown}$
\textit{N}}}
}} . \\
\hspace*{1em} \textit{VP}\d{$\wedge$}\smallrecord{\smallmfield{cont}{$u$[2].syn.daughters[2].cont}{\textit{ContType}}} 

  \end{description}
  


\item[Constituent structure rules] \mbox{}

Let \textit{CSRules} be the set of constituent structure rules,
defined inductively as follows.  The following set is included in \textit{CSRules}.
  
  \begin{tabbing}
    \{\=\textit{S} $\longrightarrow$ \textit{NP VP} $\mid$
    $B$(\textit{NP}$'$($_{\text{@}}$\textit{VP}$'$) \textbf{Revised!},\\
    \>\textit{NP} $\longrightarrow$ \textit{Det N} $\mid$
    \textit{Det}$'$($_{\text{@\!@}}$\textit{N}$'$) \textbf{Revised!},\\
    \>\textit{VP} $\longrightarrow$ \textit{V} \textit{NP}
    \d{\d{$\wedge$}} CnstrIsA, \\
    \>\textit{VP} $\longrightarrow$ \textit{V} \textit{NP} $\mid$
    $\mathfrak{A}$(\textit{V}$'$($_{\text{@}}$\textit{NP}$'$)) \textbf{Revised!}\}
  \end{tabbing}



  \begin{description}

    
  \item[Relative clauses] \mbox{}

    If $i$ and $j$ are natural numbers, then
    \begin{quote}
      \textit{Rel} $\longrightarrow$ \textit{NP}$_{\text{wh}_i}$
      \textit{S}/$j$ $\mid$
      \textit{NP}$_{\text{wh}_i}'$($_{\text{@}_{\text{wh}_{i,j}}}$\textit{S}/$j'$) \textbf{Revised!}
    \end{quote}
    is a member of \textit{CSRules}

    \bigskip

    If $T$ is a type, then
    \begin{quote}
      $\textit{N}\longrightarrow{^T\textit{N}}\ {^T\textit{Rel}}\
      \mid\
      {^T\textit{N}}'(_{\text{@}_\&}{^T\textit{Rel}}')$ \textbf{Revised!}
    \end{quote}
    is a member of \textit{CSRules}

\end{description}

\end{description}


\section{Summary}

In this chapter we have explored how to use types to characterize
underspecified interpretation.  A type underspecifies its witnesses
since any one of them can be used as a specific object of the type.
We thus characterize types of contents to play the role of
underspecified representations as employed, for example, in DRT by
\cite{Reyle1993} or using Quasi Logical Form \citep{Alshawi1992}.
Using underspecified representations of meaning is important in
computational applications where a large number of alternative
specified representations can be generated.  The idea here is that we
can compute a single type of content for an utterance and then
determine a precise content as needed and check that it is of the type
generated.  This helps us understand how speakers of a natural
language can rapidly process utterances in real time when in fact many
different contents are available.

We started by recreating the storage algorithm of \cite{Cooper1983},
showing how the stores for quantifiers can be incorporated into
context on the structure view of context we have developed using
record types.  We then showed how a treatment of anaphora can be added
to the elaboration of content types
including discourse anaphora, donkey anaphora and aspects of Chomskian
binding theory.

While we do not claim that we have covered everything that has been
accounted for in the considerable linguistic literature on these
topics, there does seem to be enough here to show that our theory of
types which we used in the earlier parts of the book to deal with
perception, interaction and mental states, is also capable of dealing
with these central concerns of linguistic semantics.




%%% Local Variables:
%%% mode: latex
%%% TeX-master: "ttl"
%%% End:
